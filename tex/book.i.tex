\documentclass{kapital}
%% should be a class option
\renewcommand{\parbreak}{\unskip\ignorespaces}
\renewcommand{\parcont}{\unskip\ignorespaces}

%% proper quote marks 
\newunicodechar{„}{«}
\newunicodechar{“}{»}

%% ditto mark
\renewcommand{\dittomark}{~}

%% overfull boxes
\vfuzz=12pt
\hfuzz=1pt

\newcommand{\Year}{2020}
\newcommand{\City}{Київ}

\newcommand{\VolumeNumber}{Том І}
\newcommand{\BookNumber}{Книга перша}
\newcommand{\BookTitle}{Процес продукції капіталу}
\newcommand{\BookSource}{Переклад з четвертого німецького видання}
\newcommand{\BookAuthors}{за редакцією Д.~Рабіновича і~С.~Трикоза}
\newcommand{\SourceYear}{Харків, 1933}
\newcommand{\Biblio}{Т. І. — Кн. І: Процес продукції капіталу /
Пер. із 4 нім. вид.; За ред. Д. Рабіновича і~С.~Трикоза.}

\newcommand{\VolumeNumberDe}{Erster Band}
\newcommand{\BookNumberDe}{Buch I}
\newcommand{\BookTitleDe}{Der Produktionsprocess des Kapitals}
\newcommand{\AuflageDe}{Vierte, durchgesehene Auflage}
\newcommand{\HerausgegebenDe}{Herausgegeben von Friedrich Engels}
\newcommand{\PublisherDe}{Verlag von Otto Meissner}
\newcommand{\CityDe}{Hamburg}
\newcommand{\YearDe}{1890}

\begin{document}
  \pagenumbering{roman}
\frontmatter
\thispagestyle{empty}
\null\vspace{6cm}
\noindent\htitlespace{\Large\bfseries\scsans Андрію Річицькому}
\cleardoublepage

\thispagestyle{empty}
\null\vspace{0.5cm}
\noindent\htitlespace{\Large\scmain Карл Маркс}

\vspace{2.5cm}
\noindent\htitlespace{\hspace{-0.05cm}\fontsize{47.22}{60}\bfseries\letterspacefont\MakeUppercase{Капітал}}

\vspace{0.1cm}
\noindent\htitlespace{\Large\scsans\bfseries Критика політичної економії}

\vspace{1.5cm}
\noindent\htitlespace{\Large\scmain \VolumeNumber{}. \BookNumber{}}

\smallskip
\noindent\htitlespace{\Large\scmain \BookTitle{}}

\vspace{1.5cm}
\noindent\htitlespace{\BookSource}

\smallskip
\noindent\htitlespace{\BookAuthors}



\vfill
\noindent\htitlespace{\scmain \City{} \bouillondot{} \Year}
\clearpage

\thispagestyle{empty}
{\footnotesize\noindent{\bfseries{}Маркс, Карл}

Капітал: Критика політичної економії / Заг. ред. А.~Потапова, І.~Зробок, Д.~Потапова. — \City, \Year. — \Biblio —
\lastpageref{pagesLTS.roman} + \lastpageref{pagesLTS.arabic} с.
(Готується до друку).
}
\vfill
\noindent{\footnotesize Друкується за виданням: Карл Маркс. Капітал: Критика політичної економії — \VolumeNumber. \BookNumber: \BookTitle. \BookSource{} \BookAuthors{} — \SourceYear{}.
}


\bigskip
\noindent{\footnotesize Цей твір ліцензовано на умовах Ліцензіїї Creative Commons Із Зазначенням Авторства — Поширення На Тих Самих Умовах 4.0 Міжнародна. Щоб ознайомитися з копією цієї ліцензії, завітайте на http://creativecommons.org/licenses/by-sa/4.0/ або направте листа за адресою Creative Commons, PO Box 1866, Mountain View, CA 94042, USA.}
\clearpage
\tableofcontents*
\cleardoublepage
\pagestyle{mypage}
\nonumsection{Подяки}{}{}

% Друкується за виданням: Карл Маркс. Капітал. Том І. Процес продукції капіталу. Переклад з четвертого німецького виддання за редакцією Д.~Рабіновича і С.~Трикоза. Харків, 1933, Партвидав «Пролетар».

\noindent{}Особлива подяка \textbf{Миколі Климчуку} за те,
що це видання вийшло охайним і гармонійним. Над виданням працювали:
\begin{itemize}[nosep]
\item \textbf{Ірина Зробок}
\item \textbf{Ернест Гук}
\item \textbf{Антон Потапов}
\item Юрий Латыш
\item Taras Bilous
\item Ivanna Kutsil
\item Max Starchevsky
\item Dan Bogynski
\item Dmytro Zhelaha
\item Богдан Бернадський
\end{itemize}
\noindent{}а також Nina Garbo, Андрій Андросович, Liza Walther, Сергій Зінченко,
Mike A. Liakh, Stas Sergienko, Volodymyr Boiko, Andriy Panchenkov, Денис Кучеренко, Настя Авдоніна, Oleg Kavaler, Volodymyr Shostak, Ілля Токар, Vika Khomovska, Maxim Sokhatsky, Mariana Potapova, Danylo Yankovskyi, Anna Potasheva, Yaroslav Kovalchuk, Денис Панкратов, Роман Козлов, \textenglish{Yevhenii Mo\-nas\-tyr\-skyi}, Andrew Zukkermann, Олександр Брайко, Anton Potapenko, Oleksandr Lapchuk, Ліда Криштоп, Anton Stepankovsky, Anton Pechenkin, Wowhura Wowhura, Ann Kurovska, Надія Йовченко, Ksena Meyta, Taras Salamaniuk, Сергей Алушкин, Kirill Kramskiy, Eugenia Virlich, Valeriy Kuropyatnik, Наталка Чех, Artem Tidva, Snizhana Umanets, Liuba Kuibida, Andriy Pogasiy, Yulia Dukach, Дмитро Вершинін, Леонид Бегунов-Новиков, Галина Новосад, Artem Borysov, Oleksandr Nykolyak, Петро Садовий, Mikhail Khokhlovych, Oleksii Parfeniuk, Hlafira Titarenko, Володимир Гунько, Денис Яшный, Картина Мира, Валентин Германович Дупак, Classic Starr, Olena Martynchuk, Zakhar Popovych, Polina Vlasenko, Vasya Opechenik та Денис Потапов.

\cleardoublepage
  
% ЧОГО vtu ХОЧЕМО? 
% Вперше без підпису надруковано польськаю мовою в газеті «Ргаса», 1879, 18 серпня, під назвою «Czego my chcemy?». В перекладі українською мовою вперше надруковано у вид.: Франко І. Твор и. В 20-ти т., т, 19, с. 215-217. ПодаЕться за першодруком. 
% ВЛАСНІСТЬ ГРУНТОВА І Уі ІСТОРІЯ 
% Вперше надруковано в кн.: Лавле Е. де. Власність грунтова і її історія. Переклав Іван Франко. Львів, 1879 («Дрібна бібліотека», VI), с. 34. (Передмова до книги). ПодаЕться за першодруком. С. 28. лавеле  Еміль де (1822-1892) — бельгійський бур-жуаsний iсторик i •    економlст. Б ю к е р Карл (1847-1930) — німецький буржуазний еконо• міст, історик народного господарства і статистик. 
% ДОПОВНЕННЯ ДО «ОСНОВ СУСПІЛЬНОУ 
% ЕкономІт" 
% Вперше надруковано в журн. «Культура», 1926, Ns 4-9, с. 56-57, та в кн.: Іван Франко, К., 1926, с. 164-166. Це герша передмова до перекладу XXIV розділу «Капіталу» К. Маркса, доданого І. Франком до написаного ним підручника «Ос• нови суспільної економії», який мав вийти у світ в кінці 1879 — на початку 1880 р., але не був надрукований; рукопис його загуб- лено. ПодаЕться за автографом, який зберігся в архіві І. Франка ((р. 3, Nё 448). 
% ГДРУГА ЛЕРЕДМОВА ДО ПЕРЕКЛАДУ 
% 24-го РОЗДІЛУ ПРАцІ К. МАРКСА «КАП[ТАЛв. т. 1] 
% Вперше надруковано в ж,урн. «Культура», 1926, Ns 4-9, с. 57-58, та в кн.: Іван Франко. К., 1926, с. 167---168. Передмову до українського перекладу 24-го розділу «Капі- талу» К. Маркса, який І. Франко мав намір видати окремим ви- пуском «Дрібної бібліотеки» (див. коментар до перекладу в цьому томі)подаеться написано, ймовірно, на початку 1880 року.  за автографом, який э6ерігаЕться в архіві І.Франка, ф. 3, Ns 448. С. 32. ...щоби сама гграця стала товаром...- Тут франківський виклад змісту першого тому «Капіталу» К. Марк- са неточний. К. Маркс мав на уваэі не працю, а робоцу силу. С. 33. Текст перекладу подаЕться у розділі «3 наукових пере- кладів» (с. 581--609). 
% 616 



\section*{Доповненя до „Основ суспільної економії“\protect\footnotemarkZ{}}
\nonumsectioncft{Доповненя до „Основ суспільної економії“}{.~}{Іван Франко}

\footnotetextZ{Вперше надруковано в журн. «Культура», 1926, № 4--9, с. 56--57, та в кн.: Іван Франко, К., 1926, с.~164--166.

Подається за автографом: відділ рукописних фондів і текстології Інституту літератури ім. Т.~Г.~Шевченка НАН України. — Ф. 3. — Од. зб. 448. — 14 арк. 
}

\noindent{}В самім початку „Основ суспільної економії“ сказано було, що економія, се наука абстрактна, т. є. що ціль єї не є виключно — розслідити закони економічні \emph{теперішної} суспільности, але \emph{загальні} закони праці людської. А позаяк с переміною суспільного ладу в протягу віків і закони ті проявляются щораз то в інших формах, випливаючих конечно з даного ладу, то наука економічна не може ніякої с тих форм вважати сталою і незмінною. Не може, значит, і нинішних форм уважати сталими, а мусит шукати таких форм, котрі \emph{після нашого теперішного знаня} булиб відповіднійші для суспільної праці і суспільного добробутку, ніж нинішні форми.

С тої то причини в сістематичнім викладі основ сусп. економії ми не могли давати надто широкого місця вислідам про \emph{нинішний} лад, а ограничились тілько головним єго нарисом. При викладі абстрактної теорії праці се була конечна річ, — але прецінь ніхто не заперечит, що на практиці для кождого дуже важне — знати передовсім докладно теперішний лад, єго почин і розвиток. Таке знанє вже тим корисне, що замісць теоретичних засад подає масу фактів, котрі самі прут розум до таких а таких виводів, між тим коли ті самі виводи, подані без підставних фактів, усякому можут видатися хиткими та схопленими з воздуха мріями. Для того то думаєм ми, що поповнимо подекуди конечний недостаток теоретичного викладу, подаючи в „Доповнених“ обширнійший огляд деяких питань, не порушених або з боку ткнених в самім викладі.

Одна з найважнійших недостач усякого чисто теоретичного викладу та, що приходится виключати з него всякі ширші \emph{історичні} перегляди. Правда, се не є недостача конечна, бо остаточно мож би бути вірним теорії, подаючи перегляд розвитку та впадку всіх економічних порядків від почину цівілізації аж до тепер. Але не кажучи вже о тім, що для такої загальної історії економічного розвитку призбирано доси дуже ще мало матеріялу, — в нашім підручнику такий виклад був би неможливий вже й за недостачею місця. А говорити обширно про розвиток одного — ниніншого — ладу, не казавши нічо про розвиток їнчих, се значилоб вважати сей лад чимось важнійшим від прочих, між тим коли в історії, як і в зрості кождого орґанізму, кожда фаза розвитку для вислідника рівноважна.

Але вважаючи потрібним познайомити наших читателів з історичним розвитком сучасного, капіталістичного ладу, ми робимо се в „Доповненях“. А для своєї ціли ми не можем найти кращого провідника над Карля Маркса, котрий в однім розділі своєї книжки „Das Kapital“ списав короткий, хоть яркий перегляд того, як розвивалася капіталістична продукція. С тим розділом ми й хочемо познакомити наших читателів.


\section*{[Друга передмова до перекладу 24-го розділу праці К.~Маркса «Капітал», т. І]\protect\footnotemarkZ{}}
\nonumsectioncft{[Друга передмова до перекладу 24-го розділу праці К.~Маркса «Капітал», т. І]}{.~}{Іван Франко}

\footnotetextZ{Вперше надруковано в журн. «Культура», 1926, № 4--9, с. 57--58, та в кн.: Іван Франко, К., 1926, с.~167--168.

Подається за автографом: відділ рукописних фондів і текстології Інституту літератури ім. Т.~Г.~Шевченка НАН України. — Ф. 3. — Од. зб. 448. — 14 арк. 
}

\noindent{}В першій части своєї великої економічної праці про „Капітал“ стараєсь Карль Маркс вияснити передовсім, \emph{як повстає капітал}? В тій ціли виказує він поперед усего, що єдиним жерелом усякої вартости є праця людська, котра з матеріалів сирих, даних природою, і при помочи сил природи витворює предмети вжиточні для чоловіка. Коли предмети такі витворюются не для власного вжитку самого витвірця, а для заміни за їнші, тоді вони звутся товарами. Капіталістична продукція полягає на витворюваню товарів, але не всяка продукція, де витворюются товарі, є вже капіталістична. До того потрібно ще одної дуже важної вимінки: \emph{щоби сама праця стала товаром}, т. є. щоб на торзі за певний товар (гроші) мож було заміняти (купити) працю людську.

Звичайно під назвою капіталу у нас розуміются беззглядно гроші. Се по части хибно. Гроші, як бачимо, тоді тілько стают капіталом, коли за них купуєся на торзі робуча сила.

Але праця людська, се не є звичайний товар. Се товар живий, котрий має тоту властивість, що \emph{надає вартість} другим предметам, і надає єї більше, ніж кілько сам коштує. Торгова ціна праці, так як і ціна кождого товару, означена звичайними економічними правилами, с котрих найважнійше — кошт витвореня товару, т. є. в тім разі — кошт удержаня робітника і єго робучої сили. Таку ціну платит капіталіст робітникови за єго працю. Між тим робітник в тім часі, на котрий нанявся, витворює далеко більше, ніж кілько виносит єго плата. Він витворив \emph{надзвишку вартости} понад вартість своєї плати, — тота надзвишка, се зиск капіталіста, — вона побільшує єго капітал. Значит, уся капіталістична продукція полягає на твореню надзвишки, котра задармо дістаєсь капіталістови. Цілий розвиток економічний капіталістичної продукції полягає на тім, що капіталісти всіми силами старалися до крайної можности вбільшити тоту надвишку. Вбільшити єї мож було двома способами: або продовжуючи день робучий (надвишка абсолютна), або приневолюючи робітників в коротшім часі працювати з більшою натугою (релятівна надвишка). Оба ті способи витрібували капіталісти, і то перший з них (продовженє робучого дня) до такої крайности, що аж уряд, затрівожений робітницькими розрухами, мусів вдатися в те діло і ограничити стало довготу робучого дня. Від тоді капіталістична продукція і доси пре в другий бік, — стараєсь той означений правно день робучий як найдоскональше використати, раз-ураз заводячи нові машини, котрі до крайности упрощуют і прискорюют продукцію, а до обслуги вимагают як найменшого числа рук.

Се головні думки, виведені Марксом з безмірної маси фактів, нагромаджених в єго книжці. При кінци книжки розбирає він ще одно важне питане: Яким способом почалася тота капіталістична продукція? Як і на якім ґрунті та при якій управі виріс той дивний порядок, оснований на щоденнім хитрім визиськуваню, на крайній бідносте незлічимих мас народа, а крайнім богацтві немногих щасливців? Сесь важний розділ Марксової книжки — прекрасний культурно-історичний очерк — зрозумілий буде і окремо від цілої книжки і ми хочемо познакомити з ним нашу громаду, як для самої єго великої стійности наукової, так і для того, щоб заохотити всіх, хто тілько владає німецькою мовою, до читаня цілої Марксової книжки. Звичайно говорится про дуже трудний і незрозумілий спосіб писаня у Маркса. Се мож би сказати хіба про перший розділ єго книжки, — а о?кілько такий суд справедливий що до прочих розділів, най посвідчит тота часть, котра отсе переведена.
  \pagenumbering{arabic}
  \mainmatter
  \thispagestyle{empty}
\null\vspace{5cm}


\noindent\htitlespace{\scmain Присвячується моєму незабутньому другові}

\smallskip
\noindent\htitlespace{\scmain Сміливому, вірному, благородному,}

\smallskip
\noindent\htitlespace{\scmain Передовому борцеві пролетаріяту}

\medskip
\noindent\htitlespace{\Large\scmain Вільгельмові Вольфу}

\smallskip
\noindent\htitlespace{\scmain Народився в Тарнау 21 червня 1809 року}

\smallskip
\noindent\htitlespace{\scmain Помер на вигнанні в Менчестері 9 травня 1864 року}

\cleardoublepage
  
\index{iii1}{0009}  %% посилання на сторінку оригінального видання
\section*{Передмова}

Нарешті мені удалось опублікувати цю третю книгу головного
твору Маркса, закінчення теоретичної частини. При виданні
другої книги в 1885 році я гадав, що третя книга становитиме
хіба тільки технічні труднощі, за винятком, звичайно, деяких
дуже важливих відділів. Так воно й було в дійсності; але про ті
труднощі, які мене чекали саме в цих найважливіших відділах
цілого, я не мав тоді ніякого уявлення, так само як і про інші
перешкоди, які так дуже загаяли виготовлення книги.

Перш за все і більш за все мені заважала постійна слабість
зору, яка протягом багатьох років обмежувала до мінімуму мій
робочий час для письмових занять, та ще й тепер тільки винятками
дозволяє мені брати в руки перо при штучному освітленні.
До цього долучилися інші невідкладні роботи: нові видання й
переклади попередніх праць Маркса та моїх, отже, перегляди,
передмови, доповнення, часто неможливі без нових досліджень,
і т. д. Передусім англійське видання першої книги, за текст
якого кінець-кінцем відповідаю я і яке через це забрало в мене
багато часу. Хто скількинебудь стежив за колосальним ростом
інтернаціональної соціалістичної літератури протягом останніх
десяти років і особливо за числом перекладів раніших праць
Маркса та моїх, той визнає, що я мав підставу вітати себе
з тим, що число тих мов, де я міг бути корисним перекладачеві
і, отже, був зобов’язаний не відмовлятись від перегляду його
праці, дуже обмежене. Але ріст літератури був тільки симптомом
відповідного зростання самого інтернаціонального робітничого
руху. А це останнє накладало на мене нові обов’язки.
З перших днів нашої громадської діяльності чимала частина праці
щодо посередництва між національними рухами соціалістів і робітників
у різних країнах падала на мене і Маркса; ця праця зростала
відповідно до зміцнення всього руху. Але тимчасом як
до самої своєї смерті Маркс і в цьому головний тягар праці
брав на себе, після його смерті постійно наростаючу працю довелося
виконувати мені одному. Між тим, безпосередні зносини
окремих національних робітничих партій між собою стали загальним
правилом, і на щастя, з дня на день стають ним дедалі більше;
не зважаючи на це, моя допомога потрібна ще далеко частіше,
ніж мені хотілося б в інтересах моїх теоретичних праць. Але
\parbreak{}  %% абзац продовжується на наступній сторінці


\index{iii2}{0010}  %% посилання на сторінку оригінального видання
Самостійний рух вартости цих титулів власности, не тільки державних
фондів, але й акцій, потверджує ілюзію, ніби вони становлять дійсний капітал
поряд того капіталу або тієї вимоги, що їхніми титулами можливо є вони.
Вони стають власне товарами, що їхня ціна має власний рух та власне усталення.
їхня ринкова вартість одержує відмінне від їхньої номінальної вартости
визначення, без того, щоб змінювалась вартість (хоч і змінюється зріст вартости)
дійсного капіталу. З одного боку, їхня ринкова вартість коливається разом з
висотою та певністю тих доходів, що на них вони дають правний титул. Коли
номінальна вартість якоїсь акції, тобто витрачена сума, що її первісно представляла
та акція, становить 100 ф. ст., а підприємство замість 5\% дає 10\%,
то її ринкова вартість серед решти однакових обставин та при рівні проценту
в 5\% зростає до 200 ф. ст., капіталізована бо з 5\%, вона становить тепер фіктивний
капітал в 200 ф. ст. Хто купує її за 200 ф. ст., одержує 5\% доходу
від цього приміщення капіталу. Навпаки буває, коли дохід підприємства меншає.
Ринкова вартість цих паперів почасти спекулятивна, бо вона визначається не
тільки дійсним доходом, але й сподіваним, наперед обрахованим. Але, коли припустити,
що розмір зростання вартости дійсного капіталу є постійний, або коли там,
де жодного капіталу немає, як от в державних боргах, вважати щорічний дохід
за зафіксований законом та загалом за досить забезпечений, то ціна цих цінних
паперів підноситься або спадає зворотно проти піднесення або спаду
рівня проценту. Якщо рівень проценту зростає від 5\% до 10\%, то якийсь
цінний папер, що забезпечує 5 ф. ст. доходу, становитиме капітал тільки
в 50 ф. ст.. Коли рівень проценту спаде до 2 1/2\%, то той самий цінний
папер становитиме капітал в 200 ф. ст. Його вартість є завжди тільки капіталізований
дохід, тобто дохід, обчислений від ілюзорного капіталу за наявним
рівнем проценту. Отже, підчас скрути на грошовому ринку ці цінні папери
падатимуть у ціні подвійно; поперше, тому, що рівень проценту зростає, а подруге
тому, що їх масами викидають на ринок, щоб реалізувати їх у грошах.
Цей спад ціни відбувається незалежно від того, чи розмір доходу, забезпечуваний
цими паперами своєму державцеві, є постійний, як от в державних фондах,
або чи зростання вартости того дійсного капіталу, що його вони представляють,
порушується перешкодами у процесі репродукції, як от це може бути в промислових
підприємствах. В останньому разі до вже згаданого знецінення долучається
тільки ще нове. Коли буря минулася, ці папери знову підносяться до своєї
колишньої висоти, оскільки вони не представляють зруйнованих або шахрайських
підприємств. їхнє знецінення підчас кризи впливає як міцний засіб до централізації
грошового майна\footnote{[Безпосередньо по лютневій революції, коли в Парижі товари й цінні папери були вкрай
знецінені та їх зовсім не можна було продати, один швайцарський купець в Ліверпулі, пан Р.
Цвільхенбарт
(що оповів про це моєму батькові) повернув на гроші все що міг, поїхав з готівкою до Парижу
та удався до Ротшільда з пропозицією зробити спільно ґешефт. Ротшільд пронизливо глянув на нього та,
кинувшись до нього, ухопив його за плечі: «Avez-vous de l’argent sur vous? — Oui M. le baron! —
Alors vous-êtes-mon homme! — («Маєте гроші? — Так, пане бароне! — Тоді ви мій чоловік!»).
І вони спільно зробили блискучий ґешефт. — Ф. Е.].}.

Оскільки знецінення або піднесення вартости цих паперів не залежить
від руху вартости того дійсного капіталу, що його вони представляють, багатство
нації лишається по знеціненні або піднесенні вартости таке саме, як і перед
тим. «23 жовтня 1847 року державні фонди й акції каналів та залізниць знецінилися
на 114.725.255 ф. ст.» (Morris, управитель Англійського банку, свідчення
у звіті про Commercial Distress 1847—48 р.). Оскільки їхнє знецінення не
свідчило про дійсний спин продукції та комунікації на залізницях та каналах,
або про ліквідацію вже початих підприємств, або про марне кидання капіталу в
\parbreak{}  %% абзац продовжується на наступній сторінці

\input{_0011.tex}

\index{iii1}{0012}  %% посилання на сторінку оригінального видання
Головну трудність становив відділ V, в якому до того ж
розглядається найзаплутаніший предмет всієї книги. І саме тоді,
коли Маркс розробляв цей відділ, його спіткав один з вищезгаданих
тяжких приступів хвороби. Отже, тут ми маємо не готовий
нарис, навіть не схему, обриси якої треба було б заповнити,
а тільки початок оброблення, який у багатьох випадках зводиться
до невпорядкованої купи заміток, уваг, матеріалів у формі
витягів. Спочатку я пробував закінчити цей відділ, як це мені
до певної міри вдалося з першим відділом, шляхом заповнення
прогалин і розроблення уривків, що були тільки намічені, так щоб
він хоч би приблизно давав усе те, що намірявся дати автор. Щонайменше
тричі я робив таку спробу, але кожного разу зазнавав
невдачі, і в утраченому таким чином часі є одна з головних причин
запізнення. Нарешті, я переконався, що цим шляхом справа
не піде. Мені довелося б переглянути всю масу літератури з цієї
галузі, і, кінець-кінцем, я виготував би щось, що все ж не було б
книгою Маркса. Мені не лишалось нічого іншого, як в певному
розумінні розрубати Гордіїв вузол, обмежитись тим, щоб по
можливості впорядкувати наявне і зробити тільки найпотрібніші
доповнення. І таким чином я навесні 1893 року закінчив головну
роботу для цього відділу.

З окремих розділів розділи 21—24 були в найголовнішому
розроблені. Розділи 25 і 26 вимагали перегляду ілюстрацій і
включення матеріалу, який був в інших місцях. Розділи 27 і 29 можна
було дати майже цілком за рукописом, навпаки — розділ 28 довелося
місцями інакше згрупувати. Але справжня трудність почалася
з розділу 30. Починаючи звідси, треба було належним чином
впорядкувати не тільки ілюстративний матеріал, але й хід думок,
який кожної хвилини переривався вставними реченнями, відхиленнями
і т. д. і розвивався далі в іншому місці, часто цілком мимохідь.
Таким чином розділ 30 склався шляхом перестановок та вилучень,
для яких знаходився вжиток в іншому місці. Розділ 31 був знову
більше розроблений у загальному зв’язку. Але далі в рукопису
йде довгий відділ, названий „Плутанина“ („Die Konfusion“), що
складається виключно з витягів з парламентських звітів про кризи
1848 і 1857 рр., в яких згрупованої місцями коротко юмористично
коментовані судження двадцяти трьох дільців і письменниківекономістів,
а саме про гроші й капітал, про відплив золота, надмірну
спекуляцію і т. д. Тут представлені, чи тими, що запитують,
чи тими, що відповідають, майже всі ходячі погляди того
часу на відношення між грішми і капіталом, і Маркс хотів критично
й сатирично розглянути „плутанину“, яка виявилась при цьому,
щодо того, що є на грошовому ринку гроші і що є капітал.
Після багатьох спроб я переконався, що виготовлення цього
розділу неможливе; матеріал, особливо матеріал, коментований
Марксом, я використав там, де це дозволяв зв’язок викладу.

Після цього йде в досить упорядкованому вигляді те, що я
вмістив в 32 розділі, але безпосередньо за цим — нова купа витягів
\index{iii1}{0013}  %% посилання на сторінку оригінального видання
з парламентських звітів про всякі можливі речі, зачеплені
в цьому відділі, перемішана з довшими чи коротшими увагами
автора. Наприкінці витяги та коментарії все більше й більше
концентруються коло руху грошових металів та вексельного
курсу і знов закінчуються всякого роду додатками. Навпаки, розділ
(36) „Передкапіталістичні відносини“ був цілком оброблений.

З усього цього матеріалу, починаючи з „Плутанини“, і оскільки
його не було вже вміщено в попередніх місцях, я склав
розділи 33—35. Звичайно, тут не обійшлося без значних вставок
з мого боку для встановлення зв’язку. Оскільки ці вставки
не чисто формального характеру, вони прямо позначені як мої.
Таким способом мені, нарешті, вдалося умістити в тексті всі так
чи інакше належні до справи судження автора; нічого не випущено,
крім незначної частини витягів, які або тільки повторювали
наведене вже в іншому місці, абож торкалися пунктів, яких
рукопис докладно не розглядав.

Відділ про земельну ренту був далеко повніше оброблений,
хоч і зовсім не впорядкований, як це видно вже з того, що
в 43 розділі (в рукопису кінець відділу про ренту) Маркс
вважав за потрібне коротко повторити план всього відділу. І це
було тим більш бажаним для видання, що рукопис починається
розділом 37, після якого йдуть розділи 45—47, і тільки після
цього розділи 38—44. Найбільше праці потребували таблиці
при диференціальній ренті II і те відкриття, що в 43 розділі
зовсім не був досліджений третій випадок цього роду ренти,
який треба було тут розглянути.

Для цього відділу про земельну ренту Маркс у семидесятих
роках взявся до цілком нових спеціальних досліджень. Протягом
ряду років він вивчав в оригіналах статистичні досліди
та інші видання про землеволодіння, які стали неминучими
в Росії після „реформи“ 1861 року і які йому постачали в бажаній
повноті його російські друзі, робив з них виписки і намірявся
їх використати при новому обробленні цього відділу. При
різноманітності форм як землеволодіння, так і експлуатації
землеробських виробників у Росії, у відділі про земельну ренту
Росія мала відігравати таку саму роль, як в першій книзі, при
розгляді промислової найманої праці, Англія. На жаль, Марксу
не вдалося здійснити цей план.

Нарешті, сьомий відділ був цілком закінчений у рукопису,
але тільки як перший нарис, безконечно заплутані періоди якого
спочатку треба було розчленувати, щоб зробити їх придатними
до друку. Від останнього розділу існує тільки початок. Тут
малося розглянути відповідні трьом головним формам доходу:
земельна рента, зиск, заробітна плата — три великі класи розвиненого
капіталістичного суспільства: землевласники, капіталісти,
наймані робітники — і неминуче дану з їх існуванням класову
боротьбу як фактично наявний результат капіталістичного
періоду. Подібні кінцеві резюме Маркс мав звичай відкладати
\parbreak{}  %% абзац продовжується на наступній сторінці

\input{_0014c.tex}
\parcont{}  %% абзац починається на попередній сторінці
\index{ii}{0015}  %% посилання на сторінку оригінального видання
авансовані гроші функціонують як грошовий капітал, бо в наслідок
циркуляції вони перетворюються на товари специфічної споживної
вартости. Тут товар може функціонувати як капітал лише остільки,
оскільки вже з продукційного процесу раніш, ніж почалась його циркуляція,
він набув характеру капіталу. Протягом процесу прядіння утворили
прядільники вартість пряжі в 128\pound{ ф. стерл}. З них, припустімо, 50 фун.
стерл. становлять для капіталіста просто еквівалент за його витрати на
робочу силу, а 78\pound{ ф. стерл.} — при рівні експлуатації робочої сили в 156\%
— становлять додаткову вартість. Отже, вартість 10.000 ф. пряжі містить у
собі, поперше, вартість зужиткованого продуктивного капіталу П, що з неї
стала частина дорівнює 372\pound{ ф. стерл.}, змінна = 50\pound{ ф. стерл.}, їх
сума = 422\pound{ ф. стерл.}, — дорівнює 8.440 ф. пряжі. Але вартість
продуктивного капіталу П дорівнює Т, вартості його складових елементів,
що на стадії $Г — Т$ протистояли капіталістові як товари в руках їхніх
продавців. — Але, подруге, вартість пряжі містить у собі додаткову вартість,
в 78\pound{ ф. стерл.} = 1560 ф. пряжі. Отже, $Т$, як вираз вартости 10.000 ф.
пряжі, дорівнює $Т + ΔТ$, $Т$ плюс приріст $Т$ (= 78\pound{ ф. стерл.}), що його ми й
позначимо $т$, бо він існує в тій самій товаровій формі, в якій тепер існує
первісна вартість $Т$. Вартість 10.000 ф. пряжі дорівнює 500\pound{ ф. стерл.}, отже,
вона дорівнює $Т + т = Т'$. Що перетворює $Т$, як вираз 10.000 ф. пряжі, на
$Т'$ — це зовсім не абсолютна величина його вартости (500\pound{ ф. стерл.}), бо вона,
як і в усіх інших $Т$, оскільки вони є вираз вартости певної суми якихбудь
інших товарів, визначається кількістю зречевленої в ньому праці.
А перетворює його відносна величина його вартости, величина його
вартости порівняно з вартістю капіталу $П$, зужиткованого на його
продукцію. $Т'$ містить у собі цю останню вартість плюс додаткову
вартість, подану продуктивним капіталом. Його вартість більша, перевищує
цю капітальну вартість на цю додаткову вартість, $т$. 10.000 ф.
пряжі — це носії вирослої у своїй вартості, збагаченої додатковою вартістю
капітальної вартости, і вони являють собою таких носіїв, як
продукт капіталістичного продукційного процесу. $Т'$ виражає відношення
вартостей — відношення вартости товарового продукту до
вартости капіталу, витраченого на його продукцію; отже, виражає, що
його вартість складається з капітальної вартости й додаткової вартости.
10.000 ф. пряжі є товаровий капітал, $Т'$, лише як перетворена форма
продуктивного капіталу $П$, отже, лише у зв’язку, що насамперед існує
тільки в кругобігу цього індивідуального капіталу, або лише для того
капіталіста, що своїм капіталом продукував пряжу. Це, так би мовити,
лише внутрішнє, а не зовнішнє відношення, що робить ці 10.000 ф.
пряжі як носіїв вартости товаровим капіталом. Капіталістична родинка
цих 10.000 ф. пряжі не в абсолютній величині їхньої вартости, а в її відносній
величині, у величині їхньої вартости порівняно з тією, яку мав
продуктивний капітал, що містився в них раніше, ніж він перетворився
на товар. Тому, коли 10.000 ф. пряжі продається за їхню вартість, за
500\pound{ ф. стерл.}, то цей акт циркуляції, розглядуваний сам по собі, $= Т — Г$,
є просте перетворення вартости, що лишається незмінною, з товарової форми
\parbreak{}  %% абзац продовжується на наступній сторінці

\input{_0016_0017_0018.tex}
\parcont{}  %% абзац починається на попередній сторінці
\index{ii}{0019}  %% посилання на сторінку оригінального видання
стерл.) існує тепер реалізований товаровий капітал у руках капіталіста.
Капітальна вартість і додаткова вартість існують тепер як гроші,
отже, в загальній еквівалентній формі.

Отже, наприкінці процесу капітальна вартість перебуває знову
в тій самій формі, що в ній вона увійшла в нього, і, отже, може знову,
як грошовий капітал, розпочати й перебігати цей процес. Саме тому, що
початкова й кінцева форма процесу являє собою форму грошового капіталу
($Г$), цю форму процесу кругобігу названо нами кругобігом
грошового капіталу. Кінець-кінцем, змінилась не форма, а лише величина
авансованої вартости.

$Г + г$ є не що інше, як грошова сума певної величини, в даному разі
500\pound{ ф. стерл}. Але як наслідок кругобігу капіталу, як реалізований
товаровий капітал, ця грошова сума має в собі капітальну вартість
і додаткову вартість, і до того ж вони вже не зрослі одна з однією, як
у пряжі, а лежать тепер поряд. Реалізація їх дала кожній з них самостійну
грошову форму. \sfrac{211}{250} цієї суми є капітальна вартість, 422\pound{ ф.
стерл.}, і \sfrac{39}{250} її є додаткова вартість в 78\pound{ ф. стерл}. Це відокремлення,
спричинене реалізацією товарового капіталу, має не тільки формальний
зміст, про що ми казатимемо зараз; воно набирає важливости в процесі
репродукції капіталу, залежно від того, чи долучається $г$ до $Г$ цілком, чи
почасти, чи зовсім не долучається, отже, залежно від того, чи функціонує
воно далі як складова частина авансованої капітальної вартости, чи ні.
$г$ і $Г$ можуть також перебігати цілком різні циркуляції.

В $Г'$ капітал знову повернувся до своєї первісної форми $Г$, до своєї
грошової форми; але повернувся він у такій формі, що в ній він зреалізований
як капітал.

Поперше, тут є кількісна ріжниця. Було $Г$, 422\pound{ ф. стерл.}; тепер є $Г'$,
500\pound{ ф. стерл.}, і цю ріжницю виражено в $Г\dots{} Г'$, в кількісно різних
крайніх членах кругобігу, що його власний рух позначено лише крапками.
$Г'$ більше за $Г$, $Г'$ мінус $Г = М$, додатковій вартості. — Але як результат цього
кругобігу $Г\dots{} Г' т$епер існує лише $Г'$; це є результат, що в ньому погас
процес його утворення. $Г'$ існує тепер самостійно само по собі, незалежно
від руху, що породив його. Рух минув, натомість маємо $Г'$.

Але $Г'$, як $Г + г$, 500\pound{ ф. стерл.}, як 422\pound{ ф. стерл.} авансованого капіталу
плюс його приріст в 78\pound{ ф. стерл.}, являє разом з тим якісне відношення,
хоч саме це якісне відношення існує лише як відношення частин тієї
самої суми, отже, як кількісне відношення. $Г$, авансований капітал, що
тепер знову перебуває в своїй первісній формі (422\pound{ ф. стерл.}), існує
тепер як реалізований капітал. Він не лише зберігся, він також реалізувався
як капітал, бо саме як капітал відрізняється він від г (78\pound{ ф. стерл.})
що до нього він стосується, як до \emph{свого} приросту, до \emph{свого} витвору, до
породженого ним самим приросту. Він реалізувався як капітал, тому що
він реалізувався як вартість, що породила вартість. $Г'$ існує як капіталістичне
відношення; Г вже виступає не як просто гроші, а виразно як
\parbreak{}  %% абзац продовжується на наступній сторінці

\input{_0020.tex}
\input{_0021_0022.tex}

\index{ii}{0023}  %% посилання на сторінку оригінального видання
\subsection{Кругобіг у цілому}

Ми бачили, що процес циркуляції по скінченні його першої фази
$Г — Т\splitfrac{Р}{Зп} $ переривається через П, що в ньому товари Р і Зп, куплені
на ринку, споживається як речеві й вартісні складові частини продуктивного
капіталу; продукт цього споживання є новий товар, $Т'$, змінений
речево і щодо вартости. Перерваний процес циркуляції, $Г — Т$, мусить
доповнитись через $Т — Г$. Але як носій цієї другої та кінцевої фази циркуляції
з’являється $Т'$, товар відмінний від першого Т речево і щодо
вартости. Отже, ряд циркуляцій має такий вигляд: 1) Г — Т1; 2) Т2' — $Г'$,
де в другій фазі першого товару Т1, підчас перерви, зумовленої функцією
П, підчас продукції $Т'$ з елементів Т, з форм буття продуктивного
капіталу П, постає другий товар, вищої вартости та іншої споживної
форми, Т2'. Навпаки, перша форма виявлення, що в ній капітал виступив
перед нами (книга І, розділ IV, І), $Г — Т — Г'$ (розкладається
на: 1) Г — Т1; 2) Т1 — $Г'$), двічі показує той самий товар. Там перед
нами обидва рази той самий товар, на який перетворюються гроші
в першій фазі і який в другій фазі перетворюється на більшу кількість
грошей. Не зважаючи на цю посутню ріжницю, обидві циркуляції мають
те спільне, що в їхній першій фазі гроші перетворюються на товар, і
в їхній другій фазі товар перетворюється на гроші, отже, що гроші, витрачені
в першій фазі, зворотно припливають у другій фазі. З одного боку,
спільне у них — зворотний приплив грошей до свого вихідного пункту,
але, з другого боку, і те, що грошей зворотно припливає більше, ніж було
авансовано. В цьому розумінні $Г — Т\dots{} Т' — Г'$ вже міститься в загальній
формулі $Г — Т — Г'$.

Далі виявляється, що в обох належних до циркуляції метаморфозах
$Г — Т$ і $Т' — Г'$ кожного разу протистоять одна одній і заступають одна
одну рівновеликі, одночасно наявні вартості. Зміна величини вартости
належить виключно метаморфозі П, продукційному процесові, що таким
чином становить реальну метаморфозу капіталу протилежно простій формальній
метаморфозі циркуляції.

А тепер розгляньмо цілий рух $Г — Т\dots{} П\dots{} Т' — Г'$, або його розгорнуту
форму $Г — Т\splitfrac{Р}{Зп}\dots{} П\dots{} Т'$ (Т \dplus{} т) — $Г'$ (Г \dplus{} г). Капітал з’являється тут
як вартість, що перебігає ряд взаємно зв’язаних, одне одним зумовлених
перетворень, ряд метаморфоз, які являють стільки ж фаз або стадій цілого
процесу. Дві з цих фаз належать до сфери циркуляції, одна — до
сфери продукції. В кожній з цих фаз капітальна вартість перебуває в
особливій формі, що їй відповідає особлива, спеціяльна функція. В цьому
русі авансована вартість не лише зберігається, але й зростає, збільшує
свою величину. Нарешті, в кінцевій стадії вона повертається до тієї самої
форми, що в ній вона з’явилась на початку цілого процесу. Тому цей
цілий процес є процес кругобігу.

\input{_0024c.tex}

\index{iii1}{0025}  %% посилання на сторінку оригінального видання
Хоч який прекрасний і ясний вищенаведений обрахунок, ми
змушені, однак, поставити панові докторові Штібелінгові \emph{одно}
питання: звідки він знає, що сума додаткової вартості, яку виробляє
фабрика І, ні на волосок не відрізняється від суми додаткової
вартості, створеної на фабриці II? Про $с, v, у$ і $х$, отже,
про всі інші фактори обрахунку, він прямо каже нам, що вони
для обох фабрик мають однакові величини, але про $m$ ні слова.
Але з того, що він обидві згадувані тут кількості додаткової
вартості алгебрично позначає через $m$, це ніяк не випливає.
Навпаки, це саме те, що має бути доведене, бо пан Штібелінг
без дальших околичностей і зиск $p$ ототожнює з додатковою
вартістю. Тут можливі тільки два випадки: або обидва $m$ рівні,
кожна фабрика виробляє однакову кількість додаткової вартості,
отже, при однаковому сукупному капіталі і однакову кількість
зиску, і тоді пан Штібелінг вже наперед припустив те, що він
ще тільки повинен довести. Абож одна фабрика виробляє більшу
суму додаткової вартості, ніж друга, і тоді розвалюється весь
його обрахунок.

Пан Штібелінг не побоявся ні праці, ні витрат для того, щоб
на цій своїй помилці в обрахунку побудувати цілі гори обчислень
і подати їх публіці. Я можу дати йому заспокійливе запевнення,
що майже всі вони однаково неправильні, і що там,
де вони як виняток правильні, вони доводять щось цілком інше,
а не те, що він хоче довести. Так, порівнюючи дані американських
переписів 1870 і 1880 років, він дійсно показує падіння
норми зиску, але пояснює його цілком помилково і гадає, що
теорія Маркса про завжди незмінну, стабільну норму зиску має
бути виправлена практикою. Але з третього відділу цієї третьої
книги виходить, що ця „нерухома норма зиску“ Маркса є чиста
вигадка і що тенденція норми зиску до падіння грунтується на
причинах, діаметрально протилежних тим, що їх наводить д-р
Штібелінг. Пан д-р Штібелінг має, без сумніву, добрі наміри,
але, якщо хто хоче займатись науковими питаннями, то він мусить
насамперед навчитися читати твори, якими хоче користуватись,
так, як їх написав автор, і перш за все не вичитувати
з них того, чого в них немає.

Результат усього дослідження: і в даному питанні щось
зроблено знов таки тільки школою Маркса. Фіреман і Конрад
Шмідт, коли читатимуть цю третю книгу, можуть бути цілком
задоволені, кожний у своїй частині, з своїх власних праць.
\begin{flushright}
  \emph{Ф. Енгельс}
\end{flushright}
Лондон, 4 жовтня 1894 р.


  \bookpaget{\BookNumber{}}{\BookTitle{}}
  
\index{iii1}{0047}  %% посилання на сторінку оригінального видання

\chapter{Перетворення додаткової вартості в зиск і норми додаткової вартості в норму зиску}

\section{Витрати виробництва (kostpreis) і зиск}

В першій книзі були досліджені ті явища, які представляє капіталістичний
\emph{процес виробництва}, взятий сам по собі, як безпосередній
процес виробництва, при чому ще залишались осторонь
усі вторинні впливи чужих йому обставин. Але цей безпосередній
процес виробництва ще не вичерпує життьового шляху капіталу.
В дійсному світі він доповнюється \emph{процесом циркуляції},
який становив предмет досліджень другої книги. Там, саме в
третьому відділі, при розгляді процесу циркуляції як опосереднення
суспільного процесу репродукції, виявилось, що капіталістичний
процес виробництва, розглядуваний у цілому, є єдність
процесу виробництва і циркуляції. Завдання цієї третьої книги
не може полягати в тому, щоб дати загальні міркування про цю
єдність. Навпаки, тут треба знайти і описати ті конкретні форми,
які виростають з \emph{процесу руху капіталу}, \emph{розглядуваного як
ціле}. В своєму дійсному русі капітали протистоять один одному
в таких конкретних формах, для яких форма капіталу в безпосередньому
процесі виробництва, як і його форма в процесі циркуляції,
виступають тільки як особливі моменти. Отже, ті форми
капіталу, які ми описуємо в цій книзі, крок за кроком наближаються
до тієї форми, в якій вони виступають на поверхні
суспільства, в діянні різних капіталів один на одного, в конкуренції
і в звичайній свідомості самих діячів виробництва.

\pfbreak

Вартість кожного капіталістично виробленого товару $Т$ зображується
у формулі: $Т = c + v + m$. Якщо ми від цієї вартості
\parbreak{}  %% абзац продовжується на наступній сторінці


\index{iii2}{0048}  %% посилання на сторінку оригінального видання
[Поки стан справ є такий, що зворотний приплив зроблених авансувань
відбувається реґулярно, отже й кредит лишається незахитаним, пошир та
скорочення циркуляції реґулюється просто потребами промисловців та купців. Що
золото, принаймні в Англії, не має ваги для гуртової торговлі, а циркуляцію
золота — якщо не вважати на сезонні коливання — можна розглядати для довшого
часу як досить сталу величину, то й становить циркуляція банкнот Англійського
банку досить точне мірило ступеня цих змін. За тихих часів по кризі розмір
циркуляції є найменший, з новим оживленням попиту постає й більша потреба
на засоби циркуляції, ця потреба зростає з розвитком розцвіту; найвищої точки
кількість засобів циркуляції доходить в період надмірного напруження та надмірної
спекуляції, — тоді вибухає криза й за ніч зникають з ринку банкноти,
що їх ще вчора було багато, а з ними зникають і дисконтери векселів, і ті хто
дають гроші під цінні папери, і купці товарів. Англійський банк має допомагати,
— але й його сили незабаром вичерпані, банковий акт 1844 року змушує
його обмежувати циркуляції своїх банкнот саме тоді, коли весь світ криком
вимагає банкнот, коли державці товарів не можуть їх продавати, а проте повинні
платити та готові на всякі жертви, аби тільки одержати банкноти. «Підчас
переляку», каже вищезгаданий банкір Wright (1. с. № 2930), «країна потребує
удвоє більшої циркуляції, ніж за звичайних часів, бо банкіри й інші скупчують
собі про запас засоби циркуляції».

Скоро вибухав криза, справа вже тільки в платіжних засобах. А що
в надході цих платіжних засобів кожен залежить від іншого та ніхто не знає,
чи той інший в стані буде платити в реченець, то й настає справжня гонитва
за тими платіжними засобами, що є на ринку, тобто за банкнотами. Кожен
скупчує тих банкнот як скарб, скільки тільки він їх може одержати, і таким
чином банкноти зникають з циркуляції того самого дня, коли їх потребують
найбільше. Samuel Gurney (C. D. 1848/57, № 1116) визначає число банкнот,
прихованих під замок в момент паніки, в жовтні 1847 р. на суму 4—5 мільйонів
ф. ст. — Ф. Е.]

Щодо цього особливо цікаві свідчення перед банковою комісією 1857 року
спільника Gurney’ового, вже згаданого Chapman’а. Я подаю тут головний
зміст їх у зв’язному викладі, хоч в них розглядаються деякі пункти, що їх ми
дослідимо тільки пізніше. Пан Chapman дає таке свідчення.

«4963. Я не вагаючися скажу, що я не вважаю за доладне, коли грошовий
ринок має бути під владою будь-якого індивідуального капіталіста (а їх
в Лондоні є досить), що в стані утворювати величезну недостачу грошей та скруту
тоді, коли циркуляція є саме дуже низька... Це можливо... є не один капіталіст,
що може витягти з циркуляції банкнот на 1 чи 2 міл. ф. ст., якщо він
може тим досягти певної мети». 4995. Якийсь великий спекулянт може продати
консолів на 1 чи 2 міл. й таким способом забрати гроші з ринку. Дещо подібне
сталося зовсім недавно, «і це утворює незвичайно гостру скруту».

4967. Певна річ, банкноти тоді є непродуктивні. «Але нема чого тим
журитися, якщо таким способом можна осягнути великої мети; його велика
мета — збити ціни на фонди, утворити грошову скруту, а зробити це — цілком
в його силі». Приклад: одного ранку був великий попит на гроші на фондовій
біржі; ніхто не знав причини; хтось запропонував Chapman’oвi, щоб останній
позичив йому 50.000 ф. ст. з 7\%. Chapman здивувався, бо в нього рівень
проценту був значно нижчий; він згодився. Скоро по тому той чоловік прийшов
знову, взяв" знову 50.000 ф. ст. з 7 \sfrac{1}{2}\%, потім 100.000 ф. ст. з 8\% і хотів
взяти ще більшу суму з 8 \sfrac{1}{2}\%. Але тоді самого Chapman’a охопила тривога.
Потім виявилося, що раптом забрано з ринку значну суму грошей. Однак, каже
Chapman, «я проте визичив значну суму з 8\%; йти далі я боявся; я не знав,
що з того вийде».


\index{iii2}{0049}  %% посилання на сторінку оригінального видання
Не треба ніколи забувати, що хоч 19—20 мільйонів банкнотами, як
кажуть, перебуває в руках публіки досить стало, проте, та частина цих банкнот,
що є дійсно в циркуляції, з одного боку, і та частина їх, що лежить в банках
незайнята як запас, з другого боку, раз-у-раз та значно змінюються одна
проти однієї. Якщо цей запас великий, отже, рівень дійсної циркуляції низький,
то з погляду грошового ринку це значить, що сфера циркуляції є повна (the
circulation is full, money is plentifull); коли запас малий, отже коли рівень
дійсної циркуляції високий, то грошовий ринок зве його низьким (the circulation
is low, money is scarce); саме та частина являє низьку суму, що представляє
позичковий незайнятий капітал. Дійсний, від фаз промислового циклу незалежний,
пошир або скорочення циркуляції — так що однак та сума, що її потребує
публіка, лишається однаковою — буває лише з технічних причин, напр., коли настає
реченець платежа податків або процентів на державний борг. При платежі податків
банкноти та золото припливають до Англійського банку понад звичайну міру,
фактично скорочуючи циркуляцію, не зважаючи на потреби останньої. Навпаки
буває, коли виплачується дивіденди на державний борг. В першому випадку
роблять позики в банку на те, щоб добути засоби циркуляції. В останньому
випадку спадає рівень проценту в приватних банках з причини тимчасового
зросту їхніх резервів. Де не має нічого до діла з абсолютною масою засобів
циркуляції, а тільки має до діла з тією банковою фірмою, що пускає ці засоби
в циркуляцію і що з погляду її той процес видається вивласненням
позичкового капіталу, що й дає їй тому змогу ховати собі до кишені зиск
від того.

В одному випадку відбувається лише часове переміщення засобів циркуляції,
що його Англійський банк вирівнює тим способом, що незадовго перед реченцем
платежа чвертьрічних податків або виплати так само чвертьрічних дивідендів
він видає короткотермінові позики за низькі проценти; отож ці отак понад міру
видані банкноти спершу заповнюють ті прогалини, що їх викликав платіж
податків, тимчасом як їх зворотний платіж до банку зараз же по тому усовує
той надмір банкнот, що до його призводить виплата дивідендів публіці.

В другому випадку низький або високий рівень циркуляції завжди становить
лише інший розподіл тієї самої маси засобів циркуляції на активну циркуляцію
та вклади, тобто знаряддя позик.

З другого боку, коли, напр., через приплив золота до Англійського банку
більшає число банкнот, виданих за те золото, то ці останні допомагають
дисконтові поза банком та припливають назад на оплату позик, так що абсолютна
маса банкнот в циркуляції збільшується лише на короткий час.

Якщо циркуляція повна з причини поширу справ (що можливе й при порівняно
низьких цінах), то рівень проценту може бути, відносно високий з причини
попиту на позичковий капітал, що зумовлюється зростом зиску та збільшенням
змоги нових приміщень капіталу. Коли рівень циркуляції є низький в наслідок
скорочення справ або й великої поточности кредиту, то рівень проценту може бути
низький і при високих цінах (див. Hubbard).

Абсолютний розмір циркуляції впливає на рівень проценту, визначаючи
його, тільки підчас пригнічення. Тут попит на поширену циркуляцію означає
або лише попит на засоби до утворення скарбів (якщо не вважати на зменшену
швидкість, з якою гроші обертаються та з якою ті ж самі монети раз-у-раз
перетворюються на позичковий капітал) в наслідок відсутности кредиту, як от
в 1847 році, коли припинення банкового акту не викликало жодного поширу
циркуляції, але його вистачило, щоб нагромаджені скарбом банкноти знову
витягти на світ денний та кинути їх до циркуляції. Або ж у певних обставинах
дійсно може бути потрібно більше засобів циркуляції, як от в 1857 році, коли
циркуляція по припиненні банкового акту дійсно зросла на деякий час.


\index{iii2}{0050}  %% посилання на сторінку оригінального видання
В інших випадках абсолютний розмір циркуляції не впливає на рівень
проценту, поперше, тому, що — припускаючи економію та швидкість циркуляції
як сталі — той розмір циркуляції визначається цінами товарів та кількістю операцій
(при чому, здебільша, один момент паралізує вплив другого) та, насамкінець,
станом кредиту, тимчасом коли навпаки сам той розмір ніяк не визначає
цих факторів; і, подруге, тому, що товарові ціни та процент не мають між собою
ніякого неминучого зв’язку.

За тих часів, коли мав силу Bank Restriction Act (1797—1820 p. p.),
був надмір засобів циркуляції, рівень проценту був завжди далеко вищий, ніж
тоді, коли відновили платежі готівкою. Він швидко впав пізніше, коли обмежили
видання банкнот та підвищились вексельні курси. В 1822,1823, 1832 роках
загальний розмір циркуляції був низький, рівень проценту теж низький. В 1824,
1825, 1836 роках розмір циркуляції був високий, рівень проценту піднісся.
Улітку 1830 року циркуляція була висока, рівень проценту низький. Від часу
відкриття нових покладів золота розмір циркуляції грошей поширився по цілій
Европі, рівень проценту підвищився. Отже рівень проценту не залежить від
кількости грошей, що перебувають в циркуляції.

Ріжниця між випуском засобів циркуляції та визичанням капіталу найкраще
виявляється в дійсному процесі репродукції. Розглядаючи його, ми бачили (Книга II,
відділ III), як обмінюється різні складові частини продукції. Напр., змінний капітал
речово складається з життьових засобів робітників, з частини їхнього власного
продукту. Але його виплачують їм частинами у грошах. Ці гроші мусить
авансувати капіталіст, і від організації кредитової справи дуже залежить, чи
зможе він ближчого тижня знову виплатити новий змінний капітал старими
грішми, що він їх платив минулого тижня. Те саме бачимо ми в актах обміну
між різними складовими частинами сукупного суспільного капіталу, напр., між
засобами спожитку та засобами продукції тих засобів спожитку. Гроші, як ми
бачили, мусять для циркуляції авансуватися однією або обома особами, що обмінюються.
Потім гроші лишаються в циркуляції, але по закінченні обміну вони
раз-у-раз вертаються назад до того, хто їх авансував, бо він авансував їх зверх
свого дійсно занятого промислового капіталу (див. Книга II, 20 розділ). За розвинутої
кредитової справи, коли гроші концентруються в руках банків, ці останні,
принаймні, номінально, являють ту установу, яка авансує гроші. Це авансування
стосується тільки до тих грошей, що перебувають в циркуляції. Це — авансування
засобів циркуляції, а не авансування капіталів, що їх циркуляція обумовлюється
цим авансуванням.

Chapman: «5062. Може надійти час, коли банкноти в руках публіки становитимуть
дуже велику суму, а проте їх не можна добути». Гроші є й підчас
паніки; але кожен стережеться перетворювати їх на позичковий капітал, на
позичкові гроші; кожен міцно тримає їх для дійсної платіжної потреби.

«5099. Чи посилають банки сільських округ свої надміри вільних грошей
до вас та до інших лондонських фірм? — Так, — 5100. З другого боку, чи дисконтують
у вас фабричні округи Ланкашайру та Іїоркшайру векселі для своїх
промислових цілей? — Так. — 5101. Отже, цим способом зайві гроші однієї частини
країни стають пожиточні для потреб другої частини країни? — Цілком слушно».

Chapman каже, що звичай банків уживати свій надмірний грошовий капітал
на короткий час на купівлю консолів та посвідок державної скарбниці, цей звичай
за останній час дуже обмежився від того часу, коли стало звичаєм визичати ці
гроші at call (з дня на день, маючи змогу кожного часу вимагати їх назад).
Сам він вважає купівлю таких паперів для свого підприємства за незвичайно
недоцільну. Тому він приміщує гроші в добрі векселі, що для частини їх щодня
надходить реченець, так що він завжди знає, на скільки вільних грошей вік
має рахувати кожного дня. (5001—5005).
\parbreak{}  %% абзац продовжується на наступній сторінці

\input{_0051.tex}

\index{iii1}{0052}  %% посилання на сторінку оригінального видання
У цій формулі частина капіталу, витрачена на працю, відрізняється
від частини капіталу, витраченої на засоби виробництва,
наприклад, на бавовну або вугілля, тільки тим, що вона
служить для оплати речево відмінного елементу виробництва,
але ніяк не тим, що в процесі творення вартості товару, а тому
і в процесі зростання вартості капіталу вона відіграє функціонально
відмінну роль. У витратах виробництва товару ціна засобів
виробництва повертається назад такою, якою вона вже
фігурувала при авансуванні капіталу, і саме тому, що ці засоби
виробництва були доцільно використані. Цілком так само у витратах
виробництва товару ціна або заробітна плата за 666 \sfrac{2}{3}
робочих днів, витрачених на його виробництво, повертається
назад такою, якою вона вже фігурувала при авансуванні капіталу,
і так само якраз тому, що ця маса праці витрачена в доцільній
формі. Ми бачимо тільки готові, наявні вартості, — ті
частини вартості авансованого капіталу, які входять в утворення
вартості продукту, — але не бачимо елементу, який ств'орює
нову вартість. Ріжниця між сталим і змінним капіталом зникла.
Всі витрати виробництва у 500 фунтів стерлінгів набувають
тепер двоякого значення: поперше, вони є та складова частина
товарної вартості в 600 фунтів стерлінгів, яка заміщає капітал
у 500 фунтів стерлінгів, витрачений на виробництво товару;
і, подруге, сама ця складова частина вартості товару існує лише
тому, що вона раніш існувала як витрати виробництва застосованих
елементів виробництва, засобів виробництва і праці, тобто
як авансований капітал. Капітальна вартість повертається назад
як витрати виробництва товару тому і остільки, що і оскільки
її було витрачено як капітальну вартість.

Та обставина, що різні складові частини вартості авансованого
капіталу витрачені на речево різні елементи виробництва,
на засоби праці, сировинні й допоміжні матеріали і працю,
зумовлює тільки те, що на витрати виробництва товару доводиться
знову купити ці речево різні елементи виробництва.
Навпаки, щодо утворення самих витрат виробництва, то тут має
значення тільки одна ріжниця, ріжниця між основним і обіговим
капіталом. В нашому прикладі 20 фунтів стерлінгів були зараховані
на зношування засобів праці (400 с = 20 фунтам стерлінгів
на зношування засобів праці + 380 фунтів стерлінгів на
матеріали виробництва). Якщо вартість цих засобів праці перед
виробництвом товару була = 1200 фунтам стерлінгів, то після
його виробництва вона існує в двох виглядах: 20 фунтів стерлінгів
як частина товарної вартості, 1200—20, або 1180 фунтів стерлінгів,
як решта вартості засобів праці, які перебувають, як і раніш,
у володінні капіталіста, або як елемент вартості не його
товарного капіталу, а його продуктивного капіталу. В протилежність
до засобів праці, матеріали виробництва і заробітна плата
цілком витрачаються на виробництво товару, а тому і вся їх
вартість входить у вартість виробленого товару. Ми бачили,
\parbreak{}  %% абзац продовжується на наступній сторінці

\input{_0053_0054.tex}

\index{iii1}{0055}  %% посилання на сторінку оригінального видання
Тепер капіталістові ясно, що цей приріст вартості виникає
з продуктивних процесів, пророблених з капіталом, що він,
отже, виникає з самого капіталу; бо після процесу виробництва
він є, а перед процесом виробництва його не було. Насамперед,
щодо капіталу, витраченого на виробництво, то здається,
що додаткова вартість виникає рівномірно з різних елементів
його вартості, які існують у вигляді засобів виробництва і праці.
Адже ці елементи рівномірно входять в утворення витрат виробництва.
Вони рівномірно додають до вартості продукту свої
вартості, що наявні як авансування капіталу, і не відрізняються
один від одного як стала і змінна величини вартості. Це стає
цілком очевидним, коли ми на один момент припустимо, що
весь витрачений капітал складається або виключно з заробітної
плати, або виключно з вартості засобів виробництва. Ми мали б
тоді в першому випадку замість товарної вартості 400 с + 100 v +
100 m товарну вартість 500 v + 100 m. Витрачений на заробітну
плату капітал у 500 фунтів стерлінгів є вартість усієї праці,
вжитої на виробництво товарної вартості в 600 фунтів стерлінгів,
J. саме тому становить витрати виробництва всього продукту. Але
утворення цих витрат виробництва, в наслідок чого вартість витраченого
капіталу знову з’являється як складова частина вартості
продукту, є єдиний відомий нам процес в утворенні цієї товарної
вартості. Як виникає та її складова частина у 100 фунтів стерлінгів,
яка становить собою додаткову вартість, ми не знаємо.
Цілком те саме було б у другому випадку, де товарна вартість
була б = 500 с + 100 m. В обох випадках ми знаємо, що додаткова
вартість виникає з даної вартості тому, що ця вартість авансована
в формі продуктивного капіталу, однаково, чи то в формі праці,
чи в формі засобів виробництва. Але, з другого боку, авансована
капітальна вартість з тієї тільки причини, що вона витрачена
і тому становить витрати виробництва товару, не може
утворити додаткової вартості. Бо саме остільки, оскільки вона
становить витрати виробництва товару, вона утворює не додаткову
вартість, а тільки еквівалент, вартість, яка заміщає витрачений
капітал. Отже, оскільки вона утворює додаткову вартість,
вона утворює її не в наслідок своєї специфічної властивості як
витрачений капітал, а як авансований і тому застосований капітал
взагалі. Тому додаткова вартість в однаковій мірі виникає як
з тієї частини авансованого капіталу, що входить у витрати
виробництва товару, так і з тієї частини його, що не входить
у витрати виробництва; одним словом — в однаковій мірі з основних
і обігових складових частин застосованого капіталу. Весь
капітал речево служить як продуктотворець — засоби праці так
само, як і матеріали виробництва та праця. Весь капітал речево
входить у дійсний процес праці, хоч тільки частина його входить
у процес зростання вартості. Може, саме це і є причиною
того, що він тільки частиною бере участь в утворенні витрат виробництва,
але цілком — в утворенні додаткової вартості. Як би там
\parbreak{}  %% абзац продовжується на наступній сторінці

\input{_0056.tex}
\input{_0057.tex}

\index{iii2}{0058}  %% посилання на сторінку оригінального видання
Проте банки мають ще й інші засоби утворювати капітал. За тим самим
Newmarch’oм провінціяльні банки, як уже вище згадано, мають звичай відправляти
свої зайві фонди (тобто банкноти Англійського банку) лондонським billbrokers'aм,
що шлють їм натомість дисконтовані векселі. Цими векселями банк
обслуговує своїх клієнтів, бо він має за правило не видавати векселів, одержаних
від своїх місцевих клієнтів, щоб комерційні операції цих клієнтів не стали відомі
в їхній власній місцевості. Оці, одержані з Лондону, векселі служать не тільки
до того, щоб видавати їх клієнтам, які мають робити платежі безпосередньо
в Лондоні, якщо вони не вважатимуть за краще доручити банкові зробити
власний переказ на Лондон; ці векселі служать ще й до сплочування платежів
у провінції, бо передатний напис банкіра забезпечує їм місцевий кредит. Таким
чином, напр., в Ланкашайрі, витиснули ті векселі з циркуляції всі власні банкноти
місцевих банків та чималу частину банкнот Англійського банку (ibidem,
1568—74).

Отже, ми бачимо, як банки утворюють кредит та капітал: 1) виданням
власних банкнот; 2) виданням переказів на Лондон реченцем до 21 дня, переказів,
що їх однак в момент їхнього видання одразу оплачується банкам готівкою;
3) платежем дисконтованими векселями, що їхня кредитоздібність перед
усім та найголовніше, принаймні для відповідної місцевої округи, забезпечується
передатним написом банку.

Сила Англійського банку виявляється в реґулюванні ним ринкової норми
рівня проценту. Підчас нормального перебігу справ може трапитися, що Англійському
банкові не сила буде припинити помірний відплив золота з свого металевого
скарбу, підвищенням норми дисконту\footnote{
На загальних зборах акційний Union Bank of London 17 січня 1894 президент пан Ritchie
оповідав, що Англійський банк підвищив в 1893 році дисконт від 2 \sfrac{1}{2} (липень) до 3 та 4\% в серпні,
та, згубивши проте протягом чотирьох тижнів повних 4 \sfrac{1}{2} міл. ф. ст. золотом, до 5\%, після чого
золото почало припливати назад і банкову норму дисконту знизили в вересні до 4\%, а в жовтні до 3\%.
Але на ринку цю банкову норму не визнали. «Коли банкова норма була 5\%, то ринкова норма була З \sfrac{1}{2},
а норма для грошей була 2 \sfrac{1}{2}\%; коли банкова норма впала до 4\%, то норма дисконту була 2 \sfrac{3}{8}\%, а
грошова норма 1 \sfrac{3}{4}\%; коли банкова норма була 3\%, то норма дисконту була 1 \sfrac{1}{2}, а грошова норма
трохи нижча». (Daily News 18 січня 1894 р.) — Ф. Е).
}, бо потребу на платіжні засоби
задовольняють приватні й акційні банки та bill-brokers’и, що за останні тридцять
років набули значної сили на полі капіталу. Тоді має він уживати інших
засобів. Але для критичних моментів все ще має силу те, про що банкір Glyn
(з фірми Glyn, Mills, Currie and C°) свідчив перед С. D. 1848/57: «1709. Підчас
великої скрути в країні Англійський банк диктує рівень проценту — 1710. Підчас
надзвичайної скрути..., коли приватні банкірі або brokers’и порівняно обмежують
дисконтові операції, ці операції випадають Англійському банкові, й тоді він має
силу усталювати ринкову норму рівня проценту».

Звичайно, як офіційна установа, що має державну охорону та державні
привилеї, не сміє банк використовувати немилосердно цю свою силу, так як
можуть собі дозволити це приватні підприємства. Тому й Hubbard ось що каже
перед банковою комісією В. А. 1857; «2844 [питання]: А хіба не правда, що
коли норма дисконту є найвища, то Англійський банк обслуговує найдешевше,
а коли вона найнижча, тоді brokers’и обслуговують найдешевше? — [Hubbard]:
Так завжди буває, бо Англійський банк ніколи не знижує норми проценту так
низько, як його конкуренти, а коли норма є найвища, ніколи не підносить її
цілком так високо, як вони».

А проте серйозною подією в комерційному житті буває, коли банк підчас
скрути починає — уживаючи ходячого вислову — наганяти рівень проценту, тобто
ще вище підносити рівень проценту, що вже піднявся вище від пересічного. «Скоро
Англійський банк починає наганяти рівень проценту, припиняються всі закупи
для вивозу закордон... експортери чекають, поки спад цін дійде найнижчої точки
\parbreak{}  %% абзац продовжується на наступній сторінці


\index{iii1}{0059}  %% посилання на сторінку оригінального видання
Безглузде уявлення, ніби витрати виробництва товару становлять
його дійсну вартість, а додаткова вартість виникає
з продажу товару вище його вартості, що, отже, товари продаються
по їх вартостях, якщо їх продажна ціна дорівнює витратам
їх виробництва, тобто дорівнює ціні спожитих на них
засобів виробництва плюс заробітна плата, — це уявлення Прудон
з звичним шахрайством, яке чваниться вченістю, просурмив
як нововідкриту таємницю соціалізму. Це зведення вартості
товарів до витрат їх виробництва становить по суті
основу його народного банку. Раніше ми з’ясували, що різні складові
частини вартості продукту можна представити в пропорціональних
частинах самого продукту. Якщо, наприклад (книга І,
розд. VIІ, 2, стор. 229 \footnote*{
Стор. 153—154 рос. вид. 1935 р. \emph{Ред. укр. перекладу}.
}), вартість 20 фунтів пряжі становить
30 шилінгів — а саме 24 шилінги засоби виробництва, 3 шилінги
робоча сила і 3 шилінги додаткова вартість, — то цю додаткову
вартість можна представити як \sfrac{1}{10} продукту = 2 фунтам пряжі.
Тепер, якщо ці 20 фунтів пряжі продаються по витратах їх
виробництва, за 27 шилінгів, то покупець дістає даром 2 фунти
пряжі, або товар продано на \sfrac{1}{10}  нижче його вартості; але робітник
так само, як і раніш, дав свою додаткову працю — тільки
для покупця пряжі, а не для капіталістичного виробника пряжі.
Було б цілком помилково припускати, що коли б усі товари
продавались по витратах їх виробництва, то результат фактично
був би той самий, як коли б усі товари продавались вище витрат
їх виробництва, але по їх вартостях. Бо навіть коли припустити,
що вартість робочої сили, довжина робочого дня
і ступінь експлуатації праці повсюди однакові, то все ж маси
додаткової вартості, які містяться у вартостях різних видів
товару, аж ніяк не рівні, залежно від різного органічного складу
капіталів, авансованих на їх виробництво\footnote{
„Вироблювані різними капіталами маси вартості і додаткової вартості, при
даній вартості і однаковому ступені експлуатації робочої сили, прямо пропорціональні
до величин змінних складових частин цих капіталів, тобто їх складових
частин, перетворених у живу робочу силу“ (книга 1, розд. ІХ, стор. 321
[стор. 227 рос. вид. 1935 р.]).
}.

\section{Норма зиску}

Загальна формула капіталу є $Г — Т — Г'$; тобто певна сума
вартості кидається в циркуляцію для того, щоб витягти з неї
більшу суму вартості. Процес, який породжує цю більшу суму
вартості, є капіталістичне виробництво; процес, який реалізує
її, є циркуляція капіталу. Капіталіст виробляє товар не ради
самого товару, не ради його споживної вартості або свого особистого
споживання. Продукт, який в дійсності цікавить капіталіста,
\index{iii1}{0060}  %% посилання на сторінку оригінального видання
це не сам відчутний продукт, а надлишок вартості продукту
понад вартість спожитого на нього капіталу. Капіталіст
авансує весь капітал, не звертаючи уваги на ті різні ролі, що їх
відіграють складові частини капіталу у виробництві додаткової
вартості. Він однаково авансує всі ці складові частини капіталу
не тільки для того, щоб репродукувати авансований капітал, але
і для того, щоб виробити певний надлишок вартості понад
цей капітал. Він може перетворити вартість змінного капіталу,
який він авансує, у вищу вартість тільки через обмін його на
живу працю, через експлуатацію живої праці. Але він може експлуатувати
працю тільки в тому разі, коли він одночасно авансує
умови для здійснення цієї праці — засоби праці і предмет
праці, машини і сировинний матеріал, тобто коли він ту суму
вартості, якою він володіє, перетворює в форму умов виробництва;
як і взагалі, він тільки тому є капіталіст, тільки тому взагалі
може взятися до процесу експлуатації праці, що він як власник
умов праці протистоїть робітникові як володільцеві тільки робочої
сили. Вже раніше, в першій книзі, було показано, що саме
те, що цими засобами виробництва володіють не-робітники, перетворює
робітників у найманих робітників, а не-робітників — у капіталістів.

Капіталістові байдуже, чи розглядати справу так, що він
авансує сталий капітал для того, щоб здобути бариш із змінного,
чи так, що він авансує змінний капітал для того, щоб збільшити
вартість сталого; чи так, що він витрачає гроші на заробітну
плату для того, щоб надати машинам і сировинному матеріалові
вищу вартість, чи так, що він авансує гроші на машини та сировинний
матеріал для того, щоб мати можливість експлуатувати працю.
Хоч додаткову вартість утворює лише змінна частина капіталу,
проте вона утворює її тільки при тій умові, що авансуються
й інші частини, виробничі умови праці. Через те що капіталіст
може експлуатувати працю тільки за допомогою авансування сталого
капіталу, що він може збільшити вартість сталого капіталу
тільки за допомогою авансування змінного, то в його уявленні
ці капітали збігаються, і це тим більше, що дійсний рівень його
баришу визначається відношенням не до змінного капіталу, а
до всього капіталу, не нормою додаткової вартості, а нормою
зиску, яка, як ми побачимо, може лишатись однаковою і все ж
виражати різні норми додаткової вартості.

До витрат виготовлення (Kosten) продукту належать усі
складові частини його вартості, які капіталіст оплатив або еквівалент
яких він кинув у виробництво. Ці витрати мусять бути
заміщені для того, щоб капітал просто зберігся або репродукувався
в своїй первісній величині.

Вартість, яка міститься в товарі, дорівнює тому робочому
часові, якого коштує його виготовлення, а сума цієї праці складається
з оплаченої і неоплаченої праці. Навпаки, для капіталіста
витрати виготовлення товару складаються тільки з тієї
\parbreak{}  %% абзац продовжується на наступній сторінці

\index{franko}{0061}

\setcounter{chapter}{23}
\section[Початок і історичний розвиток
капіталістичної продукції в Англії]{Початок і історичний розвиток капіталістичної продукції в Англії\footnotemarkZ{}}
\markboth{Початок і історичний розвиток
капіталістичної продукції в Англії}{Фрагмент «Капіталу» у~перекладі Івана~Франка}

\footnotetextZ{
Вперше надруковано в журн. «Культура», 1926, № 4--9, с. 61--87.

\nopagebreak[4]
Подається за автографом перекладача: відділ рукописних фондів і текстології Інституту літератури ім. Т.~Г.~Шевченка НАН України. — Ф. 3. — Од. зб. 448. — 14 арк. Кінець автографа не зберігся. 

\nopagebreak[4]
Переклад зроблено з другого німецького видання: \textgerman{Das Kapital. Kritik der politischen Oekonomie. Von Karl Marx. Erster Band. Zweite verbesserte Aufgabe. Hamburg. Verlag von Otto Meissner, 1872.} Про це є згадка І. Франка на початку тексту перекладу «Гл[яди] К. Marx. Das Kapital, 2 вид. з р. 1872, стор. 742--794».}

\subsection{Первісне нагромадженє капіталу}

Ми бачили, що гроші стают капіталом тоді, коли служат
до купованя робучої сили. Ми бачили, що капітал
раз-ураз намагає — творити надзвишку вартости, а надзвишка —
вбільшує капітал. Між тим, щоб капітал міг нагромаджуватись,
мусит уже вперед витворюватись надзвишка;
щоб могла витворюватись надзвишка, мусит істнувати капіталістична
продукція, а щоб тота істнувала, мусит уже
вперед більша маса капіталу бути нагромаджена в руках
поєдинчих богатирів. Здаєсь затим, що весь той процес
полягає на якімось „первіснім“ нагромадженю, котре мало
місце перед капіталістичною продукцією, котре, значит, не
було випливом капіталістичної продукції, а єї жерелом.

\index{franko}{0062}
Тото первісне нагромадженє капіталу („previous accumulation“,
як каже А.~Сміт) грає в суспільній економії
майже таку саму ролю, як „гріхопаденіє“ в теольоґії. Адам
зїв яблоко і через те стягнув гріх на рід людський. Початок
гріха обяснений казкою про давнину. Колись-колись
в давнину були з одного боку пильні вибранці, а з другого —
ліниві нероби. Через те сталося, що перші нагромадили
богацтво, а другі зійшли на таке, що остаточно не мали
вже що продавати крім себе самих. І від того гріхопаденія
почалася бідність великої маси, котра ще й доси, хоть і як
тяжко працює, не має що продавати крім себе самих, —
і богацтво деяких, що й доси змагаєся, хоть самі вони
давно перестали працювати\footnote{
Такі безглузді дитиньства плете ще д. Тйер (звісний французький
муж стану) дотепним колись Французам с повагою великого мудрця —
для оборони святої власности. Ну і справді, — скоро діло йде о власність,
то святий обовязок кождого — міцно стояти на становищи букваря,
ще й других переконувати, що те становище для всякого „віка
і возраста“ єдино відповідне і належне.
}. В правдивій історії грали, як
звісно, завойованя, гнет, рабунки, вбійства, — одним словом,
усілякі насиля велику ролю. Але в сумирній політичній
економії з давен-давна — все іділлія. Право і „праця“, се
здавна були єдині способи до збогаченя, тілько, розумієся,
завсігди с тим застереженєм, що аж „сего року воно щось
не так“. Але на ділі способи первісного нагромадженя капіталу
були всякі, які хочете, — тілько не іділлічні.

Гроші і товар не є зразу капіталом, так само, як не
є ним зразу средства продукційні і знадоби до житя. Вони
мусят бути перемінені в капітал. Але та переміна може настати
тілько серед певних обставин, котрі зводятся ось на
що: двоякі дуже відмінні посідачі товарів мусят стати супротів
себе і зіткнутися с собою, — з одного боку властивці
грошей, средств продукційних і знадіб до житя, котрим
о то йде, щоб свою суму вартостей побільшити купівлею
чужої робучої сили; а з другого боку вільні робітники, продавці
власної робучої сили і, значит, продавці \so{праці}.
Вільні вони мусят бути в двоякім значіню, т. є. щоб ані
самі вони беспосередно не були средствами продукційними,
як невольники, кріпаки і т. д., ані шоб вони самі не посідали
средств продукційних, як ґазди-селяне, дрібні властивці
ґрунтові і т. д. Такий розділ товарів між дві крайности
— се основні вимінки для капіталістичної продукції.
Без відділеня робітників від власности не може настати
капіталістична продукція. Але скоро вона раз настала, то
не тілько підтримує те відділенє, але й сама доводит до
него раз-ураз наново і раз-ураз на більший розмір. Коли
затим спитаємо: де є жерело капіталістичного ладу? то
\parbreak{}

\parcont{}
\index{franko}{0063}
відповідь на те дуже проста: жерело капіталістичного ладу,
се не що їнше, як той процес \so{відділюваня робітника
від власности, від средств продукційних}.
Сей процес з одного боку перемінює суспільну істенину
(средства продукційні і знадоби до житя) в капітал, а з другого
боку перемінює беспосередних витвірців в наємних
робітників. Так назване „первісне нагромадженє капіталу“,
се затим не що їнше, як історичний процес відділюваня
продуцента від средств продукційних. Він і справді „первісний“,
бо становит вступ до історії капіталу і капіталістичної
продукції.

На перший погляд видно, що той процес роскладовий
обнимає собою цілий ряд історичних процесів і то ряд двоякий:
з одного боку нищенє тих відносин, котрі робітника
робили власністю третих осіб, їх средством продукційним,
— з другого боку вивласнюванє беспосередних витвірців,
витисканє їх с посіданя средств продукційних. Процес роскладовий,
се затим на ділі ціла історія розвитку новійшої
буржоазної суспільности. Се булаб зовсім не трудна історія,
колиб буржоазні історики не були єї нам вказали виключно
в рожевім світлі еманціпації робітників, а булиб звернули
увагу й на то, якими способами в тій історії визискуванє
феодальне перемінилося в визискуванє капіталістичне. Початок
розвитку становила неволя робітника. В дальшім тягу
того розвитку неволя осталась, тілько в зміненій формі.
Але ми ту не будем вдаватися в розбір середновікових рухів.
Хоть капіталістична продукція вже в \RNum{14} і \RNum{15} віці розпочалася
в деяких місцях над Середземним морем, то прецінь
ера капіталістична починаєся аж від \RNum{16} віку. Там,
де вона росцвитає, давно вже знесено панщину і середновікове
міщанство також як раз хилится до впадку.

Епохи в історії того роскладового процесу становят ті
хвилі, коли великі маси людей нараз і силою відривано від
усіх средств до житя та праці і як свобідних і голих пролєтаріїв
перто на робучий торг. Вивласнюванє робітників
з ґрунту і посідлости становит основу цілого процесу. Тож
і ми насамперед мусимо переглянути історію того вивласненя.
В різних краях вона проявляєся в різних окрасках
і переходит різні фази в неоднакім порядку. Тілько в Англії,
котру ми проте беремо за примір, вона має клясичну форму\footnote{
В Італії, де капіталістична продукція розвилась найраньше, найраньше
також увільнено кріпаків. Тількож при тім увільненю вони не
одержали права на ґрунти, хотьби й за сплатою індемнізації, так що
„воля“ перемінила італіянських кріпаків відразу в голих пролєтаріїв, котрі
крім того по містах, стоячих ще переважно від римських часів, найшли
вже готових нових панів.
}.


\index{iii1}{0064}  %% посилання на сторінку оригінального видання
Величина вартості всього капіталу сама по собі не стоїть
у будьякому внутрішньому відношенні до величини додаткової
вартості, принаймні не стоїть безпосередньо. Щодо своїх речових
елементів весь капітал мінус змінний капітал, отже, сталий капітал,
складається з речових умов здійснення праці, з засобів праці
і матеріалу праці. Для того, щоб певна кількість праці реалізувалась
у товарах і, отже, утворила вартість, потрібна певна
кількість матеріалу праці і засобів праці. Залежно від особливого
характеру додаваної праці існує певне технічне відношення
між масою праці і масою засобів виробництва, до яких повинна
бути додана ця жива праця. Отже, остільки існує також певне
відношення між масою додаткової вартості або додаткової праці
і масою засобів виробництва. Якщо, наприклад, час, необхідний
для виробництва заробітної плати, становить 6 годин на день,
то робітник мусить працювати 12 годин, щоб дати 6 годин додаткової
праці, щоб створити додаткову вартість у 100\%. Він
споживає за ці 12 годин удвоє більше засобів виробництва, ніж
за ці 6 годин. Але від цього додаткова вартість, яку він додає
за 6 годин, зовсім не стає в будьяке безпосереднє відношення
до вартості засобів виробництва, спожитих за ці 6 чи навіть
за ці 12 годин. Ця вартість тут не має ніякого значення; ідеться
тільки про технічно необхідну масу. Чи сировинний матеріал або
засоби праці дешеві, чи дорогі, це не має ніякого значення;
аби тільки вони мали потрібну споживну вартість і були наявні
в технічно встановленій пропорції до тієї живої праці, яку треба
поглинути. Однак, якщо я знаю, що за одну годину перепрядається
$х$ фунтів бавовни, які коштують $а$ шилінгів, то я, звичайно,
знаю і те, що за 12 годин перепрядається 12 $х$ фунтів
бавовни = 12 $а$ шилінгам, і тоді я можу обчислити відношення
додаткової вартості до вартості цих 12, так само як і до вартості
цих 6. Але відношення живої праці до \emph{вартості} засобів
виробництва тут привходить лиш остільки, оскільки $а$ шилінгів
служать назвою для $х$ фунтів бавовни; бо певна кількість бавовни
має певну ціну, а тому й навпаки, певна ціна може служити
показником певної кількості бавовни, поки ціна бавовни
не зміниться. Якщо я знаю, що для того, щоб привласнити 6 годин
додаткової праці, я повинен примушувати працювати 12 годин,
отже, мушу мати напоготові бавовни на 12 годин, і якщо я знаю
ціну цієї потрібної для 12 годин кількості бавовни, то посередньо
існує відношення між ціною бавовни (як показником необхідної
кількості) і додатковою вартістю. Навпаки, з ціни сировинного
матеріалу я ніколи не можу зробити висновок про масу сировинного
матеріалу, яка може бути перепрядена, наприклад, за
одну годину, а не за 6. Отже, немає ніякого внутрішнього, необхідного
відношення між вартістю сталого капіталу, — а тому
і між вартістю всього капіталу ($= с + v$) і додатковою вартістю.

Якщо норма додаткової вартості відома і величина додаткової
вартості дана, то норма зиску виражає не що інше, як
\parbreak{}  %% абзац продовжується на наступній сторінці

\index{franko}{0065}

Перший крок перевороту, що поклав основу капталістичній продукції, припадає в послідній третині \RNum{15} і
в першій чверти \RNum{16} віку. Тоді скасовано феодальне дворацтво, котре, як справедливо замічає Джемс
Стюерт, „залякало  всі хати і двори безхосенно“. Через те викинено масу голих пролєтаріїв на
робучий торг. Хоть королівська власть, що й сама виросла з буржуазного розвитку, намагаючи до
неограниченого панованя, силою скасувала те великопанське дворацтво, то прецінь вона не була єдиною
причиною нового перевороту. Ні, в упертім опорі протів королівства та
парляменту витворили великі пани-феодали далеко більшу масу пролєтаріяту, прогонюючи силою
хліборобів з ґрунту і посідлости, хоть хлібороби мали до тих ґрунтів більше право, ніж вони, і
забираючи для себе громадські ґрунти. Беспосередний товчок до того в Англії дав іменно росцвіт
фляндрійської вовняної мануфактури і звязане з ним підскоченє цін вовни. Стара феодальна шляхта
вигибла в великих феодальних війнах, а нова шляхта — се були діти свого часу, для котрих гроші були
силою понад всі сили. З вірного поля пасовиська для овець! — се став тепер їх загальний оклик.
Гаррізен в своїй „Description of England. Prefixed to Holinshed’s Chronicles“ описує, як
вивласнюванє дрібних ґаздів руйнує край. „Але що нашим великим самозванцям до того?“ Мешканя ґаздів
та коттеджі робітників валят вони силою або прогнавши людей лишают пустками. „Коли перездримо
давнійші інвентарі кождої домінії, то побачимо, що незлічимі хати та дрібні ґаздівства пощезали, що
ґрунт годує далеко меньше люда, що богато міст підупало, хоть деякі нові підносятся\dots{} Мож би
чимало наросповідатися про місточка та села, зруйновані для того, щоб було місце на толоки для
овець; тілько самотні панські двори стоят серед тих толок“. Правда, наріканя тих старих літописів
усе пересаджені, але вони досадно малюют те вражінє, яке на самих сучасників робив переворот
обставин продукційних. Порівнанє між письмами канцлєрів Фортеске і Томаса Моруса вказує наглядно
пропасть між \RNum{15} а \RNum{16} віком. „Із золотого віку — каже справедливо Зорнтон — попали англійські
робітники без ніяких перехідних ступнів прямо в зелізну“.

Праводавство злякалось сего перевороту. Воно не стояло ще на такім високім ступни цівілізації, де
„богацтво народне“, т. є. богацтво капіталістів і безграничне висисанє та зубожінє маси люду
становит верх премудрости
політичної. В своїй історії Генріха VII каже Бекон: „В тім часі (1489) посипалися скарги на то, що
вірне поле перемінюєсь в пасовиська, котрих лехко може дозирати кілька пастухів. Ґрунти, що вперед
виарендовувались на кілька літ, на доживотну або щорічну умову, тепер зіллято разом
\index{franko}{0066}
с панськими. Се підкопало добробуток люду, а через те й міста, церкви, десятини\dots{} Щоб зарадити
тому лиху, проявили король і парлямент дивну на ті часи мудрість\dots{} Вони видали право протів того
обезлюднюючого край загарбуваня громадських ґрунтів (depopulating inclosures) і невідлучної
від него обезлюднюючої ґосподарки толочної (depopulating pasture[s])“. Оден акт Генріха VII з р.~
1489 заказує руйнувати хліборобські хати, до котрих належит що найменьше 20 екрів ґрунту. Генріх
VIII відновив той самий указ. Говорится там між їншим, що „многі аренди і огромні отари, особливо
овець, нагромаджуются в немногих руках, через що дохід
з ґрунту дуже вбільшився, а рільництво дуже підупало, церкви і хати повалено, дивовижні маси народа
стали неспосібні вдержувати себе і свої родини“. Указ наказує затим відбудовувати повалені хутори,
означує, кілько має бути вірного поля в стосунку до овечих толок і т. д. Їнший акт з р.~1533
жалуєсь, що деякі властивці мают по \num{24000} овець, і ограничує їх число на 2000\footnote{
В своїй „Утопії“ говорит Томас Морус про дивовижний край, де
„вівці їдят людей“.
}. Наріканя народа і
праводавство протів вивласнюваня дрібних арендаторів та хліборобів, що почалось від Генріха VII і
трівало зо 150 літ
— не помогли нічо. Чому не помогли, пояснює нам Бекон, сам того не знаючи. „Акт Генріха~VII, — каже
він в своїх „Essays, civil and moral“, Sect. 20, — був глибоко і дивно обдуманий. Він утворив
сільскі ґаздівства і хліборобські доми певного нормального розміру, т. є. вдержав для них таку
пропорцію ґрунту, котра давала їм змогу плодити на світ підданих доста заможних і не придавлених
нуждою, так що плуг був в руках властивців, а не наємників\footnote{
Бекон пояснює далі звязок між свобідним, заможним селянством
а доброю інфантерією. „Се була дивно важна річ для сили і мужности
королівства — мати аренди достаточного розміру, щоб дільних мужів
забеспечити від нужди і велику часть ґрунту краєвого запевнити в посіданє джоменам, т. є. людім
середної заможности між шляхтою а халупниками (cottagers) та наймитами. Бо се загальна думка
найліпших знавців воєнного діла\dots{} що головна сила армії, се інфантерія або піхота. Але щоб
витворити добру інфантерію, тре людей вихованих не в притиску ані в нужді, але свобідно і в певній
заможности. Коли затим яка держава вросте переважно в шляхту та делікатне панство, а хлібороби та
ратаї зійдут на простих зарібників та наймитів або халупників, т. є. жебраків з власною хатою, то
така держава може мати добру кінницю, але доброї піхоти не буде мати. Се видно в Італії і Франції і
деяких других заграничних краях, де справді все або шляхта або нужденні зарібники\dots{} Дійшло там до
того, що ті краї мусят уживати наємного зброду Швейцарів та др. для своєї піхоти: відти то й пішло,
що ті держави мают богато людий, а мало вояків“. („The Reign of Henry VII“ і т.~д.).
}. А між
\parbreak{}


\index{iii1}{0067}  %% посилання на сторінку оригінального видання
Ми зберігаємо позначення, вжиті в першій і другій книгах.
Весь капітал $К$ поділяється на сталий капітал $с$ і змінний капітал
$v$ і виробляє додаткову вартість $m$. Відношення цієї додаткової
вартості до авансованого змінного капіталу, отже $\frac{m}{v}$, ми
називаємо нормою додаткової вартості і позначаємо її через $m'$.
Отже, $\frac{m}{v} = m'$, і тому $m = m'v$. Якщо цю додаткову вартість
віднести не до змінного капіталу, а до всього капіталу, то вона
зветься зиском $(р)$, а відношення додаткової вартості $m$ до
всього капіталу $К$, отже $\frac{m}{K}$, зветься нормою зиску $р'$. Звідси ми
маємо:\[
р' = \frac{m}{K} = \frac{m}{с + v},
\]
якщо ми замість m підставимо його знайдену вище величину
$m'v$, то матимем:\[
р' = m'\frac{v}{К} = m'\frac{v}{c + v},
\]
рівняння, яке можна виразити також у пропорції:\[
р':m' = v:К,
\]
норма зиску відноситься до норми додаткової вартості, як змінний
капітал до всього капіталу.

З цієї пропорції випливає, що $р'$, норма зиску, завжди менша
від $m'$, норми додаткової вартості, бо $v$, змінний капітал, завжди
менший від $К$, суми $v + c$, змінного і сталого капіталу; за винятком
єдиного, практично неможливого випадку, коли $v = K$, отже,
коли капіталістом зовсім не авансувався б сталий капітал, засоби
виробництва, а тільки заробітна плата.

В нашому дослідженні треба, однак, звернути увагу ще на
ряд інших факторів, які визначально впливають на величину $с$,
$v$ і $m$ і тому мають бути коротко згадані.

Поперше, \emph{вартість грошей}. Її ми можемо повсюди приймати
за сталу.

Подруге, \emph{оборот}. Цей фактор ми покищо лишаємо осторонь,
бо його вплив на норму зиску розглядається окремо в одному
з дальших розділів. [Тут ми, забігаючи наперед, згадаємо тільки
про той один пункт, що формула $р' = m'\frac{v}{К}$ є строго правильна
лиш для \emph{одного} періоду обороту змінного капіталу, і що ми,
однак, можемо її зробити правильною для річного обороту, поставивши
замість $m'$, простої норми додаткової вартості, $m'n$,
річну норму додаткової вартості; при чому n є число оборотів
змінного капіталу протягом одного року (див. книгу II, розд.
XVI, 1) — Ф. Е]


\index{iii1}{0068}  %% посилання на сторінку оригінального видання
Потретє, треба взяти до уваги \emph{продуктивність праці}, вплив
якої на норму додаткової вартості докладно розглянуто в книзі І,
відділ IV. Але вона може також справляти і безпосередній
вплив на норму зиску, принаймні окремого капіталу, коли, як
де розвинено в книзі І, розділ X, цей окремий
капітал працює з продуктивністю більшою, ніж суспільнопересічна,
дає свої продукти по вартості нижчій, ніж суспільно-пересічна
вартість таких самих товарів, і таким чином
реалізує надзиск. Але цей випадок лишається тут поза нашим
розглядом, бо і в цьому відділі ми все ще виходимо з припущення,
що товари виробляються при суспільно-нормальних
умовах і продаються по їх вартостях. Отже, в кожному окремому
випадку ми виходимо з припущення, що продуктивність
праці лишається сталою. Справді, вартісний склад капіталу,
вкладеного в певну галузь промисловості, тобто певне відношення
змінного капіталу до сталого капіталу, кожного разу
виражає певний ступінь продуктивності праці. Отже, як тільки
це відношення зазнає зміни з іншої причини, а не в наслідок
простої зміни вартості матеріальних складових частин сталого
капіталу або зміни заробітної плати, то й продуктивність праці
мусить зазнати зміни, і тому ми досить часто бачитимем, що
зміни, які відбуваються з факторами $с$, $v$ і $m$, включають також
і зміни в продуктивності праці.

Те саме стосується і до трьох інших факторів: \emph{довжини робочого
дня, інтенсивності праці і заробітної плати}. Їх вплив
на масу і норму додаткової вартості докладно досліджено в першій
книзі. Отже, зрозуміло, що коли ми задля спрощення завжди
виходимо з припущення, що ці три фактори лишаються сталими,
то все ж ті зміни, які відбуваються з $v$ і $m$, можуть також включати
в собі зміну в величині цих трьох визначаючих їх моментів.
Тут слід тільки коротко нагадати про те, що заробітна плата
діє на величину додаткової вартості і висоту норми додаткової
вартості зворотно до того, як діє довжина робочого дня і інтенсивність
праці; що підвищення заробітної плати зменшує додаткову
вартість, тимчасом як здовження робочого дня і підвищення
інтенсивності праці збільшують її.

Якщо припустимо, наприклад, що капітал у 100 з 20 робітниками
виробляє при десятигодинній праці та загальній тижневій
заробітній платі в 20 додаткову вартість у 20, то ми
матимем:\[
80 с + 20 v + 20 m; m' = 100\%, р' = 20\%.
\]
Припустімо, що робочий день здовжується, без підвищення
заробітної плати, до 15 годин; в наслідок цього вся нововироблена
20 робітниками вартість підвищується з 40 до 60 (10 : 15 = 40 : 60);
\parbreak{}  %% абзац продовжується на наступній сторінці

\input{_0069c.tex}
\input{_0070c.tex}

\index{iii1}{0071}  %% посилання на сторінку оригінального видання
Якщо ми тепер визначимо відношення $К$ і $К_1$, а також
$v$ і $v_1$, припустимо, наприклад, що значення дробу $\frac{К_1}{К} = Е$, а дробу
$\frac{v_1}{v} = е$, то $К_1 = Е К$, а $v_1 = е v$. Підставивши тепер у попередньому
рівнянні для $р'_1$, $К_1$ і $v_1$ здобуті таким чином значення,
ми матимем:\[
р'_1 = m'\frac{еv}{ЕК}.
\]

Але з обох попередніх рівнянь ми можемо вивести ще й другу
формулу, перетворивши їх у пропорцію:\[
р': р'_1 = m'\frac{v}{К} : m' \frac{v_1}{K_1} = \frac{v}{К} : \frac{v_1}{К_1}.
\]
Через те, що величина дробу не змінюється, коли чисельник
і знаменник помножити або поділити на те саме число, ми можемо
$\frac{v}{К}$ і $\frac{v_1}{К_1}$ звести до процентних чисел, тобто припустити, що
$К$ і $К_1 = 100$. Тоді ми матимем $\frac{v}{К} = \frac{v}{100}$ і $\frac{v_1}{К_1} = \frac{v_1}{100}$, і можемо відкинути
у наведеній пропорції знаменники; ми одержуємо:\[
р' : р'_1 = v : v_1\text{; або:}
\]
При двох довільно взятих капіталах, які функціонують з однаковою
нормою додаткової вартості, норми зиску відносяться
одна до одної як змінні частини капіталу, обчислені у процентах
до своїх відповідних цілих капіталів.

Ці дві формули охоплюють усі випадки змін $\frac{v}{К}$.

Раніш ніж дослідити ці випадки кожний окремо, зробимо
ще одно зауваження. Через те, що $К$ є сума $c$ і $v$, сталого і змінного
капіталу, і через те що норма додаткової вартості, як
і норма зиску, звичайно виражається у процентах, то взагалі
зручно суму $c + v$ теж припустити рівною сотні, тобто $c$ і $v$
виражати в процентах. Для визначення, — правда, не маси, а
норми зиску, — однаково, чи ми скажемо: капітал у 15000,
з них 12000 сталого і 3000 змінного капіталу, виробляє додаткову
вартість у 3000, чи зведемо цей капітал до процентів:
\begin{align*}
15000 K &= 12000c + 3000 v (+ 3000 m) \\
100 K &= 80 c + 20 v (+ 20 m).
\end{align*}

В обох випадках норма додаткової вартості $m' = 100\%$, норма
зиску = 20\%.
Те саме, коли ми порівнюємо один з одним два капітали,
наприклад, з попереднім якийсь інший капітал:
\begin{align*}
12000 K &= 10800 c + 1200 v (+ 1200 m) \\
100 K &= 90 c + 10 v (+ 10 m),
\end{align*}


\input{_0072c.tex}

\index{ii}{0073}  %% посилання на сторінку оригінального видання
Процес циркуляції промислового капіталу, що становить лише частину
процесу його індивідуального кругобігу, визначається раніш розвиненими
загальними законами (книга I, розділ III), оскільки він являє лише ряд
актів у межах загальної товарової циркуляції. Та сама маса грошей, напр.,
500\pound{ ф. стерл.}, втягує по черзі в циркуляцію то більше промислових капіталів
(або також індивідуальних капіталів в їхній формі товарових капіталів), що
більша обігова швидкість грошей, що швидше, отже, кожний поодинокий
капітал перебігає ряд своїх товарових або грошових метаморфоз. Тому
та сама кількість капітальної вартости потребує для своєї циркуляції то
менше грошей, що більше гроші функціонують як засіб виплати, отже,
що більше, прим., при заміщенні товарового капіталу засобами його продукції,
доводиться оплачувати лише різність і що коротші строки виплати,
прим., при виплаті заробітної плати. З другого боку, коли швидкість
циркуляції та всі інші обставини дано як незмінні, то кількість грошей,
що мусить циркулювати як грошовий капітал, визначається сумою товарових
цін (ціна, помножена на кількість товарів), або, коли дано кількість
і вартість товарів, — вартісно самих грошей.

Але закони загальної товарової циркуляції мають силу лише остільки,
оскільки процес циркуляції капіталу утворює ряд актів простої циркуляції,
і не мають сили, коли ці акти становлять функціонально визначені
етапи в кругобігу індивідуальних промислових капіталів.

Щоб пояснити це, найкраще розглядати процес циркуляції в його
безперервному зв’язку, яким він з’являється в обох формах:
\[
\text{\phantom{I}II) } П\dots{} Т' 
\left\{\begin{array}{cc}
Т' & — \\
—  & Г'\\
т & —
\end{array}
\right.
\left\{\begin{array}{l}
Г — Т\splitfracm{Р}{Зп}\dots{} \samewidth{$П \dots{} Г'$}{$П (П')\hfill{}$} \\
~ \\
г — т
\end{array}
\right.
\]
\[
\text{III) } \samewidth{$П\dots{} Т'$}{\hfill{}$Т'$}
\left\{\begin{array}{cc}
Т' & — \\
—  & Г'\\
т & —
\end{array}
\right.
\left\{\begin{array}{l}
Г — Т\splitfracm{Р}{Зп}\dots{} П \dots{} Г' \\
~ \\
г — т
\end{array}
\right.
\]

Як ряд актів циркуляції взагалі, процес циркуляції (хоч є він $Т — Г — Т$,
хоч $Г — Т — Г$) являє лише два протилежні ряди товарових метаморфоз,
що з них кожна окрема метаморфоза знову таки має собі й протилежну
метаморфозу на боці чужого товару або чужих грошей, що протистоять
даному товарові.

Те, що з боку товаровласника $Т — Г$, з боку покупця є $Г — Т$; перша
метаморфоза товару в $Т — Г$ є друга метаморфоза товару, що виступає
як $Г$; в формулі $Г — Т$ справа стоїть навпаки. Отже, все, що сказано
про те, як метаморфоза товару на одній стадії переплітається з метаморфозою
товару на другій стадії, має силу для циркуляції капіталу,
оскільки капіталіст виконує функції покупця і продавця товарів, і
оскільки в наслідок цього його капітал функціонує як гроші проти
чужих товарів або як товар проти чужих грошей. Але це переплітання
метаморфоз не є разом з тим вираз для переплітання метаморфоз капіталів.

\parcont{}  %% абзац починається на попередній сторінці
\index{i}{0074}  %% посилання на сторінку оригінального видання
зовсім не потрібно, щоб одночасно підносились або падали ціни
всіх товарів. Досить підвищення цін певного числа головних
товарів в одному випадку або зниження їхніх цін у другому,
щоб підвищити або понизити належну до реалізації суму цін
усіх товарів, що циркулюють, отже, і щоб пустити в циркуляцію
більше або менше грошей. Чи зміна цін товарів одбиває дійсну
зміну вартостей чи просто коливання ринкових цін, вплив на
масу засобів циркуляції лишається той самий.

Припустімо, що дано якесь число продажів, або частинних
метаморфоз, не зв’язаних між собою, що відбуваються одночасно,
отже, просторово одна побіч однієї, приміром, 1 квартера пшениці,
20 метрів полотна, 1 біблії, 4 ґальонів горілки-житнівки. Коли
ціна кожного товару є 2\pound{ фунти стерлінґів}, отже, належна до реалізації
сума цін є 8\pound{ фунтів стерлінґів}, то в циркуляцію мусить
увійти маса грошей в 8\pound{ фунтів стерлінґів}. Навпаки, коли ці самі
товари є члени відомого нам уже ряду метаморфоз: 1 квартер
пшениці — 2\pound{ фунти стерлінґів} — 20 метрів полотна — 2\pound{ фунти
стерлінґів} — 1 біблія — 2\pound{ фунти стерлінґів} — 4 ґальони горілки —
2\pound{ фунти стерлінґів}, то 2\pound{ фунти стерлінґів} спричинюють послідовно
циркуляцію різних товарів, реалізуючи послідовно їхні
ціни, отже, і суму цін в 8\pound{ фунтів стерлінґів}, щоб спочити, кінець-кінцем,
у руках гуральника. Вони роблять чотири обіги. Ця
кількаразова зміна місць тієї самої монети репрезентує подвійну
зміну форми товару, його рух через дві протилежні стадії циркуляції
і сплетіння метаморфоз різних товарів\footnote{
«Лише продукти пускають їх (гроші) в рух і примушують їх
циркулювати\dots{} Швидкість їхнього (грошей) руху заступає їхню кількість.
Коли виникає потреба в них, вони тільки переходять із рук до рук, не
зупиняючись ані на хвилину». («Ce sont les productions qui (l’argent
mettent en mouvement et le font circuler\dots{} La célérité de son mouvement
(sc. de l’argent) supplée à sa quantité. Lorsqu’il en est besoin, il ne fait
que glisser d’une main dans l’autre sans s’arrêter un instant»). (\emph{Le~Trosne}:
«De l’Intérêt Social», Physiocrates, éd. Daire. Paris 1846, p. 915 sq.).
}. Протилежні й
що одна одну доповнюють фази, які перебігає цей процес, не можуть
відбуватися одна поруч однієї просторово, а наступають
одна по одній лише в часі. Тому переміжки часу становлять міру
тривання цього процесу, або швидкість обігу грошей вимірюється
числом обігів тієї самої монети за даний час. Нехай процес
циркуляції зазначених чотирьох товарів триває, приміром, один
день. Тоді сума цін, що має бути зреалізована, становитиме
8\pound{ фунтів стерлінґів}, число обігів тієї самої монети за день — 4
і кількість грошей, що циркулюють — 2\pound{ фунти стерлінґів}, або
для даного переміжку часу процесу циркуляції:
$\frac{\text{Сума цін товарів}}{\text{Число обігів однойменних монет}} \deq{}$
масі грошей, що функціонують
як засіб циркуляції. Цей закон має загальне значення.
Процес циркуляції якоїсь країни за якийсь даний відтинок часу
охоплює, правда, з одного боку, багато ізольованих, що відбуваються
одночасно й просторово один поруч одного, продажів
\parbreak{}  %% абзац продовжується на наступній сторінці


\index{iii2}{0075}  %% посилання на сторінку оригінального видання
\emph{Повосьме.} Відпливи металу, здебільша, є симптом зміни в стані закордонної
торговлі, а ця зміна, своєю чергою, є ознака того, що знову достигають
умови для кризи\footnote{
За Newmarch’oм відплив золота закордон може поставати з причин троякого роду, а саме:
1) від суто-комерційних причин, тобто тоді, коли довіз був більший, ніж вивіз, як то було між 1836
та 1844 роками та знову в 1847 роді, коли був головне значний довіз збіжжя; 2) від того, що треба
добувати засоби для приміщення англійського капіталу закордоном, як то було в 1857 роді, коли
будували
залізниці в Індії; та 3) від того, що остаточно витрачається кошти закордоном, як от в 1853
та 1854 роках на військові справи на Сході.
}.

\emph{Подев’яте.} Платіжний баланс може бути сприятливий для Азії й несприятливий
для Европи й Америки\footnote{
1918. Newmarch «Якщо ви візьмете Індію та Китай разом, якщо ви візьмете на увагу
обороти між Індією та Австралією та ще важливіші обороти між Китаєм та Сполученими Штатами — а в
цих випадках торговля є трибічна й вирівнюються рахунки за нашим посередництвом\dots{} тоді слушно,
що торговельний балянс був несприятливий не тільки для Англії, але й для Франції та Сполучених
Штатів». (В. А 1857).
}.

\pfbreak

Довіз благородного металу відбувається переважно при двох моментах.
З одного боку, за тієї першої фази низького рівня проценту, що настає по
кризі та є вияв обмеження продукції; а потім за другої фази, коли рівень проценту
підноситься, але ще не досягнув своєї середньої висоти. Це — фаза, коли
зворотні припливи капіталів відбуваються легко, комерційний кредит великий,
а тому й попит на позичковий капітал зростає непропорційно поширові продукції.
В обох фазах, коли позичкового капіталу є порівняно багато, надмірний
приплив капіталу, що існує в формі золота та срібла, отже, в такій формі,
в якій він насамперед може функціонувати лише як позичковий капітал, — мусить
значно виливати на рівень проценту, а тому й на загальний розвиток справ.

З другого боку: відплив, невпинний значний вивіз благородного металу
настає тоді, коли вже постають труднощі щодо зворотного припливу капіталів,
коли ринки переповнені, а подоба розцвіту зберігається тільки за допомогою
кредиту; отже, коли вже є дуже збільшений попит на позичковий капітал, а тому
й рівень проценту вже досяг, принаймні, своєї середньої величини. Серед цих
обставин, що відбиваються у відпливі саме благородного металу, значно більшає
вплив невпинного витягування капіталу в такій формі, в якій він існує безпосередньо
як грошовий позичковий капітал. Це мусить безпосередньо впливати
на рівень проценту. Але замість обмежувати кредитові операції, це піднесення
рівня проценту поширює їх та приводить до надмірного напруження всіх їхніх
допоміжних засобів. Тому цей період передує крахові.

Newmarch’а питають (В. А. 1857): «1520. Отже, число векселів у циркуляції
зростає разом з рівнем проценту? — Здається, так. — 1522. За спокійних
звичайних часів головна книга є дійсне знаряддя обміну; але коли постають
труднощі, коли, напр., серед таких обставин, що мною вище були наведені, підвищується
норма банкового дисконту\dots{} тоді операції сходять цілком сами собою
до виписування векселів; ці векселі не тільки придатніші до того, щоб бути за
законний доказ зробленої справи, але вони й зручніші для дальших закупів
і насамперед їх можна уживати, як засіб кредиту, щоб визичати капітал», —
До цього долучається те, що, коли банк серед до певної міри загрозливих обставин
підвищує свою норму дисконту, а в наслідок цього одночасно стає ймовірним,
що банк обмежить реченець векселів, котрі йому доводиться дисконтувати,
— постає загальне побоювання, що це буде розвиватися crescendo\footnote{
Crescendo. від лат. cresco, — росту, чим раз більшаючи, зростаючи. \emph{Пр. Ред.}
}. Отже,
\parbreak{}  %% абзац продовжується на наступній сторінці

\parcont{}  %% абзац починається на попередній сторінці
\index{ii}{0076}  %% посилання на сторінку оригінального видання
праця стає найманою працею; тому капіталістична продукція (а значить,
і товарова продукція) з’являється в цілому своєму об’ємі лише тоді,
коли й безпосередній сільський продуцент є найманий робітник. В відношенні
між капіталістом і найманим робітником грошове відношення, відношення покупця і продавця, стає
відношенням іманентним самій продукції. Але це відношення в основі своїй ґрунтується на суспільному
характері продукції, а не способу обміну; цей останній, навпаки, випливає з першого.
А проте, буржуазному світоглядові, де всю увагу звертається на практичні
операції, саме й відповідає погляд, що не в характері способу продукції
треба вбачати основу відповідного йому способу обміну, а навпаки\footnote{
До цього місця, рукопис V, — Все, що далі до кінця розділу, це замітка, яка є в зшитку з 1877 або
1878 року серед витягів з книжок.
}.

\pfbreak{}

\label{original-76}
Капіталіст кидає в циркуляцію менше вартости в грошовій формі,
ніж бере з неї, бо він кидає в неї більше вартости в товаровій формі,
ніж узяв звідти в товаровій формі. Оскільки він функціонує лише як персоніфікація
капіталу, як промисловий капіталіст, остільки його подання
товарових вартостей завжди більше, ніж його попит на товарові вартості.
Коли б його подання й попит взаємно покривались, то з цього погляду
це значило б, що його капітал не зростає вартістю; капітал не функціонував
би як продуктивний капітал; продуктивний капітал перетворився б
на товаровий капітал, не запліднений додатковою вартістю; підчас продукційного
процесу він не видобував би з робочої сили жодної додаткової
вартости в товаровій формі, отже, зовсім не функціонував би як
капітал; капіталіст дійсно мусить „продавати дорожче, ніж купив“, але
це вдається йому лише тому, що він за допомогою капіталістичного
продукційкого процесу перетворив куплений ним дешевший товар, — бо він
є товар меншої вартости, — на товар більшої вартости, тобто на дорожчий.
Він продає дорожче не тому, що продає свій товар вище понад його
вартість, а тому, що продає товар, який має вартість вищу, ніж сума
вартостей складових елементів його продукції.

Норма, що за нею капіталіст збільшує вартість свого капіталу, то більша,
що більша різність між його поданням і попитом, тобто що більший
надлишок тієї товарової вартости, яку він подає, проти тієї товарової
вартости, на яку він ставить попит. Його мета не та, щоб попит і
подання навзаєм покривались, а щоб вони якомога більше не покривались,
щоб його подання перекривало його попит.

Те, що має силу для поодинокого капіталіста, має силу й для кляси
капіталістів.

Оскільки капіталіст є лише персоніфікація промислового капіталу,
остільки його власний попит є лише попит на засоби продукції та
робочу силу. Його попит на $Зп$, розглядуваний щодо вартости,
менший, ніж його авансований капітал; він купує засоби продукції
за меншу вартість, ніж вартість його капіталу, а тому й за
\parbreak{}  %% абзац продовжується на наступній сторінці

\parcont{}  %% абзац починається на попередній сторінці
\index{ii}{0077}  %% посилання на сторінку оригінального видання
значно меншу вартість, ніж вартість того товарового капіталу, що
його він подає.

Щодо його попиту на робочу силу, то він, щодо вартости, визначається
відношенням його змінного капіталу до його цілого капіталу,
тобто дорівнює v: С, і тому зростає в капіталістичній продукції
відносно менше, ніж його попит на засоби продукції. Капіталіст
у дедалі більшій мірі більше купує Зп, ніж купує Р.

Що робітник перетворює свою заробітну плату переважно на засоби
існування, а найбільшу частину її — на доконечні засоби існування, то попит
капіталіста на робочу силу посередньо є разом з тим попит на засоби споживання,
що ввіходять у споживання робітничої кляси. Але цей попит
дорівнює v, і він на жоден атом не є більший від v (коли робітник зоощаджує
з своєї заробітної плати — кредитові відносини ми повинні лишити
тут осторонь — то це значить, що він частину своєї заробітної плати
перетворює на скарб і pro tanto\footnote*{
Остільки, відповідно до цього. \emph{Ред.}
} вже не виступає як особа, що ставить
попит, не як покупець). Максимальна межа попиту капіталіста дорівнює
C \deq{} $c \dplus{} v$, а його подання дорівнює $c \dplus{} v \dplus{} m$; отже, коли будова
його товарового капіталу є $80c \dplus{} 20v \dplus{} 20m$, то попит його дорівнює
80с \dplus{} 20v, отже, розглядуваний щодо його вартости, він на \sfrac{1}{5} менший
від його подання. Що більше відсоткове відношення спродукованої від нього
маси m (норма зиску),\footnote*{Норма зиску це є відношення маси додаткової вартости до цілого авансованого
капіталу. Про це дивись: Маркс, „Капітал“, т. III, ч. І, розд. II. —
\emph{Ред.}}
) то менший стає його попит порівняно з його
поданням. Хоч попит капіталіста на робочу силу, а тому, посередньо, і
на доконечні засоби існування, з розвитком продукції дедалі меншає порівняно
з його попитом на засоби продукції, все ж, з другого боку, не
треба забувати, що його попит на Зп завжди менший, ніж його капітал,
обчислюючи з дня на день. Отже, попит його на засоби продукції
мусить завжди бути меншої вартости, ніж товаровий продукт капіталіста,
що постачає йому ці засоби продукції та працює з однаковим
капіталом і за однакових інших обставин. Та обставина, що є
багато капіталістів, а не один, справи аж ніяк не змінює. Припустімо,
що його капітал дорівнює 1000\pound{ ф. стерл.}, стала частина його дорівнює
800\pound{ ф. стерл}. Тоді попит його до всіх капіталістів дорівнює 800\pound{ ф. стерл.},
а всі вони разом на кожні 1000\pound{ ф. стерл.} (хоч скільки з цієї суми припадає
на кожного з них зокрема і хоч яку частину цілого капіталу його
становить сума, яка припадає кожному) постачають, за однакової норми
зиску, засобів продукції вартістю в 1200\pound{ ф. стерл.}; отже, його попит покриває
лише \sfrac{2}{3} їхнього подання, тимчасом як увесь його власний попит,
розглядуваний щодо величини його вартости, дорівнює лише \sfrac{4}{5}
його власного подання.

Тепер ми ще мусимо, забігаючи наперед, розглянути, між іншим, оборот
капіталу\footnote*{Про поняття „оборот капіталу“ див. далі розділ VII. — \emph{Ред.}}. Припустімо, що ввесь капітал даного капіталіста дорівнює
\parbreak{}  %% абзац продовжується на наступній сторінці


\index{iii1}{0078}  %% посилання на сторінку оригінального видання
b) Норма зиску лишається незмінною тільки в тому разі, коли
$е = Е$, тобто коли дріб $\frac{v}{К}$ при позірній зміні зберігає те саме
значення, тобто якщо чисельник і знаменник помножуються або
діляться на те саме число. $80 с + 20 v + 20 m$ і $160 с + 40 v + 40 m$,
очевидно, мають однакову норму зиску в 20\%, бо ті лишається
= 100\%, а $\frac{v}{К} = \frac{20}{100} = \frac{40}{200}$ в обох прикладах
представляє ту саму величину.

c) Норма зиску підвищується, якщо $е$ більше за $Е$, тобто
якщо змінний капітал зростає в більшій пропорції, ніж весь капітал.
Якщо $80с + 20v + 20m$ стає $120 с + 40 v + 40 m$, то норма
зиску підвищується від 20\% до 25\%, бо, при незмінному $m'$,
$\frac{v}{К} = \frac{20}{100}$ підвищилось до \frac{40}{160}, з \sfrac{1}{5}
до \sfrac{1}{4}.

При зміні $v$ і $К$ в одному напрямі ми можемо цю зміну величин
розглядати так, ніби обидві величини змінюються до певної
межі в однаковій пропорції, так що до цієї межі $\frac{v}{К}$ лишається
незмінним. Поза цією межею стала б змінюватись тільки
одна з двох величин, і ми таким чином звели б цей складніший
випадок до одного з попередніх простіших.

Якщо, наприклад, $80 с + 20 v + 20 m$ переходить у
$100 c + 30 v + 30 m$, то при цій зміні до $100с + 25v + 25m$ відношення
$v$ до $с$, отже, і до $К$, лишається незмінним. Отже, до цього
пункту і норма зиску лишається незачепленою. Тому ми можемо
взяти тепер за вихідний пункт $100 с + 25v + 25m$; ми бачимо,
що v підвищилось на 5, до $30v$, а в наслідок цього $К$ підвищилось
від 125 до 130, і маємо таким чином перед собою другий
випадок, випадок простої зміни $v$ та спричиненої цим
зміни $К$. Норма зиску, яка спочатку була 20\%, в наслідок такої
додачі в $5v$, при попередній нормі додаткової вартості, підвищується
до $23\sfrac{1}{13}\%$.

Таке саме зведення до простішого випадку може мати місце
навіть тоді, коли $v$ і $К$ змінюють свою величину в протилежному
напрямі. Коли б ми знову виходили, наприклад, з
$80с + 20v + 20m$ і від цього перейшли б до форми: $110с + 10v + 10m$, то
при зміні до $40с + 10v + 10m$ норма зиску лишилася б така
сама, як і спочатку, а саме 20\%. В наслідок додачі $70с$ до цієї
перехідної форми норма зиску знижується до $8\sfrac{1}{3}\%$. Отже, цей
випадок ми знову звели до випадку зміни однієї тільки змінної,
а саме $с$.

Отже, одночасна зміна $v$, $с$ і $К$ не дає нових точок зору і в
кінцевому рахунку завжди веде назад до випадку, коли змінюється
тільки один фактор.

Навіть єдиний випадок, який ще лишається, уже фактично
вичерпаний, а саме той випадок, коли $v$ і $К$ чисельно зберігають
\parbreak{}  %% абзац продовжується на наступній сторінці

\input{_0079.tex}

\index{iii1}{0080}  %% посилання на сторінку оригінального видання
\begin{center}
  \textbf{II. $m'$ змінюється}
\end{center}

Загальну формулу норм зиску при різних нормах додаткової
вартості, однаково, чи  \frac{v}{K} лишається незмінним, чи теж змінюється,
ми одержимо, коли рівняння:\[p' = m' \frac{v}{К}\]

перетворимо в інше:
\[p'\textsubscript{1} = m'\textsubscript{1} \frac{v\textsubscript{1}}{K\textsubscript{1}},\]

де $р'\textsubscript{1}, m'\textsubscript{1}, v\textsubscript{1}, і К\textsubscript{1}$ означають змінені величини $р', m', v$ і $К$.
Тоді ми маємо: \[p': p'\textsubscript{1} = m' \frac{v}{K}: m'\textsubscript{1} \frac{v\textsubscript{1}}{K\textsubscript{1}},\]

і звідси:\[p'\textsubscript{1} = \frac{m'\textsubscript{1}}{m'} × \frac{v\textsubscript{1}}{v} × \frac{K}{K\textsubscript{1}} × p'\].

\begin{center}
\textbf{1. $m'$ змінюється, $\frac{v}{K}$ не змінюється}
\end{center}

В цьому випадку ми маємо рівняння:

\[p' = m'\frac{v}{K}; p'\textsubscript{1} = m'\textsubscript{1} \frac{v}{K},\]

в обох рівняннях \frac{v}{K} має однакову величину. Тому одержуємо
відношення:

\[р': р'\textsubscript{1} = m': m'\textsubscript{1}.\]

Норми зиску двох капіталів однакового складу відносяться
одна до одної, як відповідні норми додаткової вартості. Через
те що в дробу $\frac{v}{K}$ важливі не абсолютні величини $v$ і $К$, а тільки
відношення між ними, то це стосується й до всіх капіталів однакового
складу, яка б не була їх абсолютна величина.

\begin{center}
$80с + 20v + 20m; K = 100, m' = 100\%, p' = 20\%$

$160c + 40v + 20m; K = 200, m' = 50\%, p' = 10\%$

 $100\%: 50\% = 20\%: 10\%.$
\end{center}

Якщо абсолютні величини $v$ і $К$ в обох випадках однакові,
то норми зиску відносяться одна до одної, крім того, як маси
додаткової вартості:

$p': p'\textsubscript{1} = m'v: m'1v = m: m\textsubscript{1}.$


\index{iii1}{0081}  %% посилання на сторінку оригінального видання
Наприклад:

80 с + 20 v + 20 m; m' = 100\%, p' = 20\%
80 с + 20 v + 10 m; m' = 50\%, p' = 10\%
20\% : 10\% = 100 × 20 : 50 × 20 = 20 m : 10 m.

Тепер ясно, що при капіталах однакового абсолютного чи
процентного складу норми додаткової вартості можуть бути
різні тільки в тому випадку, коли різні або заробітна плата,
або довжина робочого дня, або інтенсивність праці. В трьох
випадках:

I.  80 с + 20 v + 10 m; m' = 50\%, p' = 10\%,
II. 80 с + 20 v + 20 m; m' = 100\%, p' = 20\%,
III. 80 с + 20 v + 40 m; m' = 200\%, p' = 40\%,

вся нововироблена вартість буде в І 30 (20 v + 10 m), в II — 40,
в III — 60. Це може статись трояким способом.

Поперше, якщо заробітні плати різні, отже, якщо 20 v в кожному
окремому випадку виражає різне число робітників. Припустім,
що в І занято 15 робітників 10 годин при заробітній
платі в 1  1/3  фунтів стерлінгів і що вони виробляють вартість
у 30 фунтів стерлінгів, з яких 20 фунтів стерлінгів заміщають
заробітну плату, а 10 фунтів стерлінгів лишаються для додаткової
вартості. Якщо заробітна плата падає до 1 фунта стерлінгів,
то можуть бути заняті 20 робітників 10 годин; тоді вони
виробляють вартість у 40 фунтів стерлінгів, з яких 20 фунтів
стерлінгів для заробітної плати і 20 фунтів стерлінгів додаткової
вартості. Якщо заробітна плата падає ще далі, до 2/3 фунтів
стерлінгів, то можуть бути заняті 30 робітників по 10 годин,
які виробляють вартість у 60 фунтів стерлінгів, що з них після
відрахування 20 фунтів стерлінгів для заробітної плати залишиться
ще 40 фунтів стерлінгів для додаткової вартості.

Цей випадок: незмінний процентний склад капіталу, незмінний
робочий день, незмінна інтенсивність праці, зміна норми
додаткової вартості, спричинена зміною заробітної плати — є
єдиний випадок, на якому справджується положення Рікардо:
„profits would be high or low, exactly in proportion as wages
would be, low or high“ [„зиск буде високий чи низький точно
в такій пропорції, в якій заробітна плата буде низька чи висока“]
(„Principles of Political Economy“, розд. І, відділ III, стор. 18.
„Works of D. Ricardo“, вид. Mac Culloch, 1852).

Або, подруге, якщо інтенсивність праці різна. Тоді, наприклад,
20 робітників при однакових засобах праці за 10 робочих
годин на день виробляють у І — 30, у II — 40, у III — 60 штук
певного товару, кожна штука якого, крім вартості спожитих
на неї засобів виробництва, представляє нову вартість в 1 фунт
стерлінгів. Через те що в кожному випадку 20 штук, = 20
фунтам стерлінгів, заміщають заробітну плату, то для додаткової
\index{iii1}{0082}  %% посилання на сторінку оригінального видання
вартості лишаються в І — 10 штук = 10 фунтам стерлінгів,
в II — 20 штук = 20 фунтам стерлінгів, в III — 40 штук = 40 фунтам
стерлінгів.

Або, потретє, робочий день — різної довжини. Якщо 20 робітників
при однаковій інтенсивності працюють у І — дев’ять,
у II — дванадцять, у III — вісімнадцять годин на день, то весь їх
продукт 30 : 40 : 60 відноситься як 9 : 12 : 18, і тому що заробітна
плата в кожному випадку = 20, то знову лишається 10, відповідно
20 і 40 для додаткової вартості.

Отже, підвищення або зниження заробітної плати діє в зворотному
напрямі, підвищення або зниження інтенсивності праці
і здовження або скорочення робочого дня діє в тому самому
напрямі на висоту норми додаткової вартості, а тому, при незмінному
v/K, і на норму зиску.

2. m' і v змінюються, К не змінюється

В цьому випадку має силу пропорція:

p': p'1 = m' v/K : m'1 v1/K = m'v : m'1v1 = m : m1.

Норми зиску відносяться одна до одної, як відповідні маси
додаткової вартості.

Зміна норми додаткової вартості при незмінній величині змінного
капіталу означала зміну у величині й розподілі нововиробленої
вартості. Одночасна зміна v і m' так само завжди включає
інший розподіл, але не завжди зміну величини нововиробленої
вартості. Можливі три випадки:

a) Зміни v і m' відбуваються в протилежному напрямі, але
на однакову величину; наприклад:

80 с + 20 v + 10 m; m' = 50\%, p' = 10\%
90 с + 10 v + 20 m; m' = 200\%, p' = 20\%.

Нововироблена вартість в обох випадках однакова, отже, однакова
й кількість витраченої праці; 20 v + 10 m = 10 v + 20 m = 30.
Ріжниця тільки в тому, що в першому випадку 20 сплачується
як заробітна плата, а 10 лишається для додаткової вартості,
тимчасом як у другому випадку заробітна плата становить
тільки 10, а тому додаткова вартість — 20. Це єдиний випадок,
коли при одночасній зміні v і m' число робітників, інтенсивність
праці і довжина робочого дня лишаються незміненими.

b) Зміни m' і v відбуваються так само в протилежному напрямі,
але не на ту саму величину. Тоді перевага буде або на
стороні зміни v, або на стороні зміни m'.

I. 80 с + 20 v + 20 m; m' = 100\%, p' = 20\%
II.72 с + 28 v + 20 m; m' = 71 3/7\%, p' = 20\%
III. 84 с + 16 v + 20 m; m' = 125\%, p' = 20\%.


\index{iii1}{0083}  %% посилання на сторінку оригінального видання
В І за нововироблену вартість у 40 сплачується $20 v$, в II
за нововироблену вартість у 48 сплачується $28 v$, в III за нововироблену
вартість у 36 сплачується $16 v$. Як нововироблена
вартість, так і заробітна плата змінились; але зміна нововиробленої
вартості означає зміну кількості витраченої праці, отже,
або числа робітників, або тривалості праці, або інтенсивності
праці, або кількох з цих трьох факторів.

с) Зміни $m'$ і $v$ відбуваються в тому самому напрямі; тоді
одна підсилює вплив другої.

\begin{gather*}
90 с + 10 v + 10 m; m' = 100\%, p' = 10\% \\
80 с + 20 v + 30 m; m' = 150\%, p' = 30\% \\
92 с + \phantom{$0$}8 v + \phantom{$0$}6 m; m' = \phantom{$1$}75\%, p' = \hphantom{$1$}6\%
\end{gather*}

\noindent{}І тут усі три нововироблені вартості різні, а саме 20, 50 і 14;
і ця ріжниця в величині витрачуваної в кожному випадку кількості
праці знову зводиться до ріжниці числа робітників, тривалості
праці, інтенсивності праці, або кількох, а то й усіх цих факторів.

\subsection[m', v і К змінюються]{m', v і К змінюються\footnotemarkZ{}}
\footnotetextZ{
В першому німецькому виданні: 3. Примітка ред. нім. вид. ІМЕЛ.
}

\noindent{}Цей випадок не дає нових точок зору і розв’язується за допомогою
загальної формули, даної в рубриці: II. $m'$ змінюється.

\pfbreak{}

Отже, вплив зміни величини норми додаткової вартості на
норму зиску дає такі випадки:

1. $р'$ збільшується або зменшується в тій самій пропорції, як
i $m'$, якщо $\frac{v}{K}$  лишається незмінним.

\begin{gather*}
80 с + 20 v + 20 m; m' = 100\%, p' = 20\% \\
80 с + 20 v + 10 m; m' = \phantom{1}50\%, p' = 10\% \\
100\% : 50\% = 20\% : 10\%.
\end{gather*}

2. $р'$ підвищується або падає в більшій пропорції, ніж $m'$,
якщо $\frac{v}{K}$ рухається в тому самому напрямі, що й $m'$, тобто
збільшується чи зменшується, коли збільшується чи зменшується
$m'$.

\begin{gather*}
80 с + 20 v + 10 m; m' = 50\phantom{\sfrac{2}{3}}\%, p' = 10\% \\
70 с + 30 v + 20 m; m' = 66\sfrac{2}{3}\%, p' = 20\% \\
50\% : 66 2/3\% < 10\% : 20\%.\text{\footnotemarkZ{}}
\end{gather*}

\footnotetextZ{
Знак $<$ означає тут, що збільшення з 50 до 66\sfrac{2}{3} є порівняно менше, ніж
збільшення з 10 до 20. Знак $>$ у дальшій формулі означає зворотне. Примітка
ред. нім. вид. ІМЕЛ.
}



\index{iii1}{0084}  %% посилання на сторінку оригінального видання
3. р' підвищується або падає в меншій пропорції, ніж m',
якщо v/K змінюється в напрямі, протилежному до зміни m', але
в меншій пропорції:\footnote{
с + 20 v + 20 m; m' = 100\%, p' = 20\%
90 с + 10 v + 15 m; m' = 150\%, p' = 15\%

m' підвищилось з 100\% до 150\%, р' зменшилось від 20\% до 15\%.
} с + 20 v + 10 m; m' = 50\%, p' = 10\%
90 с + 10 v + 15 m; m' = 150\%, p' = 15\%
50\% : 150\% > 10\% : 15\%.

4. р' підвищується, хоч m' падає, або падає, хоч m' підвищується,
якщо v/K змінюється в напрямі, протилежному до зміни
m', і в більшій пропорції, ніж m'.

5. Нарешті, р' лишається незмінним, хоч m' підвищується або
падає, якщо v/K змінює свою величину в протилежному напрямі,
але точно в тій самій пропорції, що й m'.

Тільки цей останній випадок потребує ще деякого пояснення.
Як ми бачили вище при змінах v/K, що одна й та сама норма
додаткової вартості може виражатися в найрізніших нормах
зиску, так ми бачимо тут, що в основі однієї і тієї самої норми
зиску можуть лежати дуже різні норми додаткової вартості.
Але в той час, як при незмінному m' першої-ліпшої зміни у відношенні
v до К досить було для того, щоб викликати відмінність
в нормі зиску, — при зміні величини m' мусить настати точно
відповідна зворотна зміна величини v/K для того, щоб норма
зиску лишилась та сама. Для одного й того ж капіталу або для
двох капіталів у тій самій країні це можливе тільки в дуже
виняткових випадках. Візьмімо, наприклад, капітал\footnote{
с + 16 v + 24 m; K = 96, m' = 150\%, p' = 25\%.

Отже, для того, щоб р' було, як і раніш, = 20\%, весь капітал
мусив би зрости до 120, отже, сталий — до 104:
} с + 20 v + 20 m; K = 100, m' = 100\%, p' = 20\%

і припустімо, що заробітна плата впала настільки, що тепер за
16 v можна було б мати те саме число робітників, як раніш за
20 v. Тоді ми, при інших незмінних умовах і звільненні 4 v,
маємо

104 с + 16 v + 24 m; K = 120, m' = 150\%, p' = 20\%


\index{iii1}{0085}  %% посилання на сторінку оригінального видання
Це було б можливе тільки тоді, коли б одночасно з зниженням
заробітної плати настала зміна в продуктивності праці, яка
вимагає цього зміненого складу капіталу; абож коли б грошова вартість
сталого капіталу підвищилась з 80 до 104; коротко кажучи —
такий випадковий збіг обставин, який буває тільки в виняткових
випадках. В дійсності така зміна $m'$, що не зумовлює одночасно
зміни $v$, а тому й зміни $\frac{v}{K}$, мислима тільки при цілком певних
обставинах, а саме в таких галузях промисловості, де застосовується
тільки основний капітал і праця, а предмет праці дається
природою.

Але при порівнянні норм зиску двох країн справа стоїть
інакше. Тут одна й та сама норма зиску в дійсності виражає
здебільшого різні норми додаткової вартості.

Отже, з усіх п’яти випадків випливає, що ростуща норма
зиску може відповідати падаючій або ростущій нормі додаткової
вартості, падаюча норма зиску — ростущій або падаючій
нормі додаткової вартості, незмінна норма зиску — ростущій або
падаючій нормі додаткової вартості. Що ростуща, падаюча або
незмінна норма зиску може також відповідати незмінній нормі
додаткової вартості, це ми бачили під рубрикою І.

\pfbreak

Отже, норма зиску визначається двома головними факторами:
нормою додаткової вартості і вартісним складом капіталу.
Вплив обох цих факторів можна коротко резюмувати, — при чому
склад ми можемо виразити в процентах, бо тут не має значення,
з якої з двох частин капіталу походить зміна, — таким чином:

Норми зиску двох капіталів або одного й того ж капіталу
в двох послідовних різних його станах

\emph{є рівні:}

1) при однаковому процентному складі капіталів і однаковій
нормі додаткової вартості;

2) при неоднаковому процентному складі і неоднаковій нормі
додаткової вартості, якщо добутки з норм додаткової вартості
і взятих у процентах змінних частин капіталу ($m'$ на $v$) є рівні,
тобто якщо \emph{маси} додаткової вартості ($m = m'v$), взяті в процентному
відношенні до всього капіталу, є рівні; інакше кажучи,
якщо в обох випадках множники $m'$ і $v$ стоять у зворотному
відношенні один до одного.

\emph{Вони нерівні:}

1) при однаковому процентному складі, якщо норми додаткової
вартості нерівні, при чому норми зиску відносяться одна до
одної, як норми додаткової вартості;
\parbreak{}  %% абзац продовжується на наступній сторінці

\parcont{}  %% абзац починається на попередній сторінці
\index{iii1}{0086}  %% посилання на сторінку оригінального видання
2) при однаковій нормі додаткової вартості і неоднаковому
процентному складі, при чому норми зиску відносяться одна до
одної, як змінні частини капіталу;

3) при неоднаковій нормі додаткової вартості і неоднаковому
процентному складі, при чому норми зиску відносяться одна до
одної, як добутки $m'v$, тобто як маси додаткової вартості,
взяті в процентному відношенні до всього капіталу.\footnote{
В рукопису є ще дуже докладні обчислення щодо різності між нормою
додаткової вартості і нормою зиску $(m' — р')$; ця різність має різні цікаві особливості,
рух її показує випадки, коли обидві норми віддаляються одна від
одної або наближаються одна до одної. Ці рухи можуть бути зображені у формі
кривих. Я відмовляюсь від відтворення цього матеріалу, бо для ближчих цілей
цієї книги він менш важливий; тут досить просто звернути на це увагу тих
читачів, які захочуть далі простежити це питання. — Ф. Е.
}

\section{Вплив обороту на норму зиску}

[Вплив обороту на виробництво додаткової вартості, отже
й зиску, з’ясовано в другій книзі. Його можна коротко зрезюмувати
таким чином, що в наслідок того, що на оборот потрібен
певний час, на виробництво не може бути застосований одночасно
весь капітал; що, отже, частина капіталу постійно лежить без
діла, чи то в формі грошового капіталу, запасних сировинних
матеріалів, готового, але ще не проданого товарного капіталу,
чи в формі боргових вимог, для яких ще не настав строк платежу;
що капітал, який діє в активному виробництві, тобто при
створенні і привласненні додаткової вартості, постійно зменшується
на цю частину, при чому в такій самій пропорції постійно
зменшується створювана і привласнювана додаткова вартість.
Чим коротший час обороту, тим меншою порівняно з усім
капіталом стає ця частина капіталу, яка лежить без діла; і тим
більшою, отже, стає, при інших незмінних умовах, привласнювана
додаткова вартість.

Уже в другій книзі докладно розвинуто, як скорочення часу
обороту або одного з двох його підрозділів, часу виробництва
і часу циркуляції, підвищує масу вироблюваної додаткової вартості.
Але через те що норма зиску виражає тільки відношення
виробленої маси додаткової вартості до всього капіталу, занятого
в її виробництві, то очевидно, що всяке таке скорочення підвищує
норму зиску. Те, що раніше, в другому відділі другої
книги, розвинуто щодо додаткової вартості, в такій самій мірі
стосується і до зиску та норми зиску і не потребує тут повторення.
Ми хочемо відзначити лиш декілька головних моментів.

Головний засіб скорочення часу виробництва є підвищення
продуктивності праці, що звичайно називають прогресом промисловості.
Якщо цим одночасно не викликається значне збільшення
\index{iii1}{0087}  %% посилання на сторінку оригінального видання
загальних витрат капіталу в наслідок застосування дорогих
машин і т. д., а тому й зниження норми зиску, обчислюваної
на весь капітал, то ця остання мусить підвищитись. І це, безперечно,
має місце при багатьох з найновіших успіхів металургії
і хемічної промисловості. Нововідкриті способи виготовлення
заліза й сталі — Бессемера, Сіменса, Гількріста-Томаса та інших —
скорочують до мінімуму, при відносно незначних витратах,
надзвичайно довгочасні раніш процеси. Виготовлення алізарину
або красильної речовини крапу з кам’яновугільного дьогтю дає
за кілька тижнів, і до того ж при фабричних приладдях, які
вже раніш уживалися для виготовлення фарб з кам’яновугільного
дьогтю, такий самий результат, який раніше вимагав цілих
років; один рік був потрібний для росту крапу, а потім ще
кілька років коріння лишали достигати, раніше ніж уживати
його для фарбування.

Головний засіб скорочення часу циркуляції є поліпшені
шляхи сполучення. І в цьому останні п’ятдесят років зробили
революцію, яку можна порівняти тільки з промисловою революцією
останньої половини минулого століття. На суходолі макадамізовані\footnote*{
Макадамізування — спосіб брукування шляхів за системою Мак-Адама,
при якому скальне каміння укочується круглими котками. Ред. укр. перекладу,
} шляхи відтиснені на задній план залізницею, на
морі повільне і нерегулярне вітрильне сполучення — швидким
і регулярним пароплавним сполученням, і вся земна куля обвивається
телеграфними дротами. Власне кажучи, тільки Суецький
канал і відкрив Східну Азію і Австралію для пароплавного сполучення.
Час циркуляції для товарів, що посилалися до Східної
Азії, який ще в 1847 році становив щонайменше дванадцять
місяців, тепер можна звести майже
до стількох же тижнів. Два великі огнища криз 1825—1857 рр.,
Америка і Індія, в наслідок цього перевороту в засобах сполучення
наблизились до європейських промислових країн на
70—90\% і тим самим утратили більшу частину своєї здатності
до вибухів. Час обороту всієї світової торгівлі скоротився
в такій самій мірі, а дієздатність капіталу, який бере в ній
участь, підвищилась більше, ніж удвоє чи утроє. Що це не
лишилось без впливу на норму зиску, зрозуміло само собою.

Щоб представити в чистому вигляді вплив обороту всього
капіталу на норму, зиску ми мусимо при порівнянні двох
капіталів припустити, що всі інші обставини однакові. Отже,
крім норми додаткової вартості і робочого дня, нехай буде
однаковий і процентний склад капіталів. Візьмім тепер капітал
$А$ з складом $80с + 20v = 100К$, що обертається двічі на рік
при нормі додаткової вартості в 100\%. Тоді річний продукт
буде:

$160 с + 40 v + 40 m$. Але для визначення норми зиску ми обчисляємо
ці $40 m$ не на капітальну вартість у 200, що обернулась,
\index{iii1}{0088}  %% посилання на сторінку оригінального видання
а на авансовану капітальну вартість у 100, і таким чином
одержуємо: $р' = 40\%$.

Порівняймо з цим капітал $В = 160с + 40v=200K$, який
функціонує при такій самій нормі додаткової вартості в 100\%,
але обертається тільки один раз на рік. Тоді річний продукт
буде, як і вище:

$160с + 40 v + 40 m$. Але на цей раз ці $40 m$ слід обчислити на
авансований капітал у 200; це дає для норми зиску тільки 20\%,
отже, тільки половину норми для $А$.

Звідси випливає: при капіталах однакового процентного
складу, при однаковій нормі додаткової вартості і однаковому
робочому дні, норми зиску двох капіталів стоять у зворотному
відношенні до часу їх оборотів. Якщо в двох порівнюваних випадках
неоднаковий склад, або норма додаткової вартості, або
робочий день, або заробітна плата, то цим, звичайно, будуть
породжені й дальші ріжниці в нормі зиску; але вони незалежні
від обороту і тому нас не цікавлять тут; вони вже розглянуті
в розділі III.

Безпосередній вплив скороченого часу обороту на виробництво
додаткової вартості, отже й зиску, полягає в підвищеній
діяльності, яка таким способом надається змінній частині капіталу,
про що див. книгу II, розділ XVI: Оборот змінного
капіталу. Там виявилось, що змінний капітал у 500, який обертається
десять разів на рік, привласнює за цей час стільки ж додаткової
вартості, як і змінний капітал у 5000, який при однаковій
нормі вартості і однаковій заробітній платі обертається
тільки один раз на рік.

Візьмемо капітал І, що складається з 10 000 основного капіталу,
— річне зношування якого становить $10\% = 1000,$ — 500
обігового сталого і 500 змінного капіталу. При нормі додаткової
вартості в 100\% цей змінний капітал обертається десять
разів на рік. Задля спрощення ми припускаємо в усіх дальших
прикладах, що обіговий сталий капітал обертається за той
самий час, як і змінний, що й на практиці здебільшого приблизно
так і буває. Тоді продукт одного такого періоду обороту буде:\[
100 с \text{(зношування)} + 500 с + 500 v + 500 m = 1600,\]
а продукт цілого року з десятьма такими оборотами:
\begin{center}
$1000 с \text{(зношування)} + 5000 с + 5000 v + 5000 m = 16 000$,
 $К = 11 000; m = 5000, р' = \frac{5000}{11000} + 45 \sfrac{5}{11}\%$.
\end{center}

Візьмемо тепер капітал II: основний капітал — 9000, його річне
зношування — 1000, обіговий сталий капітал — 1000, змінний
капітал — 1000, норма додаткової вартості — 100\%, число річних
оборотів змінного капіталу — 5. Отже, продукт кожного періоду
обороту змінного капіталу буде:\[
200c \text{(зношування)} + 1000 c + 1000 v + 1000 m = 3200,\]

а весь річний продукт при п’яти оборотах:

\begin{center}
$1000с\text{(зношування)} + 5000 с + 5000 v + 5000 m = 16 000$,

$К = 11 000, m = 5000, р' = \frac{5000}{11000} = 45 \sfrac{5}{11}\%$.
\end{center}

Візьмімо далі капітал III, в якому зовсім немає основного капіталу,
але є 6000 обігового сталого і 5000 змінного капіталу. При нормі додаткової
вартості в 100\% він обертається один раз на рік. Тоді весь продукт за рік буде:
\begin{center}
$6000c + 5000 v + 5000 m = 16 000,$

$К = 11000, m = 5000, р' = \frac{5000}{11000} = 45\sfrac{5}{11}\%.$
\end{center}
Отже, в усіх трьох випадках ми маємо однакову річну масу
додаткової вартості = 5000, а через те що весь капітал в усіх
трьох випадках теж однаковий, а саме = 11 000, то маємо
й однакову норму зиску в 45\sfrac{5}{11}\%.

Навпаки, якщо при капіталі І ми мали б не 10, а тільки
5 річних оборотів змінної частини, то справа стояла б інакше.
Тоді продукт одного обороту був би:\[
200 с \text{(зношування)} + 500 с + 500 v + 500 m = 1700.\]

Або річний продукт:

\index{iii1}{0089}  %% посилання на сторінку оригінального видання
\begin{center}
$1000с \text{(зношування)} + 2500 с + 2500 v + 2500 m = 8500,$

$К = 11 000, m = 2500, р' = \frac{2500}{11000} = 22 \sfrac{8}{11}\%.$
\end{center}
Норма зиску знизилася б наполовину, бо час обороту подвоївся.

Отже, маса додаткової вартості, привласнювана протягом року,
дорівнює масі додаткової вартості, привласнюваній за один період
обороту \emph{змінного} капіталу, помноженій на число таких оборотів
за рік. Якщо привласнювану за рік додаткову вартість або зиск
ми назвемо $М$, привласнювану за один період обороту додаткову
вартість — $m$, число річних оборотів змінного капіталу — $n$, то
$М = mn$, а річна норма додаткової вартості $М' = m'n$, як це
вже показано в книзі II, розд. XVI, 1.

Само собою зрозуміло, що формула норми зиску $р' = m' \frac{v}{K} =
m' \frac{v}{c+v}$ правильна тільки тоді, коли v чисельника однакове
з $v$ знаменника. У знаменнику $v$ є вся та частина всього капіталу,
яка пересічно застосована як змінний капітал на заробітну
плату; $v$ чисельника насамперед визначається тільки тим, що
воно виробило і привласнило певну кількість додаткової вартості
\index{iii1}{0090}  %% посилання на сторінку оригінального видання
 $= m$, відношення якої до нього, $m/v$, є норма додаткової
вартості $m'$. Тільки таким шляхом рівняння $р' = \frac{m}{c + v}$ перетворилось
в друге: $р' = m' \frac{v}{c + v}$. Тепер $v$ чисельника ближче визначається
тим, що воно мусить бути рівне $v$ знаменника, тобто
всій змінній частині капіталу $К$. Інакше кажучи, рівняння
$р' = m/K$ можна тільки тоді без помилки перетворити в друге рівняння
$р' = m' \frac{v}{c + v}$, коли $m$ означає додаткову вартість, вироблену
за один період обороту змінного капіталу. Якщо $m$ охоплює
тільки частину цієї додаткової вартості, то хоч $m — m'v$ є
правильне рівняння, але це $v$ тут менше, ніж $v$ в $K = с + v$, бо
воно менше, ніж весь змінний капітал, витрачений на заробітну
плату. Якщо ж $m$ охоплює більше, ніж додаткову вартість від
одного обороту $v$, то частина цього $v$ або навіть все $v$ функціонує
двічі: спочатку в першому, потім у другому або в другому
й дальших оборотах; отже, це $v$, яке виробляє додаткову
вартість і яке становить суму всієї виплаченої заробітної плати,
є більше, ніж $v$ в $c + v$, і тому обчислення стає неправильним.

Для того, щоб формула річної норми зиску стала цілком
правильною, ми повинні замість простої норми додаткової вартості
поставити річну норму додаткової вартості, тобто замість
$m'$ поставити $М'$, або $m'n$. Інакше кажучи, ми повинні помножити
$m'$, норму додаткової вартості — або, що зводиться до
того самого, вміщену в $К$ змінну частину капіталу $v$, — на $n$,
число оборотів цього змінного капіталу за рік, і таким чином
ми одержуємо: $р' = m'n \frac{v}{K}$, формулу для обчислення річної
норми зиску.

Але яка є величина змінного капіталу в певному підприємстві,
цього в більшості випадків не знає і сам капіталіст.
У восьмому розділі другої книги ми бачили і побачимо ще
далі, що єдина ріжниця в капіталі капіталіста, яка нав’язується
йому як істотна, є ріжниця основного й обігового капіталу.
З каси, в якій знаходиться частина обігового капіталу, яку він
має в своїх руках у грошовий формі, — оскільки вона не лежить
у банку, — він бере гроші для заробітної плати, з тієї самої
каси він бере гроші для сировинних і допоміжних матеріалів
і записує ті і другі на той самий рахунок каси. А коли б йому
й довелося вести окремий рахунок виплачуваної заробітної
плати, то цей рахунок в кінці року, правда, показав би виплачену
на заробітну плату суму, тобто $vn$, але не показав би
самого змінного капіталу $v$. Щоб визначити цей останній, капіталістові
\index{iii1}{0091}  %% посилання на сторінку оригінального видання
довелося б зробити окреме обчислення, приклад якого
ми хочемо тут навести.

Для цього ми візьмемо бавовнопрядільну фабрику на 10 000
мюльних веретен, описану в книзі І, стор. 227 \footnote*{
Стор. 152 рос. вид. 1935 р. Ред. укр. перекладу.
}, і припустимо
при цьому, що дані, взяті для одного тижня квітня 1871 року,
зберігають своє значення для цілого року. Основний капітал, вміщений
у машинах, становив 10 000 фунтів стерлінгів. Обіговий
капітал не був указаний; припустімо, що він становив 2 500 фунтів
стерлінгів, — досить висока сума, що виправдується, однак, тим
припущенням, яке ми тут весь час мусимо робити, а саме, що
не відбувається ніяких кредитних операцій, отже, що немає
тривалого чи тимчасового користування чужим капіталом. Тижневий
продукт щодо своєї вартості складався з 20 фунтів стерлінгів
на зношування машин, 358 фунтів стерлінгів авансованого
обігового сталого капіталу (плата за найом — 6 фунтів стерлінгів,
бавовна — 342 фунти стерлінгів, вугілля, газ, мастило — 10 фунтів
стерлінгів), 52 фунтів стерлінгів витраченого на заробітну
плату змінного капіталу і 80 фунтів стерлінгів додаткової вартості,
отже:\[
20с \text{(зношування)} + 358 с + 52 v + 80 m = 510.\]

Отже, щотижневе авансування обігового капіталу становило
$358 с + 52 v = 410$, і його процентний склад $= 87,3 с + 12,7 v$. При
обчисленні на весь обіговий капітал у 2500 фунтів стерлінгів
це дає 2182 фунти стерлінгів сталого і 318 фунтів стерлінгів
змінного капіталу. Через те, що вся витрата на заробітну плату
становила на рік 52 рази по 52 фунти стерлінгів, отже, 2704 фунти
стерлінгів, то виходить, що змінний капітал у 318 фунтів стерлінгів
обернувся за рік майже точно $8 1/2$ разів. Норма додаткової
вартості була  $\sfrac{80}{52} = 153 \sfrac{11}{13}\%$. За цими елементами ми обчисляємо
норму зиску, підставивши в формулу $р' = m'n \frac{v}{K}$ значення:
$m' = 153 \sfrac{11}{13}\%, n = 8 \sfrac{1}{2}, v = 318, K = 12500;$ отже:\[
р' = 153 \sfrac{11}{13} × 8 \sfrac{1}{2} × \frac{318}{12 500} = 33,27\%\]

Для перевірки цього ми скористуємось простою формулою
$р' = \frac{m}{K}$. Вся додаткова вартість, або зиск, становить за рік
52 фунти стерлінгів $× 80 = 4160$ фунтів стерлінгів; поділене на
весь капітал у 12 500 фунтів стерлінгів, це дає майже стільки ж, як
вище, 33,28\%, ненормально високу норму зиску, яка пояснюється
тільки надзвичайно сприятливими умовами даного моменту (дуже
дешеві ціни на бавовну поряд з дуже високими цінами на пряжу)
і в дійсності існувала, без сумніву, не на протязі всього року.



\index{iii1}{0092}  %% посилання на сторінку оригінального видання
В формулі $р' = m'n \frac{v}{K}$, як сказано, $m'n$ є те, що в другій
книзі названо річною нормою додаткової вартості. У вищенаведеному
випадку вона становить 153 \sfrac{11}{13}\% × 8 \sfrac{1}{2}, або точно
1307 \sfrac{9}{13}\%. Отже, якщо якийсь бравий чоловік сплеснув руками
з приводу потворності річної норми додаткової вартості в 1000\%,
наведеної в одному прикладі в другій книзі, то він, може, заспокоїться
на факті річної норми додаткової вартості понад
1300\%, який наведено йому тут з живої практики Манчестера.
В часи найвищого розквіту, яких ми, правда, давно вже не переживали,
така норма аж ніяк не є рідкість.

До речі сказати, ми маємо тут приклад дійсного складу капіталу
в сучасній великій промисловості. Весь капітал поділяється на
12182 фунти стерлінгів сталого і 318 фунтів стерлінгів змінного
капіталу, разом 12500 фунтів стерлінгів. Або в процентах:
$97 \sfrac{1}{2}c+ 2 \sfrac{1}{2}v = 100K$. Тільки сорокова частина всього капіталу,
але, повторно обертаючись більше ніж вісім разів на рік, служить
для виплати заробітної плати.

Через те що, звичайно, тільки небагатьом капіталістам спадає
на думку робити такі обчислення щодо свого власного підприємства,
то статистика майже абсолютно мовчить про відношення
сталої частини всього суспільного капіталу до змінної
частини. Тільки американський перепис дає те, що можливе при
сучасних відносинах: суму заробітної плати, виплаченої в кожній
галузі підприємств, і одержаних зисків. Хоч і які підозрілі ці
дані, — бо вони основані тільки на неперевірених повідомленнях
самих промисловців, — проте вони надзвичайно цінні і становлять
усе, що ми маємо про цей предмет. В Европі ми занадто делікатні,
щоб вимагати від наших великих промисловців подібних
викрить. — \emph{Ф. Е.}]

\section{Економія в застосуванні сталого капіталу}

\subsection{Загальні положення}

Збільшення абсолютної додаткової вартості, або здовження
додаткової праці, отже й робочого дня, при незмінній величині
змінного капіталу, тобто при вживанні того самого числа робітників
за ту саму номінально заробітну плату, — при чому байдуже,
чи оплачується надурочний час чи ні, — відносно знижує
вартість сталого капіталу порівняно з вартістю всього капіталу
і змінного капіталу і підвищує цим норму зиску, знов таки
незалежно від зростання й маси додаткової вартості і можливого
підвищення норми додаткової вартості. Розмір основної
частини сталого капіталу, фабричних будівель, машин тощо лишається
той самий, однаково, чи працюють за його допомогою 16,
чи 12 годин. Здовження робочого дня не вимагає ніяких нових
затрат на цю найдорожчу частину сталого капіталу. До цього долучається
\index{iii1}{0093}  %% посилання на сторінку оригінального видання
ще й те, що вартість основного капіталу таким чином репродукується
за коротший ряд періодів обороту, отже, скорочується
час, на який він мусить бути авансований, щоб одержати
певний зиск. Тому здовження робочого дня збільшує зиск навіть
тоді, коли надурочний час оплачується, а до певної міри навіть
тоді, коли він оплачується вище, ніж нормальні робочі години.
Тому постійно зростаюча при сучасній промисловій системі необхідність
збільшення основного капіталу була для ненаситно
жадливих до зиску капіталістів головним стимулом до здовження
робочого дня.\footnote{
„Через те що на всіх фабриках дуже висока сума основного капіталу
вкладена в будівлі і машини, зиск буде тим більший, чим більше число годин,
протягом яких ці машини можуть бути в роботі“ („Rep. of Insp. of Fact., 31.
October 1858“, стор. 8).
}

Інші умови маємо при сталому робочому дні. В цьому випадку
для того, щоб експлуатувати більшу масу праці, треба
або збільшити число робітників, і разом з тим до певної міри
масу основного капіталу, будівель, машин і т. д. (бо ми тут
залишаємо осторонь відрахування з заробітної плати або зниження
заробітної плати нижче її нормальної висоти). Абож, якщо
збільшується інтенсивність праці чи підвищується продуктивність
праці, якщо взагалі виробляється більше відносної додаткової
вартості, то в тих галузях промисловості, які застосовують
сировинний матеріал, зростає маса обігової частини
сталого капіталу, бо за даний період часу переробляється
більше сировинного матеріалу і т. д.; і, подруге, зростає кількість
машин, які приводяться в рух тим самим числом робітників,
отже, і відповідна частина сталого капіталу. Зростання додаткової
вартості супроводиться, отже, зростанням сталого капіталу, зростаюча
експлуатація праці — подорожчанням тих умов виробництва,
за допомогою яких експлуатується праця, тобто більшими
витратами капіталу. Отже, через це норма зиску з одного
боку зменшується, тимчасом як з другого боку вона підвищується.

Цілий ряд поточних затрат лишається майже або цілком
однаковий як при довшому, так і при коротшому робочому дні.
Витрати нагляду менші при 500 робітниках і 18-годинному робочому
дні, ніж при 750 робітниках і 12-годинному робочому дні.
„Витрати ведення фабрики при десятигодинній праці майже однаково
високі, як і при дванадцятигодинній“ („Rep. of Insp. of
Fact., Oct. 1848“, стор. 37). Державні та комунальні податки,
страхування від огню, заробітна плата різних постійних службовців,
зневартнення машин і різні інші затрати фабрики лишаються
незмінними при довгому чи короткому робочому дні;
в міру того, як скорочується виробництво, вони підвищуються
коштом зиску („Rep. of Insp. of Fact., Oct. 1862“, стор. 19).

Період часу, протягом якого репродукується вартість машин
і інших складових частин основного капіталу, на практиці визначається
\index{iii1}{0094}  %% посилання на сторінку оригінального видання
не тим часом, протягом якого вони просто існують,
а загальною тривалістю процесу праці, на протязі якого вони
функціонують і використовуються. Якщо робітники мусять працювати
18 годин замість 12, то це становить за тиждень на три
дні більше, тиждень перетворюється в півтора тижня, два
роки — в три. Отже, якщо надурочний час не оплачується, то
робітники, крім нормального часу додаткової праці, дають задарма
на кожні два тижні третій, на кожні два роки третій.
І таким чином репродукція вартості машин прискорюється на 50\%
і закінчується за \sfrac{2}{3} часу, необхідного при звичайних умовах.

У цьому дослідженні, так само як і в дослідженні коливань
ціни сировинного матеріалу (в розд. VI), ми, щоб уникнути
зайвих ускладнень, виходимо з припущення, що масу і норму
додаткової вартості дано.

Як уже зазначено при розгляді кооперації, поділу праці
і ролі машин, економія в умовах виробництва, яка характеризує
виробництво у великому масштабі, в істотному виникає з того,
що ці умови функціонують як умови суспільної, суспільно-комбінованої
праці, отже, як суспільні умови праці. Вони
споживаються у процесі виробництва спільно, колективним робітником,
замість споживатись у роздрібненій формі масою
незв’язаних між собою робітників або в кращому разі робітниками,
в незначній мірі безпосередньо зв’язаними відносинами співробітництва.
На великій фабриці з одним або двома центральними
двигунами витрати на ці двигуни зростають не в тій самій пропорції,
в якій зростає кількість їх кінських сил, і отже можлива сфера
їх діяння; витрати на передатні механізми зростають не в тій самій
пропорції, в якій зростає маса робочих машин, яким вони передають
рух; самий корпус робочої машини дорожчає не в тій
пропорції, в якій збільшується число знарядь, якими вона діє
як своїми органами, і т. д. Далі, концентрація засобів виробництва
дає заощадження на будівлях усякого роду, не тільки
на власне майстернях, але й на складських приміщеннях і т. д.
Так само стоїть справа з видатками на опалення, освітлення
і т. д. Інші умови виробництва лишаються ті самі, все одно,
багато чи мало людей використовує їх.

Але вся ця економія, яка виникає з концентрації засобів виробництва
та їх масового застосування, передбачає, як істотну
умову, скупчення й спільну діяльність робітників, тобто суспільну
комбінацію праці. Отже, вона виникає з суспільного характеру
праці цілком так само, як додаткова вартість виникає
з додаткової праці кожного окремого робітника, розглядуваного
ізольовано. Навіть постійні поліпшення, які тут можливі й потрібні,
виникають виключно з суспільних дослідів і спостережень,
що їх дає і уможливлює виробництво комбінованого у
великому масштабі колективного робітника.

Те саме стосується і другої великої галузі економії в умовах
виробництва. Ми маємо на увазі зворотне перетворення
\parbreak{}  %% абзац продовжується на наступній сторінці

\input{_0095_0096.tex}
\input{_0097_0098_0099.tex}
\parcont{}  %% абзац починається на попередній сторінці
\index{ii}{0100}  %% посилання на сторінку оригінального видання
і безперервність процесу циркуляції, а тому й процесу репродукції, що
має в собі і процес циркуляції.

Треба згадати, що $Т' — Г'$ може вже відбутись для продуцента $Т$,
хоч $Т$ все ще перебуває на ринку. Коли б сам продуцент захотів тримати
свій власний товар у себе на складах, доки його продасться остаточному
споживачеві, то він мусів би пустити в рух подвійний капітал:
один — як продуцент товару, другий — як купець. Для самого товару,
хоч розглядати його як поодинокий товар, хоч як складову частину
суспільного капіталу, справа зовсім не змінюється від того, чи витрати
на утворення запасу припадають на продуцента товару, чи на ряд купців,
від $А$ до $Z$.

Оскільки товаровий запас є не щось інше, як товарова форма запасу,
що при даному маштабі суспільної продукції, коли б він не існував у
формі товарового запасу, існував би або як продуктивний запас (лятентний
фонд продукції), або як споживний фонд (резерв засобів споживання),
остільки й витрати, що їх потребує зберігання запасу, отже, витрати на
утворення запасу, — тобто вжита для цього зречевлена або жива праця —
є лише витрати зберігання хоч суспільного продукційного фонду,
хоч суспільного фонду споживання. Підвищення вартости товару,
зумовлене ними, розподіляє ці витрати лише pro rata між різними
товарами, бо вони різні для різних сортів товару. Як і раніш,
витрати на утворення запасу лишаються одбавою із суспільного багатства,
хоч вони є умова його існування.

Лише оскільки товаровий запас є умова циркуляції товарів і навіть
форма, що доконечно постала в товаровій циркуляції, оскільки, отже,
цей позірний застій є форма самого руху, цілком так само, як утворення
грошового резерву є умова грошової циркуляції, — лише остільки
він є нормальний. Навпаки, скоро товари, що затрималися в резервуарах
циркуляції, не звільняють місця для наступної хвилі продукції,
скоро, отже, резервуари переповнюються, товаровий запас збільшується
в наслідок застою в циркуляції, цілком так само, як зростають
скарби, коли затримується грошова циркуляція. При цьому
байдуже, чи цей застій постає в амбарах промислових капіталістів, чи
на складах купця. Товаровий запас тоді є вже не умова безперервного
продажу, а наслідок того, що товари не сила продати. Витрати лишаються
ті самі, але що вони тепер випливають виключно з форми, а саме з
доконечности перетворити товари на гроші, та з труднощів у цій метаморфозі,
то вони не входять у вартість товару, а становлять одбаву, втрату
вартости при реалізації вартости. Що нормальна і анормальна форма
запасу не відрізняється щодо форми, і обидві являють застій циркуляції,
то явища можна сплутати, та й самі агенти продукції допускаються
цієї помилки тим легше, що для продуцента процес циркуляції
його капіталу може перебігати, хоч процес циркуляції його товарів,
які перейшли в руки купців, зупинився. Коли більшають розміри
продукції та споживання, то, за інших незмінних обставин, збільшується
й товаровий запас. Він відновлюється й поглинається так само швидко,
\parbreak{}  %% абзац продовжується на наступній сторінці

\input{_0101.tex}
\parcont{}  %% абзац починається на попередній сторінці
\index{iii2}{0102}  %% посилання на сторінку оригінального видання
марнотратникам-вельможам, вони шукали й знаходили поза межами своєї країни
бланковий вексельний кредит, тобто такий кредит, що собі за основу не мав
ніякісінької товарової торговлі; кредит, що його закордонні трасати терпляче
акцептували доти, доки ще надходили римеси, утворені цим вексельовим шахрайством.
За це вони дуже тяжкого лиха зазнали з причини банкрутства такого
банкіра, як Тапер, та інших вельми поважних варшавських банкірів» (І. G. Büsch,
Teoretisch-praktische Darstellung der Handlung etc 3. Auflage. Hamburg 1808. Band
II, p. 232, 233.)

\subsubsection{Користь для церкви від заборони проценту}

«Проценти брати церква забороняла; але не забороняла продавати власність,
щоб зарадити собі в нужді; навіть не забороняла віддавати цю власність
на певний час, аж до оплати боргу, грошовому позикодавцеві, щоб він міг собі
мати в тій власності забезпечення, і також, щоб протягом того часу, поки та
власність перебуває в його руках, міг він користуючися мати винагородження
за визичені гроші. Сама церква або приналежні до неї комуни й ріа corpora\footnote*{
Ріа corpora — дослівно «благочестиві, побожні тіла», тобто вірні, в церковних громадах,
об’єднані. Прим. Ред.
} добували
собі від того значну користь, особливо підчас хрестових походів. Таким
чином значна частина національного доходу опинилася з цієї причини в володінні
так званої «мертвої руки», особливо тому, що єврей не міг лихварювати таким
способом, бо володіння такою нерушною заставою не сила було затаїти... Без
заборони проценту церкви й манастирі ніколи б не мали змоги стати такими
багатими». (1. с., р. 55.)



\index{iii2}{0103}  %% посилання на сторінку оригінального видання
\chapter{Перетворення надзиску на земельну ренту}

\section{Вступ}

Аналіза земельної власности в її різних історичних формах лежить поза
межами цієї праці. Ми спиняємось на ній лише остільки, оскільки частина додаткової
вартости, випродукуваної капіталом, припадає земельному власникові.
Отже, ми припускаємо, що в хліборобстві, цілком так само, як в мануфактурі
панує капіталістичний спосіб продукції, тобто що сільське господарство провадять
капіталісти, які відрізняються від решти капіталістів передусім лише тим
елементом, до якого прикладається їхній капітал та наймана праця, яку цей
капітал пускає в рух. На наш погляд фармер продукує пшеницю і т. ін. так
само, як фабрикант — пряжу або машини. Та передумова, що капіталістичний спосіб
продукції опанував сільське господарство, має в собі й те, що цей спосіб продукції
опановує всі сфери продукції й буржуазного суспільства, що, отже,
є наявні і його цілком розвинуті умови, як от вільна конкуренція капіталів,
змога переносити їх з однієї сфери продукції до іншої, однакова висота пересічного
зиску і т. ін. Та форма земельної власности, що її ми розглядаємо,
становить специфічно історичну її форму, \emph{перетворену} — в наслідок впливу
капіталу та капіталістичного способу продукції — форму або февдальної земельної
власности, або дрібно-селянського хліборобства, що провадиться як ділянка
для прохарчування, хліборобства, що в ньому \emph{володіння} землею для безпосередного
продуцента є одна з умов продукції, а його, того продуцента \emph{власність}
на землю є найвигідніша умова розцвіту \emph{його} способу продукції. Якщо
капіталістичний спосіб продукції взагалі має собі за передумову експропріацію
умов праці в робітників, то в хліборобстві він має собі за передумову
експропріацію землі в сільських робітників та підпорядкування їх капіталістові,
що провадить хліборобство за-для зиску. Отже, для нашої аналізи цілком байдуже,
коли нам заперечуватимуть, нагадуючи, що були або ще й досі є й інші
форми земельної власности та хліборобства. Це заперечення може вразити тільки
тих економістів, що розглядають капіталістичний спосіб продукції в сільському
господарстві та відповідну йому форму земельної власности не як історичні, а як
вічні категорії.

Для нас розгляд новітньої форми земельної власности потрібний тому, що
взагалі справа йде про розгляд тих певних відносин продукції й обміну, що
\parbreak{}  %% абзац продовжується на наступній сторінці


\index{ii}{0104}  %% посилання на сторінку оригінального видання
\chapter{Оборот капіталу}

\section{Час обороту й число оборотів}

Ми бачили: сукупний час циркуляції даного\footnote*{
Термін „сукупний час циркуляції“ тут Маркс вживає в тому самому розумінні,
в якому він далі в цьому ж розділі вживає термін „час обороту“, тимчасом
як взагалі він з цій книзі термін „час циркуляції“ вживає в тому самому
розумінні, що і „час обігу“, тобто в розумінні того часу, що протягом його капітал
перебуває в сфері циркуляції. (Дивись розділ V). \emph{Ред.}
} капіталу дорівнює сумі
часу його обігу та часу його продукції. Це є відтинок часу від моменту
авансування капітальної вартости в певній формі до моменту, коли капітальна
вартість, що процесує, повертається в тій самій формі.

Мета, що визначає капіталістичну продукцію, завжди є зростання
авансованої вартости, чи авансовано цю вартість в її самостійній формі,
тобто в грошовій формі, чи в формі товару, так що його форма вартости
має лише ідеальну самостійність у ціні авансованих товарів.
В обох випадках ця капітальна вартість перебігає протягом свого кругобігу
різні форми існування. Її тотожність з самою собою констатується
в книгах капіталіста або в формі рахункових грошей.

Хоч візьмемо ми форму $Г\dots{} Г'$, хоч форму $П\dots{} П$, обидві форми
значать: 1) що авансована вартість функціонувала як капітальна вартість
і зросла своєю вартістю; 2) що по закінченні процесу вона повернулась
до тієї форми, в якій почала його. Зростання авансованої вартости Г і
разом з тим поворот капіталу до цієї форми (до грошової форми) виразно
помітно в $Г\dots{} Г'$. Але те саме відбувається і в другій формі. Бо
вихідний пункт для П є наявність елементів продукції, товарів даної
вартости. Ця форма має в собі зростання цієї вартости (Т' і $Г'$) і поворот
до первісної форми, бо в другому П авансована вартість
знову має форму елементів продукції, що в ній її первісно авансовано.

Раніше ми бачили: „Якщо продукція має капіталістичну форму, то
і репродукція має ту саму форму. Як процес праці за капіталістичного
способу продукції є лише засіб для процесу зростання вартости, так
\parbreak{}  %% абзац продовжується на наступній сторінці

\parcont{}  %% абзац починається на попередній сторінці
\index{ii}{0105}  %% посилання на сторінку оригінального видання
само й репродукція є лише засіб репродукувати авансовану вартість
як капітал, тобто як вартість, що зростає сама з себе“. (Книга І,
розд. XXI).

Три форми: І) $Г\dots{} Г'$, II) $П\dots{} П$ і III) $Т'\dots{} Т'$ відрізняються між
собою ось чим: в формі II (П\dots{} П) відновлення процесу, процесу репродукції,
виражено як дійсне, а в формі І лише як можливе. Але обидві ці
форми відрізняються від форми III тим, що авансована капітальна вартість
— хоч її авансовано як гроші, хоч в вигляді речових елементів продукції
— становить вихідний пункт, а тому й пункт повороту. В $Г\dots{} Г' п$оворот
є $Г' \deq{} Г \dplus{} г$. Коли процес відновлюється знову в тих самих розмірах,
то Г знову становить вихідний пункт, а г не входить в цей процес і лише
показує нам, що Г зросло своєю вартістю як капітал і тому створило додаткову
вартість г, але відштовхнуло її від себе. В формі $П\dots{} П$ капітальна
вартість, авансована в формі П, елементів продукції, знову таки
становить вихідний пункт. Ця форма має в собі й зростання цієї вартости.
Коли відбувається проста репродукція, то та сама капітальна вартість
в тій самій формі П знову починає свій процес. Коли відбуваєтьсяакумуляція,
то тепер процес починає $П'$ (що величиною вартости дорівнює
$Г' \deq{} Т'$), як збільшена капітальна вартість. Але процес починається знову
авансованою капітальною вартістю в початковій формі, хоч і капітальною
вартістю більшою, ніж раніш. Навпаки, в формі III капітальна вартість
починає процес не як авансована, але як уже виросла, як усе багатство,
що перебуває в формі товарів, і що лише деяка частина його являє
авансовану капітальну вартість. Остання форма важлива для третього
відділу, де рух поодиноких капіталів береться в зв’язку з рухом сукупного
суспільного капіталу. Але, навпаки, з неї не можна користатись,
коли досліджується оборот капіталу, що завжди починається авансуванням
капітальної вартости, чи то у формі грошей, чи то у формі товару, і
який завжди зумовлює, що капітальна вартість, яка чинить оборот, повертається
в тій формі, що в ній її авансовано. З кругобігів І і II
треба триматися першого, коли мають на увазі переважно той вплив, що
його справляє оборот на утворення додаткової вартости; другого — коли
мають на увазі вплив обороту на утворення продукту.

Як мало економісти відрізняли різні форми кругобігів, так само мало
вони розглядали ці різні форми кругобігів відокремлено щодо обороту
капіталу. Звичайно береться форму $Г\dots{} Г'$, бо вона панує над поодиноким
капіталістом і служить йому в його розрахунках навіть тоді, коли
гроші становлять вихідний пункт лише в формі рахункових грошей. Інші
беруть за вихідний пункт витрати в формі елементів продукції, поки
не настане поворот, при цьому про форму повороту — чи буде цей поворот
в товарі, чи в грошах — у них немає й мови. Напр.: „Економічний
цикл\dots{} тобто ввесь перебіг продукції від часу, коли зроблено витрати,
до часу, коли настає поворот. В сільському господарстві час засіву є
початок економічного циклу, а жнива — закінчення“. (Economic Cycle\dots{}
the whole course of production, from the time that outlays are made till
returns are received. In agriculture seedtime is its commencement, and
\parbreak{}  %% абзац продовжується на наступній сторінці

\parcont{}  %% абзац починається на попередній сторінці
\index{iii1}{0106}  %% посилання на сторінку оригінального видання
сухот, в Блекберні й Скіптоні — 167, в Конглетоні й Бредфорді —
168, в Лейстері — 171, в Ліку — 182, в Маккльсфільді — 184,
в Больтоні — 190, в Ноттінгемі — 192, в Ронделі — 193, в Дербі —
198, в Сальфорді і Аштоні-на-Лайні — 203, в Лідсі — 218, в Престоні — 220 і в Манчестері — 263
(стор. 24). Нижченаведена таблиця дає ще разючіший приклад. Вона наводить випадки смерті
внаслідок хвороб легенів окремо для обох статей між 15 і
25 роками, обчислені на кожні 10 0000 мешканців. Вибрано такі
округи, де тільки жінки зайняті в промисловості, провадженій
у закритих приміщеннях, а чоловіки — в усяких можливих галузях праці.

\begin{table}[ht]
  \small
  \begin{tabular}{c c c c}
    \toprule
    Округи &
    Головна промисловість &
    \multicolumn{2}{c}{\makecell{Число випадків смерті від \\легеневих захворувань між 15 і 25\\
роками на 100 000 жителів}}\\
    \cmidrule(rl){3-4}
    & & Чоловіки & Жінки \\
    \midrule
Berkhampstead    & \makecell{Плетіння з соломи,\\ працюють жінки} & 219 & 578 \\
Leighton Buzzard & \makecell{Плетіння з соломи,\\ працюють жінки} & 309 & 554 \\
Newport Pagnell  & \makecell{Плетіння мережива\\ жінками}         & 301 & 617 \\
Towcester        & \makecell{Плетіння мережива\\ жінками}                         & 239 & 577 \\
Yeovil           & \makecell{Виробництво рукавичок,\\ здебільшого працюють жінки} & 280 & 409 \\
Leek             & \makecell{Шовкова промисловість,\\ переважно жінки}            & 437 & 856 \\
Congleton        & \makecell{Шовкова промисловість,\\ переважно жінки}            & 566 & 790 \\
Macclesfield     & \makecell{Шовкова промисловість,\\ переважно жінки}            & 593 & 890 \\
\makecell{Здорова сільська\\ місцевість} &   Землеробство                         & 331 & 333 \\
  \end{tabular}
\end{table}

В округах шовкової промисловості, де участь чоловіків
у фабричній праці більша, більша також і смертність серед них.
Норма смертності від сухот і т. п. як чоловіків, так і жінок
виявляє тут, як сказано в звіті, „обурливі (atrocious) санітарні
умови, за яких провадиться значна частина нашої шовкової
промисловості“. І це, якраз, та сама шовкова промисловість,
фабриканти якої, посилаючись на винятково сприятливі санітарні умови свого виробництва, вимагали і
почасти добилися
винятково довгого робочого часу для дітей, молодших 13 років.
(книга І, розд. VIII, 6, стор. 306\footnote*{Стор. 214 рос. вид. 1935 р. Ред. укр. перекладу.}).

„Без сумніву, жодна з досліджених досі галузей промисловості
не дає сумнішої картини, ніж та, що її дає доктор Сміт
щодо кравецтва\dots{} Майстерні, каже він, дуже неоднакові щодо

\parbreak{}  %% абзац продовжується на наступній сторінці

\parcont{}  %% абзац починається на попередній сторінці
\index{iii1}{0107}  %% посилання на сторінку оригінального видання
санітарного стану; але майже всі вони переповнені, погано провітрюються і в високій мірі
несприятливі для здоров’я\dots{} В таких
кімнатах, крім усього, неодмінно жарко; а коли запалюють газ,
як це роблять удень під час туману або зимою вечорами, то температура підвищується до 80 і навіть до
90 градусів (за Фаренгейтом = 27—33° Цельсія) і викликає надзвичайне пітніння робітників
і згущення пари на шибках, так що вода безупинно стікає або
крапає з вікна в стелі, і робітники змушені держати відчиненими
кілька вікон, хоча вони при цьому неминуче простуджуються. —
Становище в 16 найзначніших майстернях лондонського Вестенду
він описує так: найбільший кубічний простір, який припадає
в цих погано провітрюваних кімнатах на одного робітника, становить 270 кубічних футів; найменший —
105 футів, пересічно —
всього тільки 156 футів на людину. В одній майстерні, яка обведена з усіх боків галереєю і має
освітлення тільки згори,
занято від 92 до 100 осіб; горить багато газових ріжків; клозети
збудовані безпосередньо коло майстерні, і на кожну людину
припадає не більше, як 150 кубічних футів простору. В другій
майстерні, в освітленому згори дворі, яку можна назвати тільки
собачою конурою і яку можна провітрювати тільки через маленьке вікно в даху, працює 5 чи 6 осіб, при
чому на кожну
з них припадає 112 кубічних футів“. І „в цих жахливих (atrocious) майстернях, які описує доктор
Сміт, кравці працюють
звичайно 12—13 годин на день, а іноді праця триває 14—16 годин“ (стор. 25, 26, 28).

\begin{table}[ht]
  \footnotesize
  \begin{tabular}{ c c c c c}
  \toprule
Число занятих людей & \makecell{Галузь промисловості\\ і місцевість} & \multicolumn{3}{c}{\makecell{Норма смертності на 100 000 осіб\\ віком}}\\
\cmidrule(rl){3-5}
& & 25\textendash{}35 р. & 35\textendash{}45 р. & 45\textendash{}55 р.\\
\midrule

958265                          & Землеробство, Англія та Уельс & 743 & 805 & 1145\\
\makecell{22 301 чоловіків і\\ 12 377 жінок} & Кравці, Лондон                & 958 & 1262 & 2093\\
13 803                          & Складачі й друкарі, Лондон    & 894 & 1747 & 2367\\

  \end{tabular}
\end{table}
(стор. 30). Треба відзначити — і це дійсно відзначено складачем
цього звіту, завідувачем медичного відділу, Джоном Сімоном, —
що для віку в 25—35 років смертність кравців, складачів і друкарів Лондона показана применшеною, бо
в обох цих галузях
промисловості лондонські майстри одержують з села велике
число молодих людей (мабуть, до 30 років), що працюють як
учні і „improvers“, тобто для дальшого удосконалення. Вони
збільшують число занятих осіб, на яке треба обчисляти норми
смертності промислового населення Лондона; але вони не збільшують в такій самій мірі число смертей у
Лондоні, бо їх перебування в Лондоні тільки тимчасове; коли вони захворіють на протязі цього часу,
то вертаються додому на село, і смерть
їх, якщо вони умирають, реєструється там. Ця обставина ще
в більшій мірі стосується до молодшого віку, і в наслідок цього
\parbreak{}  %% абзац продовжується на наступній сторінці

\input{_0108.tex}
\input{_0109.tex}
\parcont{}  %% абзац починається на попередній сторінці
\index{i}{0110}  %% посилання на сторінку оригінального видання
товарів, покупець або продавець, а саме, в обох рядах оборудок
виступаю проти одного контраґента лише як покупець, а
проти другого лише як продавець, проти одного — лише як
гроші, проти другого — лише як товар; ані проти одного, ані
проти другого я не виступаю як капітал або як капіталіст або
як представник чогось такого, що було б більше, ніж гроші або
товар, або могло б заподіяти інший вплив, крім того, що його
можуть справляти гроші або товар. Для мене купівля в \emph{А} і продаж
\emph{В} становлять послідовний ряд. Але зв’язок поміж цими
обома актами існує лише для мене. А немає жодного діла до моєї
оборудки з \emph{В}, а \emph{В} — до моєї оборудки з \emph{А}. Коли б я захотів
пояснити їм особливу заслугу, яку я маю перед ними, обертаючи
послідовність ряду, то вони довели б мені, що я помиляюсь щодо
самого порядку послідовности, і що вся операція почалася не
від купівлі й кінчається не продажем, а, навпаки, почалася від
продажу й завершується купівлею. Справді, мій перший акт,
купівля, з погляду \emph{А} є продаж, а мій другий акт, продаж, з
погляду \emph{В} — купівля. Не задовольнившися цим, \emph{А} й \emph{В} заявляють,
що цілий цей порядок послідовности був зайвий фокус-покус.
\emph{А} продасть товар безпосередньо \emph{В}, а \emph{В} купить його безпосередньо
в \emph{А}. Разом з тим вся операція стискується в однобічний
акт звичайної товарової циркуляції, — просто продаж з погляду
\emph{А} і просто купівлю з погляду \emph{В}. Отже, обернувши порядок послідовности,
ми не вийшли поза сферу простої товарової циркуляції,
а тому ми мусимо розглянути, чи допускає вона з своєї природи
зростання вартостей, що входять у неї, тобто чи допускає вона
творення додаткової вартости.

Візьмімо процес циркуляції у формі, в якій він виявляється
як простий обмін товарів. Це завжди буває тоді, коли обидва
посідачі товарів купують один в одного товари і в термін платежу
вирівнюють балянс своїх взаємних грошових зобов’язань. Гроші
служать тут за рахункові гроші, щоб виразити вартості товарів
у їхніх цінах, але вони не виступають проти самих товарів речово.
Ясна річ, що, оскільки йдеться про споживну вартість,
виграти можуть обидва обмінювані. Обидва відчужують товари,
які є некорисні для них як споживні вартості, і одержують товари,
що їх вони потребують для споживання. І користь од цього може
бути не лише ця одна. \emph{А}, що продає вино й купує збіжжя, продукує,
може, більше вина, ніж його зміг би випродукувати за той
самий робочий час рільник \emph{В}, а рільник \emph{В} за той самий робочий
час продукує більше збіжжя, ніж його зміг би випродукувати
винар \emph{А}. Отже, \emph{А} дістає за таку саму мінову вартість більше
збіжжя, а \emph{В} — більше вина, ніж дістав би відповідно кожний
із них без обміну, коли б вони мусили продукувати сами для себе
вино і збіжжя. Таким чином щодо споживної вартости можна
сказати, що «обмін є оборудка, в якій виграють обидві сторони»\footnote{
«Обмін є дивна оборудка, в якій виграють обидва контраґенти —
завжди (!)» («L’échange est une transaction admirable, dans la quelle les
deux contractants gagnent — toujours (!)»). (\emph{Destutt de Tracy}: «Traité de
la Volonté et de ses effets», Paris 1826, p. 68). Ta сама книга появилася пізніше
під назвою «Traité d’Economie Politique».
}.
\index{i}{0111}  %% посилання на сторінку оригінального видання
Інша справа з міновою вартістю. «Людина, що має багато вина,
а не має збіжжя, веде торг з людиною, що в неї багато збіжжя,
але немає вина, і вони обмінюють пшеницю вартістю в 50 на
вино вартістю в 50. Цей обмін не є збільшення мінової вартости
ні для одного, ані для другого, бо вже перед обміном кожний
з них мав вартість, рівну тій, що її він здобув собі за допомогою
цієї операції»\footnote{
\emph{Mercier de la Rivière}: «L’Ordre naturel et essentiel», Physiocrates, éd.
Daire, IІ.~Partie, p. 544.
}. Справа зовсім не змінюється, коли між товарами
виступають гроші як засіб циркуляції і акт купівлі почуттєво
відокремлюється від акту продажу\footnote{
«Само по собі цілком байдуже, чи є одна з цих двох вартостей
гроші, чи обидві вони є звичайні товари» («Que l’une de ces deux valeurs
soit argent, ou qu’elles soient toutes deux marchandises usuelles, rien de
plus indifférent en soi»). (Mercier de la Rivière: «L’Ordre naturel et essentiel».
Physiocrates, éd. Daire, II.~Partie, p. 543).
}. Вартість товарів є виражена
в їхніх цінах раніш, ніж вони вступають до циркуляції, отже,
вона є передумова циркуляції, а не її результат\footnote{
«Не контраґенти визначають вартість, її визначено ще до оборудки»
(«Ce ne sont pas les contractants, qui prononcent sur la valeur; elle est
décidée avant la convention»). (Le Trosne: «De l’Intérêt Social», Physiocrates,
éd. Daire, Paris 1846, p. 906).
}.

Розглядаючи справу абстрактно, тобто залишаючи осторонь
обставини, які не випливають з іманентних законів простої товарової
циркуляції, ми побачимо, що, крім заміни однієї споживної
вартости на іншу, в ній відбувається лише метаморфоза, проста
зміна форми товару. Та сама вартість, тобто та сама кількість
упредметненої суспільної праці, лишається в руках того самого
посідача товарів спочатку в формі товару, потім у формі грошей,
на які товар перетворився, нарешті, у формі товару, на який
знову перетворилися ці гроші. Ця зміна форми не містить у собі
жодної зміни величини вартости. Зміна, якої зазнає в цьому процесі
сама вартість товару, обмежується на зміні її грошової
форми. Спочатку вона існує як ціна подаваного на продаж товару,
потім як грошова сума, що була вже однак виражена в ціні, і,
нарешті, як ціна еквівалентного товару. Ця зміна форм сама по
собі так само мало містить у собі зміну величини вартости, як
ось розмін п’ятифунтової банкноти на соверени, півсоверени й
шилінґи. Отже, оскільки циркуляція товару зумовлює лише
зміну форми його вартости, вона зумовлює, — якщо явище відбувається
в чистій формі, — обмін еквівалентів. Тому навіть вульґарна
політична економія, хоч і як мало вона тямить, що таке
вартість, кожного разу, коли вона на свій штиб хоче розглянути
явище в його чистій формі, припускає, що попит і подання урівноважуються,
тобто, що вплив їхній взагалі припиняється. Отже,
коли щодо споживної вартости обидва контраґенти можуть виграти,
то на міновій вартості вони не можуть обидва виграти. Навпаки,
\parbreak{}  %% абзац продовжується на наступній сторінці

\parcont{}  %% абзац починається на попередній сторінці
\index{ii}{0112}  %% посилання на сторінку оригінального видання
відбирає їм його в другому випадку. Однак та обставина, що засоби праці льокально прикріплені,
пустили своє коріння в землю, надає цій частині основного капіталу особливої ролі в економії націй.
Їх не можна відіслати за кордон, вони не можуть циркулювати як товари на світовому ринку. Титули
власности на цей основний капітал можуть змінюватись, їх можна купувати й продавати, і остільки вони
можуть ідеально
циркулювати. Ці титули власности можуть навіть циркулювати на закордонних ринках, напр., в формі
акцій. Але від зміни осіб, що є власники такого виду основного капіталу, не змінюється відношення
між нерухомою, матеріяльно фіксованою частиною багатства даної країни і рухомою частиною того таки
багатства \footnote{До цього місця рукопис IV.~Відси рукопис II.~Ф.~Е.}).

Своєрідна циркуляція основного капіталу зумовлює своєрідний оборот. Та частина вартости, що її
втрачається в її натуральній формі в наслідок зношування, циркулює, як частина вартости продукту.
Продукт через свою циркуляцію перетворюється з товару на гроші, отже, на гроші перетворюється й та
частина вартости засобів праці, що її продукт несе в циркуляцію, а саме: ця частина вартости падає
краплями як гроші з процесу циркуляції, в тій самій пропорції, що в ній даний засіб праці перестає
бути носієм вартости в продукційному процесі. Отже, вартість цього засобу праці набирає тепер
двоїстого існування. Частина її лишається зв’язана з його споживною або натуральною формою, належною
продукційному процесові, а друга частина відокремлюється від неї як гроші. В перебігу свого
функціонування та частина вартости засобів праці, що існує в натуральній формі, постійно меншає,
тимчасом як перетворена на гроші частина вартости постійно більшає, поки, нарешті, засоби праці
одживуть свій вік, і вся їхня вартість, відокремившись від мертвого тіла, перетвориться на гроші.
Тут виявляється своєрідність в обороті цього елемента продуктивного капіталу. Його вартість
перетворюється на гроші рівнобіжно з тим, як на грошову лялечку перетворюється той товар, що є носій
його вартости. Але його зворотне перетворення з грошової форми на споживну форму відділяється від
зворотного перетворення товару на інші елементи продукції цього товару і визначається періодом його
власної репродукції, тобто часом, що протягом його засоби праці одживають свій вік, і треба їх
замінити на нові екземпляри такого самого роду. Коли час функціонування якоїсь машини, напр.,
вартістю в \num{10.000}\pound{ ф. стерл.}, дорівнює, припустімо, 10 рокам, то час обороту вартости, первісно
авансованої на неї, дорівнює 10 рокам. Поки не мине цей час, її не треба поновлювати, і вона
функціонує далі в своїй натуральній формі. Тимчасом її вартість частинами циркулює як частина
вартости товарів, що до їх безперервної продукції вона придається, — і таким чином поступінно
перетворюється на гроші, поки, нарешті, по десятьох роках, вона цілком перетвориться на гроші, а з
грошей знову на машину, вивершуючи, отже, свій оборот. До цього моменту
\parbreak{}  %% абзац продовжується на наступній сторінці

\input{_0113.tex}
\input{_0114.tex}
\parcont{}  %% абзац починається на попередній сторінці
\index{ii}{0115}  %% посилання на сторінку оригінального видання
й поновлювати зворотною купівлею, зворотним перетворенням з грошової форми на елементи продукції.
Одним заходом їх вилучається з ринку меншими масами, ніж елементи основного капіталу, але тим
частіше доводиться їх вилучати з ринку, а тому авансування витраченого на них капіталу поновлюється
через коротші періоди. Це постійне поновлення упосереднюється постійним збутом того продукту, що в
ньому циркулює вся їхня вартість. Нарешті, вони безупинно пророблюють увесь кругобіг метаморфоз не
лише своєю вартістю, але й у своїй речовій формі; з товару вони постійно перетворюються знову на
елементи продукції цього самого товару.

Разом із своєю власною вартістю робоча сила постійно долучає до продукту додаткову вартість,
втілення неоплаченої праці. Отже, готовий продукт так само подає її постійно в циркуляцію, і вона
разом з ним перетворюється на гроші так само, як і інші елементи вартости продукту. Однак, тут, де
йдеться насамперед про оборот капітальної вартости, а не додаткової вартости, що обертається разом з
нею, — тут ми лишаємо це покищо осторонь.

З наведеного вище випливає ось що:

1) Визначеності форми основного й поточного капіталу походять лише з ріжниці в обороті капітальної
вартости, що функціонує в процесі продукції, або продуктивного капіталу. Ця ріжниця в обороті
походить і собі з ріжниці в способі, що ним різні складові частини продуктивного капіталу переносять
свою вартість на продукт, а не з їхньої різної участи в утворенні вартости продукту або не з
характеристичної ролі їх у процесі зростання вартости. Нарешті, ріжниця в передачі вартости
продуктові, — а тому й різні способи, що ними ця вартість вводиться через продукт у циркуляцію і в
наслідок його метаморфоз поновлюється в своїй первісній натуральній формі, — ця ріжниця походить з
відмінности тих речових форм, що в них існує продуктивний капітал, і що з них одна частина під час
утворення окремого продукту споживається цілком, а другу зужитковується лише поступінно. Отже, лише
продуктивний капітал може розподілятись на основний і поточний. Навпаки, цієї протилежности не існує
для обох інших способів буття промислового капіталу, отже, ні для товарового капіталу, ні для
грошового капіталу; не існує її також як і протилежности цих обох форм проти продуктивного капіталу.
Вона існує лише для продуктивного капіталу і в межах його. Грошовий капітал і товаровий капітал
можуть скільки завгодно функціонувати як капітал і можуть хоч як швидко циркулювати, але зробитись
поточним капіталом протилежно до основного вони можуть лише тоді, коли перетворяться на поточні
складові частини продуктивного капіталу. Але через те, що ці обидві форми капіталу перебувають у
сфері циркуляції, то, як ми побачимо, економія від часів А.~Сміса не могла стриматися від спокуси
сплутати їх з поточною частиною продуктивного капіталу, об’єднуючи їх в категорію обіговий капітал.
А справді грошовий капітал і товаровий капітал є капітал циркуляції протилежно до продуктивного, але
не обіговий капітал протилежно до основного.

\input{_0116_0117.tex}
\input{_0118.tex}
\input{_0119.tex}
\input{_0120_0121.tex}
\input{_0122.tex}
\input{_0123.tex}
\parcont{}  %% абзац починається на попередній сторінці
\index{iii1}{0124}  %% посилання на сторінку оригінального видання
перетворена в гроші; друга частина існує як гроші в будьякій
формі і мусить бути знову перетворена в умови виробництва;
нарешті, третя частина перебуває у сфері виробництва, почасти
у первісній формі засобів виробництва, сировинних матеріалів,
допоміжних матеріалів, куплених на ринку півфабрикатів,
машин та іншого основного капіталу, почасти як продукт, який
ще тільки виготовляється. Як діє тут підвищення вартості або
зниження вартості, це в великій мірі залежить від того відношення,
в якому стоять одні до одних ці складові частини. Щоб
спростити питання, залишмо спочатку осторонь весь основний
капітал і розгляньмо тільки ту частину сталого капіталу, яка
складається з сировинних матеріалів, допоміжних матеріалів,
півфабрикатів і товарів, які ще тільки виготовляються або вже
є готові на ринку.

Якщо підвищується ціна сировинного матеріалу, наприклад,
бавовни, то підвищується й ціна бавовняних товарів — півфабрикатів,
як от пряжа, і готових товарів, як от тканини і т. д., —
сфабрикованих з дешевшої бавовни; так само підвищується
і вартість як ще непереробленої бавовни, яка є на складі, так
і тієї, що перебуває ще в процесі оброблення. Ця остання,
через те що вона в наслідок зворотного впливу стає виразом
більшої кількості робочого часу, додає до продукту, в який
вона входить як складова частина, більшу вартість, ніж та, яку
вона первісно мала сама і яку капіталіст заплатив за неї.

Отже, якщо підвищення цін сировинного матеріалу супроводиться
наявністю на ринку значної маси готового товару, —
все одно, на якому ступені готовості, — то підвищується вартість
цього товару і разом з тим відбувається підвищення вартості
наявного капіталу. Те саме стосується і до запасів сировинного
матеріалу і т. д., які перебувають в руках виробників.
Це підвищення вартості може відшкодувати або й більш ніж
відшкодувати окремих капіталістів або навіть і цілу окрему
сферу виробництва капіталу за падіння норми зиску, яке виникає
з підвищення ціни сировинного матеріалу. Не входячи тут
у деталі впливу конкуренції, можна, однак, ради повноти відзначити,
що 1) коли запаси сировинного матеріалу, які перебувають
на складах, значні, то вони протидіють підвищенню цін,
що виникає в місці виробництва сировинного матеріалу; 2) коли
півфабрикати або готові товари, які перебувають на ринку, дуже
тиснуть на ринок, то вони заважають ціні готових товарів і півфабрикатів
зростати відповідно до ціни їх сировинного матеріалу.

Зворотне маємо при падінні цін сировинного матеріалу, яке
при інших однакових умовах підвищує норму зиску. Товари, які
перебувають на ринку, речі, які ще тільки виготовляються,
запаси сировинного матеріалу знецінюються і цим самим протидіють
одночасному підвищенню норми зиску.

Чим менші запаси, які перебувають у сфері виробництва
і на ринку, наприклад наприкінці операційного року, коли сировинний
\index{iii1}{0125}  %% посилання на сторінку оригінального видання
матеріал знову постачається великими масами, як от у землеробстві після жнив, — тим
виразніше виступає вплив зміни цін сировинного матеріалу.

В усьому нашому дослідженні ми виходимо з того припущення, що підвищення або зниження цін є вираз
дійсних коливань вартості. Але через те що тут мова йде про той вплив, який ці коливання цін
справляють на норму зиску, то в дійсності не має значення, яка є причина цих коливань; отже,
розвинуте тут має силу також і тоді, коли ціни підвищуються і падають не в наслідок коливань
вартості, а в наслідок діяння системи кредиту, конкуренції і т. д.

Через те що норма зиску дорівнює відношенню надлишку вартості продукту до вартості всього
авансованого капіталу, то підвищення норми зиску, що походить із зниження вартості авансованого
капіталу, може бути сполучене з втратою капітальної вартості; так само зниження норми зиску, що
походить з підвищення вартості авансованого капіталу, може бути сполучене з виграшем.

Щодо другої частини сталого капіталу, машин і взагалі основного капіталу, то підвищення вартості,
які тут відбуваються і стосуються саме до будівель, землі і т. д., не можуть бути розглянуті до
викладу вчення про земельну ренту і тому, вони не належать сюди. Але для зниження вартості цієї
частини капіталу загальне значення мають:

1. Постійні поліпшення, які позбавляють наявні машини, фабричне устаткування і т. д. частини їх
споживної вартості,
а тому і їх вартості. Цей процес діє з особливою силою в перший період введення нових машин, раніше
ніж вони досягають певної міри зрілості, і коли вони через це постійно стають застарілими раніше,
ніж встигають репродукувати свою вартість. Це одна з причин звичайного в такі епохи безмірного
здовження робочого часу, праці вдень і вночі почережно змінами, для того, щоб протягом коротшого
часу репродукувати вартість машин, не відраховуючи при цьому занадто багато на їх зношування. Якщо
ж, навпаки, короткий період діяльності
машин (короткий строк їх життя в зв’язку з можливими поліпшеннями) не буде таким способом
скомпенсовано, то внаслідок їх морального зношування вони переносять на продукт занадто велику
частину своєї вартості, так що не можуть конкурувати навіть з ручною працею\footnote{Приклади, між іншим,
у Беббеджа. Звичайний засіб - зниження заробітної плати - застосовується і тут, і таким чином це постйно
знецінення діє цілком інакше, ніж це уявляє собі в своєму гармонійному мозку пан Кері.
}.

Якщо машини, устаткування будівель, взагалі основний капітал досяг певної зрілості, так що протягом
довшого часу
він, принаймні в своїй основній конструкції, лишається незмінним, то подібне ж зниження вартості
настає в наслідок поліпшень
\index{iii1}{0126}  %% посилання на сторінку оригінального видання
у методах репродукції цього основного капіталу. Вартість
машин і т. д. знижується тепер не тому, що вони швидко
витісняються або до певної міри знецінюються новими продуктивнішими
машинами і т. д., а тому, що вони тепер можуть
бути дешевше репродуковані. Це одна з причин, чому великі підприємства
часто процвітають тільки в других руках, після того
як збанкрутує перший власник, а другий, що дешево купив
підприємство, таким чином уже з самого початку починає своє
виробництво з меншими витратами капіталу.

В землеробстві особливо впадає в очі, що ті самі причини,
які підвищують або знижують ціну продукту, підвищують або
знижують також і вартість капіталу, бо цей останній у значній
частині сам складається з цього продукту — хліба, худоби і т. ін.
(Рікардо).

\pfbreak

Тепер треба було б згадати ще про змінний капітал.

Якщо вартість робочої сили підвищується внаслідок підвищення
вартості потрібних для її репродукції засобів існування,
або, навпаки, знижується в наслідок зниження вартості цих засобів
існування, — а підвищення вартості і зниження вартості
змінного капіталу не виражає нічого іншого, крім цих обох випадків, — то при незмінній довжині
робочого дня цьому підвищенню вартості відповідає падіння додаткової вартості, а цьому
зниженню вартості — зростання додаткової вартості. Але в той
самий час з цим можуть бути зв’язані й інші обставини — звільнення і зв’язування капіталу — які не
були ще досліджені і які
треба тепер коротко розглянути.

Якщо заробітна плата знижується внаслідок падіння вартості робочої сили (з чим може бути зв’язане
навіть підвищення
реальної ціни праці), то таким чином звільняється частина капіталу, яка досі витрачалась на
заробітну плату. Відбувається
звільнення змінного капіталу. На нововкладуваний капітал це
справляє тільки той вплив, що він працює з підвищеною нормою додаткової вартості. Та сама кількість
праці приводиться
в рух за допомогою меншої кількості грошей, ніж раніше, і таким чином неоплачена частина праці
збільшується коштом
оплаченої. Але для капіталу, який був вкладений уже раніше,
не тільки підвищується норма додаткової вартості, але, крім
того, звільняється частина капіталу, яка досі витрачалась на
заробітну плату. Досі вона була зв’язана і становила постійну
частину, яка відділялась від виручки за продукт і мусила витрачатись на заробітну плату,
функціонувати як змінний капітал,
якщо підприємство мало й далі провадитися в попередніх розмірах. Тепер ця частина стає вільною, і
може, отже, бути використана як нове капіталовкладення, чи для розширення того самого підприємства,
чи для функціонування в іншій сфері
виробництва.

\parcont{}  %% абзац починається на попередній сторінці
\index{ii}{0127}  %% посилання на сторінку оригінального видання
мости, тунелі, віадуки тощо, являє приклад того, що можна назвати
віковим зношуванням. А швидше й помітніше зневартнення, відшкодуване
протягом коротких переміжків часу ремонтом і заміщенням, є подібне
до періодичних неправильностей. У витрати на річний ремонт заводиться
й полагодження тієї випадкової шкоди, що її зазнають час від часу
зовнішні частини навіть довготриваліших споруд; але й незалежно від
такого ремонту, час не минає для них безслідно, і хоч як далекий той момент,
коли стан цих будов потребуватиме перебудувати їх наново, а все ж
мусить він надійти. В усякому разі щодо фінансової та економічної сторони
цей момент може бути дуже віддалений, щоб його брати на увагу
в практичних обчисленнях“ (Lardner, 1. c., 38, 39).

Це має силу до всіх таких споруд вікової тривалости, що в них,
отже, не доводиться поступінно, рівнобіжно з їхнім зношенням, заміщувати
авансований на них капітал, а доводиться переносити на ціну продукту
лише щорічні пересічні витрати на підтримання і ремонт.

Хоча — як ми бачили — більшість грошей, які щороку або навіть
через коротший час повертаються на заміщення зношуваного основного
капіталу, знову перетворюються на натуральну форму цього капіталу,
проте, кожному поодинокому капіталістові потрібен фонд амортизації для тієї
частини основного капіталу, що для неї лише по багатьох роках надходить час
репродукції, і її треба тоді цілком заміщувати. Значна складова частина основного
капіталу вже в наслідок своїх властивостей виключає часткову репродукцію.
Крім того, там, де частинна репродукція відбувається таким способом, що
через короткі переміжки до зневартненого складу додається нозий, то, щоб це
заміщення було можливе, потрібне попереднє грошове нагромадження
в більших або менших розмірах, залежно від специфічного характеру
даної галузі продукції. Для цього досить не якої завгодно суми грошей,
а грошової суми певних розмірів.

Коли ми розглянемо що справу, припускаючи лише просту грошову
циркуляцію, лишаючи цілком осторонь кредитову систему, що про неї
мова буде далі, то механізм руху такий: в першій книзі (розділ II, 3 а)
показано, що коли одна частина наявних у суспільстві грошей завжди
лежить без діла як скарб, а друга функціонує як засіб циркуляції, зглядно
як безпосередній резервний фонд для грошей, що безпосередньо циркулюють,
то постійно змінюється пропорція, що в ній уся маса грошей
розподіляється на скарб і на засоби циркуляції. В нашому прикладі
гроші — що їх досить великий капіталіст повинен нагромадити як скарб
чималих розмірів, — підчас закупу основного капіталу разом пускається
в циркуляцію. Потім вони знову сами собою розпадаються в суспільстві
на засоби циркуляції та скарб. За допомогою амортизаційного фонду,
куди, як до свого вихідного пункту, повертається вартість основного капіталу
в міру його зношування, частина грошей, що циркулюють, знову
утворює скарб — на більший або менший час — в руках того самого капіталіста,
що від нього підчас закупу основного капіталу віддалився його
скарб, перетворившись на засіб циркуляції. Отже, ми маємо повсякчас
змінний розподіл наявного в суспільстві скарбу, що по черзі функціонує
\parbreak{}  %% абзац продовжується на наступній сторінці

\parcont{}  %% абзац починається на попередній сторінці
\index{iii1}{0128}  %% посилання на сторінку оригінального видання
400 фунтів стерлінгів. Далі, через те що сталий капітал вартістю в 2000 фунтів стерлінгів потребує
для свого функціонування 500 робітників, то 400 робітників можуть привести в рух тільки сталий
капітал вартістю в 1600 фунтів стерлінгів. Отже,
для того, щоб виробництво і далі провадилося в попередніх
розмірах і щоб \sfrac{1}{5} машин не стояла без діла, змінний капітал
мусить бути підвищений на 100 фунтів стерлінгів, щоб, як і раніш, вживати 500 робітників; а цього
можна досягти тільки за
допомогою того, що вільний досі капітал зв’язується, при чому
та частина нагромадження, яка повинна була б служити для розширення виробництва, тепер служить
тільки для поповнення, або ж
до попереднього капіталу додається та частина, яка призначена
була для витрачання як дохід. Із збільшеною на 100 фунтів
стерлінгів витратою змінного капіталу тепер виробляється на
100 фунтів стерлінгів менше додаткової вартості. Щоб привести
в рух те саме число робітників, потрібно більше капіталу, і разом
з тим зменшується додаткова вартість, яку дає кожний окремий робітник.

Вигоди, які випливають із звільнення, і втрати, які випливають із зв’язування змінного капіталу,
існують тільки для капіталу, який уже вкладений і який, отже, репродукується при даних відношеннях.
Для нововкладуваного капіталу вигоди на
одному боці, втрати на другому зводяться до підвищення або
зниження норми додаткової вартості і відповідної, хоч і зовсім
не пропорціональної, зміни норми зиску.

\pfbreak

Щойно досліджене звільнення і зв’язування змінного капіталу є наслідок зниження вартості або
підвищення вартості
елементів змінного капіталу, тобто витрат репродукції робочої
сили. Але змінний капітал може звільнятися й тоді, коли внаслідок розвитку продуктивної сили, при
незмінній нормі заробітної плати, потрібно менше робітників для того, щоб привести
в рух ту саму масу сталого капіталу. Так само, навпаки, зв’язування додаткового змінного капіталу
може мати місце, якщо
в наслідок зниження продуктивної сили праці потрібно більше
робітників для тієї самої маси сталого капіталу. Якщо ж, з другого боку, частина капіталу, який
раніш застосовувався як змінний капітал, застосовується тепер у формі сталого капіталу, отже, якщо
відбувається тільки зміна розподілу між складовими
частинами того самого капіталу, то, хоч це і справляє вплив
на норму додаткової вартості й зиску, але не належить до розглядуваної тут рубрики зв’язування і
звільнення капіталу.

Сталий капітал, як ми вже бачили, також може зв’язуватись
або звільнятись в наслідок підвищення вартості або зниження
вартості тих елементів, з яких він складається. Залишаючи це
осторонь, зв’язування його можливе (без перетворення будь-якої
частини змінного капіталу в сталий) тільки тоді, коли збільшується
\index{iii1}{0129}  %% посилання на сторінку оригінального видання
продуктивна сила праці, отже, коли та сама маса праці
створює більше продукту і тому приводить в рух більше сталого капіталу. Те саме при певних
обставинах може мати місце
тоді, коли продуктивна сила зменшується, як, наприклад, у землеробстві, так що та сама кількість
праці потребує для створення
того самого продукту більше засобів виробництва, наприклад,
більше насіння або добрива, дренування і т. д. Без знецінення
сталий капітал може звільнятись тоді, коли в наслідок удосконалень, застосовування сил природи і т.
ін. сталий капітал меншої вартості стає спроможним технічно виконувати ту саму службу, яку раніше
виконував капітал вищої вартості.

У книзі II ми бачили, що після того як товари перетворені
в гроші, продані, певна частина цих грошей знову мусить бути
перетворена в речові елементи сталого капіталу і саме в тих
пропорціях, яких вимагає певний технічний характер кожної
даної сфери виробництва. Щодо цього в усіх галузях — залишаючи осторонь заробітну плату, отже,
змінний капітал — найважливішим елементом є сировинний матеріал, включаючи й допоміжні матеріали,
які особливо важать у тих галузях виробництва, в які не входить сировинний матеріал у власному
значенні, як от у копальнях і добувній промисловості взагалі. Та частина ціни, яка мусить замістити
зношування машин, поки
машини ще взагалі здатні функціонувати, входить в обрахунок
більше ідеально; не має особливого значення, коли саме ця
частина буде оплачена й заміщена грішми, сьогодні чи завтра,
чи в якийсь інший період часу обороту капіталу. Інакше стоїть
справа з сировинним матеріалом. Якщо ціна сировинного матеріалу підвищується, то може стати
неможливим, після відрахування заробітної плати, цілком замістити ціну його з вартості товару. Тому
сильні коливання цін викликають перерви, великі
колізії і навіть катастрофи в процесі репродукції. Продукти
землеробства у власному розумінні слова, сировинні матеріали,
які походять з органічної природи, особливо підпадають таким
коливанням вартості в наслідок мінливих врожаїв і т. д. — кредитну систему ми тут ще цілком
залишаємо осторонь. Та сама
кількість праці може тут в наслідок непіддатних контролеві природних умов, сприятливості чи
несприятливості діб року та ін.,
виражатися в дуже різних кількостях споживних вартостей,
і тому певна кількість цих споживних вартостей матиме дуже
різну ціну. Якщо вартість x представлена в 100 фунтах товару $a$, то ціна одного фунта $a = \frac{x}{100}$; якщо
ж вона представлена в 1000 фунтах $а$, то ціна одного фунта $a = \frac{x}{1000}$ і т. д. Такий, отже, є один
елемент цих коливань ціни сировинного матеріалу.
Другий елемент, про який тут згадується тільки для повноти, — бо конкуренція, як і кредитна система,
лежить тут поки що поза
межами нашого розгляду, — є такий: з самої природи речей
рослинні й тваринні речовини, ріст і виробництво яких підлягають
\parbreak{}  %% абзац продовжується на наступній сторінці

\parcont{}  %% абзац починається на попередній сторінці
\index{ii}{0130}  %% посилання на сторінку оригінального видання
і вартостетворчий, і треба буде його замінити. Отже, авансована капітальна
вартість повинна зробити деякий цикл оборотів, в даному разі, приміром, цикл
у десять річних оборотів — і визначається цей цикл часом існування,
а тому й часом репродукції або часом обороту застосованого основного
капіталу.

Отже, в тій самій мірі, в якій з розвитком капіталістичного способу
продукції збільшуються розміри вартости і протяг життя застосовуваного
основного капіталу, — в тій самій мірі розвивається життя промисловости
й промислового капіталу в кожній особливій галузі приміщення в багаторічне
життя, скажімо, пересічно в десятирічне. Якщо, з одного боку,
розвиток основного капіталу подовжує це життя, то, з другого боку,
його скорочують постійні перевороти в засобах продукції, перевороти,
що з розвитком капіталістичної продукції так само набирають дедалі
більшої сили. Відси випливає й зміна засобів продукції та потреба постійно
їх заміщувати, бо вони зазнають морального зношування за довгий
час до того, як фізично доживуть свого віку. Можна припустити, що
для вирішальніших галузей промисловости цей цикл життя становить тепер
пересічно десять років. Однак, тут має значення не певне число. В усякому
разі ясно: цим багаторічним циклом взаємно зв’язаних оборотів, що
в них капітал є зв’язаний своєю основною складовою частиною, дається
матеріяльна основа періодичних криз, при чому підприємство послідовно
переживає періоди послаблення, середньої жвавости, раптового розмаху,
кризи. Правда, періоди, коли капітал вкладається, дуже різні й зовсім не
збігаються один з одним. Проте, криза завжди становить вихідний пункт
для нових великих капіталовкладань. Отже, розглядаючи справу з суспільного
погляду — вона також дає більш або менш нову матеріяльну
основу для наступного циклу оборотів\footnote{
„Міська продукція зв’язана з оборотом, що охоплює кілька днів, а сільська,
навпаки, з оборотом, що охоплює кілька років“. Adam G. Müller: „Die Elemente
der Staatskunst“. Berlin. 1809, II, ct. 178. Таке наївне уявлення романтики
про промисловість і хліборобство.
}.

5) Щодо способу обчислення обороту, дамо слово одному американському
економістові. „В деяких галузях підприємств ввесь авансований
капітал обертається або циркулює кілька разів протягом року; в інших
одна частина обертається більш як один раз на рік, а друга не так швидко.
Капіталіст повинен обчислювати свій зиск, зважаючи на той пересічний
період, що потрібен для цілого його капіталу, щоб перейти через його
руки або обернутись один раз. Припустімо, що людина вклала в певне
підприємство половину свого капіталу на будівлі й машини, що їх відновлюється
один раз на десять років; четверту частину — на знаряддя і т. ін.,
що їх відновлюється раз на два роки, і остання четверта частина, витрачена
на заробітну плату й сировинний матеріял, обертається двічі на рік.
Хай ввесь її капітал буде 50.000 долярів. Тоді її річні витрати будуть
такі:

\parcont{}  %% абзац починається на попередній сторінці
\index{iii1}{0131}  %% посилання на сторінку оригінального видання
його репродукцію, і таким чином відновлюється монополія тих
країн — джерел його постачання, які виробляють при найсприятливіших
умовах, — відновлюється, може, з певними обмеженнями,
але все ж відновлюється. Правда, репродукція сировинних
матеріалів в наслідок даного поштовху відбувається в розширеному
масштабі, особливо в країнах, які в більшій чи меншій мірі
володіють монополією цього виробництва. Але та база, на якій в наслідок
збільшення кількості машин і т. д. відбувається виробництво
і яка тепер після кількох коливань має стати новою нормальною
базою, новим вихідним пунктом, дуже розширилася в наслідок
процесів, що відбувались протягом останнього циклу обороту.
При цьому, однак, в частині другорядних джерел постачання
сировинного матеріалу репродукція, яка щойно збільшилась, знову
значно гальмується. Так, наприклад, з таблиць експорту ясно
видно, як протягом останніх 30 років (до 1865 року) зростало
індійське виробництво бавовни, коли наставала недостача в американському
виробництві, і потім раптом знову починалося
більш-менш тривале скорочення. Протягом часу подорожчання
сировинного матеріалу промислові капіталісти об’єднуються,
утворюють асоціації, щоб регулювати виробництво. Так було,
наприклад, в Манчестері після підвищення цін на бавовну в
1848 році, так само як і в виробництві льону в Ірландії. Але як
тільки безпосередній привід мине і знову суверенно запанує загальний
принцип конкуренції „купувати на найдешевшому ринку“
(замість того, щоб намагатися, як це робили згадані асоціації,
підвищити виробничу здатність відповідних країн — джерел постачання
сировинного матеріалу, незалежно від безпосередньої
ціни даного моменту, по якій ці країни можуть у даний час постачати
продукт), — отже, як тільки знову суверенно запанує
принцип конкуренції, регулювати подання знову полишається
„ціні“. Всяка думка про спільний, рішучий і передбачливий контроль
над виробництвом сировинного матеріалу — контроль,
який загалом і в цілому ніяк несполучний з законами капіталістичного
виробництва і тому завжди лишається благочестивим
побажанням або обмежується винятковими спільними кроками
в моменти великої безпосередньої небезпеки й безпорадності —
поступається місцем вірі в те, що попит і подання взаємно
регулюватимуть одно одне.\footnote{
Після того, як це було написано (1865 р.), конкуренція на світовому ринку
значно посилилася в наслідок швидкого розвитку промисловості в усіх культурних
країнах, 'особливо в Америці і Німеччині. Той факт, що швидко й колосально
зростаючі сучасні продуктивні сили з кожним днем все більше переростають
закони капіталістичного товарообміну, в межах яких вони повинні
рухатись, — цей факт нині все більше й більше проникає навіть у свідомість
самих капіталістів. Це виявляється особливо в двох симптомах. Поперше, в новій
загальній манії охоронних мит, яка відрізняється від старої системи охоронних
мит особливо тим, що вона найбільше захищає якраз товари, придатні до
експорту. Подруге, в картелях (trusts) фабрикантів цілих великих сфер виробництва
для регулювання виробництва і разом з тим цін і зисків. Само собою
зрозуміло, що ці експерименти здійснимі тільки при відносно сприятливій
економічній погоді. Перша ж буря повинна їх зруйнувати і довести, що, хоч
виробництво і потребує регулювання, але, без сумніву, не капіталістичний клас
покликаний здійснити його. Покищо ці картелі мають лиш одну мету —
дбати про те, щоб дрібні капіталісти пожиралися великими ще швидше, ніж
досі. — Ф. Е.
} Суєвірство капіталістів тут таке
грубе, що навіть фабричні інспектори в своїх звітах знов і знов
з приводу цього здивовано розводять руками. Чергування сприятливих
\index{iii1}{0132}  %% посилання на сторінку оригінального видання
і несприятливих років, звичайно, знов таки приводить до
здешевлення сировинного матеріалу. Незалежно від того безпосереднього
впливу, який ця обставина справляє на розширення
попиту, сюди долучається ще як стимул вищезгаданий
вплив на норму зиску. І зазначений вище процес ступневого
випереджання виробництва сировинних матеріалів виробництвом
машин і т. д. повторюється тоді в ширшому масштабі. Дійсне
поліпшення сировинного матеріалу, так щоб він постачався не
тільки в потрібній кількості, але й потрібної якості, наприклад,
бавовна американської якості з Індії, вимагало б тривалого, регулярно
зростаючого і постійного попиту з боку Европи (цілком
залишаючи осторонь ті економічні умови, в які поставлений індійський
виробник на своїй батьківщині). Але при таких умовах
сфера виробництва сировинних матеріалів змінюється тільки
стрибками, то раптом розширюється, то знову дуже скорочується.
Все це, як і дух капіталістичного виробництва взагалі, можна
дуже добре вивчати на бавовняному голоді 1861—1865 років, до
якого долучалася ще й та обставина, що часами зовсім не було
сировинного матеріалу, одного з найістотніших елементів репродукції.
Ціна може підвищуватись навіть і тоді, коли подання
цілком достатнє, але достатнє при тяжчих умовах. Або може
мати місце справжня недостача сировинного матеріалу. Під час
бавовняної кризи спочатку мала місце така недостача сировинного
матеріалу.

Отже, чим більше ми наближаємось в історії виробництва
до безпосередньої сучасності, тим регулярніше ми знаходимо,
особливо у вирішальних галузях виробництва, постійне повторення
чергувань відносного подорожчання і виникаючого з нього
пізнішого знецінення сировинних матеріалів органічного походження.
Ілюстрації до вищесказаного дано в наведених нижче
прикладах, взятих із звітів фабричних інспекторів.

Мораль історії, яку можна здобути також з дослідження землеробства
взагалі, полягає в тому, що капіталістична система
протидіє раціональному землеробству, або що раціональне землеробство
несполучне з капіталістичною системою (хоч ця остання
і сприяє його технічному розвиткові) і потребує або руки самостійно
працюючого дрібного селянина, або контролю асоційованих
виробників.

\pfbreak

Тепер ми наводимо щойно згадані ілюстрації з англійських
фабричних звітів.



\index{iii1}{0133}  %% посилання на сторінку оригінального видання
„Стан справ покращав; але цикл сприятливих і несприятливих
періодів скорочується із збільшенням кількості машин, і
коли при цьому збільшується попит на сировинний матеріал,
то частіше повторюються також і коливання в стані справ...
В даний момент не тільки відновилось довір’я після паніки
1857 року, але й сама паніка, здається, майже цілком забута.
Чи це покращання буде тривалим, чи ні, це в дуже значній
мірі залежить від ціни сировинних матеріалів. Я бачу вже ознаки
того, що в деяких випадках уже досягнуто максимуму, після
якого фабрикація ставатиме все менш зисковною, поки, нарешті,
вона й зовсім перестане давати зиск. Якщо ми візьмемо, наприклад,
зисковні роки у підприємствах чесаної вовни, 1849 і 1850,
то ми побачимо, що ціна англійської чесаної вовни стояла на
рівні 13 пенсів, австралійської — від 14 до 17 пенсів за фунт,
і що на протязі десяти років, з 1841 до 1850, пересічна ціна
англійської вовни ніколи не перевищувала 14 пенсів, а австралійської
— 17 пенсів за фунт. Але на початку нещасливого
1857 року австралійська вовна стояла на рівні 23 пенсів; у грудні,
в найлихіший час паніки, вона впала до 18 пенсів, але на протязі
1858 року знову підвищилась до теперішньої ціни в 21 пенс.
Англійська вовна в 1857 році так само почала з 20 пенсів, у квітні
й вересні вона підвищилась до 21 пенса, в січні 1858 року впала
до 14 пенсів, а потім підвищилась до 17 пенсів, так що тепер
ціна її за фунт на 3 пенси вища, ніж пересічна ціна протягом
наведених 10 років... Це свідчить, на мою думку, про те, що або
банкрутства 1857 року, викликані подібними цінами, забуті, або
що вовни виробляється ледве-ледве стільки, скільки можуть
перепрясти наявні веретена, абож має місце тривале підвищення
ціни тканин... Але в моїй дотеперішній практиці я бачив,
як на протязі неймовірно короткого часу не тільки збільшилась
кількість веретен і ткацьких верстатів, але й швидкість
їх роботи; далі, що майже в тій самій мірі підвищився наш вивіз
вовни до Франції, тимчасом як пересічний вік овець, вирощуваних
як усередині країни, так і за кордоном, стає все нижчим,
бо населення швидко збільшується, і вівчарі бажають якомога
швидше перетворити своїх овець у гроші. Тому я часто переживав
тяжке почуття, коли бачив людей, які, не знаючи цього, вкладали
свою долю і свій капітал у підприємства, успіх яких залежить від
подання такого продукту, який може збільшуватись тільки
згідно з певними органічними законами... Стан попиту й подання
всіх сировинних матеріалів... пояснює, як видно, багато коливань
у бавовняній промисловості, а також стан англійського вовняного
ринку восени 1857 року і викликану ним промислову кризу“\footnote{Само собою зрозуміло, що ми не \emph{пояснюємо}, разом з паном Бекером,
вовняну кризу 1857 року невідповідністю між цінами сировинного матеріалу
і фабрикату. Ця невідповідність сама була тільки симптомом, а криза була
загальною. — \emph{Ф. Е.}}
(\emph{R. Baker} в „Rep. of. Insp. of Fact., Oct. 1858“, стор. 56—61).


\index{i}{0134}  %% посилання на сторінку оригінального видання
Отже, ми бачимо: чи з’являється споживна вартість як сировинний
матеріял, чи як засіб праці або продукт — це геть чисто
залежить від певної її функції у процесі праці, від місця, яке
вона займає в ньому, і зі зміною цього місця змінюються й ті
її визначення.

Тому, увіходячи як засоби продукції в нові процеси праці,
продукти втрачають характер продуктів. Вони функціонують
ще тільки як предметні фактори живої праці. Прядун ставиться
до веретена лише як до засобу, яким він пряде, до льону — лише
як до предмету, що його він пряде. Звичайно, не можна прясти
без матеріялу для прядіння й без веретен. Тому наявність цих
продуктів доводиться припустити вже з самого початку прядіння.
Але для самого цього процесу той факт, що льон і веретена є
продукти минулої праці, не має ніякого значення цілком так
само, як для акту харчування не має ніякого значення той факт,
що хліб є продукт минулої праці рільника, мірошника, пекаря
й~\abbr{т. ін.} Навпаки, якщо засоби продукції й виявляють у процесі
праці свій характер продуктів минулої праці, то лише через їхні
вади. Ніж, що не ріже, пряжа, що раз-у-раз рветься, і~\abbr{т. ін.}
живо нагадують ножівника $А$ і прядуна $В$. На вдалому продукті
не помітно слідів минулої праці, що надала йому його споживних
властивостей.

Машина, що не функціонує в процесі праці, є некорисна.
Окрім того, вона зазнає руйнаційного впливу природного обміну
речовин. Залізо ржавіє, дерево гниє. Пряжа, що її не тчуть або
не плетуть, є зіпсована бавовна. Жива праця мусить охопити
ці речі, з мертвих зробити їх живими, перетворити їх з лише
можливих у дійсні й діяльні споживні вартості. Охоплені вогнем
праці, яка асимілює їх як своє тіло, надхнені в процесі праці на виконання
функцій, що відповідають їхній ідеї і призначенню, вони
хоч і будуть спожиті, але спожиті доцільно, як елементи утворення
нових споживних вартостей, нових продуктів, які здатні
увійти як засоби існування в особисте споживання або як засоби
продукції в новий процес праці.

Отже, коли наявні продукти є не лише результат, але й умови
існування процесу праці, то, з другого боку, вкладання їх у
процес праці, отже, їхній контакт з живою працею, є єдиний
засіб для того, щоб зберегти й реалізувати ці продукти минулої
праці як споживні вартості.

Праця споживає свої речові елементи, свій предмет і свої
засоби, з’їдає їх, отже, це є процес споживання. Це продуктивне
споживання відрізняється тим від особистого споживання, що в
останньому продукти споживаються як засоби існування живого
індивіда, а в першому — як засоби існування праці, робочої
сили індивіда, яка виявляється в діяльності. Отже, продукт особистого
споживання є сам споживач, а результат продуктивного
споживання є відмінний від споживача продукт.

Оскільки засоби й предмет праці сами вже є продукти, праця
споживає продукти, щоб утворювати продукти, або застосовує
\parbreak{}  %% абзац продовжується на наступній сторінці

\parcont{}  %% абзац починається на попередній сторінці
\index{ii}{0135}  %% посилання на сторінку оригінального видання
продуктивного капіталу та їхній вплив на характер обороту. Ба навіть
він одразу наводить, як приклад, купецький капітал у такому питанні,
де йдеться виключно про ріжниці частин продуктивного капіталу
в процесі утворення продукту й вартости — ріжниці, що й собі утворюють
ріжниці в обороті й репродукції капіталу.

Він каже далі: „Капітал, застосовуваний таким способом, не дає своєму
власникові доходу або зиску, поки він лишається в його посіданні або
зберігає ту саму форму“\footnote*{
„The capital employed in this manner yields no revenue or profit to its employer,
while it either remains in his possession or continues in the same shape“.
}. — Капітал, застосовуваний таким способом!
Але ж А.~Сміс каже про капітал, вкладений у сільське господарство або
промисловість, і далі каже нам, що приміщений таким способом капітал
розподіляється на основний та обіговий. Отже, приміщення капіталу цим
способом само собою не може зробити його ні основним, ні обіговим.

Але, може, він хотів сказати, що капітал, застосований для того, щоб
продукувати товари й продавати ці товари з зиском, мусить, по перетворенні
на товари, продаватись і через продаж, поперше, переходити з
власности продавця у власність покупця, а подруге, зміняти свою натуральну
форму товару на грошову форму, і тому капітал не є корисний
для свого власника, поки він лишається в його посіданні або зберігає —
для нього — ту саму форму? Однак тоді справа сходить ось на що: та
сама капітальна вартість, яка раніш функціонувала в формі продуктивного
капіталу, в формі належній до продукційного процесу, функціонує
тепер як товаровий капітал і грошовий капітал, — в формах капіталу, належних
до процесу циркуляції, і тому вона вже не є ні основний, ні поточний
капітал. І це має силу так само для тих елементів вартости, що
долучаються сировинними та допоміжними матеріялами, отже, поточним
капіталом, як і для тих, що долучаються в наслідок зношування засобів
праці, отже, основним капіталом. Таким чином, ми тут ні на крок не
наблизились до висвітлення ріжниці між основним і поточним капіталом.

Далі: „Товари торговця не дають йому жодного доходу або зиску,
поки він не продасть їх за гроші, і гроші так само мало дають йому,
поки він знову не обміняє їх на товари. Його капітал безупинно одходить
від нього в одній формі й повертається до нього в другій і тільки
за допомогою такої циркуляції або послідовних актів обміну може дати
йому будь-який зиск. Тому такі капітали можна назвати у власному значенні
слова обіговими капіталами“\footnote*{
„The goods of the merchant yield him no revenue or profit tilt he sells them
for money, and the money yields him as little till it is again exchanged for goods.
His capital is continually going from him in one shape, and returning to him in
another, and it is only by means of such circulation, or successive exchanges, that
it can yield him any profit. Such capitals, therefore, may very properly be called
circulating capitals“.
}.

Те, що А.~Сміс визначає тут як обіговий капітал, я хочу назвати
капіталом циркуляції (Zirkulationskapital). Це капітал в формі,
належній до процесу циркуляції, капітал, що змінює форму за допомогою
обміну (зміни речовин і зміни власника), отже, товаровий капітал і грошовий
\parbreak{}  %% абзац продовжується на наступній сторінці

\parcont{}  %% абзац починається на попередній сторінці
\index{ii}{0136}  %% посилання на сторінку оригінального видання
капітал протилежно до його форми, належної до процесу продукції,
тобто протилежно до форми продуктивного капіталу. Це не різні
відміни, що на них поділяє промисловий капіталіст свій капітал, а різні
форми, що їх поступінно завжди набирає й скидає та сама авансована
капітальна вартість протягом свого curriculum vitae\footnote*{
Curriculum vitae (лат.) — перебіг життя. — \emph{Ред.}
}. А.~Сміс сплутує це —
роблячи великий крок назад порівняно з фізіократами — з тими відмінностями
форми, що постають у межах циркуляції капітальної вартости, в її кругобігу
через ряд її послідовних форм тоді, коли капітальна вартість перебуває
в формі продуктивного капіталу; і постають вони саме в наслідок
різних способів, що ними різні елементи продуктивного капіталу беруть
участь в процесі утворення вартости й переносять свою вартість на продукт.
Ми розглянемо далі наслідки цього основного сплутування капіталу
продуктивного й капіталу, що перебуває в сфері циркуляції (товарового
капіталу й грошового капіталу), з одного боку, і основного та поточного
капіталу, з другого. Капітальна вартість, авансована на основний капітал,
так само циркулює разом з продуктом, як і вартість, авансована на поточний,
капітал, і через циркуляцію товарового капіталу перша так само перетворюється
на грошовий капітал, як і друга. Ріжниця виникає лише з того,
що вартість, авансована на основний капітал, циркулює частинами, а тому
й мусить вона також частинами, протягом довших або коротших періодів,
заміщуватись, репродукуватися в натуральній формі.

Що А.~Сміс розуміє тут під обіговим капіталом не що інше, як капітал
циркуляції, тобто капітальну вартість в її формах, належних до процесу
циркуляції (товаровий капітал і грошовий капітал), це доводить приклад,
обраний ним особливо невлучно. Він бере як приклад відміну капіталу, що
зовсім не належить до процесу продукції, а існує лише в сфері циркуляції,
складається лише з капіталу циркуляції: він бере купецький капітал.

Як безглуздо починати прикладом, де капітал взагалі фігурує не як
продуктивний капітал, він сам каже про це зараз же далі: „Капітал торговця
складається цілком з обігового капіталу“. („The capital of a merchant
is altogether a circulating capital“). Але ріжниця між обіговим і основним
капіталом постає, як нам далі скажуть, з посутніх ріжниць в середині
самого продуктивного капіталу. З одного боку, А.~Сміс має на
увазі визначену в фізіократів ріжницю, з другого боку, — відмінності форми;
що їх пророблює капітальна вартість у процесі свого кругобігу. І те
й друге сплутує він в одну строкату купу.

Але як може утворюватись зиск в наслідок зміни форми грошей і
товару, в наслідок простого перетворення вартости з однієї з цих форм
на другу, це лишається цілком незрозуміло. Та й не можна зовсім цього
пояснити, бо він починає тут з купецького капіталу, що функціонує
лише в сфері циркуляції. Ми ще повернемось до цього, а покищо послухаймо,
що каже А.~Сміс про основний капітал:

„Подруге, його (капітал) можна застосовувати на поліпшення грунту,
на закуп корисних машин і знарядь праці та подібні речі, що дають
\parbreak{}  %% абзац продовжується на наступній сторінці

\parcont{}  %% абзац починається на попередній сторінці
\index{ii}{0137}  %% посилання на сторінку оригінального видання
дохід або зиск, не змінюючи власника, не циркулюючи далі. Тому такі
капітали можна назвати основними капіталами у власному значенні цього
слова. Різні підприємства потребують поділу вкладеного в них капіталу
на основний та обіговий в дуже різних пропорціях\dots{} Кожен ремісник
або фабрикант мусить деяку частину свого капіталу зв’язати в формі
засобів праці своєї галузі. Ця частина однак іноді дуже мала, іноді дуже
велика\dots{} Куди більша частина капіталу всіх цих ремісників (кравців, шевців,
ткачів) перебуває однак в циркуляції, то як заробітна плата їхніх
робітників, то як ціна їхнього сировинного матеріялу, і її треба оплатити
з зиском в ціні їхніх продуктів“\footnote*{
„Secondly, it (capital) may be employed in the improvement of land, in the
purchase of useful machines and instruments of trade, or in such like things as
yield a revenue or profit without changing masters, or circulating any further. Such
capitals therefore, may very properly be called fixed capitals. Different occupations
require very different proportions between the fixed and circulating capitals employed
in them\dots{} Some part of the capital of every master artificer or manufacturer
must be fixed in the instruments of his trade. This part, however, is very small in
some, and very great in others.. The far greater part of the capital of all such master
artificers however is circulated, either in the wages of their workmen, or in the
price of their materials, and to be repaid with a profit by the price of the work“.
}.

Не кажучи вже про дитяче визначення джерела зиску, хибність і заплутаність
видно ось з чого: для фабриканта-машинобудівника, напр.,
машина є продукт, що циркулює як товаровий капітал, або, кажучи словами
А.~Сміса: „is parted with, changes masters, circulates further“ (відокремлюється,
змінює власника, циркулює далі). Отже, машина згідно з
його власним визначенням була б не основним, а обіговим капіталом. Ця
плутанина знову таки постає тому, що Сміс сплутує ріжницю між основним
і поточним капіталом, яка постає в наслідок неоднакових способів
циркуляції різних елементів продуктивного капіталу, з відмінностями
форми, що їх перебігає той самий капітал, оскільки він функціонує
в продукційному процесі як продуктивний капітал, а в сфері
циркуляції, навпаки, як капітал циркуляції, тобто як товаровий капітал
або як грошовий капітал. Тому ті самі речі, залежно від того місця, що
його вони мають у життьовому процесі капіталу, можуть, за А.~Смісом,
функціонувати і як основний капітал (як засоби праці, елементи продуктивного
капіталу) і як „обіговий“ капітал, товаровий капітал (як продукт,
виштовхнутий з сфери продукції в сферу циркуляції).

Але А.~Сміс сплутує разом з тим самі основи цього розподілу й суперечить
тому, з чого він почав кількома рядками вище цілий свій
дослід. Це саме сталось у реченні: „Є два способи застосувати капітал
так, щоб він давав своєму власникові дохід або зиск“, а саме — застосувати
його або як обіговий або як основний капітал. Тут мають на увазі,
очевидно, різні способи застосування різних і незалежних один від одного
капіталів, як, напр., капітали, що їх можна вкласти або в промисловість,
або в хліборобство. Але далі ми читаємо: „Різні підприємства
потребують поділу вкладеного в них капіталу на основний та обіговий
в дуже різних пропорціях“. Тепер основний та обіговий капітал є вже
\parbreak{}  %% абзац продовжується на наступній сторінці

\input{_0138.tex}

\index{iii1}{0139}  %% посилання на сторінку оригінального видання
\subsubsection{1861—1864 рр. Американська громадянська війна. Cotton Famine [бавовняний
голод]. Найбільший приклад перерви в процесі виробництва в наслідок
недостачі й дорожнечі сировинного матеріалу}

1860 рік. Квітень. „Щодо стану справ, то я радий можливості
повідомити вас, що, не зважаючи на високу ціну сировинних
матеріалів, всі галузі текстильної промисловості, за винятком
шовку, працювали протягом останнього півроку дуже добре...
В деяких бавовняних округах робітників шукали шляхом оголошень,
і робітники йшли туди з Норфолька та інших землеробських
графств... Як видно, в усіх галузях промисловості панує велика недостача
сировинного матеріалу. Тільки... ця недостача тримає нас
у певних межах. В бавовняній промисловості число новозбудованих
фабрик, розширення наявних фабрик і попит на робітників,
мабуть, ніколи ще не досягали такого високого рівня, як
тепер. Скрізь і всюди шукають сировинного матеріалу“ („Rep.
of Insp. of Fact., April 1860“ [стор. 57]).

1860 рік. Жовтень. „Стан справ у бавовняних, шерстяних
і льонопрядільних округах був добрий; в Ірландії він, як кажуть,
вже більше року навіть „дуже добрий“, і був би ще кращий,
коли б не висока ціна на сировинний матеріал. Прядільники
льону, здається, з більшим нетерпінням, ніж будьколи,
чекають відкриття індійських джерел постачання за допомогою
залізниць і відповідного розвитку індійського землеробства, щоб,
нарешті... добитися відповідного їх потребам подання льону“
(„Rep. of Insp. of Fact., Oct. 1860“, стор. 37).

1861 рік. Квітень. „Стан справ у даний момент пригнічений...
деякі бавовняні фабрики працюють неповний час і багато шовкових
фабрик працюють тільки частково. Сировинний матеріал
дорогий. Майже в усіх галузях текстильної промисловості
ціна його вища, ніж та, при якій він міг би бути перероблений
для маси споживачів“ („Rep. of Insp. of Fact., April 1861“, стор. 33).

Тепер виявилось, що в 1860 році в бавовняній промисловості
була перепродукція; наслідки цього давалися взнаки ще протягом
ближчих років. „Потрібно було від двох до трьох років,
поки світовий ринок поглинув перепродукцію 1860 року“ („Rep.
of Insp. of Fact., October 1863“, стор. 127). „Пригнічений стан
ринків бавовняних фабрикатів у Східній Азії, на початку 1860 року,
справив відповідний зворотний вплив на стан справ у Блекберні,
де пересічно 30 000 механічних ткацьких верстатів майже виключно
заняті у виробництві тканин для цього ринку. В наслідок
цього попит на працю був уже тут обмеженим багато місяців
перед тим, як став відчутним вплив бавовняної блокади...
На щастя, це уберегло багатьох фабрикантів від краху. Запаси,
поки їх тримали на складах, підвищились у своїй вартості, і таким
чином уникнуто було того жахливого знецінення, яке
інакше при такій кризі було б неминучим“ („Rep. of Insp. of
Fact., Oct. 1862“, стор. 28, 29 [30]).


\index{iii1}{0140}  %% посилання на сторінку оригінального видання
1861 рік. Жовтень. „Стан справ з деякого часу був дуже
пригнічений... Немає нічого неймовірного, що протягом зимніх
місяців багато фабрик дуже скоротять робочий час. Це, зрештою,
можна було передбачити... цілком незалежно від тих причин,
які припинили наш звичайний довіз бавовни з Америки і наш вивіз,
скорочення робочого часу протягом наступної зими стало б
необхідним в наслідок сильного збільшення виробництва за
останні три роки і в наслідок порушень на індійському й китайському
ринках“ („Rep. of Insp. of Fact., Oct. 1861“, стор. 19).

\subsubsection{Бавовняні відпади. Ост-індська бавовна (Surat). Вплив на заробітну плату
робітників. Поліпшення машин. Заміна бавовни крохмальним борошном і
мінералами. Вплив цього шліхтування крохмальним борошном на робітників.
Прядільники тонких нумерів пряжі. Ошуканство фабрикантів}

„Один фабрикант пише мені таке: „Щодо оцінки споживання
бавовни на одно веретено, то ви, мабуть, не досить берете до
уваги той факт, що коли бавовна дорога, кожен прядільник
звичайної пряжі (скажімо, до № 40, переважно № 12—32) пряде
такі тонкі нумери, які тільки може, тобто він прястиме № 16
замість попереднього № 12 або № 22 замість № 16 і т. д.,
і ткач, який тче з цієї тонкої пряжі, доведе свій ситець до
звичайної ваги, додаючи до нього відповідно більше шліхти.
Цим способом користуються тепер в справді ганебних розмірах.
Я чув з надійного джерела, що є звичайний Shirting [тканина
для сорочок] для експорту вагою в 8 фунтів штука, з яких
2 \sfrac{3}{4} фунти є шліхта. В тканинах інших сортів часто є до 50\%
шліхти, так що фабрикант аж ніяк не бреше, коли вихваляється,
що він багатіє, продаючи фунт своєї тканини дешевше, ніж він
заплатив за пряжу, з якої ця тканина зроблена“ („Rep. of Insp.
of Fact., April 1864“, стор. 27).

„Мені також казали, що ткачі приписують свою підвищену
захворюваність шліхті, яку застосовують для основи, випряденої
з ост-індської бавовни, і яка вже не складається, як раніше,
з чистого борошна. Однак, цей сурогат борошна дає, як кажуть,
ту велику вигоду, що він значно збільшує вагу тканини, так
що з 15 фунтів пряжі стає 20 фунтів тканини“ („Rep, of Insp.
of Fact., Oct. 1863“, стор. 63. Цим сурогатом був перемолотий
тальк, називаний China clay [китайською глиною], або гіпс, називаний
French chalk [французькою крейдою].) — „Заробіток ткачів
(тобто тут робітників) дуже зменшується в наслідок застосовування
сурогатів борошна для шліхтування основи. Ця шліхта
робить пряжу важчою, але також твердою і ламкою. Кожна
нитка основи проходить у ткацькому верстаті через так званий
реміз, міцні нитки якого тримають основу в правильному положенні;
твердо нашліхтована основа спричинює постійні розриви
ниток у ремізі; при кожному розриві ткач втрачає п’ять хвилин
на виправлення; тепер ткачеві доводиться виправляти такі
пошкодження принаймні в 10 разів частіше, ніж раніш, і верстат,
\index{iii1}{0141}  %% посилання на сторінку оригінального видання
розуміється, дає протягом робочих годин настільки ж менше
тканини“ (там же, стор. 42, 43).

„В Аштоні, Стелібріджі, Мослеї, Ольдгемі і т. д. робочий
час скорочений на цілу третину, і з кожним тижнем робочі години
скорочуються ще більше... Одночасно з цим скороченням
робочого часу в багатьох галузях відбувається також зниження
заробітної плати“ (стор. 13). — На початку 1861 року стався
страйк механічних ткачів у деяких частинах Ланкашіра. Деякі
фабриканти заявили про зниження заробітної плати на 5—7 \sfrac{1}{2}\%;
робітники настоювали на тому, щоб рівень заробітної плати лишити
незмінним, а робочий день скоротити: Фабриканти на це не
згодились, і почався страйк. Через місяць робітники мусили поступитися.
Але тепер вони одержали і те і друге: „Крім зниження
заробітної плати, на що робітники кінець-кінцем згодились,
вони на багатьох фабриках працюють тепер неповний час“.
(„Rep. of Insp. of Fact., April 1861“, стор. 23).

1862 рік. Квітень. „Страждання робітників від часу мого
останнього звіту значно збільшились; але ще ніколи в історії
промисловості такі раптові і такі тяжкі страждання не переносилися
з такою мовчазною покірливістю і таким терпеливим
самовладанням“ („Rep. of Insp. of Fact., April 1862“, стор. 10). —
„Відносне число цілком безробітних робітників в даний момент,
здається, не дуже перевищує число безробітних 1848 року,
коли панувала звичайна паніка, яка, однак, була досить значною,
щоб спонукати занепокоєних фабрикантів складати такі
самі статистичні відомості про бавовняну промисловість, які
тепер публікують щотижня... В травні 1848 року з усіх бавовняних
робітників Манчестера 15\% було без роботи, 12\% працювало
неповний час, тоді як понад 70\% працювало повний час. 28 травня
1862 року без роботи було 15\%, 35\% працювало неповний час,
49\% — повний час... В сусідніх місцевостях, наприклад, в Стокпорті,
процент тих, що працюють неповний час, і тих, що зовсім
не працюють, вищий, процент тих, що працюють повний час,
нижчий“, бо тут випрядаються грубіші нумери, ніж у Манчестері
(стор. 16).

1862 рік. Жовтень. „За останніми офіціальними статистичними
даними, в 1861 році в Сполученому Королівстві було 2887 бавовняних
фабрик, з них 2109 в моїй окрузі (Ланкашір і Чешір). Я, звичайно,
знав, що дуже значна частина з цих 2109 фабрик моєї округи
були дрібні підприємства, які вживали небагато робітників. Я
був, однак, дуже здивований, коли виявив, як багато таких підприємств.
В 392, або 19\%, рушійна сила, пара або вода, менша за
10 кінських сил; в 345, або 16\%, між 10 і 20 кінськими силами;
в 1372 — 20 кінських сил і більше... Дуже значна частина цих
дрібних фабрикантів — більше ніж третина загального числа їх —
не дуже давно самі були робітниками; це — люди, які не мають
у своєму розпорядженні капіталу... Центр ваги падає, отже, на
інші \sfrac{2}{3}“ („Rep. of Insp. of Fact., Oct. 1862“, стор. 18, 19).


\index{iii1}{0142}  %% посилання на сторінку оригінального видання
За даними того самого звіту, з бавовняних робітників Ланкашіра
й Чешіра тоді працювали повний час 40 146 робітників,
або 11,3\%, неповний робочий час — 134 767 робітників, або 38\%,
зовсім без роботи було 179 721 робітник, або 50,7\%. Коли виключити
звідси дані про Манчестер і Больтон, де випрядаються
головним чином тонкі нумери, — галузь, що порівняно мало потерпіла
від недостачі бавовни, — то справа виявиться ще несприятливішою, 'а
саме: таких, що працюють повний час — 8,5\%,
неповний час — 38\%, безробітних — 53,5\% (стор. 19, 20).

„Для робітників становить істотну ріжницю, чи переробляють
вони добру чи погану бавовну. В перші місяці року, коли фабриканти
намагались тримати свої фабрики в русі тим, що вживали
всяку бавовну, яку тільки можна було купити по помірних цінах,
багато поганої бавовни потрапило на ті фабрики, де раніше звичайно
застосовували добру; ріжниця в заробітній платі робітників
була така велика, що відбулося багато страйків, бо робітники
при старій відштучній платі тепер не могли вже добути
собі зносного щоденного заробітку... В деяких випадках ріжниця
в наслідок застосовування поганої бавовни становила навіть при
повному робочому часі половину всього заробітку“ (стор. 27).

1863 рік. Квітень. „На протязі цього року зможуть бути заняті
повний час трохи більше половини бавовняних робітників“
(„Rep. of Insp. of Fact., April 1863“, стор. 14).

„Дуже серйозна невигода при застосуванні ост-індської бавовни,
яку тепер фабрики мусять споживати, полягає в тому,
що швидкість машин при цьому мусить бути дуже уповільнена.
Протягом останніх років було вжито всіх заходів для збільшення
цієї швидкості, так щоб ті самі машини виконували більше
роботи. Але зменшена швидкість зачіпає робітника в такій самій
мірі, як фабриканта, бо більшість робітників одержують відштучну
плату — прядільники стільки то за фунт випряденої
пряжі, ткачі стільки то за витканий кусок; і навіть у інших
робітників, які одержують тижневу плату, заробітна плата повинна
знизитися в наслідок зменшення виробництва. На підставі
моїх досліджень... і переданих мені даних про заробіток бавовняних
робітників на протязі цього року... виявляється зменшення
заробітної плати пересічно на 20\%, в деяких випадках
на 50\% порівняно з висотою заробітної плати 1861 року“
(стор. 13). — „Зароблена сума залежить... від того, який матеріал
переробляється... Становище робітників, щодо суми заробленої
плати, тепер (жовтень 1863 року) багато краще, ніж минулого
року в цей час. Машини поліпшено, сировинний матеріал
знають краще, і робітники легше справляються з тими труднощами,
з якими їм доводилося боротись спочатку. Минулої весни
я був у Престоні в одній швацькій школі“ [благодійна установа
для безробітних]; „дві молоді дівчини, які за день перед тим
були послані до ткацької фабрики, де, за заявою фабриканта, вони
могли б заробити 4 шилінга на тиждень, просили, щоб їх знову
\parbreak{}  %% абзац продовжується на наступній сторінці

\input{_0143_0144_0145.tex}
\parcont{}  %% абзац починається на попередній сторінці
\index{i}{0146}  %% посилання на сторінку оригінального видання
упредметненої праці, отже, їх не береться до рахуби й не входять
вони у продукт утворення вартости\footnote{
Це одна з обставин, які удорожчують продукцію, основану на
рабстві. Робітник, як влучно висловлювалися за старовини, відрізняється
тут лише як instrumentum vocale\footnote*{
знаряддя, обдароване мовою. \emph{Ред.}
} від тварини як instrumentum semivocale\footnote*{
знаряддя, обдарованого голосом. \emph{Ред.}
} і від мертвого знаряддя праці як від instrumentum mutum\footnote*{
знаряддя німого. \emph{Ред.}
}. Але сам робітник дає відчути тварині і знаряддю праці, що він їм не
рівня, а що він людина. Збиткуючися з них і con amore\footnote*{
з насолодою. \emph{Ред.}
} руйнуючи їх, він з самозадоволенням переконує себе самого в своїй відмінності
від них. Тому за цього способу продукції вважається за економічний принцип
вживати лише найгрубіших, найтяжчих знарядь праці, бо саме через
цю грубість і незграбність їх важко знівечити. Тому в рабовласницьких
державах, які лежать над Мехіканською затокою, до вибуху громадянської
війни, вживали плугів старокитайської конструкції, що рили землю,
як свиня або кріт, але не робили борозни, не повертали її. Порівн.
J.~С.~Cairns: «The Slave Power», London 1862, p. 46 і далі. У своїй праці
«Sea Bord Slave States» (p. 46, 47) Олмстед оповідає, між іншим, таке:
«Мені тут показували знаряддя, що їх у нас жодна людина із здоровим
розумом ніколи не дала б найманому робітникові, бо вони обтяжали б
його; на мою думку, їхня надзвичайна вага й незграбність збільшують
працю щонайменше на 10\% порівняно з тим знаряддям, що його звичайно
вживають у нас. І я певен, що за недбалого й грубого поводження рабів
із знаряддям праці було б неекономно дати їм легше й не таке грубе знаряддя.
А ті знаряддя, що їх ми з користю завжди даємо нашим робітникам,
не збереглися б жодного дня на хлібних полях Вірґінії, хоч ґрунт там
і легший і не такий кам’янистий, як наш. Так само, коли я спитав, чому
на всіх фармах замість коней вживають мулів, то перший арґумент, звичайно,
найдовідніший, був той, що коні не могли б витримати поводження
з боку негрів; у наслідок такого поводження коні завжди швидко нівечаться
або калічіють, тоді як мули витримують биття і брак харчів, не
зазнаючи від цього жодної матеріяльної шкоди, не перестуджуються й не
хоріють, навіть коли нехтувати ними й обтяжати їх працею. Алеж мені
досить підійти до вікна кімнати, де я пишу, щоб побачити завжди таке
поводження з худобою, за яке північний фармер негайно прогнав би погонича».
(«І am here shown tools that no man in his senses, with us, would
allow a labourer for whom he was paying wages, to be encumbered with:
and the excessive weight and clumsiness of which, I would judge, would
make work at least ten per cent greater than with those ordinarily used
with us. And I am assured that, in the careless and clumsy way they must
be used by the slaves, anything ligther or less rude could not be furnished
them with good economy, and that such tools as we constantly give our
bourers, and find our profit in giving them, would not last out a day in
Virginia cornfield — much lighter and more free from stones though it be
than ours. So, too, when I ask why mules are so universally substituted for
horses on the farm, the first reason given, and confessedly the most conclusive
one, is that horses cannot bear the treatment that they always must get
from negroes; horses are always soon foundered or crippled by them while
mules will bear cudgelling, and lose a meal or two now and then, and not
be materially injured, and they do not take cold or get sick, if neglected
or overworked. But I do not need to go further than to the window of the room
in which I am writing, to see at almost any time, treatment of cattle
that would insure the immediate discharge of the driver by almost any farmer
owning them in the North»).
}.

Ми бачимо, що встановлена вже раніш аналізою товару ріжниця
між працею, оскільки вона утворює споживну вартість, і
\parbreak{}  %% абзац продовжується на наступній сторінці

\parcont{}  %% абзац починається на попередній сторінці
\index{ii}{0147}  %% посилання на сторінку оригінального видання
вони не становлять жодного елементу продуктивного капіталу, хоч яке
буде їхнє остаточне призначення, тобто чи кінець-кінцем входять вони
відповідно до свого призначення (своєї споживної вартости) в сферу особистого,
чи продуктивного споживання. В пункті 2 ці продукти є засоби
харчування, в пункті 4 — всі інші готові продукти, що, отже, знову таки
складаються з готових засобів праці або готових засобів споживання
(інші, ніж засоби харчування, зазначені в пункті 2).

Що А.~Сміс при цьому каже і про торговця, це виявляє його плутанину.
Оскільки продуцент продав торговцеві свій продукт, то цей останній
вже взагалі не становить жодної форми його капіталу. З погляду
суспільства це, правда, все ще товаровий капітал, хоч він перебуває в
інших руках, ніж руки його продуцента; але саме тому, що це капітал
товаровий, він не може бути ні основним, ні обіговим капіталом.

В кожній продукції, що не має на меті безпосередньо задовольняти
власні потреби, продукт мусить циркулювати як товар, тобто його
треба продати не для того, щоб одержати таким чином зиск, а для того,
щоб взагалі продуцент міг існувати. За капіталістичної продукції до
цього долучається та обставина, що під час продажу товару реалізується
й додаткову вартість, що міститься в ньому. Продукт виходить з процесу
продукції як товар, а тому він не є ні основний, ні обіговий елемент
цього процесу.

А проте, А.~Сміс тут сам себе збиває. Всі готові продукти, хоча
яка буде їхня речова форма або їхня споживна вартість, їхній корисний
ефект, є тут товаровий капітал, тобто капітал в формі, належній до процесу
циркуляції. Перебуваючи в цій формі, вони зовсім не становлять
складових частин продуктивного капіталу їхнього власника; це ні в якому
разі не заважає тому, що скоро тільки їх продано, вони \so{стають} в
руках покупця складовими частинами продуктивного капіталу, все одно —
обігового, чи основного. Тут виявляється, що ті самі речі, що деякий
час виступали на ринку як товаровий капітал протилежно до продуктивного,
пізніше, скоро тільки їх взято з ринку, можуть функціонувати або
не функціонувати, як поточна або основна складова частина продуктивного
капіталу.

Продукт бавовнопрядника — пряжа — є товарова форма його капіталу,
товаровий капітал для нього. Пряжа не може функціонувати знову, як
складова частина його продуктивного капіталу, ні як матеріял праці, ні
як засіб праці. Але в руках ткача, що її купив, вона входить в його
продуктивний капітал, як одна з поточних складових частин його. Але
для прядуна пряжа є носій вартости частини його капіталу — так основного,
як і поточного (додаткову вартість ми лишаємо осторонь). Так
машина, як продукт фабриканта машин, є товарова форма його капіталу,
товаровий капітал для нього, і доки вона зберігає цю форму, вона не є
ні поточний, ні основний капітал. Продана одному з фабрикантів, що
вживають її, вона стає основною складовою частиною продуктивного капіталу.
Навіть тоді, коли продукт має таку споживну форму, що почасти
може, як засіб продукції, ввійти знову в той самий процес, що з нього
\parbreak{}  %% абзац продовжується на наступній сторінці

\parcont{}  %% абзац починається на попередній сторінці
\index{iii1}{0148}  %% посилання на сторінку оригінального видання
сприятливого вибору ринку, може бути дуже різна залежно від
більшої чи меншої дешевини сировинного матеріалу, закупівлі
його з більшим чи меншим знанням справи; залежно від того,
наскільки застосовувані машини є продуктивні, доцільні й дешеві;
залежно від більшої чи меншої досконалості загальної
організації різних ступенів процесу виробництва, від того, наскільки
усунено марнування матеріалу, наскільки просто й доцільно
організовано управління й нагляд і т. п. Коротко кажучи,
якщо додаткова вартість для певного змінного капіталу є дана,
то та сама додаткова вартість може виражатися в більшій чи
меншій нормі зиску, отже, може давати більшу чи меншу масу
зиску залежно від особистої ділової спритності самого капіталіста
або його наглядачів і прикажчиків. Припустім, що та сама додаткова
вартість в 1000 фунтів стерлінгів, продукт 1000 фунтів
стерлінгів заробітної плати, в підприємстві $A$ припадає на
9000 фунтів стерлінгів, а в іншому підприємстві $В$ — на 11000
фунтів стерлінгів сталого капіталу. У випадку $А$ ми маємо
$р' = \frac{1000}{10000} = 10\%$. У випадку $В$ ми маємо $р' = \frac{1000}{12000} = 8\sfrac{1}{3}\%$.
Весь капітал виробляє в $А$ порівняно більше зиску, ніж у $В$, бо
там норма зиску вища, ніж тут, хоч в обох випадках авансований
змінний капітал = 1000 і здобута з нього додаткова
вартість також = 1000, отже в обох випадках має місце однакова
експлуатація однакового числа робітників. Ця ріжниця
виразу однієї і тієї ж маси додаткової вартості, або ріжниця
норм зиску, а тому й самих зисків, при однаковій експлуатації
праці, може походити і з інших джерел; але вона може також
походити цілком і виключно з ріжниці в діловій вправності, з
якою ведуться обидва підприємства. І ця обставина приводить
капіталіста до ілюзії — переконує його, — що його зиск завдячує
своє існування не експлуатації праці, а, принаймні почасти і
іншим, від цієї експлуатації праці незалежним, обставинам, особливо
його індивідуальній діяльності.

\pfbreak

З викладеного в цьому першому відділі видно помилковість
того погляду (Родбертуса), згідно з яким (відмінно від
земельної ренти, де, наприклад, площа землі лишається незмінною,
в той час як рента зростає) зміна величини капіталу не впливає
на відношення між зиском і капіталом, а тому й на норму
зиску, бо тоді, коли зростає маса зиску, зростає і маса капіталу,
на яку цей зиск обчислюється, і навпаки.

Це правильно тільки в двох випадках. Поперше, тоді, коли
при незмінності всіх інших умов, отже, особливо норми додаткової
вартості, настає зміна вартості товару, який є грошовим
товаром. (Те саме має місце при самій тільки номінальній
зміні вартості, підвищенні чи падінні знаків вартості при інших
однакових умовах). Припустім, що весь капітал = 100 фунтам
стерлінгів, зиск = 20 фунтам стерлінгів, отже, норма зиску = 20\%.
\parbreak{}  %% абзац продовжується на наступній сторінці

\parcont{}  %% абзац починається на попередній сторінці
\index{ii}{0149}  %% посилання на сторінку оригінального видання
товарів, що їх робітник купує на свою заробітну плату, в формі засобів
існування. В цій формі капітальна вартість, витрачена на заробітну плату,
належить, за Смісом, до обігового капіталу. Але те, що вводиться в
продукційний процес, є робоча сила, є сам робітник, а не засоби існування,
що ними робітник підтримує своє життя. А проте, ми бачили
(кн. І, розд. XXI), що, з суспільного погляду, репродукція самого робітника
його особистим споживанням теж належить до процесу репродукції
суспільного капіталу. Але цього не можна сказати про поодинокий,
замкнений в собі продукційний процес, що його ми досліджуємо тут.
Надбані й корисні вмілості, acquired and useful abilities (ст. 187), що їх
Сміс подає під рубрикою основного капіталу, в дійсності становлять
складові частини поточного капіталу, оскільки це є abilities найманого
робітника й оскільки робітник продає свою працю разом з своїми abilities.

Велика помилка Сміса в тому, що він усе суспільне багатство поділяє
на: 1) фонд безпосереднього споживання, 2) основний капітал,
3) обіговий капітал. Згідно з цим усе багатство треба було б розподілити
на 1) фонд споживання, що зовсім не становить частини діющого
суспільного капіталу, хоч поодинокі частини його завжди можуть функціонувати
як капітал, і 2) капітал. Одна частина багатства функціонує
таким чином як капітал, друга частина — як некапітал або як фонд
споживання. І для всякого капіталу тут виявляється неминучість бути
або основним або поточним, так само, як кожен з ссавців неминуче мусить
бути або самцем або самицею. Ми однак бачили, що протилежність
між основним і обіговим капіталом має силу тільки для елементів \so{продуктивного
капіталу}, і що, значить, поряд них є ще дуже значна
маса капіталу — товаровий капітал і грошовий капітал — яка перебуває в
такій формі, що в ній \so{не може} вона бути ні основним, ні поточним
капіталом.

А що за винятком тієї частини продуктів, яку — в натуральній формі —
безпосередньо, без продажу й купівлі, знову вживають самі поодинокі
капіталістичні продуценти як засоби продукції, вся маса продуктів суспільної
продукції — на капіталістичній основі — циркулює на ринку як
товаровий капітал, то очевидно, що з товарового капіталу треба вилучити
так основні й поточні елементи продуктивного капіталу, як і всі
елементи споживного фонду. Фактично це значить, що засоби продукції,
як і засоби споживання, на основі капіталістичної продукції виступають
спочатку як товаровий капітал, хоч вони й мали б призначення в
дальшому служити як засіб споживання або як засоби продукції; так само
навіть робоча сила перебуває на ринку як товар, хоч і не як товаровий
капітал.

Відси така нова плутанина в А.~Сміса. Він каже:

„З цих чотирьох частин (cilculating капіталу, тобто капіталу в його
належних до процесу циркуляції формах товарового капіталу й грошового
капіталу — дві частини, що перетворюються на чотири частини через
те, що Сміс відрізняє складові частини товарового капіталу знову на
основі речових ознак) три: харчові засоби, матеріяли та готові вироби,
\parbreak{}  %% абзац продовжується на наступній сторінці

\parcont{}  %% абзац починається на попередній сторінці
\index{ii}{0150}  %% посилання на сторінку оригінального видання
щороку, або в інші, більш-менш короткі періоди реґулярно вилучається
з нього й приміщується або в основний капітал, або в фонд, призначений
для безпосереднього споживання. Кожний основний капітал первісно
постав з обігового й йому потрібна повсякчасна підтримка від цього
останнього. Всі корисні машини та знаряддя праці первісно постали з
обігового капіталу, який дав матеріяли, що з них їх зроблено, і утримання
робітникам, що їх зробили. Вони потребують також, щоб капітал,
зазначеного виду, підтримував їх завжди справними“\footnote*{
Of these four parts three — provisions, materials, and finished work, are either
annually or in a longer or shorter period, regularly withdrawn from it, and placed
either in the fixed capital, or in the stock reserved for immediate consumption.
Every fixed capital is both originally derived from, and requires to be continually
supported by a circulating capital. All useful machines and instruments of trade are
originally derived from a circulating capital, which furnishes the materials of which
they are made and the maintenance of the workmen who make them. They require,
too, a capital of the same kind to keep them in constant repair“ (p. 188).
}.

З винятком частини продукту, що її продуценти завжди безпосередньо
знову зуживають як засоби продукції, для капіталістичної продукції
має силу таке загальне правило: всі продукти подається як товари на ринок,
вони циркулюють для капіталіста як товарова форма його капіталу, як товаровий
капітал незалежно від того, чи мусять і чи можуть ці продукти
своєю натуральною формою, своєю споживною вартістю, функціонувати як
елементи продуктивного капіталу (продукційного процесу), тобто як засоби
продукції, а тому і як основні або поточні елементи продуктивного капіталу,
або чи можуть вони служити лише як засоби особистого, а не продуктивного
споживання. Всі продукти як товари подається на ринок;
тому всі засоби продукції та споживання, всі елементи продуктивного та
особистого споживання треба знову вилучити з ринку купівлею. Ця тривіяльність
(truism), звичайно, правильна. Отже, це однаково має силу й
щодо основних і щодо поточних елементів продуктивного капіталу, і
для засобів праці і для матеріялів праці в усіх їхніх формах. (При цьому
ще забувають, що елементи продуктивного капіталу дані самою природою,
отже, вони не є продукти). Машину купується на ринку так само,
як і бавовну. Але відси ні в якому разі не випливає, що кожний основний
капітал первісно походить із поточного капіталу — це випливає лише
з Смісового сплутування капіталу циркуляції з обіговим або поточним
капіталом, тобто неосновним капіталом. І до того ж Сміс сам себе збиває.
Машини як товар, за його словами, становлять частину зазначеного
в пункті 4 обігового капіталу. Їхнє походження з обігового капіталу значить,
отже, лише те, що вони функціонували як товаровий капітал раніш,
ніж почали функціонувати як машини, але — що речово вони походять
з самих себе; цілком так само, як бавовна, як поточний елемент капіталу
прядуна, походить з бавовни, що циркулювала на ринку. Але коли
Сміс в дальшому викладі висновує основний капітал з обігового на
тій підставі, що для машинобудівництва потрібні праця й сировинні матеріяли,
то, поперше, для цього потрібні також засоби праці, тобто основний
капітал і, подруге, щоб виготувати сировинні матеріяли, теж
\parbreak{}  %% абзац продовжується на наступній сторінці


  \parcont{}  %% абзац починається на попередній сторінці
\index{ii}{0151}  %% посилання на сторінку оригінального видання
потрібен основний капітал, машини тощо, бо продуктивний капітал завжди
включає засоби праці, але він не завжди включає матеріял праці. Сам
Сміс каже безпосередньо по цьому: „Землі, копальні та рибні промисли
потребують так основного, як і обігового капіталу, щоб їх розробляти“
(отже, він згоджується з тим, що не лише поточний, а й основний капітал
потрібен для продукції сировинного матеріялу, „і“ (тут нова помилка)
„їхній продукт покриває з зиском не лише ці капітали, але й\so{ усі
інші, що є в суспільстві}“\footnote*{
„Lands, mines, and fisheries, require all both a fixed and circulating capital
to cultivate them; and their produce replaces with a profit, not only those capitals,
but \so{all the others in society}“ (p. 188).
}. Це зовсім неправильно. Їхній продукт
дає сировинні матеріяли, допоміжні матеріяли тощо для всіх інших галузей
промисловости. Але їхня вартість не покриває вартости всіх інших
суспільних капіталів; вона покриває лише свою власну капітальну вартість
(\dplus{} додаткова вартість). В цьому в А.~Сміса знову виявляється
вплив фізіократів.

З суспільного погляду правильно, що частина товарового капіталу,
яка складається з продуктів, що можуть служити лише засобами праці,
раніше або пізніше — якщо тільки не спродуковано їх взагалі марно,
якщо вони не лишаються непродані — функціонуватимуть як засоби праці;
інакше кажучи, на основі капіталістичної продукції такі продукти,
переставши бути товарами, справді мусять стати згідно з своїм призначенням
елементами основної частини суспільного продуктивного капіталу.

Тут перед нами ріжниця, що випливає з натуральної форми продукту.

Напр., прядільна машина немає споживної вартости, коли її не вживається
на прядіння, отже, коли вона не функціонує як елемент продукції,
отже, з капіталістичного погляду як основна частина продуктивного
капіталу. Але прядільна машина рухома. Її можна вивезти з країни, де
її випродукувано, і продати в іншій країні в обмін, безпосередньо або
посередньо, на сировинний матеріял тощо або на шампанське. В країні,
де її випродукувано, вона функціонує тоді лише як товаровий капітал,
але зовсім не функціонує — навіть після її продажу — як основний капітал.

Навпаки, продукти, що через прикріплення їх до ґрунту є льокалізовані,
і які, отже, можна використовувати лише на місці, напр., фабричні
будівлі, залізниці, мости, тунелі, доки й~\abbr{т. ін.}, меліорації тощо —
всі такі продукти не можна вивезти матеріяльно, так, як вони є. Вони
нерухомі. Або їх марно спродуковано, або, якщо їх продано, вони мусять
функціонувати як основний капітал, — у тій країні, де їх випродукувано.
Для капіталістичного продуцента, що за для спекуляції, маючи
на меті продаж, будує фабрики або поліпшує ґрунт, ці речі мають форму
його товарового капіталу, отже, за А.~Смісом, форму обігового капіталу.
Але, з суспільного погляду, ці речі, щоб не лишитись некорисними,
мусять, кінець-кінцем, функціонувати у власній країні як основний
капітал у процесі продукції, фіксованому в місці їхнього перебування. Відси
ні в якому разі не випливає, що нерухомі речі, як такі, вже самі
собою є основний капітал; вони, як, напр., житлові будинки тощо,
\parbreak{}  %% абзац продовжується на наступній сторінці

\parcont{}  %% абзац починається на попередній сторінці
\index{iii1}{0152}  %% посилання на сторінку оригінального видання
Хоч і яке велике значення має вивчення таких тертів для кожної
спеціальної праці про заробітну плату, все ж при загальному
дослідженні капіталістичного виробництва їх треба залишити
осторонь як випадкові і неістотні. При такому загальному
дослідженні взагалі завжди припускається, що дійсні відносини
відповідають своєму поняттю, або, що є те саме, дійсні відносини
зображаються лиш остільки, оскільки вони виражають свій
власний загальний тип.

Ріжниця норм додаткової вартості в різних країнах, отже,
національні ріжниці в ступенях експлуатації праці, для даного
дослідження зовсім не мають значення. Адже в цьому відділі ми
саме хочемо показати, яким чином в межах даної країни утворюється
певна загальна норма зиску. Однак, ясно, що при порівнянні
різних національних норм зиску треба тільки зіставити розвинуте
нами раніше з тим, що ми маємо розвинути тут. Спочатку
треба розглянути ріжницю в національних нормах додаткової
вартості, а потім, на основі цих даних норм додаткової вартості,
порівняти ріжницю національних норм зиску. Оскільки ріжниця
цих останніх не є результатом ріжниці національних норм додаткової
вартості, вона мусить виникати з обставин, при яких,
як і в нашому дослідженні в цьому розділі, додаткова вартість
припускається повсюди однаковою, постійною.

В попередньому розділі було показано, що, коли норму додаткової
вартості припустити незмінною, норма зиску, яку дає певний
капітал, може підвищуватись чи падати в наслідок обставин,
які підвищують або знижують вартість тієї чи іншої частини
сталого капіталу і тому взагалі впливають на відношення між
сталою і змінною складовими частинами капіталу. Далі було
відзначено, що обставини, які здовжують або скорочують час
обороту капіталу, можуть справляти аналогічний вплив на
норму зиску. Через те що маса зиску тотожна з масою додаткової
вартості, з самою додатковою вартістю, то виявилось також,
що \emph{маса} зиску — відмінно від \emph{норми} зиску — не зачіпається
щойно згаданими коливаннями вартості. Вони модифікують тільки
норму, в якій виражається дана додаткова вартість, отже й зиск
даної величини, тобто модифікують його відносну величину,
його величину порівняно з величиною авансованого капіталу.
Оскільки в наслідок таких коливань вартості відбувається зв’язування
або звільнення капіталу, таким посереднім шляхом може
бути зачеплена не тільки норма зиску, але й самий зиск. Однак,
це завжди має силу тільки для капіталу, уже вкладеного, а не
для нових капіталовкладень; і, крім того, збільшення або зменшення
самого зиску завжди залежить від того, наскільки більше
чи менше праці в наслідок таких коливань вартості може бути
приведено в рух тим самим капіталом, отже, від того, наскільки
більшу чи меншу масу додаткової вартості може — при незмінній
нормі додаткової вартості — виробити той самий капітал.
Аж ніяк не суперечачи загальному законові і не становлячи винятку
\index{iii1}{0153}  %% посилання на сторінку оригінального видання
з нього, цей позірний виняток в дійсності був тільки
окремим випадком застосування загального закону.

Якщо в попередньому відділі виявилось, що, при незмінному
ступені експлуатації праці, із зміною вартості складових частин
сталого капіталу, а також із зміною часу обороту капіталу змінюється
норма зиску, то з цього само собою випливає, що
норми зиску різних одночасно існуючих, одна поряд одної, сфер
виробництва будуть різні, якщо при інших незмінних умовах
час обороту застосовуваних капіталів різний або якщо вартісне
відношення між органічними складовими частинами цих капіталів
у різних галузях виробництва є різне. Те, що ми раніш
розглядали як зміни, що відбуваються послідовно в часі
з тим самим капіталом, ми розглядаємо тепер як одночасно
наявні ріжниці між існуючими одно поряд одного капіталовкладеннями
в різних сферах виробництва.

При цьому нам доведеться дослідити: 1) ріжницю в \emph{органічному
складі} капіталів, 2) ріжницю в часі їх обороту.

В усьому цьому дослідженні, коли ми говоримо про склад
або оборот капіталу в певній галузі виробництва, ми завжди
маємо на увазі — припущення, яке само собою зрозуміле, — пересічні
нормальні відношення капіталу, вкладеного в цю галузь
виробництва; взагалі, мова йде про пересічні відношення сукупного
капіталу, вкладеного в дану сферу, а не про випадкові
ріжниці між окремими вкладеними в цю сферу капіталами.

Через те що, далі, припускається, що норма додаткової вартості
і робочий день є незмінні, і через те що це припущення
включає також і незмінність заробітної плати, то певна кількість
змінного капіталу виражає певну кількість приведеної
в рух робочої сили, а тому й певну кількість праці, яка упредметнюється.
Отже, якщо 100 фунтів стерлінгів виражають тижневу
заробітну плату 100 робітників, тобто в дійсності 100 робочих
сил, то 100 фунтів стерлінгів $×n $виражають тижневу
заробітну плату $100 × n$ робітників, а $\frac{100 фунтів стерлінгів}{n}$ тижневу
заробітну плату $\frac{100}{n}$ робітників. Отже, змінний капітал служить
тут (як і завжди при даній величині заробітної плати) показником
маси праці, яку приводить в рух весь капітал певної величини;
тому ріжниці у величині застосовуваного змінного капіталу
служать показниками ріжниці в масі вживаної робочої сили. Якщо
100 фунтів стерлінгів представляють 100 робітників на тиждень
і, отже, при 60 годинах тижневої праці — 6000 робочих годин, то
200 фунтів стерлінгів представляють 12 000 робочих годин, а 50
фунтів стерлінгів тільки 3000 робочих годин.

Під складом капіталу ми розуміємо, як це сказано вже у
книзі першій, відношення між його активною і пасивною складовою
частиною, між змінним і сталим капіталом. При цьому
треба розглянути два відношення, які мають неоднакову важливість,
\index{iii1}{0154}  %% посилання на сторінку оригінального видання
хоч при певних обставинах можуть спричиняти однаковий
вплив.

Перше відношення грунтується на технічній базі, і на певному
ступені розвитку продуктивної сили його треба розглядати
як дане. Потрібна певна маса робочої сили, представлена
певним числом робітників, щоб виробити певну масу продукту,
наприклад, протягом одного дня, і, отже — що при цьому само
собою розуміється — привести в рух, продуктивно спожити
певну масу засобів виробництва, машин, сировинних матеріалів
і т. д. Певне число робітників припадає на певну кількість засобів
виробництва, отже певна кількість живої праці припадає
на певну кількість праці, вже упредметненої в засобах виробництва.
Це відношення дуже різне в різних сферах виробництва,
часто в різних галузях однієї й тієї ж промисловості, хоч, з другого
боку, випадково воно може бути цілком або приблизно
однаковим в дуже віддалених одна від одної галузях промисловості.
Це відношення становить технічний склад капіталу і є дійсна
основа його органічного складу.

Але можливо також, що це відношення однакове в різних
галузях промисловості, оскільки змінний капітал є простий показник
робочої сили, а сталий капітал — простий показник маси
засобів виробництва, приведеної в рух цією робочою силою.
Наприклад, певні роботи з міддю й залізом можуть вимагати
однакового відношення між робочою силою і масою засобів
виробництва. Але через те що мідь дорожча, ніж залізо, то вартісне
відношення між змінним і сталим капіталом в обох випадках
буде різне і разом з тим буде різний і вартісний склад
обох цілих капіталів. Ріжниця між технічним складом і вартісним
складом виявляється в кожній галузі промисловості
в тому, що при незмінному технічному складі вартісне відношення
обох частин капіталу може змінюватись, а при зміні
технічного складу вартісне відношення може лишатись незмінним;
останнє має місце, звичайно, тільки тоді, коли зміна
відношення між застосованою масою засобів виробництва і масою
робочої сили вирівнюється протилежною зміною їх вартостей.

Вартісний склад капіталу, оскільки він визначається його
технічним складом і відображає цей останній, ми звемо \emph{органічним}
складом капіталу.\footnote{
Вищевикладене коротко було розвинуте уже в третьому виданні першої
книги, стор. 628. на початку розділу XXIII. Через
те що в перших двох виданнях немає цього місця, повторення його тут мало
тим більше підстав. — \emph{Ф. Е.}
}

Отже, щодо змінного капіталу ми припускаємо, що він є показник
певної кількості робочої сили, певного числа робітників
або певної маси приводжуваної в рух живої праці. В попередньому
\index{iii1}{0155}  %% посилання на сторінку оригінального видання
відділі ми бачили, що зміна величини вартості змінного
капіталу іноді виражає не що інше, як більшу або меншу ціну
тієї самої маси праці; але тут, де норма додаткової вартості
і робочий день розглядаються як незмінні, а заробітна плата
за певний робочий час як величина дана, це відпадає. Навпаки,
ріжниця у величині сталого капіталу може, правда, бути також
показником зміни маси засобів виробництва, приведених в рух
певною кількістю робочої сили; але вона може також походити
з ріжниці у вартості засобів виробництва, приведених в рух
у певній сфері виробництва, порівняно з іншими сферами. Тим
то тут треба взяти до уваги обидві ці точки зору.

Нарешті, треба зробити ще таке істотне зауваження:

Припустім, що 100 фунтів стерлінгів становлять тижневу
заробітну плату 100 робітників. Припустім, що тижневий робочий час дорівнює 60 годинам. Припустімо,
далі, що норма
додаткової вартості = 100\%. В цьому випадку робітники 30 годин з 60 працюють на себе самих, а 30
даром на капіталіста.
В 100 фунтах стерлінгів заробітної плати в дійсності втілено
тільки 30 робочих годин 100 робітників, або разом 3000 робочих годин, тимчасом як інші 3000 годин,
які вони працюють,
втілені в 100 фунтах стерлінгів додаткової вартості, відповідно — зиску, що його забирає собі
капіталіст. Тому, хоч заробітна плата в 100 фунтів стерлінгів не виражає тієї вартості,
в якій упредметнюється тижнева праця 100 робітників, вона
все ж показує (бо довжина робочого дня і норма додаткової
вартості є дані), що цим капіталом приведено в рух 100 робітників на протязі загалом 6000 робочих
годин. Капітал в 100 фунтів стерлінгів показує це, тому що він, по-перше, показує
число приведених в рух робітників, бо 1 фунт стерлінгів = 1 робітникові за тиждень, отже 100 фунтів
стерлінгів = 100 робітникам; і, по-друге, тому що кожний приведений
в рух робітник, при даній нормі додаткової вартості в 100\%,
виконує вдвоє більше праці, ніж міститься в його заробітній
платі, отже, 1 фунт стерлінгів, його заробітна плата, що є виразом півтижневої праці, приводить в
рух працю цілого тижня,
і так само 100 фунтів стерлінгів, хоч вони містять в собі тільки 50
тижнів праці, приводять в рух працю 100 робочих тижнів. Отже,
тут треба мати на увазі дуже істотну ріжницю між змінним капіталом, витраченим на заробітну плату,
оскільки його вартість,
сума заробітних плат, представляє певну кількість упредметненої праці, і цим капіталом, оскільки
його вартість є простий показник маси живої праці, яку він приводить в рух. Ця
остання завжди більша, ніж кількість праці, яка міститься
в змінному капіталі, і тому вона виражається також у вартості
більшій, ніж вартість змінного капіталу — у вартості, яка визначається, з одного боку, числом
приведених в рух змінним капіталом робітників, а з другого боку, кількістю виконуваної ними
додаткової праці.


\index{iii2}{0156}  %% посилання на сторінку оригінального видання
Загальна сума грошової ренти становила б якраз половину того, що було
в таблиці II, де додаткові капітали були вкладені за незмінних цін продукції.

Найважливіше є порівняти вищенаведені таблиці з таблицею І.

Ми бачимо, що з пониженням ціни продукції на половину, з 60 шил. до
30 шил. за квартер, загальна сума грошової ренти залишилась та сама = 18 ф.
ст. і відповідно до цього збіжжева рента подвоїлась, саме зросла з 6 кварт. до
12 кварт. Рента з В відпала; з С грошова рента в ІVс збільшилась на половину,
але на половину зменшилась в ІVс; з D вона лишилась та сама = 9 ф.
стерл. у таблиці ІVс, і піднеслась з 9 ф. стерл. до 15 ф. стерл. у таблиції ІVd.
Продукц я піднеслась з 10 квартерів до 34 в ІVс, і до 30 квартер в в IVd;
зиск підвищився з 2 ф. стерл. до 5\sfrac{1}{2} в ІVс і до 4 \sfrac{1}{2} в IVd. Загальна сума
вкладеного капіталу зросла в одному випадку з 10 ф. стерл. до 27\sfrac{1}{2} ф. стерл.,
в другому — з 10 до 22\sfrac{1}{2} ф. стерл.; отже, обидва рази більше, ніж удвоє. Норма
ренти, рента, обчислена у відношенні до авансованого капіталу, в усіх таблицях
від IV до IVd для кожного роду землі всюди та сама, що вже було дано тим припущенням,
що норма продуктивности обох послідовних витрат капіталу на землях
усіх родів не змінюється. Проти таблиці І вона, проте, понизилась пересічно
щодо всіх родів землі і для кожного окремого роду землі. В таблиці І вона =
180\% пересічно, в таблиці ІVс вона$ = \frac{18}{27\sfrac{1}{2}} × 100 = 65\sfrac{5}{11}\%$ і
IVd = $\frac{18}{22\sfrac{1}{2}} × 100 = 80\%$. Пересічна грошова рента з акра підвищилась. Її пересічна
величина давніш в таблиці І була 4\sfrac{1}{2} ф. стерл. з акра для всіх 4 акрів,
а тепер у таблицях IVс і d вона дорівнює 6 ф. стерл. з акра для 3 акрів.
Її пересічна величина для землі, що дає ренту, була раніш 6 ф. стерл., а тепер
зона дорівнює 9 ф. стерл. з акра. Отже, грошова вартість ренти з акра підвищилась
і репрезентує тепер удвоє більше продукту в збіжжі, ніж давніш, але
12 квартерів збіжжевої ренти тепер становлять менше, ніж половину всього продукту
в 34, зглядно 30\footnote*{В німецькому тексті стоїть: усього «продукту в 33, зглядно 27 квартерів» Явна помилка,
як це можна бачити з таблиць ІVс і IVd. \emph{Прим. Ред.}} квартерів, тимчасом як у таблиці І 6 квартерів становлять
\sfrac{3}{5}  усього продукту в 10 квартерів. Отже, хоч рента, коли розглядати
її як відповідну частину всього продукту, а також коли обчислити її у відношенні
до витраченого капіталу, і знизилась, одначе її грошова вартість,
обчислена на акр. збільшилась, а її вартість в продукті, збільшилась ще дужче.
Коли ми візьмемо землю D в таблиці IVd, то ціна продукції тут дорівнює
15 ф. стерл., що з них витрачений капітал = 12\sfrac{1}{2} ф. стерл. Грошова рента = 15
ф. стер. У таблиці І на тій самій землі D ціна продукції була 3 ф. стерл., витрачений
капітал = 2\sfrac{1}{2} ф. стерл., грошова рента = 9 ф. стерл., отже, остання
утроє більша за ціну продукції й майже у чотири рази більша за витрачений
капітал. У таблиці IVd для D грошова рента в 15 ф. стерл. якраз дорівнює ціні
продукції і лише на \sfrac{1}{5}  більша за витрачений капітал. А все ж грошова рента
з акра на \sfrac{2}{3}  більша, 15 ф. стерл. замість 9 ф. стерл. В таблиці І збіжжева
рента в 3 квартери = \sfrac{3}{4}  усього продукту, що становить 4 квартери, в таблиці
IVd вона = 10 квартерам, половині всього продукту (20 квартерів) з акра
землі D. Це показує, що грошова і збіжжева рента з акра може зрости, хоч
вона і становить відносно меншу частину всього здобутку і знизилась у відношенні
до авансованого капіталу.

Вартість всього продукту в таблиці І = 30 ф. стерл.; рента = 18 ф.
стерл. більше від половини цієї вартости. Вартість усього продукту в таблиці
IV = 45 ф. стерл., що з них 18 ф. стерл., менш від половини, становлять
ренту.

\input{_0157.tex}
\parcont{}  %% абзац починається на попередній сторінці
\index{ii}{0158}  %% посилання на сторінку оригінального видання
протиставиться другій складовій частині сталого капіталу, витраченій на
засоби праці. Додаткова вартість, отже, саме та обставина, що перетворює
витрачену суму вартости на капітал, лишається при цьому цілком
поза розглядом. Так само поза розглядом лишається й те, що частину
вартости, яку долучає до продукту витрачений на заробітну плату капітал,
випродукувано знову (тобто справді репродуковано), тимчасом як
частину вартости, що її долучає до продукту сировинний матеріял, не
випродукувано знову, не репродуковано в дійсності, а лише збережено
в вартості продукту, консервовано, і тому вона лише знову з’являється як
складова частина вартости продукту. Ріжниця, як вона виявляється тепер
з погляду протилежности між поточним і основним капіталом, сходить
лише ось на що: вартість засобів праці, вжитих для продукції товару,
лише частинами входить у вартість товару, а тому й лише частинами
покривається через продаж товарів, а значить, і взагалі покривається вона
тільки частинами й поступінно. З другого боку, вартість робочої
сили та предметів праці (сировинні матеріяли тощо), вжитих для
продукції товару, цілком увіходить у товар і тому цілком покривається
через продаж його. В цьому розумінні, отже, щодо процесу
циркуляції одна частина капіталу виступає як основний, а друга як поточний
або обіговий капітал. В обох випадках ідеться про перенесення
даної, авансованої вартости на продукт і про покриття її через продаж
продукту. Ріжниця тут лише в тому, як відбувається це перенесення вартости,
а, значить, і покриття вартости: чи частинами й поступінно, чи
одразу одним заходом. Цим самим затушковується найвирішальнішу ріжницю
між змінним і сталим капіталом, отже, затушковується всю таємницю
утворення додаткової вартости і всю таємницю капіталістичної продукції,
затушковується обставини, що перетворюють на капітал певні вартості
й речі, що в них ці вартості втілюються. Всі складові частини капіталу
відрізняються тут тільки способом циркуляції (а циркуляція товару,
звичайно, має чинення тільки до наявних уже, даних вартостей); але особливий
спосіб циркуляції капіталу, витраченого на заробітну плату, спільний і
частині капіталу, витраченій на сировинні матеріяли, напівфабрикати,
допоміжні матеріяли, протилежно до частини капіталу, витраченої на засоби
праці.

Відси зрозуміло, чому буржуазна політична економія інстинктивно
зберігала Смісову плутанину категорій „сталого й змінного капіталу“
з категоріями „основного й обігового капіталу“ і без будь-якої критики
протягом цілого століття передавала цю плутанину з покоління в покоління.
На її погляд, витрачена на заробітну плату частина капіталу зовсім
уже не відрізняється від частини капіталу, витраченої на сировинний
матеріял, і відрізняється лише формально — лише тим, чи циркулює вона
разом з продуктом частинами, чи цілком — від сталого капіталу. Цим
самим одним ударом руйнується основи, потрібні для того, щоб зрозуміти
справжній рух капіталістичної продукції, а, значить, і капіталістичної
експлуатації. Для неї справа сходить лише на відновлення авансованих
вартостей.


\index{ii}{0159}  %% посилання на сторінку оригінального видання
Ця некритично запозичена в А. Сміса плутанина заважає Рікардо
не тільки більше, ніж пізнішим апологетам — останнім плутанина понять
не тільки не заважає, а скорше допомагає — а й більше, ніж самому
А. Смісові, бо Рікардо, дотримуючись на ділі езотеричного вчення
А. Сміса проти екзотеричного А. Сміса, протилежно до нього, послідовніше
і гостріше розвинув вчення про вартість і додаткову вартість.

У фізіократів немає й сліду цієї плутанини. Ріжниця між avances
annuelles і avances primitives стосується лише до різних періодів репродукції
різних складових частин капіталу, спеціально хліборобського капіталу,
тимчасом як їхні погляди на продукцію додаткової вартости становлять
незалежну від цих ріжниць частину їхньої теорії, а саме частину,
що її вони виставляють як основу теорії. Утворення додаткової вартости
пояснюється в них не з капіталу, як такого, а визнається як властивість
лише певної продукційної сфери капіталу — хліборобства.

2) Найпосутніше для визначення змінного капіталу — а тому й для
перетворення будь-якої суми вартости на капітал — в тому, що капіталіст
обмінює певну, дану (і в цьому розумінні сталу) величину вартости на
силу, яка творить вартість; певну кількість вартости обмінюється на продукцію
вартости, на процес її самозростання. Чи платить капіталіст робітникові
грішми, чи засобами існування, — це нічого не змінює в цьому
найпосутнішому визначенні. Від цього змінюється тільки спосіб існування
авансованої капіталістом вартости, яка в одному разі існує у формі грошей,
що на них робітник сам собі купує на ринку засоби свого існування,
а в другому разі — у формі засобів існування, що їх робітник
споживає безпосередньо. На ділі розвинена капіталістична продукція
припускає, що робітника оплачується грішми, як вона взагалі має собі
за передумову процес продукції, упосереднюваний процесом циркуляції,
тобто має за передумову грошове господарство. Але творення додаткової
вартости — і, значить, капіталізація авансованої суми вартости — не випливає
ні з грошової, ні з натуральної форми заробітної плати, або капіталу,
витраченого на закуп робочої сили. Воно випливає з обміну вартости
на вартостетворчу силу, — з перетворення сталої величини на змінну.

Більша або менша закріпленість засобів праці залежить від ступеня
їхньої довготривалости, тобто від фізичної властивости. Залежно від
ступеня довготривалости вони, за інших незмінних умов, зношуються
швидше або повільніше, отже, функціонують як основний капітал довший
або коротший час. Але вони функціонують як основний капітал зовсім
не в наслідок самої цієї, фізичної властивости — довготривалости. Сировинний
матеріял на металевих фабриках так само довготривалий, як і машини,
що його обробляють, і довготриваліший, ніж деякі складові частини цих
машин: шкіра, дерево тощо. А проте, металь, що служить як сировинний
матеріял, становить частину обігового капіталу, а засіб праці, що функціонує,
зроблений, може, з того самого металю, становить частину основного
капіталу. Отже, не в наслідок фізичної природи речовини, не
в наслідок більшої або меншої незнищуваности той самий металь одного
разу заводиться під рубрику основного, а другого — під рубрику обігового
\parbreak{}  %% абзац продовжується на наступній сторінці

\input{_0160.tex}
\parcont{}  %% абзац починається на попередній сторінці
\index{ii}{0161}  %% посилання на сторінку оригінального видання
капіталові як сталий капітал — це з погляду процесу зростання вартости.
Або, коли тут мова повинна бути про речову ріжницю, оскільки вона
впливає на процес циркуляції, то справа така: з природи вартости, яка є
не що інше, як зречевлена праця, і з природи діющої робочої сили, яка
є не що інше, як праця, що зречевлює себе, випливає, що робоча сила
протягом періоду її функціонування постійно утворює вартість і долярову
вартість; і що те, що на боці робочої сили виявляється як рух, як
утворення вартости, на боці її продукту виявляється у формі спокою,
як уже утворена вартість. Коли робоча сила вже діяла, то капітал не
складається вже більше з робочої сили на одному боці, із засобів продукції
на другому. Капітальна вартість, витрачена на робочу силу, є тепер
вартість, що її (\dplus{} додаткову вартість) долучено до продукту. Щоб
повторити процес, треба продати продукт і на вторговані гроші знову й
знову купувати робочу силу і вводити її в продуктивний капітал. Це
надає тоді частині капіталу, витраченій на робочу силу, так само, як і частинам
його, витраченим на матеріял праці тощо, характер обігового капіталу,
протилежно до того капіталу, що лишається закріплений у засобах праці.

Коли, навпаки, другорядне визначення обігового капіталу, спільне
йому з частиною сталого капіталу (сировинними й допоміжними матеріялами)
— саме те визначення, що вартість, витрачену на обіговий капітал,
цілком переноситься на продукт, в продукції якого його зуживається, а
не поступінно й частинами, як в основного капіталу, що вартість ця,
отже, мусить цілком заміститися через продаж продукту, — перетворити
на посутню характеристику частини капіталу, витраченої на робочу силу,
то й частина капіталу, витрачена на заробітну плату, речово мусить
складатися не з діющої робочої сили, а з речових елементів, що їх робітник
купує на свою плату, отже, з частини суспільного товарового капіталу,
яка ввіходить у споживання робітника — з засобів існування.
Основний капітал складається при такому погляді на справу з засобів
праці, що зношуються повільніше, а тому й доводиться їх рідше відновлювати,
а капітал, витрачений на робочу силу, з засобів існування, що
їх треба заміщувати швидше.

Однак межі швидшої та повільнішої зношуваности стираються.

„Харч і одяг що їх зуживає робітник, будівлі, де він працює, знаряддя,
що допомагають йому в роботі, всі ці речі з своєї природи минущі.
Але є величезна ріжниця в часі, що протягом його зберігаються
ці різні капітали: парова машина зберігається довший час, ніж корабель,
корабель — довший час, ніж одяг робітника, одяг робітника знову таки
довший час, ніж харч, що його він споживає“\footnote{
„The food and clothing consumed the labourer, the buildings in which he
works, the implements with which his labour is assisted, are all of a perishable
nature. There is, however, a vast difference in the time for which these different
capitals will endure: a steam-engine will last longer than a ship, a ship than the
clothing of the labourer, and the clothing of the labourer longer than the food which
he consumes“. (Ricardo, etc., p. 27).
}.

\parcont{}  %% абзац починається на попередній сторінці
\index{iii1}{0162}  %% посилання на сторінку оригінального видання
сферах однакові при рівновеликих затратах капіталу, як би не
відрізнялись між собою вироблені вартості і додаткові вартості.
Ця рівність витрат виробництва становить базу конкуренції капіталовкладень, за допомогою якої
встановлюється пересічний зиск.

\section{Утворення загальної норми зиску (пересічної норми зиску) і перетворення вартостей товарів у ціни
виробництва}

Органічний склад капіталу в кожний даний момент залежить
від двох обставин: по-перше, від технічного відношення між уживаною робочою силою і масою
застосовуваних засобів виробництва; подруге, від ціни цих засобів виробництва. Його, як ми бачили,
треба розглядати в його процентному відношенні. Органічний склад капіталу, який складається на \sfrac{4}{5} з
сталого і на \sfrac{1}{5} з змінного капіталу, ми виражаємо формулою $80 c + 20 v$. Далі, при
порівнянні припускається незмінна і при тому довільна норма
додаткової вартості, наприклад, в 100\%. Отже, капітал в $80 c + 20 v$ дає додаткову вартість в $20 m$,
що становить норму зиску
в 20\% на весь капітал. Величина дійсної вартості його продукту
залежить від величини основної частини сталого капіталу і від
того, скільки з неї входить в продукт як зношування і скільки
не входить. Але через те що ця обставина зовсім не має значення для норми зиску, отже, і для даного
дослідження, то для
спрощення ми припускаємо, що сталий капітал повсюди однаково цілком входить у річний продукт цих
капіталів. Ми припускаємо далі, що капітали в різних сферах виробництва у
відношенні до величини їх змінної частини реалізують щорічно
однакову кількість додаткової вартості; отже, ми покищо залишаємо осторонь ту ріжницю, яку тут може
викликати ріжниця
в періодах обороту. Цей пункт розглядається пізніше.

Візьмімо п’ять різних сфер виробництва з різним органічним
складом вкладених у них капіталів, наприклад:
  \begin{center}
  \begin{tabular}{c c c c c}
  \toprule
Капітали  &  \makecell{Норма\\додаткової\\вартості}  & \makecell{Додаткова \\вартість} & \makecell{Вартість\\продукту} & \makecell{Норма\\зиску}\\
  \midrule
\phantom{II}І. $80 c + 20 v$          &  100\% &   20            &  120 &   20\% \\
\phantom{I}II. $70 c + 30 v$          &  100\% &   30            &  130 &   30\% \\
III. $60 c + 40 v$                    &  100\% &   40            &  140 &   40\% \\
IV. $85 c + 15 v$                     &  100\% &   15            &  115 &   15\% \\
\phantom{I}V. $95 c + \phantom{0}5 v$ &  100\% &   \phantom{0}5  &  105 &   \phantom{0}5\% \\
  \end{tabular}
  \end{center}

Ми маємо тут для різних сфер виробництва при однаковій
експлуатації праці дуже різні норми зиску, відповідно до різного органічного складу капіталів.

Сукупна сума капіталів, вкладених у цих п’яти сферах, $= 500$;
сукупна сума виробленої ними додаткової вартості $= 110$; сукупна вартість вироблених ними товарів $=
610$. Якщо ми розглядатимем 500 як один єдиний капітал, що з нього І — V становлять
\parbreak{}  %% абзац продовжується на наступній сторінці

\parcont{}  %% абзац починається на попередній сторінці
\index{iii1}{0163}  %% посилання на сторінку оригінального видання
тільки окремі частини (як, наприклад, на якійсь бавовняній фабриці, в різних відділах якої —
кардувальному, підготовчому, прядільному й ткацькому — існує різне відношення між змінним і
сталим капіталом і де пересічне відношення для всієї фабрики
ще тільки має бути обчислене), то, по-перше, пересічний склад
капіталу в 500 був би $= 390 c + 110 v$, або в процентах $78 c + 22 v$.
Кожний з цих капіталів в 100, розглядуваний тільки як \sfrac{1}{5} сукупного капіталу, мав би своїм складом цей
пересічний склад
в $78 c + 22 v$; так само на кожні 100 припадало б 22 як пересічна
додаткова вартість; тому пересічна норма зиску була б $=$ 22\%,
і, нарешті, ціна кожної п’ятої частини сукупного продукту, виробленого цими 500, дорівнювала б 122.
Отже, продукт кожної
п’ятої частини сукупного авансованого капіталу мусив би продаватись за 122.

Однак, щоб не прийти до цілком хибних висновків, не слід
усі витрати виробництва рахувати рівними 100.

При $80 c + 20 v$ і нормі додаткової вартості = 100\% вся вартість товару, виробленого капіталом І =
100, була б $= 80 c + 20 v + 20 m = 120$, коли б весь сталий капітал входив у річний продукт. При
певних обставинах це, звичайно, може мати місце
в певних сферах виробництва. Однак, ледве чи це можливе там,
де відношення $c : v = 4 : 1$. Отже, при дослідженні вартостей товарів, вироблюваних кожними 100
одиницями різних капіталів,
треба взяти до уваги те, що ці вартості можуть бути різні, залежно від різного складу $c$ з основних і
обігових складових
частин, і що основні складові частини різних капіталів, в свою
чергу, зношуються швидше або повільніше, отже, за однакові
періоди часу додають до продукту неоднакові кількості вартості. Але для норми зиску це не має
значення. Чи $80 c$ віддають
річному продуктові вартість в 80, чи в 50, чи в 5, отже, чи річний
продукт $= 80 c + 20 v + 20 m = 120$, чи $= 50 c + 20 v + 20 m = 90$, чи $= 5 c + 20 v + 20 m = 45$, — в
усіх цих випадках надлишок
вартості продукту понад його витрати виробництва = 20, і в усіх
цих випадках при встановленні норми зиску ці 20 обчислюються
на капітал в 100; отже, норма зиску для капіталу І в усіх випадках $=$ 20\%. Щоб зробити це ще яснішим,
ми в нижченаведеній
таблиці припускаємо для тих самих п’яти капіталів, про які мова
йшла вище, що у вартість продукту з цих п’яти капіталів входять різні частини сталого капіталу.
\begin{footnotesize}
\footnotesize
\begin{tabular}{c@{ } c@{ } c@{ } c@{ } c@{ } c@{ } c@{ } c@{ } }
\toprule
\multicolumn{2}{c}{Капітали} &
\makecell{Норма\\додаткової\\вартості} &
\makecell{Додаткова\\вартість} &
\makecell{Норма\\зиску} &
\makecell{Зношування\\$c$} &
\makecell{Вартість\\товарів} &
\makecell{Витрати\\виробництва} \\
\midrule
І.        & $\phantom{0}80 c + \phantom{0}20 v$ & 100\%  &  \phantom{0}20   & 20\%           & 50 & \phantom{0}90  & 70  \\
II.       & $\phantom{0}70 c + \phantom{0}30 v$ & 100\%  &  \phantom{0}30   & 30\%           & 51 & 111 & 81  \\
III.      & $\phantom{0}60 c + \phantom{0}40 v$ & 100\%  &  \phantom{0}40   & 40\%           & 51 & 131 & 91  \\
IV.       & $\phantom{0}85 c + \phantom{0}15 v$ & 100\%  &  \phantom{0}15   & 15\%           & 40 & \phantom{0}70  & 55  \\
V.        & $\phantom{0}95 c + \phantom{00}5 v$ & 100\%  &  \phantom{00}5   & \phantom{0}5\% & 10 & \phantom{0}20  & 15  \\
Сума      & $390 c + 110 v $                    & \textemdash  &  110             &  \textemdash   & \textemdash & \textemdash & \textemdash \\
Пересічно & $\phantom{0}78 c + \phantom{0}22 v$ & \textemdash &  \phantom{0}22   &  22\%          & \textemdash & \textemdash & \textemdash \\
\end{tabular}
\end{footnotesize}


\index{iii1}{0164}  %% посилання на сторінку оригінального видання
Якщо ми капітали І — V знову розглядатимемо як єдиний сукупний капітал, то побачимо, що і в цьому
випадку склад суми
п’яти капіталів = 500 = 390 c + 110 v, отже, пересічний склад, = 78 c + 22 v, лишається той самий;
так само й пересічна додаткова вартість = 22\%. Розподіливши цю додаткову вартість рівномірно між
І—V, ми одержали б такі товарні ціни:

Капітали
Додаткова вартість
Вартість товарів
Витрати виробництва
Ціна товарів
Норма зиску
Відхилення ціни від вартості

І. 80 c + 20 v    20    90    70    92    22\%    + 2
II. 70 c + 30 v   30   111   81   103   22\%  — 8
III. 60 c + 40 v  40   131   91   113   22\% — 18
IV. 85 c + 15 v   15    70    55    77    22\%    + 7
V. 95 c + 5 v        5     20    15    37    22\%  + 17

В загальній сумі товари продаються на 2 + 7 + 17 = 26 вище і
на 8 + 18 = 26 нижче вартості, так що відхилення цін взаємно
знищуються в наслідок рівномірного розподілу додаткової вартості, тобто в наслідок додання
пересічного зиску в 22 на
100 одиниць авансованого капіталу до відповідних витрат виробництва товарів І—V; в тому самому
відношенні, в якому одна
частина товарів продається вище, друга продається нижче її
вартості. І тільки продаж їх по таких цінах уможливлює те, що
норма зиску для І—V є однакова, 22\%, не зважаючи на різний
органічний склад капіталів І—V. Ціни, які виникають таким чином, що з різних норм зиску різних сфер
виробництва береться
пересічна і ця пересічна додається до витрат виробництва в різних сферах виробництва, — такі ціни є
ціни виробництва. Передумовою їх є існування однієї загальної норми зиску, а ця
остання знов таки передбачає, що норми зиску в кожній окремій сфері виробництва, взяті самі по собі,
вже зведені до
відповідної кількості пересічних норм. Ці окремі норми зиску в кожній сфері виробництва = m/K, і їх
треба виводити з вартості товару, як це і було зроблено в першому відділі цієї книги. Без такого
виведення загальна норма зиску (а тому й ціна виробництва товару) була б безглуздим і ірраціональним
уявленням. Отже, ціна виробництва товару дорівнює витратам його
виробництва плюс доданий до них зиск, обчислений у процентах
відповідно до загальної норми зиску, тобто дорівнює витратам
виробництва товару плюс пересічний зиск.

В наслідок різного органічного складу капіталів, вкладених
у різні галузі виробництва, а тому в наслідок тієї обставини,
що — залежно від різного процентного відношення змінної частини до всього капіталу даної величини —
рівновеликими капіталами приводяться в рух дуже різні кількості праці, ними привласнюються також
дуже різні кількості додаткової праці, або
виробляються дуже різні маси додаткової вартості. Відповідно
до цього норми зиску, які панують в різних галузях виробництва,
\index{iii1}{0165}  %% посилання на сторінку оригінального видання
первісно є дуже різні. Ці різні норми зиску за допомогою конкуренції вирівнюються в загальну
норму зиску, яка
є пересічною всіх цих різних норм зиску. Зиск, який відповідно
до цієї загальної норми зиску припадає на капітал даної величини, який би не був його органічний
склад, зветься пересічним
зиском. Ціна товару, яка дорівнює витратам його виробництва
плюс та частина річного пересічного зиску на застосований для
виробництва товару (а не тільки на спожитий для його виробництва) капітал, яка припадає на товар
залежно від умов його
обороту, є його ціна виробництва. Візьмімо, наприклад, капітал
в 500, в тому числі 100 основного капіталу, з якого зношується
10\% протягом одного періоду обороту обігового капіталу в 400.
Припустімо, що пересічний зиск протягом цього періоду обороту становить 10\%. Тоді витрати
виробництва виготовленого
протягом цього обороту продукту будуть: 10 с на зношування
плюс 400 (c + v) обігового капіталу = 410, а його ціна виробництва: 410 витрати виробництва плюс
(10\% зиску на 500) 50 = 460.

Тому, хоч капіталісти різних сфер виробництва при продажу
своїх товарів повертають собі капітальні вартості, спожиті на
виробництво цих товарів, але реалізують вони не ту додаткову
вартість, отже, і не той зиск, що виробляється в їх власній
сфері при виробництві цих товарів, а лише стільки додаткової вартості, отже й зиску, скільки при
рівному розподілі
припадає на кожну відповідну частину всього капіталу суспільства з усієї додаткової вартості або
всього зиску, який виробляється сукупним капіталом суспільства за даний період часу
в усіх сферах виробництва, взятих разом. Кожен авансований
капітал, який би не був його склад, одержує кожного року або
за якийсь інший період часу стільки зиску на кожні 100, скільки
його за цей період часу припадає на кожні 100 як певну частину
сукупного капіталу. Оскільки справа стосується зиску, різні капіталісти відносяться тут один до
одного, як прості акціонери
одного акційного товариства, в якому зиск розподіляється між
ними рівномірно на кожні 100 одиниць, і тому зиски для різних
капіталістів відрізняються тільки залежно від величини капіталу,
вкладеного кожним з них у спільне підприємство, залежно від
відносного розміру участі кожного в спільному підприємстві,
залежно від числа акцій кожного з них. Отже, тимчасом як та
частина цієї товарної ціни, яка заміщає спожиті на виробництво
товарів частини вартості капіталу і за яку, отже, знову мусять
бути куплені ці спожиті капітальні вартості, — тимчасом як ця
частина, яка становить витрати виробництва, цілком визначається
видатками в межах відповідних сфер виробництва, — друга складова частина товарної ціни, зиск,
доданий до цих витрат виробництва, визначається не масою зиску, виробленою цим певним
капіталом у цій певній сфері виробництва протягом даного
часу, а тією масою зиску, яка за даний період часу пересічно припадає на кожний застосований капітал
як певну частину
\index{iii1}{0166}  %% посилання на сторінку оригінального видання
сукупного суспільного капіталу, вкладеного в сукупне
виробництво.\footnote{
Cherbuliez [„Riche ou Pauvre“, Paris-Genève 1840, стор. 116 і далі].
}

Отже, якщо капіталіст продає свій товар по його ціні виробництва, то він повертає собі кількість
грошей, відповідну величині вартості спожитого ним у виробництві капіталу, і добуває зиск
пропорціонально до його авансованого капіталу, просто
як до певної частини сукупного суспільного капіталу. Витрати
виробництва в кожній сфері виробництва мають специфічний
характер. Доданий до цих витрат виробництва зиск не залежить від його окремої сфери виробництва, він
є проста пересічна
на кожні 100 авансованого капіталу.

Припустімо, що п’ять різних капіталів І—V у вищенаведеному прикладі належать одній людині. Кількість
змінного і сталого капіталу, спожита на виробництво товарів у кожному окремому підрозділі І—V на
кожні 100 застосованого капіталу,
є дана; ця частина вартості товарів І—V, само собою зрозуміло, становитиме частину їх ціни, бо
принаймні ця ціна потрібна для заміщення авансованої і спожитої частини капіталу. Отже, ці витрати
виробництва були б різні для кожного роду
товарів І—V і, як такі, вони були б по-різному фіксовані їх власником. Що ж до різних мас додаткової
вартості або зиску, вироблених у підрозділах І—V, то капіталіст мав би всі підстави вважати їх за
зиск на весь свій авансований капітал, так що
на кожні 100 одиниць капіталу припадала б певна відповідна
частина. Отже, витрати виробництва товарів, вироблених в окремих підрозділах І—V, були б різні; але
в усіх цих товарів
була б однаковою частина продажної ціни, яка походить з доданого до витрат виробництва зиску на
кожні 100 одиниць капіталу. Отже, сукупна ціна товарів І—V дорівнювала б їх сукупній вартості, тобто
дорівнювала б сумі витрат виробництва
І—V плюс сума додаткової вартості, або зиску, вироблена в
І—V; отже, в дійсності ця ціна була б грошовим виразом сукупної кількості минулої і новододаної
праці, вміщеної в товарах
І—V. І таким чином, у самому суспільстві — якщо розглядати
всі галузі виробництва в їх сукупності — сума цін виробництва
вироблених товарів дорівнює сумі їх вартостей.

Цьому твердженню, здається, суперечить той факт, що в
капіталістичному виробництві елементи продуктивного капіталу
звичайно купуються на ринку, отже, ціни їх містять у собі вже
реалізований зиск, тобто ціну виробництва певної галузі промисловості разом з уміщеним в ній зиском,
так що зиск однієї
галузі промисловості входить у витрати виробництва іншої.
Але якщо ми підрахуємо на одному боці суму витрат виробництва товарів цілої країни, а на другому —
суму її зиску або додаткової вартості, то, очевидно, матимемо правильний обрахунок. Візьмімо,
наприклад, товар А; нехай витрати його
\parbreak{}  %% абзац продовжується на наступній сторінці

\input{_0167.tex}
\parcont{}  %% абзац починається на попередній сторінці
\index{iii1}{0168}  %% посилання на сторінку оригінального видання
сама дорівнює витратам виробництва плюс додаткова вартість,
отже, в даному разі дорівнює витратам виробництва плюс зиск,
а цей зиск знов таки може бути більший або менший, ніж додаткова вартість, місце якої він заступає.
Щодо змінного капіталу, то хоч пересічна денна заробітна плата завжди дорівнює вартості, виробленій
за те число годин яке робітник
мусить працювати, щоб виробити необхідні засоби існування,
однак саме число цих годин знов таки фальсифікується в наслідок того, що ціни виробництва необхідних
засобів існування
відхиляються від їх вартостей. Однак, це завжди розв’язується
таким чином, що наскільки в один товар входить більше додаткової вартості, настільки її в другий
товар входить менше,
і тому ті відхилення від вартості, які містяться в цінах виробництва товарів, взаємно знищуються.
Взагалі в цілому капіталістичному виробництві загальний закон здійснюється завжди
тільки як панівна тенденція, дуже заплутаним і приблизним
способом, тільки як якась пересічна вічних коливань, яка ніколи
не може бути точно встановлена.

Через те що загальна норма зиску утворюється з пересічної
різних норм зиску на кожні 100 авансованого капіталу за певний
період часу, скажімо, за рік, то в ній стирається також ріжниця,
викликана ріжницею в часі оборотів різних капіталів. Але ці
ріжниці є визначальним фактором для тих різних норм зиску
різних сфер виробництва, що з їх пересічної утворюється загальна норма зиску.

В попередній ілюстрації утворення загальної норми зиску
кожний капітал у кожній сфері виробництва припускався = 100,
і це було зроблено саме для того, щоб з’ясувати процентну
ріжницю в нормах зиску, а тому й ріжницю у вартостях товарів,
вироблюваних рівновеликими капіталами. Але само собою зрозуміло: дійсні маси додаткової вартості,
створювані в кожній
окремій сфері виробництва, залежать від величини застосованих
капіталів, бо в кожній такій даній сфері виробництва склад капіталу є даний. Тимчасом особлива \emph{норма}
зиску кожної окремої
сфери виробництва не змінюється від того, чи застосовується
капітал в 100, $100 × m$ чи $100 × xm$. Норма зиску однаково лишається 10\% — чи становить весь зиск $10 :
100$, чи $1000 : 10000$.

Але через те що норми зиску в різних сферах виробництва
є різні, бо в них залежно від відношення змінного капіталу до
всього капіталу виробляються дуже різні маси додаткової вартості, отже й зиску, то очевидно, що
пересічний зиск на кожні
100 суспільного капіталу, отже, пересічна норма зиску або загальна норма зиску, буде дуже різна
залежно від відповідної
величини капіталів, вкладених у різні сфери виробництва. Візьмімо чотири капітали $А$, $В$, $C$, $D$. Нехай
норма додаткової вартості для всіх них буде = 100\%. Нехай на кожні 100 сукупного
капіталу змінного капіталу буде для $А = 25$, для $B = 40$, для
$C = 15$, для $D = 10$. На кожні 100 сукупного капіталу тоді припадало
\index{iii1}{0169}  %% посилання на сторінку оригінального видання
б додаткової вартості або зиску у $А = 25$, $B = 40$, $C = 15$, $D = 10$; разом $= 90$; отже, якщо всі
чотири капітали є рівновеликі, пересічна норма зиску є \frac{90}{4} = 22\sfrac{1}{2}\%.

Але якщо загальні величини цих капіталів будуть: $А = 200$, $B = 300$, $C = 1000$, $D = 4000$, то вироблені
зиски будуть відповідно 50, 120, 150 і 400. Разом на 5500 капіталу зиску буде 720, або пересічна
норма зиску буде 13 \sfrac{1}{11}\%.

Маси всієї виробленої вартості є різні залежно від різних загальних величин відповідних капіталів,
авансованих в $А$, $В$, $C$, $D$.
Тому при утворенні загальної норми зиску справа йде не тільки
про ріжницю \emph{норм} зиску в різних сферах виробництва, з яких
просто треба було б вивести пересічну, але й про відносну вагу,
з якою ці різні норми зиску входять в утворення пересічної.
Але це залежить від відносної величини капіталу, вкладеного
в кожну окрему сферу виробництва, тобто від того, яку частину
сукупного суспільного капіталу становить капітал, вкладений
в кожну окрему сферу виробництва. Дуже велика ріжниця мусить, звичайно, бути залежно від того, чи
більша чи менша
частина сукупного капіталу дає вищу або нижчу норму зиску.
Але це знов таки залежить від того, скільки капіталу вкладено
в ті сфери виробництва, в яких відношення змінного капіталу
до всього капіталу є високе або низьке. Тут справа стоїть цілком
так само, як з пересічним процентом, що його одержує лихвар,
який віддає в позику різні капітали за різні норми процента, наприклад, за 4, 5, 6, 7\% і т. д.
Пересічна норма цілком залежить
від того, скільки з свого капіталу він позичив за кожну з цих
різних норм процента.

Отже, загальна норма зиску визначається двома факторами:

1) органічним складом капіталів у різних сферах виробництва,
отже, різними нормами зиску в окремих сферах;

2) розподілом сукупного суспільного капіталу між цими різними сферами, отже, відносною величиною
капіталу, вкладеного
в кожну окрему сферу, і отже з окремою нормою зиску, тобто
відносною масою сукупного суспільного капіталу, яку поглинає
кожна окрема сфера виробництва.

В I і II книгах ми мали справу тільки з \emph{вартостями} товарів.
Тепер, з одного боку, відокремились \emph{витрати виробництва}, як
частина цієї вартості, з другого боку, розвинулась \emph{ціна виробництва} товару, як перетворена форма
вартості товару.

Припустім, що склад пересічного суспільного капіталу є
$80 c + 20 v$, а норма річної додаткової вартості $m' = 100\%$; тоді
річний пересічний зиск для капіталу в 100 буде $= 20$, а загальна
річна норма зиску $= 20\%$. Хоч би які були $k$, витрати виробництва
товарів, вироблених за рік капіталом в 100, їх ціна виробництва була б $= k + 20$. В сферах
виробництва, де склад капіталу $= (80 — x) c + (20 x) v$, дійсно створена додаткова вартість,
відповідно річний зиск, вироблений у цій сфер і виробництва,
\index{iii1}{0170}  %% посилання на сторінку оригінального видання
був би $= 20 + x$, отже, більший ніж 20, і вироблена
товарна вартість була б $= k + 20 + x$, більша ніж $k + 20$, або
більша, ніж ціна виробництва. У сферах виробництва, в яких
склад капіталу $(80 + x) c + (20 — x) v$, створювана протягом року
додаткова вартість, або зиск, була б $= 20 — x$, отже, менша, ніж 20,
а тому товарна вартість $k + 20 — x$ була б менша, ніж ціна виробництва, яка $= k + 20$. Якщо залишити
осторонь можливі ріжниці в часі оборотів, то ціна виробництва товарів дорівнювала б їх вартості
тільки в тих сферах, в яких склад капіталу випадково був би $= 80 c + 20 v$.

В кожній окремій сфері виробництва специфічний розвиток
суспільної продуктивної сили праці є різний щодо ступеня,
вищий чи нижчий, відповідно до того, наскільки велика є кількість засобів виробництва, що їх
приводить в рух певна кількість праці, тобто, при даному робочому дні, певне число робітників; отже,
він вищий чи нижчий, відповідно до того, наскільки
мала є кількість праці, потрібна для певної кількості засобів
виробництва. Тому капітали, які містять у собі більший процент
сталого, отже, менший процент змінного капіталу, ніж пересічний суспільний капітал, ми звемо
капіталами \emph{вищого} складу.
Навпаки, такі капітали, в яких сталий капітал займає відносно
менше, а змінний відносно більше місце, ніж у пересічному
суспільному капіталі, ми звемо капіталами \emph{нижчого} складу.
Нарешті, ми звемо капіталами пересічного складу такі капітали,
склад яких збігається з складом пересічного суспільного капіталу. Якщо пересічний суспільний капітал
в процентах складається з $80 c + 20 v$, то капітал $90 c + 10 v$ стоїть \emph{вище}, а капітал $70 c + 30 v$
\emph{нижче}, ніж пересічний суспільний. Взагалі, при
складі пересічного суспільного капіталу, рівному $mc + nv$, де
$m$ і $n$ є сталі величини і $m + n = 100$, $(m + x) c + (n — x) v$ репрезентує вищий, а $(m — x) c + (n + x)
v$ — нижчий склад окремого капіталу або групи капіталів. Як функціонують ці капітали після
встановлення пересічної норми зиску, — припускаючи, що
вони обертаються один раз за рік, — це показує нижченаведена
таблиця, в якій I представляє пересічний склад, і тому пересічна норма зиску = 20\%.

\begin{center}
\begin{tabular}{c c c c}

\toprule
Капітал & Норма зиску & Ціна продукту & Вартість \\
\midrule

\phantom{II}I. $80 c + 20 v + 20 m$ & 20\% & 120 &  120 \\

\phantom{I}II.    $90 c + 10 v + 10 m$ & 20\% & 120 & 110\\

III. $70 c + 30 v + 30 m$ & 20\% & 120 & 130 \\
\end{tabular}
\end{center}
Отже, для товарів, вироблених капіталом II, їхня вартість була б
менша, ніж їхня ціна виробництва, для товарів капіталу III ціна
виробництва була б менша, ніж вартість, і тільки для капіталів I,
\parbreak{}  %% абзац продовжується на наступній сторінці

\parcont{}  %% абзац починається на попередній сторінці
\index{ii}{0171}  %% посилання на сторінку оригінального видання
то звичайно руйнується все підприємство; в кращому разі будинки
лишаються недобудовані до ліпших часів, а в найгіршому їх продається
з авкціону й за півціни. Без спекуляційних будов, та до того ще в широких
розмірах, не може тепер обійтись жоден підприємець. Зиск від
самих будов надзвичайно малий; його головний бариш у підвищенні
земельної ренти, у влучному виборі та використанні забудовуваної площі.
Таким способом, тобто спекуляцією, що антиципує попит на будинки,
збудовано майже цілу Бельґравію й Тібурнію, а також багато тисяч вілл
в околицях Лондону. (Скорочений виклад з „Report from the Select
Committee on Bank Acts. Part I, 1857. Evidence. Запитання 5413--18,
5435--36).

Виконання робіт, що потребують дуже довгого робочого періоду й
широкого маштабу, цілком потрапляє в руки капіталістичної
продукції лише тоді, коли концентрація капіталу вже дуже велика, і
коли, з другого боку, розвиток кредитової системи дає капіталістові зручний
засіб авансувати чужий капітал замість власного, а тому й ризикувати
чужим капіталом. Однак, само собою зрозуміло, що на швидкість обороту
й на час обороту капіталу не впливає зовсім та обставина, чи належить
капітал, чи не належить тому, хто авансує його на підприємство.

Обставини, що збільшують продукт окремого робочого дня, як от кооперація,
поділ праці, застосування машин, разом з тим скорочують робочий
період при актах продукції, що зв’язані між собою. Напр., машини
скорочують час будування будинків, мостів тощо; жатки, молотарки
тощо скорочують робочий період, потрібний на те, щоб перетворити
достигле зерно на готовий товар. Удосконалене суднобудівництво, збільшуючи
швидкість суден, скорочує час обороту капіталу, витраченого на
судноплавство. Однак, ці поліпшення, що скорочують робочий період, а
через це й час, що на нього треба авансувати обіговий капітал, сполучаються
здебільша із збільшеною витратою основного капіталу. З другого
боку, в деяких галузях робочий період може скоротитись в наслідок
простого поширення кооперації: будування залізниць скорочується
в наслідок того, що в роботу пускається великі армії робітників, які
беруться до роботи одночасно в багатьох пунктах. Час обороту скорочується
тут в наслідок збільшення авансованого капіталу. Більше засобів
продукції і більше робочої сили мусять об’єднатись під командою
капіталіста.

Отже, якщо скорочення робочого періоду здебільша сполучається зі
збільшенням капіталу, авансованого на коротший час, так що в міру того,
як скорочується час авансування, більшає маса авансованого капіталу, — то
треба тут згадати, що, незалежно від того, яка взагалі є маса суспільного
капіталу, справа сходить на те, якою мірою засоби продукції й засоби
існування, зглядно порядкування ними, є розпорошені або зосереджені
в руках поодиноких капіталістів, отже, якого розміру вже
досягла концентрація капіталів. Оскільки кредит упосереднює концентрацію
капіталу в одних руках, прискорює та підвищує її, остільки він
сприяє скороченню робочого періоду, а тим самим і скороченню часу обороту.


\index{iii1}{0172}  %% посилання на сторінку оригінального видання
Формула, згідно з якою ціна виробництва товару $= k + p$,
дорівнює витратам виробництва плюс зиск, визначилась тепер
ближче таким чином, що $p = kp'$ (де $p'$ є загальна норма зиску),
і, отже, ціна виробництва $= k + kp'$. Якщо $k = 300$, а $p' = 15\%$,
то ціна виробництва $k + kp' = 300 + 300$. $\frac{15}{100} = 345$.

Ціна виробництва товарів у кожній окремій сфері виробництва може змінювати свою величину:

1) при незмінній вартості товарів (тобто при умові, що у виробництво товару після зміни ціни
виробництва входить та сама
кількість мертвої і живої праці, як і до зміни) в наслідок незалежної від даної окремої сфери зміни
загальної норми зиску;

2) при незмінній загальній нормі зиску в наслідок зміни вартості — чи то в самій даній сфері
виробництва, в результаті
технічних змін, чи в наслідок зміни вартості тих товарів, які
входять у сталий капітал цієї сфери як його складові елементи;

3) нарешті, внаслідок спільного впливу обох цих обставин.
Не зважаючи на великі зміни, які постійно — як це виявиться
далі — відбуваються у фактичних нормах зиску окремих сфер
виробництва, дійсна зміна в загальній нормі зиску, оскільки
вона викликається не винятковими, надзвичайними економічними
подіями, є дуже пізній результат ряду коливань, які охоплюють
дуже довгі періоди часу, тобто коливань, що потребують багато часу, поки вони сконсолідуються і
вирівняються у зміну
загальної норми зиску. Тому при всіх коротших періодах (цілком незалежно від коливань ринкових цін)
зміну цін виробництва
треба завжди пояснювати prima facie [очевидно] дійсною зміною
вартості товарів, тобто зміною всієї суми робочого часу, потрібного для їх виробництва. Проста зміна
грошового виразу тих самих вартостей тут, само собою зрозуміло, зовсім не береться до уваги.\footnote{
\emph{Corbett} [„An Inquiry into the Causes and Modes of the Wealth of Individuals“.
London 1841], стор. [33 і далі] 174.
}

З другого боку, очевидно, що коли розглядати сукупний
суспільний капітал, то сума вартості вироблених ним товарів
(або, в грошовому виразі, їхня ціна), = вартості сталого капіталу + вартість змінного капіталу +
додаткова вартість. Якщо припустити, що ступінь експлуатації праці є незмінний, то норма зиску
при незмінній масі додаткової вартості може змінюватись тут
тільки в тому випадку, коли вартість сталого капіталу змінюється,
або коли вартість змінного капіталу змінюється, або ж коли змінюються обидві ці вартості, так що
змінюється $K$, а тому й $\frac{m}{K}$, загальна норма зиску. Отже, в кожному випадку зміна загальної
норми зиску передбачає зміну вартості товарів, що входять як
складові елементи в сталий капітал, або в змінний капітал, або
одночасно в той і в другий.


\index{iii1}{0173}  %% посилання на сторінку оригінального видання
Або ж загальна норма зиску може змінюватись при незмінній вартості товарів, якщо змінюється ступінь
експлуатації праці.

Або, при незмінному ступені експлуатації праці, загальна
норма зиску може змінюватися в тому випадку, коли сума вживаної праці змінюється відносно сталого
капіталу в наслідок
технічних змін у процесі праці. Але такі технічні зміни завжди
мусять виявлятись у зміні вартості товарів, а тому й супроводитись цією зміною вартості товарів,
виробництво яких вимагає
тепер більше або менше праці, ніж раніше.

В першому відділі ми бачили, що додаткова вартість і зиск,
розглядувані щодо їхньої маси, тотожні. Однак, норма зиску вже
з самого початку відрізняється від норми додаткової вартості,
при чому спочатку це виявляється тільки як інша форма обрахунку; але через те що норма зиску може
підвищуватись або
падати при незмінній нормі додаткової вартості і навпаки, і що
капіталіста практично цікавить виключно норма зиску, то це
знов таки вже з самого початку цілком затемнює і містифікує
дійсне поводження додаткової вартості. Проте, кількісна ріжниця
була тільки між нормою додаткової вартості і нормою зиску,
а не між самими додатковою вартістю і зиском. Через те що
в нормі зиску додаткова вартість обчислюється на весь капітал
і відноситься до нього як до своєї міри, то вже внаслідок
цього здається, що сама додаткова вартість виникла з усього
капіталу, і при тому рівномірно з усіх його частин, так що
в понятті зиску стирається органічна ріжниця між сталим і змінним капіталом; тому, дійсно, у цій
своїй перетвореній формі,
у формі зиску, додаткова вартість сама заперечує своє походження, втрачає свій характер, стає
непізнаванною. Однак, досі
ріжниця між зиском і додатковою вартістю зводилась тільки до
якісної зміни, до зміни форми, тимчасом як дійсна кількісна
ріжниця на цьому першому ступені перетворення існує тільки
між нормою зиску і нормою додаткової вартості, але ще не існує
між зиском і додатковою вартістю.

Інакше стоїть справа, коли вже встановлюється загальна
норма зиску і за її допомогою пересічний зиск, відповідний до
даної для різних сфер виробництва величини застосовуваного
капіталу.

Тепер тільки випадково додаткова вартість, — а тому й зиск, — дійсно створена в якійсь окремій сфері
виробництва, може збігтися з зиском, який міститься в продажній ціні товару. Як загальне правило,
тепер зиск і додаткова вартість, а не тільки їх
норми, є дійсно різні величини. Тепер, при даному ступені
експлуатації праці, маса додаткової вартості, створена в якійсь
окремій сфері виробництва, має важливіше значення для сукупного пересічного зиску суспільного
капіталу, отже, для капіталістичного класу взагалі, ніж безпосередньо для капіталістів кожної
окремої галузі виробництва. Для них це має значення
\parbreak{}  %% абзац продовжується на наступній сторінці

\parcont{}  %% абзац починається на попередній сторінці
\index{i}{0174}  %% посилання на сторінку оригінального видання
знайомства з вами в кращому світі. Addio\elli{!..}\footnote{
Однак пан професор мав деяку користь із своєї подорожі до Менчестеру.
В «Letters on the Factory Act» увесь чистий прибуток, «зиск» і «процент» і
навіть «something more»\footnote*{
щось більше. \emph{Ред}.
}, залежить від
однієї неоплаченої
робочої години робітника! Роком раніш у своїх «Outlines of Political Economy»,
складених для насолоди оксфордських студентів і освічених філістерів, Сеніор,
полемізуючи проти Рікардового визначення вартости робочим часом, «відкрив», що
зиск постає з праці капіталіста, а процент з його
аскетичности, з його «поздержливости». Сама побрехенька була стара, але слово
«поздержливість» («Abstinenz») було нове. Пан Рошер правильно переклав його
німецькою мовою словом «Enthaltung» («поздержливість»). А його компатріоти,
менше биті в латині, Вірти, Шульци
й інші Міхелі, переклали його на чорнече «самовідречення» («Entsagung»).
} Сиґнал «останньої години», що її винайшов Сеніор 1836~\abbr{р.}, наново протрубив
був 15 квітня 1848~\abbr{р.} в «London Economist» Джемc Вілсон, один з головних
мандаринів економічної науки, у своїй полеміці проти
закону про десятигодинний робочий день.

\manualpagebreak{}
\subsection{Додатковий продукт}

Ту частину продукту (\sfrac{1}{10} від 20 фунтів пряжі, або 2 фунти пряжі, у
прикладі §2), яка репрезентує додаткову вартість, ми
називаємо додатковим продуктом (surplus produce, produit net). Як норму
додаткової вартости визначає відношення додаткової
вартости не до цілої суми капіталу, а лише до його змінної складової частини,
так і рівень додаткового продукту визначає відношення останнього не до решти
цілого продукту, а до тієї частини його, яка репрезентує доконечну працю.
Як продукція додаткової вартости є визначальна мета
капіталістичної продукції, так і ступінь багатства вимірюється не абсолютною
величиною продукту, а відносною величиною додаткового продукту\footnote{
«Для індивіда, що має капітал у \num{20.000}\pound{ фунтів стерлінґів}, і що його зиски
становлять \num{2.000}\pound{ фунтів стерлінґів} на рік, було б цілком байдуже, чи його
капітал вживає 100 чи \num{1.000} робітників, чи випродуковані товари продається
за \num{10.000}\pound{ фунтів стерлінґів} чи за \num{20.000}\pound{ фунтів стерлінґів}, аби лише
його зиски в усіх цих випадках не падали нижче як \num{2.000}\pound{ фунтів
стерлінґів}. Хіба реальний інтерес націй не такий самий? Коли припустити, що
реальний чистий прибуток нації, її ренти й зиски лишаються однакові, то не має
найменшої ваги, чи нація складається з 10 чи
12 мільйонів людности». (\emph{Ricardo}: «The Principles of Political Economy»,
3 rd. ed, London 1821, p. 416). Задовго перед Рікардом Артур Юнґ, фанатик
додаткового продукту, взагалі язикатий, неспроможний
на будь-яку критику письменник, що його слава стоїть у зворотному відношенні
до його заслуг, сказав, між іншим: «Що за користь була б для сучасного
королівства з якоїсь цілої провінції, що в ній землю обробляли б на
староримський лад дрібні незалежні селяни, про мене хоч би
й як і найкраще? Яка мета була б у цьому, крім одним-однієї мети продукувати
людей («the mere purpose of breeding men»), а це саме по собі не має
ніякої мети» (is a most useless purpose»). (\emph{Arthur Young}: «Political
Arithmetic etc.», 1774, p. 47).
Додаток до примітки 34. Дивний є «великий нахил малювати чистий прибуток
корисним для робітничої кляси\dots{} та проте ясно, що це стається не через те,
що він чистий» («the strong inclination to
represent net wealth as beneficial to the labouring class\dots{} though it
is evidently not on account
of being net»). (\emph{Th.~Hopkins}: «On Rent of Land etс.», London 1823, p. 126).
}.

\input{_0175.tex}
\parcont{}  %% абзац починається на попередній сторінці
\index{i}{0176}  %% посилання на сторінку оригінального видання
Тому робочий день можна визначити, але сам по собі він є величина
невизначена\footnote{
«Робочий день є величина невизначена, він може бути довгий або
короткий» («А day’s labour is vague, it may be long or short»). («An
Essay on Trade and Commerce, containing Observations on Taxation etc.»,
London 1770, p. 73).
}.

Хоч робочий день є не стала, а змінна величина, однак, з
другого боку, він може змінятися лише в певних межах. Але мінімальні
межі його визначити не можна. Певна річ, коли ми припустимо,
що лінія здовження \emph{bc}, або додаткова праця, дорівнює
нулеві, то тоді матимемо мінімальну межу, а саме ту частину дня,
яку робітник доконечно мусить працювати, щоб утримати себе
самого. Але на базі капіталістичного способу продукції доконечна
праця завжди може становити лише частину його робочого дня,
отже, робочий день ніколи не може бути скорочений до цього
мінімуму. Навпаки, робочий день має певну максимальну межу.
Його не можна здовжити поза певну межу. Цю максимальну
межу визначається подвійно. Поперше, фізичними межами робочої
сили. Людина може протягом природного дня в 24 години
витратити тільки певну кількість життєвої сили. Приміром, кінь
може працювати з дня на день лише 8 годин. Протягом однієї
частини дня сила мусить відпочивати, спати, протягом другої
частини людині треба задовольняти інші фізичні потреби: живитися,
підтримувати чистоту, одягатися й~\abbr{т. ін.} Крім цих суто
фізичних меж здовження робочого дня натрапляє на моральні
межі. Робітник потребує часу, щоб задовольняти інтелектуальні
й соціяльні потреби, що їхній обсяг і кількість визначаються
загальним станом культури. Отже, зміни, що їх зазнає робочий
день, рухаються в цих фізичних і соціяльних межах. Але обидві
ці межі дуже елястичної природи й дають найбільший простір
для варіяцій. Так ми находимо робочі дні у 8, 10, 12, 14, 16, 18
годин, отже, дні дуже різноманітної довжини.

Капіталіст купив робочу силу за її денною вартістю. Йому
належить її споживна вартість протягом одного робочого дня.
Отже, він здобув собі право примушувати робітника працювати
для нього протягом одного дня. Але що таке робочий день?\footnote{
Питання це безконечно важливіше, ніж знамените питання сера
Роберта Піла, звернене до бермінґемської торговельної палати: «What
is a pound?»\footnote*{
Що таке фунт стерлінґів? \emph{Ред.}
} — питання, яке він міг поставити лише через те, що для
нього була так само неясна природа грошей, як і для «little shilling men»\footnote*{
людців від шилінґів. \emph{Ред.}
}
із Бермінґему.
}
У всякому разі щось менше за природний день життя. На
скільки? У капіталіста є свій власний погляд на цю ultima Thule\footnote*{
останню межу. \emph{Ред.}
},
доконечну межу робочого дня. Як капіталіст він є лише персоніфікований
капітал. Його душа — душа капіталу. Але капітал
має одним-однісіньке життєве прагнення — прагнення самозростати,
утворювати додаткову вартість, вбирати своєю сталою
\parbreak{}  %% абзац продовжується на наступній сторінці

\input{_0177.tex}

\index{iii1}{0178}  %% посилання на сторінку оригінального видання
\section{Вирівнення загальної норми зиску через
Конкуренцію. Ринкові ціни і ринкові вартості.
Надзиск}

Частина сфер виробництва має середній або пересічний склад
застосовуваного в них капіталу, тобто склад капіталу, який цілком чи приблизно збігається з складом
пересічного суспільного
капіталу.

Ціна виробництва товарів, вироблюваних у цих сферах виробництва, цілком чи приблизно збігається з їх
вартістю, вираженою
в грошах. І коли б ніяким іншим способом не можна було досягти математичної границі, то цього можна
було б досягти цим
способом. Конкуренція так розподіляє суспільний капітал між
різними сферами виробництва, що ціни виробництва в кожній
сфері утворюються на зразок цін виробництва в цих сферах
середнього складу, тобто = $k + kp'$ (витрати виробництва плюс
добуток пересічної норми зиску і витрат виробництва). Але
ця пересічна норма зиску є не що інше, як обчислений в процентах зиск у сфері виробництва середнього
складу, де, отже,
зиск збігається з додатковою вартістю. Отже, норма зиску в усіх
сферах виробництва є одна й та ж, а саме вирівнена до норми
зиску цих середніх сфер виробництва, в яких панує пересічний
склад капіталу. Тому сума зисків усіх різних сфер виробництва
мусить дорівнювати сумі додаткових вартостей і сума цін виробництва сукупного суспільного продукту
мусить дорівнювати
сумі його вартостей. Але ясно, що це вирівнювання між сферами виробництва з різним складом завжди
мусить прагнути
урівняти ці сфери з сферами середнього складу, однаково, чи
ці останні точно чи тільки приблизно відповідають пересічному
суспільному складові. У сферах виробництва, які більш чи менш
наближаються до середньої, знову таки має місце тенденція
до вирівнення, яка прагне до ідеального, тобто в дійсності не
наявного середнього рівня, тобто тенденція до вирівнення
навколо нього, як норми. Таким чином у цьому відношенні
необхідно панує тенденція зробити ціни виробництва просто
перетвореними формами вартості, або перетворити зиски в
прості частини додаткової вартості, які, однак, розподіляються
не пропорційно до додаткової вартості, створеної в кожній
окремій сфері виробництва, а пропорційно до маси капіталу,
застосовуваного в кожній сфері виробництва, так що на рівновеликі маси капіталу, хоч би який був їх
склад, припадають відповідно рівновеликі частини сукупної додаткової вартості, створеної сукупним
суспільним капіталом.

Отже, для капіталів середнього чи приблизно середнього
складу ціна виробництва збігається цілком або приблизно з вартістю,
\index{iii1}{0179}  %% посилання на сторінку оригінального видання
а зиск — із створеною ними додатковою вартістю. Всі інші
капітали, хоч би який був їх склад, під тисненням конкуренції
прагнуть зрівнятися з капіталами середнього або приблизно
середнього складу. Але через те що капітали середнього складу
е рівні або приблизно рівні пересічному суспільному капіталові,
то всі капітали, яка б не була величина створеної ними самими
додаткової вартості, прагнуть замість цієї додаткової вартості
реалізувати в цінах своїх товарів пересічний зиск, тобто прагнуть реалізувати ціни виробництва.

З другого боку, можна сказати, що повсюди, де встановлюється пересічний зиск, отже загальна норма
зиску — яким би
шляхом не досягався цей результат, — цей пересічний зиск не
може бути нічим іншим, як зиском на пересічний суспільний
капітал, зиском, сума якого дорівнює сумі додаткових вартостей,
а ціни, які утворюються в наслідок надбавки цього пересічного
зиску до витрат виробництва, не можуть бути нічим іншим, як
перетвореними в ціни виробництва вартостями. Справа ні трохи
не змінилася б, коли б капітали в певних сферах виробництва
з будь-яких причин не підлягали цьому процесові вирівнення.
Тоді пересічний зиск обчислювався б на ту частину суспільного
капіталу, яка входить у процес вирівнення. Очевидно, що пересічний зиск не може бути нічим іншим, як
сукупною масою
додаткової вартості, розподіленою в кожній сфері виробництва
між масами капіталів пропорційно до їхніх величин. Це — сума
реалізованої неоплаченої праці, і вся ця маса праці, так само як
і оплачена, мертва й жива праця, виражається в сукупній масі,
товарів і грошей, яка припадає капіталістам.

Справжня трудність питання тут ось у чому: як відбувається
це вирівнення зисків у загальну норму зиску, раз воно, очевидно, є результат і не може бути вихідним
пунктом.

Насамперед, очевидно, що оцінка товарних вартостей, наприклад, у грошах, може бути тільки
результатом обміну їх і що,
припускаючи таку оцінку, ми повинні розглядати її як результат
дійсного обміну товарної вартості на товарну вартість. Але
яким же чином може здійснитись цей обмін товарів по їх дійсних вартостях?

Припустімо, спочатку, що всі товари в різних сферах виробництва продаються по їх дійсних вартостях.
Що сталося б
тоді? Згідно з вищевикладеним, в різних сферах виробництва
тоді панували б дуже різні норми зиску. Чи продаються товари
по їх вартостях (тобто чи обмінюються вони один на один пропорційно до вміщеної в них вартості, по
цінах їх вартості),
чи продаються вони по таких цінах, що продаж їх дає рівновеликі зиски на рівновеликі маси капіталів,
авансованих на відповідне виробництво їх, — це prima facie [очевидно] цілком різні речі.

Та обставина, що капітали, які приводять в рух неоднакову
кількість живої праці, виробляють неоднакову кількість додаткової
\index{iii1}{0180}  %% посилання на сторінку оригінального видання
вартості, передбачає, принаймні до певної міри, що ступінь
експлуатації праці або норма додаткової вартості однакова, або
що існуючі в цьому відношенні ріжниці вирівнюються за допомогою
дійсних або уявних (умовних) компенсуючих причин.
Це передбачає конкуренцію між робітниками і вирівнювання
ступеня їх експлуатації в наслідок постійного переходу їх
з однієї сфери виробництва до іншої. Така загальна норма додаткової
вартості — як тенденція, подібно до всіх економічних
законів, — припускається нами як теоретичне спрощення; але
в дійсності вона є фактична передумова капіталістичного способу
виробництва, хоч вона й гальмується в більшій чи меншій
мірі практичними тертями, які викликають більш чи менш
значні місцеві ріжниці, такі є, наприклад, закони про осілість
(settlement laws) для землеробських поденників в Англії. Але
в теорії припускається, що закони капіталістичного способу
виробництва розвиваються в чистому вигляді. В дійсності існує
завжди тільки наближення; однак, це наближення тим більше,
чим більше розвинений капіталістичний спосіб виробництва і чим
більше усунене його забарвлення рештками попередніх економічних
становищ і переплетення з ними.

Вся трудність постає з того, що товари обмінюються не
просто як \emph{товари}, а як \emph{продукти капіталів}, які претендують
на пропорціональну до їх величини або, при рівній величині, на
рівну участь у сукупній масі додаткової вартості. І сукупна
ціна товарів, вироблених даним капіталом за даний період часу,
повинна задовольнити цю вимогу. Але сукупна ціна цих товарів
є просто сума цін окремих товарів, які становлять продукт
капіталу.

Punctum saliens [вирішальний пункт] виступить найбільше, якщо
ми підійдемо до справи так: Припустім, що самі робітники
володіють своїми відповідними засобами виробництва і обмінюють
свої товари один з одним. Ці товари не були б тоді
продуктами капіталу. Залежно від технічної природи їх робіт,
вартість засобів праці і матеріалів праці, застосовуваних у різних
галузях праці, була б різна; так само, незалежно від неоднакової
вартості застосовуваних засобів виробництва, потрібна була б
різна маса цих засобів виробництва для даної маси праці, залежно
від того, що один певний товар може бути виготовлений
за одну годину, а інший тільки за день і т. д. Припустімо
далі, що ці робітники пересічно працюють однакову кількість
часу, враховуючи вирівнення, які випливають з різної інтенсивності
праці та ін. Двоє робітників замістили б тоді в товарах,
що становлять продукт їх денної праці, поперше, свої
видатки, витрати (die Kostpreise) на спожиті засоби виробництва.
Ці останні були б різні залежно від технічної природи їх галузей
праці. Подруге, вони обидва створили б однакові кількості
нової вартості, а саме робочий день, доданий ними до засобів
виробництва. Ця нова вартість містила б у собі їх заробітну
\parbreak{}  %% абзац продовжується на наступній сторінці

\input{_0181.tex}

\index{iii1}{0182}  %% посилання на сторінку оригінального видання
Отже, незалежно від панування закону вартості над цінами
і рухом цін, цілком відповідає справі розглядати вартість товарів
не тільки теоретично, але й історично як prius [те, що передує]
відносно цін виробництва. Це стосується до таких економічних
відносин, коли засоби виробництва належать робітникові,
а таке є становище як у стародавньому, так і в новітньому
світі, у селянина, що працює сам і володіє землею, і в ремісника.
Це погоджується, також з тим висловленим нами раніше
поглядом\footnote{
Тоді, в 1865 році, це було тільки „поглядом“ Маркса. Тепер, після широких
досліджень первісної громади, починаючи від Маурера і кінчаючи Морганом,
це факт, навряд чи кимсь заперечуваний. — \emph{Ф. Е.}
}, що розвиток продуктів у товари постає через обмін
між різними громадами, а не між членами однієї і тієї ж громади.
Так само як до цього первісного становища, це стосується
також і до пізніших відносин, основаних на рабстві
й кріпацтві, а також до цехової організації ремесла — поки засоби
виробництва, закріплені в кожній галузі виробництва, тільки
з труднощами можуть бути перенесені з однієї сфери в іншу,
і тому різні сфери виробництва відносяться одна до одної
до певної міри так само, як чужі країни або комуністичні
громади.

Для того, щоб ціни, по яких взаємно обмінюються товари,
приблизно відповідали їх вартостям, потрібно тільки, щоб 1) обмін
різних товарів перестав бути чисто випадковим або лише
принагідним; 2) щоб ці товари, оскільки ми розглядаємо безпосередній
товарообмін, вироблялися з тієї і другої сторони у відносних
кількостях, приблизно відповідних взаємній потребі в них,
що встановлюється взаємним досвідом при збуті, отже, виростає
як результат з самого триваючого обміну, і 3) оскільки ми
говоримо про продаж, щоб ніяка природна або штучна монополія
не давала можливості сторонам-контрагентам продавати
вище вартості і не примушувала їх збувати товари нижче вартості.
Під випадковою монополією ми розуміємо монополію, яка створюється
для покупця або продавця з випадкового стану попиту
й подання.

Припущення, що товари різних сфер виробництва продаються
по їх вартостях, означає, звичайно, тільки те, що їх вартість
є центр тяжіння, навколо якого обертаються їх ціни і за яким
вирівнюються їх постійні коливання вгору і вниз. Крім того,
треба завжди відрізняти \emph{ринкову вартість} — про яку мова буде
пізніше — від індивідуальної вартості окремих товарів, які виробляються
різними виробниками. Індивідуальна вартість деяких
з цих товарів стоятиме нижче ринкової вартості (тобто для їх
виробництва потрібно менше робочого часу, ніж виражає ринкова
вартість), індивідуальна вартість інших товарів — вище
ринкової вартості. Ринкову вартість треба розглядати, з одного
боку, як пересічну вартість товарів, вироблених у певній сфері
\parbreak{}  %% абзац продовжується на наступній сторінці

\input{_0183.tex}
\input{_0184.tex}
\input{_0185.tex}
\parcont{}  %% абзац починається на попередній сторінці
\index{i}{0186}  %% посилання на сторінку оригінального видання
законом час, ви вкладали б мені до кишені річно \num{1.000}\pound{ фунтів
стерлінґів}»\footnote{
Там же, стор. 48.
}. «Атоми часу є елементи баришу»\footnote{
«Moments are the element of profit». («Reports of the Insp. etc.
30 th April 1860», p. 56).
}.

З цього боку немає нічого характеристичнішого, як назва
«\textenglish{full times}»\footnote*{
повний час. \emph{Ред.}
} для робітників, що працюють повний час, і «half
times»\footnote*{
половина часу. \emph{Ред.}
} для дітей до тринадцятилітнього віку, яким дозволяється
працювати лише по 6 годин\footnote{
Цей вислів має офіціяльне право громадянства так на фабриці,
як і у фабричних звітах.
}. Робітник є тут не що більше,
як персоніфікований робочий час. Усі індивідуальні ріжниці
сходять на ріжницю між «Vollzeitler» і «Halbzeitler»\footnote*{
робітником повночасним і робітником півчасним. \emph{Ред.}
}.

\subsection[Галузі англійської промисловости \\
без~законодавчих~меж~експлуатації]{%
Галузі англійської промисловости
без~законодавчих~меж~експлуатації}

Досі ми розглядали прагнення здовжувати робочий день,
ненажерливий вовчий голод за додатковою працею, на такому
полі, де безмірні зловживання, не перевищені і навіть — як каже
один буржуазний англійський економіст, — жорстокостями еспанців
проти червоношкурих Америки\footnote{
«Ненажерливість власників фабрик призводить до того, що в погоні
за баришем вони допускаються таких жорстокостей, яких ледве чи
перевищили жорстокості еспанців підчас завойовування Америки в гонитві
за золотом» («The cupidity of mill-owners, whose cruelties in pursuit
of gain, have hardly been exceeded by those perpetrated by the Spaniards
on the conquest of America in the pursuit of gold»). (John Wade:
«History of the Middle and Working Classes», 3 rd ed. London 1835, p-114).
Теоретична частина цієї книги, свого роду нарис політичної економії,
містить у собі дещо ориґінальне для свого часу, приміром, про торговельні
кризи. Щождо історичної частини, то вона є безсоромний пляґіят із Sir
М.~Eden: «History of the Poor», London 1799.
}, спричинилися, нарешті,
до того, що капітал закували у ланцюги законодавчого реґулювання.
А тепер киньмо оком на деякі галузі промисловости, де
висисання робочої сили або ще й тепер вільне від тих законодавчих
пут, або було таким ще зовсім недавно.

«Пан Бровтон, суддя графства, як голова мітингу, який
відбувся в нотінгемському міському будинку 14 січня 1860~\abbr{р.},
заявив, що серед частини міської людности, занятої виробництвом
мережива, панують такі страшні злидні й нужда, що решта цивілізованого
світу ще таких не знає\dots{} О 2, 3, 4 годині ранку 9--10-літніх
дітей виривають із їхніх брудних ліжок і примушують
тільки за мізерний харч працювати до 10, 11, 12 години вночі,
в наслідок чого нидіють їхні члени, корчиться тіло, тупіють риси
їх обличчя, і їхнє ціле людське єство дубіє в німій нерухомості, на
яку навіть глянути страшно. Це для нас не диво, що пан Малет
і інші фабриканти виступили з протестом проти всякої дискусії.
Система, як її описав панотець Монтегіо Вальпі, — це система безмежного
\index{i}{0187}  %% посилання на сторінку оригінального видання
рабства, — рабства з кожного погляду, соціяльного, фізичного,
морального й інтелектуального\dots{} Що подумати про місто,
яке скликає прилюдний мітинг на те, щоб просити про обмеження
робочого часу для чоловіків на 18 годин на добу\elli{!..} Ми деклямуємо
проти вірджінських і каролінських плянтаторів. Але хіба їхня
торговля неграми з усіма страхіттями батога й баришування людським
м’ясом огидніша, ніж це повільне душогубство, яке відбувається
для того, щоб на користь капіталістам вироблялося
серпанки й комірчини?»\footnote{
«London Daily Telegraph», з 17 січня 1860~\abbr{р.}
}

Ганчарня (Pottery) Стафордшіру була протягом останніх
22 років предметом трьох парляментських слідств. Результати
цих слідств наведено у звіті пана Скрайвена з 1841~\abbr{р.}, поданому
членам «Children’s Employment Commission», у звіті д-ра
Ґрінхов з 1860~\abbr{р.}, опублікованому за розпорядженням лікарського
урядовця Privy Council («Public Health», 3 rd Report,
I, 112--113), нарешті, y звіті пана Льон Ге з 1863~\abbr{р.}, у «First
Report of the Children’s Employment Commission» з 13 червня
1863~\abbr{р.} Для мого завдання досить узяти із звітів з 1860 і 1863~\abbr{рр.}
деякі свідчення дітей, що сами були об’єктом експлуатації. Із
становища дітей можна робити висновки й про становище дорослих,
а особливо дівчат і жінок, і до того ж в такій галузі промисловости,
поруч з якою бавовнопрядіння й~\abbr{т. ін.} може видаватися
дуже приємною й здоровою працею\footnote{
Порівн. Engels: «Lage der arbeitenden Klasse in England»,
Leipzig 1845, S. 249--251. (Енгельс: «Становище робітничої кляси в
Англії». Партвидав «Пролетар», 1932~\abbr{р.}, стор. 233--236).
}.

Вільгельм Вуд, дев’яти років, «почав працювати, мавши
7 років 10 місяців». Спочатку він був «van moulds» (носив до
сушні виготовлений товар у формах і приносив назад порожні
форми). Цілий тиждень день-у-день приходив о 6 годині вранці
й кінчав роботу так десь коло 9 години вечора. «Я цілий тиждень
працюю щодня до 9 години вечора. Так було, приміром, протягом
останніх 7--8 тижнів». Отже, п’ятнадцять годин праці для семилітньої
дитини! Дж.~Меррей, дванадцятилітнє хлоп’я, свідчить:
«І run moulds and turn jigger (я ношу форми та кручу колесо).
Я приходжу о 6, іноді о 4 годині вранці. Я працював цілу останню
ніч до 8 години сьогоднішнього ранку. Я не спав від минулої
ночі. Крім мене працювало ще 8 або 9 хлопчиків цілу останню
ніч без перерви. За винятком одного, всі знов прийшли сьогодні
вранці. Я дістаю 3\shil{ шилінґи} 6\pens{ пенсів} (1 таляр 5 шагів) на тиждень.
Я не дістаю більше, коли працюю цілу ніч. Останнього тижня
я працював дві ночі». Фернігав, десятилітнє хлоп’я: «Мені не
завжди лишається ціла година на обід, часто лише півгодини;
це буває щочетверга, п’ятниці й суботи»\footnote{«Children’s Employment Commission. 1 st Report etc. 1863», Appendix,
p. 16, 19, 18.
}.

Д-р Ґрінхов заявляє, що вік життя в ганчарняних округах
Stoke-upon-Trent і Wolstanton надзвичайно короткий. Хоч в
\parbreak{}  %% абзац продовжується на наступній сторінці

\parcont{}  %% абзац починається на попередній сторінці
\index{i}{0188}  %% посилання на сторінку оригінального видання
окрузі Stoke працює по ганчарнях лише 30,6\%, а у Wolstanton
лише 30,4\% чоловічої людности у віці понад 20 років, однак
з-поміж чоловіків цієї категорії в першій окрузі більше, ніж
половина, а у другій окрузі приблизно дві п’ятих із загального
числа смертних випадків з грудних недуг припадає на ганчарів.
Д-р Бутройд, лікар, що практикує в Hanley, свідчить: «Кожне
наступне покоління ганчарів карликуватіше й слабіше, ніж попереднє».
Так само інший лікар, пан Мак-Бін, каже: «Від того
часу, як я перед 25 роками почав свою практику серед ганчарів,
наочне виродження цієї кляси виявилось у дедалі швидшому
зменшенні зросту й ваги». Ці свідчення взято із звіту д-ра
Ґрінхова з 1860~\abbr{р.}\footnote{
«Public Health. 3 rd Report etc.», p. 102, 104, 105.
}

Із звіту комісарів 1863~\abbr{р.} ми наводимо ось що: Д-р Дж.~Т.~Ерледж, головний лікар шпиталю в північному Стафордшірі,
каже: «Як кляса, ганчарі, чоловіки й жінки\dots{} являють собою
фізично й морально вироджену людність. Вони звичайно низькі
на зріст, поганої статури й часто нездужають на викрив грудини.
Вони передчасно старіються й живуть недовго; флегматичні й
малокровні, вони виявляють слабість своєї статури нападами
впертої диспепсії, розладів діяльности печінки й нирок та ревматизму.
Але передусім піддаються вони грудним недугам: запаленню
легенів, сухотам, бронхітові й астмі. Одна форма цієї
останньої спеціяльно властива їм і відома під назвою ганчарської
астми, або ганчарських сухот. Золотуха, що захоплює залози,
кості й інші частини тіла, є недуга, на яку слабує більше ніж
дві третини ганчарів. Те, що виродження (degenerescence) людности
цієї округи не є ще значно більше, завдячує виключно
припливові нових елементів із суміжних селянських округ і
шлюбам із здоровшими расами». Пан Чарлз Пірсон, що недавно
ще перед тим був лікарем у тому самому шпиталі, пише
в одному листі до члена комісії Льонге, між іншим, таке: «Я можу
говорити лише на основі своїх особистих спостережень, а не з
статистичних даних, але я не вагаюся запевняти вас, що в мене
раз-у-раз скипало обурення, коли я дивився на цих бідних
дітей, що їхнє здоров’я кидано на поталу ненажерливости їхніх
батьків і працедавців». Він перелічує причини недуг серед ганчарів
і закінчує найголовнішою з них — «long hours» («довгий
робочий день»). Звіт комісії висловлює надію, що «мануфактура,
яка має таке видне становище в очах цілого світу, не плямуватиме
себе більш тим, що її величезні успіхи супроводяться психічним
виродженням, численними тілесними хоробами і передчасною
смертю робітників, що через працю і вправність їхню
досягнуто таких великих результатів»\footnote{
«Children’s Employment Commission, 1863», p. 24, 22, XI.
}. Сказане тут про ганчарні
в Англії стосується також і до ганчарень у Шотляндії\footnote{
Там же, стор. XLVII.
}.

Мануфактура сірників починається з 1833~\abbr{р.}, коли винайдено
спосіб прикріпляти фосфор до самого сірникового патичка. Від
\parbreak{}  %% абзац продовжується на наступній сторінці


\index{iii2}{0189}  %% посилання на сторінку оригінального видання
Одно з найкумедніших явищ є в тому, що всі противники Рікардо, які
заперечують визначення вартости виключно працею, в справі з диференційною
рентою, що випливає з ріжниць землі, надають ваги тому, що тут вартість
визначається природою, а не працею; і одночасно приписують це визначення
положенню, або, і ще більше, процентові на капітал, вкладений в землю при
обробітку. Та сама праця дає однакову вартість для продукту, створеного
протягом даного часу; але величина або кількість цього продукту, отже, і та
частина вартости, яка припадає на відповідну частину цього продукту за даної
кількости праці, залежить єдино від кількости продукту, а це знову від продуктивности
даної кількости праці, не від величини цієї кількости. Чи завдячує
ця продуктивність своїм походженням природі, чи суспільству — цілком байдуже.
Тільки в тому разі, коли вона сама коштує праці, отже, капіталу, вона
збільшує ціну продукції новою складовою частиною, чого природа сама по собі
не робить.

\section{Абсолютна земельна рента}

Аналізуючи диференційну ренту, ми виходили з припущення, що найгірша
земля не виплачує земельної ренти, або, висловлюючись загальніше, що земельну
ренту виплачує тільки така земля, для продукту якої індивідуальна ціна продукції
нижча від ціни продукції, що реґулює ринок, так що в такий спосіб
виникає надзиск, що перетворюється на ренту. Потрібно насамперед зауважити,
що закон диференційної ренти, як днференційної ренти, зовсім не залежить від
правильности чи неправильности того припущення.

Коли загальну ціну продукції, що реґулює ринок, ми назвемо Р, то Р для
продукту найгіршого роду землі А збігається з індивідуальною ціною продукції
на цій землі; тобто вона оплачує зужиткований у продукції сталий і змінний капітал
плюс пересічній зиск (= підприємницькому баришеві плюс процент).

Рента тут дорівнює нулеві. Індивідуальна ціна продукції найближчого
кращого роду землі В = Р', і Р>Р'; тобто Р оплачує більше, ніж дійсну
ціну продукції продукту на клясі землі В. Хай тепер Р — Р' = d; тому
d, надмір Р над Р', є той надзиск, що його добуває орендар з цієї кляси В.
Це d перетворюється на ренту, яку доводиться виплачувати власникові землі.
Хай для третьої кляси землі С за дійсну ціну продукції буде Р", і хай Р —
Р'' = 2d; отже, ці 2d перетворюються на ренту; так само для четвертої кляси
D індивідуальна ціна продукції хай буде Р'", а Р — Р'" = 3d, які перетворюються
на земельну ренту і т. д. Даймо тепер, що припущення, ніби для
кляси землі А рента = 0, а тому ціна її продукту = Р + 0, помилкове. Хай,
навпаки, і вона дає ренту = г. В цьому випадку маємо двоякі наслідки.

\emph{Поперше}: ціна продукту землі кляси А не реґулювалася б ціною продукції
на цій землі, а мала б деякий надмір над цією ціною, вона була б =
P — r. Бо, коли припускається, нормальний перебіг капіталістичного способу
продукції, отже, коли припускається, що надмір r, виплачуваний від орендаря
земельному власникові, не становить вирахування ані з заробітної плати, ані
з пересічного зиску на капітал, то орендар може виплачувати його лише тому,
що його продукт продається понад ціну продукції, що він, отже, дав би йому
надзиск, коли б не доводилося відступати цей надмір у формі ренти земельному
власникові. Реґуляційна ринкова ціна всього наявного на ринку продукту
всіх родів землі була б тоді не та ціна продукції, яку дає капітал взагалі
у всіх сферах продукції, тобто не ціна рівна витратам плюс пересічний
зиск, а була б ціною продукції плюс рента, Р + r, не Р. Бо ціна продукту
кляси А визначає взагалі межу реґуляційної загальної ринкової ціни, тієї ціни,
\parbreak{}  %% абзац продовжується на наступній сторінці

\parcont{}  %% абзац починається на попередній сторінці
\index{ii}{0190}  %% посилання на сторінку оригінального видання
з 100 до 75, або на одну чверть. Ціла сума, що на неї скорочується
продуктивний капітал, який функціонує протягом дев’ятитижневого робочого
періоду, становить 9 × 25 \deq{} 225\pound{ ф. стерл.}, або четверту частину
900\pound{ ф. стерл}. Але відношення часу обігу до періоду обороту, як і раніш,
становить \frac{3}{12} \deq{} \sfrac{1}{4}. З цього випливає ось що. Для того, щоб продукція
не припинялась протягом часу обігу продуктивного капіталу, перетвореного
на товаровий капітал, щоб вона однаково невпинно продовжувалась тиждень
у тиждень, коли немає для цього окремого обігового капіталу, то
цього можна досягти, лише скоротивши продукцію, зменшивши поточну
складову частину діющого продуктивного капіталу. Поточна частина капіталу,
звільнена таким чином для процесу продукції протягом часу обігу,
відноситься до цілого авансованого поточного капіталу, як час обігу до
періоду обороту. Як ми вже зауважили, це має силу тільки для тих галузей
продукції, де процес праці відбувається тиждень-у-тиждень у
тому самому маштабі, де, отже, не треба, як у хліборобстві, в різні робочі
періоди витрачати різні кількості капіталу.

Навпаки, якщо ми припустимо, що самий характер підприємства виключає
можливість скорочення маштабу продукції, а тому й розмірів щотижнево
авансовуваного поточного капіталу, то безперервности продукції
можна досягти лише додачею поточного капіталу, в наведеному вище
прикладі додачею 300\pound{ ф. стерл}. Протягом дванадцятитижневого періоду
обороту послідовно авансується 1200\pound{ ф. стерл.}, з них 300 являють четверту
частину, як 3 тижні від 12. По скінченні 9-тижневого робочого
періоду капітальна вартість в 900\pound{ ф. стерл.} перетворюється з форми
продуктивного капіталу на форму товарового капіталу. Її робочий період
закінчено, але його не можна відновити з тим самим капіталом. Протягом
трьох тижнів, поки цей капітал перебуває в сфері циркуляції, функціонуючи
як товаровий капітал, він щодо продукційного процесу перебуває
в такому самому стані, ніби його взагалі не існувало. Ми лишаємо тут
осторонь усі кредитові відносини, а тому припускаємо, що капіталіст
господарює тільки своїм власним капіталом. Але тимчасом як капітал,
авансований на перший робочий період, завершивши процес продукції,
протягом 3 тижнів перебуває в процесі циркуляції, — у цей самий час
функціонує додатково витрачений капітал в 300\pound{ ф. стерл.}, так що безперервність
продукції не порушується.

Тут треба зауважити ось що:

Поперше. Робочий період авансованого спочатку капіталу в 900\pound{ ф.
стерл.} закінчується по 9 тижнях, але капітал припливає назад не раніш,
як по трьох тижнях, отже, лише на початку 13-го тижня. Однак новий
робочий період починається негайно за допомогою додаткового капіталу
в 300\pound{ ф. стерл}. Саме в наслідок цього підтримується безперервність
продукції.

Подруге. Функції первісного капіталу в 900\pound{ ф. стерл.} і новододаного
наприкінці першого дев’ятитижневого робочого періоду капіталу в 300\pound{ ф. стерл.}, який відкриває другий робочий період безпосередньо по закінченні
\parbreak{}  %% абзац продовжується на наступній сторінці

\parcont{}  %% абзац починається на попередній сторінці
\index{ii}{0191}  %% посилання на сторінку оригінального видання
першого, ці функції за першого періоду обороту точно відмежовані одна
від однієї, або принаймні їх можна точно відмежувати, тимчасом як протягом
другого періоду обороту вони, навпаки, переплітаються одна з
однією.

Уявімо собі справу наочніше:

Перший період обороту триває 12 тижнів. Перший робочий період —
9 тижнів; оборот авансованого на нього капіталу закінчується на початку
13-го тижня. Протягом останніх 3 тижнів функціонує додатковий капітал
в 300\pound{ ф. стерл.}, який починає другий дев’ятитижневий робочий
період.

Другий період обороту. На початку 13-го тижня 900\pound{ ф. стерл.} припливають
назад і можуть почати новий оборот. Але другий робочий
період уже на десятому тижні почато за допомогою додаткових 300\pound{ ф.
стерл.}; на початку 13-го тижня за допомогою тих самих 300\pound{ ф. стерл.}
уже закінчено третину робочого періоду, 300\pound{ ф. стерл.} з продуктивного
капіталу перетворено на продукт. А що для закінчення другого робочого
періоду треба ще лише 6 тижнів, то в процес продукції другого робочого
періоду можуть ввійти лише дві третини капіталу в 900\pound{ ф. стерл.},
який повернувся назад, а саме 600\pound{ ф. стерл}. З первісних 900\pound{ ф. стерл.}
звільнилося 300\pound{ ф. стерл.}, щоб відігравати ту саму ролю, яку відігравав
у першому робочому періоді додатковий капітал в 300\pound{ ф. стерл}. Наприкінці
6-го тижня другого періоду обороту закінчено другий робочий
період. Витрачений на нього капітал в 900\pound{ ф. стерл.} повертається за три
тижні, отже, наприкінці 9-го тижня другого дванадцятитижневого періоду
обороту. Протягом 3 тижнів його часу обігу ввіходить у робочий період
звільнений капітал в 300\pound{ ф. стерл}. З ним починається на 7-й тиждень
другого періоду обороту або на 19-й тиждень року третій робочий
період капіталу в 900\pound{ ф. стерл}.

Третій період обороту. Наприкінці 9-го тижня другого періоду обороту
знову зворотно припливають 900\pound{ ф. стерл}. Але третій робочий
період почався вже на сьомому тижні попереднього періоду обороту й
6 тижнів його вже минуло. Тому він триває тільки три тижні. Отже,
з 900\pound{ ф. стерл.}, що повернулись назад, у процес продукції ввіходять
лише 300\pound{ ф. стерл}. Четвертий робочий період заповнює дев’ятитижневу
решту цього періоду обороту, і таким чином з 37-го тижня року починається
одночасно четвертий період обороту й п’ятий робочий період.

Щоб спростити обчислення, ми припустимо робочий період в 5 тижнів,
час обігу в 5 тижнів, отже, період обороту в 10 тижнів; рік рахуватимемо
в 50 тижнів, а щотижневу витрату капіталу рахуватимемо в 100\pound{ ф.
стерл}. Отже, робочий період потребує поточного капіталу в 500\pound{ ф. стерл.},
а час обігу потребує додаткового капіталу — нових 500\pound{ ф. стерл}. Робочі
періоди й періоди оборотів позначиться тоді так:

1-й робочий період: тижні 1--5 (500\pound{ ф. стерл.} товару повертаються
наприкінці 10 тижня).

2-й робочий період: тижні 6--10 (500\pound{ ф. стерл.} товару повертаються
наприкінці 15 тижня).


\index{ii}{0192}  %% посилання на сторінку оригінального видання
3-й робочий період: тижні 11--15 (500\pound{ ф. стерл.} товару повертаються
наприкінці 20 тижня).

4-й робочий період: тижні 16--20 (500\pound{ ф. стерл.} товару повертаються
наприкінці 25 тижня).

5-й робочий період: тижні 21--25 (500\pound{ ф. стерл.} товару повертаються
наприкінці 30 тижня) і~\abbr{т. д.}

Коли час обігу дорівнює 0, отже, коли період обороту дорівнює
робочому періодові, то число оборотів дорівнює числу робочих періодів
на рік.
Отже, при п’ятитижневому робочому періоді воно було б \deq{} \sfrac{50}{5}
тижнів, тобто 10, а вартість капіталу, що обернувся, була б 6 \deq{} 500×10 \deq{}
5000.
В таблиці, де час обігу припущено в 5 тижнів, так само щороку
продукується товарів вартістю в 5000\pound{ ф. стерл.}, але з них \sfrac{1}{10} \deq{} 500\pound{ ф.
стерл.} завжди перебуває у вигляді товарового капіталу й повертається
назад лише по 5 тижнях. Наприкінці року продукт десятого робочого
періоду (46--50 робочі тижні) закінчив лише половину свого часу
обороту, при чому його час обігу припадає на перші 5 тижнів наступного року.

Візьмімо ще третій приклад: робочий період 6 тижнів, час обігу
З тижні, щотижневе авансування на процес праці 100\pound{ ф. стерл}.

1-й робочий період: тижні 1--6. Наприкінці 6-го тижня є товаровий
капітал в 600\pound{ ф. стерл.}, він повертається наприкінці 9-го тижня.

2-й робочий період: тижні 7--12. Протягом 7--9-го тижнів авансовано
300\pound{ ф. стерл.} додаткового капіталу. Наприкінці 9-го тижня повертаються
назад 600\pound{ ф. стерл}. З них протягом 10--12 тижнів авансовано 300\pound{ ф.
стерл.}; отже, наприкінці 12 тижня є вільних 300\pound{ ф. стерл.} і в товаровому
капіталі є 600\pound{ ф. стерл.}, що повертаються наприкінці 15 тижня.

3-й робочий період: тижні 13--18. Протягом 13--15-го тижнів авансується
вищезгадані 300\pound{ ф. стерл.}, потім повертаються назад 600\pound{ ф. стерл.},
з них 300 ф. сторл. авансується на 16--18 тижні. Наприкінці 18-го тижня
є вільних 300\pound{ ф. стерл.} грішми; 600\pound{ ф. стерл.} є в товаровому капіталі,
що повертаються наприкінці 21 тижня (див. докладніший виклад
цього випадку далі під II).

Отже, протягом 9 робочих періодів (= 54 тижням) продукується товару
на 600×9 \deq{} 5400\pound{ ф. стерл}. Наприкінці дев’ятого робочого періоду
капіталіст має 300\pound{ ф. стерл.} грішми і 600\pound{ ф. стерл.} товаром, що не
проробив ще свого часу обігу.

Порівнюючи ці три приклади, ми бачимо, поперше, що лише в другому
прикладі відбувається послідовна зміна капіталу І в 500\pound{ ф. стерл.}
і додаткового капіталу II так само в 500\pound{ ф. стерл.}, так що ці дві частини
капіталу рухаються відокремлено одна від однієї і саме лише тому, що
тут припускається цілком винятковий випадок, що робочий період і час
обігу становлять дві однакові половини періоду обороту. В усіх інших
випадках, хоч яка буде нерівність між двома періодами цілого періоду
обороту, рух обох капіталів навзаєм переплітається, як у прикладах
І і III, вже починаючи з другого періоду обороту. В цих випадках додатковий
\index{ii}{0193}  %% посилання на сторінку оригінального видання
капітал II разом з частиною капіталу І становлять капітал, що функціонує
в другому періоді обороту, тимчасом як решта капіталу І звільняється
для первісної функції капіталу II. Капітал, діющий протягом часу
обігу товарового капіталу, тут не тотожній з капіталом II, первісно авансованим
для цього, але він дорівнює йому вартістю і становить таку саму
частину цілого авансованого капіталу.

Подруге. Капітал, що функціонував протягом робочого періоду,
лежить без діла протягом часу обігу. В другому прикладі капітал функціонує
протягом 5 тижнів робочого періоду й лежить без діла протягом
5 тижнів часу обігу. Отже, увесь цей час, що його капітал І тут на протязі
року лежить без діла, дорівнює половині року. На цей час тоді ввіходить
додатковий капітал II, що, отже, і собі у даному випадку теж лежить без діла
протягом півроку. Але додатковий капітал, потрібний для того, щоб підтримати
безперервну продукцію протягом часу обігу, визначається не
всією величиною, зглядно не сумою часів обігу протягом року, а лише
відношенням часу обігу до періоду обороту. (Тут, звичайно, припускається,
що всі обороти відбуваються в однакових умовах). Тому в прикладі
II додаткового капіталу треба 500\pound{ ф. стерл.}, а не 2500\pound{ ф. стерл}.
Це пояснюється просто тим, що додатковий капітал увіходить в оборот
цілком так само, як і первісно авансований, отже, цілком так само, як
і цей останній, числом своїх оборотів заміщує свою масу.

Потретє. Коли час продукції довший, ніж робочий час, то це нічого
не змінює в розглянутих тут обставинах. В наслідок цього в усякому
разі подовшає цілий період обороту, але при такому подовшанні обороту
не треба жодного додаткового капіталу для процесу праці. Додатковий
капітал призначається тільки на те, щоб заповнити прогалини в процесі
праці, зумовлені часом обігу; отже, він повинен лише захищати
продукцію від тих порушень, що походять з часу обігу, а порушення,
що постають з власне умов продукції, вирівнюється іншим способом, що
його ми тут не будемо розглядати. Навпаки, є такі підприємства, де роблять
лише з перервами, на замовлення, де, отже, можуть бути перерви
між робочими періодами. В таких підприємствах pro tanto відпадає потреба
в додатковому капіталі. З другого боку, в більшості випадків сезоновнх
робіт дано й певну межу для часу зворотного припливу капіталу.
Ту саму роботу в наступному році не можна відновити тим самим капіталом,
коли час циркуляції цього капіталу в проміжний час не скінчився. Навпаки,
час обігу може бути й коротший від переміжку між одним періодом
продукції й наступним. В цьому випадку капітал лежить без діла,
якщо в цей проміжний час не застосується його інакше.

Почетверте. Капітал, авансований на один робочий період, напр.,
600\pound{ ф. стерл.} в прикладі III, витрачається почасти на сировинні й допоміжні
матеріяли, на продуктивний запас для робочого періоду, на сталий обіговий
капітал, а почасти на змінний обіговий капітал, на оплату самої
праці. Частина, витрачена на сталий обіговий капітал, може існувати в
формі продуктивного запасу не однаково довгий час, напр., сировинний
матеріял можна запасати не на ввесь робочий період, вугілля можна
\parbreak{}  %% абзац продовжується на наступній сторінці

\input{_0194_0195.tex}
\input{_0196.tex}
\parcont{}  %% абзац починається на попередній сторінці
\index{i}{0197}  %% посилання на сторінку оригінального видання
«Наші «білі раби», — вигукнув «Morning Star», орган панів
фритредерів Кобдена й Брайта, — наші білі раби запрацьовуються
на смерть і гинуть і вмирають без найменшого шуму»\footnote{
«Morning Star» з 23 липня 1863~\abbr{р.} «Times» скористався цим випадком
для оборони американських рабовласників проти Брайта й~\abbr{т. ін.}
«Дуже багато з нас, — каже «Times», — гадають, що лоти, доки ми сами
вимучуємо на смерть працею наших власних молодих жінок, погрожуючи
їм ударами голоду замість свисту батога, доти ми ледве чи маємо право
йти мечем і вогнем на ті родини, що їхні члени родилися рабовласниками
та які принаймні добре годують своїх рабів і вимагають від них лише
помірної праці» («Times», а 2 липня 1863~\abbr{р.}). Газета торів «Standart»
розправлялась у тому самому дусі з його преподобієм Ньюмен Холлом:
«Він відлучує від церкви рабовласників, але молиться разом із порядними
людьми, що примушують працювати за собачу плату лондонських візників
та кондукторів омнібусів і~\abbr{т. ін.} лише по 16 годин на день». Нарешті,
пролунав голос оракула, винахідника культу генія, пана Томаса Карлейля,
про якого я вже року 1850 писав: «Геній пішов к чорту, лишився культ».
В коротенькій притчі він зводить єдину величну подію сучасної історії,
американську громадянську війну, на те, що Петро з півночі з усіх сил
намагається переломити черепа Павлові з півдня, бо Петро з півночі
наймає свого робітника «поденно», а Павло з півдня — «на ціле життя».
(«Macmillan’s Magazine». Ilias Americana in nuce. Серпневий зошит
1863~\abbr{р.}). Так луснув, нарешті, шумовинний пухир торійських симпатій
до міських — але ні в якому разі не до сільських! — найманих робітників.
Основа цих симпатій — це рабство!
}.

«Запрацьовуватись на смерть — це є порядок дня не лише
в майстернях кравчих, але в тисячах місць, ба на кожному місці,
де справи йдуть добре\dots{} Візьмімо як приклад коваля. Як вірити
поетам, то немає в світі людини сильнішої, веселішої за коваля.
Він устає раннім ранком і викрешує іскри перед тим, як засяє
сонце; нема такої людини, що так їла б, так пила б і спала, як
він. Якщо поглянути на долю коваля чисто з фізичного боку, то,
дійсно, за помірної праці, становище його одне з найкращих.
Але ходімо за ним до міста й погляньмо на той тягар праці, який
накладають на його дужі плечі, погляньмо на місце, яке він посідає
у таблицях смертности нашої країни? У Marylebone (один із
найбільших міських кварталів Лондону) смертність ковалів становить
31 на 1000 на рік, а це на 11 перевищує пересічну смертність
дорослих чоловіків Англії. Праця ця, майже інстинктова вмілість
людини, сама по собі бездоганна, через саму лише надмірність
стає руйнаційною для людини. Людина може зробити стільки й
стільки ударів молотом на день, стільки й стільки кроків, стільки
й стільки разів дихнути, стільки й стільки зробити якоїсь роботи
й прожити пересічно, приміром, 50 років. Її примушують робити
на стільки більше вдарів, на стільки більше кроків, стільки частіш
віддихувати, а це все разом збільшує її життєве завдання на
одну четвертину на день. Вона силкується це все робити, а результат
такий, що за обмежений період вона виконує на четвертину
більшу роботу і вмирає на 37 році замість на 50»\footnote{
\emph{Dr.~Richardson}: «Work and Overwork» y «Social Science Review»,
18 липня 1863.
}.

\input{_0198_0199_0200.tex}
\parcont{}  %% абзац починається на попередній сторінці
\index{iii1}{0201}  %% посилання на сторінку оригінального видання
спеціально ним занятих робітників, окремий капіталіст, в відміну
від своєї сфери виробництва, має в експлуатації робітників, експлуатованих
ним особисто.

З другого боку, кожна окрема сфера капіталу і кожний окремий
капіталіст однаково заінтересовані в продуктивності суспільної
праці, вживаної сукупним капіталом. Бо від цього залежать
дві обставини: поперше, маса споживних вартостей, в якій виражається
пересічний зиск; а це подвійно важливо, оскільки цей
зиск служить як фондом нагромадження нового капіталу, так
і фондом доходу, призначеного для споживання. Подруге, висота
вартості авансованого сукупного капіталу (сталого і змінного),
яка, при даній величині додаткової вартості або зиску всього
класу капіталістів, визначає норму зиску, або зиск на певну
кількість капіталу. Особлива продуктивність праці в певній
особливій сфері виробництва або в певному окремому підприємстві
цієї сфери інтересує тільки тих капіталістів, які безпосередньо
беруть участь у цій сфері або в цьому підприємстві,
оскільки така продуктивність дає можливість окремій сфері
виробництва одержувати додатковий зиск порівняно з сукупним
капіталом або окремому капіталістові — порівняно з його сферою.

Отже, ми маємо тут математично точне пояснення того,
чому капіталісти, якими б вони не виявлялись зрадливими братами
у своїй конкуренції між собою, становлять, проте, справжній
масонський союз проти робітничого класу як цілого.

Ціна виробництва включає в собі пересічний зиск. Ми назвали
її ціною виробництва; фактично вона є те саме, що А. Сміт називає
natural price [природною ціною], Рікардо — price of production,
cost of production [ціною виробництва, витратами виробництва],
фізіократи — prix nécessaire [необхідною ціною], — при
чому ніхто з них не дослідив відмінності ціни виробництва від
вартості, — бо ціна виробництва є постійна умова подання, репродукції
товарів кожної окремої сфери виробництва.\footnote{
\emph{Мальтус}
} Зрозуміло
також, чому ті самі економісти, які повстають проти визначення
вартості товарів робочим часом, кількістю вміщеної в них праці,
завжди говорять про ціни виробництва, як про центри, навколо
яких коливаються ринкові ціни. Вони можуть собі дозволити це,
бо ціна виробництва є вже цілком відчуженою і prima facie [явно]
ірраціональною формою товарної вартості, формою, як вона виступає
в конкуренції, отже, у свідомості вульгарного капіталіста,
а тому також і в свідомості вульгарних економістів.

\pfbreak

З вищевикладеного виявилось, яким чином ринкова вартість
(а все сказане про неї стосується з необхідними обмеженнями
і до ціни виробництва) містить у собі надзиск тих, що в кожній
окремій сфері виробництва виробляють при найкращих
\parbreak{}  %% абзац продовжується на наступній сторінці

\input{_0202.tex}

\index{iii1}{0203}  %% посилання на сторінку оригінального видання
Візьмімо тепер капітал, склад якого є нижчий, ніж первісний
склад пересічного суспільного капіталу $80 c + 20 v$ (який
перетворився тепер в $76\sfrac{4}{21}c + 23\sfrac{17}{21}v$), наприклад, $50 c + 50 v$.
Тут ціна виробництва річного продукту, — якщо ми для спрощення
припустимо, що весь основний капітал увійшов як зношування
в річний продукт і що час обороту такий самий, як
і в випадку I, — становила перед підвищенням заробітної плати
$50 c + 50 v + 20 p = 120$. Підвищення заробітної плати на 25\%
дає для тієї самої кількості приведеної в рух праці підвищення
змінного капіталу з 50 до 62\sfrac{1}{2}. Коли б річний продукт був
проданий по попередній ціні виробництва в 120, то це дало б
$50 c + 62\sfrac{1}{2}v + 7\sfrac{1}{2}p$, тобто норму зиску в 6\sfrac{2}{3}\%.
Але нова пересічна норма зиску є 14\sfrac{2}{7}\%, і через те що ми всі інші умови
припускаємо незмінними, цей капітал в $50 c + 62\sfrac{1}{2}v$ так само
мусить дати вказаний зиск. Але капітал в 112\sfrac{1}{2}, при нормі зиску
в 14\sfrac{2}{7}, дає 16\sfrac{1}{14} зиску.\footnote*{
В першому німецькому виданні тут сказано: „в круглих числах 16\sfrac{1}{12}
зиску“; відповідно до цього Енгельс обчислює потім ціну виробництва в 128\sfrac{7}{12}
В рукопису Маркса дано точне число в 16\sfrac{3}{42}, яке нами взяте з відповідним
скороченням дробу і застосоване при обчисленні ціни виробництва.
\emph{Примітка ред. нім. вид. ІМЕЛ.}
} Отже, ціна виробництва вироблених
ним товарів є тепер $50 c + 62\sfrac{1}{2}v + 16\sfrac{1}{14}p = 128\sfrac{8}{14}$. Отже, в наслідок
підвищення заробітної плати на 25\% ціна виробництва
тієї самої кількості того самого товару підвищилась тут з 120
до 128\sfrac{8}{14}, або більше ніж на 7\%.

Візьмім, навпаки, сферу виробництва вищого складу, ніж пересічний
капітал, наприклад, $92 c + 8 v$. Отже, первісний пересічний
зиск і тут = 20, і якщо ми знову припустимо, що весь
основний капітал входить у річний продукт і що час обороту
такий самий, як і в випадках І і II, то ціна виробництва товару
й тут = 120.

В наслідок підвищення заробітної плати на 25\% змінний капітал
для тієї самої кількості праці зростає з 8 до 10, отже
витрати виробництва товарів зростають з 100 до 102; з другого
боку, пересічна норма зиску впала з 20\% до 14\sfrac{2}{7}\%. Але
$100 : 14\sfrac{2}{7} = 102 : 14\sfrac{4}{7}$\footnote*{
В першому німецькому виданні тут стоїть: „(приблизно)“. В рукопису
Маркса цього слова немає. В дійсності тут рівняння точне, а не тільки приблизне.
\emph{Примітка ред. нім. вид. ІМЕЛ.}
}. Отже, зиск, що припадає тепер на 102,
становить 14\sfrac{4}{7}. І тому весь продукт продається за
$k + kp' = 102 + 14\sfrac{4}{7} = 116\sfrac{4}{7}$. Отже, ціна виробництва впала
з 120 до 116\sfrac{4}{7}, або майже на 3\%\footnote*{
В першому німецькому виданні тут сказано: „більше ніж на 3\%. В рукопису
Маркса стоїть: „на 3\sfrac{3}{7}“, тобто дано абсолютне число. В процентах воно
дорівнює 2\sfrac{6}{7}\%. \emph{Примітка ред. нім. вид. ІМЕЛ.}
}.

Отже, в наслідок підвищення заробітної плати на 25\%:

1) для капіталу пересічного суспільного складу ціна виробництва
товару лишилась незмінною;

2) для капіталу нижчого складу ціна виробництва товару
\parbreak{}  %% абзац продовжується на наступній сторінці

\input{_0204.tex}
\input{_0205.tex}
\input{_0206.tex}

\index{ii}{0207}  %% посилання на сторінку оригінального видання
\subsection{Висновки}

З попереднього досліду випливає:

А.~Різні частки, на які треба поділити капітал, щоб одна з частин
його постійно могла бути в робочому періоді, тимчасом як друга перебуває
в періоді циркуляції, — чергуючись змінюють одна одну, як різні
самостійні приватні капітали, в двох випадках: 1) коли робочий період
дорівнює періодові циркуляції, коли, отже, період обороту розпадається
на два однакові відділи; 2) коли період циркуляції довший, ніж робочий
період, але разом з тим становить просте кратне робочого періоду, так
що один період циркуляції дорівнює n робочим періодам, де n мусить
бути цілим числом. В цих випадках жодна частина послідовно авансованого
капіталу не звільняється.

В.~Навпаки, в усіх тих випадках, коли 1) період циркуляції більший,
ніж робочий період, і не становить простого кратного йому і 2) коли
робочий період більший, ніж період циркуляції, то наприкінці кожного
робочого періоду, починаючи з другого обороту, постійно й періодично
звільняється частина цілого поточного капіталу. При цьому, коли робочий
період більший, ніж період циркуляції, то цей звільнений капітал
дорівнює частині цілого капіталу, авансованій на період циркуляції, а
коли період циркуляції більший, ніж робочий період, то цей звільнений
капітал дорівнює частині капіталу, що має поповнювати надлишок періоду
циркуляції проти робочого періоду або проти кратного робочих періодів.

С.~З цього випливає, що для сукупного суспільного капіталу, розглядуваного
з погляду його поточної частини, звільнення капіталу
мусить бути загальним правилом, а просте чергування частин капіталу,
що послідовно функціонують у процесі продукції, — винятком. Бо однаковість
робочого періоду й періоду циркуляції або однаковість періоду
циркуляції та простого кратного робочого періоду, така правильна пропорційність
двох складових частин періоду обороту не має жодного
чинення до суті справи і тому взагалі та в цілому може траплятись лише
винятково.

Отже, дуже значна частина суспільного обігового капіталу, що робить
кілька оборотів протягом року, буде протягом річного циклу оборотів
періодично перебувати в формі звільненого капіталу.

Далі зрозуміло, що, припускаючи всі інші умови незмінними, величина
цього звільненого капіталу зростає разом з розміром процесу праці
або з маштабом продукції, отже, взагалі з розвитком капіталістичної
продукції. У випадку, позначеному під В, 2) — тому, що зростає ввесь
авансований капітал; в випадку В, 1) — тому, що з розвитком капіталістичної
продукції зростає протяг періоду циркуляції, а значить, і період
обороту, в тих випадках, коли немає правильного відношення між робочим
періодом і періодом циркуляції.

В першому випадку ми повинні були щотижня витрачати, напр.,
100\pound{ ф. стерл}. Для шеститижневого робочого періоду 600\pound{ ф. стерл.}, для тритижневого
періоду циркуляції 300\pound{ ф. стерл.}, разом 900\pound{ ф. стерл}. Тут буде
\parbreak{}  %% абзац продовжується на наступній сторінці


\index{iii1}{0208}  %% посилання на сторінку оригінального видання
\subsection{Ціна виробництва товарів середнього складу}

Ми бачили, яким чином відхилення цін виробництва від вартостей постає в наслідок того:

1) що до витрат виробництва товару додається не додаткова вартість, вміщена в ньому, а
пересічний зиск;

2) що ціна виробництва товару, яка таким чином відхиляється від вартості, входить як елемент
у витрати виробництва інших товарів, в наслідок чого, отже, вже у витратах виробництва товару може
міститись відхилення від вартості спожитих на нього засобів виробництва, незалежно від того
відхилення, що може постати для самого цього товару в наслідок ріжниці між пересічним зиском і
додатковою вартістю.

Таким чином, можливо, що і в товарів, вироблених капіталами середнього складу, витрати виробництва
відхилятимуться від суми вартості елементів, з яких складається ця складова частина їх ціни
виробництва. Припустім, що середній склад є $80 c + 20 v$. Можливо, що в дійсних капіталах, які мають
такий склад, $80  c$ більше або менше вартості $с$, сталого капіталу, бо це $с$ складається з товарів,
ціна виробництва яких відхиляється від їх вартості. Так само $20 v$  могли б відхилятися від своєї
вартості, якщо в споживання заробітної плати входять товари, ціна виробництва яких відрізняється від
їх вартості; отже, робітник, щоб купити ці товари (замістити їх), мусить витратити більше або менше
робочого часу, отже, мусить виконати більше або менше необхідної праці, ніж потрібно було б, коли б
ціни виробництва
необхідних засобів існування збігалися з їх вартостями.

Однак, ця можливість зовсім не міняє правильності положень, встановлених для товарів
середнього складу. Кількість зиску, що припадає на ці товари, дорівнює кількості вміщеної в них
самих додаткової вартості. Наприклад, при наведеному вище капіталі з складом у $80 с + 20 v$ для
визначення додаткової вартості важливе не те, чи ці числа є вирази дійсних вартостей, а те, як вони
відносяться одне до одного; а саме, що $v = \frac{1}{5}$, а $с = \frac{4}{5}$  всього капіталу. Якщо це так, то
додаткова вартість, вироблена $v$, дорівнює, як ми це припустили вище, пересічному зискові. З другого
боку: через те що додаткова вартість дорівнює пересічному зискові, ціна виробництва = витратам
виробництва + зиск = $k + p = k + m$, на практиці дорівнює вартості товару. Тобто підвищення або
зниження заробітної плати лишає $k + p$  в цьому випадку так само незмінним, як воно лишило б
незмінною вартість товару, і викликає тільки відповідний зворотний рух,
зниження або підвищення, на стороні норми зиску. А саме, якщо в наслідок підвищення або зниження
заробітної плати тут змінилася б ціна товарів, то норма зиску в цих сферах середнього складу стала б
вищою або нижчою порівняно з її рівнем в інших
\index{iii1}{0209}  %% посилання на сторінку оригінального видання
сферах. Лиш оскільки ціна лишається незмінною, сфера
середнього складу зберігає свій рівень зиску однаковим з іншими сферами. Отже, на практиці в цій
сфері справа відбувається цілком так само, як коли б продукти цієї сфери продавались по їх дійсній
вартості. А саме, якщо товари продаються по їх дійсних вартостях, то очевидно, що при інших
однакових умовах підвищення або зниження заробітної плати викликає відповідне зниження або підвищення зиску,
але не викликає ніякої зміни вартості товарів, і що при всіх обставинах підвищення або зниження
заробітної плати ніколи не може вплинути на вартість товарів, а завжди тільки на величину додаткової
вартості.

\subsection{Підстави капіталіста для компенсації}

Уже було сказано, що, конкуренція вирівнює норми зиску різних сфер виробництва в пересічну норму
зиску і саме тим перетворює вартості продуктів цих різних сфер виробництва в ціни виробництва. І це
стається саме в наслідок постійного перенесення капіталу з однієї сфери виробництва до іншої, де в
даний момент зиск стоїть вище пересічного рівня; при цьому, однак, слід взяти до уваги коливання
зиску, зв’язані з чергуванням худих і ситих років в даній галузі промисловості на протязі даного
періоду часу. Ця безперервна еміграція та імміграція капіталу, яка відбувається між різними сферами
виробництва, породжує висхідні і низхідні рухи норми зиску, які більше чи менше взаємно
урівноважуються і через це мають тенденцію повсюди зводити норму зиску до того самого спільного й загального рівня.

Цей рух капіталів завжди викликається в першу чергу станом ринкових цін, які в одному місці
підвищують зиск понад загальний пересічний рівень, в другому — знижують його нижче цього рівня. Ми
покищо залишаємо осторонь купецький капітал, з яким ми тут ще не маємо справи і який, як це
показують пароксизми спекуляції з певними улюбленими товарами, що раптово вибухають, може з
надзвичайною швидкістю витягати маси капіталу з одної галузі застосування і так само швидко кидати
їх до іншої. Але в кожній сфері виробництва у власному розумінні слова — в промисловості,
землеробстві, рудниках і т. д. — перенесення капіталу з однієї сфери в іншу становить значні
труднощі, особливо в наслідок наявності основного капіталу. До того ж досвід показує, що коли
яканебудь галузь промисловості,
наприклад, бавовняна промисловість, в певний час дає надзвичайно високий зиск, то вона потім, в
інший час, дає дуже незначний зиск, а то навіть і збиток, так що за певний цикл років пересічний
зиск в ній приблизно такий самий, як і в інших галузях. І капітал швидко привчається зважати на цей
досвід.

Але чого конкуренція \emph{не} показує, так це визначення вартості, яке керує рухом виробництва; так це
вартостей, які стоять за
\parbreak{}  %% абзац продовжується на наступній сторінці

\input{_0210_0211.tex}

  
\index{iii1}{0212}  %% посилання на сторінку оригінального видання

\chapter{Закон тенденції норми зиску до падіння}

\section{Закон як такий}

При даній заробітній платі і при даному робочому дні змінний
капітал, наприклад, в 100, представляє певне число приведених
у рух робітників; він є показник цього числа. Припустімо,
наприклад, що 100 фунтів стерлінгів становлять заробітку плату
100 робітників, скажімо, за 1 тиждень. Якщо ці 100 робітників
виконують стільки ж необхідної праці, скільки додаткової праці,
якщо вони, отже, щодня працюють стільки ж часу на себе
самих, тобто для репродукції своєї заробітної плати, скільки
на капіталістів, тобто для виробництва додаткової вартості, то
вся вироблена ними вартість буде = 200 фунтам стерлінгів,
а вироблена ними додаткова вартість становитиме 100 фунтів
стерлінгів. Норма додаткової вартості \frac{m}{v} була б = 100\%. Однак,
ця норма додаткової вартості, як ми бачили, виражалася б у дуже
різних нормах зиску, залежно від різного розміру сталого капіталу
$c$, а тому й усього капіталу $K$, бо норма зиску $ = \frac{m}{K}$. При нормі
додаткової вартості в 100\%,

\begin{center}
якщо $c = \phantom{0}50$, $v = 100$, то $р' = \frac{100}{150} = 66\frac{2}{3}\%$;

якщо $c = 100$, $v = 100$, то $р' = \frac{100}{200} = 50\phantom{\frac{1}{1}}\%$;

якщо $c = 200$, $v = 100$, то $р' = \frac{100}{300} = 33\frac{1}{3}\%$;

якщо $c = 300$, $v = 100$, то $р' = \frac{100}{400} = 25\phantom{\frac{1}{1}}\%$;

якщо $c = 400$, $v = 100$, то $р' = \frac{100}{500} = 20\phantom{\frac{1}{1}}\%$.
\end{center}

Таким чином при незмінному ступені експлуатації праці та
сама норма додаткової вартості виражалася б у падаючій нормі
зиску, бо разом з матеріальним розміром сталого капіталу зростає,
\index{iii1}{0213}  %% посилання на сторінку оригінального видання
хоч і не в тій самій пропорції, і розмір вартості сталого,
а разом з ним і всього капіталу.

Якщо ми далі припустимо, що ця ступнева зміна в складі
капіталу відбувається не тільки в окремих сферах виробництва,
але більш-менш в усіх або, принаймні, у вирішальних сферах
виробництва, так що вона таким чином рівнозначна зміні в пересічному
органічному складі сукупного капіталу, належного певному
суспільству, то таке ступневе наростання сталого капіталу
порівняно з змінним неминуче мусить мати своїм результатом
\emph{ступневе зниження загальної норми зиску} при незмінній нормі
додаткової вартості, або при незмінному ступені експлуатації
праці капіталом. Але виявилось, як закон капіталістичного способу
виробництва, що з розвитком цього способу виробництва
відбувається відносне зменшення змінного капіталу порівняно
з сталим капіталом і, отже, порівняно з усім капіталом, який
приводиться в рух. Це означає тільки те, що те саме число
робітників, та сама кількість робочої сили, якою можна розпоряджатися
при змінному капіталі даного розміру вартості, в наслідок
особливих методів виробництва, що розвиваються в капіталістичному
виробництві, за той самий час приводить в рух,
переробляє, продуктивно споживає постійно зростаючу масу
засобів праці, машин і всякого роду основного капіталу, сировинних
і допоміжних матеріалів, отже і сталий капітал постійно
зростаючого розміру вартості. Це прогресуюче відносне зменшення
змінного капіталу порівняно з сталим і, отже, з усім капіталом,
тотожне з дедалі вищим пересічним органічним складом
суспільного капіталу. Це — так само тільки інший вираз
прогресуючого розвитку суспільної продуктивної сили праці,
який виявляється саме в тому, що за допомогою зростаючого
застосування машин і взагалі основного капіталу при тому самому
числі робітників за той самий час, тобто з меншою кількістю
праці, перетворюється в продукти більша кількість сировинних
і допоміжних матеріалів. Цьому зростаючому розмірові вартості
сталого капіталу — хоч він тільки віддалено представляє зростання
дійсної маси споживних вартостей, з яких речево складається
сталий капітал — відповідає зростаюче здешевлення продукту.
Кожний індивідуальний продукт, розглядуваний сам по
собі, містить у собі меншу суму праці, ніж на нижчому ступені виробництва,
де відношення капіталу, витраченого на працю, до
капіталу, витраченого на засоби виробництва, є незрівняно
більша величина. Отже, гіпотетичний ряд, наведений нами на
початку цього розділу, виражає дійсну тенденцію капіталістичного
виробництва. Це останнє разом з прогресуючим відносним
зменшенням змінного капіталу порівняно з сталим створює дедалі
вищий органічний склад сукупного капіталу, безпосереднім
наслідком чого є те, що норма додаткової вартості при незмінному
і навіть при зростаючому ступені експлуатації праці
виражається в дедалі нижчій загальній нормі зиску. (Далі буде
\parbreak{}  %% абзац продовжується на наступній сторінці

\input{_0214.tex}
\input{_0215.tex}
\input{_0216_0217.tex}
\parcont{}  %% абзац починається на попередній сторінці
\index{iii1}{0218}  %% посилання на сторінку оригінального видання
з двох мільйонів до трьох. Проте, не зважаючи на це зростання
абсолютної маси додаткової праці, а тому й додаткової
вартості на 50\%, відношення змінного капіталу до сталого
впало б з $2 : 4$ до $3 : 15$, і відношення додаткової вартості до
всього капіталу було б таке (в мільйонах):

\begin{gather*}
\text{\phantom{I}I. }\phantom{1}4с + 2v + 2m; K = \phantom{1}6, p' = 33\sfrac{1}{3}\%.\\
\text{II. }15с + 3v + 3m; K = 18, p' = 16\sfrac{1}{3}\%.
\end{gather*}

\noindent{}Тимчасом як маса додаткової вартості підвищилась наполовину,
норма зиску впала наполовину порівняно з попередньою.
Але зиск є тільки додаткова вартість, обчислена на суспільний
капітал, і тому маса зиску, його абсолютна величина, розглядувана
з точки зору всього суспільства, дорівнює абсолютній величині
додаткової вартості. Отже, абсолютна величина зиску, його
сукупна маса, зросла б на 50\%, не зважаючи на величезне зменшення
цієї маси зиску відносно авансованого сукупного капіталу
або не зважаючи на величезне зменшення загальної норми зиску.
Отже, число вживаних капіталом робітників, тобто абсолютна
маса праці, яка ним приводиться в рух, тому й абсолютна маса
вбираної ним додаткової праці, тому й маса виробленої ним додаткової
вартості, тому й абсолютна маса виробленого ним зиску
\emph{може} зростати і зростати прогресивно, не зважаючи на прогресивне
падіння норми зиску. Це не тільки \emph{може} бути. Це — залишаючи
осторонь минущі коливання — \emph{мусить} так бути на базі
капіталістичного виробництва.

Капіталістичний процес виробництва є разом з тим істотно і
процес нагромадження. Ми показали, як з розвитком капіталістичного
виробництва маса вартості, яка мусить бути просто
репродукована, збережена, збільшується і зростає разом з
підвищенням продуктивності праці, навіть якщо вживана робоча
сила лишається незмінною. Але з розвитком суспільної продуктивної
сили праці ще більше зростає маса вироблюваних споживних
вартостей, частину яких становлять засоби виробництва.
А добавна праця, через привласнення якої це додаткове багатство
може бути знову перетворене в капітал, залежить не від вартості,
а від маси цих засобів виробництва (включаючи й засоби
існування), бо в процесі праці робітник має справу не з вартістю,
а з споживною вартістю засобів виробництва. Однак, само
нагромадження і дана разом з ним концентрація капіталу є
матеріальний засіб підвищення продуктивної сили. Але це зростання
засобів виробництва передбачає зростання робітничого
населення, створення населення робітників, яке відповідає додатковому
капіталові і загалом і в цілому навіть завжди перевищує
його потреби, отже, створення перенаселення робітників.
Тимчасовий надлишок додаткового капіталу порівняно з робітничим
населенням, яке є в його розпорядженні, справляв би
двоякий вплив. З одного боку, він ступнево збільшував би робітниче
\index{iii1}{0219}  %% посилання на сторінку оригінального видання
населення шляхом підвищення заробітної плати, отже,
пом’якшенням згубних впливів, що скорочують приріст робітників,
і полегшенням шлюбів; а з другого боку, шляхом застосування
методів, які створюють відносну додаткову вартість (введення
й поліпшення машин), він ще далеко швидше створив би
штучне відносне перенаселення, яке з свого боку — бо в капіталістичному
виробництві злидні породжують населення, — знов таки є
теплицею дійсного швидкого збільшення чисельності населення.
Тому з природи капіталістичного процесу нагромадження —
який є тільки моментом капіталістичного процесу виробництва —
само собою випливає, що збільшена маса засобів виробництва,
призначених для перетворення в капітал, завжди знаходить під
рукою відповідно збільшене і навіть надлишкове робітниче населення,
яке можна експлуатувати. Отже, з розвитком процесу
виробництва і нагромадження \emph{мусить} зростати маса придатної
до привласнення і привласнюваної додаткової праці, а тому й абсолютна
маса зиску, привласнюваного суспільним капіталом. Але ті
самі закони виробництва і нагромадження разом з масою сталого
капіталу підвищують у дедалі більшій прогресії і його вартість, —
швидше, ніж вони підвищують вартість змінної частини капіталу,
обмінюваної на живу працю. Отже, одні й ті самі закони зумовлюють
для суспільного капіталу зростаючу абсолютну масу
зиску і падаючу норму зиску.

Ми тут цілком залишаємо осторонь те, що та сама величина
вартості з прогресом капіталістичного виробництва і відповідного
йому розвитку продуктивної сили суспільної праці та при помноженні
галузей виробництва, отже й продуктів, представляє прогресивно
зростаючу масу споживних вартостей і насолод.

Хід розвитку капіталістичного виробництва і нагромадження
зумовлює процеси праці в дедалі більшому масштабі, отже, в дедалі
більших розмірах, і відповідно до цього зумовлює зростаюче
авансування капіталу на кожне окреме підприємство. Тому
зростаюча концентрація капіталів (супроводжена в той самий
час, хоч і в меншій мірі, зростанням числа капіталістів) є так
само однією з матеріальних умов капіталістичного виробництва
і нагромадження, як і одним із створюваних ним самим результатів.
Рука в руку і у взаємодії із цим відбувається прогресуюча експропріація
більш чи менш безпосередніх виробників. Таким чином
для одиничних капіталістів стає зрозумілим, що вони мають
у своєму розпорядженні дедалі зростаючі робітничі армії (як би
сильно не падав їх змінний капітал порівняно з сталим), що маса
привласнюваної ними додаткової вартості, а тому й зиску, зростає
одночасно з падінням норми зиску і не зважаючи на це падіння.
Якраз ті самі причини, які концентрують маси робітничих армій
під командою окремих капіталістів, збільшують також масу застосовуваного
основного капіталу, як і сировинних та допоміжних
матеріалів,— збільшують відносно швидше, ніж масу вживаної
живої праці.

\parcont{}  %% абзац починається на попередній сторінці
\index{ii}{0220}  %% посилання на сторінку оригінального видання
\frac{\num{25.000}}{5} = 5000\pound{ ф. стерл}. Коли поділити ці 5000\pound{ ф. стерл.} на 500, то матимемо число оборотів 10,
цілком таке саме, як і для цілого капіталу в 2500\pound{ ф. стерл}.

Це пересічне обчислення, що за ним вартість річного продукту ділиться на вартість авансованого
капіталу, а не на вартість частини цього капіталу, постійно застосовуваної в одному робочому періоді
(отже, в нашому прикладі, не на 400, а на 500, не на капітал І, а на капітал І + капітал II), — це
пересічне обчислення тут, де йдеться лише про продукцію додаткової вартости, є абсолютно точне. Далі
ми побачимо, що, з іншого погляду, воно не зовсім точне, як і взагалі це пересічне обчислення не
зовсім точне. Інакше кажучи, воно задовільне для практичних цілей капіталіста,
але воно не виражає точно й гаразд усіх реальних обставин обороту.

Досі ми одну частину вартости товарового капіталу лишали цілком осторонь, а саме вміщену в ньому
додаткову вартість, спродуковану та долучену до продукту протягом процесу продукції. На неї тепер і
треба нам звернути увагу.

Коли припустити, що витрачуваний щотижня змінний капітал в 100\pound{ ф. стерл.}, продукує додаткову
вартість в 100\% = 100\pound{ ф. стерл.}, то змінний капітал в 500\pound{ ф. стерл.},  витрачуваний протягом
п’ятитижневого періоду обороту, випродукує додаткову вартість в 500\pound{ ф. стерл.}, тобто половина
робочого дня складається з додаткової праці.

Але коли 500\pound{ ф. стерл.} змінного капіталу продукують 500\pound{ ф. стерл.} додаткової вартости, то 5000\pound{ ф.
стерл.} випродукують її 500 × 10 = 5000\pound{ ф. стерл}. Але авансований змінний капітал = 500\pound{ ф. стерл}.
Відношення всієї маси додаткової вартости, спродукованої протягом року, до суми вартости
авансованого змінного капіталу ми звемо річною нормою додаткової вартости. Отже, в даному випадку,
вона = \frac{5000}{500} = 1000\%.
Коли ближче аналізувати цю норму, то виявиться, що вона дорівнює тій нормі додаткової вартости, яку
авансований змінний капітал продукує протягом одного періоду обороту, помноженій на число оборотів
змінного капіталу (а воно збігається з числом оборотів цілого обігового капіталу).

Авансований протягом одного періоду обороту змінний капітал в даному випадку = 500\pound{ ф. стерл.};
створена ним додаткова вартість теж = 500\pound{ ф. стерл}. Тому норма додаткової вартости протягом одного
періоду обороту = \frac{500m}{500v} = 100\%. Ці 100\%, помножені на 10, на число оборотів протягом року,
дають \frac{5000m}{5000v} = 1000\%.

Це має силу щодо річної норми додаткової вартости. Щождо маси додаткової вартости, здобуваної
протягом певного періоду обороту, то ця маса дорівнює вартості авансованого протягом цього періоду
змінного капіталу — в даному випадку = 500\pound{ ф. стерл.}, помноженій на норму
\parbreak{}  %% абзац продовжується на наступній сторінці

\parcont{}  %% абзац починається на попередній сторінці
\index{ii}{0221}  %% посилання на сторінку оригінального видання
додаткової вартости, в даному випадку, отже, 500 × \frac{100}{100} \deq{} 500 × 1 \deq{} 500\pound{ ф. стерл}. Коли б
авансований капітал був \deq{} 1500\pound{ ф. стерл.} при незмінній
нормі додаткової вартости, то маса додаткової вартости була б \deq{}
1500 × \frac{100}{100} \deq{} 1500\pound{ ф. стерл}.

Змінний капітал у 500\pound{ ф. стерл.}, що обертається 10 разів на рік, і
що продукує протягом року додаткову вартість в 5000\pound{ ф. стерл.}, отже,
капітал, що для нього річна норма додаткової вартости \deq{} 1000\%, ми
будемо називати капіталом $А$.

Припустімо тепер, що інший змінний капітал $В$ в 5000\pound{ ф. стерл.}
авансується на цілий рік (тобто, тут на 50 тижнів) і тому обертається
лише один раз на рік. Припустімо при цьому далі, що наприкінці року
продукт оплачується в той самий день, як його виготовлено, і, значить,
грошовий капітал, що на нього його перетворюється, повертається в той
самий день. Отже, період циркуляції тут \deq{} 0, період обороту дорівнює
робочому періодові, а саме, одному рокові. Як і в попередньому випадку,
в процесі праці щотижня перебуває змінний капітал в 100\pound{ ф. стерл.},
а тому протягом 50 тижнів — в 5000\pound{ ф. стерл}. Далі, норма додаткової
вартости хай буде та сама \deq{} 100\%, тобто за однакової довжини робочого
дня половина його складається з додаткової праці. Коли ми візьмемо
5 тижнів, то вкладений змінний капітал \deq{} 500\pound{ ф. стерл.}, норма додаткової
вартости \deq{} 100\%, отже, маса додаткової вартости, створена протягом
5 тижнів \deq{} 500\pound{ ф. стерл}. Кількість робочої сили, що її тут експлуатується,
і ступінь її експлуатації, згідно з нашим припущенням, тут
точно такі самі, як і при капіталі $А$.

Вкладений змінний капітал в 100\pound{ ф. стерл.} щотижня створює додаткову
вартість в 100\pound{ ф. стерл.}, тому протягом 50 тижнів вкладений капітал
в 100 × 50 \deq{} 5000\pound{ ф. стерл.} створить додаткову вартість в 5000\pound{ ф. стерл}. Маса щороку створюваної
додаткової вартости буде така сама, як і в попередньому випадку \deq{} 5000\pound{ ф. стерл.}, але річна норма
додаткової
вартости цілком інша. Вона дорівнює спродукованій протягом року
додатковій вартості, поділеній на авансований змінний капітал:
\frac{5000m}{5000v} \deq{} 100\%, тимчасом як раніш для капіталу $А$ вона дорівнювала 1000\%.

При капіталі $А$, як і при капіталі $В$, ми витрачали щотижня 100\pound{ ф. стерл.} змінного капіталу; ступінь
зростання вартости або норма додаткової
вартости цілком та сама, вона дорівнює 100\%; величина змінного
капіталу теж та сама \deq{} 100\pound{ ф. стерл}. Експлуатується цілком таку
саму кількість робочої сили, величина й ступінь експлуатації в обох випадках
однакові, робочі дні однакові і однаково поділяються на доконечну
й додаткову працю. Сума змінного капіталу, застосованого протягом
року, однакова величиною \deq{} 5000\pound{ ф. стерл.}, вона пускає в рух таку
саму масу праці й витягує з робочої сили, пущеної в рух обома рівними
капіталами, однакову масу додаткової вартости, 5000\pound{ ф. стерл}. І,
\parbreak{}  %% абзац продовжується на наступній сторінці

\input{_0222.tex}
\input{_0223_0224.tex}
\parcont{}  %% абзац починається на попередній сторінці
\index{iii1}{0225}  %% посилання на сторінку оригінального видання
зиску, ніж дрібний капіталіст, який, видимо, одержує високий зиск. Далі, найповерховіше
спостереження конкуренції показує, що при певних обставинах, коли більший капіталіст хоче захопити
для себе місце на ринку, витиснути дрібніших капіталістів, — як, наприклад, за часів кризи, — він
використовує це практично, тобто навмисно знижує свою норму зиску, щоб витиснути з ринку дрібніших
капіталістів. Так само й купецький капітал — про який ми пізніше скажемо докладніше — показує явища,
завдяки яким зниження зиску здається наслідком розширення підприємства, а разом з тим і капіталу.
Власне науковий вираз замість помилкового розуміння ми дамо пізніше. Подібні поверхові погляди є
результатом порівнення норм зиску, одержуваних в окремих галузях підприємств залежно від того, чи
підпорядковані вони режимові вільної конкуренції чи монополії. Цілком банальне уявлення, яке
створюється в головах агентів конкуренції, ми знаходимо в нашого Рошера, а саме, що таке зниження
норми зиску є „розумніше й гуманніше“.\footnote*{
„Die Grundlagen der Nationalökonomie“. 2 Aufl. Stuttgart und Augsburg 1857,
стор. 190. Примітка ред. нім. вид. ІМЕЛ.
} Зменшення норми зиску представлено тут як \emph{наслідок}
збільшення капіталу і зв’язаного з цим розрахунку капіталістів, що при меншій нормі зиску маса
зиску, яку вони кладуть собі в кишеню, буде більша. Все це (за винятком того, що є в А. Сміта, про
що пізніше) основане на цілковитому нерозумінні того, що таке взагалі є загальна норма зиску, і на
тому грубому уявленні, що ціни дійсно визначаються шляхом надбавки більш-менш довільної частки зиску
до дійсної вартості товарів. Хоч які грубі ці уявлення, все ж вони з необхідністю виникають з того
перекрученого способу й вигляду, в якому імманентні закони капіталістичного виробництва виявляються
в сфері конкуренції.

\pfbreak{}

Закон, згідно з яким падіння норми зиску, викликуване розвитком продуктивної сили, супроводиться
збільшенням маси зиску, виражається і в тому, що падіння цін товарів, вироблюваних капіталом,
супроводиться відносним збільшенням мас зиску, які містяться в них і реалізуються через їх продаж.

Через те що розвиток продуктивної сили і відповідний цьому вищий склад капіталу приводить в рух
дедалі більшу кількість засобів виробництва за допомогою дедалі меншої кількості праці, то кожна
пропорціональна частина всього продукту, кожна одиниця товару або кожна певна окрема кількість
товару, яка служить одиницею міри для сукупної маси вироблених товарів, вбирає менше живої праці і
містить у собі, крім того, менше упредметненої праці як щодо зношення застосованого основного
капіталу,
так і щодо спожитих сировинних і допоміжних матеріалів. Отже, кожна одиниця товару містить у собі
меншу суму праці як упредметненої
\index{iii1}{0226}  %% посилання на сторінку оригінального видання
в засобах виробництва, так і новододаної під час виробництва. Тому ціна одиниці товару
падає. Маса зиску, яка міститься в кожній одиниці товару, може, не зважаючи на це, збільшитись, якщо
норма абсолютної чи відносної додаткової вартості зростає. Кожний окремий товар містить у собі менше
новододаної праці, але неоплачена частина її зростає в порівнянні з оплаченою. Однак, це
відбувається тільки в певних межах. Разом з дуже значним абсолютним зменшенням новододаної до кожної
одиниці товару суми живої праці, яке відбувається в ході розвитку виробництва, зменшуватиметься
абсолютно і маса неоплаченої праці, яка міститься в ній, як би вона не зростала відносно, а саме в
порівнянні з оплаченою частиною. Маса зиску, яка припадає на кожну одиницю товару, дуже
зменшуватиметься з розвитком продуктивної сили праці, не зважаючи на зростання норми додаткової
вартості; і це зменшення цілком так само, як падіння норми зиску, тільки уповільнюється здешевленням
елементів сталого капіталу та іншими наведеними в першому відділі цієї книги обставинами, які
підвищують норму зиску при незмінній і навіть при падаючій нормі додаткової вартості.

Те, що ціна окремих товарів, з суми яких складається сукупний продукт капіталу, падає, не означає
нічого іншого, як те, що дана кількість праці реалізується в більшій масі товарів, що, отже, кожна
одиниця товару містить у собі менше праці, ніж раніше. Це відбувається навіть у тому випадку, коли
ціна якоїсь частини сталого капіталу, сировинного матеріалу та ін. зростає. За винятком окремих
випадків (наприклад, коли продуктивна сила праці рівномірно здешевлює всі елементи як сталого, так і
змінного капіталу), норма зиску знижуватиметься, не зважаючи на підвищену норму додаткової вартості,
1) тому що навіть більша неоплачена частина зменшеної загальної суми новододаної праці є менша, ніж
була менша відповідна неоплачена частина більшої загальної суми, і 2) тому що вищий склад капіталу в
окремому товарі виражається в тому, що та частина його вартості, яка взагалі представляє новододану
працю, зменшується порівняно з тією частиною вартості, яка представляє сировинний матеріал,
допоміжний матеріал і зношування основного капіталу. Ця переміна у відношенні різних складових
частин ціни окремого товару, зменшення тієї частини ціни, яка представляє новододану живу працю, і
збільшення тих частин ціни, які представляють раніше упредметнену працю, є та форма, в якій у ціні
окремого товару виражається зменшення змінного капіталу порівняно з сталим. Наскільки таке зменшення
є абсолютним для капіталу даної величини, наприклад, для 100, настільки ж воно є абсолютним для
кожного окремого товару як відповідної частини репродукованого капіталу. Однак, норма зиску, якщо
тільки обчисляти її на елементи ціни окремих товарів, виступила б іншою, ніж вона є в дійсності. І
саме з такої причини:

\index{iii1}{0227}  %% посилання на сторінку оригінального видання
[Норма зиску обчислюється на весь застосований капітал, але за певний час, фактично за один рік.
Відношення виробленої за рік і реалізованої додаткової вартості або зиску до всього капіталу,
обчислене в процентах, є норма зиску. Отже, вона не неодмінно дорівнює тій нормі зиску, при якій в
основу обчислення кладеться не рік, а період обороту капіталу, про який іде мова; тільки в тому
випадку, коли цей капітал обертається саме один раз за рік, обидві ці норми збігаються.

З другого боку, зиск, одержаний на протязі року, є тільки сума зисків на товари, вироблені і продані
на протязі того самого року. Якщо ж ми обчислюватимем зиск на витрати виробництва товарів, то
одержимо норму зиску $= \frac{p}{k}$, де $р$ становить реалізований на протязі року зиск, а $k$ — суму витрат
виробництва товарів, вироблених і проданих протягом того самого часу. Очевидно, що ця норма зиску
$\frac{p}{k}$ тільки в тому випадку може збігатися з дійсною нормою зиску $\frac{p}{K}$, — маса зиску, поділена на весь
капітал, — коли $k = К$, тобто коли капітал обертається, саме один раз за рік.

Візьмімо три різні стани якогонебудь промислового капіталу.

І. Капітал в 8000 фунтів стерлінгів виробляє і продає щороку 5000 штук товару по 30 шилінгів за
штуку, отже, має річний оборот в 7500 фунтів стерлінгів. На кожну штуку товару
він дає зиск в 10 шилінгів = 2500 фунтам стерлінгів на рік. Отже, в кожній штуці містяться 20
шилінгів авансованого капіталу і 10 шилінгів зиску, отже норма зиску на кожну штуку становить
$\frac{10}{20}= 50\%$. На суму в 7500 фунтів стерлінгів, що обернулась, припадає 5000 фунтів стерлінгів
авансованого капіталу і 2500 фунтів стерлінгів зиску; норма зиску на кожний оборот, $\frac{p}{k}$, так само =
50\%. Навпаки, норма зиску, обчислена на весь капітал, $\frac{p}{K} = \frac{2500}{8000} = 31\sfrac{1}{4}\%$.

II. Припустім, що капітал збільшується до 10000 фунтів стерлінгів. Припустім, що в наслідок
збільшеної продуктивної сили праці він може виробляти щороку 10000 штук товару при витратах
виробництва в 20 шилінгів на штуку. Він продає їх із зиском в 4 шилінги на штуку, отже, по 24
шилінги за штуку. Тоді ціна річного продукту = 12000 фунтам стерлінгів, з яких 10000 фунтів
стерлінгів авансованого капіталу і 2000 фунтів стерлінгів зиску $\frac{p}{k}$ на кожну штуку $= \frac{4}{20}$, для річного
обороту $= \frac{2000}{10000}$, отже, в обох випадках = 20\%, а через те що весь
\parbreak{}  %% абзац продовжується на наступній сторінці

\parcont{}  %% абзац починається на попередній сторінці
\index{ii}{0228}  %% посилання на сторінку оригінального видання
вартости виражає не що інше, як відношення застосованого протягом
певного часу змінного капіталу до спродукованої протягом того самого
часу додаткової вартости; або — масу тієї неоплаченої праці, що її пускає
в рух змінний капітал, застосований протягом цього часу. Вона абсолютно
не має чинення до тієї частини змінного капіталу, яку авансовано,
але протягом певного часу не застосовується, отже, так само не
має ніякою чинення вона й до відношення між частиною капіталу, авансованого
в певний протяг часу, і тією частиною його, що її застосовано
протягом цього самого часу — відношення, що для різних капіталів
під впливом періодів обороту модифікується й є різне.

З наведеного вище скорше випливає, що річна норма додаткової вартости
лише в одному єдиному випадку збігається з справжньою нормою
додаткової вартости, яка виражає ступінь експлуатації праці: а саме в
тому разі, коли авансований капітал обертається тільки один раз на рік,
коли тому авансований капітал дорівнює капіталові, що обернувся протягом
року, а відношення маси додаткової вартости, спродукованої протягом
року, до капіталу, застосованого на її продукцію протягом року, збігається
і є тотожне з відношенням маси додаткової вартости, спродукованої
протягом року, до капіталу, авансованого протягом року.

A) Річна норма додаткової вартости дорівнює:\[
\frac{\text{маса додаткової вартости, спродукованої протягом року}}{\text{авансований змінний капітал}}
\]

\noindent{}Але маса додаткової вартости, спродукованої протягом року, дорівнює
справжній нормі додаткової вартости, помноженій на змінний капітал,
застосований на її продукцію. Капітал, застосований на продукцію
річної маси додаткової вартости, дорівнює авансованому капіталові, помноженому
на число оборотів його, яке ми позначатимемо $n$. Тому формула А) перетворюється на таку:

B) Річна норма додаткової вартости дорівнює:\[
\frac{\text{справжня норма додаткової вартости} × \text{аванс. змінний капітал} × n}{\text{авансований змінний капітал}}
\]

\noindent{}Наприклад, для капіталу $В \deq{} \frac{100\%×5000×1}{5000}$ або 100\%. Тільки коли
$n \deq{} 1$, тобто, коли авансований змінний капітал обертається тільки
один раз на рік, отже, дорівнює застосованому протягом року капіталові,
або капіталові, що обернувся протягом року, — тільки тоді річна норма додаткової
вартости, дорівнює справжній нормі додаткової вартости.

Коли ми позначимо річну норму додаткової вартости $М'$, справжню норму
додаткової вартости $m'$, авансований змінний капітал — $v$, число оборотів
— $n$, то $М' \deq{} \frac{m'vn}{v}= m'n$; отже, $М' \deq{} m'n$, і лише тоді \deq{} $m'$, коли
$n \deq{} 1$; отже, $М' \deq{} m' × 1 \deq{} m$.

\parcont{}  %% абзац починається на попередній сторінці
\index{i}{0229}  %% посилання на сторінку оригінального видання
сцени протягом усього часу тривання драми, так і робітники належали
тепер до фабрики протягом 15 годин, не рахуючи часу на
дорогу до фабрики й назад. Таким чином години відпочинку перетворювалися
на години примусового безділля, що гнали молодого
робітника до шинку, а молоду робітницю в дім розпусти. За
кожної нової витівки, що її день-у-день вигадував капіталіст,
щоб тримати свої машини в русі 12 або 15 годин, не збільшуючи
робочого персоналу, робітник мусів проковтнути свою їжу то в
той, то в інший шматок часу. Під час агітації за десятигодинний
робочий день фабриканти кричали, що робітнича наволоч подає
петиції, сподіваючись дістати за десятигодинну працю дванадцятигодинну
заробітну плату. Тепер вони обернули медалю. Вони
виплачували десятигодинну заробітну плату за дванадцяти й
п’ятнадцятигодинне порядкування робочими силами!\footnote{
Див. «Reports etc. for 30 th April 1849», p. 6 і докладне пояснення
«shifting system»\footnote*{
системи пересувань. \emph{Ред.}
}, яке фабричні інспектори Хоуелл і Савндер дають
у «Reports etc. for 31 st October 1848». Див. також петицію проти
«shift system», подану королеві духівництвом Ashton’a й околиць на весні
1849~\abbr{р.}
} Так ось
у чім була річ; це було фабрикантське видання десятигодинного
закону! Це були ті самі фритредери, сповнені благодаті й любови
до людства, що підчас аґітації проти хлібних законів цілих десять
років до останнього шага обчислювали робітникам, що за вільного
довозу хліба, при тих засобах, що їх має англійська промисловість,
цілком досить було б десяти годин праці, щоб збагатити
капіталістів\footnote{
Порівн., наприклад, «The Factory Question and the Ten Hours
Bill. By R.~H.~Greg. 1837».
}.

Дворічний бунт капіталу увінчався нарешті присудом однієї
з чотирьох вищих судових установ Англії, Court of Exchequer,
який в одному з випадків, що дійшов до нього, 8 лютого 1850~\abbr{р.}
вирішив, що хоч фабриканти й чинили проти змісту закону
1844~\abbr{р.}, але самий цей закон містить у собі деякі слова, що роблять
його безглуздим. «Цей вирок знищив закон про десятигодинну
працю»\footnote{
\emph{F.~Engels}: «Die englische Zehnstundenbill» (у видаваній мною
«Neue Rheinische Zeitung». Politish-ökonomische Revue, Aprilheft
1850», p. 13). Той самий «високий» суд так само винайшов підчас американської
громадянської війни словесну зачіпку, яка перетворювала закон
проти озброєння піратських кораблів у його пряму протилежність.
}. Маса фабрикантів, що досі боялись застосовувати
систему змін для підлітків і робітниць, ухопилися за неї тепер
обома руками\footnote{
«Reports etc. for 30 th April 1850».
}.

Але за цією, здавалось, остаточною перемогою капіталу
настав зараз же поворот. Робітники досі ставили пасивний, хоч
і впертий і день-у-день відновлюваний опір. Тепер вони почали
голосно протестувати на загрозливих мітинґах у Ланкашірі і
Йоркшірі. Значить, так званий десятигодинний закон — це лише
ошуканство, парляментське шахрайство, а на ділі він ніколи не
існував! Фабричні інспектори пильно попереджали уряд, що
\parbreak{}  %% абзац продовжується на наступній сторінці

\input{_0230.tex}
\input{_0231.tex}

\index{i}{0232}  %% посилання на сторінку оригінального видання
Закон з 1850~\abbr{р.} перетворив лише для «підлітків і жінок»
п’ятнадцятигодинний період від пів на шосту ранку до пів на
дев’яту вечора на дванадцятигодинний період від шостої години
ранку й до шостої години вечора. Отже, не для дітей, яких усе
ще можна було експлуатувати \sfrac{1}{2} години перед початком і 2\sfrac{1}{2}
години по скінченні цього періоду, хоч загальний час тривання
їхньої праці не повинен був перевищувати 6\sfrac{1}{2} годин. Під час обговорення
закону фабричні інспектори подали до парляменту статистичні
дані про ганебні зловживання, до яких приводила ця
аномалія. Але все це даремно. Потайний намір був у тому, щоб
за допомогою дітей у роки розцвіту знов догнати робочий день
дорослих до 15 годин. Досвід дальших трьох років довів, що
така спроба мусіла б розбитись об опір дорослих робітників-чоловіків\footnote{
«Reports etc. for 30 th April 1853», p. 31.
}.
Тим то закон з 1850~\abbr{р.} й доповнено, нарешті, 1853~\abbr{р.}
забороною «вживати праці дітей ранком перед початком і ввечері
по скінченні праці підлітків і жінок». Починаючи з цього часу,
фабричний закон 1850~\abbr{р.} реґулював, за деякими винятками, в
підпорядкованих йому галузях промисловости робочий день усіх
робітників\footnote{
За часів найвищого розквіту англійської бавовняної промисловости,
в роках 1859--1860, деякі фабриканти приманою високої заробітної
плати за наднормовий час пробували підохотити дорослих прядунів до
здовження робочого дня. Прядуни на ручних варстатах і на сельфакторах
поклали кінець цій спробі, подавши меморіял своїм підприємцям, де, між
іншим, зазначено: «Сказати по правді, наше життя є тягар для нас, і
поки ми прикуті до фабрики майже на 2 дні (20 годин) у тижні
більше, ніж інші робітники, то ми почуваємо себе в країні гелотами й
сами собі докоряємо за те, що увіковічнюємо таку систему, яка фізично
й морально шкодить нам самим і нашим нащадкам\dots{} Тим то з повною
пошаною доводимо до вашого відома, що від першого дня нового року
не працюватимемо й хвилини довше понад 60 годин тижнево, від шостої
години до шостої години, відлічуючи законом призначені перерви на
1\sfrac{1}{2} години». («Reports etc. for 30 th April 1860», p. 30).
}. Від часу оголошення першого фабричного закону
проминуло тепер півстоліття\footnote{
Про засоби порушувати цей закон, що їх дає редакція цього закону,
див. Parliamentary Return: «Factory Regulations Acts» (6 серпня 1859~\abbr{р.}) і
там само \emph{Leonhard Horner} : «Suggestions for Amending the Factory Acts to
enable the Inspectors to prevent illegal working, now become very prevalent».
}.

Поза свою первісну сферу законодавство вийшло вперше через
«Printworks’ Act» (закон про перкалеві фабрики тощо), виданий
1845~\abbr{р.} Нехіть, з якою капітал допустив цю нову «екстраваґантність»,
промовляє з кожного рядка закону! Він обмежує
робочий день дітей 8--13 років і жінок 16 годинами — від шостої
години ранку до десятої години вечора, не призначаючи жодної
взаконеної перерви на їжу. Він дозволяє примушувати до праці
робітників-чоловіків старших від 13 років довільно цілий
день і цілу ніч\footnote{
«За останнє півріччя (1857) у моїй окрузі дітей 8 років і старших
справді катують від 6 години ранку й аж до 9 години вечора». («Reports
etc. for 31 st October 1857», p. 39).
}. Це — парляментський викидень\footnote{
«Закон про перкалеві фабрики вважається за невдалий так шодо його
постанов про навчання, як і щодо його постанов про охорону праці» («The Printworks Act is admitted to be a failure, both with reference to its educational
and protective provisions»). («Reports etc. for 31 st Oct. 1862», p. 52).
}.

\index{i}{0233}  %% посилання на сторінку оригінального видання
А все ж принцип\footnote*{
Мається на увазі принцип законодавчого втручання в промислові
справи. \emph{Ред.}
}, перемігши у великих галузях промисловости,
які є найспецифічніший витвір сучасного способу продукції,
переміг остаточно. Дивовижний розвиток цих галузей промисловости
на протязі часу від 1853~\abbr{р.} до 1860~\abbr{р.}, який відбувався
поруч фізичного й морального відродження фабричних робітників,
розкрив очі найдурнішим. Сами фабриканти, що в них у
півстолітній громадянській війні крок за кроком відвойовано
законодавчі обмеження й реґулювання робочого дня, хвальковито
вказували на контраст між цими галузями промисловости
й тими сферами експлуатації, що лишилися ще «вільними»\footnote{
Так, приміром, висловлюється E.~Поттер у листі до «Times’y»
з 24 березня 1863~\abbr{р.} «Times» нагадав йому про бунт фабрикантів проти
десятигодинного закону.
}.
Фарисеї «політичної економії» проголосили тепер переконання
про доконечність законодавчим шляхом реґулювати робочий день
новим характеристичним здобутком їхньої «науки»\footnote{
Так, між іншим, висловлюється пан В.~Ньюмарч, співробітник
і видавець «History of Prices» Тука. Невже ж це науковий проґрес —
робити боягузливі поступки громадській думці?
}. Легко
зрозуміти, що після того, як фабричні маґнати скорилися перед
неминучим і примирилися з ним, сила опору капіталу помалу
слабшала, тоді як у той самий час сила наступу робітничої кляси
зростала разом із зростом числа її спільників серед суспільних
верств, безпосередньо не заінтересованих. Цим то й пояснюється
порівняно швидкий проґрес від 1860~\abbr{р.}

Фарбарні й білильні\footnote{
Виданий 1860~\abbr{р.} закон про білильні та фарбарні установляє,
що робочий день від 1 серпня 1861~\abbr{р.} тимчасово скорочується до 12, а
від 1 серпня 1862~\abbr{р.} остаточно до 10 годин, тобто до 10\sfrac{1}{2} годин у робочі
дні та 7\sfrac{1}{2} годин суботами. Але ось настав лихий 1862~\abbr{р.}, і повторився
старий фарс. Пани фабриканти звернулись до парляменту з петицією
стерпіти ще один-однісінький рік дванадцятигодинну працю підлітків
і жінок\dots{} «За сучасного стану справ (підчас бавовняного голоду) було б
дуже корисно для робітників, коли б їм дозволили працювати по 12 годин
щодня й діставати по змозі якнайбільшу заробітну плату\dots{} Вже було
пощастило внести до парляменту біл у цьому дусі, але він провалився
через аґітацію робітників у білильнях Шотляндії». («Reports etc. for
31 st October 1862», p. 14, 15). Капітал, побитий таким чином тими самими
робітниками, що іменем їхнім він претендував говорити, відкрив тепер
за допомогою юридичних окулярів, що закон з 1860~\abbr{р.}, подібно до всіх
парляментських законів про «охорону праці», складений поплутаними,
покрученими словами, дає привід не поширювати його на категорії робітників
«calenderers»\footnote*{пресувальники сукна. \emph{Ред.}
} і «finishers»\footnote*{
апретери. \emph{Ред.}
}. Англійська юрисдикція, завжди
вірний наймит капіталу, санкціонувала це закарлюцтво постановою так
званого «Common Pleas»\footnote*{цивільний суд. \emph{Ред.}
}. «Це викликало велике незадоволення серед
робітників, і дуже шкода, що ясні наміри законодавства нищиться під
приводом хибного окреслення слів». (Там же, стор. 18).
} підведено під фабричний закон 1850~\abbr{р.}
ще 1860~\abbr{р.}, а мереживні й панчішні — 1861~\abbr{р.} В наслідок першого
\parbreak{}  %% абзац продовжується на наступній сторінці

\input{_0234.tex}

\index{iii1}{0235}  %% посилання на сторінку оригінального видання
З вищесказаним зв’язане знецінення наявного капіталу (тобто його речових елементів), яке
відбувається з розвитком промисловості. Воно також є однією з постійно діючих причин, які затримують
падіння норми зиску, хоч воно при певних обставинах може зменшувати масу зиску через зменшення маси
капіталу, який дає зиск. Тут знову виявляється, що ті самі причини, які породжують тенденцію норми
зиску до падіння, уміряють також здійснення цієї тенденції.

\subsection{Відносне перенаселення}

Утворення відносного перенаселення невідривне від розвитку продуктивної сили праці і прискорюється
цим розвитком, який виражається в зменшенні норми зиску. Відносне перенаселення виявляється в певній
країні тим яскравіше, чим більше розвинений в ній капіталістичний спосіб виробництва. В свою чергу
воно є, з одного боку, причиною того, що в багатьох галузях виробництва продовжує існувати більш чи
менш неповне підпорядкування праці капіталові, і продовжує існувати довше, ніж це на перший погляд
відповідає загальному станові розвитку; це — наслідок дешевини і великої кількості наявних в
розпорядженні капіталістів або звільнених найманих робітників, а також більшого опору, що його деякі
галузі виробництва, відповідно до
їх природи, чинять перетворенню ручної праці в машинну. З другого боку, відкриваються нові галузі
виробництва, особливо предметів розкоші, галузі, які мають своєю базою саме те відносне населення,
яке часто звільняється в наслідок переважання сталого капіталу в інших галузях виробництва, і які з
свого боку знов таки базуються на переважанні елементів живої праці і тільки ступнево пророблюють
той самий шлях розвитку, що й інші галузі виробництва. В обох випадках змінний капітал становить
значну частину всього капіталу, а заробітна плата стоїть нижче пересічної, так що в цих галузях
виробництва як норма додаткової вартості, так і маса додаткової вартості незвичайно високі. А через
те що загальна норма зиску утворюється в наслідок вирівнення норм зиску окремих галузей виробництва,
то
тут знову таки та сама причина, яка породжує тенденцію норми зиску до падіння, викликає протидію цій
тенденції, яка більш або менш паралізує вплив цієї тенденції.

\subsection{Зовнішня торгівля}

Оскільки зовнішня торгівля здешевлює почасти елементи сталого капіталу, почасти необхідні засоби
існування, в які перетворюється змінний капітал, остільки вона діє на норму зиску в напрямі
підвищення, підвищуючи норму додаткової вартості і знижуючи вартість сталого капіталу. Вона взагалі
діє в цьому напрямі, даючи змогу розширювати розміри виробництва. Таким
\parbreak{}  %% абзац продовжується на наступній сторінці

\input{_0236_0237.tex}
\parcont{}  %% абзац починається на попередній сторінці
\index{ii}{0238}  %% посилання на сторінку оригінального видання
можуть вийти лише з певних галузей, як, напр., сільське господарство тощо,
де працюють виключно дужі парубки. Це діється й після того, як нові підприємства
стали вже постійною галуззю продукції і, значить, після того,
як уже утворилась потрібна для них бродяча робітнича кляса. Напр.,
коли залізниця раптом почне будуватись у ширшому від пересічного
маштабі. Тоді вбирається частину резервної армії робітників, що її тиск
тримав заробітну плату на порівняно низькому рівні. Тоді заробітна плата
скрізь підвищується, навіть у тих частинах робочого ринку, де робітники
й раніш легко знаходили собі працю. Це триває доти, доки неминучий
крах знову звільняє резервну армію робітників, і заробітну плату
знову знижується до її мінімуму й нижче.\footnote{
В рукопису тут вставлено таку замітку, щоб пізніш її розвинути: „Суперечність
в капіталістичному способі продукції: робітники як покупці товару,
важать для ринку. Але як продавців свого товару — робочої сили капіталістичне
суспільство має тенденцію обмежувати їх мінімумом ціни. Дальша суперечність:
ті епохи, коли капіталістична продукція напружує всі свої сили, регулярно з’являються
як епохи перепродукції, бо продуктивні сили ніколи не можна застосувати
так, щоб у наслідок цього можна було не лише випродукувати, а й зреалізувати
більше вартости; але продаж товарів, реалізація товарового капіталу, отже,
і додаткової вартости, обмежена не просто споживними потребами суспільства
взагалі, з споживними потребами такого суспільства, що його переважна
більшість завжди бідна й мусить завжди лишатися бідною. Однак це стосується
лише до наступного відділу.“ \emph{Ф.~Е.}
}

Оскільки більший або менший протяг періоду обороту залежить від
робочого періоду у власному значенні, тобто від періоду, потрібного на
те, щоб виготувати продукт для ринку, він ґрунтується на кожного
разу даних речових умовах продукції різних капіталовкладень, на
умовах, що в хліборобстві мають більше характер природних умов продукції,
а в мануфактурі і в більшій частині видобувної промисловости
змінюються разом із суспільним розвитком самого продукційного процесу.

Оскільки протяг робочого періоду ґрунтується на величині поставок
(на кількісному розмірі, що в ньому продукт звичайно подається на ринок
як товар), він має умовний характер. Але сама ця умовність має за
матеріяльну базу розміри продукції, а тому вона є випадкова лише остільки,
оскільки ми розглядаємо її ізольовано.

Нарешті, оскільки протяг періоду обороту залежить від протягу періоду
циркуляції, він почасти зумовлюється постійною зміною ринкових
коньюнктур, більшою або меншою легкістю продажу і неминучою, відси
посталою, потребою подавати частину продукту на ближчий або дальший
ринок. Лишаючи осторонь розмір попиту взагалі, рух цін відіграє
тут головну ролю, оскільки при зниженні цін продаж навмисно обмежується,
тимчасом як продукція розвивається далі; навпаки буває при
підвищенні цін, коли продукція та продаж не відстають одне від одного,
або коли продаж може відбуватися заздалегідь. Однак, за власне матеріяльну
базу треба вважати справжнє віддалення місця продукції від
ринку збуту.

\index{iii1}{0239}  %% посилання на сторінку оригінального видання
\section{Розвиток внутрішніх суперечностей закону}
\subsection{Загальні зауваження}

В першому відділі цієї книги ми бачили, що норма зиску завжди виражає норму додаткової вартості
нижчою, ніж вона є. Тепер ми побачили, що навіть зростаюча норма додаткової вартості має тенденцію
виражатись у падаючій нормі зиску. Норма
зиску дорівнювала б нормі додаткової вартості тільки в тому випадку, коли $с$ було б $= 0$, тобто коли б
увесь капітал витрачався на заробітну плату. Падаюча норма зиску тільки тоді виражає падаючу норму
додаткової вартості, коли відношення між вартістю сталого капіталу і масою робочої сили, яка
приводить його в рух, лишається незмінним, або коли ця остання збільшується у відношенні до вартості
сталого капіталу.

Рікардо, досліджуючи, як він гадав, норму зиску, в дійсності досліджував тільки норму додаткової
вартості і цю останню тільки при тому припущенні, що робочий день щодо інтенсивності й довжини є
стала величина.

Падіння норми зиску і прискорене нагромадження лиш остільки є різні вирази одного й того ж процесу,
оскільки і те і друге є виразом розвитку продуктивної сили. Нагромадження, з свого боку, прискорює
падіння норми зиску, оскільки разом з ним
дана концентрація робіт у великому масштабі, а тому й вищий склад капіталу. З другого боку, падіння
норми зиску знову таки прискорює концентрацію капіталу і його централізацію шляхом експропріації
дрібних капіталістів, шляхом експропріації
останніх решток безпосередніх виробників, у яких лишається ще щонебудь експропріювати. В наслідок
цього, з другого боку, прискорюється — щодо маси — нагромадження, хоча з падінням норми зиску падає
і норма нагромадження.

З другого боку, оскільки норма зростання вартості всього капіталу, норма зиску, є стимулом
капіталістичного виробництва (подібно до того, як збільшення вартості капіталу є його єдиною метою),
падіння цієї норми уповільнює утворення нових самостійних капіталів і виступає таким чином як
загроза для розвитку капіталістичного процесу виробництва; воно сприяє перепродукції, спекуляції,
кризам, утворенню надмірного капіталу поряд з надмірним населенням. Отже, ті економісти, які,
подібно до Рікардо, вважають капіталістичний спосіб виробництва за абсолютний, відчувають тут, що
цей спосіб виробництва сам собі створює межу, і тому приписують цю межу не виробництву, а природі (в
ученні про ренту). Але важливим в їх жаху перед падаючою нормою зиску є відчуття того, що
капіталістичний спосіб виробництва в розвитку продуктивних сил має таку межу, яка не стоїть ні в
якому зв’язку з виробництвом багатства як таким;
\index{iii1}{0240}  %% посилання на сторінку оригінального видання
і ця особлива межа свідчить про обмеженість і тільки
історичний, минущий характер капіталістичного способу виробництва;
свідчить про те, що він не є абсолютний спосіб виробництва
для виробництва багатства і що, навпаки, на певному
ступені він вступає в конфлікт із своїм дальшим розвитком.

Рікардо і його школа розглядають, в усякому разі, тільки промисловий
зиск, в якому міститься і процент. Але й норма земельної
ренти має тенденцію до падіння, хоч її абсолютна маса зростає
і хоч вона може зростати й відносно, порівняно з промисловим
зиском (див. \emph{Ед. Уест} [„Essay on Application of Capital to Land“,
Лондон 1815], який виклав закон земельної ренти \emph{раніше} від
Рікардо). Якщо ми розглядатимем сукупний суспільний капітал
$К$ і, позначимо через $р_1$ той промисловий зиск, який залишається
після відрахування процента й земельної ренти, через $z$
процент і через $r$ земельну ренту, то
$\frac{m}{K} = \frac{p}{K} = \frac{p_1 + z + r}{K} = \frac{p_1}{K} + \frac{z}{K} + \frac{r}{K}$.
Ми бачили, що хоч у ході розвитку капіталістичного
виробництва сукупна сума додаткової вартості, $m$, постійно
зростає, проте $\frac{m}{K}$ так само постійно зменшується, бо $К$
зростає ще швидше, ніж $m$. Отже, немає ніякої суперечності
в тому, що $p_1$, $z$ і $r$, кожне само по собі, можуть постійно зростати,
тимчасом як $\frac{m}{K} = \frac{p}{K}$, а також $\frac{p_1}{K}$, $\frac{z}{K}$ і $\frac{r}{K}$, кожне само
по собі, постійно зменшуються, або що $р_1$ порівняно з $z$, або
$r$ порівняно з $р_1$, абож порівняно з $р_1$ і $z$ відносно зростає.
При зростаючій сукупній додатковій вартості або зиску $m = p$, але
при одночасно падаючій нормі зиску $\frac{m}{K} = \frac{p}{K}$ відношення величин
частин $p_1$, $z$ і $r$, на які розпадається $m = p$, може як завгодно
змінюватись у межах, даних сукупною сумою $m$, при чому
на величину $m$ або $\frac{m}{K}$ це не впливає.

Взаємна зміна $p_1$, $z$ і $r$ є тільки різний розподіл $m$ між різними
рубриками. Тому і $\frac{p_1}{K}$, $\frac{z}{K}$ або $\frac{r}{K}$, норма індивідуального
промислового зиску, норма процента і відношення ренти до
сукупного капіталу можуть підвищуватись одно порівняно з одним,
хоч $\frac{m}{K}$, загальна норма зиску, падає; умовою при цьому
лишається тільки те, щоб сума всіх трьох $= \frac{m}{K}$. Якщо норма
зиску падає з 50\% до 25\%, якщо, наприклад, склад капіталу,
при нормі додаткової вартості в 100\%, змінюється з $50 с + 50 v$
у $75 c + 25 v$, то в першому випадку капітал в 1000 дасть зиск
\parbreak{}  %% абзац продовжується на наступній сторінці

\parcont{}  %% абзац починається на попередній сторінці
\index{ii}{0241}  %% посилання на сторінку оригінального видання
свого функціонування само підприємство через капіталізацію певної частини
додаткової вартости. Для капіталіста \emph{В} це не можливо. Частина
капіталу, що про неї мовиться, мусить складати в нього частину первісно
авансованого капіталу. В обох випадках ця частина капіталу фігуруватиме
в книгах капіталіста як авансований капітал — і ним вона є в
дійсності — бо, згідно з нашим припущенням, вона становить частину
продуктивного капіталу, доконечного для провадження підприємства
в даному маштабі. Але величезна ріжниця в тому, з якого фонду
її авансується. У \emph{В} вона дійсно є частина первісного авансованого
капіталу або капіталу, що його треба мати в розпорядженні.
Навпаки, в \emph{А} вона є частина додаткової вартости, застосованої як
капітал. Цей останній випадок показує нам як не лише акумульований
капітал, а й частина первісно авансованого капіталу може бути просто
капіталізованою додатковою вартістю.

Скоро сюди долучається розвиток кредиту, відношення первісно авансованого
капіталу й капіталізованої додаткової вартости заплутується
ще більше. Напр., \emph{А} позичає в банкіра \emph{С} частину продуктивного капіталу,
що з ним, він починає або продовжує справу протягом року. З
самого початку він не має власного капіталу, достатнього для провадження
справи. Банкір \emph{С} позичає йому суму, що складається виключно з
додаткової вартости, покладеної до нього підприємцями \emph{D}, \emph{E}, \emph{F} і~\abbr{т. ін.}
З погляду А тут ще не йдеться про акумульований капітал. А в дійсності
для \emph{D}, \emph{E}, \emph{F} і~\abbr{т. ін.} \emph{А} є не що інше, як аґент, що капіталізує привласнену
ними додаткову вартість.

В книзі 1, розділ 22, ми бачили, що акумуляція, перетворення додаткової
вартости на капітал, своїм реальним змістом є процес репродукції
в поширеному маштабі, все одно, чи виявляється таке поширення
екстенсивно у вигляді долучення нових фабрик до старих, чи в інтенсивному
поширенні попереднього маштабу підприємства.

Розмір продукції може поширюватись малими дозами, оскільки
частину додаткової вартости застосовується на такі поліпшення, що або
тільки підвищують продуктивну силу вживаної праці, або разом з тим
дають змогу і визискувати її інтенсивніше. Або ж, коли робочий день не
обмежено законом, досить додаткової витрати обігового капіталу (на матеріяли
продукції та заробітну плату), щоб поширити розміри підприємства,
не збільшуючи основного капіталу, що його денний протяг вживання
таким чином лише подовжується, тимчасом як період обороту його
відповідно скорочується. Або, за сприятливих ринкових коньюнктур, капіталізована
додаткова вартість може дати змогу спекулювати на сировинному
матеріялі, отже, переводити такі операції, що для них не вистачило
б первісно авансованого капіталу, й~\abbr{т. ін.}

А проте, очевидно, що там, де порівняно велике число періодів обороту
зумовлює частішу реалізацію додаткової вартости протягом року,
будуть наставати періоди, коли не можна буде ні подовжувати робочий
день, ні заводити частинні поліпшення; тимчасом як, з другого боку, пропорційне
поширення цілого підприємства, зумовлене почасти загальним
\parbreak{}  %% абзац продовжується на наступній сторінці

\parcont{}  %% абзац починається на попередній сторінці
\index{ii}{0242}  %% посилання на сторінку оригінального видання
характером підприємства, напр., будівель, почасти поширенням фонду
робочої сили, як у сільському господарстві, можливе лише в певних
більш-менш вузьких межах, і для цього треба додаткового капіталу
такого розміру, що його може дати лише багаторічна акумуляція додаткової
вартости.

Отже, поряд справжньої акумуляції або перетворення додаткової
вартости на продуктивний капітал (і відповідної репродукції в поширеному
розмірі) відбувається акумуляція грошей, нагромадження частини додаткової
вартости як лятентного грошового капіталу, який лише пізніше,
досягши певних розмірів, має функціонувати як додатковий активний
капітал.

Так стоїть справа з погляду поодинокого капіталіста. Однак, з розвитком
капіталістичної продукції розвивається одночасно кредитова система.
Грошовий капітал, що його капіталіст ще не може застосувати в своєму
власному підприємстві, застосовує інший і платить за це йому проценти. Він
функціонує для свого власника як грошовий капітал в особливому
значенні, як особливий ґатунок капіталу, відмінний від продуктивного
капіталу. Але він діє як капітал в руках другого. Очевидно, що при
частішій реалізації додаткової вартости і при збільшенні маштабу, що
в ньому її продукується, зростає пропорція, що в ній новий грошовий
капітал, або гроші як капітал, подається на грошовий ринок, а відси
знову вбирається — принаймні більшу частину його — для поширення
продукції.

Найпростіша форма, що в ній може виявлятися цей додатковий лятентний
грошовий капітал, є форма скарбу. Можливо, що цей скарб є
додаткове золото або срібло, одержане безпосередньо або посередньо
в обміні з країнами, що продукують благородні металі. І тільки таким
способом в країні абсолютно зростає грошовий скарб. З другого боку,
можливо — і так здебільша буває, — що цей скарб є не що інше, як
гроші, вилучені з циркуляції всередині країни, що набрали форму скарбу
в руках поодиноких капіталістів. Можливо далі, що цей лятентний грошовий
капітал складається просто з знаків вартости — кредитові гроші
ми тут ще лишаємо осторонь — або з простих, потверджених леґальними
документами вимог (юридичних титулів) капіталістів до третіх осіб. В
усіх цих випадках, хоч яка буде форма буття цього додаткового грошового
капіталу, він, оскільки він є капітал in spe\footnote*{
In spe — досл.; „в надії, в перспективі“, тобто потенціяльно. \emph{Ред.}
}, репрезентує не
що інше, як додаткові та в запасі тримані юридичні титули капіталістів на
майбутню додаткову річну продукцію суспільства.

„Таким чином, маса справді акумульованого багатства, розглядувана
з кількісного боку,\dots{} надзвичайно мала порівняно з продуктивними
силами суспільства, що йому воно належить, хоч на якому щаблі цивілізації
стояло б те суспільство; або навіть порівняно з дійсним споживанням
цього самого суспільства протягом лише небагатьох років; остільки
мала, що головну увагу законодавців та політико-економів треба було б
\parbreak{}  %% абзац продовжується на наступній сторінці

\parcont{}  %% абзац починається на попередній сторінці
\index{ii}{0243}  %% посилання на сторінку оригінального видання
спрямувати на продуктивні сили та на їхній майбутній вільний розвиток,
а не на саме лише акумульоване багатство, що впадає на очі, як це
було до цього часу. Куди більша частина так званого акумульованого
багатства є лише номінальна й складається не з справжніх речей, кораблів,
будинків, бавовняних товарів, меліорацій, а з простих юридичних
титулів, з вимог на майбутні річні продуктивні сили суспільства, з юридичних
титулів, що утворились і увічнились в наслідок засобів або
інституцій незабезпечености\dots{} Вживання таких предметів (нагромаджених
фізичних речей, або справжнього багатства) як простого
засобу, для присвоювання їхніми власниками багатства, яке лише мають
утворити майбутні продуктивні сили суспільства, таке вживання їм поступінно
відібралось би природними законами розподілу, не вживаючи
сили; за допомогою кооперованої праці (Cooperative labour) його
відібралось би їм протягом небагатьох років“. (William Thompson, „An
Inguiry into the principles of the Distribution of Wealth. London 1850, p. 453.
Ця книга вийшла першим виданням 1824 року).

„Мало хто думає, а більшість навіть і гадки не має, яка надто
незначна й масою своєю і силою свого впливу дійсна акумуляція суспільства
порівняно з продуктивними силами людства і навіть порівняно
з звичайним споживанням одного покоління протягом небагатьох лише
років. Причина очевидна, але вплив дуже шкідливий. Багатство, споживане
щороку, зникає разом із споживанням його; воно лише одну мить
навіч перед нами і справляє вражіння лише, поки з нього користуються
або поки його споживають. Але тільки повільно споживана частина
багатства, меблі, машини, будівлі, стоять перед нашими очима з нашого
дитинства й до старости, як довговічні пам’ятники людської праці.
Маючи цю сталу, довговічну, лише повільно споживану частину суспільного
багатства — землю та сировинний матеріял, що до них прикладається
працю, знаряддя, що ними працюють, будівлі, що дають притулок підчас
праці, — маючи все це, власники цих речей в своїх інтересах захоплюють
річні продуктивні сили всіх дійсно продуктивних робітників суспільства,
хоч би які незначні були ці речі порівняно з постійно відновлюваними
продуктами цієї праці. Людність Брітанії та Ірляндії дорівнює 20 мільйонам;
пересічне споживання кожної людини, — чоловіка, жінки, дитини —
становить, мабуть, щось 20\pound{ ф. стерл.}, увесь щорічно споживаний продукт
праці становить багатство приблизно в 400 мільйонів ф. стерл. За оцінкою,
загальна сума акумульованого капіталу в цих країнах не перевищує
1200 мільйонів, або потроєного річного продукту праці; поділивши
нарівно, маємо 60\pound{ ф. стерл.} на душу; тут для нас радше має вагу відношення,
ніж більш або менш точні абсолютні підсумки сум цієї оцінки.
Процентів з цілого цього капіталу було б досить для того, щоб утримувати
всю людність при її теперішньому рівні життя приблизно два
місяці на рік, а всього акумульованого капіталу (коли б знайшлися для
нього покупці) вистачило б на утримання цієї людности протягом цілих
трьох років без якоїбудь роботи! Але потім, опинившись без будівель,
одягу й харчу, люди мусіли б загинути з голоду, або зробитись рабами
\parbreak{}  %% абзац продовжується на наступній сторінці

\input{_0244_0245.tex}
\parcont{}  %% абзац починається на попередній сторінці
\index{ii}{0246}  %% посилання на сторінку оригінального видання
безпосередній або посередній обмін частини річного продукту країни на
продукт країн, що продукують золото й срібло. Однак такий інтернаціональний
характер оборудки замасковує її простоту. Тому, щоб звести
проблему до найпростішого та найвиразнішого виразу, треба припустити,
що продукція золота й срібла відбувається в самій країні, отже, що продукція
золота й срібла становить частину сукупної суспільної продукції
кожної країни.

Лишаючи осторонь золото й срібло, продуковані для речей розкошів,
мінімум щорічної продукції їх мусить дорівнювати зношуванню грошового
металю, зумовленому річною грошовою циркуляцією. Далі: коли
зростає сума вартости маси товарів, яка щорічно продукується й циркулює,
то мусить зростати й річна продукція золота й срібла, оскільки
виросла сума вартости товарів, що циркулюють, і маса грошей, потрібних
для їхньої циркуляції (та для утворення відповідного скарбу), не компенсується
більшою швидкістю грошового обігу та поширенішою функцією
грошей як засобу виплати, тобто частішими взаємними вирівнюваннями
купівель і продажів без посередництва дійсних грошей.

Отже, частину суспільної робочої сили та частину суспільних засобів
продукції треба щороку витрачати на продукцію золота й срібла.

Капіталісти, які провадять продукцію золота й срібла, провадять її, —
як ми тут, за умов простої репродукції, припускаємо — лише в межах
пересічного річного зношування та зумовленого ким пересічного річного
споживання золота й срібла; свою додаткову вартість, що її вони, згідно
з нашим припущенням, споживають щорічно, нічого не капіталізуючи з
неї, вони пускають у циркуляцію безпосередньо в грошовій формі, яка
для них є натуральна форма, а не перетворена форма продукту, як в інших
галузях продукції.

Далі: щодо заробітної плати — грошової форми, що в ній авансується
змінний капітал — то тут її так само заміщується не через продаж
продукту, не через перетворення його на гроші, а самим продуктом, що
натуральна форма його з самого початку є грошова форма.

Нарешті, так само стоїть справа і з тією частиною продукту благородного
металю, яка дорівнює вартості періодично споживаного сталого
капіталу, так сталого обігового, як і сталого основного, споживаного протягом
року.

Розгляньмо кругобіг, зглядно оборот, капіталу, вкладеного в продукцію
благородних металів, насамперед у формі $Г — Т\dots{} П\dots{} Г'$. Оскільки
$Т$ в $Г — Т$ складається не лише з робочої сили та засобів продуції, а також
із основного капіталу, що з нього в $П$ споживається тільки частину
його вартости, то очевидно, що $Г'$ — продукт — є грошова сума, яка
дорівнює змінному капіталові, витраченому на заробітну плату, плюс обіговий
сталий капітал, витрачений на засоби продукції, плюс частина
вартости, яка відповідає зношуванню основного капіталу, плюс додаткова
вартість. Коли б, при незмінній загальній вартості золота, ця сума була
менша, то вкладення капіталу в золоті копальні було б непродуктивне
або, — коли б це явище набрало загального характеру, — то вартість золота
\parbreak{}  %% абзац продовжується на наступній сторінці

\parcont{}  %% абзац починається на попередній сторінці
\index{ii}{0247}  %% посилання на сторінку оригінального видання
в майбутньому зросла б порівняно з товарами, котрих вартість не змінюється;
тобто ціни товарів знизились би, отже, в майбутньому грошова
сума, витрачувана в акті $Г — Т$, зменшилась би.

Якщо розглядати насамперед тільки обігову частину капіталу, авансовуваного
в $Г$, вихідному пунктові $Г — Т\dots{} П\dots{} Г'$, то виявиться, що певну
грошову суму авансується, пускається в циркуляцію на оплату робочої
сили й на закуп матеріялів продукції. Але через кругобіг цього капіталу
її не вилучається знову з циркуляції, щоб знову ж таки подати її
в неї. Продукт є гроші вже в своїй натуральній формі, отже, йому не
доводиться перетворюватись на гроші через обмін, через процес
циркуляції. З процесу продукції в сферу циркуляції він увіходить не в
формі товарового капіталу, що повинен перетворитись на грошовий капітал,
а як грошовий капітал, що повинен перетворитись знову на продуктивний
капітал, тобто знову купувати робочу силу й матеріяли продукції.
Грошову форму обігового капіталу, зужитого на робочу силу й на засоби
продукції, заміщується не через продаж продукту, а натуральною
формою самого продукту; отже, не через зворотне вилучення з циркуляції
вартости продукту в її грошовій формі, а через додаткові новоспродуковані
гроші.

Припустімо, що цей обіговий капітал \deq{} 500\pound{ ф. стерл.}, період обороту
\deq{} 5 тижням, робочий період \deq{} 4 тижням, період циркуляції \deq{} лише 1
тижневі. Гроші з самого початку треба авансувати на 5 тижнів почасти
на продукційний запас, почасти вони мають бути в запасі, щоб можна
було поступінно виплачувати з них заробітну плату. На початку шостого
тижня повертаються назад 400\pound{ ф. стерл.} і звільняються 100\pound{ ф. стерл}. Це
повторюється раз-у-раз. Тут, як і раніш, протягом певного часу обороту
100\pound{ ф. стерл.} завжди будуть в формі вільних грошей. Але вони складаються
з додаткових новоспродукованих грошей, цілком так само, як і останні
400\pound{ ф. стерл}. Ми маємо тут 10 оборотів на рік і спродукований річний
продукт \deq{} 5000\pound{ ф. стерл.} золотом. (Період циркуляції складається тут
не з того часу, що його потребує перетворення товару на гроші, а з
часу, потрібного для перетворення грошей на елементи продукції).

Для всякого іншого капіталу в 500\pound{ ф. стерл.}, що обертається в таких
самих умовах, раз-у-раз відновлювана грошова форма є перетворена форма
спродукованого товарового капіталу, який що чотири тижні подається
в циркуляцію і який у наслідок продажу — отже, в наслідок періодичного
вилучення такої кількости грошей, яка спочатку ввійшла в процес, —
знову й знову набирає цієї грошової форми. Навпаки, тут в кожний період
обороту з самого процесу продукції подається в циркуляцію нову
додаткову суму грошей в 500\pound{ ф. стерл.} для того, щоб постійно вилучати
відти матеріяли продукції та робочу силу. Цих поданих у циркуляцію
грошей не вилучається знову з неї через кругобіг цього капіталу, а їх
дедалі збільшують, додаючи раз-у-раз новоспродуковані маси золота.

Коли ми розглянемо змінну частину цього обігового капіталу й припустимо,
як і перше, що вона дорівнює 100\pound{ ф. стерл.}, то при звичайній
товаровій продукції цих 100\pound{ ф. стерл.} при десятьох оборотах на рік було б
\parbreak{}  %% абзац продовжується на наступній сторінці

\parcont{}  %% абзац починається на попередній сторінці
\index{ii}{0248}  %% посилання на сторінку оригінального видання
досить, щоб постійно оплачувати робочу силу. Тут, у продукції грошей
вистачить такої самої суми; але зворотно приплилі 100\pound{ ф. стерл.}, що ними
кожні 5 тижнів оплачується робочу силу, є не перетворена форма продукту
цієї робочої сили, а частина цього самого постійно відновлюваного
продукту. Золотопромисловець платить своїм робітникам безпосередньо
частиною золота, що вони сами його випродукували. Тому 1000\pound{ ф. стерл.},
щорічно витрачувані таким чином на робочу силу й подавані робітниками
в циркуляцію, не повертаються через циркуляцію до свого вихідного
пункту.

Далі, щодо основного капіталу, то при першому заснуванні підприємства
треба витратити порівняно великий грошовий капітал, що його,
отже, пускається в циркуляцію. Як кожний основний капітал, він повертається
назад лише частинами протягом кількох років. Але він повертається
назад як безпосередня частина продукту, золота, не через продаж
продукту, не через перетворення його таким способом на золото. Отже,
він поступінно набирає своєї грошової форми не через вилучення грошей
з циркуляції, а через нагромадження відповідної частини продукту.
Відновлений таким чином грошовий капітал не є грошова сума, поступінно
вилучувана з циркуляції на покриття грошової суми, первісно кинутої
в циркуляцію на придбання основного капіталу. Це — додаткова
маса грошей.

Нарешті, щодо додаткової вартости, то вона так само дорівнює тій
частині нового продукту — золота, яку в кожний новий період обороту
пускається в циркуляцію, щоб, згідно з нашим припущенням, витратити
Її непродуктивно, на оплату засобів існування та речей розкошів.

Але згідно з нашим припущенням, вся ця річна продукція золота —
що нею постійно вилучається з ринку робочу силу й матеріяли продукції,
але не вилучається з нього грошей, а постійно подається додаткові гроші
— вся ця річна продукція золота тільки заміщує гроші, зношувані протягом
року, отже, лише підтримує в суспільстві сповна ту кількість грошей,
яка постійно, хоч і в змінних долях, існує в двох формах — в формі
скарбу і в формі грошей, що перебувають у циркуляції.

Згідно з законом товарової циркуляції загальна маса грошей мусить
дорівнювати масі грошей, потрібних для циркуляції, плюс кількість грошей,
що перебуває в формі скарбу, а ця остання кількість більшає або
меншає залежно від скорочення або поширення циркуляції; вона ж служить
також і для утворення потрібного резервного фонду засобів виплати.
Вартість товарів мусить сплачуватись грішми, оскільки виплати
взаємно не урівноважуються. Та обставина, що частина цієї вартости
складається з додаткової вартости, тобто нічого не коштувала продавцеві
товарів, абсолютно нічого не змінює в справі. Припустімо, що всі
продуценти є самостійні власники їхніх засобів продукції, отже, що циркуляція
відбувається між самими безпосередніми продуцентами. Коли залишити
осторонь сталу частину їхнього капіталу, то їхній річний додатковий
продукт за аналогією з капіталістичним станом, можна було б поділити
на дві частини: одну — \emph{а}, що тільки заміщує потрібні засоби
\parbreak{}  %% абзац продовжується на наступній сторінці

\parcont{}  %% абзац починається на попередній сторінці
\index{ii}{0249}  %% посилання на сторінку оригінального видання
їхнього існування, другу — b, що її вони почасти витрачають на речі
розкошів, а почасти застосовують на поширення продукції; а — в такому
разі репрезентує змінний капітал, b — додаткову вартість. Але такий поділ
не мав би жодного впливу на величину тієї маси грошей, яка потрібна
для циркуляції цілого їхнього продукту. За інших незмінних умов, вартість
товарової маси, що циркулює, була б та сама, а значить, і маса
потрібних для цього грошей була б та сама. Крім того, при однаковому
поділі періодів обороту продуценти мусили б мати такі самі грошові запаси,
тобто постійно мати в грошовій формі таку саму частину свого
капіталу, бо, згідно з нашим припущенням, їхня продукція, як і раніш,
була б товаровою продукцією. Отже, та обставина, що частина товарової
вартости складається з додаткової вартости, абсолютно не змінює маси
грошей доконечних для провадження підприємства.

Один з супротивників Тука, що тримається формули $Г — Т — Г$, запитує
його, як капіталістові вдається постійно вилучати з циркуляції більше
грошей, ніж він подає туди. Це цілком зрозуміло. Тут ідеться не про
утворення додаткової вартости. Останнє, являючи єдину таємницю, з
капіталістичного погляду само собою зрозуміле. Застосована бо сума вартости
не була б капіталом, коли б вона не збагачувалась додатковою
вартістю. А що згідно з припущенням вона є капітал, то додаткова вартість
сама собою зрозуміла.

Отже, питання не в тім, відки береться додаткова вартість, а в тім,
відки беруться гроші, що на них вона перетворюється.

Але для буржуазної економії існування додаткової вартости зрозуміле
само собою. Отже, її не лише припускається, але разом з нею припускається
й те, що частина товарової маси, пущеної в циркуляцію, складається
з додаткового продукту, отже, репрезентує таку вартість, що її капіталіст
не кинув у циркуляцію, кидаючи туди свій капітал; що, отже, капіталіст,
разом з своїм продуктом кидає в циркуляцію певний надлишок
порівняно з своїм капіталом, а потім знову вилучає з неї цей надлишок.

Товаровий капітал, що його капіталіст подає в циркуляцію, має більшу
вартість (звідки це постає, не пояснюється або не розуміється, але з
погляду буржуазної економії c’est un fait\footnote*{
Це — факт. \emph{Ред.}
}, ніж продуктивний капітал,
що його він вилучив з циркуляції в формі робочої сили плюс засоби
продукції. Тому при цьому припущенні ясно, чому не лише капіталіст
$А$, але й $В$, $С$, $D$ і~\abbr{т. ін.} можуть постійно вилучати з циркуляції через
обмін своїх товарів більшу вартість, ніж вартість їхнього первісно авансованого
капіталу, що його потім знову й знову авансується. $А$, $В$, $С$,
$D$ і~\abbr{т. ін.} завжди подають в циркуляцію в формі товарового капіталу, —
а ця операція так само багатобічна, як і самостійно діющі капітали, —
більшу товарову вартість, ніж та, що її вони вилучають з циркуляції в
формі продуктивного капіталу. Отже, їм постійно доводиться розподіляти
між собою суму вартости (тобто кожному доводиться вилучати для себе
з циркуляції продуктивний капітал), що дорівнює сумі вартости їхніх
\parbreak{}  %% абзац продовжується на наступній сторінці


\index{iii1}{0250}  %% посилання на сторінку оригінального видання
Частина старого капіталу мусила б лишитись без діла при
всяких обставинах, лишитись без діла щодо своєї властивості як
капіталу — функціонувати і зростати в своїй вартості. Яка саме
частина лишилася б без діла, це вирішила б конкурентна боротьба.
Поки все йде добре, конкуренція, як це виявилось при
вирівненні загальної норми зиску, діє, як практичний братерський
союз класу капіталістів, так що вони спільно ділять між собою загальну
здобич пропорціонально до величини частки, вкладеної
кожним з них. Але як тільки справа йде вже не про розподіл
зиску, а про розподіл збитку, то кожний з них намагається
якомога зменшити свою участь в ньому і перекласти його на
шию іншому. Для всього класу збиток є неминучий. Але скільки
з нього припаде на кожного окремого капіталіста, наскільки
взагалі кожний з них повинен взяти участь в ньому, це стає
тоді питанням сили й хитрості, і конкуренція перетворюється
тоді в боротьбу ворогуючих братів. Протилежність між інтересами
кожного окремого капіталіста і інтересами класу капіталістів виявляється
при цьому цілком так само, як перед тим за допомогою
конкуренції проявлялась на практиці тотожність цих інтересів.

Яким же чином міг би бути знов усунений цей конфлікт
і як могли б відновитись відносини, відповідні „здоровому“
рухові капіталістичного виробництва? Спосіб усунення міститься
вже в простому констатуванні конфлікту, про усунення якого
йде мова. Він полягає в залишенні без діла і навіть частковому
знищенні капіталу, рівного своєю вартістю всьому додатковому
капіталові $ΔК$ або принаймні частині його. Хоча — як
це вже випливає з викладу конфлікту — розподіл цього збитку
ні в якому разі не поширюється рівномірно на поодинокі окремі
капітали, а вирішується в конкурентній боротьбі, в якій збиток
розподіляється дуже нерівно і в дуже різних формах, залежно
від особливих переваг або особливих завойованих уже позицій,
так що один капітал лишається лежати без діла, другий знищується,
третій має тільки відносний збиток або зазнає тільки
тимчасового знецінення і т. д.

Але при всяких обставинах рівновага відновилась би в наслідок
бездіяльності і навіть знищення капіталу в більшому чи
меншому розмірі. Це почасти поширилося б на матеріальну
субстанцію капіталу; тобто частина засобів виробництва, основний
і обіговий капітал, не функціонувала б, не діяла б як капітал;
частина підприємств, що вже почали функціонувати, припинила
б роботу. Хоча в цьому відношенні час робить своє
і погіршує всі засоби виробництва (за винятком землі), але тут
в наслідок припинення функціонування мало б місце значно
сильніше справжнє руйнування засобів виробництва. Головний
результат у цьому відношенні був би, однак, у тому, що ці засоби
виробництва перестали б діяти як засоби виробництва, — в зруйнуванні
їх функції як засобів виробництва на коротший чи довший
час.


\index{iii1}{0251}  %% посилання на сторінку оригінального видання
Найбільш руйнівного впливу, і при тому найгострішого
характеру, зазнав би капітал, оскільки він має властивість
вартості, зазнали б капітальні \emph{вартості}. Частина капітальної
вартості, яка перебуває просто у формі посвідок на майбутню
участь в додатковій вартості, в зиску, і яка в дійсності становить
тільки боргові зобов’язання в різних формах на виробництво,
відразу знецінюється з падінням доходів, на які вона розрахована.
Частина готівки золота й срібла лежить без діла, не функціонує
як капітал. Частина товарів, що перебувають на ринку, може
здійснити свій процес циркуляції і репродукції тільки шляхом
надзвичайного зниження своїх цін, отже, шляхом знецінення
того капіталу, який вона представляє. Цілком так само більше
чи менше знецінюються елементи основного капіталу. До цього
долучається ще й те, що певні припущені відношення цін
обумовлюють процес репродукції, і тому цей останній в наслідок
загального падіння цін приходить до застою і розладу. Цей
розлад і застій паралізує функцію грошей як платіжного засобу,
яка розвивається одночасно з розвитком капіталу і грунтується
на згаданих припущених відношеннях цін; він розриває у сотнях
місць ланцюг платіжних зобов’язань на певні строки і ще більше
загострюється в наслідок зумовленого цим краху (Zusammenbrechen)
кредитної системи, що розвинулась одночасно з капіталом,
і таким чином веде до сильних і гострих криз, до раптових
насильних знецінень і дійсного застою й розладу\footnote*{
В першому німецькому виданні тут стоїть: „застою й занепаду (Sturz)“;
виправлено на підставі рукопису Маркса. \emph{Примітка ред. нім. вид, ІМЕЛ.}
} процесу
репродукції, і тим самим до дійсного зменшення репродукції.

Але одночасно діяли б і інші фактори. Застій виробництва
позбавив би роботи частину робітничого класу і цим поставив би
заняту частину його в такі відносини, при яких вона мусила б
згоджуватись на зниження заробітної плати навіть нижче пересічного
рівня; обставина, яка дає для капіталу такий самий результат,
як коли б при пересічній заробітній платі була підвищена
відносна чи абсолютна додаткова вартість. Період процвітання
сприяв би шлюбам серед робітників і зменшив би смертність їх
дітей, обставини, які — хоч би й яке вони означали дійсне збільшення
населення — не означають збільшення дійсно працюючого
населення, але на відношення робітників до капіталу впливають
цілком так само, як коли б збільшилося число дійсно функціонуючих
робітників. З другого боку, падіння цін і конкурентна
боротьба спонукали б кожного капіталіста підвищувати індивідуальну
вартість свого сукупного продукту понад його загальну
вартість за допомогою застосування нових машин, нових поліпшених
методів праці, нових комбінацій, тобто підвищувати продуктивну
силу даної кількості праці, знижувати відношення
змінного капіталу до сталого, і тим самим звільняти робітників,
\parbreak{}  %% абзац продовжується на наступній сторінці

\parcont{}  %% абзац починається на попередній сторінці
\index{iii2}{0252}  %% посилання на сторінку оригінального видання
які витрачається дохід, тобто правити за засоби споживання, перебігають протягом
року різні ступені, наприклад, вовняна пряжа, сукно. На одному ступені
вони становлять частину сталого капіталу, на другому — їх особисто споживається,
отже, цілком, входять в склад доходу. Можна, отже, уявити собі разом
з А. Смітом, що сталий капітал є лише позірний елемент товарової вартости,
який в загальному зв’язку зникає. Таким самим чином відбувається далі обмін
змінного капіталу на дохід. Робітник купує на свою заробітну плату частину
товарів, що становить його дохід. Одночасно він покриває цим самим для капіталіста
грошову форму змінного капіталу. Нарешті: частина продуктів, що становлять
сталий капітал, покривається або in natura, або за посередництвом обміну
між самими продуцентами сталого капіталу; процес, до якого споживачі
не мають жодного чинення. Коли спустити це з уваги, то може постати зовнішня
видимість, що дохід споживачів покриває ввесь продукт, отже і сталу частину
вартости.

5) Крім плутанини, яку вносить перетворення вартостей на ціни продукції,
виникає ще й інша в наслідок перетворення додаткової вартости на різні окремі
форми доходу, самостійні одна проти однієї і залічені до різних елементів продукції,
на зиск і ренту. При цьому забувається, що вартості товарів є основою, і що
розпадання цієї товарової вартости на окремі складові частини, і дальший розвиток
цих складових частин вартости у форми доходу, їх перетворення на відносини
різних посідачів різних чинників продукції до цих окремих складових
частин вартости, їх розподіл між цими посідачами згідно з певними категоріями
і титулами, нічого не змінює у самому визначенні вартости й законів її.
Так само мало змінюється закон вартости тією обставиною, що вирівнювання
зиску, тобто розподіл сукупної додаткової вартости між різними капіталами, і
перешкоди, що почасти (в абсолютній ренті) ставляться землеволодінням на
шляху цього вирівнювання, призводять до відхилу регуляційних пересічних цін
товарів від їхніх індивідуальних вартостей. Це впливає знов таки тільки на
добавку додаткової вартости до цін різних товарів, але не знищує самої додаткової
вартости і сукупної вартости товарів як джерела цих різних складових
частин ціни.

Тут перед нами quid pro quo, яке ми розглядаємо в дальшім розділі, і
яке неминуче зв’язане з ілюзією, що нібито вартість виникає з її власних
складових частин. А саме: спершу різні складові частини вартости товару набувають
в доходах самостійних форм, і як такі доходи їх залічують до окремих
речових елементів продукції, як до їхніх джерел, замість залічити їх до
вартости товару як до їхнього джерела. Вони дійсно залічуються до зазначених
окремих джерел, але не як складові частини вартости, а як доходи, як складові
частини вартости, що дістаються цим певним категоріям агентів продукції: робітникові,
капіталістові, земельному власникові. А проте, можна уявити собі, що
ці складові частини вартости замість виникати від розкладу товарової вартости,
навпаки, лише створюють її своїм сполученням; тоді саме й виникає чудове
порочне коло: вартість товарів виникає з суми вартости заробітної плати, зиску,
ренти, а вартість заробітної плати, зиску, ренти в свою чергу визначається
вартістю товарів і т. ін.\footnote{
«Оборотний капітал, витрачений на матеріяли, сировий матеріял і викінчені вироби, сам
складається з товарів, що їх потрібна ціна складена з тих самих елементів; так що, розглядаючи
сукупність товарів у певній країні, було б зайво зачислювати цю частину оборотного капіталу до
елементів потрібної ціни». (Storch, Cours d’Ec. Pol., II, p. 140). Під цими елементами оборотного
капіталу Шторх розуміє (основний — це тільки змінена форма оборотного) сталу частину вартости.
«Правда, що заробітна плата робітника так само як і частина зиску підприємця, яка складається з
заробітних плат, коли розглядати їх як частину засобів існування, і собі складається з товарів, що
куплені по ринковій ціні, і теж мають в собі заробітні плати, ренти на капітали, земельні ренти і
підприємницькі зиски\dots{} спостереження це доводить лише неможливість розкласти потрібну ціну на її
простіші елементи» (ib., примітка). — У своїх Considérations sur la nature du revenu national (Paris
1824)
Шторх y своїй полеміці s Сеєм, правда, розуміє все безглуздя, що до нього призводить помилкова
аналіза товарової вартости, що розкладає її без остачі тільки на доходи, і правильно висловлюється
про все безглуздя цих висновків — з погляду не поодиноких капіталістів, а нації, — але сам він не
робить і кроку вперед в аналізі prix nécessaire (потрібної ціни), відносно якої він, замість
відсувати розв’язання
завдання до безконечности, заявляє в своєму «Cours» що її неможливо розкласти на її дійсні елементи.
«Ясно, що вартість річного продукту поділяється почасти на капітали, почасти на зиски, і що кожна з
цих частин вартости річного продукту регулярно купує продукти, потрібні нації так для збереження її
капіталу, як і для відновлення її споживного фонду (р. 134--135)\dots{} Чи зможе вона (селянська родина,
що працює самостійно) жити у своїх клунях і стайнях, живитись тільки насінням
і травою, одягатися з своєї робочої худоби, витрачати свої хліборобські зваряддя? Згідно з
твердженими п. Сея, слід було б відповісти позитивно на всі ці питання (135--136)\dots{} Коли визнати, що
дохід нації дорівнює її гуртовому продуктові, тобто, що з нього не доводиться вираховувати капітали,
то доведеться також визнати, що вона може непродуктивно витратити всю вартість свого річного
продукту,
не роблячи найменшої шкоди своєму майбутньому доходові. (147) Продукти, що складають капітал нації,
не підлягають споживанню», (р. 150)
}.

\index{iii2}{0253}  %% посилання на сторінку оригінального видання
При нормальному стані продукції тільки частина новодолученої праці
вживається на продукцію і тому на покриття сталого капіталу: саме якраз та
частина, що покриває сталий капітал, витрачений у продукції засобів споживання,
речових елементів доходу. Це вирівнюється тим, що ця стала частина,
кляси II не коштує новодолучуваної праці. Але сталий капітал, який (коли розглядати
сукупний процес репродукції, що в ньому, отже, включено і те вирівнювання
кляс І і II), не є продукт новодолученої праці, хоч цей продукт неможливо
було б випродукувати без нього, — цей сталий капітал, розглядуваний з речового
боку, підлягає підчас процесу репродукції, випадковостям і небезпекам,
які можуть його зменшити. (Але далі, коли розглядати його щодо вартости, то
він також може знецінитися в наслідок зміни у продуктивній силі праці;
проте, це стосується лише до поодиноких капіталістів). Відповідно до цього частина
зиску, отже, додаткової вартости, а тому і додаткового продукту, що в ньому
(коли розглядати його з погляду вартости) репрезентується лише новодолучена
праця, служить страховим фондом. При цьому суть справи ані трохи не змінюється
від того, чи порядкує цим страховим фондом страхове товариство як
окреме підприємство, чи ні. Це є однісінька частина доходу, що не споживається
як такий, і не служить неодмінно фондом акумуляції. Чи служить вона фактично
фондом акумуляції, чи лише покриває прогріхи репродукції, це залежить від
випадку. Це також однісінька частина додаткової вартости і додаткового продукту,
отже, додаткової праці, що крім частини, яка служить для акумуляції, отже,
для поширення процесу репродукції, мусить існувати і далі по знищенні
капіталістичного способу продукції. Звичайно, це має своєю передумовою, що
частина, регулярно споживувана безпосередніми продуцентами, не лишиться обмеженою на своєму
теперішньому мінімальному рівні. За винятком додаткової
праці на тих, хто через свій вік ще не може або вже не може брати участи у продукції,
відпаде всяка праця на утримання тих, хто не працює. Коли ми уявимо
собі суспільство при його виниканні, то побачимо, що тут немає ще випродукованих
засобів продукції, отже, немає сталого капіталу, що його вартість увіходить
у продукт, і при репродукції в тому самому маштабі мусить покриватися
in natura з продукту в розмірі, визначуваному його вартістю. Але природа
безпосередньо дає тут засоби існування, їх не доводиться продукувати. Тому
вона залишає також дикунові, що йому доводиться задовольняти лише малі потреби,
час, — не на те, щоб використати ще не сущі в наявності засоби продукції для
нової продукції, а на те, щоб, крім праці, якої коштує привласнення наявних
у природі засобів існування, витрачати працю на перетворення інших продуктів
природи на засоби продукції, лук, кам’яний ніж, човен і т. ін. Процес цей,
\parbreak{}  %% абзац продовжується на наступній сторінці

\input{_0254.tex}
\input{_0255_0256.tex}
\input{_0257.tex}
\parcont{}  %% абзац починається на попередній сторінці
\index{iii1}{0258}  %% посилання на сторінку оригінального видання
виробництва товару $= \sfrac{1}{2} + 17\sfrac{1}{2} + 2 = 20\text{ шилінгам}$, пересічна
норма зиску $\frac{2}{20} = 10\%$, а ціна виробництва штуки товару дорівнює
його вартості = 22 шилінгам або маркам.

Припустім, що винайдено машину, яка наполовину скорочує
потрібну на кожну штуку товару живу працю, але зате
збільшує втроє ту частину вартості, яка складається з зношування
основного капіталу. Тоді справа стоятиме так: зношування
= 1\sfrac{1}{2} шилінгам, сировинний та допоміжний матеріал, як
і раніше, 17\sfrac{1}{2} шилінгів, заробітна плата 1 шилінг, додаткова
вартість 1 шилінг, разом 21 шилінг або 21 марка. Вартість товару
зменшилась тепер на 1 шилінг; нова машина безперечно підвищила
продуктивну силу праці. Але для капіталіста справа стоятиме
так: його витрати виробництва є тепер: 1\sfrac{1}{2} шилінги зношування,
17\sfrac{1}{2} шилінгів — сировинний і допоміжний матеріал,
1 шилінг — заробітна плата, разом 20 шилінгів, як і раніш.
Через те що норма зиску безпосередньо в наслідок застосування
нової машини не змінюється, він мусить одержати 10\%
понад витрати виробництва, що становить 2 шилінги; отже, ціна
виробництва лишилась незмінною = 22 шилінгам, але вона на 1 шилінг
вища вартості. Для суспільства, яке виробляє при капіталістичних
умовах, товар \emph{не} став дешевшим, нова машина не являє
собою \emph{ніякого} поліпшення. Отже, капіталіст не має ніякого
інтересу в тому, щоб вводити нову машину. А через те що
введенням нової машини він просто зробив би нічого невартою
свою стару, ще не зношену машину, перетворив би її просто
в старе залізо, отже, зазнав би позитивного збитку, то він дуже
стережеться такої утопічної для нього дурості.

Отже, для капіталу закон зростаючої продуктивної сили праці
має не безумовне значення. Для капіталу ця продуктивна сила
підвищується не тоді, коли взагалі заощаджується жива праця,
а тільки тоді, коли на \emph{оплачуваній} частині живої праці заощаджується
більше, ніж додається минулої праці, як це вже коротко
зазначено було в книзі І, розділ XIII, 2, стор. 411\footnote*{
Стор. 297--298 рос. вид. 1935 р. Ред. укр. перекладу.
}. Тут
капіталістичний спосіб виробництва впадає в нову суперечність.
Його історичне покликання — нестримний розвиток продуктивності
людської праці, підготований вперед у геометричній прогресії.
Він зраджує це покликання, оскільки він, як у даному
випадку, перешкоджає розвиткові продуктивності. Цим він тільки
знову доводить, що він хиріє від старості і все більше й більше
переживає себе.]\footnote{
Вищенаведене стоїть у дужках, тому що хоч це і є переробка з примітки
оригіналу рукопису, але у викладі деяких моментів воно виходить за
межі того матеріалу, що є в оригіналі. — \emph{Ф. Е.}
}

\pfbreak{}

В конкуренції збільшення мінімуму капіталу, який з підвищенням
продуктивної сили стає потрібним для успішного ведення
\index{iii1}{0259}  %% посилання на сторінку оригінального видання
самостійного промислового підприємства, виявляється
так: як тільки нове дорожче промислове устаткування стає
загальнопоширеним, дрібніші капітали на майбутнє виключаються
з цього виробництва. Тільки на перших порах механічних винаходів
у різних сферах виробництва дрібніші капітали можуть
в них самостійно функціонувати. З другого боку, дуже великі
підприємства, з надзвичайно високим відношенням сталого капіталу,
як залізниці, дають не пересічну норму зиску, а тільки
частину її, процент. Інакше загальна норма зиску знизилась би
ще більше. Навпаки, і тут великі капітали, зібрані в формі акцій,
знаходять собі поле для безпосереднього застосування.

Зростання капіталу, отже, нагромадження капіталу, включає
зменшення норми зиску лиш остільки, оскільки разом з цим зростанням
настають розглянуті нами вище зміни у відношенні органічних
складових частин капіталу. Однак, не зважаючи на постійні,
повсякденні перевороти в способі виробництва, та чи інша,
більша чи менша частина всього капіталу протягом певного
часу продовжує нагромаджуватися на базі даного пересічного відношення
цих складових частин, так що з зростанням цієї частини
не сполучена ніяка органічна переміна, отже й ніякі причини
падіння норми зиску. Це постійне збільшення капіталу, а тому
й розширення виробництва на основі старих методів виробництва,
яке спокійно триває далі, тимчасом як поряд з ними
вводяться вже нові методи, знов таки є причиною того, що
норма зиску зменшується не в тій мірі, в якій зростає сукупний
капітал суспільства.

Збільшення абсолютного числа робітників, не зважаючи на
відносне зменшення змінного капіталу, витрачуваного на заробітну
плату, відбувається не в усіх галузях виробництва і не
в усіх рівномірно. В землеробстві зменшення елементу живої
праці може бути абсолютним.

Зрештою, абсолютне збільшення числа найманих робітників,
не зважаючи на його відносне зменшення, є тільки потреба капіталістичного
способу виробництва. Для нього робочі сили
стають уже зайвими, як тільки немає вже необхідності примушувати
їх працювати 12--15 годин на день. Розвиток продуктивних
сил, який зменшував би абсолютне число робітників, тобто
в дійсності давав би змогу всій нації виконувати своє сукупне
виробництво за коротший час, викликав би революцію, бо він
вивів би в тираж більшість населення. В цьому знову виявляється
специфічна межа капіталістичного виробництва, а також
те, що воно ніяк не є абсолютною формою для розвитку продуктивних
сил і створення багатства, що воно, навпаки, на певному
пункті вступає в колізію з цим розвитком. Частково така колізія
виявляється в періодичних кризах, які походять з того, що
то одна, то друга частина робітничого населення робиться зайвою
в своїй старій професії. Межа капіталістичного виробництва
— надлишковий час робітників. Абсолютний надлишковий
\parbreak{}  %% абзац продовжується на наступній сторінці

\parcont{}  %% абзац починається на попередній сторінці
\index{iii1}{0260}  %% посилання на сторінку оригінального видання
час, який виграє суспільство, не цікавить капіталістичне виробництво.
Розвиток продуктивної сили для нього важливий лиш
остільки, оскільки він збільшує додатковий робочий час робітничого
класу, а не оскільки він взагалі зменшує робочий час
матеріального виробництва; таким чином капіталістичне виробництво
рухається в суперечностях.

Ми бачили, що зростаюче нагромадження капіталу включає
зростаючу концентрацію його. Таким чином зростає влада капіталу,
персоніфіковане в капіталісті усамостійнення суспільних
умов виробництва проти дійсних виробників. Капітал дедалі більше
виявляє себе як суспільна сила, яка функціонує через капіталіста
і яка не стоїть уже в ніякому відношенні до того, що
може створити праця окремого індивіда, — але як відчужена,
усамостійнена суспільна сила, що як річ, і за допомогою цієї
речі як влада капіталіста, протистоїть суспільству. Суперечність
між загальною суспільною силою, в яку перетворюється капітал,
і приватною владою окремих капіталістів над цими суспільними
умовами виробництва розвивається в дедалі більш кричущу суперечність
і включає в собі розв’язання цього відношення,
оскільки воно разом з тим передбачає вироблення умов виробництва
у загальні, колективні, суспільні умови виробництва.
Це вироблення визначається розвитком продуктивних сил при
капіталістичному виробництві і тим способом, яким відбувається
цей розвиток.

\pfbreak{}

Жоден капіталіст не застосовує добровільно нового способу
виробництва, хоч би наскільки він був продуктивніший і хоч би
наскільки він збільшував норму додаткової вартості, якщо він
зменшує норму зиску. Але кожен такий новий спосіб виробництва
здешевлює товари. Тому капіталіст спочатку продає
їх вище їх ціни виробництва, може, вище їх вартості. Він кладе
собі в кишеню ріжницю між їх витратами виробництва і ринковою
ціною всіх інших товарів, вироблених при вищих витратах
виробництва. Він може це робити тому, що пересічний робочий
час, суспільно потрібний для виробництва цих товарів, є більший,
ніж робочий час, потрібний при новому способі виробництва.
Його методи виробництва стоять вище пересічних суспільних.
Але конкуренція робить їх загальними і підпорядковує їх загальному
законові. Тоді починається зниження норми зиску, — спочатку,
може, в цій сфері виробництва, а потім вона вирівнюється
з іншими, — яке, отже, цілком незалежне від волі капіталістів.

З приводу цього треба ще зауважити, що цей самий закон
панує і в тих сферах виробництва, продукт яких ні безпосередньо,
ні посередньо не входить у споживання робітника або
в умови виробництва його засобів існування; отже, і в тих сферах
виробництва, в яких ніяке здешевлення товарів не може збільшити
відносну додаткову вартість, здешевити робочу силу.
\parbreak{}  %% абзац продовжується на наступній сторінці

\input{_0261.tex}
\parcont{}  %% абзац починається на попередній сторінці
\index{iii1}{0262}  %% посилання на сторінку оригінального видання
і зв’язаної з цим легкості перетворювати гроші в капітал без того,
щоб самому ставати промисловим капіталістом. Поп’яте, в наслідок
зростання потреб і жадоби збагачення. Пошосте, в наслідок
зростання маси вкладуваного основного капіталу і т. д.

\pfbreak{}

Три головні факти капіталістичного виробництва:

1) Концентрація засобів виробництва в небагатьох руках,
в наслідок чого вони перестають бути власністю безпосередніх
робітників і перетворюються, навпаки, в суспільні сили виробництва.
Хоча такими вони стають спочатку як приватна власність
капіталістів. Ці останні є trustees [довірені] буржуазного суспільства,
але вони кладуть у свою кишеню всі плоди цього довірення.

2) Організація самої праці як суспільної праці: за допомогою
кооперації, поділу праці і сполучення праці з природознавством.

Як з того, так і з другого боку капіталістичний спосіб виробництва
знищує приватну власність і приватну працю, хоч
знищує в антагоністичних формах.

3) Утворення світового ринку.

Величезна порівняно з населенням продуктивна сила, яка розвивається
при капіталістичному способі виробництва, і зростання
— хоч і не в тій самій пропорції — капітальних вартостей
(не тільки їх матеріального субстрату), які зростають далеко
швидше, ніж населення, суперечать базі, яка порівняно з зростаючим
багатством стає дедалі вужчою і для якої діє ця величезна
продуктивна сила, і відносинам зростання вартості цього
дедалі наростаючого капіталу. Звідси кризи.



  \input{part.4.5.tex}
  
\index{iii2}{0103}  %% посилання на сторінку оригінального видання
\chapter{Перетворення надзиску на земельну ренту}

\section{Вступ}

Аналіза земельної власности в її різних історичних формах лежить поза
межами цієї праці. Ми спиняємось на ній лише остільки, оскільки частина додаткової
вартости, випродукуваної капіталом, припадає земельному власникові.
Отже, ми припускаємо, що в хліборобстві, цілком так само, як в мануфактурі
панує капіталістичний спосіб продукції, тобто що сільське господарство провадять
капіталісти, які відрізняються від решти капіталістів передусім лише тим
елементом, до якого прикладається їхній капітал та наймана праця, яку цей
капітал пускає в рух. На наш погляд фармер продукує пшеницю і т. ін. так
само, як фабрикант — пряжу або машини. Та передумова, що капіталістичний спосіб
продукції опанував сільське господарство, має в собі й те, що цей спосіб продукції
опановує всі сфери продукції й буржуазного суспільства, що, отже,
є наявні і його цілком розвинуті умови, як от вільна конкуренція капіталів,
змога переносити їх з однієї сфери продукції до іншої, однакова висота пересічного
зиску і т. ін. Та форма земельної власности, що її ми розглядаємо,
становить специфічно історичну її форму, \emph{перетворену} — в наслідок впливу
капіталу та капіталістичного способу продукції — форму або февдальної земельної
власности, або дрібно-селянського хліборобства, що провадиться як ділянка
для прохарчування, хліборобства, що в ньому \emph{володіння} землею для безпосередного
продуцента є одна з умов продукції, а його, того продуцента \emph{власність}
на землю є найвигідніша умова розцвіту \emph{його} способу продукції. Якщо
капіталістичний спосіб продукції взагалі має собі за передумову експропріацію
умов праці в робітників, то в хліборобстві він має собі за передумову
експропріацію землі в сільських робітників та підпорядкування їх капіталістові,
що провадить хліборобство за-для зиску. Отже, для нашої аналізи цілком байдуже,
коли нам заперечуватимуть, нагадуючи, що були або ще й досі є й інші
форми земельної власности та хліборобства. Це заперечення може вразити тільки
тих економістів, що розглядають капіталістичний спосіб продукції в сільському
господарстві та відповідну йому форму земельної власности не як історичні, а як
вічні категорії.

Для нас розгляд новітньої форми земельної власности потрібний тому, що
взагалі справа йде про розгляд тих певних відносин продукції й обміну, що
\parbreak{}  %% абзац продовжується на наступній сторінці


\index{ii}{0104}  %% посилання на сторінку оригінального видання
\chapter{Оборот капіталу}

\section{Час обороту й число оборотів}

Ми бачили: сукупний час циркуляції даного\footnote*{
Термін „сукупний час циркуляції“ тут Маркс вживає в тому самому розумінні,
в якому він далі в цьому ж розділі вживає термін „час обороту“, тимчасом
як взагалі він з цій книзі термін „час циркуляції“ вживає в тому самому
розумінні, що і „час обігу“, тобто в розумінні того часу, що протягом його капітал
перебуває в сфері циркуляції. (Дивись розділ V). \emph{Ред.}
} капіталу дорівнює сумі
часу його обігу та часу його продукції. Це є відтинок часу від моменту
авансування капітальної вартости в певній формі до моменту, коли капітальна
вартість, що процесує, повертається в тій самій формі.

Мета, що визначає капіталістичну продукцію, завжди є зростання
авансованої вартости, чи авансовано цю вартість в її самостійній формі,
тобто в грошовій формі, чи в формі товару, так що його форма вартости
має лише ідеальну самостійність у ціні авансованих товарів.
В обох випадках ця капітальна вартість перебігає протягом свого кругобігу
різні форми існування. Її тотожність з самою собою констатується
в книгах капіталіста або в формі рахункових грошей.

Хоч візьмемо ми форму $Г\dots{} Г'$, хоч форму $П\dots{} П$, обидві форми
значать: 1) що авансована вартість функціонувала як капітальна вартість
і зросла своєю вартістю; 2) що по закінченні процесу вона повернулась
до тієї форми, в якій почала його. Зростання авансованої вартости Г і
разом з тим поворот капіталу до цієї форми (до грошової форми) виразно
помітно в $Г\dots{} Г'$. Але те саме відбувається і в другій формі. Бо
вихідний пункт для П є наявність елементів продукції, товарів даної
вартости. Ця форма має в собі зростання цієї вартости (Т' і $Г'$) і поворот
до первісної форми, бо в другому П авансована вартість
знову має форму елементів продукції, що в ній її первісно авансовано.

Раніше ми бачили: „Якщо продукція має капіталістичну форму, то
і репродукція має ту саму форму. Як процес праці за капіталістичного
способу продукції є лише засіб для процесу зростання вартости, так
\parbreak{}  %% абзац продовжується на наступній сторінці

\parcont{}  %% абзац починається на попередній сторінці
\index{ii}{0105}  %% посилання на сторінку оригінального видання
само й репродукція є лише засіб репродукувати авансовану вартість
як капітал, тобто як вартість, що зростає сама з себе“. (Книга І,
розд. XXI).

Три форми: І) $Г\dots{} Г'$, II) $П\dots{} П$ і III) $Т'\dots{} Т'$ відрізняються між
собою ось чим: в формі II (П\dots{} П) відновлення процесу, процесу репродукції,
виражено як дійсне, а в формі І лише як можливе. Але обидві ці
форми відрізняються від форми III тим, що авансована капітальна вартість
— хоч її авансовано як гроші, хоч в вигляді речових елементів продукції
— становить вихідний пункт, а тому й пункт повороту. В $Г\dots{} Г' п$оворот
є $Г' \deq{} Г \dplus{} г$. Коли процес відновлюється знову в тих самих розмірах,
то Г знову становить вихідний пункт, а г не входить в цей процес і лише
показує нам, що Г зросло своєю вартістю як капітал і тому створило додаткову
вартість г, але відштовхнуло її від себе. В формі $П\dots{} П$ капітальна
вартість, авансована в формі П, елементів продукції, знову таки
становить вихідний пункт. Ця форма має в собі й зростання цієї вартости.
Коли відбувається проста репродукція, то та сама капітальна вартість
в тій самій формі П знову починає свій процес. Коли відбуваєтьсяакумуляція,
то тепер процес починає $П'$ (що величиною вартости дорівнює
$Г' \deq{} Т'$), як збільшена капітальна вартість. Але процес починається знову
авансованою капітальною вартістю в початковій формі, хоч і капітальною
вартістю більшою, ніж раніш. Навпаки, в формі III капітальна вартість
починає процес не як авансована, але як уже виросла, як усе багатство,
що перебуває в формі товарів, і що лише деяка частина його являє
авансовану капітальну вартість. Остання форма важлива для третього
відділу, де рух поодиноких капіталів береться в зв’язку з рухом сукупного
суспільного капіталу. Але, навпаки, з неї не можна користатись,
коли досліджується оборот капіталу, що завжди починається авансуванням
капітальної вартости, чи то у формі грошей, чи то у формі товару, і
який завжди зумовлює, що капітальна вартість, яка чинить оборот, повертається
в тій формі, що в ній її авансовано. З кругобігів І і II
треба триматися першого, коли мають на увазі переважно той вплив, що
його справляє оборот на утворення додаткової вартости; другого — коли
мають на увазі вплив обороту на утворення продукту.

Як мало економісти відрізняли різні форми кругобігів, так само мало
вони розглядали ці різні форми кругобігів відокремлено щодо обороту
капіталу. Звичайно береться форму $Г\dots{} Г'$, бо вона панує над поодиноким
капіталістом і служить йому в його розрахунках навіть тоді, коли
гроші становлять вихідний пункт лише в формі рахункових грошей. Інші
беруть за вихідний пункт витрати в формі елементів продукції, поки
не настане поворот, при цьому про форму повороту — чи буде цей поворот
в товарі, чи в грошах — у них немає й мови. Напр.: „Економічний
цикл\dots{} тобто ввесь перебіг продукції від часу, коли зроблено витрати,
до часу, коли настає поворот. В сільському господарстві час засіву є
початок економічного циклу, а жнива — закінчення“. (Economic Cycle\dots{}
the whole course of production, from the time that outlays are made till
returns are received. In agriculture seedtime is its commencement, and
\parbreak{}  %% абзац продовжується на наступній сторінці

\parcont{}  %% абзац починається на попередній сторінці
\index{ii}{0106}  %% посилання на сторінку оригінального видання
harvesting its ending. — S. P. Newman. Elements of Pol. Econ.; Andover
and New-York, p. 81). Інші починають з $T'$ (III форми): „Можна вважати,
що світ продукційного обміну рухається колом, що його ми можемо
назвати економічним циклом, і де кожний обіг вивершується, скоро підприємство,
проробивши ряд послідовних оборудок, знову доходить пункта,
відки вийшло. За початок можна вважати той пункт, коли капіталіст
одержує надходження, що за їх посередництвом до нього повертається
його капітал, з цього пункту він знову переходить до того, щоб навербувати
собі робітників і розподілити між ними в формі заробітної плати
засоби їхнього існування або скорше силу, потрібну на придбання цих
засобів; одержати від них готові речі, що їх він виробляє; подати ці
речі на ринок і там довести до кінця кругобіг ряду цих рухів, продаючи
ці речі й одержуючи в уторгованих за товар грошах покриття всіх своїх
капітальних витрат“. (Th. Chalmers, on Pol. Econ., 2. ed., London, 1832 p.
84 і далі).

Скоро тільки вся капітальна вартість, вкладена поодинокими капіталістами
в якубудь галузь продукції, вивершить у своєму русі кругобіг,
вона знову опиняється в своїй початковій формі і може тепер повторити
той самий процес. Щоб вартість увічнилась і далі зростала, як капітальна
вартість, вона мусить повторювати цей кругобіг. В житті капіталу поодинокий
кругобіг становить лише один постійно повторюваний відділ,
отже, період. Наприкінці періоду $Г\dots{} Г'$ капітал знову перебуває в формі
грошового капіталу, що знову перебігає ряд перетворень форми, які
охоплюють процес його репродукції, зглядно процес зростання вартости.
При закінченні періоду $П\dots{} П$ капітал знову перебуває в формі елементів
продукції, які є передумова для відновлення його кругобігу. Кругобіг
капіталу, визначуваний не як поодинокий акт, а як періодичний процес,
зветься оборотом капіталу. Протяг цього обороту дано сумою часу
його продукції та часу його обігу. Ця сума часу становить час обороту
капіталу. Отже, вона охоплює переміжок часу від одного періоду кругобігу
цілої капітальної вартости до наступного, вона позначає періодичність
у життьовому процесі капіталу, або, коли хочете, час відновлення, повторення
процесу зростання вартости, зглядно процесу продукції тієї самої
капітальної вартости.

Лишаючи осторонь індивідуальні пригоди, що можуть для окремого
капіталу подовжити або скоротити час обороту, цей час є різний залежно
від різних сфер приміщення капіталу.

Так само, як для функції робочої сили за природну вимірну одиницю
є робочий день, так само рік є природна вимірна одиниця для оооротів капіталу,
що процесує. Природну основу такої одиниці виміру являє та обставина,
що в помірній смузі, батьківщині капіталістичної продукції, найважливіші
плоди земні є річні продукти.

Коли рік як вимірну одиницю часу обороту ми позначимо $О$, час
обороту певного капіталу — $о$, число його оборотів $n$, то $n \deq{} \frac{О}{о}$ Отже,
\parbreak{}  %% абзац продовжується на наступній сторінці

\input{_0107.tex}
\input{_0108.tex}
\input{_0109.tex}
\input{_0110.tex}
\input{_0111.tex}
\parcont{}  %% абзац починається на попередній сторінці
\index{ii}{0112}  %% посилання на сторінку оригінального видання
відбирає їм його в другому випадку. Однак та обставина, що засоби праці льокально прикріплені,
пустили своє коріння в землю, надає цій частині основного капіталу особливої ролі в економії націй.
Їх не можна відіслати за кордон, вони не можуть циркулювати як товари на світовому ринку. Титули
власности на цей основний капітал можуть змінюватись, їх можна купувати й продавати, і остільки вони
можуть ідеально
циркулювати. Ці титули власности можуть навіть циркулювати на закордонних ринках, напр., в формі
акцій. Але від зміни осіб, що є власники такого виду основного капіталу, не змінюється відношення
між нерухомою, матеріяльно фіксованою частиною багатства даної країни і рухомою частиною того таки
багатства \footnote{До цього місця рукопис IV.~Відси рукопис II.~Ф.~Е.}).

Своєрідна циркуляція основного капіталу зумовлює своєрідний оборот. Та частина вартости, що її
втрачається в її натуральній формі в наслідок зношування, циркулює, як частина вартости продукту.
Продукт через свою циркуляцію перетворюється з товару на гроші, отже, на гроші перетворюється й та
частина вартости засобів праці, що її продукт несе в циркуляцію, а саме: ця частина вартости падає
краплями як гроші з процесу циркуляції, в тій самій пропорції, що в ній даний засіб праці перестає
бути носієм вартости в продукційному процесі. Отже, вартість цього засобу праці набирає тепер
двоїстого існування. Частина її лишається зв’язана з його споживною або натуральною формою, належною
продукційному процесові, а друга частина відокремлюється від неї як гроші. В перебігу свого
функціонування та частина вартости засобів праці, що існує в натуральній формі, постійно меншає,
тимчасом як перетворена на гроші частина вартости постійно більшає, поки, нарешті, засоби праці
одживуть свій вік, і вся їхня вартість, відокремившись від мертвого тіла, перетвориться на гроші.
Тут виявляється своєрідність в обороті цього елемента продуктивного капіталу. Його вартість
перетворюється на гроші рівнобіжно з тим, як на грошову лялечку перетворюється той товар, що є носій
його вартости. Але його зворотне перетворення з грошової форми на споживну форму відділяється від
зворотного перетворення товару на інші елементи продукції цього товару і визначається періодом його
власної репродукції, тобто часом, що протягом його засоби праці одживають свій вік, і треба їх
замінити на нові екземпляри такого самого роду. Коли час функціонування якоїсь машини, напр.,
вартістю в \num{10.000}\pound{ ф. стерл.}, дорівнює, припустімо, 10 рокам, то час обороту вартости, первісно
авансованої на неї, дорівнює 10 рокам. Поки не мине цей час, її не треба поновлювати, і вона
функціонує далі в своїй натуральній формі. Тимчасом її вартість частинами циркулює як частина
вартости товарів, що до їх безперервної продукції вона придається, — і таким чином поступінно
перетворюється на гроші, поки, нарешті, по десятьох роках, вона цілком перетвориться на гроші, а з
грошей знову на машину, вивершуючи, отже, свій оборот. До цього моменту
\parbreak{}  %% абзац продовжується на наступній сторінці

\input{_0113c.tex}
\input{_0114.tex}
\parcont{}  %% абзац починається на попередній сторінці
\index{ii}{0115}  %% посилання на сторінку оригінального видання
й поновлювати зворотною купівлею, зворотним перетворенням з грошової форми на елементи продукції.
Одним заходом їх вилучається з ринку меншими масами, ніж елементи основного капіталу, але тим
частіше доводиться їх вилучати з ринку, а тому авансування витраченого на них капіталу поновлюється
через коротші періоди. Це постійне поновлення упосереднюється постійним збутом того продукту, що в
ньому циркулює вся їхня вартість. Нарешті, вони безупинно пророблюють увесь кругобіг метаморфоз не
лише своєю вартістю, але й у своїй речовій формі; з товару вони постійно перетворюються знову на
елементи продукції цього самого товару.

Разом із своєю власною вартістю робоча сила постійно долучає до продукту додаткову вартість,
втілення неоплаченої праці. Отже, готовий продукт так само подає її постійно в циркуляцію, і вона
разом з ним перетворюється на гроші так само, як і інші елементи вартости продукту. Однак, тут, де
йдеться насамперед про оборот капітальної вартости, а не додаткової вартости, що обертається разом з
нею, — тут ми лишаємо це покищо осторонь.

З наведеного вище випливає ось що:

1) Визначеності форми основного й поточного капіталу походять лише з ріжниці в обороті капітальної
вартости, що функціонує в процесі продукції, або продуктивного капіталу. Ця ріжниця в обороті
походить і собі з ріжниці в способі, що ним різні складові частини продуктивного капіталу переносять
свою вартість на продукт, а не з їхньої різної участи в утворенні вартости продукту або не з
характеристичної ролі їх у процесі зростання вартости. Нарешті, ріжниця в передачі вартости
продуктові, — а тому й різні способи, що ними ця вартість вводиться через продукт у циркуляцію і в
наслідок його метаморфоз поновлюється в своїй первісній натуральній формі, — ця ріжниця походить з
відмінности тих речових форм, що в них існує продуктивний капітал, і що з них одна частина під час
утворення окремого продукту споживається цілком, а другу зужитковується лише поступінно. Отже, лише
продуктивний капітал може розподілятись на основний і поточний. Навпаки, цієї протилежности не існує
для обох інших способів буття промислового капіталу, отже, ні для товарового капіталу, ні для
грошового капіталу; не існує її також як і протилежности цих обох форм проти продуктивного капіталу.
Вона існує лише для продуктивного капіталу і в межах його. Грошовий капітал і товаровий капітал
можуть скільки завгодно функціонувати як капітал і можуть хоч як швидко циркулювати, але зробитись
поточним капіталом протилежно до основного вони можуть лише тоді, коли перетворяться на поточні
складові частини продуктивного капіталу. Але через те, що ці обидві форми капіталу перебувають у
сфері циркуляції, то, як ми побачимо, економія від часів А.~Сміса не могла стриматися від спокуси
сплутати їх з поточною частиною продуктивного капіталу, об’єднуючи їх в категорію обіговий капітал.
А справді грошовий капітал і товаровий капітал є капітал циркуляції протилежно до продуктивного, але
не обіговий капітал протилежно до основного.


\index{ii}{0116}  %% посилання на сторінку оригінального видання
2) Оборот основної складової частини капіталу, а, значить, і потрібний для цього час обороту,
охоплює кілька оборотів поточної складової частини капіталу. Протягом того самого часу, коли
основний капітал зробив один оборот, поточний капітал робить їх кілька. Одна з складових частин
вартости продуктивного капіталу набуває визначености форми основного капіталу лише остільки,
оскільки засіб продукції, що в ньому вона існує, не зужитковується цілком в той час, що протягом
його продукт виготовляється і з продукційного процесу викидається як товар. Деяка частина йою
вартости мусить лишатися зв’язаною в старій, далі збереженій споживній формі, тимчасом як друга
частина циркулює в наслідок циркуляції готового продукту; навпаки, щодо поточних складових частин
капіталу, то разом з циркуляцією готового продукту циркулює вся їхня вартість.

3) Витрачувану на основний капітал частину вартости продуктивного капіталу авансується цілком одним
заходом на ввесь час функціонування тієї частини засобів продукції, що з неї складається основний
капітал. Отже, капіталіст одним заходом кидає цю вартість в циркуляцію, а вилучається її знову з
циркуляції лише частинами й поступінно через реалізацію тих частин вартости, що їх основний капітал
частинами долучає до товарів. З другого боку, сами засоби продукції, що в них фіксується одна з
складових частин продуктивного капіталу, вилучаються з циркуляції одним заходом, і на ввесь час
свого функціонування їх зв’язується з продукційним процесом. Але протягом цього часу не треба їх
замінювати на нові екземпляри того самого роду, ні репродукувати. Протягом довшого або коротшого
часу вони й далі беруть участь в утворенні товарів, подаваних в циркуляцію, не вилучаючи однак з
циркуляції елементів свого власного поновлення. Отже, протягом цього часу вони також і собі не
потребують поновлення авансування з боку капіталіста. Нарешті, капітальна вартість, витрачена на
основний капітал, перебігає кругобіг своїх форм протягом часу функціонування тих засобів продукції,
що в них вона існує, — перебігає не речово, а лише своєю вартістю, та й то лише частинами й
поступінно. Тобто, частина вартости основного капіталу циркулює безупинно як частина вартости товару
й перетворюється на гроші, не перетворюючись знову з грошей на свою первісну натуральну форму. Це
зворотне перетворення грошей на натуральну форму засобів продукції відбувається лише наприкінці
періоду їх функціонування, коли засоби продукції цілком зужитковано.

4) Елементи поточного капіталу так само постійно повинні бути зафіксовані в процесі продукції — в
разі він має перебігати безупинно — як і елементи основного капіталу. Але фіксовані таким чином
елементи першого постійно поновлюється in natura (засоби продукції замінюється на нові екземпляри
того самого роду, а робочу силу — через постійно поновлюваний закуп), тимчасом як елементи основного
капіталу протягом їхнього існування ні самі не поновлюються, ані доводиться поновлювати їх купівлю.
В продукційному процесі постійно є сировинні й допоміжні матеріяли, але їх завжди замінюється на
нові екземпляри того самого
\parbreak{}  %% абзац продовжується на наступній сторінці

\parcont{}  %% абзац починається на попередній сторінці
\index{ii}{0117}  %% посилання на сторінку оригінального видання
роду, коли старі цілком зужитковано на вироблення готового продукту. Так само в процесі продукції
завжди є і робоча сила, але тільки в наслідок постійного поновлювання її закупу, при чому часто
змінюються й особи. Навпаки, підчас повторюваних оборотів поточного капіталу в тих самих
повторюваних процесах продукції й далі функціонують ті самі будівлі, машини тощо.

\subsection{Складові частини, заміщення, ремонт, акумуляція основного капіталу}
В тих самих капіталовкладеннях життьова тривалість поодиноких елементів основного капіталу різна, а
тому різний і час їх обороту. Напр., на залізниці, час функціонування й час репродукції рейок,
злежнів, земляних споруд, станційних будинків, мостів, тунелів, льокомотивів і вагонів різний, а
тому й різний час обороту авансованого на них капіталу. Протягом багатьох років будівлі, плятформи,
водоймища, віядуки, тунелі,
земляні виїмки й насипи, — коротше кажучи, все те, що в англійському залізничному господарстві
зветься works of art, не потребує жодного поновлення. Речі, що найбільше зношуються — це рейковий
шлях і рухома частина (rolling stock).

Первісно, коли будували сучасні залізниці, панував той погляд, — його поширювали видатні
інженери-практики, — ніби залізниця своєю тривалістю вічна, а зношування рейок таке непомітне, що
його можна й зовсім не брати на увагу в усіх фінансових і практичних розрахунках; життьову
тривалість добрих рейок тоді обчислювалось в 100--150 років. Але незабаром виявилось, що життьова
тривалість рейки, звичайно залежна від швидкости паротягів, ваги та числа потягів, грубини самих
рейок та багатьох інших бічних обставин, пересічно не перевищує 20 років.
На деяких станціях, центрах великого обороту, рейки зношуються навіть щорічно. Щось близько 1867
року почали вводити сталеві рейки, які коштували майже вдвоє дорожче, ніж залізні, але зате й
тривають більш як удвоє. Життьова тривалість дерев’яних злежнів становила 12--15 років. Щодо рухомої
частини, то товарові вагони зношуються куди швидше, ніж пасажирські. Життьову тривалість паротягів
1867 року обчислювалось в 10--12 років.

Зношування постає, поперше, в наслідок самого уживання. Взагалі рейки зношуються пропорційно до
числа потягів (R. C. № \num{17655})\footnote{
Цитати, позначені R. С., взято з Royal Commission on Railways. Minutes of Evidence taken before
the Commissioners. Presented to both Houses of Parliament. London, 1867. — Запитання й відповіді
перенумеровано й нумери тут зазначено.
}. Коли швидкість збільшувалась, то зношування зростало більш ніж
відповідно до квадратів швидкости, тобто, коли швидкість потягів збільшувалась удвоє, то зношування
зростало більше ніж у чотири рази (R. С. № \num{17046}).

Далі, зношування постає в наслідок впливу природних сил. Приміром, злежні псуються не лише в
наслідок дійсного зношування, але й через гниття. „Витрати на утримання залізниці залежать не
стільки від
\parbreak{}  %% абзац продовжується на наступній сторінці

\input{_0118.tex}
\input{_0119.tex}

\index{ii}{0120}  %% посилання на сторінку оригінального видання
Здебільша це залежить від площі, яка є в розпорядженні. При деяких будівлях можна надбудовувати
горішні поверхи, при інших треба поширювати в боки, тобто треба більше землі. За капіталістичної
продукції, з одного боку, багато засобів витрачається марно, а з другого боку, при поступінному
поширенні підприємства спостерігається багато випадків такого роду недоцільного поширення будівель в
боки (почасти це шкодить робочій силі), бо нічого не робиться за суспільним пляном, а все залежить
від безлічі різних обставин, засобів і~\abbr{т. ін.}, що з ними має діло капіталіст. А з цього постає
велике марнотратство продуктивних сил.

Таке повторне вкладання грошового резервного фонду частинами (тобто частини основного капіталу,
знову перетвореної на гроші) найлегше робиться в хліборобстві. Просторове обмежене поле продукції
тут якнайбільш здібне поступінно вбирати капітал. Так само стоїть справа й там, де відбувається
природна репродукція, як, напр., у скотарстві.

Основний капітал спричиняє особливі витрати на зберігання. Почасти це зберігання здійснюється самим
процесом праці; основний капітал псується, коли він не функціонує в процесі праці (див. кн. І, розд.
VI і розд. XIII.~Зношування машин, що постає від їх невживання). Тому англійський закон вважає
буквально за шкоду (waste), коли орендовані ділянки не обробляється заведеним у країні способом. (W.~A.~Holdsworth, Barrister at Law, „The Law of Landlord and Tenant“, London, 1857, p. 96). Це
зберігання, що походить з ужитку в процесі праці, є безплатний природний дар живої праці. Ця
властива праці сила зберігання має двоїстий характер. З одного боку, праця зберігає вартість
матеріялів праці, переносячи їх на продукт; з другого боку, оскільки вона й не переносить на продукт
вартости засобів праці, вона все ж зберігає їхню вартість, зберігаючи їхню споживну вартість тим, що
вони функціонують у процесі продукції.

Однак, для того, щоб основний капітал зберігався в належному стані, потрібні й безпосередні витрати
праці. Машини треба час від часу чистити. Тут справа йде про новододавану працю, що без неї вони
будуть непридатні до вжитку, про безпосереднє зберігання від шкідливих стихійних впливів, завжди
сполучених з продукційним процесом, отже, про зберігання машин у стані працездатности в прямому
значенні цього слова. Само собою зрозуміло, нормальну життьову тривалість основного капіталу
обчислюється, зважаючи на те, що здійсняться умови, в яких він може нормально існувати протягом
цього часу, так само як припускається, що, коли людина живе пересічно 30 років, вона також і
миється. Отже, тут ходить не про те, щоб замінити працю, яка є в машині, а про постійну новододавану
працю, потрібну в наслідок уживання машини. Тут ідеться не про ту працю, що її виконує машина, а про
ту, що прикладається до машини, тимчасом як машина є не чинник продукції, а сировинний
матеріял. Капітал, витрачений на цю працю, — хоч і не входить власне в той процес праці, що йому
продукт завдячує своїм походженням, — належить до поточного капіталу. Цю працю доводиться постійно
витрачати на продукцію, а тому й вартість цієї праці завжди мусить покриватись
\parbreak{}  %% абзац продовжується на наступній сторінці

\parcont{}  %% абзац починається на попередній сторінці
\index{ii}{0121}  %% посилання на сторінку оригінального видання
вартістю продукту. Витрачений на цю працю капітал належить до тієї частини поточного капіталу, яка
має покрити загальні затрати (Unkosten), і треба її розподілити на новоспродуковану вартість
відповідно до пересічного річного розрахунку. Ми бачили, що у власне промисловості цю працю чищення
робітники виконують безплатно під час павз для відпочинку, і саме через це вони часто виконують її
під час самого процесу продукції; від цього походить більшість нещасних випадків. Цю працю
не оплачується в ціні продукту. Отже, споживач остільки й має її безплатно. З другого боку,
капіталіст ощаджує таким чином витрати на зберігання машин. Робітник платить сам, власного особою, і
це становить одну з тих таємниць самозберігання капіталу, що в дійсності утворюють юридичні права
робітника на машину й перетворюють його навіть з буржуазного погляду на співвласника машини. Однак,
в різних галузях продукції, там, де машини для чищення треба вилучати з продукційного процесу, а
тому й не можна їх чистити між іншим, — як, прим., при чищенні паровозів, — ця праця зберігання
належить до поточних витрат, тобто є елемент поточного капіталу. Після трьох щонайбільше днів праці
паровоза треба подати в депо й там чистити; щоб не зіпсувати казан, промиваючи його, треба спочатку
його охолодити. (R.~С. №\num{17823}).

Власне ремонт або полагодження потребують таких витрат капіталу й праці, що їх немає в первісно
авансованому капіталі, а значить, і не можна їх — в усякому разі не завжди можна — замістити й
покрити поступінним заміщенням вартости основного капіталу. Коли, напр., вартість основного капіталу
дорівнює \num{10.000}\pound{ ф. стерл.}, а його загальна життьова тривалість дорівнює 10 рокам, то ці \num{10.000}\pound{ ф.
стерл.}, по 10 роках цілком перетворившись на гроші, заміщують лише вартість первісно вкладеного
капіталу, але вони не заміщують капіталу, зглядно праці, новодолученого під час ремонту. Це є
додаткова складова частина вартости, що її теж авансується не одним заходом, а залежно від потреби,
коли саме надходять різні моменти її авансування, це з самої природи речей залежить від випадку.
Кожен основний капітал потребує таких пізніших,
часткових, додаткових витрат капіталу на засоби праці й робочу силу.

Ушкодження, що їх зазнають поодинокі частини машин і~\abbr{т. ін.}, випадкові своєю природою, а тому так
само випадкові й зумовлювані цим полагодження. Однак з таких ремонтних робіт відзначаються дві
відміни їх, що мають більш-менш сталий характер і припадають на різні періоди життя основного
капіталу — це недуги дитинства й куди численніші недуги віку, що вийшов поза межі середнього віку
життя. Хоч яка досконала конструкція, прим., машини, яка входить у продукційний процес, на практиці,
при її застосуванні, завжди виявляються хиби, що їх треба виправляти добавочною працею. З другого
боку, що більше вона виходить за середній свій вік, отже, що більше стає її нормальне зношування, а
матеріял, що з нього вона складається, зуживається й старіє, то частішого й більшого ремонту треба,
щоб підтримати функціонування машини до скінчення її пересічного життьового періоду; так само як
старій людині, шоб не вмерти передчасно, доводиться більше
\parbreak{}  %% абзац продовжується на наступній сторінці

\input{_0122.tex}
\input{_0123_0124.tex}
\input{_0125.tex}

\index{iii2}{0126}  %% посилання на сторінку оригінального видання
Після цих попередніх зауважень я хочу коротко подати особливості мого
дослідження в відміну від Рікардо та ін.

\pfbreak

Ми розглянемо спочатку неоднакові наслідки, що їх дають однакові маси
капіталу, застосовані на різних земельних дільницях однакової величини: або,
при земельних дільницях неоднакової величини, наслідки, обчислені щодо однакової
земельної площі.

Дві незалежні від капіталу загальні причини цієї неоднаковости наслідків
є: 1) \emph{Родючість}. (До цього пункту (1) слід вияснити, що взагалі і які
різні моменти розуміються під природною родючістю земель). 2) \emph{Положення}
земельних дільниць. Остання причина є вирішальна для колоній, і взагалі — для
послідовности, в якій можуть іти під обробіток земельні дільниці одна по одній.
Далі ясно, що ці дві різні основи диференційної ренти, родючість і положення,
можуть впливати в протилежному напрямку. Земля може бути добре розташована
і мало родюча, і навпаки.

Ця обставина є важлива, бо вона пояснює нам, чому, обробляючи землі
даної країни, можна поступово переходити від кращої землі до гіршої, так само,
як і навпаки. Нарешті ясно, що прогрес суспільної продукції взагалі, з одного
боку, нівелює вплив положення, як основу диференційної ренти, бо він створює
місцеві ринки і, створюючи засоби сполучення й транспорту, змінює умови
положення; з другого боку, цей проґрес збільшує ріжниці в місцевому положенні
земельних дільниць, відокремлюючи хліборобство від мануфактури і створюючи
великі промислові центри, з одного боку, і відносне відокремлення села, з другого.

Але спочатку ми залишимо цей пункт, положення, не будемо звертати на
нього уваги і розглянемо лише природну родючість. Лишаючи осторонь кліматичні
та інші моменти, ріжниця у природній родючості сходить на ріжницю
хемічного складу верхнього шару ґрунту, тобто на ріжницю в кількості потрібних
для виростання рослин поживних речовин, що містяться в ньому. Проте, коли
припустити дві земельні дільниці з однаковим хемічним складом ґрунту і в
цьому розумінні однакової природної родючости, то дійсна ефективна родючість
буде різна залежно від тієї форми, в якій перебувають ці поживні речовини і в якій
вони більш-менш засвоюються, більш або менш безпосередньо йдуть на живлення
рослин. Отже, почасти від розвитку хліборобської хемії, почасти від розвитку
хліборобської механіки залежить те, якою мірою на земельних дільницях однакової
природної родючости можна дійсно використати цю природну родючість.
Отже, хоч родючість і є об’єктивна властивість ґрунту, проте, економічно вона
постійно має в собі певне відношення, відношення до даного рівня розвитку
хліборобської хемії і механіки, і змінюється разом з цим рівнем розвитку.
Як з допомогою хемічних засобів (наприклад, застосуванням певного текучого
гною на щільнім глинястім ґрунті, абож обпалюванням важкого глинястого
ґрунту), так і з допомогою механічних засобів (наприклад, особливих
плугів для важких ґрунтів) можна усунути перешкоди, що робили такі самі
родючі ґрунти фактично менше родючими (сюди ж належить і дренування ґрунту).
Це може змінити і саму послідовність в обробітку різних родів землі, як це
було, наприклад, щодо легкого піщаного і важкого глинястого ґрунтів за одного
з періодів розвитку англійського хліборобства. Це знов таки показує, яким
чином історично — в послідовному перебізі обробітку — перехід може однаково відбуватися
так від родючіших земель до менш родючих, як і навпаки. Те саме
може статись і в умовах штучно переведених поліпшень у складі ґрунту, або
в умовах простої зміни в методах хліборобства. Нарешті, такий самий результат
може постати з зміни в ієрархії щодо родів ґрунту в наслідок різних умов
підґрунтя, скоро тільки підґрунтя теж починає оброблятися й перетворюється
на зорану землю. Це зумовлюється почасти застосуванням нових хліборобських
\parbreak{}  %% абзац продовжується на наступній сторінці

\input{_0127c.tex}
\parcont{}  %% абзац починається на попередній сторінці
\index{ii}{0128}  %% посилання на сторінку оригінального видання
як засіб циркуляції, а потім знову як скарб відокремлюється від маси
грошей, що циркулюють. З розвитком кредитової системи, — а він неминуче
відбувається рівнобіжно з розвитком великої промисловости й капіталістичної
продукції, — гроші функціонують уже не як скарб, а як капітал,
однак в руках не їхнього власника, а другого капіталіста, що йому
їх передається в розпорядження.

\section{Цілий оборот авансованого капіталу. Цикли
оборотів}

Ми бачили, що основні й поточні складові частини продуктивного
капіталу різним способом і в різні періоди обертаються, і що різні складові
частини основного капіталу в тому самому підприємстві знову таки
мають різні періоди обороту залежно від різного часу їхнього життя, а,
значить, і репродукції. (Про справжні або позірні відмінності в обороті
різних складових частин поточного капіталу в тому самому підприємстві
див. наприкінці цього розділу під цифрою 6).

1. Цілий оборот авансованого капіталу є пересічний оборот його різних
складових частин; спосіб обчислення подається нижче. Оскільки
йдеться лише про різні періоди часу, немає, звичайно, нічого простішого,
як обчислити з них пересічне, але:

2. Тут маємо не лише кількісні, а й якісні відмінності.

Поточний капітал, що входить в процес продукції, переносить на
продукт усю свою вартість, а тому, щоб продукційний процес відбувався
безупинно, він мусить завжди заміщуватись in natura через продаж
продукту. Основний капітал, що входить у процес продукції, переносить
на продукт лише частину своєї вартости (зношування) і, не зважаючи на
зношування, і далі функціонує в продукційному процесі; тому лише через
коротші або довші переміжки, в усякому разі не так часто, як поточний
капітал, треба його заміщувати in natura. Ця потреба в заміщенні, строк
репродукції, не лише кількісно різна для різних складових частин капіталу,
але, як ми бачили вище, частина довготривалішого, багатолітнього
капіталу може бути заміщена і долучена in natura до старого основного
капіталу щорічно або навіть через коротші переміжки часу; щождо основного
капіталу іншої властивости, то його заміщення може статися лише
одним заходом наприкінці його життя.

Ось чому й треба звести особливі обороти різних частин основного
капіталу до однорідної форми обороту, так щоб вони відрізнялись один
від одного лише кількісно, триванням обороту.

Цієї якісної тотожности немає, коли ми візьмемо за вихідний пункт
$П\dots{} П$, — форму безперервного продукційного процесу. Бо певні елементи
П мусять заміщуватись in natura, а інші ні. Але форма $Г\dots{} Г'$ дає безперечно
цю тотожність обороту. Коли ми візьмемо, напр., машину вартістю
\num{10.000}\pound{ ф. стерл.}, яка живе 10 років, то тоді щороку знов перетворюється
\parbreak{}  %% абзац продовжується на наступній сторінці

\input{_0129c.tex}

\index{iii2}{0130}  %% посилання на сторінку оригінального видання
Перше і головне припущення є, що поліпшення в хліборобстві нерівномірно
впливає на землі різних родів, і тут воно більше впливає на кращі землі
С і D, ніж на А і В. Досвід довів, що, звичайно, справа так і стоїть, хоч може
статись і зворотне. Коли б поліпшення більше впливало на гірші землі, ніж на
кращі, то рента з останніх понизилася б замість підвищитись. — Але з абсолютним
зростом родючости всіх родів землі у таблиці одночасно припускається зріст
вищої відносної родючості кращих родів землі С і D, а тому зріст ріжниці в продукті за однакової
величини застосованого капіталу, а тому і зріст диференційної ренти.
Друге припущення є в тому, що з зростанням всього продукту відповідно зростає і загальна потреба в
ньому. \emph{По-перше}, не слід уявляти собі це зростання раптовим; воно відбувається поступово, доти, доки
не встановиться ряд III. \emph{По-друге}, невірно, нібито споживання потрібних засобів існування не зростає разом з їхнім
здешевленням. Скасування хлібних законів в Англії (дивись Newman) довело зворотне, і протилежне
уявлення постало лише тому, що великі і раптові ріжниці в урожаях, які пояснюються тільки погодою,
спричиняють то неспіврозмірне пониження, то неспіврозмірне підвищення цін збіжжя.
Коли в цьому разі раптове і скороминуще здешевлення не встигає справити повного впливу на поширення
споживання, то зворотне явище спостерігається в тому випадку, коли здешевлення випливає із зменшення
самої регуляційної ціни продукції, отже, коли воно має тривалий характер. \emph{По-трете}, частина збіжжя
може бути спожита у вигляді горілки або пива. А зростаюуче споживання обох цих продуктів ніяк не
обмежено вузькими межами. \emph{По-четверте}, тут справа залежить почасти від приросту людности, почасти
від експорту збіжжя в тих країнах, що вивозять збіжжя — як от Англія, до і
пізніше половини XYIII століття, і де тому потребу реґулюється межами не самого
тільки національного споживання. \emph{Нарешті}, збільшення і здешевлення
продукції пшениці може мати своїм наслідком, що замість жита або вівса за
головний засіб харчування маси народу стане пшениця, так що вже в наслідок
самого цього ринок для неї зросте подібно до того, як при зменшенні кількості
продукту і збільшенні його ціни може постати зворотне явище. — При цих припущеннях, отже, і при
взятих числових відношеннях, ряд III дає той наслідок,
що ціна знижується з 60 до ЗО шил. за квартер, отже на 50\%; що продукція проти ряду І зростає з 10
до 23 квартерів, отже, на 130\%; що рента з землі В лишається незмінною, рента з землі С
подвоюється, а з D більше, ніж подвоюється, і що загальна сума ренти підвищується з 18 до 22 ф.
стерл., отже, на 22\sfrac{1}{9}\%.

З порівняння цих трьох таблиць (при чому ряд I треба брати подвійно:
у висхідному напрямку від А до D і в низхідному від D до А), що їх можна
розглядати або як дані ступені хліборобства, за даного стану суспільства, наприклад,

\begin{table}[h]
  \begin{center}
  \footnotesize
    \emph{Таблиця III}

  \begin{tabular}{c c c c c c с с c}
    \toprule
      \multirowcell{2}{\makecell{Рід \\землі}} &
      \multicolumn{2}{c}{Продукт} &
      \multirowcell{2}{\makecell{Витрата \\капіталу}} &
      \multicolumn{2}{c}{Зиск} &
      \multicolumn{2}{c}{Рента} &
      \multirowcell{2}{\makecell{Ціна про-\\дукції \\квартера}}
      \\
    \cmidrule(rl){2-3}
    \cmidrule(l){5-6}
    \cmidrule(l){7-8}
    &
    \makecell{Квар-\\тери} &
    \makecell{Ши-\\лінґи} &
    &
    \makecell{Квар-\\тери} &
    \makecell{Ши-\\лінґи} &
    \makecell{Квар-\\тери} &
    \makecell{Ши-\\лінґи} &
    \\
    \midrule
      А  &  \phantom{0}2  &  \phantom{0}60  & 50 & \phantom{0}\sfrac{1}{3}  & \phantom{0}10  & \phantom{0}0 & \phantom{00}0  &  30\\
      B  &  \phantom{0}4  &  120            & 50 & 2\sfrac{1}{3}            & \phantom{0}70  & \phantom{0}2 & \phantom{0}60  &  15\\
      C  &  \phantom{0}7  &  210            & 50 & 5\sfrac{1}{3}            & 160            & \phantom{0}5 & 150            &  8\sfrac{4}{7} \\
      D  &  10            &  300            & 50 & 8\sfrac{1}{3}            & 250            & \phantom{0}8 & 240            &  6\phantom{0} \\
      \cmidrule(rl){2-2}
      \cmidrule(l){7-8}
      Разом & 23          &                 &    &                          &                & 15           & 450           & \\
  \end{tabular}
  \end{center}
\end{table}

\parbreak{}  %% абзац продовжується на наступній сторінці

\input{_0131.tex}
\input{_0132c.tex}
\input{_0133c.tex}
\input{_0134c.tex}
\parcont{}  %% абзац починається на попередній сторінці
\index{ii}{0135}  %% посилання на сторінку оригінального видання
продуктивного капіталу та їхній вплив на характер обороту. Ба навіть
він одразу наводить, як приклад, купецький капітал у такому питанні,
де йдеться виключно про ріжниці частин продуктивного капіталу
в процесі утворення продукту й вартости — ріжниці, що й собі утворюють
ріжниці в обороті й репродукції капіталу.

Він каже далі: „Капітал, застосовуваний таким способом, не дає своєму
власникові доходу або зиску, поки він лишається в його посіданні або
зберігає ту саму форму“\footnote*{
„The capital employed in this manner yields no revenue or profit to its employer,
while it either remains in his possession or continues in the same shape“.
}. — Капітал, застосовуваний таким способом!
Але ж А.~Сміс каже про капітал, вкладений у сільське господарство або
промисловість, і далі каже нам, що приміщений таким способом капітал
розподіляється на основний та обіговий. Отже, приміщення капіталу цим
способом само собою не може зробити його ні основним, ні обіговим.

Але, може, він хотів сказати, що капітал, застосований для того, щоб
продукувати товари й продавати ці товари з зиском, мусить, по перетворенні
на товари, продаватись і через продаж, поперше, переходити з
власности продавця у власність покупця, а подруге, зміняти свою натуральну
форму товару на грошову форму, і тому капітал не є корисний
для свого власника, поки він лишається в його посіданні або зберігає —
для нього — ту саму форму? Однак тоді справа сходить ось на що: та
сама капітальна вартість, яка раніш функціонувала в формі продуктивного
капіталу, в формі належній до продукційного процесу, функціонує
тепер як товаровий капітал і грошовий капітал, — в формах капіталу, належних
до процесу циркуляції, і тому вона вже не є ні основний, ні поточний
капітал. І це має силу так само для тих елементів вартости, що
долучаються сировинними та допоміжними матеріялами, отже, поточним
капіталом, як і для тих, що долучаються в наслідок зношування засобів
праці, отже, основним капіталом. Таким чином, ми тут ні на крок не
наблизились до висвітлення ріжниці між основним і поточним капіталом.

Далі: „Товари торговця не дають йому жодного доходу або зиску,
поки він не продасть їх за гроші, і гроші так само мало дають йому,
поки він знову не обміняє їх на товари. Його капітал безупинно одходить
від нього в одній формі й повертається до нього в другій і тільки
за допомогою такої циркуляції або послідовних актів обміну може дати
йому будь-який зиск. Тому такі капітали можна назвати у власному значенні
слова обіговими капіталами“\footnote*{
„The goods of the merchant yield him no revenue or profit tilt he sells them
for money, and the money yields him as little till it is again exchanged for goods.
His capital is continually going from him in one shape, and returning to him in
another, and it is only by means of such circulation, or successive exchanges, that
it can yield him any profit. Such capitals, therefore, may very properly be called
circulating capitals“.
}.

Те, що А.~Сміс визначає тут як обіговий капітал, я хочу назвати
капіталом циркуляції (Zirkulationskapital). Це капітал в формі,
належній до процесу циркуляції, капітал, що змінює форму за допомогою
обміну (зміни речовин і зміни власника), отже, товаровий капітал і грошовий
\parbreak{}  %% абзац продовжується на наступній сторінці

\input{_0136_0137c.tex}
\input{_0138.tex}

\index{iii1}{0139}  %% посилання на сторінку оригінального видання
\subsubsection{1861—1864 рр. Американська громадянська війна. Cotton Famine [бавовняний
голод]. Найбільший приклад перерви в процесі виробництва в наслідок
недостачі й дорожнечі сировинного матеріалу}

1860 рік. Квітень. „Щодо стану справ, то я радий можливості
повідомити вас, що, не зважаючи на високу ціну сировинних
матеріалів, всі галузі текстильної промисловості, за винятком
шовку, працювали протягом останнього півроку дуже добре...
В деяких бавовняних округах робітників шукали шляхом оголошень,
і робітники йшли туди з Норфолька та інших землеробських
графств... Як видно, в усіх галузях промисловості панує велика недостача
сировинного матеріалу. Тільки... ця недостача тримає нас
у певних межах. В бавовняній промисловості число новозбудованих
фабрик, розширення наявних фабрик і попит на робітників,
мабуть, ніколи ще не досягали такого високого рівня, як
тепер. Скрізь і всюди шукають сировинного матеріалу“ („Rep.
of Insp. of Fact., April 1860“ [стор. 57]).

1860 рік. Жовтень. „Стан справ у бавовняних, шерстяних
і льонопрядільних округах був добрий; в Ірландії він, як кажуть,
вже більше року навіть „дуже добрий“, і був би ще кращий,
коли б не висока ціна на сировинний матеріал. Прядільники
льону, здається, з більшим нетерпінням, ніж будьколи,
чекають відкриття індійських джерел постачання за допомогою
залізниць і відповідного розвитку індійського землеробства, щоб,
нарешті... добитися відповідного їх потребам подання льону“
(„Rep. of Insp. of Fact., Oct. 1860“, стор. 37).

1861 рік. Квітень. „Стан справ у даний момент пригнічений...
деякі бавовняні фабрики працюють неповний час і багато шовкових
фабрик працюють тільки частково. Сировинний матеріал
дорогий. Майже в усіх галузях текстильної промисловості
ціна його вища, ніж та, при якій він міг би бути перероблений
для маси споживачів“ („Rep. of Insp. of Fact., April 1861“, стор. 33).

Тепер виявилось, що в 1860 році в бавовняній промисловості
була перепродукція; наслідки цього давалися взнаки ще протягом
ближчих років. „Потрібно було від двох до трьох років,
поки світовий ринок поглинув перепродукцію 1860 року“ („Rep.
of Insp. of Fact., October 1863“, стор. 127). „Пригнічений стан
ринків бавовняних фабрикатів у Східній Азії, на початку 1860 року,
справив відповідний зворотний вплив на стан справ у Блекберні,
де пересічно 30 000 механічних ткацьких верстатів майже виключно
заняті у виробництві тканин для цього ринку. В наслідок
цього попит на працю був уже тут обмеженим багато місяців
перед тим, як став відчутним вплив бавовняної блокади...
На щастя, це уберегло багатьох фабрикантів від краху. Запаси,
поки їх тримали на складах, підвищились у своїй вартості, і таким
чином уникнуто було того жахливого знецінення, яке
інакше при такій кризі було б неминучим“ („Rep. of Insp. of
Fact., Oct. 1862“, стор. 28, 29 [30]).


\index{i}{0140}  %% посилання на сторінку оригінального видання
Отож, надзвичайно важливо, щоб підчас процесу, тобто перетворення
бавовни на пряжу, споживано тільки суспільно-доконечний
робочий час. Коли за нормальних, тобто пересічних
суспільних умов продукції, \emph{a} фунтів бавовни за одну робочу
годину мусять бути перетворені на \emph{b} фунтів пряжі, то значення
12-годинного робочого дня має лише такий робочий день, який
$12 × а$ фунтів бавовни перетворює на $12 × b$ фунтів пряжі, бо
при творенні вартости береться до уваги лише суспільно-доконечний
робочий час.

Так сама праця, як і сировинний матеріял і продукт з’являються
тут у зовсім іншому світлі, ніж з погляду власне процесу
праці. Сировинний матеріял має тут значення лише остільки,
оскільки він вбирає в себе певну кількість праці. Через це вбирання
він дійсно перетворюється на пряжу, бо робочу силу витрачено
й додано до нього у формі прядіння. Але продукт, пряжа,
є тепер лише мірило праці, увібраної бавовною. Коли за одну
годину випрядається 1\sfrac{2}{3} фунта бавовни, або її перетворюється на
1\sfrac{2}{3} фунта пряжі, то 10 фунтів пряжі свідчать про шість увібраних
годин праці. Певні й установлені досвідом кількості продукту
репрезентують тепер не що інше, як певні кількості праці, певні
маси застиглого робочого часу. Вони є лише матеріялізація однієї
години, двох годин, одного дня суспільної праці.

Те, що праця є саме праця прядіння, її матеріял — бавовна,
а її продукт — пряжа, тут так само не має значення, як і те, що
самий предмет праці вже є продукт, тобто сировинний матеріял.
Коли б робітник працював не в прядільні, а в копальні, то предмет
праці, вугілля, був би даний природою. Проте певна кількість
вугілля, видобутого з вугляних покладів, приміром, один
центнер, репрезентувала б певну кількість увібраної праці.

При продажу робочої сили припускалось, що її денна вартість
дорівнює 3\shil{ шилінґам}, і що в останніх утілено 6 робочих годин,
отже, що ця кількість праці потрібна на те, щоб випродукувати
пересічну суму засобів існування для робітника на один день.
Коли наш прядун за 1 робочу годину перетворює 1\sfrac{2}{3} фунта
бавовни в 1\sfrac{2}{3} фунта пряжі\footnote{Числа тут цілком довільні}, то за 6 годин він перетворить
10 фунтів бавовни в 10 фунтів пряжі. Отже, протягом процесу
прядіння бавовна вбирає в себе 6 робочих годин. Цей самий робочий
час виражається в кількості золота в 3\shil{ шилінґи.} Отже, самим
прядінням додано до бавовни вартість у 3\shil{ шилінґи.}

Погляньмо тепер на цілу вартість продукту, цих 10 фунтів
пряжі. У них упредметнено 2\sfrac{1}{2}, робочих днів: 2 дні містяться в
бавовні та веретенах, \sfrac{1}{2} дня праці увібрано протягом процесу прядіння.
Цей самий робочий час виражається в масі золота в 15\shil{ шилінґів.}
Отже, ціна цих 10 фунтів пряжі, адекватна їхній вартості,
становить 15\shil{ шилінґів}, ціна 1 фунта пряжі — 1\shil{ шилінґ} 6\pens{ пенсів.}

Наш капіталіст збентежений. Вартість продукту дорівнює
вартості авансованого капіталу. Авансована вартість не зросла,
\parbreak{}  %% абзац продовжується на наступній сторінці

\input{_0141c.tex}

\index{iii1}{0142}  %% посилання на сторінку оригінального видання
За даними того самого звіту, з бавовняних робітників Ланкашіра
й Чешіра тоді працювали повний час 40 146 робітників,
або 11,3\%, неповний робочий час — 134 767 робітників, або 38\%,
зовсім без роботи було 179 721 робітник, або 50,7\%. Коли виключити
звідси дані про Манчестер і Больтон, де випрядаються
головним чином тонкі нумери, — галузь, що порівняно мало потерпіла
від недостачі бавовни, — то справа виявиться ще несприятливішою, 'а
саме: таких, що працюють повний час — 8,5\%,
неповний час — 38\%, безробітних — 53,5\% (стор. 19, 20).

„Для робітників становить істотну ріжницю, чи переробляють
вони добру чи погану бавовну. В перші місяці року, коли фабриканти
намагались тримати свої фабрики в русі тим, що вживали
всяку бавовну, яку тільки можна було купити по помірних цінах,
багато поганої бавовни потрапило на ті фабрики, де раніше звичайно
застосовували добру; ріжниця в заробітній платі робітників
була така велика, що відбулося багато страйків, бо робітники
при старій відштучній платі тепер не могли вже добути
собі зносного щоденного заробітку... В деяких випадках ріжниця
в наслідок застосовування поганої бавовни становила навіть при
повному робочому часі половину всього заробітку“ (стор. 27).

1863 рік. Квітень. „На протязі цього року зможуть бути заняті
повний час трохи більше половини бавовняних робітників“
(„Rep. of Insp. of Fact., April 1863“, стор. 14).

„Дуже серйозна невигода при застосуванні ост-індської бавовни,
яку тепер фабрики мусять споживати, полягає в тому,
що швидкість машин при цьому мусить бути дуже уповільнена.
Протягом останніх років було вжито всіх заходів для збільшення
цієї швидкості, так щоб ті самі машини виконували більше
роботи. Але зменшена швидкість зачіпає робітника в такій самій
мірі, як фабриканта, бо більшість робітників одержують відштучну
плату — прядільники стільки то за фунт випряденої
пряжі, ткачі стільки то за витканий кусок; і навіть у інших
робітників, які одержують тижневу плату, заробітна плата повинна
знизитися в наслідок зменшення виробництва. На підставі
моїх досліджень... і переданих мені даних про заробіток бавовняних
робітників на протязі цього року... виявляється зменшення
заробітної плати пересічно на 20\%, в деяких випадках
на 50\% порівняно з висотою заробітної плати 1861 року“
(стор. 13). — „Зароблена сума залежить... від того, який матеріал
переробляється... Становище робітників, щодо суми заробленої
плати, тепер (жовтень 1863 року) багато краще, ніж минулого
року в цей час. Машини поліпшено, сировинний матеріал
знають краще, і робітники легше справляються з тими труднощами,
з якими їм доводилося боротись спочатку. Минулої весни
я був у Престоні в одній швацькій школі“ [благодійна установа
для безробітних]; „дві молоді дівчини, які за день перед тим
були послані до ткацької фабрики, де, за заявою фабриканта, вони
могли б заробити 4 шилінга на тиждень, просили, щоб їх знову
\parbreak{}  %% абзац продовжується на наступній сторінці

\input{_0143.tex}
\input{_0144.tex}
\parcont{}  %% абзац починається на попередній сторінці
\index{ii}{0145}  %% посилання на сторінку оригінального видання
поточний капітал, а не основний, оскільки І) вартість його цілком входить
у продукт і 2) оскільки його in natura цілком заміщено новим екземпляром
з нового продукту.

А.~Сміс каже нам, з чого складається обіговий і основний капітал.
Він перелічує ті речі, ті речові елементи, що становлять основний капітал,
і ті, що становлять обіговий капітал, ніби таке призначення властиве
цим речам речово, з природи, а не випливає з певних функцій
цих речей в капіталістичному процесі продукції. І однак в тому самому
розділі (Book II, chap. 1) він зауважує, що, хоч певна річ, напр., житлова
будівля, призначена для безпосереднього споживання „може давати
дохід своєму власникові, а значить, служити йому, \so{функціонуючи як
капітал}, однак вона не може ні давати дохід суспільству, ані служити
йому, функціонучи як капітал, отже, вона ані трохи не збільшує доходу
всього суспільства“\footnote*{
\dots{} may yield a revenue to its proprietor, and thereby serve \so{in the function
of a capital} to him, it cannot yield any to the public, nor serve in the function
of a capital to it, and the revenue of the whole body of the people can never be
in the smallest degree increased by it“(p. 186).
}. Тут А.~Сміс цілком виразно висловлює думку, що
властивість бути капіталом речі мають не як такі і не за всяких обставин,
але що це є така функція, яку вони, залежно від обставин, іноді мають,
а іноді не мають. Але те, що має силу для капіталу взагалі, те має
силу й для його підрозділів.

Ті самі речі становлять складову частину поточного або основного
капіталу залежно від того, яку функцію вони виконують в процесі праці.
Напр., худоба, як робоча худоба (засіб праці) становить речову форму
існування основного капіталу; навпаки, як худоба, відгодовувана на
заріз (сировинний матеріял), вона становить складову частину обігового
капіталу фармера. З другого боку, та сама річ може іноді функціонувати
як складова частина продуктивного капіталу, а іноді належати до
фонду безпосереднього споживання. Напр., будинок, функціонучи як місце
праці, є основна складова частина продуктивного капіталу, а функціонуючи
як житлова будівля власника, зовсім не має форми капіталу.
Ті самі засоби праці можуть у багатьох випадках функціонувати то як
засоби продукції, то як засоби споживання.

Це була одна з помилок, що випливають із Смісового уявлення: особливості
основного та обігового капіталу розглядати як особливості,
властиві речам. Аналіза процесу праці („Капітал“, книга 1, розділ V)
вже показала, як змінюються визначення засобу праці, матеріялу праці,
продукту, залежно від різної ролі, що та сама річ відіграє в цьому
процесі. Але визначення основного і не основного капіталу ґрунтуються
й собі на тих певних ролях, що їх ці елементи відіграють у процесі праці,
а, значить, і в процесі утворення вартости.

Але, подруге, при перелічуванні речей, що з них складається основний
і обіговий капітал, виразно виявляється, що А.~Сміс сплутує ріжницю
між основними й поточними складовими частинами капіталу, яка має
силу й рацію лише щодо продуктивного капіталу (капіталу в його
\parbreak{}  %% абзац продовжується на наступній сторінці

\parcont{}  %% абзац починається на попередній сторінці
\index{i}{0146}  %% посилання на сторінку оригінального видання
упредметненої праці, отже, їх не береться до рахуби й не входять
вони у продукт утворення вартости\footnote{
Це одна з обставин, які удорожчують продукцію, основану на
рабстві. Робітник, як влучно висловлювалися за старовини, відрізняється
тут лише як instrumentum vocale\footnote*{
знаряддя, обдароване мовою. \emph{Ред.}
} від тварини як instrumentum semivocale\footnote*{
знаряддя, обдарованого голосом. \emph{Ред.}
} і від мертвого знаряддя праці як від instrumentum mutum\footnote*{
знаряддя німого. \emph{Ред.}
}. Але сам робітник дає відчути тварині і знаряддю праці, що він їм не
рівня, а що він людина. Збиткуючися з них і con amore\footnote*{
з насолодою. \emph{Ред.}
} руйнуючи їх, він з самозадоволенням переконує себе самого в своїй відмінності
від них. Тому за цього способу продукції вважається за економічний принцип
вживати лише найгрубіших, найтяжчих знарядь праці, бо саме через
цю грубість і незграбність їх важко знівечити. Тому в рабовласницьких
державах, які лежать над Мехіканською затокою, до вибуху громадянської
війни, вживали плугів старокитайської конструкції, що рили землю,
як свиня або кріт, але не робили борозни, не повертали її. Порівн.
J.~С.~Cairns: «The Slave Power», London 1862, p. 46 і далі. У своїй праці
«Sea Bord Slave States» (p. 46, 47) Олмстед оповідає, між іншим, таке:
«Мені тут показували знаряддя, що їх у нас жодна людина із здоровим
розумом ніколи не дала б найманому робітникові, бо вони обтяжали б
його; на мою думку, їхня надзвичайна вага й незграбність збільшують
працю щонайменше на 10\% порівняно з тим знаряддям, що його звичайно
вживають у нас. І я певен, що за недбалого й грубого поводження рабів
із знаряддям праці було б неекономно дати їм легше й не таке грубе знаряддя.
А ті знаряддя, що їх ми з користю завжди даємо нашим робітникам,
не збереглися б жодного дня на хлібних полях Вірґінії, хоч ґрунт там
і легший і не такий кам’янистий, як наш. Так само, коли я спитав, чому
на всіх фармах замість коней вживають мулів, то перший арґумент, звичайно,
найдовідніший, був той, що коні не могли б витримати поводження
з боку негрів; у наслідок такого поводження коні завжди швидко нівечаться
або калічіють, тоді як мули витримують биття і брак харчів, не
зазнаючи від цього жодної матеріяльної шкоди, не перестуджуються й не
хоріють, навіть коли нехтувати ними й обтяжати їх працею. Алеж мені
досить підійти до вікна кімнати, де я пишу, щоб побачити завжди таке
поводження з худобою, за яке північний фармер негайно прогнав би погонича».
(«І am here shown tools that no man in his senses, with us, would
allow a labourer for whom he was paying wages, to be encumbered with:
and the excessive weight and clumsiness of which, I would judge, would
make work at least ten per cent greater than with those ordinarily used
with us. And I am assured that, in the careless and clumsy way they must
be used by the slaves, anything ligther or less rude could not be furnished
them with good economy, and that such tools as we constantly give our
bourers, and find our profit in giving them, would not last out a day in
Virginia cornfield — much lighter and more free from stones though it be
than ours. So, too, when I ask why mules are so universally substituted for
horses on the farm, the first reason given, and confessedly the most conclusive
one, is that horses cannot bear the treatment that they always must get
from negroes; horses are always soon foundered or crippled by them while
mules will bear cudgelling, and lose a meal or two now and then, and not
be materially injured, and they do not take cold or get sick, if neglected
or overworked. But I do not need to go further than to the window of the room
in which I am writing, to see at almost any time, treatment of cattle
that would insure the immediate discharge of the driver by almost any farmer
owning them in the North»).
}.

Ми бачимо, що встановлена вже раніш аналізою товару ріжниця
між працею, оскільки вона утворює споживну вартість, і
\parbreak{}  %% абзац продовжується на наступній сторінці

\input{_0147c.tex}
\parcont{}  %% абзац починається на попередній сторінці
\index{iii1}{0148}  %% посилання на сторінку оригінального видання
сприятливого вибору ринку, може бути дуже різна залежно від
більшої чи меншої дешевини сировинного матеріалу, закупівлі
його з більшим чи меншим знанням справи; залежно від того,
наскільки застосовувані машини є продуктивні, доцільні й дешеві;
залежно від більшої чи меншої досконалості загальної
організації різних ступенів процесу виробництва, від того, наскільки
усунено марнування матеріалу, наскільки просто й доцільно
організовано управління й нагляд і т. п. Коротко кажучи,
якщо додаткова вартість для певного змінного капіталу є дана,
то та сама додаткова вартість може виражатися в більшій чи
меншій нормі зиску, отже, може давати більшу чи меншу масу
зиску залежно від особистої ділової спритності самого капіталіста
або його наглядачів і прикажчиків. Припустім, що та сама додаткова
вартість в 1000 фунтів стерлінгів, продукт 1000 фунтів
стерлінгів заробітної плати, в підприємстві $A$ припадає на
9000 фунтів стерлінгів, а в іншому підприємстві $В$ — на 11000
фунтів стерлінгів сталого капіталу. У випадку $А$ ми маємо
$р' = \frac{1000}{10000} = 10\%$. У випадку $В$ ми маємо $р' = \frac{1000}{12000} = 8\sfrac{1}{3}\%$.
Весь капітал виробляє в $А$ порівняно більше зиску, ніж у $В$, бо
там норма зиску вища, ніж тут, хоч в обох випадках авансований
змінний капітал = 1000 і здобута з нього додаткова
вартість також = 1000, отже в обох випадках має місце однакова
експлуатація однакового числа робітників. Ця ріжниця
виразу однієї і тієї ж маси додаткової вартості, або ріжниця
норм зиску, а тому й самих зисків, при однаковій експлуатації
праці, може походити і з інших джерел; але вона може також
походити цілком і виключно з ріжниці в діловій вправності, з
якою ведуться обидва підприємства. І ця обставина приводить
капіталіста до ілюзії — переконує його, — що його зиск завдячує
своє існування не експлуатації праці, а, принаймні почасти і
іншим, від цієї експлуатації праці незалежним, обставинам, особливо
його індивідуальній діяльності.

\pfbreak

З викладеного в цьому першому відділі видно помилковість
того погляду (Родбертуса), згідно з яким (відмінно від
земельної ренти, де, наприклад, площа землі лишається незмінною,
в той час як рента зростає) зміна величини капіталу не впливає
на відношення між зиском і капіталом, а тому й на норму
зиску, бо тоді, коли зростає маса зиску, зростає і маса капіталу,
на яку цей зиск обчислюється, і навпаки.

Це правильно тільки в двох випадках. Поперше, тоді, коли
при незмінності всіх інших умов, отже, особливо норми додаткової
вартості, настає зміна вартості товару, який є грошовим
товаром. (Те саме має місце при самій тільки номінальній
зміні вартості, підвищенні чи падінні знаків вартості при інших
однакових умовах). Припустім, що весь капітал = 100 фунтам
стерлінгів, зиск = 20 фунтам стерлінгів, отже, норма зиску = 20\%.
\parbreak{}  %% абзац продовжується на наступній сторінці


\index{iii2}{0149}  %% посилання на сторінку оригінального видання
Цей випадок не припускає далі жодного продуктивнішого застосування
капіталу, а лише застосування більшого капіталу до тієї самої площі і з тими
самими наслідками, як і до того часу.

Усі відносні величини тут лишаються ті самі. Звичайно, коли розглядати
не відносні ріжниці, а суто аритметичні, то диференційна рента з різних земельможе
змінитися. Припустімо, наприклад, що додатковий капітал вкладено лише
в В і Б. Тоді ріжниця між D і А = 7 квартерам, давніш вона = 3; ріжниця
між В і А = 3 кварт., давніш вона = 1; ріжниця між С і В = — 1, давніш
вона = + 1 і т. ін. Але ця аритметична ріжниця, вирішальна щодо диференційної
ренти І, оскільки в ній виражається ріжниця в продуктивності за однакового
розміру вкладеного капіталу, тут цілком не має ваги, бо вона є лише
наслідок того, чи вкладено, чи ні різні додаткові капітали, за незмінної ріжниці
для кожної рівної частини капіталу на ріжних дільницях.

IIІ. Додаткові капітали дають надмірний продукт і створюють тому надзиски,
але при понижуваній нормі, не пропорційно їхньому збільшенню.

\begin{table}[h]
  \begin{center}
    \emph{Таблиця ІII}
    \footnotesize

  \begin{tabular}{c@{ } c@{ } c@{ } c@{ } c@{ } c@{ } c@{ } c@{ } c@{ } c@{ } c}
    \toprule
      \multirowcell{2}{\makecell{Рід \\землі}} &
      \multirowcell{2}{\rotatebox[origin=c]{90}{Акри}} &
      Капітал &
      \rotatebox[origin=c]{90}{Зиск} &
      \rotatebox[origin=c]{90}{\makecell{Ціна про- \\ дукції}} &
      \multirowcell{2}{\makecell{Продукт в\\ квартерах}} &
      \rotatebox[origin=c]{90}{\makecell{Продажна \\ ціна}} &
      \rotatebox[origin=c]{90}{Здобуток} &
      \multicolumn{2}{c}{Рента} &
      \multirowcell{2}{\makecell{Норма \\надзиску}} \\

      \cmidrule(r){3-3}
      \cmidrule(r){4-4}
      \cmidrule(r){5-5}
      \cmidrule(r){7-7}
      \cmidrule(r){8-8}
      \cmidrule(r){9-10}

       &  &  ф. ст. & ф. ст. & ф. ст. & & ф. ст. & ф. ст. & Кварт. & ф. ст. &  \\
      \midrule

      A & 1 & \phantom{2\sfrac{1}{2} + 2\sfrac{1}{2} =} 2\sfrac{1}{2} & \phantom{0}\sfrac{1}{2} & \phantom{0}3 & \phantom{2 + 1\sfrac{1}{2} =} 1\phantom{\sfrac{1}{2}} & 3 & \phantom{0}3\phantom{\sfrac{1}{2}} &\phantom{0} 0\phantom{\sfrac{1}{2}} & \phantom{0}0\phantom{\sfrac{1}{2}} & \phantom{00}0\phantom{\%} \\
      B & 1 & 2\sfrac{1}{2} + 2\sfrac{1}{2} = 5\phantom{\sfrac{1}{2}} & 1\phantom{\sfrac{1}{2}} & \phantom{0}6 & 2 + 1\sfrac{1}{2} = 3\sfrac{1}{2}           & 3           & 10\sfrac{1}{2}                     & \phantom{0}1\sfrac{1}{2}           & \phantom{0}4\sfrac{1}{2}           & 90\% \\
      C & 1 & 2\sfrac{1}{2} + 2\sfrac{1}{2} = 5\phantom{\sfrac{1}{2}} & 1\phantom{\sfrac{1}{2}} & \phantom{0}6 & 3 + 2\phantom{\sfrac{1}{2}} = 5\phantom{\sfrac{1}{2}} & 3 & 15\phantom{\sfrac{1}{2}}           & \phantom{0}3\phantom{\sfrac{1}{2}} & \phantom{0}9\phantom{\sfrac{1}{2}} & 180\%\\
      D & 1 & 2\sfrac{1}{2} + 2\sfrac{1}{2} = 5\phantom{\sfrac{1}{2}} & 1\phantom{\sfrac{1}{2}} & \phantom{0}6 & 4 + 3\sfrac{1}{2} = 7\sfrac{1}{2}           & 3           & 22\sfrac{1}{2}                     & \phantom{0}5\sfrac{1}{2}           & 16\sfrac{1}{2}                     & 330\%\\
     \cmidrule(r){1-1}
     \cmidrule(r){3-3}
     \cmidrule(r){4-4}
     \cmidrule(r){5-5}
     \cmidrule(r){6-6}
     \cmidrule(r){8-8}
     \cmidrule(r){9-9}
     \cmidrule(r){10-10}

     Разом &  & \phantom{2\sfrac{1}{2} + 2\sfrac{1}{2} =} 17\sfrac{1}{2} & 3\sfrac{1}{2} & 21 & \phantom{2 + 1\sfrac{1}{2} =}17\phantom{\sfrac{1}{2}} & & 51\phantom{\sfrac{1}{2}}  & 10 & 30\phantom{\sfrac{1}{2}} &\\
  \end{tabular}

  \end{center}
\end{table}

При цьому третьому припущені знов таки байдуже, чи повторні додаткові
капітали вкладаються рівномірно або нерівномірно на землі різних родів або ні
в однакових чи неоднакових відношеннях відбувається зменшення продукції
надзиску; чи всі додаткові капітали вкладаються в той самий сорт землі, щодає
ренту, чи розподіляються вони рівномірно або нерівномірно, між землями
різної якости, що дають ренту. Всі ці обставини байдужі для закону, що його тут
розвиваємо. Єдине наше припущення є в тому, що додатковий капітал,
вкладений в будь-який сорт землі, що дає ренту, дає надзиск, але в зменшуваній
пропорції проти розміру збільшення капіталу. Межі цього зменшення
коливаються в прикладах вищенаведеної таблиці, між 4 квартерами = 12 ф. стерл.,
продуктом першого капіталовкладення на найкращій землі В і 1 квартером
= 3 ф. стерл., продуктом такого самого вкладення капіталу на найгіршій
землі А. Продукт з найкращої землі при витраті капіталу і становить максимальну
межу, а продукт з найгіршої землі А, що не дає ні ренти, ні надзиску,
становить, за однакового вкладення капіталу, мінімальну межу продукту,
який дають послідовні вкладення капіталу па будь-якого роду землях, що дають надзиск за зменшуваної
продуктивности послідовних вкладень капіталу. Як
припущення ІІ відповідає тому, що нові однакові якістю дільниці землі кращих
родів приєднується до оброблюваної площі, так що кількість якогось роду обробленої
землі збільшується, так припущення ІІІ відповідає тому, що оброблюються
\parbreak{}  %% абзац продовжується на наступній сторінці

\parcont{}  %% абзац починається на попередній сторінці
\index{ii}{0150}  %% посилання на сторінку оригінального видання
щороку, або в інші, більш-менш короткі періоди реґулярно вилучається
з нього й приміщується або в основний капітал, або в фонд, призначений
для безпосереднього споживання. Кожний основний капітал первісно
постав з обігового й йому потрібна повсякчасна підтримка від цього
останнього. Всі корисні машини та знаряддя праці первісно постали з
обігового капіталу, який дав матеріяли, що з них їх зроблено, і утримання
робітникам, що їх зробили. Вони потребують також, щоб капітал,
зазначеного виду, підтримував їх завжди справними“\footnote*{
Of these four parts three — provisions, materials, and finished work, are either
annually or in a longer or shorter period, regularly withdrawn from it, and placed
either in the fixed capital, or in the stock reserved for immediate consumption.
Every fixed capital is both originally derived from, and requires to be continually
supported by a circulating capital. All useful machines and instruments of trade are
originally derived from a circulating capital, which furnishes the materials of which
they are made and the maintenance of the workmen who make them. They require,
too, a capital of the same kind to keep them in constant repair“ (p. 188).
}.

З винятком частини продукту, що її продуценти завжди безпосередньо
знову зуживають як засоби продукції, для капіталістичної продукції
має силу таке загальне правило: всі продукти подається як товари на ринок,
вони циркулюють для капіталіста як товарова форма його капіталу, як товаровий
капітал незалежно від того, чи мусять і чи можуть ці продукти
своєю натуральною формою, своєю споживною вартістю, функціонувати як
елементи продуктивного капіталу (продукційного процесу), тобто як засоби
продукції, а тому і як основні або поточні елементи продуктивного капіталу,
або чи можуть вони служити лише як засоби особистого, а не продуктивного
споживання. Всі продукти як товари подається на ринок;
тому всі засоби продукції та споживання, всі елементи продуктивного та
особистого споживання треба знову вилучити з ринку купівлею. Ця тривіяльність
(truism), звичайно, правильна. Отже, це однаково має силу й
щодо основних і щодо поточних елементів продуктивного капіталу, і
для засобів праці і для матеріялів праці в усіх їхніх формах. (При цьому
ще забувають, що елементи продуктивного капіталу дані самою природою,
отже, вони не є продукти). Машину купується на ринку так само,
як і бавовну. Але відси ні в якому разі не випливає, що кожний основний
капітал первісно походить із поточного капіталу — це випливає лише
з Смісового сплутування капіталу циркуляції з обіговим або поточним
капіталом, тобто неосновним капіталом. І до того ж Сміс сам себе збиває.
Машини як товар, за його словами, становлять частину зазначеного
в пункті 4 обігового капіталу. Їхнє походження з обігового капіталу значить,
отже, лише те, що вони функціонували як товаровий капітал раніш,
ніж почали функціонувати як машини, але — що речово вони походять
з самих себе; цілком так само, як бавовна, як поточний елемент капіталу
прядуна, походить з бавовни, що циркулювала на ринку. Але коли
Сміс в дальшому викладі висновує основний капітал з обігового на
тій підставі, що для машинобудівництва потрібні праця й сировинні матеріяли,
то, поперше, для цього потрібні також засоби праці, тобто основний
капітал і, подруге, щоб виготувати сировинні матеріяли, теж
\parbreak{}  %% абзац продовжується на наступній сторінці

\parcont{}  %% абзац починається на попередній сторінці
\index{ii}{0151}  %% посилання на сторінку оригінального видання
потрібен основний капітал, машини тощо, бо продуктивний капітал завжди
включає засоби праці, але він не завжди включає матеріял праці. Сам
Сміс каже безпосередньо по цьому: „Землі, копальні та рибні промисли
потребують так основного, як і обігового капіталу, щоб їх розробляти“
(отже, він згоджується з тим, що не лише поточний, а й основний капітал
потрібен для продукції сировинного матеріялу, „і“ (тут нова помилка)
„їхній продукт покриває з зиском не лише ці капітали, але й\so{ усі
інші, що є в суспільстві}“\footnote*{
„Lands, mines, and fisheries, require all both a fixed and circulating capital
to cultivate them; and their produce replaces with a profit, not only those capitals,
but \so{all the others in society}“ (p. 188).
}. Це зовсім неправильно. Їхній продукт
дає сировинні матеріяли, допоміжні матеріяли тощо для всіх інших галузей
промисловости. Але їхня вартість не покриває вартости всіх інших
суспільних капіталів; вона покриває лише свою власну капітальну вартість
(\dplus{} додаткова вартість). В цьому в А.~Сміса знову виявляється
вплив фізіократів.

З суспільного погляду правильно, що частина товарового капіталу,
яка складається з продуктів, що можуть служити лише засобами праці,
раніше або пізніше — якщо тільки не спродуковано їх взагалі марно,
якщо вони не лишаються непродані — функціонуватимуть як засоби праці;
інакше кажучи, на основі капіталістичної продукції такі продукти,
переставши бути товарами, справді мусять стати згідно з своїм призначенням
елементами основної частини суспільного продуктивного капіталу.

Тут перед нами ріжниця, що випливає з натуральної форми продукту.

Напр., прядільна машина немає споживної вартости, коли її не вживається
на прядіння, отже, коли вона не функціонує як елемент продукції,
отже, з капіталістичного погляду як основна частина продуктивного
капіталу. Але прядільна машина рухома. Її можна вивезти з країни, де
її випродукувано, і продати в іншій країні в обмін, безпосередньо або
посередньо, на сировинний матеріял тощо або на шампанське. В країні,
де її випродукувано, вона функціонує тоді лише як товаровий капітал,
але зовсім не функціонує — навіть після її продажу — як основний капітал.

Навпаки, продукти, що через прикріплення їх до ґрунту є льокалізовані,
і які, отже, можна використовувати лише на місці, напр., фабричні
будівлі, залізниці, мости, тунелі, доки й~\abbr{т. ін.}, меліорації тощо —
всі такі продукти не можна вивезти матеріяльно, так, як вони є. Вони
нерухомі. Або їх марно спродуковано, або, якщо їх продано, вони мусять
функціонувати як основний капітал, — у тій країні, де їх випродукувано.
Для капіталістичного продуцента, що за для спекуляції, маючи
на меті продаж, будує фабрики або поліпшує ґрунт, ці речі мають форму
його товарового капіталу, отже, за А.~Смісом, форму обігового капіталу.
Але, з суспільного погляду, ці речі, щоб не лишитись некорисними,
мусять, кінець-кінцем, функціонувати у власній країні як основний
капітал у процесі продукції, фіксованому в місці їхнього перебування. Відси
ні в якому разі не випливає, що нерухомі речі, як такі, вже самі
собою є основний капітал; вони, як, напр., житлові будинки тощо,
\parbreak{}  %% абзац продовжується на наступній сторінці

\parcont{}  %% абзац починається на попередній сторінці
\index{ii}{0152}  %% посилання на сторінку оригінального видання
можуть належати до фонду споживання, отже, взагалі не належать до
суспільного капіталу, хоч і становлять елемент суспільного багатства, що
з нього капітал є лише частина. Продуцент цих речей, кажучи словами
Сміса, одержує зиск, продаючи їх. Отже, обіговий капітал! Людина, що
користується з них, їхній остаточний покупець, може використати їх, лише
вживаючи їх в процесі продукції. Отже, основний капітал!

Титули власности, напр., на залізницю, можуть щодня переходити з
рук в руки, і власники їх, продаючи ці титули, можуть одержувати зиск
навіть за кордоном; отже, титули власности на залізницю, протилежно
самій залізниці можна вивозити. А проте, сами ці речі мусять саме в тій
країні, де вони льокалізовані, або лежати без діла, або функціонувати як
основна складова частина продуктивного капіталу. Так само фабрикант
$А$ може одержати зиск, продавши свою фабрику фабрикантові $В$, що однак
не перешкоджає фабриці й тепер, як раніше, функціонувати як основний
капітал.

Отже, якщо фіксовані в певному місці, невідокремлювані від ґрунту
засоби праці доконечно мусять — згідно з їхнім призначенням — функціонувати
як основний капітал в самій країні, хоча б для їхнього продуцента
вони функціонузали як товаровий капітал, не являючи елементів його
основного капіталу (останній складається для нього з засобів праці, що
вони потрібні на будування будівель, залізниць тощо), то відси ні в
якому разі не випливає зворотний висновок, що основний капітал мусить
складатись з нерухомих речей. Корабель або льокомотив працюють
лише рухаючись; і все ж вони функціонують — не для їхнього продуцента,
а для їхнього споживача — як основний капітал. З другого боку,
речі, що якнайочевидніше фіксовані в продукційному процесі, у ньому
живуть та вмирають і, одного разу ввійшовши в нього, вже ніколи його
не облишають, є поточні складові частини продуктивного капіталу. Напр.,
вугілля, зуживане машиною в процесі продукції, газ, що ним освітлюється
фабричну будову тощо. Вони поточні не тому, що вони разом з
продуктом матеріяльно облишають процес продукції і циркулюють як
товар, а тому, що їхня вартість цілком ввіходить у вартість товару, що
його продукувати вони допомагають, і, значить, її цілком треба покрити
через продаж товару.

В щойно цитованому місці з А.~Сміса треба зазначити ще таке речення:
„Обіговий капітал, що дає\dots{} утримання робітникам, які виробляють
їх“ (машини і т.~інше).

У фізіократів частина капіталу, авансована на заробітну плату, правильно
фігурує під назвою avances annuelles протилежно до avances primitives.
З другого боку, в них виступає як складова частина продуктивного
капіталу, вживаного фармером, не сама робоча сила, а засоби існування,
що їх видається сільсько-господарським робітникам („утримання
робітників“, як каже Сміс). Це точно відповідає їхній специфічній доктрині.
А саме — у них частину вартости, долучувану працею до продукту
(цілком так само, як і ту частину вартости, що її долучають до продукту
сировинні матеріяли, знаряддя праці та інші речові складові частини
\parbreak{}  %% абзац продовжується на наступній сторінці

\input{_0153c.tex}
\input{_0154c.tex}

\index{iii2}{0155}  %% посилання на сторінку оригінального видання
На землі D збіжжева рента проти таблиці I зросла з 3 \footnote*{
В німецькому тексті стоїть: «з 2 квартерів». Явна помилка, як це можна бачити з таблиці І \emph{Прим. Ред.}
} квартерів до 6
тимчасом як грошова рента лишилася, як і давніш, 9 ф. стерл.. Проти таблиці II
збіжжева рента з D лишилася колишня, 6 квартерів, але грошова рента знизилась
з 18 ф. стер, до 9 ф. стерл.

Коли розглядати загальні суми ренти, то збіжева рента таблиці IVb = 8
квартерам, більша, ніж рента в таблиці І, що дорівнює 6 квартерам, і більша,
ніж рента в таблиці IVа, що дорівнює 7 квартерам; і навпаки, вона менша, ніж
рента в таблиці II = 12 кварт. Грошова рента в таблиці IVb = 12 ф. стерл.,
більша, ніж грошова рента в таблиці ІVа = 10\sfrac{1}{2} ф. стерл., і менша від грошової
ренти таблиці І = 18 ф. стерл. і таблиці II = 36 ф. стерл.

Щоб по відпаданні ренти з В в умовах таблиці IVb загальна сума ренти
дорівнювала такій у таблиці I, ми мусимо одержати ще на 6 ф. стерл.
надпродукту, тобто 4 квартери по 1\sfrac{1}{2} ф. стерл., що є новою ціною продукції.
Тоді ми знову маємо загальну суму ренти в 18 ф. стерл., як у таблиці І. Величина
потрібного на це додаткового капіталу буде різна залежно від того, чи
вкладемо ми його в С або D, чи розподілимо його між обома родами землі.

На С капітал в 5 ф. стерл. дає 2 квартери надпродукту, отже, 10 ф. ст.
додаткового капіталу дадуть 4 квартери додаткового надпродукту. На D було б
досить додаткової витрати в 5 ф. стерл., щоб випродукувати 4 квартери додаткової
збіжжевої ренти при зробленому тут основному припущенні, що продуктивність
додаткових капіталовкладень лишається та сама. Тому здобуваємо
такі наслідки.

\begin{table}[h]
  \begin{center}
    \emph{Таблиця ІVc}
    \footnotesize

  \begin{tabular}{c c c c c c c c c c c}
    \toprule
      \multirowcell{2}{\makecell{Рід \\землі}} &
      \multirowcell{2}{\rotatebox[origin=c]{90}{Акри}} &
      \rotatebox[origin=c]{90}{Капітал} &
      \rotatebox[origin=c]{90}{Зиск} &
      \rotatebox[origin=c]{90}{\makecell{Ціна про- \\ дукції}} &
      \multirowcell{2}{\rotatebox[origin=c]{90}{\makecell{Продукт \\ в кварт.}}} &
      \rotatebox[origin=c]{90}{\makecell{Продажна \\ ціна}} &
      \rotatebox[origin=c]{90}{Здобуток} &
      \multicolumn{2}{c}{Рента} &
      \multirowcell{2}{\makecell{Норма \\надзиску}} \\

      \cmidrule(rl){3-3}
      \cmidrule(rl){4-4}
      \cmidrule(rl){5-5}
      \cmidrule(rl){7-7}
      \cmidrule(rl){8-8}
      \cmidrule(rl){9-10}

       &  &  ф. ст. & ф. ст. & ф. ст. & & ф. ст. & ф. ст. & Кварт. & ф. ст. &  \\
      \midrule

      B & 1 &  \phantom{0}5\phantom{\sfrac{1}{2}} & 1\phantom{\sfrac{1}{2}} & \phantom{0}6 & \phantom{0}4 & 1\sfrac{1}{2} & \phantom{0}6 & 0 & \phantom{0}0 & \phantom{00}0\% \\
      C & 1 & 15\phantom{\sfrac{1}{2}}            & 3\phantom{\sfrac{1}{2}} & 18           & 18           & 1\sfrac{1}{2} & 27           & 6 & \phantom{0}9 & \phantom{0}60\%\\
      D & 1 &  \phantom{0}7\sfrac{1}{2}           & 1\sfrac{1}{2}           & \phantom{0}9 & 12           & 1\sfrac{1}{2} & 18           & 6 & \phantom{0}9 & 120\%\\
     \cmidrule(rl){1-1}
     \cmidrule(rl){2-2}
     \cmidrule(rl){3-3}
     \cmidrule(rl){4-4}
     \cmidrule(rl){5-5}
     \cmidrule(rl){6-6}
     \cmidrule(rl){8-8}
     \cmidrule(rl){9-9}
     \cmidrule(rl){10-10}

     Разом & 3 & 27\sfrac{1}{2} & 5\sfrac{1}{2} & 33 & 34 & & 51 & 12 & 18 &\\
  \end{tabular}

  \end{center}
\end{table}

\begin{table}[h]
  \begin{center}
    \emph{Таблиця ІVd}
    \footnotesize

  \begin{tabular}{c c c c c c c c c c c}
    \toprule
      \multirowcell{2}{\makecell{Рід \\землі}} &
      \multirowcell{2}{\rotatebox[origin=c]{90}{Акри}} &
      \rotatebox[origin=c]{90}{Капітал} &
      \rotatebox[origin=c]{90}{Зиск} &
      \rotatebox[origin=c]{90}{\makecell{Ціна про- \\ дукції}} &
      \multirowcell{2}{\rotatebox[origin=c]{90}{\makecell{Продукт \\ в кварт.}}} &
      \rotatebox[origin=c]{90}{\makecell{Продажна \\ ціна}} &
      \rotatebox[origin=c]{90}{Здобуток} &
      \multicolumn{2}{c}{Рента} &
      \multirowcell{2}{\makecell{Норма \\надзиску}} \\

      \cmidrule(rl){3-3}
      \cmidrule(rl){4-4}
      \cmidrule(rl){5-5}
      \cmidrule(rl){7-7}
      \cmidrule(rl){8-8}
      \cmidrule(rl){9-10}

       &  &  ф. ст. & ф. ст. & ф. ст. & & ф. ст. & ф. ст. & Кварт. & ф. ст. &  \\
      \midrule

      B & 1 & \phantom{0}5\phantom{\sfrac{1}{2}} & 1\phantom{\sfrac{1}{2}} & \phantom{0}6 & \phantom{0}4 & 1\sfrac{1}{2} & \phantom{0}6 & \phantom{0}0 & \phantom{0}0 & \phantom{00}0\% \\
      C & 1 & \phantom{0}5\phantom{\sfrac{1}{2}} & 1\phantom{\sfrac{1}{2}} & \phantom{0}6 & \phantom{0}6 & 1\sfrac{1}{2} & \phantom{0}9 & \phantom{0}2 & \phantom{0}3 & \phantom{0}60\%\\
      D & 1 & 12\sfrac{1}{2}                     & 2\sfrac{1}{2}           & 15           & 20           & 1\sfrac{1}{2} & 30           & 10           & 15           & 120\%\\
     \cmidrule(rl){1-1}
     \cmidrule(rl){2-2}
     \cmidrule(rl){3-3}
     \cmidrule(rl){4-4}
     \cmidrule(rl){5-5}
     \cmidrule(rl){6-6}
     \cmidrule(rl){8-8}
     \cmidrule(rl){9-9}
     \cmidrule(rl){10-10}

     Разом & 3 & 22\sfrac{1}{2} & 4\sfrac{1}{2} & 27 & 30 & & 45 & 12 & 18 &\\
  \end{tabular}

  \end{center}
\end{table}



\index{iii2}{0156}  %% посилання на сторінку оригінального видання
Загальна сума грошової ренти становила б якраз половину того, що було
в таблиці II, де додаткові капітали були вкладені за незмінних цін продукції.

Найважливіше є порівняти вищенаведені таблиці з таблицею І.

Ми бачимо, що з пониженням ціни продукції на половину, з 60 шил. до
30 шил. за квартер, загальна сума грошової ренти залишилась та сама = 18 ф.
ст. і відповідно до цього збіжжева рента подвоїлась, саме зросла з 6 кварт. до
12 кварт. Рента з В відпала; з С грошова рента в ІVс збільшилась на половину,
але на половину зменшилась в ІVс; з D вона лишилась та сама = 9 ф.
стерл. у таблиці ІVс, і піднеслась з 9 ф. стерл. до 15 ф. стерл. у таблиції ІVd.
Продукц я піднеслась з 10 квартерів до 34 в ІVс, і до 30 квартер в в IVd;
зиск підвищився з 2 ф. стерл. до 5\sfrac{1}{2} в ІVс і до 4 \sfrac{1}{2} в IVd. Загальна сума
вкладеного капіталу зросла в одному випадку з 10 ф. стерл. до 27\sfrac{1}{2} ф. стерл.,
в другому — з 10 до 22\sfrac{1}{2} ф. стерл.; отже, обидва рази більше, ніж удвоє. Норма
ренти, рента, обчислена у відношенні до авансованого капіталу, в усіх таблицях
від IV до IVd для кожного роду землі всюди та сама, що вже було дано тим припущенням,
що норма продуктивности обох послідовних витрат капіталу на землях
усіх родів не змінюється. Проти таблиці І вона, проте, понизилась пересічно
щодо всіх родів землі і для кожного окремого роду землі. В таблиці І вона =
180\% пересічно, в таблиці ІVс вона$ = \frac{18}{27\sfrac{1}{2}} × 100 = 65\sfrac{5}{11}\%$ і
IVd = $\frac{18}{22\sfrac{1}{2}} × 100 = 80\%$. Пересічна грошова рента з акра підвищилась. Її пересічна
величина давніш в таблиці І була 4\sfrac{1}{2} ф. стерл. з акра для всіх 4 акрів,
а тепер у таблицях IVс і d вона дорівнює 6 ф. стерл. з акра для 3 акрів.
Її пересічна величина для землі, що дає ренту, була раніш 6 ф. стерл., а тепер
зона дорівнює 9 ф. стерл. з акра. Отже, грошова вартість ренти з акра підвищилась
і репрезентує тепер удвоє більше продукту в збіжжі, ніж давніш, але
12 квартерів збіжжевої ренти тепер становлять менше, ніж половину всього продукту
в 34, зглядно 30\footnote*{В німецькому тексті стоїть: усього «продукту в 33, зглядно 27 квартерів» Явна помилка,
як це можна бачити з таблиць ІVс і IVd. \emph{Прим. Ред.}} квартерів, тимчасом як у таблиці І 6 квартерів становлять
\sfrac{3}{5}  усього продукту в 10 квартерів. Отже, хоч рента, коли розглядати
її як відповідну частину всього продукту, а також коли обчислити її у відношенні
до витраченого капіталу, і знизилась, одначе її грошова вартість,
обчислена на акр. збільшилась, а її вартість в продукті, збільшилась ще дужче.
Коли ми візьмемо землю D в таблиці IVd, то ціна продукції тут дорівнює
15 ф. стерл., що з них витрачений капітал = 12\sfrac{1}{2} ф. стерл. Грошова рента = 15
ф. стер. У таблиці І на тій самій землі D ціна продукції була 3 ф. стерл., витрачений
капітал = 2\sfrac{1}{2} ф. стерл., грошова рента = 9 ф. стерл., отже, остання
утроє більша за ціну продукції й майже у чотири рази більша за витрачений
капітал. У таблиці IVd для D грошова рента в 15 ф. стерл. якраз дорівнює ціні
продукції і лише на \sfrac{1}{5}  більша за витрачений капітал. А все ж грошова рента
з акра на \sfrac{2}{3}  більша, 15 ф. стерл. замість 9 ф. стерл. В таблиці І збіжжева
рента в 3 квартери = \sfrac{3}{4}  усього продукту, що становить 4 квартери, в таблиці
IVd вона = 10 квартерам, половині всього продукту (20 квартерів) з акра
землі D. Це показує, що грошова і збіжжева рента з акра може зрости, хоч
вона і становить відносно меншу частину всього здобутку і знизилась у відношенні
до авансованого капіталу.

Вартість всього продукту в таблиці І = 30 ф. стерл.; рента = 18 ф.
стерл. більше від половини цієї вартости. Вартість усього продукту в таблиці
IV = 45 ф. стерл., що з них 18 ф. стерл., менш від половини, становлять
ренту.


\index{iii2}{0157}  %% посилання на сторінку оригінального видання
Причина ж того, що не зважаючи на пониження ціни на 1\sfrac{1}{2} ф. стерл.
за квартер, отже на 50\%, і не зважаючи на зменшення площі конкурентної
землі з 4 до 3 акрів, загальна сума грошової ренти лишається та сама, а збіжжева
рента подвоюється, тимчасом як збіжжева й грошова рента, обчислена на акр, підвищується,
— причина цього в тому, що вироблено більше квартерів надпродукту.
Ціна збіжжя знижується на 50\%, надпродукт зростає на 100\%.
Але для досягнення такого наслідку вся продукція, згідно з нашими умовами,
мусить збільшитись утроє, а капітал, вкладений у кращу землю, мусить більше
ніж подвоїтись. В якому відношенні він мусить збільшуватись, залежить насамперед
від того, як розподіляється додаткові вкладення капіталу між кращими та
найкращими землями, припускаючи завжди, що продуктивність капіталу на
кожній категорії землі зростає пропорційно його величині.

Коли б пониження ціни продукції було менш значне, то потрібно було б
менше додаткового капіталу, щоб випродукувати ту саму грошову ренту. Коли б
подання збіжжя потрібне для того, щоб вилучити А з числа оброблюваних земель,
— а це залежить не тільки від кількости продукту з акра землі А, але
також і від того, яку частину всієї оброблюваної земельної площі становить А, —
отже, коли б потрібне для цього подання було більше, отже, коли б також
потрібно було і більшої маси додаткового капіталу на кращій, ніж А землі, то,
за інших незмінних відношень грошова і збіжжева ренти зросли б ще більше,
не зважаючи на те, що земля В перестала б давати грошову і збіжжеву ренту.

Коли б капітал, що перестав функціонувати на землі А, дорівнював 5 ф.
стерл., то для цього випадку треба було б взяти для порівняння обидві таблиці:
II і ІVd. Весь продукт збільшився б з 20 до 30 квартерів. Грошова рента
зменшилася б удвоє, вона дорівнювала б 18 ф. стерл. замість 36 ф. стерл.,
збіжжева рента залишилась би та сама = 12 квартерів.

Коли б можна було випродукувати на землі D 44 квартери загального
продукту = 66 ф. стерл., вкладаючи капітал в 27\sfrac{1}{2} ф. стерл., — що відповідало б
колишньому припущенню для D: 4 квартери на 2\sfrac{1}{2} ф. стерл. капіталу, —
то загальна сума\footnote*{
Тут очевидно справа йде про загальну грошову ренту. \emph{Прим. Ред.}
} ренти знову досягла б тієї висоти, яку вона мала в таблиці
II, і таблиця набула б такого вигляду:

\begin{table}[h]
  \begin{center}
  \begin{tabular}{c c c c c}
  \toprule
  \makecell{Рід\\землі}  & \makecell{Капітал\\ф. ст.} & \makecell{Продукт в \\ квартерах} & \makecell{Збіжжева \\ рента \\ в кварт.}& \makecell{Грошова\\рента \\ф. ст.}\\
  \midrule
  B &    \phantom{0}5\phantom{\sfrac{1}{2}} & \phantom{0}4  & \phantom{0}0  & \phantom{0}0\\
  C &    \phantom{0}5\phantom{\sfrac{1}{2}} & \phantom{0}6  & \phantom{0}2  & \phantom{0}3\\
  D &   27\sfrac{1}{2}                      & 44            & 22            & 33\\
  \cmidrule(rl){1-1}
  \cmidrule(rl){2-2}
  \cmidrule(rl){3-3}
  \cmidrule(rl){4-4}
  \cmidrule(rl){5-5}
  Разом. & 37\sfrac{1}{2} &      54  &  24  &  36\\
  \end{tabular}
  \end{center}
\end{table}

Уся продукція була б 54 квартери проти 20 квартерів у таблиці II, грошова рента була
б та сама = 36 ф. стерл. Але весь капітал був би 37\sfrac{1}{2} ф. стерл., тимчасом як у таблиці II
він був = 20 ф. стерл. Весь авансований капітал майже подвоївся б, тимчасом як продукція майже
потроїлася б; збіжжева рента збільшилася б удвоє, грошова рента залишилася б та сама.
Отже, коли ціна, за незмінної продуктивности, знижується в наслідок приміщення
додаткового грошового капіталу у кращі землі, що дають ренту, отже
в усі землі кращі від А, то весь капітал має тенденцію зростати не в такій
самій пропорції, як продукція і збіжжева рента; так що зростання збіжжевої ренти
може урівноважити падіння грошової ренти, яке постає в наслідок пониження ціни.
Той самий закон виявляється і в тому, що авансований капітал мусить бути більший
відповідно до того, як його вживається більше на землі С, ніж на D, — на землі,
що дає менше ренти, ніж на тій, яка дає більше ренти. Це визначає лише ось що:

\parbreak{}  %% абзац продовжується на наступній сторінці

\parcont{}  %% абзац починається на попередній сторінці
\index{ii}{0158}  %% посилання на сторінку оригінального видання
протиставиться другій складовій частині сталого капіталу, витраченій на
засоби праці. Додаткова вартість, отже, саме та обставина, що перетворює
витрачену суму вартости на капітал, лишається при цьому цілком
поза розглядом. Так само поза розглядом лишається й те, що частину
вартости, яку долучає до продукту витрачений на заробітну плату капітал,
випродукувано знову (тобто справді репродуковано), тимчасом як
частину вартости, що її долучає до продукту сировинний матеріял, не
випродукувано знову, не репродуковано в дійсності, а лише збережено
в вартості продукту, консервовано, і тому вона лише знову з’являється як
складова частина вартости продукту. Ріжниця, як вона виявляється тепер
з погляду протилежности між поточним і основним капіталом, сходить
лише ось на що: вартість засобів праці, вжитих для продукції товару,
лише частинами входить у вартість товару, а тому й лише частинами
покривається через продаж товарів, а значить, і взагалі покривається вона
тільки частинами й поступінно. З другого боку, вартість робочої
сили та предметів праці (сировинні матеріяли тощо), вжитих для
продукції товару, цілком увіходить у товар і тому цілком покривається
через продаж його. В цьому розумінні, отже, щодо процесу
циркуляції одна частина капіталу виступає як основний, а друга як поточний
або обіговий капітал. В обох випадках ідеться про перенесення
даної, авансованої вартости на продукт і про покриття її через продаж
продукту. Ріжниця тут лише в тому, як відбувається це перенесення вартости,
а, значить, і покриття вартости: чи частинами й поступінно, чи
одразу одним заходом. Цим самим затушковується найвирішальнішу ріжницю
між змінним і сталим капіталом, отже, затушковується всю таємницю
утворення додаткової вартости і всю таємницю капіталістичної продукції,
затушковується обставини, що перетворюють на капітал певні вартості
й речі, що в них ці вартості втілюються. Всі складові частини капіталу
відрізняються тут тільки способом циркуляції (а циркуляція товару,
звичайно, має чинення тільки до наявних уже, даних вартостей); але особливий
спосіб циркуляції капіталу, витраченого на заробітну плату, спільний і
частині капіталу, витраченій на сировинні матеріяли, напівфабрикати,
допоміжні матеріяли, протилежно до частини капіталу, витраченої на засоби
праці.

Відси зрозуміло, чому буржуазна політична економія інстинктивно
зберігала Смісову плутанину категорій „сталого й змінного капіталу“
з категоріями „основного й обігового капіталу“ і без будь-якої критики
протягом цілого століття передавала цю плутанину з покоління в покоління.
На її погляд, витрачена на заробітну плату частина капіталу зовсім
уже не відрізняється від частини капіталу, витраченої на сировинний
матеріял, і відрізняється лише формально — лише тим, чи циркулює вона
разом з продуктом частинами, чи цілком — від сталого капіталу. Цим
самим одним ударом руйнується основи, потрібні для того, щоб зрозуміти
справжній рух капіталістичної продукції, а, значить, і капіталістичної
експлуатації. Для неї справа сходить лише на відновлення авансованих
вартостей.


\index{iii2}{0159}  %% посилання на сторінку оригінального видання
Для зручности порівняння поновимо насамперед таку таблицю:

\begin{table}[h]
  \begin{center}
    \emph{Таблиця І}
    \footnotesize

  \begin{tabular}{c c c c c c c c c}
    \toprule
      \multirowcell{2}{Земля} &
      \multirowcell{2}{Акри} &
      Капітал &
      Зиск &
      \multirowcell{2}{\makecell{Ціна про- \\ дукції в \\ квартерах}} &
      \multirowcell{2}{\makecell{Продукт в\\ квартерах}} &
      \multicolumn{2}{c}{Рента} &
      \multirowcell{2}{\makecell{Норма \\надзиску}} \\

      \cmidrule(rl){3-3}
      \cmidrule(rl){4-4}
      \cmidrule(rl){7-7}
      \cmidrule(rl){8-8}

       &  &  ф. ст. & ф. ст. & & & Кварт. & ф. ст. &   \\
       \\
      \midrule

       A & 1 & \phantom{0}2\sfrac{1}{2} & \sfrac{1}{2} & 3\phantom{\sfrac{1}{2}} & \phantom{0}1 & 0 & \phantom{0}0 & \phantom{00}0\% \\
       B & 1 & \phantom{0}2\sfrac{1}{2} & \sfrac{1}{2} & 1\sfrac{1}{2}           & \phantom{0}2 & 1 & \phantom{0}3 & 120\% \\
       C & 1 & \phantom{0}2\sfrac{1}{2} & \sfrac{1}{2} & 1\phantom{\sfrac{1}{2}} & \phantom{0}3 & 2 & \phantom{0}6 & 240\%\\
       D & 1 & \phantom{0}2\sfrac{1}{2} & \sfrac{1}{2} & \phantom{0}\sfrac{3}{4} & \phantom{0}4 & 3 & \phantom{0}9 & 360\%\\
     \cmidrule(rl){1-1}
     \cmidrule(rl){2-2}
     \cmidrule(rl){3-3}
     \cmidrule(rl){6-6}
     \cmidrule(rl){7-7}
     \cmidrule(rl){8-8}
     \cmidrule(rl){9-9}

      Разом & 4 & 10 & & & 10 & 6 & 18 & \makecell{180\% \\ пересічно}\\
  \end{tabular}

  \end{center}
\end{table}

Коли ми тепер припустимо, що цифра 16 квартерів, що їх даватимуть землі
В, С, D за низхідної норми продуктивности, достатня для того, щоб вилучити
А з числа оброблюваних земель, то таблиця III перетворюється на таку:

\begin{table}[h]
  \begin{center}
    \emph{Таблиця V}
    \footnotesize

  \begin{tabular}{c@{  } c@{  } c@{  } c@{  } c@{  } c@{  } c@{  } c@{  } c@{  } с}
    \toprule
      \multirowcell{2}{Земля} &
      \multirowcell{2}{Акри} &
      \makecell{Вкладення \\ капіталу} &
      Зиск &
      \multirowcell{2}{\makecell{Продукт в\\ квартерах}} &
      \makecell{Продажна \\ ціна} &
      \makecell{Здо-\\буток} &
      \multicolumn{2}{c}{Рента} &
      \multirowcell{2}{\makecell{Норма \\надзиску}} \\

      \cmidrule(r){3-3}
      \cmidrule(r){4-4}

      \cmidrule(r){6-6}
      \cmidrule(r){7-7}
      \cmidrule(r){8-8}
      \cmidrule(r){9-9}

       &  & ф. ст. & ф. ст. & & ф. ст. & ф. ст. & Кварт. & ф. ст. &   \\
      \midrule

       B & 1 & 2\sfrac{1}{2} + 2\sfrac{1}{2} & 1 & 2 + 1\sfrac{1}{2} = 3\sfrac{1}{2}                     & 1\sfrac{5}{7} & \phantom{0}6\phantom{\sfrac{1}{2}} & 0\phantom{\sfrac{1}{2}} & 0\phantom{\sfrac{1}{2}} & \phantom{00}0\phantom{\sfrac{1}{2}}\% \\
       C & 1 & 2\sfrac{1}{2} + 2\sfrac{1}{2} & 1 & 3 + 2\phantom{\sfrac{1}{2}} = 5\phantom{\sfrac{1}{2}} & 1\sfrac{5}{7} & \phantom{0}8\sfrac{4}{7}           & 1\sfrac{1}{2}           & 2\sfrac{4}{7}           & \phantom{0}51\sfrac{2}{5}\%\\
       D & 1 & 2\sfrac{1}{2} + 2\sfrac{1}{2} & 1 & 4 + 3\sfrac{1}{2} = 7\sfrac{1}{2}                     & 1\sfrac{5}{7} & 12\sfrac{6}{7}                     & 4\phantom{\sfrac{1}{2}} & 6\sfrac{6}{7}           & 137\sfrac{1}{5}\%\\
     \cmidrule(r){1-1}
     \cmidrule(r){2-2}
     \cmidrule(r){3-3}
     \cmidrule(r){5-5}
     \cmidrule(r){7-7}
     \cmidrule(r){8-8}
     \cmidrule(r){9-9}
     \cmidrule(r){10-10}

      Разом & 3 & 15 & &  \phantom{2 + 1\sfrac{1}{2} =}16\phantom{\sfrac{1}{2}} & & 27\sfrac{3}{7} & 5\sfrac{1}{2} & 9\sfrac{3}{7} & \makecell{94\sfrac{3}{10}\% \\ пересічно\footnotemarkZ{}}\\
  \end{tabular}

  \end{center}
\end{table}
\footnotetextZ{Тут пересічну норму надзиску обчислено не до всього вкладеного капіталу, а тільки до капіталу, вкладеного в рентодайні дільниці С і D. \emph{Прим. Ред.}} % текст примітки прямо під заголовком

Тут за низхідної норми продуктивности додаткових капіталів і за різного
ступеня цього зменшення на різних землях, реґуляційна ціна продукції знизилася
з 3 ф. стерл. до 1\sfrac{5}{7} ф. стерл. Вкладення капіталу збільшилося наполовину з 10 ф.
стерл. до 15 ф. стерл. Грошова рента зменшилася майже удвоє, з 18 до 9 \sfrac{3}{7} ф.
стерл., але збіжжева рента лише на \sfrac{1}{2} \footnote*{
В німецькому тексті тут стоїть «\sfrac{1}{22}». Очевидна помилка. \emph{Прим. Ред.}
}, з 6 квартерів до 5\sfrac{1}{2}. Весь продукт
збільшився з 10 до 16, або на 60\%\footnote*{
В німецькому тексті тут помилково стоїть: «160\%». \emph{Прим. Ред.}
}. Збіжжева рента становить небагато більше
від третини всього продукту. Авансований капітал відноситься до грошової ренти
як 15: 9\sfrac{3}{7}, тимчасом як давніш це відношення було 10:18.

\subsection{За висхідної норми продуктивности додаткових капіталів.}

Цей випадок відрізняється від варіянту І, наведеного на початку цього
розділу, де ціна продукції за незмінної норми продуктивности знижується, тільки
тим, що коли потрібна додаткова кількість продукту для того, щоб вилучити
землю А, то це відбувається тут швидше.

Так за низхідної, як і за висхідної продуктивности додаткових вкладень
капіталу може це різно впливати, залежно від того, як ці вкладення розподіляються
між різними родами землі. В міру того, як цей різний вплив
\parbreak{}  %% абзац продовжується на наступній сторінці

\input{_0160c.tex}
\parcont{}  %% абзац починається на попередній сторінці
\index{ii}{0161}  %% посилання на сторінку оригінального видання
капіталові як сталий капітал — це з погляду процесу зростання вартости.
Або, коли тут мова повинна бути про речову ріжницю, оскільки вона
впливає на процес циркуляції, то справа така: з природи вартости, яка є
не що інше, як зречевлена праця, і з природи діющої робочої сили, яка
є не що інше, як праця, що зречевлює себе, випливає, що робоча сила
протягом періоду її функціонування постійно утворює вартість і долярову
вартість; і що те, що на боці робочої сили виявляється як рух, як
утворення вартости, на боці її продукту виявляється у формі спокою,
як уже утворена вартість. Коли робоча сила вже діяла, то капітал не
складається вже більше з робочої сили на одному боці, із засобів продукції
на другому. Капітальна вартість, витрачена на робочу силу, є тепер
вартість, що її (\dplus{} додаткову вартість) долучено до продукту. Щоб
повторити процес, треба продати продукт і на вторговані гроші знову й
знову купувати робочу силу і вводити її в продуктивний капітал. Це
надає тоді частині капіталу, витраченій на робочу силу, так само, як і частинам
його, витраченим на матеріял праці тощо, характер обігового капіталу,
протилежно до того капіталу, що лишається закріплений у засобах праці.

Коли, навпаки, другорядне визначення обігового капіталу, спільне
йому з частиною сталого капіталу (сировинними й допоміжними матеріялами)
— саме те визначення, що вартість, витрачену на обіговий капітал,
цілком переноситься на продукт, в продукції якого його зуживається, а
не поступінно й частинами, як в основного капіталу, що вартість ця,
отже, мусить цілком заміститися через продаж продукту, — перетворити
на посутню характеристику частини капіталу, витраченої на робочу силу,
то й частина капіталу, витрачена на заробітну плату, речово мусить
складатися не з діющої робочої сили, а з речових елементів, що їх робітник
купує на свою плату, отже, з частини суспільного товарового капіталу,
яка ввіходить у споживання робітника — з засобів існування.
Основний капітал складається при такому погляді на справу з засобів
праці, що зношуються повільніше, а тому й доводиться їх рідше відновлювати,
а капітал, витрачений на робочу силу, з засобів існування, що
їх треба заміщувати швидше.

Однак межі швидшої та повільнішої зношуваности стираються.

„Харч і одяг що їх зуживає робітник, будівлі, де він працює, знаряддя,
що допомагають йому в роботі, всі ці речі з своєї природи минущі.
Але є величезна ріжниця в часі, що протягом його зберігаються
ці різні капітали: парова машина зберігається довший час, ніж корабель,
корабель — довший час, ніж одяг робітника, одяг робітника знову таки
довший час, ніж харч, що його він споживає“\footnote{
„The food and clothing consumed the labourer, the buildings in which he
works, the implements with which his labour is assisted, are all of a perishable
nature. There is, however, a vast difference in the time for which these different
capitals will endure: a steam-engine will last longer than a ship, a ship than the
clothing of the labourer, and the clothing of the labourer longer than the food which
he consumes“. (Ricardo, etc., p. 27).
}.

\input{_0162.tex}
\parcont{}  %% абзац починається на попередній сторінці
\index{iii1}{0163}  %% посилання на сторінку оригінального видання
тільки окремі частини (як, наприклад, на якійсь бавовняній фабриці, в різних відділах якої —
кардувальному, підготовчому, прядільному й ткацькому — існує різне відношення між змінним і
сталим капіталом і де пересічне відношення для всієї фабрики
ще тільки має бути обчислене), то, по-перше, пересічний склад
капіталу в 500 був би $= 390 c + 110 v$, або в процентах $78 c + 22 v$.
Кожний з цих капіталів в 100, розглядуваний тільки як \sfrac{1}{5} сукупного капіталу, мав би своїм складом цей
пересічний склад
в $78 c + 22 v$; так само на кожні 100 припадало б 22 як пересічна
додаткова вартість; тому пересічна норма зиску була б $=$ 22\%,
і, нарешті, ціна кожної п’ятої частини сукупного продукту, виробленого цими 500, дорівнювала б 122.
Отже, продукт кожної
п’ятої частини сукупного авансованого капіталу мусив би продаватись за 122.

Однак, щоб не прийти до цілком хибних висновків, не слід
усі витрати виробництва рахувати рівними 100.

При $80 c + 20 v$ і нормі додаткової вартості = 100\% вся вартість товару, виробленого капіталом І =
100, була б $= 80 c + 20 v + 20 m = 120$, коли б весь сталий капітал входив у річний продукт. При
певних обставинах це, звичайно, може мати місце
в певних сферах виробництва. Однак, ледве чи це можливе там,
де відношення $c : v = 4 : 1$. Отже, при дослідженні вартостей товарів, вироблюваних кожними 100
одиницями різних капіталів,
треба взяти до уваги те, що ці вартості можуть бути різні, залежно від різного складу $c$ з основних і
обігових складових
частин, і що основні складові частини різних капіталів, в свою
чергу, зношуються швидше або повільніше, отже, за однакові
періоди часу додають до продукту неоднакові кількості вартості. Але для норми зиску це не має
значення. Чи $80 c$ віддають
річному продуктові вартість в 80, чи в 50, чи в 5, отже, чи річний
продукт $= 80 c + 20 v + 20 m = 120$, чи $= 50 c + 20 v + 20 m = 90$, чи $= 5 c + 20 v + 20 m = 45$, — в
усіх цих випадках надлишок
вартості продукту понад його витрати виробництва = 20, і в усіх
цих випадках при встановленні норми зиску ці 20 обчислюються
на капітал в 100; отже, норма зиску для капіталу І в усіх випадках $=$ 20\%. Щоб зробити це ще яснішим,
ми в нижченаведеній
таблиці припускаємо для тих самих п’яти капіталів, про які мова
йшла вище, що у вартість продукту з цих п’яти капіталів входять різні частини сталого капіталу.
\begin{footnotesize}
\footnotesize
\begin{tabular}{c@{ } c@{ } c@{ } c@{ } c@{ } c@{ } c@{ } c@{ } }
\toprule
\multicolumn{2}{c}{Капітали} &
\makecell{Норма\\додаткової\\вартості} &
\makecell{Додаткова\\вартість} &
\makecell{Норма\\зиску} &
\makecell{Зношування\\$c$} &
\makecell{Вартість\\товарів} &
\makecell{Витрати\\виробництва} \\
\midrule
І.        & $\phantom{0}80 c + \phantom{0}20 v$ & 100\%  &  \phantom{0}20   & 20\%           & 50 & \phantom{0}90  & 70  \\
II.       & $\phantom{0}70 c + \phantom{0}30 v$ & 100\%  &  \phantom{0}30   & 30\%           & 51 & 111 & 81  \\
III.      & $\phantom{0}60 c + \phantom{0}40 v$ & 100\%  &  \phantom{0}40   & 40\%           & 51 & 131 & 91  \\
IV.       & $\phantom{0}85 c + \phantom{0}15 v$ & 100\%  &  \phantom{0}15   & 15\%           & 40 & \phantom{0}70  & 55  \\
V.        & $\phantom{0}95 c + \phantom{00}5 v$ & 100\%  &  \phantom{00}5   & \phantom{0}5\% & 10 & \phantom{0}20  & 15  \\
Сума      & $390 c + 110 v $                    & \textemdash  &  110             &  \textemdash   & \textemdash & \textemdash & \textemdash \\
Пересічно & $\phantom{0}78 c + \phantom{0}22 v$ & \textemdash &  \phantom{0}22   &  22\%          & \textemdash & \textemdash & \textemdash \\
\end{tabular}
\end{footnotesize}

\input{_0164c.tex}

\index{iii2}{0165}  %% посилання на сторінку оригінального видання
У вищенаведеному випадку ми припускали, що продуктивна сила другого
капіталовкладення вища, ніж первісна продуктивність першого вкладення. Справа
не зміниться, коли ми припустимо для другого капіталовкладення лише таку саму
продуктивність, що її мала первісна продуктивність першого вкладення, як от у
таблиці VIII.

\begin{table}[h]
  \begin{center}
    \emph{Таблиця VIII}
    \footnotesize

  \begin{tabular}{c@{  } c@{  } c@{  } c@{  } c@{  } c@{  } c@{  } c@{  } c@{  } c@{  } с}
    \toprule
      \multirowcell{2}{\makecell{Рід\\ землі}} &
      \multirowcell{2}{Акри} &
      Капітал &
      Зиск &
      \makecell{Ціна\\ продук.} &
      \multirowcell{2}{\makecell{Продукт в\\ квартерах}} &
      \makecell{Продажна \\ ціна} &
      \makecell{Здо-\\буток} &
      \multicolumn{2}{c}{Рента} &
      \multirowcell{2}{\makecell{Норма \\надзиску}} \\

      \cmidrule(r){3-3}
      \cmidrule(r){4-4}
      \cmidrule(r){5-5}
      \cmidrule(r){7-7}
      \cmidrule(r){8-8}
      \cmidrule(r){9-9}
      \cmidrule(r){10-10}

       &  & ф. ст. & ф. ст. & ф. ст. & & ф. ст. & ф. ст. & Кварт. & ф. ст. &   \\
      \midrule
      A & 1 & 2\sfrac{1}{2} + 2\sfrac{1}{2} = 5 & 1 & 6 & \phantom{0}\sfrac{1}{2} + 1 = 1\sfrac{1}{2}                                 & 4 & \phantom{0}6 & 0\phantom{\sfrac{1}{2}} & \phantom{0}0 & \phantom{00}0\% \\
      B & 1 & 2\sfrac{1}{2} + 2\sfrac{1}{2} = 5 & 1 & 6 & 1\phantom{\sfrac{0}{0}} + 2 = 3\phantom{\sfrac{0}{0}}                       & 4 & 12           & 1\sfrac{1}{2}           & \phantom{0}6 & 120\% \\
      C & 1 & 2\sfrac{1}{2} + 2\sfrac{1}{2} = 5 & 1 & 6 & 1\sfrac{1}{2} + 3 = 4\sfrac{1}{4}                                           & 4 & 18           & 3\phantom{\sfrac{1}{2}} & 12           & 240\%\\
      D & 1 & 2\sfrac{1}{2} + 2\sfrac{1}{2} = 5 & 1 & 6 & 2\phantom{\sfrac{0}{0}} + 4 = 6\phantom{\sfrac{0}{0}} & 4 & 24           & 4\sfrac{1}{2}           & 18           & 360\%\\

     \cmidrule(r){3-3}
     \cmidrule(l){6-6}
     \cmidrule(r){8-8}
     \cmidrule(r){9-9}
     \cmidrule(r){10-10}
     \cmidrule(r){11-11}

      Разом & & \phantom{2\sfrac{1}{2} + 2\sfrac{1}{2} =}20 & & & \phantom{2 + 1\sfrac{1}{2} =}15 & & 60 & 9 & 36 & 240\%\\
  \end{tabular}

  \end{center}
\end{table}

І тут ціна продукції, яка підвищується в тому самому відношенні, зумовлює
те, що зменшення продуктивности цілком урівноважується так щодо здобутку,
як і щодо грошової ренти.

У своєму чистому вигляді третій випадок виступає лише за низхідної продуктивности
другого капіталовкладення, тимчасом як продуктивність першого вкладення,
як це всюди припускалось для першого і другого випадків, лишається сталою.
Диференційна рента І тут не зачіпається, зміна відбувається лише з тією частиною,
що походить з диференційної ренти II. Ми подаємо два приклади: в
першому продуктивність другого капіталовкладення зводиться до \sfrac{1}{2}, у другому
— до \sfrac{1}{4} продуктивности першого вкладення.

\begin{table}[h]
  \begin{center}
    \emph{Таблиця IX}
    \footnotesize

  \begin{tabular}{c@{  } c@{  } c@{  } c@{  } c@{  } c@{  } c@{  } c@{  } c@{  } c@{  } с}
    \toprule
      \multirowcell{2}{\makecell{Рід\\ землі}} &
      \multirowcell{2}{Акри} &
      Капітал &
      Зиск &
      \makecell{Ціна\\ продук.} &
      \multirowcell{2}{\makecell{Продукт в\\ квартерах}} &
      \makecell{Продажна \\ ціна} &
      \makecell{Здо-\\буток} &
      \multicolumn{2}{c}{Рента} &
      \multirowcell{2}{\makecell{Норма \\ренти}} \\

      \cmidrule(r){3-3}
      \cmidrule(r){4-4}
      \cmidrule(r){5-5}
      \cmidrule(r){7-7}
      \cmidrule(r){8-8}
      \cmidrule(r){9-9}
      \cmidrule(r){10-10}

       &  & ф. ст. & ф. ст. & ф. ст. & & ф. ст. & ф. ст. & Кварт. & ф. ст. &   \\
      \midrule
      A & 1 & 2\sfrac{1}{2} + 2\sfrac{1}{2} = 5 & 1 & 6 & 1 + \phantom{0}\sfrac{1}{2} = 1\sfrac{1}{2}                                 & 4 & \phantom{0}6 & 0\phantom{\sfrac{1}{2}} & \phantom{0}0 & \phantom{00}0\% \\
      B & 1 & 2\sfrac{1}{2} + 2\sfrac{1}{2} = 5 & 1 & 6 & 2 + 1\phantom{\sfrac{0}{0}} = 3\phantom{\sfrac{0}{0}}                       & 4 & 12           & 1\sfrac{1}{2}           & \phantom{0}6 & 120\% \\
      C & 1 & 2\sfrac{1}{2} + 2\sfrac{1}{2} = 5 & 1 & 6 & 3 + 1\sfrac{1}{2} = 4\sfrac{1}{2}                                           & 4 & 18           & 3\phantom{\sfrac{1}{2}} & 12           & 240\%\\
      D & 1 & 2\sfrac{1}{2} + 2\sfrac{1}{2} = 5 & 1 & 6 & 4 + 2\phantom{\sfrac{0}{0}} = 6\phantom{\sfrac{0}{0}} & 4 & 24           & 4\sfrac{1}{2}           & 18           & 360\%\\

     \cmidrule(r){3-3}
     \cmidrule(l){6-6}
     \cmidrule(r){8-8}
     \cmidrule(r){9-9}
     \cmidrule(r){10-10}
     \cmidrule(r){11-11}

      Разом & & \phantom{2\sfrac{1}{2} + 2\sfrac{1}{2} =}20 & & & \phantom{2 + 1\sfrac{1}{2} =}15 & & 60 & 9 & 36 & 240\%\\
  \end{tabular}

  \end{center}
\end{table}

Таблиця IX та сама, що й таблиця VIII, тільки в таблиці VIII зменшення
продуктивности припадає на перше, в таблиці IX — на друге капіталовкладення.


\index{i}{0166}  %% посилання на сторінку оригінального видання
Жакоб, припускаючи ціну пшениці в 80\shil{ шилінґів} за квартер і пересічний урожай в 22 бушлі з одного
акра, так що один акр приносить 11\pound{ фунтів стерлінґів}, наводить для 1815~\abbr{р.} обрахунок, який через те,
що в ньому вже переведено компенсацію різних
пунктів, дуже хибний, але все ж для нашої мети придатний.

\begin{table}[H]
\caption*{Продукція вартости на 1 акр}
\noindent\begin{tabularx}{\textwidth}{@{}X*{4}{@{~}r}@{\hspace{5em}}X*{4}{@{~}r}@{}}
Насіння (пшениця)\dotfill{} & 1 & \pound{ф. ст.} & 9 &\shil{шил.} &
Десятини, податки\dotfill{} & 1 & \pound{ф. ст.} & 1 &\shil{шил.} \\
Добриво\dotfill{} & 2 & \dittomark{} & 10 & \dittomark{} &
Рента\dotfill{} & 1 & \dittomark{} &  8 & \dittomark{} \\

Заробітна плата\dotfill{} &  3  & \dittomark{} &  10 & \dittomark{} &
Зиск фармера й проц.\dotfill{}& 1 & \dittomark{} & 2 & \dittomark{} \\

\cmidrule(r{5em}){1-5}  \cmidrule{6-10} 

Разом\dotfill{} & 7 & \pound{ф. ст.} & 9 & \shil{шил.} &
Разом\dotfill{} & З & \pound{ф. ст.} & 11 & \shil{шил.}
\end{tabularx}
\end{table}

\noindent{}Додаткова вартість, припускаючи завжди, що ціна продукту дорівнює його вартості, розподіляється тут
між різними рубриками: зиск, процент, десятина й~\abbr{т. ін.} Ці рубрики для нас не мають значення. Ми
складаємо їх і як результат маємо додаткову вартість у 3\pound{ фунти стерлінґів} 11\shil{ шилінґів.} Ті 3\pound{ фунти
стерлінґів} 19\shil{ шилінґів}, що коштують насіння і добриво, ми, як сталу частину капіталу, прирівнюємо
нулеві. Лишається авансований змінний капітал у 3\pound{ фунти стерлінґів} 10\shil{ шилінґів}, замість якого
спродуковано нову вартість у 3\pound{ фунти стерлінґів} 10\shil{ шилінґів} \dplus{} 3\pound{ фунти стерлінґів} 11\shil{ шилінґів.} Отже,
$\frac{m}{v} \deq{} \frac{3\text{\pound{ фунти ст.} }11\text{\shil{ шилінґів}}}{3\text{\pound{ фунти ст.} }10\text{\shil{ шилінґів}}}$ становить більше, ніж 100\%. Робітник більш
ніж половину свого робочого дня вживає на продукцію додаткової вартости, яку різні особи під різними
приводами  розподіляють проміж себе\footnoteA{
Наведені обчислення мають значення лише як ілюстрація. Справді, ми припускаємо, що ціни
дорівнюють вартостям. У третій книзі ми побачимо,
що це прирівняння робиться не так просто навіть для пересічних цін.
}.

\subsection{Вираз вартости продукту у відносних частинах продукту}

Вернімось тепер до того прикладу, що показав нам, як капіталіст із грошей робить капітал. Доконечна
праця його прядуна
становила 6 годин, додаткова праця — стільки ж, отже, ступінь експлуатації робочої сили — 100\%.

Продукт дванадцятигодинного робочого дня є 20 фунтів пряжі вартістю в 30\shil{ шилінґів.} Не менше як \sfrac{8}{10}
вартости цієї пряжі (24\shil{ шилінґи}) становить вартість зужиткованих засобів продукції, що лише знову
з’являється (20 фунтів бавовни на 20\shil{ шилінґів}, веретена й~\abbr{т. ін.} на 4\shil{ шилінґи}), або, інакше кажучи,
складається з сталого капіталу. Решта, \sfrac{2}{10}, є нова вартість у 6\shil{ шилінґів}, яка постала підчас
процесу прядіння, що з них половина компенсує авансовану денну вартість робочої сили, або змінний
капітал, а друга половина становить додаткову вартість у 3\shil{ шилінґи.} Отже, сукупна вартість цих 20
фунтів пряжі складаєься ось як:
вартість пряжі в $30\text{\shil{ шилінґів}}
\deq{} 24\text{\shil{ шилінґи}} \dplus{}
\oversetl{v}{v-166-node-1}{3\text{\shil{ шилінґи}}} \dplus{}
\oversetr{m}{m-166-node-1}{3\text{\shil{ шилінґи.}}}$
\begin{tikzpicture}[overlay]
    \path[-,thick,black] (v-166-node-1) edge [out=5,in=175] (m-166-node-1);
\end{tikzpicture}%


\index{iii2}{0167}  %% посилання на сторінку оригінального видання

\begin{table}[h]
  \begin{center}
    \emph{Таблиця Xa}
    \footnotesize

  \begin{tabular}{c@{  } c@{  } c@{  } c@{  } c@{  } c@{  } c@{  } c@{  } c@{  } c@{  } с}
    \toprule
      \multirowcell{2}{\makecell{Рід\\ землі}} &
      \multirowcell{2}{Акри} &
      Капітал &
      Зиск &
      \makecell{Ціна\\ продук.} &
      \multirowcell{2}{\makecell{Продукт в\\ квартерах}} &
      \makecell{Продажна \\ ціна} &
      \makecell{Здо-\\буток} &
      \multicolumn{2}{c}{Рента} &
      \multirowcell{2}{Підвищення} \\

      \cmidrule(r){3-3}
      \cmidrule(r){4-4}
      \cmidrule(r){5-5}
      \cmidrule(r){7-7}
      \cmidrule(r){8-8}
      \cmidrule(r){9-9}
      \cmidrule(r){10-10}

       &  & ф. ст. & ф. ст. & ф. ст. & & ф. ст. & ф. ст. & Кварт. & ф. ст. &   \\
      \midrule
      a & 1 & \phantom{2\sfrac{1}{2} + }5\phantom{\sfrac{1}{2}} & 1 & 6 & \phantom{1\sfrac{1}{2} + 3 = }1\sfrac{1}{8}           & 5\sfrac{1}{3} & \phantom{0}6\phantom{\sfrac{1}{5}} & 0\phantom{\sfrac{1}{2}}  & \phantom{0}0\phantom{\sfrac{1}{1}} & 0\phantom{\sfrac{1}{5} + 3 × 7\sfrac{1}{5}} \\
      A & 1 & 2\sfrac{1}{2} + 2\sfrac{1}{2}                     & 1 & 6 & 1 + \phantom{0}\sfrac{1}{4} = 1\sfrac{1}{4}           & 5\sfrac{1}{3} & \phantom{0}6\sfrac{2}{3}           & \phantom{0}\sfrac{1}{8}  & \phantom{00}\sfrac{2}{3}           & \sfrac{2}{3}\phantom{ + 3 × 7\sfrac{1}{5}} \\
      B & 1 & 2\sfrac{1}{2} + 2\sfrac{1}{2}                     & 1 & 6 & 2 + \phantom{0}\sfrac{1}{2} = 2\sfrac{1}{2}           & 5\sfrac{1}{3} & 13\sfrac{1}{3}                     & 1\sfrac{3}{8}            & \phantom{0}7\sfrac{1}{3}           & \sfrac{2}{3} + 6\sfrac{2}{3}\phantom{ 1 ×} \\
      C & 1 & 2\sfrac{1}{2} + 2\sfrac{1}{2}                     & 1 & 6 & 3 + \phantom{0}\sfrac{3}{4} = 3\sfrac{3}{4}           & 5\sfrac{1}{3} & 20\phantom{\sfrac{3}{5}}           & 2\sfrac{5}{8}            & 14\phantom{\sfrac{3}{5}}           & \sfrac{2}{3} + 2 × 6\sfrac{2}{3}\\
      D & 1 & 2\sfrac{1}{2} + 2\sfrac{1}{2}                     & 1 & 6 & 4 + 1\phantom{\sfrac{0}{0}} = 5\phantom{\sfrac{0}{0}} & 5\sfrac{1}{3} & 26\sfrac{2}{3}                     & 3\sfrac{7}{8}            & 20\sfrac{2}{3}                     & \sfrac{2}{3} + 3 × 6\sfrac{2}{3}\\

     \cmidrule(r){5-5}
     \cmidrule(r){6-6}
     \cmidrule(r){8-8}
     \cmidrule(r){9-9}
     \cmidrule(r){10-10}

      Разом & & & & 30 & \phantom{2 + 1\sfrac{1}{2} =}13\sfrac{5}{8} & & 72\sfrac{2}{3} & 8\phantom{\sfrac{1}{2}} & 42\sfrac{2}{3} & \\
  \end{tabular}

  \end{center}
\end{table}

Приєднанням землі \emph{а} породжується нову диференційну ренту І; на цій
новій основі розвивається потім диференційна рента II теж у зміненому вигляді.
Земля \emph{а} має в кожній з трьох вищенаведених таблиць ріжну родючість; ряд
відповідно висхідних ступенів родючости починається лише з А. Відповідно до
цього розміщується і ряд висхідних рент. Рента з найгіршої рентодайної землі,
що раніш ренти не давала, становить постійну величину, яка просто приєднується
до всіх вищих рент; лише за вирахуванням цієї сталої величини ясно виступає
при порівнянні вищих рент ряд ріжниць і його паралелізм з рядом, що
визначає родючість різних земель. У всіх таблицях різні ступені родючости, починаючи
з А до D, стосуються один до одного, як 1: 2 : 3 : 4, і відповідно до
цього ренти стосуються одна до однієї:

\begin{tabular}{l}
в VIIa, як 1 : 1 + 7 : 1 + 2 × 7 : 1 + 3 × 7,\\
в VIIIa, як 1\sfrac{1}{5}:1\sfrac{1}{5} + 7\sfrac{1}{5} : 1\sfrac{1}{5} + 2 × 7\sfrac{1}{5} : 1\sfrac{1}{5} + 3 × 7\sfrac{1}{5},\\
в Xa, як \sfrac{2}{3} : \sfrac{2}{3} + 6 \sfrac{2}{3} : \sfrac{2}{3} + 2 × 6 \sfrac{2}{3} : \sfrac{2}{3} + 3 × 6\sfrac{2}{3}.\\
\end{tabular}

Коротко: коли рента з А = n, а рента з землі безпосередньо вищої родючости
$= n + m$, то ряд буде такий: $n: n + m: n + 2m : n + З m$ і т. д. — Ф. Е.]

\pfbreak

[А що вищенаведений третій випадок в рукопису не був опрацьований —
там є лише його заголовок, — то завдання редактора було по змозі доповнити
це, як зроблено вище. Але йому лишається ще зробити загальні висновки, що
випливають з усього попереднього дослідження диференційної ренти II в її трьох
головних випадках і дев’ятьох похідних випадках. Але для цієї мети наведені
в рукопису випадки придаються лише дуже мало. По-перше, в них порівнюються
дільниці землі, що з них здобутки для площ однакової величини стосуються
як 1: 2 : 3 : 4; отже, беруться ріжниці, що вже від самого початку дуже перебільшені,
і які в дальшому розвитку зроблених на цій основі припущень і обчислень
призводять до цілком насильницьких числових відношень. Але подруге,
\parbreak{}  %% абзац продовжується на наступній сторінці


\index{i}{0168}  %% посилання на сторінку оригінального видання
З цих 4 фунтів пряжі, в яких таким чином існує вся новоспродукована вартість денного процесу
прядіння, одна половина репрезентує лише еквівалент вартости спожитої робочої сили, тобто змінний капітал у 3\shil{ шилінґи}, другі
2 фунти пряжі — лише додаткову вартість у 3\shil{ шилінґи.}

А що 12 робочих годин прядуна упредметнюються в 6\shil{ шилінґах} вартости, то в 30\shil{ шилінґах} вартости пряжі
упредметнюються 60 робочих годин. Вони існують у 20 фунтах пряжі, з яких \sfrac{8}{10},
або 16 фунтів, є матеріялізація тих 48 робочих годин, що минули перед процесом прядіння, а саме,
тієї праці, що упредметнена в засобах продукції пряжі; навпаки, \sfrac{2}{10}, або 4 фунти пряжі, є
матеріялізація 12 робочих годин, витрачених у самому процесі прядіння.

Раніш ми бачили, що вартість пряжі дорівнює сумі нової вартости, створеної в процесі продукції
пряжі, плюс вартості, які вже перед тим існували в засобах її продукції. Тепер виявилось, яким чином функціонально або в
понятті різні складові
частини вартости продукту можуть бути виражені у відносних частинах самого продукту.

Цей розклад продукту — результату процесу продукції — на кількість продукту, що репрезентує лише
працю, вміщену в
засобах продукції, або сталу частину капіталу, далі на іншу кількість, що репрезентує лише додану в
процесі продукції доконечну працю, або змінну частину капіталу, і, нарешті, на ту
кількість, що репрезентує лише додаткову працю, додану в самому процесі, або додаткову вартість, —
цей розклад є так само простий, як і важливий, як це покаже пізніше застосовування його в заплутаних
і не розв’язаних іще проблемах.

Ми щойно розглядали ввесь продукт як готовий результат дванадцятигодинного робочого дня. Але ми
можемо простежити
його і в процесі його постання, і все ж таки частинні продукти виразити як функціонально відмінні
частини продукту.

Прядун продукує за 12 годин 20 фунтів пряжі, значить, за одну годину 1\sfrac{2}{3} фунта і за 8 годин 13\sfrac{1}{3}
фунтів, отже, частинний
продукт, вартість якого дорівнює цілій вартості бавовни, що її випрядається протягом цілого робочого
дня. Так само частинний продукт дальших 1 години 36 хвилин дорівнює 2\sfrac{2}{3} фунтів пряжі,
і тому репрезентує вартість засобів праці, зужиткованих протягом 12 робочих годин. Так само й за
дальші 1 годину 12 хвилин
прядун продукує 2 фунти пряжі \deq{} 3\shil{ шилінґам} — вартість продукту, що дорівнює тій цілій
новоспродукованій вартості, що її він створює за 6 годин доконечної праці. Нарешті, за останні \sfrac{6}{5}
годин він продукує так само 2 фунти пряжі, що їхня вартість дорівнює додатковій вартості, створеній
його полуденною додатковою працею. Цей спосіб обчислення служить англійському фабрикантові для домашнього вжитку, і він скаже,
приміром, що за перші 8 годин, або \sfrac{2}{3} робочого дня, він одержує назад вартість своєї бавовни
й~\abbr{т. ін.} Ми бачимо, що ця формула правильна, що фактично це лише перша формула, перенесена з
простору, де готові частини
\parbreak{}  %% абзац продовжується на наступній сторінці


\index{iii2}{0169}  %% посилання на сторінку оригінального видання
Варіянт III: Висхідна продуктивність другої витрати (таблиця XXI); це знов
таки зумовлює низхідну продуктивність першої витрати.

Друга видозміна: земля гіршої якости, (позначувана: літерою а)
вступає в конкуренцію; земля А дає ренту.

Варіянт 1: Незмінна продуктивність другої витрати (таблиця XXII).

Варіант 2: Низхідна продуктивність (таблиця XXIII).

Варіант 3: Висхідна продуктивність (таблиця XXIV).

Ці три варіянти відповідають загальним умовам проблеми і не дають
приводу до будь-яких зауважень.

Тепер ми наведемо таблиці:

\begin{table}[h]
  \begin{center}
    \emph{Таблиця XI}
    \footnotesize

  \begin{tabular}{c@{  } c@{  } c@{  } c@{  } c@{  } c@{  } c}
    \toprule
      \multirowcell{2}{\makecell{Рід\\ землі}} &
      Ціна продукції &
      Продукт &
      \makecell{Продажна \\ ціна} &
      \makecell{Здо-\\буток} &
      Рента &
      \multirowcell{2}{Підвищення ренти} \\

      \cmidrule(r){2-2}
      \cmidrule(r){3-3}
      \cmidrule(r){4-4}
      \cmidrule(r){5-5}
      \cmidrule(r){6-6}

       & Шил. & Бушелі & Шил. & Шил. & Шил. & &   \\
      \midrule
      A & 60 & 10 & 6 & 60  & \phantom{00}0 & \phantom{00 × 0}0 \\
      B & 60 & 12 & 6 & 72  & \phantom{0}12 & \phantom{01 × }12 \\
      C & 60 & 14 & 6 & 84  & \phantom{0}24 & \phantom{0}2 × 12           \\
      D & 60 & 16 & 6 & 96  & \phantom{0}36 & \phantom{0}3 × 12           \\
      E & 60 & 18 & 6 & 108 & \phantom{0}48 & \phantom{0}4 × 12           \\

     \cmidrule(r){6-6}
     \cmidrule(r){7-7}

      & & & & & 120 & 10 × 12 \\
  \end{tabular}

  \end{center}
\end{table}

За другої витрати капіталу на тій самій землі.

Перший випадок: за незмінної ціни продукції.

Варіянт 1: за незмінної продуктивности другої витрати капіталу.

\begin{table}[h]
  \begin{center}
    \emph{Таблиця XII}
    \footnotesize

  \begin{tabular}{c@{  } c@{  } c@{  } c@{  } c@{  } c@{  } c}
    \toprule
      \multirowcell{2}{\makecell{Рід\\ землі}} &
      Ціна продукції &
      Продукт &
      \makecell{Продажна \\ ціна} &
      \makecell{Здо-\\буток} &
      Рента &
      \multirowcell{2}{Підвищення ренти} \\

      \cmidrule(r){2-2}
      \cmidrule(r){3-3}
      \cmidrule(r){4-4}
      \cmidrule(r){5-5}
      \cmidrule(r){6-6}

       & Шил. & Бушелі & Шил. & Шил. & Шил. & &   \\
      \midrule
      A & 60 + 60 = 120 & 10 + 10 = 20 & 6 & 120  & \phantom{00}0 & \phantom{00 × 0}0 \\
      B & 60 + 60 = 120 & 12 + 12 = 24 & 6 & 144  & \phantom{0}24 & \phantom{01 × }24 \\
      C & 60 + 60 = 120 & 14 + 14 = 28 & 6 & 168  & \phantom{0}48 & \phantom{0}2 × 24 \\
      D & 60 + 60 = 120 & 16 + 16 = 32 & 6 & 192  & \phantom{0}72 & \phantom{0}3 × 24 \\
      E & 60 + 60 = 120 & 18 + 18 = 36 & 6 & 216  & \phantom{0}96 & \phantom{0}4 × 24 \\

     \cmidrule(r){6-6}
     \cmidrule(r){7-7}

      & & & & & 240 & 10 × 24 \\
  \end{tabular}

  \end{center}
\end{table}

Варіянт 2: за низхідної продуктивности другої витрати капіталу: на землі
А не зроблено другої витрати.

1) Коли земля В стає землею, що не дає ренти.


\index{iii2}{0170}  %% посилання на сторінку оригінального видання

\begin{table}[h]
  \begin{center}
    \emph{Таблиця XIII}
    \footnotesize

  \begin{tabular}{c@{  } c@{  } c@{  } c@{  } c@{  } c@{  } c}
    \toprule
      \multirowcell{2}{\makecell{Рід\\ землі}} &
      Ціна продукції &
      Продукт &
      \makecell{Продажна \\ ціна} &
      \makecell{Здо-\\буток} &
      Рента &
      \multirowcell{2}{Підвищення ренти} \\

      \cmidrule(r){2-2}
      \cmidrule(r){3-3}
      \cmidrule(r){4-4}
      \cmidrule(r){5-5}
      \cmidrule(r){6-6}

       & Шил. & Бушелі & Шил. & Шил. & Шил. & &   \\
      \midrule
      A & \phantom{60 + 60 = 0}60 & \phantom{12 + 10\sfrac{1}{3} =} 10\phantom{\sfrac{2}{3}}           & 6 & \phantom{0}60 & \phantom{00}0 & \phantom{0 × 0}0 \\
      B & 60 + 60 = 120           & 12 + \phantom{0}8\phantom{\sfrac{1}{3}} = 20\phantom{\sfrac{2}{3}} & 6 & 120           & \phantom{00}0 & \phantom{0 × 0}0 \\
      C & 60 + 60 = 120           & 14 + \phantom{0}9\sfrac{1}{3} = 23\sfrac{1}{3}                     & 6 & 140           & \phantom{0}20 & \phantom{1 × }20 \\
      D & 60 + 60 = 120           & 16 + 10\sfrac{2}{3} = 26\sfrac{2}{3}                               & 6 & 160           & \phantom{0}40 & 2 × 20 \\
      E & 60 + 60 = 120           & 18 + 12\footnotemarkZ{}\phantom{/}= 30\phantom{\sfrac{2}{3}}                  & 6 & 180           & \phantom{0}60 & 3 × 20 \\

     \cmidrule(r){6-6}
     \cmidrule(r){7-7}

      & & & & & 120 & 6 × 20 \\
  \end{tabular}

  \end{center}
\end{table}
\footnotetextZ{ В німецькому тексті тут стоїть «20». Очевидна помилка. Прим. Ред.} % текст примітки прямо під заголовком

2) Кола земля В не стає землею, що зовсім не дає ренти.

\begin{table}[h]
  \begin{center}
    \emph{Таблиця XIV}
    \footnotesize

  \begin{tabular}{c@{  } c@{  } c@{  } c@{  } c@{  } c@{  } c}
    \toprule
      \multirowcell{2}{\makecell{Рід\\ землі}} &
      Ціна продукції &
      Продукт &
      \makecell{Продажна \\ ціна} &
      \makecell{Здо-\\буток} &
      Рента &
      \multirowcell{2}{Підвищення ренти} \\

      \cmidrule(r){2-2}
      \cmidrule(r){3-3}
      \cmidrule(r){4-4}
      \cmidrule(r){5-5}
      \cmidrule(r){6-6}

       & Шил. & Бушелі & Шил. & Шил. & Шил. & &   \\
      \midrule
      A & \phantom{60 + 60 = 0}60 & \phantom{12 + 10\sfrac{1}{3} =} 10\phantom{\sfrac{2}{3}}           & 6 & \phantom{0}60 & \phantom{00}0 & \phantom{4 ×}0\phantom{ + 3 × 21}\\
      B & 60 + 60 = 120           & 12 + \phantom{0}9\phantom{\sfrac{1}{3}} = 21\phantom{\sfrac{2}{3}} & 6 & 126           & \phantom{00}6 & \phantom{4 ×}6\phantom{ + 3 × 21}\\
      C & 60 + 60 = 120           & 14 + 10\sfrac{1}{2} = 24\sfrac{1}{2}                               & 6 & 147           & \phantom{0}27 & \phantom{4 ×}6 + 21\phantom{1 × } \\
      D & 60 + 60 = 120           & 16 + 12\phantom{\sfrac{2}{3}} = 28\phantom{\sfrac{2}{3}}           & 6 & 168           & \phantom{0}48 & \phantom{4 ×}6 + 2 × 21 \\
      E & 60 + 60 = 120           & 18 + 13\sfrac{1}{2}= 31\sfrac{1}{2}                                & 6 & 189           & \phantom{0}69 & \phantom{4 ×}6 + 3 × 21 \\

     \cmidrule(r){6-6}
     \cmidrule(r){7-7}

      & & & & & 150 & 4 × 6 + 6 × 21 \\
  \end{tabular}

  \end{center}
\end{table}

Варіянт 3: за висхідної продуктивности другої витрати капіталу; на землі
А тут теж не робиться другої витрати.

\begin{table}[h]
  \begin{center}
    \emph{Таблиця XV}
    \footnotesize

  \begin{tabular}{c@{  } c@{  } c@{  } c@{  } c@{  } c@{  } c}
    \toprule
      \multirowcell{2}{\makecell{Рід\\ землі}} &
      Ціна продукції &
      Продукт &
      \makecell{Продажна \\ ціна} &
      \makecell{Здо-\\буток} &
      Рента &
      \multirowcell{2}{Підвищення ренти} \\

      \cmidrule(r){2-2}
      \cmidrule(r){3-3}
      \cmidrule(r){4-4}
      \cmidrule(r){5-5}
      \cmidrule(r){6-6}

       & Шил. & Бушелі & Шил. & Шил. & Шил. & &   \\
      \midrule
      A & \phantom{60 + 60 = 0}60 & \phantom{12 + 10\sfrac{1}{3} =} 10\phantom{\sfrac{2}{3}}           & 6 & \phantom{0}60 & \phantom{00}0 & \phantom{4 ×0}0\phantom{ + 3 × 27}\\
      B & 60 + 60 = 120           & 12 + 15\phantom{\sfrac{1}{3}} = 27\phantom{\sfrac{2}{3}}           & 6 & 162           & \phantom{0}42 & \phantom{4 ×}42\phantom{ + 3 × 27}\\
      C & 60 + 60 = 120           & 14 + 17\sfrac{1}{2} = 31\sfrac{1}{2}                               & 6 & 189           & \phantom{0}69 & \phantom{4 ×}42 + 27\phantom{1 × } \\
      D & 60 + 60 = 120           & 16 + 20\phantom{\sfrac{2}{3}} = 36\phantom{\sfrac{2}{3}}           & 6 & 216           & \phantom{0}96 & \phantom{4 ×}42 + 2 × 27 \\
      E & 60 + 60 = 120           & 18 + 22\sfrac{1}{2}= 40\sfrac{1}{2}                                & 6 & 243           & 123           & \phantom{4 ×}42 + 3 × 27 \\

     \cmidrule(r){6-6}
     \cmidrule(r){7-7}

      & & & & & 330 & 4 × 42 + 6 × 27 \\
  \end{tabular}

  \end{center}
\end{table}


\index{iii2}{0171}  %% посилання на сторінку оригінального видання
Другий випадок за низхідної ціни продукції.

Варіянт 1: за незмінної продуктивности другої витрати капіталу: земля
А випадає з конкуренції, земля В стає землею, що не дає ренти.

\begin{table}[h]
  \begin{center}
    \emph{Таблиця XVI}
    \footnotesize

  \begin{tabular}{c@{  } c@{  } c@{  } c@{  } c@{  } c@{  } c}
    \toprule
      \multirowcell{2}{\makecell{Рід\\ землі}} &
      Ціна продукції &
      Продукт &
      \makecell{Продажна \\ ціна} &
      \makecell{Здо-\\буток} &
      Рента &
      \multirowcell{2}{Підвищення ренти} \\

      \cmidrule(r){2-2}
      \cmidrule(r){3-3}
      \cmidrule(r){4-4}
      \cmidrule(r){5-5}
      \cmidrule(r){6-6}

       & Шил. & Бушелі & Шил. & Шил. & Шил. &  \\
      \midrule
      B & 60 + 60 = 120 & 12 + 12 = 24 & 5 & 120  & \phantom{00}0 & \phantom{01 × }0 \\
      C & 60 + 60 = 120 & 14 + 14 = 28 & 5 & 140  & \phantom{0}20 & \phantom{1 ×} 20 \\
      D & 60 + 60 = 120 & 16 + 16 = 32 & 5 & 160  & \phantom{0}40 & 2 × 20 \\
      E & 60 + 60 = 120 & 18 + 18 = 36 & 5 & 180  & \phantom{0}60 & 3 × 20 \\

     \cmidrule(r){6-6}
     \cmidrule(r){7-7}

      & & & & & 120 & 6 × 20 \\
  \end{tabular}

  \end{center}
\end{table}

Варіант 2: за низхідної продуктивности другої витрати капіталу; земля
А випадає з конкуренції, земля В стає землею, що не дає ренти.

\begin{table}[h]
  \begin{center}
    \emph{Таблиця XVII}
    \footnotesize

  \begin{tabular}{c@{  } c@{  } c@{  } c@{  } c@{  } c@{  } c}
    \toprule
      \multirowcell{2}{\makecell{Рід\\ землі}} &
      Ціна продукції &
      Продукт &
      \makecell{Продажна \\ ціна} &
      \makecell{Здо-\\буток} &
      Рента &
      \multirowcell{2}{Підвищення ренти} \\

      \cmidrule(r){2-2}
      \cmidrule(r){3-3}
      \cmidrule(r){4-4}
      \cmidrule(r){5-5}
      \cmidrule(r){6-6}

       & Шил. & Бушелі & Шил. & Шил. & Шил. &  \\
      \midrule
      B & 60 + 60 = 120 & 12 + \phantom{0}9\phantom{\sfrac{1}{2}} = 21\phantom{\sfrac{1}{2}} & 5\sfrac{5}{7} & 120  & \phantom{00}0 & \phantom{01 × }0 \\
      C & 60 + 60 = 120 & 14 + 10\sfrac{1}{2} = 24\sfrac{1}{2}                               & 5\sfrac{5}{7} & 140  & \phantom{0}20 & \phantom{1 ×} 20 \\
      D & 60 + 60 = 120 & 16 + 12\phantom{\sfrac{1}{2}} = 28\phantom{\sfrac{1}{2}}           & 5\sfrac{5}{7} & 160  & \phantom{0}40 & 2 × 20 \\
      E & 60 + 60 = 120 & 18 + 13\sfrac{1}{2} = 31\sfrac{1}{2}                               & 5\sfrac{5}{7} & 180  & \phantom{0}60 & 3 × 20 \\

     \cmidrule(r){6-6}
     \cmidrule(r){7-7}

      & & & & & 120 & 6 × 20 \\
  \end{tabular}

  \end{center}
\end{table}

Варіант 3: за висхідної продуктивности другої витрати капіталу; земля
А залишається конкурентною. Земля В дає ренту.

\begin{table}[h]
  \begin{center}
    \emph{Таблиця XVIII}
    \footnotesize

  \begin{tabular}{c@{  } c@{  } c@{  } c@{  } c@{  } c@{  } c}
    \toprule
      \multirowcell{2}{\makecell{Рід\\ землі}} &
      Ціна продукції &
      Продукт &
      \makecell{Продажна \\ ціна} &
      \makecell{Здо-\\буток} &
      Рента &
      \multirowcell{2}{Підвищення ренти} \\

      \cmidrule(r){2-2}
      \cmidrule(r){3-3}
      \cmidrule(r){4-4}
      \cmidrule(r){5-5}
      \cmidrule(r){6-6}

       & Шил. & Бушелі & Шил. & Шил. & Шил. &   \\
      \midrule
      A & 60 + 60 = 120 & 10 + 15 = 25 & 4\sfrac{4}{5} & 120  & \phantom{00}0 & \phantom{00 × 0}0 \\
      B & 60 + 60 = 120 & 12 + 18 = 30 & 4\sfrac{4}{5} & 144  & \phantom{0}24 & \phantom{01 × }24 \\
      C & 60 + 60 = 120 & 14 + 21 = 35 & 4\sfrac{4}{5} & 168  & \phantom{0}48 & \phantom{0}2 × 24 \\
      D & 60 + 60 = 120 & 16 + 24 = 40 & 4\sfrac{4}{5} & 192  & \phantom{0}72 & \phantom{0}3 × 24 \\
      E & 60 + 60 = 120 & 18 + 27 = 45 & 4\sfrac{4}{5} & 216  & \phantom{0}96 & \phantom{0}4 × 24 \\

     \cmidrule(r){6-6}
     \cmidrule(r){7-7}

      & & & & & 240 & 10 × 24 \\
  \end{tabular}

  \end{center}
\end{table}


\index{iii2}{0172}  %% посилання на сторінку оригінального видання
Третій випадок: За висхідної ціни продукції.

А. Коли земля А не дає ренти й продовжує реґулювати ціну.

Варіянт 1: За незмінної продуктивности другої витрати капіталу, що зумовлює
низхідну продуктивність першої витрати.

 % ця мітка у заголовку
 % текст примітки прямо під заголовком

\begin{table}[h]
  \begin{center}
    \emph{Таблиця XIX\footnotemarkZ{}}
    \footnotesize

  \begin{tabular}{c@{  } c@{  } c@{  } c@{  } c@{  } c@{  } c}
    \toprule
      \multirowcell{2}{\makecell{Рід\\ землі}} &
      Ціна продукції &
      Продукт &
      \makecell{Продажна \\ ціна} &
      \makecell{Здо-\\буток} &
      Рента &
      \multirowcell{2}{Підвищення ренти} \\

      \cmidrule(r){2-2}
      \cmidrule(r){3-3}
      \cmidrule(r){4-4}
      \cmidrule(r){5-5}
      \cmidrule(r){6-6}

       & Шил. & Бушелі & Шил. & Шил. & Шил. &  \\
      \midrule
      A & 60 + 60 = 120 & 5 + 12\sfrac{1}{2} = 17\sfrac{1}{2}                      & 6\sfrac{6}{7} & 120  & \phantom{00}0 & \phantom{01 × }0 \\
      B & 60 + 60 = 120 & 6 + 15\phantom{\sfrac{1}{2}} = 21\phantom{\sfrac{1}{2}}  & 6\sfrac{6}{7} & 144  & \phantom{0}24 & \phantom{1 ×} 24 \\
      C & 60 + 60 = 120 & 7 + 17\sfrac{1}{2} = 24\sfrac{1}{2}                      & 6\sfrac{6}{7} & 168  & \phantom{0}48 & 2 × 24 \\
      D & 60 + 60 = 120 & 8 + 20\phantom{\sfrac{1}{2}} = 28\phantom{\sfrac{1}{2}}  & 6\sfrac{6}{7} & 192  & \phantom{0}72 & 3 × 24 \\
      E & 60 + 60 = 120 & 9 + 22\sfrac{1}{2} = 31\sfrac{1}{2}                      & 6\sfrac{6}{7} & 216  & \phantom{0}96 & 4 × 24 \\

     \cmidrule(r){6-6}
     \cmidrule(r){7-7}

      & & & & & 240 & 10 × 24 \\
  \end{tabular}

  \end{center}
\end{table}

\footnotetextZ{Це є таблиця висхідної продуктивности другої витрати капіталу. Порівн. табл. XXI. \emph{Прим. Ред}}

Варіянт 2: За низхідної продуктивности другої витрати капіталу, що не виключає
незмінюваної продуктивности першої витрати.

\begin{table}[h]
  \begin{center}
    \emph{Таблиця XX}
    \footnotesize

  \begin{tabular}{c@{  } c@{  } c@{  } c@{  } c@{  } c@{  } c}
    \toprule
      \multirowcell{2}{\makecell{Рід\\ землі}} &
      Ціна продукції &
      Продукт &
      \makecell{Продажна \\ ціна} &
      \makecell{Здо-\\буток} &
      Рента &
      \multirowcell{2}{Підвищення ренти} \\

      \cmidrule(r){2-2}
      \cmidrule(r){3-3}
      \cmidrule(r){4-4}
      \cmidrule(r){5-5}
      \cmidrule(r){6-6}

       & Шил. & Бушелі & Шил. & Шил. & Шил. &  \\
      \midrule
      A & 60 + 60 = 120 & 10 + 5 = 15  & 8 & 120  & \phantom{00}0 & \phantom{01 × }0 \\
      B & 60 + 60 = 120 & 12 + 6 = 18  & 8 & 144  & \phantom{0}24 & \phantom{1 ×} 24 \\
      C & 60 + 60 = 120 & 14 + 7 = 21  & 8 & 168  & \phantom{0}48 & 2 × 24 \\
      D & 60 + 60 = 120 & 16 + 8 = 24  & 8 & 192  & \phantom{0}72 & 3 × 24 \\
      E & 60 + 60 = 120 & 18 + 9 = 27  & 8 & 216  & \phantom{0}96 & 4 × 24 \\

     \cmidrule(r){6-6}
     \cmidrule(r){7-7}

      & & & & & 240 & 10 × 24 \\
  \end{tabular}

  \end{center}
\end{table}

Варіянт 3: За висхідної продуктивности другої витрати капіталу, що, за даних
припущень, обумовлює низхідну продуктивність першої витрати.

\begin{table}[h]
  \begin{center}
    \emph{Таблиця XXI}
    \footnotesize

  \begin{tabular}{c@{  } c@{  } c@{  } c@{  } c@{  } c@{  } c}
    \toprule
      \multirowcell{2}{\makecell{Рід\\ землі}} &
      Ціна продукції &
      Продукт &
      \makecell{Продажна \\ ціна} &
      \makecell{Здо-\\буток} &
      Рента &
      \multirowcell{2}{Підвищення ренти} \\

      \cmidrule(r){2-2}
      \cmidrule(r){3-3}
      \cmidrule(r){4-4}
      \cmidrule(r){5-5}
      \cmidrule(r){6-6}

       & Шил. & Бушелі & Шил. & Шил. & Шил. &  \\
      \midrule
      A & 60 + 60 = 120 & 5 + 12\sfrac{1}{2} = 17\sfrac{1}{2}                      & 6\sfrac{6}{7} & 120  & \phantom{00}0 & \phantom{01 × }0 \\
      B & 60 + 60 = 120 & 6 + 15\phantom{\sfrac{1}{2}} = 21\phantom{\sfrac{1}{2}}  & 6\sfrac{6}{7} & 144  & \phantom{0}24 & \phantom{1 ×} 24 \\
      C & 60 + 60 = 120 & 7 + 17\sfrac{1}{2} = 24\sfrac{1}{2}                      & 6\sfrac{6}{7} & 168  & \phantom{0}48 & 2 × 24 \\
      D & 60 + 60 = 120 & 8 + 20\phantom{\sfrac{1}{2}} = 28\phantom{\sfrac{1}{2}}  & 6\sfrac{6}{7} & 192  & \phantom{0}72 & 3 × 24 \\
      E & 60 + 60 = 120 & 9 + 22\sfrac{1}{2} = 31\sfrac{1}{2}                      & 6\sfrac{6}{7} & 216  & \phantom{0}96 & 4 × 24 \\

     \cmidrule(r){6-6}
     \cmidrule(r){7-7}

      & & & & & 240 & 10 × 24 \\
  \end{tabular}

  \end{center}
\end{table}


\index{iii2}{0173}  %% посилання на сторінку оригінального видання
В. Коли гірша (позначувана літерою а) земля стає землею, яка реґулює
ціну і через те А починає давати ренту. Де не виключає можливости незмінюваної
продуктивности другої витрати для всіх варіянтів.

Варіянт 1: Незмінювана продуктивність другої витрати капіталу.

\begin{table}[h]
  \begin{center}
    \emph{Таблиця XXII}
    \footnotesize

  \begin{tabular}{c@{  } c@{  } c@{  } c@{  } c@{  } c@{  } c}
    \toprule
      \multirowcell{2}{\makecell{Рід\\ землі}} &
      Ціна продукції &
      Продукт &
      \makecell{Продажна \\ ціна} &
      \makecell{Здо-\\буток} &
      Рента &
      \multirowcell{2}{Підвищення ренти} \\

      \cmidrule(r){2-2}
      \cmidrule(r){3-3}
      \cmidrule(r){4-4}
      \cmidrule(r){5-5}
      \cmidrule(r){6-6}

       & Шил. & Бушелі & Шил. & Шил. & Шил. &  \\
      \midrule
      a & \phantom{60 + 60 = }120 & \phantom{10 + 10 = }16 & 7\sfrac{1}{2} & 120  & \phantom{00}0  & \phantom{01 × }0 \\
      A & 60 + 60 = 120           & 10 + 10 = 20            & 7\sfrac{1}{2} & 150  & \phantom{0}30 & \phantom{1 ×} 30 \\
      B & 60 + 60 = 120           & 12 + 12 = 24            & 7\sfrac{1}{2} & 180  & \phantom{0}60 & 2 × 30 \\
      C & 60 + 60 = 120           & 14 + 14 = 28            & 7\sfrac{1}{2} & 210  & \phantom{0}90 & 3 × 30 \\
      D & 60 + 60 = 120           & 16 + 16 = 32            & 7\sfrac{1}{2} & 240  & 120           & 4 × 30 \\
      E & 60 + 60 = 120           & 18 + 18 = 36            & 7\sfrac{1}{2} & 270  & 150           & 5 × 30 \\

     \cmidrule(r){6-6}
     \cmidrule(r){7-7}

      & & & & & 450 & 15 × 30 \\
  \end{tabular}

  \end{center}
\end{table}

Варіянт 2: Низхідна продуктивність другої витрати капіталу.

\begin{table}[h]
  \begin{center}
    \emph{Таблиця XXIII}
    \footnotesize

  \begin{tabular}{c@{  } c@{  } c@{  } c@{  } c@{  } c@{  } c}
    \toprule
      \multirowcell{2}{\makecell{Рід\\ землі}} &
      Ціна продукції &
      Продукт &
      \makecell{Продажна \\ ціна} &
      \makecell{Здо-\\буток} &
      Рента &
      \multirowcell{2}{Підвищення ренти} \\

      \cmidrule(r){2-2}
      \cmidrule(r){3-3}
      \cmidrule(r){4-4}
      \cmidrule(r){5-5}
      \cmidrule(r){6-6}

       & Шил. & Бушелі & Шил. & Шил. & Шил. &  \\
      \midrule
      a & \phantom{60 + 60 = }120 & \phantom{10 + 10\sfrac{1}{2} = }15\phantom{\sfrac{1}{2}}  & 8 & 120 & \phantom{00}0 & \phantom{5 × 0}0 \phantom{+ 01 × 28} \\
      A & 60 + 60 = 120           & 10 + \phantom{0}7\sfrac{1}{2} = 17\sfrac{1}{2}                       & 8 & 140 & \phantom{0}20 & \phantom{5 × }20 \phantom{+ 01 × 28} \\
      B & 60 + 60 = 120           & 12 + \phantom{0}9\phantom{\sfrac{1}{2}} = 21\phantom{\sfrac{1}{2}}   & 8 & 168 & \phantom{0}48 & \phantom{5 × }20 + \phantom{01 × }28\\
      C & 60 + 60 = 120           & 14 + 10\sfrac{1}{2} = 24\sfrac{1}{2}                      & 8 & 194 & \phantom{0}76 & \phantom{5 × }20 + \phantom{0}2 × 28 \\
      D & 60 + 60 = 120           & 16 + 12\phantom{\sfrac{1}{2}} = 28\phantom{\sfrac{1}{2}}  & 8 & 224 & 104           & \phantom{5 × }20 + \phantom{0}3 × 28 \\
      E & 60 + 60 = 120           & 18 + 13\sfrac{1}{2} = 31\sfrac{1}{2}                      & 8 & 252 & 132           & \phantom{5 × }20 + \phantom{0}4 × 28 \\

     \cmidrule(r){6-6}
     \cmidrule(r){7-7}

      & & & & & 380 & 5 × 20 + 10 × 28 \\
  \end{tabular}

  \end{center}
\end{table}

Варіянт 3: Висхідна продуктивність другої витрати капіталу.

\begin{table}[h]
  \begin{center}
    \emph{Таблиця XXIV}
    \footnotesize

  \begin{tabular}{c@{  } c@{  } c@{  } c@{  } c@{  } c@{  } c}
    \toprule
      \multirowcell{2}{\makecell{Рід\\ землі}} &
      Ціна продукції &
      Продукт &
      \makecell{Продажна \\ ціна} &
      \makecell{Здо-\\буток} &
      Рента &
      \multirowcell{2}{Підвищення ренти} \\

      \cmidrule(r){2-2}
      \cmidrule(r){3-3}
      \cmidrule(r){4-4}
      \cmidrule(r){5-5}
      \cmidrule(r){6-6}

       & Шил. & Бушелі & Шил. & Шил. & Шил. &  \\
      \midrule
      a & \phantom{60 + 60 = }120 & \phantom{10 + 10\sfrac{1}{2} = }16\phantom{\sfrac{1}{2}}  & 7\sfrac{1}{2} & 120\phantom{\sfrac{1}{2}} & \phantom{00}0\phantom{\sfrac{1}{2}} & \phantom{5 × 15 + 15 × }0\phantom{\sfrac{3}{4}} \\
      A & 60 + 60 = 120           & 10 + 12\sfrac{1}{2} = 22\sfrac{1}{2}                      & 7\sfrac{1}{2} & 168\sfrac{3}{4}           & \phantom{0}48\sfrac{3}{4}           & \phantom{5 × }15 + \phantom{1 × }33\sfrac{3}{4} \\
      B & 60 + 60 = 120           & 12 + 15\phantom{\sfrac{1}{2}} = 27\phantom{\sfrac{1}{2}}  & 7\sfrac{1}{2} & 202\sfrac{1}{2}           & \phantom{0}82\sfrac{1}{2}           & \phantom{5 × }15 + 2 × 33\sfrac{3}{4} \\
      C & 60 + 60 = 120           & 14 + 17\sfrac{1}{2} = 31\sfrac{1}{2}                      & 7\sfrac{1}{2} & 236\sfrac{1}{4}           & 116\sfrac{1}{4}                     & \phantom{5 × }15 + 3 × 33\sfrac{3}{4} \\
      D & 60 + 60 = 120           & 16 + 20\phantom{\sfrac{1}{2}} = 36\phantom{\sfrac{1}{2}}  & 7\sfrac{1}{2} & 270\phantom{\sfrac{1}{2}} & 150\phantom{\sfrac{1}{2}}           & \phantom{5 × }15 + 4 × 33\sfrac{3}{4} \\
      E & 60 + 60 = 120           & 18 + 22\sfrac{1}{2} = 40\sfrac{1}{2}                      & 7\sfrac{1}{2} & 303\sfrac{3}{4}           & 183\sfrac{3}{4}                     & \phantom{5 × }15 + 5 × 33\sfrac{3}{4} \\

     \cmidrule(r){6-6}
     \cmidrule(r){7-7}

      & & & & & 581\sfrac{3}{4} & 5 × 15 + 15 × 33\sfrac{3}{4} \\
  \end{tabular}

  \end{center}
\end{table}

\parbreak{}  %% абзац продовжується на наступній сторінці

\parcont{}  %% абзац починається на попередній сторінці
\index{i}{0174}  %% посилання на сторінку оригінального видання
знайомства з вами в кращому світі. Addio\elli{!..}\footnote{
Однак пан професор мав деяку користь із своєї подорожі до Менчестеру.
В «Letters on the Factory Act» увесь чистий прибуток, «зиск» і «процент» і
навіть «something more»\footnote*{
щось більше. \emph{Ред}.
}, залежить від
однієї неоплаченої
робочої години робітника! Роком раніш у своїх «Outlines of Political Economy»,
складених для насолоди оксфордських студентів і освічених філістерів, Сеніор,
полемізуючи проти Рікардового визначення вартости робочим часом, «відкрив», що
зиск постає з праці капіталіста, а процент з його
аскетичности, з його «поздержливости». Сама побрехенька була стара, але слово
«поздержливість» («Abstinenz») було нове. Пан Рошер правильно переклав його
німецькою мовою словом «Enthaltung» («поздержливість»). А його компатріоти,
менше биті в латині, Вірти, Шульци
й інші Міхелі, переклали його на чорнече «самовідречення» («Entsagung»).
} Сиґнал «останньої години», що її винайшов Сеніор 1836~\abbr{р.}, наново протрубив
був 15 квітня 1848~\abbr{р.} в «London Economist» Джемc Вілсон, один з головних
мандаринів економічної науки, у своїй полеміці проти
закону про десятигодинний робочий день.

\manualpagebreak{}
\subsection{Додатковий продукт}

Ту частину продукту (\sfrac{1}{10} від 20 фунтів пряжі, або 2 фунти пряжі, у
прикладі §2), яка репрезентує додаткову вартість, ми
називаємо додатковим продуктом (surplus produce, produit net). Як норму
додаткової вартости визначає відношення додаткової
вартости не до цілої суми капіталу, а лише до його змінної складової частини,
так і рівень додаткового продукту визначає відношення останнього не до решти
цілого продукту, а до тієї частини його, яка репрезентує доконечну працю.
Як продукція додаткової вартости є визначальна мета
капіталістичної продукції, так і ступінь багатства вимірюється не абсолютною
величиною продукту, а відносною величиною додаткового продукту\footnote{
«Для індивіда, що має капітал у \num{20.000}\pound{ фунтів стерлінґів}, і що його зиски
становлять \num{2.000}\pound{ фунтів стерлінґів} на рік, було б цілком байдуже, чи його
капітал вживає 100 чи \num{1.000} робітників, чи випродуковані товари продається
за \num{10.000}\pound{ фунтів стерлінґів} чи за \num{20.000}\pound{ фунтів стерлінґів}, аби лише
його зиски в усіх цих випадках не падали нижче як \num{2.000}\pound{ фунтів
стерлінґів}. Хіба реальний інтерес націй не такий самий? Коли припустити, що
реальний чистий прибуток нації, її ренти й зиски лишаються однакові, то не має
найменшої ваги, чи нація складається з 10 чи
12 мільйонів людности». (\emph{Ricardo}: «The Principles of Political Economy»,
3 rd. ed, London 1821, p. 416). Задовго перед Рікардом Артур Юнґ, фанатик
додаткового продукту, взагалі язикатий, неспроможний
на будь-яку критику письменник, що його слава стоїть у зворотному відношенні
до його заслуг, сказав, між іншим: «Що за користь була б для сучасного
королівства з якоїсь цілої провінції, що в ній землю обробляли б на
староримський лад дрібні незалежні селяни, про мене хоч би
й як і найкраще? Яка мета була б у цьому, крім одним-однієї мети продукувати
людей («the mere purpose of breeding men»), а це саме по собі не має
ніякої мети» (is a most useless purpose»). (\emph{Arthur Young}: «Political
Arithmetic etc.», 1774, p. 47).
Додаток до примітки 34. Дивний є «великий нахил малювати чистий прибуток
корисним для робітничої кляси\dots{} та проте ясно, що це стається не через те,
що він чистий» («the strong inclination to
represent net wealth as beneficial to the labouring class\dots{} though it
is evidently not on account
of being net»). (\emph{Th.~Hopkins}: «On Rent of Land etс.», London 1823, p. 126).
}.


\index{i}{0175}  %% посилання на сторінку оригінального видання
Сума доконечної праці й додаткової праці, періодів часу,
протягом яких робітник продукує еквівалент вартости своєї робочої
сили й додаткову вартість, становить абсолютну величину
його робочого часу — робочий день (working day).

\section{Робочий день}
\subsection{Межі робочого дня}

Ми виходили з тієї передумови, що робочу силу купується
й продається за її вартістю. Вартість її, як і вартість кожного
іншого товару, визначається робочим часом, потрібним на
її продукцію. Отже, коли на продукцію пересічних денних засобів
існування робітника потрібно 6 годин, то він мусить працювати
пересічно 6 годин на день, щоб продукувати щоденно свою робочу
силу, або щоб репродукувати вартість, яку він одержав при її
продажу. Доконечна частина його робочого дня становить тоді
6 годин; отже, за інших незмінних обставин, вона є дана величина.
Але цим іще величину самого робочого дня не дано.

Припустимо, що лінія $a\linerule{6}b$ репрезентує тривання
або довжину доконечного робочого часу, приміром, 6 годин. Відповідно
до того, чи праця буде здовжена поза межі $ab$ на 1, 3,
6 годин і~\abbr{т. ін.}, ми матимемо три різні лінії:

\begin{table}[H]
\centering
\noindent\begin{tabular}{l}
Робочий день I \\
$a\linerule{6}b\linerule{1}c$ \\
\addlinespace
Робочий день II \\
$a\linerule{6}b\linerule{3}c$ \\
\addlinespace
Робочий день III \\
$a\linerule{6}b\linerule{6}c$ \\
\end{tabular}
\end{table}

\noindent{}які репрезентують три різні робочі дні — 7, 9 і 12 годин. Лінія
здовження $bc$ репрезентує довжину додаткової праці. А що робочий
день $\deq{} ab \dplus{} bc$, або $ac$, то він змінюється разом із змінною
величиною $bc$. Тому що $ab$ дано, то відношення $bc$ до $ab$ завжди
можна виміряти. Воно становить у робочому дні I \sfrac{1}{6}, у робочому
дні II \sfrac{3}{6} і у робочому дні III \sfrac{6}{6} $ab$. Далі, через те що, відношення
$\frac{\text{додатковий робочий час}}{\text{доконечний робочий час}}$ визначає норму додаткової вартости, то
останню дано цим відношенням. Вона становить для трьох різних
робочих днів відповідно 16\sfrac{2}{3}, 50 і 100\%. Навпаки, сама норма
додаткової вартости не дала б нам величини робочого дня. Коли
б, приміром, вона дорівнювала 100\%, то робочий день міг би
тривати 8, 10 і 12 годин і~\abbr{т. ін.} Вона показувала б, що обидві
складові частини робочого дня, доконечна праця й додаткова
праця є однаково великі, але не показувала б, яка велика кожна
з цих частин.

Отже, робочий день є не стала, а змінна величина. Правда,
одну з його частин визначається робочим часом, потрібним для
постійної репродукції самого робітника, але його ціла величина
змінюється разом з довжиною, або триванням додаткової праці.
\parbreak{}  %% абзац продовжується на наступній сторінці

\input{_0176c.tex}
\parcont{}  %% абзац починається на попередній сторінці
\index{ii}{0177}  %% посилання на сторінку оригінального видання
часом і часом продукції, то й час зуживання вкладеного основного капіталу раз-у-раз переривається на
більш-менш протяжні періоди, як, напр., у хліборобстві при вживанні робочої худоби, знарядь праці та
машин. Оскільки цей основний капітал складається з робочої худоби, він потребує завжди однакових або
майже однакових витрат на корм і~\abbr{т. ін.}, все одно, чи в роботі вона, чи не в роботі. Щодо мертвих
засобів праці, то коли їх не вживається, вони теж дещо зневартнюються. Тому продукт взагалі
дорожчає, бо передачу вартости на продукт обчислюється не на той час, коли основний капітал
функціонує, але на той час, коли він втрачає вартість. В цих галузях продукції бездіяльність
основного капіталу, хоч сполучена вона з поточними витратами, хоч ні, становить так само умову
нормального його вжитку, як, наприклад, втрата певної кількости бавовни в процесі прядіння; так само
в кожному процесі праці непродуктивна — але неминуча — витрата робочої сили, що відбувається в
нормальних технічних умовах, береться на увагу так само, як і продуктивна. Кожне поліпшення, що
зменшує непродуктивну витрату засобів праці, сировинного матеріялу та робочої сили, зменшує також і
вартість продукту.

В сільському господарстві поєднуються й порівняно довгий робочий період і велика ріжниця між робочим
часом і часом продукції. Годскін слушно зауважує про це: „Ріжниця в часі [хоч він тут і не відрізняє
робочого часу й часу продукції], потрібному на те, щоб виготовити продукти в сільському
господарстві, і тим часом, що потрібен в інших галузях праці, є головна причина великої залежности
сільських господарств. Вони не можуть подавати свої товари на ринок раніше, ніж через рік. Протягом
цілого цього часу вони мусять боргуватись у шевця, кравця, коваля, колісника та різних інших
продуцентів, що їхніх продуктів вони потребують, і що їхні продукти можна виготувати протягом
небагатьох днів або тижнів. В наслідок цієї природної обставини і в наслідок швидкого збільшення
багатства в інших галузях праці, землевласники, що монополізували землю цілої держави, хоч вони,
крім цього, захопили й монополію законодавства, все ж таки не можуть врятувати себе й своїх слуг
фармерів від долі найбільш залежних людей в країні“. (Thomas Hodgskin, Popular Political Economy,
London, 1827, p. 147, примітка).

Всі методи, що ними в хліборобстві почасти рівномірніше розподіляється на цілий рік витрати на
заробітну плату й засоби праці, почасти скорочується оборот у наслідок культивування різноманітних
продуктів, яке уможливлює кілька зборів урожаю на рік, — всі ці методи потребують збільшення
авансовуваного обігового капіталу, витрачуваного на заробітну плату, добриво, насіння тощо. Так
буває, коли переходять від трипільного господарства з паром до сівозмінного без пару. Так буває у
Фляндрії при cultures dérobées\footnote*{
Culture dérobée — дослівно: „потайна культура“. Так зветься культура корінняків, що їх засівають
після збору основної культури; назва походить з того, що така культура, потребуючи менше часу,
вистигає між двома основними культурами, ніби потай. \emph{Ред.}
}“. „В culture dérobée застосовують корінняки; те саме поле спочатку
дає збіжжя, льон, рапс на задоволення потреб людини,
\index{ii}{0178}  %% посилання на сторінку оригінального видання
а по жнивах його засівають корінняками на годівлю худоби. Ця система, за якої рогата худоба
може ввесь час перебувати в стійлі, дає чималі запаси угноєння і стає таким чином за основу
сівозмінного господарства. В піскуватих місцевостях більше, ніж третину оброблюваної землі
відводиться під cultures dérobées, а наслідок такий, ніби оброблюваної землі побільшало на третину“.
Поряд корінняків тут культивують також конюшину та інші кормові рослини. „Рільництво доведене таким
чином до того пункту, де воно перетворюється на городництво, потребує, звичайно, порівняно чималого
основного капіталу (Anlagekapital). В Англії основний капітал обчислюється в 250 франків на гектар.
У Фляндрії основний капітал в 500 франків на гектар наше селянство, мабуть, визнало б за дуже
низький“. (Essais sur l’Economie Rurale de la Belgique par Emile de Laveleye. Paris, 1863, p. 59,
60, 63).

Візьмімо нарешті лісівництво. — „Продукція дерева посутньо відрізняється від більшости інших
продукцій тим, що тут сила природи діє самостійно і при природному поновленні не потребує сили
людської або сили капіталу. А проте, навіть там, де ліси розводять штучно, застосування сили
людської та капіталу порівняно з дією сил природи є лише незначне. Крім того ліс може добре рости на
таких ґрунтах і місцях, де хліб не удається або продукція його не оплачується. Але для
лісорозведення при правильному господарюванні потрібна також більша площа, ніж для культури хліба,
бо на маленьких парцелях не можна розбити ліс на правильні дільниці, побічних плодів майже не можна
використати, важче зберігати дерево й~\abbr{т. д.} Однак процес продукції тут сполучено також з такими
довгими періодами, що він виходить поза пляни приватного господарства, а іноді навіть поза межі
людського життя. Капітал, витрачений на закуп землі“ [при громадській продукції
цей капітал відпадає, і справа лише в тому,
скільки землі може громада відібрати під ліс від поля та пасовиська], „дає помітні плоди лише через
довгий час і обертається тільки почасти, а цілий оборот при деяких ґатунках дерев потребує до 150
років. Крім того, для правильної продукції дерева треба, щоб був запас живого дерева в 10--40 разів
більший, ніж щорічне споживання. Тому той, хто не має інших прибутків і посідає чимало площі лісу,
не може вести правильне лісове господарство“ (Kirchhof, р. 58).

Довгий час продукції (що має в собі відносно лише незначну частку робочого часу) і сполучений з ними
довгий період обороту робить лісівництво несприятливим для приватних, а значить, і для
капіталістичних підприємств, бо останні суттю своєю є приватні підприємства, хоча б замість
поодинокого капіталіста виступав капіталіст асоційований. Розвиток культури і взагалі промисловости
остільки енергійно виявив себе щодо знищення лісів, що порівняно з цим усе, зроблене ним для
підтримання й насадження лісу, є цілком незначна величина.

Особливо треба зауважити в цитаті Кірхгофа таке місце: „Крім того, для правильної продукції дерева
треба, щоб був запас живого дерева в 10--40 разів більший, ніж щорічне споживання“. — Отже, один
оборот дорівнює 10--40 і більше рокам.

\parcont{}  %% абзац починається на попередній сторінці
\index{i}{0179}  %% посилання на сторінку оригінального видання
вільний чи невільний, мусить до робочого часу, доконечного для
утримання себе самого, додавати надлишок робочого часу, щоб
продукувати засоби існування для власника засобів продукції\footnote{
«Ті, що працюють\dots{} у дійсності годують і пенсіонерів, яких називають
багатими, і самих себе» («Those who labour\dots{} in reality feed both
the pensioners called the rich, and themselves»). (\emph{Edmund Burke}: «Thoughts and
Details on Scarcity». London, 1800, p. 2)
},
хоч буде цей власник атенський \textgreek{χαλος χαγανος}\footnote*{
аристократ. \emph{Ред.}
}, етруський теократ,
civis romanus\footnote*{римський громадянин. \emph{Ред.}},
норманський барон, американський рабовласник,
волоський боярин, сучасний лендлорд або капіталіст\footnote{
У своїй «Römische Geschichte» Нібур дуже наївно зауважує: «Нічого
таїти, що такі витвори, як етруські, що будять у нас подив навіть у
своїх руїнах, у маленьких (!) державах мають собі за передумову існування
панів і рабів». Багато глибше висловився Сісмонді, що «брюссельські
мережива» мають собі за передумову існування панів наймачів і найманих
робітників.
}.
Та проте ясно, то коли в якійсь суспільній економічній формації
має перевагу не мінова вартість, а споживна вартість продукту,
то додаткова праця обмежується на вужчому або ширшому колі
потреб, але з самого характеру продукції не випливає безмежна
потреба додаткової праці. Тому ми находимо жахливу надмірну
працю в старовинному світі там, де йдеться про здобуття мінової
вартости в її самостійній грошовій формі — у продукції золота
й срібла. Поневільна, що тягне за собою смерть робітника, праця
є тут офіціяльна форма надмірної праці. Досить прочитати лише
Діодора Сіцілійського\footnote{
«Не можна без жалю до їхньої злиденної долі дивитися на цих нещасних
(що працюють на копальнях золота між Єгиптом, Етіопією й
Арабією), які навіть не мають змоги подбати про чистоту свого тіла або
покрити свою голизну. Бо тут немає поблажливости, немає жалю до
хорих, покалічених, дідусів, до жіночої слабости. Всі мусять, приневолені
ударами київ, працювати й працювати аж доки смерть покладе кінець
їхнім мукам і злидням». (\emph{Diodorus Siculus}: «Historische Bibliothek», Buch З,
kар. 13).
}. Однак це є винятки у старовинному
світі. Але скоро тільки народи, що в них продукція рухається
ще в низьких формах рабської праці, панщини й~\abbr{т. ін.}, втягуються
у світовий ринок, опанований капіталістичним способом продукції,
в наслідок чого переважним інтересом для них стає продаж виробів
своєї продукції за кордон, — до варварських страхіть рабства,
кріпацтва й~\abbr{т. ін.} прищеплюється цивілізоване страхіття надмірної
праці. Тому праця негрів у південних штатах Американського
союзу мала помірно-патріархальний характер доти, доки продукцію
зверталося головним чином на задоволення власних потреб.
Але в міру того як експорт бавовни стає життєвим інтересом цих
держав, в міру цього й надмірна праця негра, а в деяких місцях
споживання його життя протягом сімох робочих років, стає складовою
частиною байдужно обрахованої системи. Тут ішлося вже
не про те, щоб видушити з нього певну масу корисних продуктів.
Тут уже йшлося про продукцію самої додаткової вартости. Те ж
саме було з панщиною, приміром, у дунайських князівствах.

\input{_0180c.tex}
\input{_0181.tex}
\input{_0182c.tex}
\input{_0183.tex}
\input{_0184c.tex}
\input{_0185.tex}
\input{_0186.tex}
\input{_0187c.tex}
\input{_0188c.tex}

\index{iii2}{0189}  %% посилання на сторінку оригінального видання
Одно з найкумедніших явищ є в тому, що всі противники Рікардо, які
заперечують визначення вартости виключно працею, в справі з диференційною
рентою, що випливає з ріжниць землі, надають ваги тому, що тут вартість
визначається природою, а не працею; і одночасно приписують це визначення
положенню, або, і ще більше, процентові на капітал, вкладений в землю при
обробітку. Та сама праця дає однакову вартість для продукту, створеного
протягом даного часу; але величина або кількість цього продукту, отже, і та
частина вартости, яка припадає на відповідну частину цього продукту за даної
кількости праці, залежить єдино від кількости продукту, а це знову від продуктивности
даної кількости праці, не від величини цієї кількости. Чи завдячує
ця продуктивність своїм походженням природі, чи суспільству — цілком байдуже.
Тільки в тому разі, коли вона сама коштує праці, отже, капіталу, вона
збільшує ціну продукції новою складовою частиною, чого природа сама по собі
не робить.

\section{Абсолютна земельна рента}

Аналізуючи диференційну ренту, ми виходили з припущення, що найгірша
земля не виплачує земельної ренти, або, висловлюючись загальніше, що земельну
ренту виплачує тільки така земля, для продукту якої індивідуальна ціна продукції
нижча від ціни продукції, що реґулює ринок, так що в такий спосіб
виникає надзиск, що перетворюється на ренту. Потрібно насамперед зауважити,
що закон диференційної ренти, як днференційної ренти, зовсім не залежить від
правильности чи неправильности того припущення.

Коли загальну ціну продукції, що реґулює ринок, ми назвемо Р, то Р для
продукту найгіршого роду землі А збігається з індивідуальною ціною продукції
на цій землі; тобто вона оплачує зужиткований у продукції сталий і змінний капітал
плюс пересічній зиск (= підприємницькому баришеві плюс процент).

Рента тут дорівнює нулеві. Індивідуальна ціна продукції найближчого
кращого роду землі В = Р', і Р>Р'; тобто Р оплачує більше, ніж дійсну
ціну продукції продукту на клясі землі В. Хай тепер Р — Р' = d; тому
d, надмір Р над Р', є той надзиск, що його добуває орендар з цієї кляси В.
Це d перетворюється на ренту, яку доводиться виплачувати власникові землі.
Хай для третьої кляси землі С за дійсну ціну продукції буде Р", і хай Р —
Р'' = 2d; отже, ці 2d перетворюються на ренту; так само для четвертої кляси
D індивідуальна ціна продукції хай буде Р'", а Р — Р'" = 3d, які перетворюються
на земельну ренту і т. д. Даймо тепер, що припущення, ніби для
кляси землі А рента = 0, а тому ціна її продукту = Р + 0, помилкове. Хай,
навпаки, і вона дає ренту = г. В цьому випадку маємо двоякі наслідки.

\emph{Поперше}: ціна продукту землі кляси А не реґулювалася б ціною продукції
на цій землі, а мала б деякий надмір над цією ціною, вона була б =
P — r. Бо, коли припускається, нормальний перебіг капіталістичного способу
продукції, отже, коли припускається, що надмір r, виплачуваний від орендаря
земельному власникові, не становить вирахування ані з заробітної плати, ані
з пересічного зиску на капітал, то орендар може виплачувати його лише тому,
що його продукт продається понад ціну продукції, що він, отже, дав би йому
надзиск, коли б не доводилося відступати цей надмір у формі ренти земельному
власникові. Реґуляційна ринкова ціна всього наявного на ринку продукту
всіх родів землі була б тоді не та ціна продукції, яку дає капітал взагалі
у всіх сферах продукції, тобто не ціна рівна витратам плюс пересічний
зиск, а була б ціною продукції плюс рента, Р + r, не Р. Бо ціна продукту
кляси А визначає взагалі межу реґуляційної загальної ринкової ціни, тієї ціни,
\parbreak{}  %% абзац продовжується на наступній сторінці

\parcont{}  %% абзац починається на попередній сторінці
\index{ii}{0190}  %% посилання на сторінку оригінального видання
з 100 до 75, або на одну чверть. Ціла сума, що на неї скорочується
продуктивний капітал, який функціонує протягом дев’ятитижневого робочого
періоду, становить 9 × 25 \deq{} 225\pound{ ф. стерл.}, або четверту частину
900\pound{ ф. стерл}. Але відношення часу обігу до періоду обороту, як і раніш,
становить \frac{3}{12} \deq{} \sfrac{1}{4}. З цього випливає ось що. Для того, щоб продукція
не припинялась протягом часу обігу продуктивного капіталу, перетвореного
на товаровий капітал, щоб вона однаково невпинно продовжувалась тиждень
у тиждень, коли немає для цього окремого обігового капіталу, то
цього можна досягти, лише скоротивши продукцію, зменшивши поточну
складову частину діющого продуктивного капіталу. Поточна частина капіталу,
звільнена таким чином для процесу продукції протягом часу обігу,
відноситься до цілого авансованого поточного капіталу, як час обігу до
періоду обороту. Як ми вже зауважили, це має силу тільки для тих галузей
продукції, де процес праці відбувається тиждень-у-тиждень у
тому самому маштабі, де, отже, не треба, як у хліборобстві, в різні робочі
періоди витрачати різні кількості капіталу.

Навпаки, якщо ми припустимо, що самий характер підприємства виключає
можливість скорочення маштабу продукції, а тому й розмірів щотижнево
авансовуваного поточного капіталу, то безперервности продукції
можна досягти лише додачею поточного капіталу, в наведеному вище
прикладі додачею 300\pound{ ф. стерл}. Протягом дванадцятитижневого періоду
обороту послідовно авансується 1200\pound{ ф. стерл.}, з них 300 являють четверту
частину, як 3 тижні від 12. По скінченні 9-тижневого робочого
періоду капітальна вартість в 900\pound{ ф. стерл.} перетворюється з форми
продуктивного капіталу на форму товарового капіталу. Її робочий період
закінчено, але його не можна відновити з тим самим капіталом. Протягом
трьох тижнів, поки цей капітал перебуває в сфері циркуляції, функціонуючи
як товаровий капітал, він щодо продукційного процесу перебуває
в такому самому стані, ніби його взагалі не існувало. Ми лишаємо тут
осторонь усі кредитові відносини, а тому припускаємо, що капіталіст
господарює тільки своїм власним капіталом. Але тимчасом як капітал,
авансований на перший робочий період, завершивши процес продукції,
протягом 3 тижнів перебуває в процесі циркуляції, — у цей самий час
функціонує додатково витрачений капітал в 300\pound{ ф. стерл.}, так що безперервність
продукції не порушується.

Тут треба зауважити ось що:

Поперше. Робочий період авансованого спочатку капіталу в 900\pound{ ф.
стерл.} закінчується по 9 тижнях, але капітал припливає назад не раніш,
як по трьох тижнях, отже, лише на початку 13-го тижня. Однак новий
робочий період починається негайно за допомогою додаткового капіталу
в 300\pound{ ф. стерл}. Саме в наслідок цього підтримується безперервність
продукції.

Подруге. Функції первісного капіталу в 900\pound{ ф. стерл.} і новододаного
наприкінці першого дев’ятитижневого робочого періоду капіталу в 300\pound{ ф. стерл.}, який відкриває другий робочий період безпосередньо по закінченні
\parbreak{}  %% абзац продовжується на наступній сторінці

\parcont{}  %% абзац починається на попередній сторінці
\index{ii}{0191}  %% посилання на сторінку оригінального видання
першого, ці функції за першого періоду обороту точно відмежовані одна
від однієї, або принаймні їх можна точно відмежувати, тимчасом як протягом
другого періоду обороту вони, навпаки, переплітаються одна з
однією.

Уявімо собі справу наочніше:

Перший період обороту триває 12 тижнів. Перший робочий період —
9 тижнів; оборот авансованого на нього капіталу закінчується на початку
13-го тижня. Протягом останніх 3 тижнів функціонує додатковий капітал
в 300\pound{ ф. стерл.}, який починає другий дев’ятитижневий робочий
період.

Другий період обороту. На початку 13-го тижня 900\pound{ ф. стерл.} припливають
назад і можуть почати новий оборот. Але другий робочий
період уже на десятому тижні почато за допомогою додаткових 300\pound{ ф.
стерл.}; на початку 13-го тижня за допомогою тих самих 300\pound{ ф. стерл.}
уже закінчено третину робочого періоду, 300\pound{ ф. стерл.} з продуктивного
капіталу перетворено на продукт. А що для закінчення другого робочого
періоду треба ще лише 6 тижнів, то в процес продукції другого робочого
періоду можуть ввійти лише дві третини капіталу в 900\pound{ ф. стерл.},
який повернувся назад, а саме 600\pound{ ф. стерл}. З первісних 900\pound{ ф. стерл.}
звільнилося 300\pound{ ф. стерл.}, щоб відігравати ту саму ролю, яку відігравав
у першому робочому періоді додатковий капітал в 300\pound{ ф. стерл}. Наприкінці
6-го тижня другого періоду обороту закінчено другий робочий
період. Витрачений на нього капітал в 900\pound{ ф. стерл.} повертається за три
тижні, отже, наприкінці 9-го тижня другого дванадцятитижневого періоду
обороту. Протягом 3 тижнів його часу обігу ввіходить у робочий період
звільнений капітал в 300\pound{ ф. стерл}. З ним починається на 7-й тиждень
другого періоду обороту або на 19-й тиждень року третій робочий
період капіталу в 900\pound{ ф. стерл}.

Третій період обороту. Наприкінці 9-го тижня другого періоду обороту
знову зворотно припливають 900\pound{ ф. стерл}. Але третій робочий
період почався вже на сьомому тижні попереднього періоду обороту й
6 тижнів його вже минуло. Тому він триває тільки три тижні. Отже,
з 900\pound{ ф. стерл.}, що повернулись назад, у процес продукції ввіходять
лише 300\pound{ ф. стерл}. Четвертий робочий період заповнює дев’ятитижневу
решту цього періоду обороту, і таким чином з 37-го тижня року починається
одночасно четвертий період обороту й п’ятий робочий період.

Щоб спростити обчислення, ми припустимо робочий період в 5 тижнів,
час обігу в 5 тижнів, отже, період обороту в 10 тижнів; рік рахуватимемо
в 50 тижнів, а щотижневу витрату капіталу рахуватимемо в 100\pound{ ф.
стерл}. Отже, робочий період потребує поточного капіталу в 500\pound{ ф. стерл.},
а час обігу потребує додаткового капіталу — нових 500\pound{ ф. стерл}. Робочі
періоди й періоди оборотів позначиться тоді так:

1-й робочий період: тижні 1--5 (500\pound{ ф. стерл.} товару повертаються
наприкінці 10 тижня).

2-й робочий період: тижні 6--10 (500\pound{ ф. стерл.} товару повертаються
наприкінці 15 тижня).

\input{_0192.tex}
\parcont{}  %% абзац починається на попередній сторінці
\index{i}{0193}  %% посилання на сторінку оригінального видання
лише у власних пекарнях. Під кінець тижня\dots{} тобто в четвер,
праця починається тут о 10 годині вечора й триває з незначною
лише перервою до пізньої ночі під неділю»\footnote{
Там же, стор. LXXI.
}.

Щождо «underselling masters», то й буржуазний погляд розуміє,
що «неоплачена праця підмайстрів (the unpaid labour of
the men) становить основу їхньої конкуренції»\footnote{
\emph{George Read}: «The History of Baking», London 1848, p. 16.
}. «Full priced
baker» (пекар, що продає за «повну ціну») виказує перед слідчою
комісією на своїх «underselling» -конкурентів як на розкрадачів
чужої праці й фалшівників. «Вони мають успіх лише
через те, що ошукують публіку, і через те, що видушують із
своїх підмайстрів 18 годин праці, оплачуючи лише 12 годин»\footnote{
«First. Report etc.», Evidence. Свідчення «full priced baker’a» Чізмена,
p. 108.
}.

Фальсифікація хліба й утворення кляси пекарів, що продають
хліб нижче від повної ціни, розвинулися в Англії на початку
XVIII сторіччя, відколи занепав цеховий характер ремества й
за спиною номінального майстра-пекаря виступив капіталіст\footnote{
\emph{George Read}: «The History of Baking», London 1848. Наприкінці
XVII й початку XVIII віків посередників (аґентів), що протиснулись
у всі можливі галузі ремісництва, офіціально ганьбили, називаючи
їх «Public Nuisances»\footnote*{суспільним лихом. \emph{Ред.}}.
Приміром, від «Grand Jury»\footnote*{Велике жюрі — суд присяжних. \emph{Ред.}}
підчас чвертьрічної
сесії мирових суддів у графстві Somerset подано до Палати громад
«presentment»\footnote*{внесення. \emph{Ред.}},
де, між іншим, сказано: «that these factors of Blackwell Hall
are a Public Nuisance and Prejudice to the Clothing Trade and ought to be
put down as a Nuisance». (Ці посередники Blackwell Hall є суспільне лихо
й перешкода торговлі одягом, і, як таку перешкоду, їх треба знищити»).
«The Case of our English Wool etc.», London 1685) p. 6, 7).
} в образі мірошника або торговця борошном. Цим покладено основу
для капіталістичної продукції, для безмірного зловження робочого
дня та нічної праці, хоч остання навіть у Лондоні стала на тверді
ноги лише в 1824~\abbr{р.}\footnote{«First Report etc.», p. VIII.}

Після всього попереднього зрозуміло, чому звіт комісії зачисляє
пекарських підмайстрів до тих робітників, які живуть
недовго; щасливо поминувши небезпеку стати жертвою жахливої
дитячої смертности, яка є нормальне явище для всіх категорій
робітничої кляси, вони рідко доживають до 42 року життя. А, проте,
пекарний промисел завжди переповнений кандидатами. Джерела,
що постачають для Лондону ці «робочі сили», є Шотляндія, західні
рільничі округи Англії і — Німеччина.

В 1858 – 1860~\abbr{рр.} пекарські підмайстри в Ірляндії зорганізували
власним коштом ряд великих мітинґів для аґітації проти
нічної й недільної праці. Публіка з суто ірляндським запалом
стала на їхній бік, як це було, приміром, 1860~\abbr{р.} на травневому
мітинґу в Дебліні. В наслідок цього руху було дійсно успішно
заведено виключну денну працю в Wexford’і, Kilkenny, Clonmel’i,
\parbreak{}  %% абзац продовжується на наступній сторінці

\parcont{}  %% абзац починається на попередній сторінці
\index{i}{0194}  %% посилання на сторінку оригінального видання
Waterford’i й~\abbr{т. ін.} «В Limerick’y, де, як відомо, страждання
найманих підмайстрів переходять усяку міру, цей рух розбився
об опір пекарів-хазяїнів, особливо ж пекарів-мірошників. Приклад
Limerick’a призвів до назаднього руху в Eniss’i та Tripperary.
В Cork’y, де громадське обурення виявилось у найжвавішій
формі, хазяї розбили рух, використавши свою силу викинути підмайстрів
на вулицю. В Дебліні хазяї виявили якнайрішучіший
опір і, переслідуючи тих підмайстрів, що стояли на чолі аґітації,
примусили решту поступитись і згодитись на нічну та недільну
працю»\footnote{
«Report of Committee on the Baking Trade in Ireland for 1861».
}. Комісія англійського уряду, озброєного в Ірляндії
з ніг до голови, жалібно нарікає, немов голосільниця, на невблаганних
пекарів-хазяїнів Дебліну, Корку й~\abbr{т. ін.} «Комітет гадає,
що робочий час обмежено природними законами, яких не можна
порушувати безкарно. Тримаючи своїх робітників під загрозою
звільнення, хазяїни примушують їх порушувати їхні релігійні
переконання, не слухатися законів країни і зневажати громадську
думку (все це останнє стосується до недільної праці), вони сіють
ворожнечу між капіталом і працею та дають приклад, небезпечний
для релігії, моральности й суспільного ладу\dots{} Комітет гадає,
що здовження робочого дня понад 12 годин є узурпаторське втручання
у родинне й приватне життя робітника, і через встрявання
в родинний побут людини і у виконання нею своїх родинних обов’язків
як сина, брата, чоловіка й батька воно призводить до лихих
моральних результатів. Праця понад 12 годин має тенденцію підточувати
здоров’я робітника, викликає передчасну старість і
ранню смерть, а тому й нещастя робітничих родин, позбавляючи
(«are deprived») їх опіки й підпори голови родини саме в такий
час, коли це їм якнайпотрібніше»\footnote{Там же.}.

Ми тільки що побували в Ірляндії. По тому боці каналу, в
Шотляндії, рільничий робітник, робітник плуга, нарікає на свою
13--14-годинну працю в найсуворішому кліматі, з чотиригодинною
додатковою працею в неділю (це в країні святкувальників
суботи!)\footnote{
Публічний мітинґ рільничих робітників у Lasswade біля Glasgow
5 січня 1866~\abbr{р.} (див. «Workman’s Advocate» з 13 січня 1860~\abbr{р.}). — Утворення
наприкінці 1865~\abbr{р.} тред-юньойону рільничих робітників, передучім
у Шотляндії, є історична подія. В одній з найпригнобленіших рільничих
округ Англії, в Buckingamshire, наймані робітники влаштували
в березні 1867~\abbr{р.} величезний страйк з метою вибороти підвищення тижневої
заробітної плати з 9--10\shil{ шилінґів} до 12\shil{ шилінґів.} (Із попереднього
видно, що рух англійського рільничого пролетаріяту, геть чисто зламаний
від часів придушення його енерґійних демонстрацій після року 1830
і особливо після заведення нового закону про бідних, знову починається
в шістдесятих роках, доки нарешті в році 1782 стає епохальним. У ІІ томі
я повертаюсь до цього питання, а так само й до Синіх Книг про становище
англійського рільничого робітника, що з’явилися після 1867~\abbr{р.} — Додаток
до третього видання).
}, одночасно з цим перед лондонським Grand Jury стало
троє залізничників: пасажирний кондуктор, машиніст і сиґнальник.
Велика залізнична катастрофа відправила сотні пасажирів
\parbreak{}  %% абзац продовжується на наступній сторінці

\input{_0195.tex}
\input{_0196.tex}
\parcont{}  %% абзац починається на попередній сторінці
\index{i}{0197}  %% посилання на сторінку оригінального видання
«Наші «білі раби», — вигукнув «Morning Star», орган панів
фритредерів Кобдена й Брайта, — наші білі раби запрацьовуються
на смерть і гинуть і вмирають без найменшого шуму»\footnote{
«Morning Star» з 23 липня 1863~\abbr{р.} «Times» скористався цим випадком
для оборони американських рабовласників проти Брайта й~\abbr{т. ін.}
«Дуже багато з нас, — каже «Times», — гадають, що лоти, доки ми сами
вимучуємо на смерть працею наших власних молодих жінок, погрожуючи
їм ударами голоду замість свисту батога, доти ми ледве чи маємо право
йти мечем і вогнем на ті родини, що їхні члени родилися рабовласниками
та які принаймні добре годують своїх рабів і вимагають від них лише
помірної праці» («Times», а 2 липня 1863~\abbr{р.}). Газета торів «Standart»
розправлялась у тому самому дусі з його преподобієм Ньюмен Холлом:
«Він відлучує від церкви рабовласників, але молиться разом із порядними
людьми, що примушують працювати за собачу плату лондонських візників
та кондукторів омнібусів і~\abbr{т. ін.} лише по 16 годин на день». Нарешті,
пролунав голос оракула, винахідника культу генія, пана Томаса Карлейля,
про якого я вже року 1850 писав: «Геній пішов к чорту, лишився культ».
В коротенькій притчі він зводить єдину величну подію сучасної історії,
американську громадянську війну, на те, що Петро з півночі з усіх сил
намагається переломити черепа Павлові з півдня, бо Петро з півночі
наймає свого робітника «поденно», а Павло з півдня — «на ціле життя».
(«Macmillan’s Magazine». Ilias Americana in nuce. Серпневий зошит
1863~\abbr{р.}). Так луснув, нарешті, шумовинний пухир торійських симпатій
до міських — але ні в якому разі не до сільських! — найманих робітників.
Основа цих симпатій — це рабство!
}.

«Запрацьовуватись на смерть — це є порядок дня не лише
в майстернях кравчих, але в тисячах місць, ба на кожному місці,
де справи йдуть добре\dots{} Візьмімо як приклад коваля. Як вірити
поетам, то немає в світі людини сильнішої, веселішої за коваля.
Він устає раннім ранком і викрешує іскри перед тим, як засяє
сонце; нема такої людини, що так їла б, так пила б і спала, як
він. Якщо поглянути на долю коваля чисто з фізичного боку, то,
дійсно, за помірної праці, становище його одне з найкращих.
Але ходімо за ним до міста й погляньмо на той тягар праці, який
накладають на його дужі плечі, погляньмо на місце, яке він посідає
у таблицях смертности нашої країни? У Marylebone (один із
найбільших міських кварталів Лондону) смертність ковалів становить
31 на 1000 на рік, а це на 11 перевищує пересічну смертність
дорослих чоловіків Англії. Праця ця, майже інстинктова вмілість
людини, сама по собі бездоганна, через саму лише надмірність
стає руйнаційною для людини. Людина може зробити стільки й
стільки ударів молотом на день, стільки й стільки кроків, стільки
й стільки разів дихнути, стільки й стільки зробити якоїсь роботи
й прожити пересічно, приміром, 50 років. Її примушують робити
на стільки більше вдарів, на стільки більше кроків, стільки частіш
віддихувати, а це все разом збільшує її життєве завдання на
одну четвертину на день. Вона силкується це все робити, а результат
такий, що за обмежений період вона виконує на четвертину
більшу роботу і вмирає на 37 році замість на 50»\footnote{
\emph{Dr.~Richardson}: «Work and Overwork» y «Social Science Review»,
18 липня 1863.
}.


\index{i}{0198}  %% посилання на сторінку оригінального видання
\subsection{Денна й нічна праця. Система змін}

З погляду процесу зростання вартости сталий капітал, засоби
продукції існують лише на те, щоб вбирати в себе працю і з кожною
краплею праці — відповідну кількість додаткової праці.
Якщо вони цього не роблять, то вже саме їхнє існування становить
для капіталіста неґативну втрату, бож протягом того часу,
коли вони лежать без діла, вони репрезентують марно авансований
капітал: утрата ця стає позитивною, скоро тільки на поновлення
перерваної продукції треба додаткових витрат. Здовження
робочого дня поза межі природного дня геть аж у ніч діє
лише як паліятив, лише приблизно заспокоює вампірову спрагу
живої крови праці. Тому присвоєння праці протягом усіх 24 годин
доби є іманентне прагнення капіталістичної продукції. А що
фізично неможливо день і ніч безупинно висисати ті самі робочі
сили, то, щоб перемогти фізичні перешкоди, потрібно зміни робочих
сил, споживаних вдень і вночі, зміни, яка допускає різні
методи, наприклад, її можна зорганізувати так, що частина робочого
персоналу один тиждень працює вдень, а другий тиждень
вночі й~\abbr{т. ін.} Як відомо, така система змін, таке перемінне господарство
панувало за часів юнацького розцвіту англійської бавовняної
промисловости і~\abbr{т. ін.}, і процвітає в наші часи, між
іншим, на бавовняних прядільнях Московської губерні. Як система
цей 24-годинний процес продукції існує ще й нині в багатьох
досі «вільних» галузях промисловости Великобрітанії, між іншим
у домнах, кузнях, вальцювальних та інших металевих мануфактурах
Англії, Велзу й Шотляндії. Робочий процес охоплює тут,
окрім 24 годин шістьох робочих днів, здебільшого ще й 24 години
неділі. Робітники складаються з чоловіків та жінок, дорослих
і дітей обох статей. У віці дітей і молоді є всі переходові ступені
від 8 (в деяких випадках від 6) до 18 років\footnote{
«Children’s Employment Commission». Third Report. London
1864, p. IV, V, VI.
}. У деяких галузях
дівчата й жінки працюють уночі разом із чоловічим персоналом\footnote{
«Так у Стафордшірі, як і в південному Велзі, молоді дівчата й жінки
працюють у кам’яновугляних копальнях та коксувальнях не лише вдень,
а й уночі. У звітах, подаваних до парляменту, не раз зазначувано це
явище як причину великого й загальновідомого лиха. Ці жінки, що працюють
разом із чоловіками й ледве відрізняються від них своїм одягом,
покриті брудом і сажею, наражаються на небезпеку згубити свій моральний
характер через утрату самоповаги, а це є неминучий наслідок їхньої
нежіночої праці». («Both in Staffordshire and in South Wales young girl
and women are employed on the pit banks and on the coke heaps, not only
by day, but also night. This practice has been often noticed in Reports presented
to Parliament, as being attended with great and notorious evils.
These females, employed with the men, hardly distinguished from them
in their dress, and begrimed with dirt and smoke, are exposed to the deterioration
of character arising from the loss of self-respect which can hardly
fail to follow from their unfeminine occupation»). (Там же, 194, p. XXVI.~Порівн. Fourth Report (1865), 61, p. XIII). Те саме й на гутах.
}.


\index{i}{0199}  %% посилання на сторінку оригінального видання
Не кажучи вже про загальні шкідливі впливи нічної праці\footnote{
«Цілком природно, — зауважує один фабрикант сталі, який
уживає до нічної праці дітей, — що молодь, яка працює вночі, не має
змоги спати вдень і не може користуватися путнім відпочинком, а лише
без перестанку тиняється на другий день («Children’s Employment Commission.
Fourth Report», 63, p. XIII). Про значення сонячного світла для
збереження й розвитку організму один лікар каже, між іншим: «Світло
безпосередньо впливає і на тканини тіла, яким дає міць і елястичність.
Мускули тварин, позбавлених нормальної кількости світла, стають як губка
і втрачають свою елястичність, сила нервів через недостачу побудливих
спонук втрачає свій тонус, і розвиток усього, що перебуває в процесі
зростання, занепадає\dots{} Щождо дітей, то для їхнього здоров’я є вельми
важливий постійний рясний приплив денного світла й безпосередній вплив
сонячного проміння протягом якоїсь частини дня. Світло помагає перетворювати
харч у добру плястичну кров і зміцнює новоутворені фібри. Воно
побуджує й органи зору і через те викликає інтенсивнішу діяльність різних
мозкових функцій». Пан В.~Стрендж, старший лікар «General Hospital»
у Worcester’i, що з його твору «Про здоров’я» (1864) запозичено це
місце, пише в одному листі до члена слідчої комісії пана Вайта: «Я мав
давніше нагоду стежити в Ланкашірі за впливом нічної праці на фабричних
дітей і, всупереч улюбленому запевненню деяких працедавців, я
рішуче заявляю, що така праця швидко підтинає здоров’я дітей». («Children’s
Employment Commission. 4 th Report», 284, p. 55). Що такі речі
можуть взагалі бути предметом серйозних суперечок, найкраще показує
те, як впливає капіталістична продукція на «мозкові функції» капіталістів
та їхніх retainers\footnote*{
прихильників. \emph{Ред.}
}.
},
безперервний процес продукції, що триває протягом двадцяти чотирьох
годин, дає незвичайно бажану нагоду для того, щоб переступати
межі номінального робочого дня. Приміром, у згаданих
вище галузях промисловости, де працюють з дуже великим напруженням,
офіціяльний робочий день становить для кожного робітника
здебільшого 12 годин нічних або денних. Але наднормова
праця, яка виходить поза ці межі, в багатьох випадках, уживаючи
слів англійського офіційного звіту, «справді повна жаху»
(«truly fearful»)\footnote{
Там же, 57, p. XII.
}. Ніякий людський розум, — каже звіт, — не
може уявити собі тієї маси праці, яку, за даними свідків, виконують
хлопці 9--12 років, і не дійти при цьому неминуче
до висновку, що такого зловживання владою батьків та працедавців
надалі не можна дозволяти»\footnote{
Там же (4 th Report, 1865), 58, p. XII.
}.

«Вже та метода, що хлопчаків взагалі примушують працювати
навпереміну то вдень то вночі, — вже це приводить так підчас
оживлення справ, як і за звичайного стану речей до ганебного здовження
робочого дня. Це здовження в багатьох випадках є не лише
жорстоке, але просто неймовірне. Часто-густо буває, що з тієї
або іншої причини іноді не прийде якийсь із хлопчаків на зміну.
Тоді один або декілька з присутніх хлопчаків, що вже скінчили
свій робочий день, мусять заступити відсутнього. Ця система
така загальновідома, що управитель однієї вальцювальні на мій запит,
як заповнюється місця відсутніх хлопчаків, відповів: «Аджеж
я знаю, що вам це так само добре відомо, як і мені», — і, ні трохи
\parbreak{}  %% абзац продовжується на наступній сторінці


\index{iii2}{0200}  %% посилання на сторінку оригінального видання
Коли б уся земля певної країни, придатна для хліборобства, була вже
здана в оренду, — при чому припускається, як загальне явище, капіталістичний
спосіб продукції і нормальні відносини, — то не було б такої землі, яка не
давала б ренти, але могли б існувати такі приміщення капіталу, окремі частини
капіталу вкладеного в землю, які не давали б ренти; бо, скоро земля здана в
оренду, земельна власність перестає діяти, як абсолютна межа потрібного вкладення
капіталу. Як відносна межа, вона продовжує ще діяти і після цього
в такій мірі, в якій перехід до земельного власника долученного до землі капіталу
ставить тут перед орендарем дуже визначені межі. Тільки в цьому випадку
вся рента перетворилася б на диференційну ренту, на диференційну ренту, яка
визначається не ріжницями в якості землі, а ріжницями між надзисками, що
постають від останніх приміщень капіталу на певній землі, і рентою, яка виплачувалася
б за оренду землі найгіршої кляси. Як межа земельна власність
діє абсолютно лише остільки, оскільки допущення до землі взагалі, як до сфери
приміщення капіталу, зумовлює данину земельному власникові. Коли це допущення
сталося, останній уже не може протиставити ніяких абсолютних меж
кількісному розмірові приміщення капіталу на даній дільниці землі. Будуванню
будинків взагалі кладеться межу земельною власністю третьої особи на ту дільницю
землі, на якій мається збудувати будинок. Але скоро лише ця земля
орендована під будівлю будинків, то вже від орендаря залежить, чи бажає він
збудувати на ній високий чи низький будинок.

Коли б пересічний склад хліборобського капіталу був такий самий або
вищий, ніж пересічний склад суспільного капіталу, то абсолютна рента, знов
таки в щойно дослідженому розумінні, відпала б; тобто відпала б рента, яка
відрізняється так від диференційної ренти, як і від ренти, що ґрунтується на
власне монопольній ціні. Тоді вартість хліборобського продукту не була б
вища від його ціни продукції, і хліборобський капітал пускав би в рух не
більшу кількість праці, отже, реалізував би також не більшу кількість додаткової
праці, ніж нехліборобський капітал. Те саме сталося б тоді, коли б з
проґресом культури склад хліборобського капіталу зрівнявся із пересічним
складом суспільного капіталу.

На перший погляд здається за суперечність припускати, що, з одного
боку, склад хліборобського капіталу підвищується, отже, зростає його стала частина
проти змінної, а з другого боку, що ціна хліборобського продукту має
піднестися остільки високо, щоб нова і гірша, ніж колишня, земля могла виплачувати
ренту, яка в цьому випадку могла б виникнути лише з надміру ринкової
ціни над вартістю і ціною продукції, коротко, лише з монопольної ціни
продукту.

Тут треба відрізняти таке.

Розглядаючи створення норм зиску, ми, насамперед, бачили, що капітали
які, технологічно розглядувані, складені однаково, тобто порівняно з кількістю
машин і сирового матеріялу пускають в рух однакову кількість праці,
можуть, проте, бути різного складу в наслідок того, що сталі частини цих капіталів
мають різну вартість. Сировий матеріял або машини можуть бути в
одному випадку дорожчі, ніж у другому. Щоб пустити в рух таку саму масу
праці (а це, згідно з припущенням, було б потрібне для перероблення такої ж
самої маси сирового матеріялу), в одному випадку довелося б авансувати більший
капітал, ніж у другому, тому що, наприклад, з капіталом 100 я не можу пустити
в рух сднакової кількости праці, коли сировий матеріял, який теж доводиться
оплачувати з цих 100, в одному випадку коштує 40, в другому 20.
Але те, що технологічно ці капітали все ж складені однаково, негайно виявилося
б, скоро ціна дорожчого сирового матеріялу знизилася б до рівня дешевшого.
Відношення вартости змінного і сталого капіталу тоді стали б однакові,
\parbreak{}  %% абзац продовжується на наступній сторінці

\input{_0201.tex}
\input{_0202.tex}

\index{iii1}{0203}  %% посилання на сторінку оригінального видання
Візьмімо тепер капітал, склад якого є нижчий, ніж первісний
склад пересічного суспільного капіталу $80 c + 20 v$ (який
перетворився тепер в $76\sfrac{4}{21}c + 23\sfrac{17}{21}v$), наприклад, $50 c + 50 v$.
Тут ціна виробництва річного продукту, — якщо ми для спрощення
припустимо, що весь основний капітал увійшов як зношування
в річний продукт і що час обороту такий самий, як
і в випадку I, — становила перед підвищенням заробітної плати
$50 c + 50 v + 20 p = 120$. Підвищення заробітної плати на 25\%
дає для тієї самої кількості приведеної в рух праці підвищення
змінного капіталу з 50 до 62\sfrac{1}{2}. Коли б річний продукт був
проданий по попередній ціні виробництва в 120, то це дало б
$50 c + 62\sfrac{1}{2}v + 7\sfrac{1}{2}p$, тобто норму зиску в 6\sfrac{2}{3}\%.
Але нова пересічна норма зиску є 14\sfrac{2}{7}\%, і через те що ми всі інші умови
припускаємо незмінними, цей капітал в $50 c + 62\sfrac{1}{2}v$ так само
мусить дати вказаний зиск. Але капітал в 112\sfrac{1}{2}, при нормі зиску
в 14\sfrac{2}{7}, дає 16\sfrac{1}{14} зиску.\footnote*{
В першому німецькому виданні тут сказано: „в круглих числах 16\sfrac{1}{12}
зиску“; відповідно до цього Енгельс обчислює потім ціну виробництва в 128\sfrac{7}{12}
В рукопису Маркса дано точне число в 16\sfrac{3}{42}, яке нами взяте з відповідним
скороченням дробу і застосоване при обчисленні ціни виробництва.
\emph{Примітка ред. нім. вид. ІМЕЛ.}
} Отже, ціна виробництва вироблених
ним товарів є тепер $50 c + 62\sfrac{1}{2}v + 16\sfrac{1}{14}p = 128\sfrac{8}{14}$. Отже, в наслідок
підвищення заробітної плати на 25\% ціна виробництва
тієї самої кількості того самого товару підвищилась тут з 120
до 128\sfrac{8}{14}, або більше ніж на 7\%.

Візьмім, навпаки, сферу виробництва вищого складу, ніж пересічний
капітал, наприклад, $92 c + 8 v$. Отже, первісний пересічний
зиск і тут = 20, і якщо ми знову припустимо, що весь
основний капітал входить у річний продукт і що час обороту
такий самий, як і в випадках І і II, то ціна виробництва товару
й тут = 120.

В наслідок підвищення заробітної плати на 25\% змінний капітал
для тієї самої кількості праці зростає з 8 до 10, отже
витрати виробництва товарів зростають з 100 до 102; з другого
боку, пересічна норма зиску впала з 20\% до 14\sfrac{2}{7}\%. Але
$100 : 14\sfrac{2}{7} = 102 : 14\sfrac{4}{7}$\footnote*{
В першому німецькому виданні тут стоїть: „(приблизно)“. В рукопису
Маркса цього слова немає. В дійсності тут рівняння точне, а не тільки приблизне.
\emph{Примітка ред. нім. вид. ІМЕЛ.}
}. Отже, зиск, що припадає тепер на 102,
становить 14\sfrac{4}{7}. І тому весь продукт продається за
$k + kp' = 102 + 14\sfrac{4}{7} = 116\sfrac{4}{7}$. Отже, ціна виробництва впала
з 120 до 116\sfrac{4}{7}, або майже на 3\%\footnote*{
В першому німецькому виданні тут сказано: „більше ніж на 3\%. В рукопису
Маркса стоїть: „на 3\sfrac{3}{7}“, тобто дано абсолютне число. В процентах воно
дорівнює 2\sfrac{6}{7}\%. \emph{Примітка ред. нім. вид. ІМЕЛ.}
}.

Отже, в наслідок підвищення заробітної плати на 25\%:

1) для капіталу пересічного суспільного складу ціна виробництва
товару лишилась незмінною;

2) для капіталу нижчого складу ціна виробництва товару
\parbreak{}  %% абзац продовжується на наступній сторінці

\input{_0204.tex}
\input{_0205.tex}
\input{_0206.tex}
\input{_0207c.tex}
\parcont{}  %% абзац починається на попередній сторінці
\index{ii}{0208}  %% посилання на сторінку оригінального видання
завжди звільнятися 300\pound{ ф. стерл}. Навпаки, коли щотижня витрачається
300\pound{ ф. стерл.}, то для робочого періоду ми маємо 1800\pound{ ф. стерл.}, для періоду
циркуляції 900\pound{ ф. стерл.}; отже, періодично звільнятиметься вже 900\pound{ ф. стерл.}
замість 300\pound{ ф. стерл}.

D. Ввесь капітал, напр., в 900\pound{ ф. стерл.}, треба поділити на дві частини,
які раніш: 600\pound{ ф. стерл.} для робочого періоду і 300\pound{ ф. стерл.} для періоду
циркуляції. Частина, дійсно витрачувана на процес праці, зменшиться в
наслідок цього на одну третину, з 900 до 600\pound{ ф. стерл.}, і тому розмір продукції
зменшиться на одну третину. З другого боку, 300\pound{ ф. стерл.}
функціонують лише для того, щоб зробити робочий період безперервним,
так, щоб на процес праці щотижня протягом року можна було витрачати
по 100\pound{ ф. стерл}.

Беручи абстрактно, цілком байдуже, чи роблять 600\pound{ ф. стерл.} протягом
6 × 8 = 48 тижнів (продукт = 4800\pound{ ф. стерл.}), чи весь капітал в 900\pound{ ф. стерл.}
витрачається на процес праці протягом 6 тижнів, а потім протягом
З тижнів періоду циркуляції він лежить без діла; в останньому випадку
він працював би на протязі 48 тижнів 6 Х 5 \sfrac{1}{3} = 32 тижні (продукт =
900 × 5\sfrac{1}{3} = 4800\pound{ ф. стерл.}) і 16 тижнів лежав би без діла. Але, не кажучи
вже про більше псування основного капіталу протягом 16 тижнів, коли
він лишається бездіяльний, та подорожчання праці, що її доведеться оплатити
за ввесь рік, хоч вона діє лише протягом частини його, така регулярна
перерва продукційкого процесу взагалі несполучна з продукцією
сучасної великої промисловости. Сама ця безперервність є продуктивна
сила праці.

Коли ми тепер ближче придивимось до звільненого капіталу, в дійсності
до капіталу, що його дію припинено, то виявиться, що чимала
частина його завжди мусить мати форму грошового капіталу. Зупинімось
на прикладі: робочий період 6 тижнів, період циркуляції 3 тижні, щотижнева
витрата 100\pound{ ф. стерл}. Посередині другого робочого періоду,
наприкінці 9-го тижня, припливають назад 600\pound{ ф. стерл.}, що з них протягом
решти робочого періоду треба витратити лише 300\pound{ ф. стерл}.
Отже, наприкінці другого робочого періоду з цієї суми звільняться
300\pound{ ф. стерл}. В якому стані перебувають ці 300\pound{ ф. стерл.}? Припустімо,
що \sfrac{1}{3} треба витратити на заробітну плату, \sfrac{2}{3} на сировинні та допоміжні
матеріяли. Отже, з 600\pound{ ф. стерл.}, що приплили назад, 200\pound{ ф. стерл.}, призначені
на заробітну плату, перебувають у грошовій формі, а 400\pound{ ф.
стерл.} — у формі продуктивного запасу, у формі елементів поточної
частини сталого продуктивного капіталу. А що для другої половини
робочого періоду II, треба лише половини цього продуктивного запасу,
то друга половина його протяюм 3 тижнів перебуває в формі надлишкового
продуктивного запасу, тобто запасу, що перевищує потреби одного
робочого періоду. Але капіталіст знає, що з цієї частини (= 400\pound{ ф. стерл.})
приплилого капіталу для поточного робочого періоду потрібна тільки
половина (= 200\pound{ ф. стерл.}). Отже, від ринкових умов залежатиме, чи
перетворить він знову ці 200\pound{ ф. стерл.} одразу цілком або тільки почасти на
надлишковий продуктивний запас, чи, вичікуючи сприятливих ринкових умов
\parbreak{}  %% абзац продовжується на наступній сторінці

\input{_0209_0210_0211_0212c.tex}
\input{_0213.tex}
\input{_0214.tex}
\input{_0215.tex}
\parcont{}  %% абзац починається на попередній сторінці
\index{ii}{0216}  %% посилання на сторінку оригінального видання
часу обігу, а разом з тим і часу обороту, виділюється в формі грошового
капіталу \sfrac{1}{9} частина авансованого капіталу \deq{} 100\pound{ ф. стерл.} і коли
ці 100\pound{ ф. стерл.} складаються з 20\pound{ ф. стерл.} періодично надлишкового
грошового капіталу, призначеного для виплати щотижневої заробітної
плати, і з 80\pound{ ф. стерл.}, що існують як періодичний надлишковий тижневий
продукційний запас, — то цьому зменшенню у фабриканта надлишкового
продукційного запасу на 80\pound{ ф. стерл.} відповідає збільшення товарового
запасу у торговця бавовною. Та сама бавовна то довше лежить
на його складах як товар, що менше лежить вона на складах у фабриканта
як продукційний запас.

Досі ми припускали, що скорочення часу обігу в підприємстві X випливає
з того, що X швидше продає свої товари або швидше одержує
за них гроші, зглядно, що при кредиті термін виплати скорочується.
Отже, це скорочення часу обігу випливає з швидкого продажу товарів,
швидкого перетворення товарового капіталу на грошовий, з $Т' — Г'$, з
першої фази процесу циркуляції. Воно могло б випливати й з другої фази,
$Г — Т$, а тому й з одночасної зміни, чи то робочого періоду, чи то часу
обігу капіталів Y, Z, і т. ін., що постачають капіталістові X продукційні
елементи його поточного капіталу.

Коли, напр., бавовна, вугілля та ін., в старих умовах транспорту перебувають
8 тижні в дорозі від місця продукції або від складів до місця
підприємства капіталіста X, то мінімуму продукційного запасу X мусить
вистачати принаймні на 3 тижні, поки надійдуть нові запаси. Поки
бавовна та вугілля перебувають в дорозі, вони не можуть служити як
засоби продукції. Вони скоріше становлять тоді предмет праці для транспортової
промисловости й приміщеного в ній капіталу, а також товаровий
капітал для вуглепродуцента або для продавця бавовни, товаровий капітал,
що перебуває в своїй циркуляції. При поліпшеному транспорті час
перевозу скорочується до 2 тижнів. Таким чином, продукційний запас може
перетворитися з тритижневого на двотижневий. Разом з тим звільняється
авансований на це додатковий капітал у 80\pound{ ф. стерл.}, а також 20\pound{ ф. стерл.}, призначені на заробітну плату, бо капітал у 600\pound{ ф. стерл.},
що обернувся, повертається на тиждень раніше.

З другого боку, коли, напр., робочий період капіталу, що постачає
сировинний матеріял, скорочується (приклади про це подано в попередніх
розділах), отже, зростає й можливість відновлювати сировинний матеріял,
то продукційний запас може зменшитись, переміжок від одного періоду
відновлення до другого може скоротитись.

Навпаки, коли час обігу, а тому й період обороту довшає, то потрібне
авансування додаткового капіталу — з кишені самого капіталіста,
коли в нього є додатковий капітал. Але цей капітал є в тій
або іншій формі приміщений, як частина грошового ринку; щоб ним
можна було порядкувати, його треба визволити з старої форми, напр.,
продати акції, взяти вклади, так що й тут постає посередній вплив на
грошовий ринок. Або капіталіст мусить десь позичити додатковий капітал
Щож до частини додаткового капіталу, потрібної для заробітної плати, то
\parbreak{}  %% абзац продовжується на наступній сторінці

\input{_0217.tex}
\input{_0218.tex}
\input{_0219.tex}
\parcont{}  %% абзац починається на попередній сторінці
\index{ii}{0220}  %% посилання на сторінку оригінального видання
\frac{\num{25.000}}{5} = 5000\pound{ ф. стерл}. Коли поділити ці 5000\pound{ ф. стерл.} на 500, то матимемо число оборотів 10,
цілком таке саме, як і для цілого капіталу в 2500\pound{ ф. стерл}.

Це пересічне обчислення, що за ним вартість річного продукту ділиться на вартість авансованого
капіталу, а не на вартість частини цього капіталу, постійно застосовуваної в одному робочому періоді
(отже, в нашому прикладі, не на 400, а на 500, не на капітал І, а на капітал І + капітал II), — це
пересічне обчислення тут, де йдеться лише про продукцію додаткової вартости, є абсолютно точне. Далі
ми побачимо, що, з іншого погляду, воно не зовсім точне, як і взагалі це пересічне обчислення не
зовсім точне. Інакше кажучи, воно задовільне для практичних цілей капіталіста,
але воно не виражає точно й гаразд усіх реальних обставин обороту.

Досі ми одну частину вартости товарового капіталу лишали цілком осторонь, а саме вміщену в ньому
додаткову вартість, спродуковану та долучену до продукту протягом процесу продукції. На неї тепер і
треба нам звернути увагу.

Коли припустити, що витрачуваний щотижня змінний капітал в 100\pound{ ф. стерл.}, продукує додаткову
вартість в 100\% = 100\pound{ ф. стерл.}, то змінний капітал в 500\pound{ ф. стерл.},  витрачуваний протягом
п’ятитижневого періоду обороту, випродукує додаткову вартість в 500\pound{ ф. стерл.}, тобто половина
робочого дня складається з додаткової праці.

Але коли 500\pound{ ф. стерл.} змінного капіталу продукують 500\pound{ ф. стерл.} додаткової вартости, то 5000\pound{ ф.
стерл.} випродукують її 500 × 10 = 5000\pound{ ф. стерл}. Але авансований змінний капітал = 500\pound{ ф. стерл}.
Відношення всієї маси додаткової вартости, спродукованої протягом року, до суми вартости
авансованого змінного капіталу ми звемо річною нормою додаткової вартости. Отже, в даному випадку,
вона = \frac{5000}{500} = 1000\%.
Коли ближче аналізувати цю норму, то виявиться, що вона дорівнює тій нормі додаткової вартости, яку
авансований змінний капітал продукує протягом одного періоду обороту, помноженій на число оборотів
змінного капіталу (а воно збігається з числом оборотів цілого обігового капіталу).

Авансований протягом одного періоду обороту змінний капітал в даному випадку = 500\pound{ ф. стерл.};
створена ним додаткова вартість теж = 500\pound{ ф. стерл}. Тому норма додаткової вартости протягом одного
періоду обороту = \frac{500m}{500v} = 100\%. Ці 100\%, помножені на 10, на число оборотів протягом року,
дають \frac{5000m}{5000v} = 1000\%.

Це має силу щодо річної норми додаткової вартости. Щождо маси додаткової вартости, здобуваної
протягом певного періоду обороту, то ця маса дорівнює вартості авансованого протягом цього періоду
змінного капіталу — в даному випадку = 500\pound{ ф. стерл.}, помноженій на норму
\parbreak{}  %% абзац продовжується на наступній сторінці

\parcont{}  %% абзац починається на попередній сторінці
\index{ii}{0221}  %% посилання на сторінку оригінального видання
додаткової вартости, в даному випадку, отже, 500 × \frac{100}{100} \deq{} 500 × 1 \deq{} 500\pound{ ф. стерл}. Коли б
авансований капітал був \deq{} 1500\pound{ ф. стерл.} при незмінній
нормі додаткової вартости, то маса додаткової вартости була б \deq{}
1500 × \frac{100}{100} \deq{} 1500\pound{ ф. стерл}.

Змінний капітал у 500\pound{ ф. стерл.}, що обертається 10 разів на рік, і
що продукує протягом року додаткову вартість в 5000\pound{ ф. стерл.}, отже,
капітал, що для нього річна норма додаткової вартости \deq{} 1000\%, ми
будемо називати капіталом $А$.

Припустімо тепер, що інший змінний капітал $В$ в 5000\pound{ ф. стерл.}
авансується на цілий рік (тобто, тут на 50 тижнів) і тому обертається
лише один раз на рік. Припустімо при цьому далі, що наприкінці року
продукт оплачується в той самий день, як його виготовлено, і, значить,
грошовий капітал, що на нього його перетворюється, повертається в той
самий день. Отже, період циркуляції тут \deq{} 0, період обороту дорівнює
робочому періодові, а саме, одному рокові. Як і в попередньому випадку,
в процесі праці щотижня перебуває змінний капітал в 100\pound{ ф. стерл.},
а тому протягом 50 тижнів — в 5000\pound{ ф. стерл}. Далі, норма додаткової
вартости хай буде та сама \deq{} 100\%, тобто за однакової довжини робочого
дня половина його складається з додаткової праці. Коли ми візьмемо
5 тижнів, то вкладений змінний капітал \deq{} 500\pound{ ф. стерл.}, норма додаткової
вартости \deq{} 100\%, отже, маса додаткової вартости, створена протягом
5 тижнів \deq{} 500\pound{ ф. стерл}. Кількість робочої сили, що її тут експлуатується,
і ступінь її експлуатації, згідно з нашим припущенням, тут
точно такі самі, як і при капіталі $А$.

Вкладений змінний капітал в 100\pound{ ф. стерл.} щотижня створює додаткову
вартість в 100\pound{ ф. стерл.}, тому протягом 50 тижнів вкладений капітал
в 100 × 50 \deq{} 5000\pound{ ф. стерл.} створить додаткову вартість в 5000\pound{ ф. стерл}. Маса щороку створюваної
додаткової вартости буде така сама, як і в попередньому випадку \deq{} 5000\pound{ ф. стерл.}, але річна норма
додаткової
вартости цілком інша. Вона дорівнює спродукованій протягом року
додатковій вартості, поділеній на авансований змінний капітал:
\frac{5000m}{5000v} \deq{} 100\%, тимчасом як раніш для капіталу $А$ вона дорівнювала 1000\%.

При капіталі $А$, як і при капіталі $В$, ми витрачали щотижня 100\pound{ ф. стерл.} змінного капіталу; ступінь
зростання вартости або норма додаткової
вартости цілком та сама, вона дорівнює 100\%; величина змінного
капіталу теж та сама \deq{} 100\pound{ ф. стерл}. Експлуатується цілком таку
саму кількість робочої сили, величина й ступінь експлуатації в обох випадках
однакові, робочі дні однакові і однаково поділяються на доконечну
й додаткову працю. Сума змінного капіталу, застосованого протягом
року, однакова величиною \deq{} 5000\pound{ ф. стерл.}, вона пускає в рух таку
саму масу праці й витягує з робочої сили, пущеної в рух обома рівними
капіталами, однакову масу додаткової вартости, 5000\pound{ ф. стерл}. І,
\parbreak{}  %% абзац продовжується на наступній сторінці

\input{_0222.tex}
\parcont{}  %% абзац починається на попередній сторінці
\index{i}{0223}  %% посилання на сторінку оригінального видання
обставини, заробітну плату зменшено щонайменше на 25\%\footnote{
«Я переконався, що в людей, які діставали 10\shil{ шилінґів} тижнево,
скорочено плату на 1\shil{ шилінґ} у наслідок загального зменшення заробітної
плати на 10\%, а потім іще на 1\shil{ шилінґ} 6\pens{ пенсів} через скорочення часу,
разом на 2\shil{ шилінґи} 6\pens{ пенсів}, і, не зважаючи на це, більшість твердо обстоювала
десятигодинний біл». (Там же).
}.
За таких, так сприятливо підготованих обставин розпочалась
аґітація серед робітників за скасування закону з 1847~\abbr{р.} Не гребували
жодним з тих засобів, що їх можуть дати омана, спокуса,
погрози, — та все було даремно. Щодо півтузіня петицій, в яких
робітники мусили нарікати на те, «що їх утискує цей закон», то
сами ж прохачі на усному допиті заявили, що ті підписи від них
вимушено. «Їх утискує, але хтось інший, а не фабричний закон»\footnote{
«Коли я підписував прохання, то я одночасно заявив, що роблю
цим щось погане. — Так чого ж ви його тоді підписували? — Тому, що коли б
я спротивився, мене б були викинули на вулицю. — Прохач дійсно почував
«себе утисненим», але зовсім не фабричним законом». (Там же, стор. 102).
}.
Але якщо фабрикантам не пощастило примусити робітників
говорити в бажаному для них тоні, то тим голосніш вони сами
кричали від імени робітників у пресі і в парляменті. Вони виставляли
фабричних інспекторів як щось на зразок комісарів конвенту,
які без милосердя приносять нещасних робітників у жертву
своїй химері про поліпшення світу. Але й цей маневр не мав
успіху. Фабричний інспектор Леонард Горнер особисто й через
своїх підінспекторів паназбирав численні свідчення на фабриках
Ланкашіру. Якихось 70\% переслуханих робітників висловилося
за 10-годинний робочий день, куди менший відсоток за 11-годинний
і зовсім незначна меншість за старий 12-годинний день\footnote{
Там же, стор. 17. Таким чином в окрузі пана Горнера переслухано
\num{10.270} дорослих робітників-чоловіків на 181 фабриці. Їхні свідчення
можна найти в додатку до фабричного звіту за півріччя, що кінчається
жовтнем 1848~\abbr{р.} Ці свідчення дають і з іншого боку цінний матеріял.
}.

Інший «полюбовний» маневр був такий: примусити дорослих
чоловіків-робітників працювати 12--15 годин, а потім оголосити
цей факт за найкращий вислів щиро пролетарських бажань.
Але «немилосердий» фабричний інспектор Леонард Горнер знову
вже був на своєму місці. Більшість робітників, що працювали
«надмірний час», висловилася, «що їм куди краще було б працювати
по 10 годин за меншу заробітну плату, але для них не було
жодного вибору: серед них так багато безробітних, так багато
прядунів, примушених працювати як звичайні piercers\footnote*{
присукальники. \emph{Ред.}
}, що
коли б вони відмовилися від здовження робочого дня, так зараз
на їхні місця прийшли б інші, так що для них, мовляв, справа
стоїть так: або працювати довший час, або опинитись на вулиці»\footnote{
Там же. Див. зібрані від самого Леонарда Горнера свідчення
№№ 69, 70, 71, 72, 92, 93 і ті, що зібрав підінспектор А., №№ 51, 52,
58, 59, 62, 70 в «Додатку». Навіть один фабрикант висловився занадто
ясно. Див. № 14 після № 265, там же.
}.

Попередній похід капіталу скінчився невдало, і закон про десятигодинний
робочий день набув чинности 1 травня 1848~\abbr{р.} Тимчасом
поразка чартистської партії, що її проводирів позамикано
\parbreak{}  %% абзац продовжується на наступній сторінці

\parcont{}  %% абзац починається на попередній сторінці
\index{ii}{0224}  %% посилання на сторінку оригінального видання
вартости, хоч яке різне буде відношення цього змінного капіталу, застосованого
протягом певного часу, до змінного капіталу, авансованого на
той самий час, і значить, хоч яке різне буде також і відношення утворених
мас додаткової вартости, не до застосованого, а до взагалі авансованого
змінного капіталу. Неоднаковість цього відношення, замість суперечити
розвиненим законам продукції додаткової вартости, навпаки,
потверджує їх і є неминучий наслідок їх.

Розгляньмо перший п’ятитижневий період продукції капіталу \emph{В}. Наприкінці
5-го тижня 500\pound{ ф. стерл.} застосовано й зужито. Новостворена
вартість \deq{} 1000\pound{ ф. стерл.}, отже, $\frac{500m}{500v} \deq{} 100\%$. Цілком так, як при капіталі \emph{А}.
Та обставина, що при капіталі \emph{А} додаткова вартість реалізується разом з
авансованим капіталом, а при \emph{В} — ні, нас тут покищо не обходить, бо
тут покищо йдеться лише про продукцію додаткової вартости і про
відношення її до змінного капіталу, авансованого під час її продукції.
Навпаки, коли ми обчислимо відношення додаткової вартости \emph{В} не
до тієї частини авансованого капіталу в 5000\pound{ ф. стерл.}, що її застосовано й
тому зужито протягом продукції цієї додаткової вартости, а до самого
цього цілого авансованого капіталу, то матимемо $\frac{500m}{5000v} \deq{} \sfrac{1}{10} \deq{} 10\%$.
Отже, для капіталу \emph{В} 10\%, а для капіталу \emph{А} 100\%, тобто вдесятеро
більше. Коли б тут сказали: така ріжниця в нормі додаткової вартости
для однакових величиною капіталів, що пускають у рух однакову кількість
праці, та ще праці, яка однаковою мірою поділяється на оплачену й
неоплачену, суперечить законам продукції додаткової вартости, — то
відповідь була б проста й випливала б з першого погляду на фактичні
відношення: для \emph{А} виражається справжня норма додаткової вартости,
тобто відношення додаткової вартости, спродукованої протягом 5 тижнів
змінним капіталом в 500 ф., до цього змінного капіталу в 500\pound{ ф. стерл}.
Для \emph{В}, навпаки, обчислення робиться таким способом, що не має жодного
чинення ні до продукції додаткової вартости, ні до відповідного їй
визначення норми додаткової вартости. 500\pound{ ф. стерл.} додаткової вартости,
спродуковані змінним капіталом у 500\pound{ ф. стерл.}, обчислюється
власне не в їхньому відношенні до 500\pound{ ф. стерл.} змінного капіталу, авансованого
протягом продукції цієї додаткової вартости, а в їхньому відношенні
до капіталу в 5000\pound{ ф. стерл.}, що \sfrac{9}{10} його, 4500\pound{ ф. стерл.}, не мають
жодного чинення до продукції цієї додаткової вартости в 500\pound{ ф. стерл.}, а
скорше мають лише поступінно функціонувати протягом наступних 45 тижнів;
отже, вони зовсім не існують для продукції протягом перших 5 тижнів,
що про них тільки й мовиться тут. Отже, в цьому випадку ріжниця
в нормі додаткової вартости капіталів \emph{А} і \emph{В} не становить жодної
проблеми.

Порівняймо тепер річні норми додаткової вартости для капіталів \emph{В} і \emph{А}.
Для капіталу \emph{В} ми маємо $\frac{5000m}{5000v} \deq{} 100\%$;
для капіталу \emph{А} $\frac{5000m}{500v} \deq{} 1000\%$.
\parbreak{}  %% абзац продовжується на наступній сторінці

\input{_0225.tex}
\input{_0226.tex}

\index{i}{0227}  %% посилання на сторінку оригінального видання
Звичайно, всі ці викрути нічого не помогли. Фабричні інспектори
вдалися до суду. Але незабаром на міністра внутрішніх
справ сера Джорджа Грея спала така хмара петицій від фабрикантів,
що в обіжнику з 5 серпня 1848~\abbr{р.} він наказав інспекторам
«не позивати взагалі за порушення букви закону, поки не буде
доведене зловживання Relaissystem’ою з метою примусити підлітків
і жінок працювати понад десять годин». Після цього фабричний
інспектор Ф.~Стюарт дозволив так звану систему змін протягом
п’ятнадцятигодинного періоду фабричного дня для цілої Шотляндії,
де вона незабаром знов розцвіла, як колись. Навпаки,
англійські фабричні інспектори заявили, що міністер не має
жодної диктаторської влади припинити чинність закону, і далі
провадили судові переслідування проти Proslavery rebels.

Алеж нащо було притягати до суду, коли суди, county magistrates\footnote{
Ці «county magistrates», «great unpaid»\footnote*{величні неоплачувані.  \emph Ред. },
як їх називає В.~Кобе — це щось наче безплатні мирові судді, що їх обирають із почесних
осіб графства. В дійсності вони являють собою патримоніяльні суди панівних
кляс.
},
виправдували притягуваних до права? По цих судах
засідали пани фабриканти, щоб самих себе судити. Ось приклад.
Якийсь Іскрідж із бавовнопрядної фірми Кершоу, Лізе і К°
подав був фабричному інспекторові своєї округи схему Relaissystem,
призначену для його фабрики. Одержавши відмову,
він спочатку тримався пасивно. Декілька місяців пізніш якийсь
індивід, на ім’я Робінзон, теж бавовняник, і коли не П’ятниця,
то в усякому разі родич Іскріджа, став перед Borough Justices\footnote*{мировими суддями. \emph {Ред.}}
у Стокпорті, обвинувачуваний у тому, що завів у себе таку
систему змін, яку вигадав Іскрідж. Засідало четверо суддів,
серед них три бавовняні фабриканти, з тим самим неминучим
Іскріджем на чолі. Іскрідж виправдав Робінзона й заявив: що є
законоправне для Робінзона, те справедливе й для Іскріджа.
Покликаючись на свій власний судовий присуд, що набрав правної
сили, він зараз же завів цю систему й на своїй власній фабриці\footnote{
«Reports etc. for 30 th April 1849», p. 21, 22. Порівн. подібні
приклади там же, \stor{}4, 5.
}.
Певна річ, уже самий склад таких суддів був явним порушенням
закону\footnote{
Законом 1 і 2 Вільяма IV, с. 24, s. 10, відомим під назвою фабричного
закону сера Джона Гобговза, забороняється кожному посідачеві
бавовнопрядної або ткацької фабрики, а також і батькові, синові або
братові такого посідача виконувати обов'язки мирового судді в питаннях,
які стосуються до фабричного закону.
}. «Такі судові фарси, — каже інспектор Хоуелл, —
аж волають по ліки\dots{} або пристосуйте закон до таких присудів,
або віддайте вирішення справ не такому вже порочному трибуналові,
який свої присуди пристосував би до закону\dots{} в усіх таких
випадках. Дуже бажано, щоб посада судді була платна!»\footnote{
«Reports etc. for 30 th April 1849».
}

Коронні юристи проголосили фабрикантську інтерпретацію
закону 1848~\abbr{р.} за недоладну, але рятівники суспільства не дали
\parbreak{}  %% абзац продовжується на наступній сторінці

\parcont{}  %% абзац починається на попередній сторінці
\index{i}{0228}  %% посилання на сторінку оригінального видання
себе збити з пантелику. «Після того, — оповідає Леонард Горнер, —
як я спробував примусити виконувати закон, розпочавши 10 процесів
у 7 різних судових округах, і лише в одному випадку найшов
підтримку в суддів\dots{} я вважаю за некорисні дальші переслідування
за оминання закону. Та частина закону, що її укладено
з метою створити одностайність у робочих годинах\dots{} вже не існує
більше в Ланкашірі. Так само я абсолютно не маю, як і мої помічники,
ніяких засобів, щоб упевнитися, що по тих фабриках, де
панує так звана Relaissystem, підлітків і жінок не примушують
працювати більш як 10 годин. Наприкінці квітня 1849~\abbr{р.} вже
114 фабрик у моїй окрузі працювали за цією методою, і число
їх останніми часами швидко зростає. Загалом же вони працюють
тепер 13\sfrac{1}{2} годин, від шостої години ранку до пів на восьму вечора;
в деяких випадках вони працюють 15 годин, від пів на шосту
ранку до пів на дев’яту вечора»\footnote{
«Reports etc. for 30 th April 1849», p. 5.
}. Вже у грудні 1848~\abbr{р.} Леонард
Горнер мав список 65 фабрикантів і 29 фабричних доглядачів,
які одноголосно заявляли, що жодна система контролю не може
за такої системи змін перешкодити поширенню якнайінтенсивнішої
надмірної праці\footnote{
«Reports etc. for 31 st October 1849», p. 6.
}. То тих самих дітей і підлітків переводять
із прядільні до ткальні й~\abbr{т. д.}, то протягом 15 годин їх
кидають (shifted) з однієї фабрики до однієї\footnote{
«Reports etc. for 30 th April 1849», p. 21.
}. Як можна контролювати
таку систему змін, «яка зловживає словом зміна, щоб із
безмежною різноманітністю перемішувати робочі руки, як карти,
і день-у-день так пересовувати години праці й відпочинку
різних осіб, що один і той самий повний асортимент рук ніколи
не працює разом на тому самому місці в той самий час»!\footnote{
«Reports etc. for 1 st December 1848», p. 95.
}

Але й залишаючи цілком осторонь дійсну надмірну працю,
ця так звана система змін була таким витвором фантазії капіталу,
що його ніколи не перевищив Фур’є у своїх гумористичних
нарисах «courtes séances»\footnote*{
коротких сеансів. \emph{Ред.}
}, з тією лише ріжницею, що притягання
праці тут перетворилося на притягання капіталу. Подивімось
на ці схеми, утворені фабрикантами і прославлені добрячою
пресою як зразок того, «що можна зробити з розумною мірою
дбайливости й методичности» («what a reasonable degree of care
and method can accomplish»). Робочий персонал розділювано
іноді на 12--15 категорій, що знову раз-у-раз зміняли свої
складові частини. Протягом п’ятнадцятигодинного періоду фабричного
дня капітал притягав робітника то на 30 хвилин, то на
годину, потім знову відштовхував його, щоб знову притягти
його на фабрику й знов одштовхнути, ганяючи його то туди, то сюди
розрізненими шматками часу, але постійно не випускаючи його
із своїх рук, доки десятигодинну працю не буде цілком закінчено.
Як на театральній сцені, мали виступати ті самі особи навпереміну
в різних явах різних дій. Але, як актор належить до
\parbreak{}  %% абзац продовжується на наступній сторінці

\parcont{}  %% абзац починається на попередній сторінці
\index{i}{0229}  %% посилання на сторінку оригінального видання
сцени протягом усього часу тривання драми, так і робітники належали
тепер до фабрики протягом 15 годин, не рахуючи часу на
дорогу до фабрики й назад. Таким чином години відпочинку перетворювалися
на години примусового безділля, що гнали молодого
робітника до шинку, а молоду робітницю в дім розпусти. За
кожної нової витівки, що її день-у-день вигадував капіталіст,
щоб тримати свої машини в русі 12 або 15 годин, не збільшуючи
робочого персоналу, робітник мусів проковтнути свою їжу то в
той, то в інший шматок часу. Під час агітації за десятигодинний
робочий день фабриканти кричали, що робітнича наволоч подає
петиції, сподіваючись дістати за десятигодинну працю дванадцятигодинну
заробітну плату. Тепер вони обернули медалю. Вони
виплачували десятигодинну заробітну плату за дванадцяти й
п’ятнадцятигодинне порядкування робочими силами!\footnote{
Див. «Reports etc. for 30 th April 1849», p. 6 і докладне пояснення
«shifting system»\footnote*{
системи пересувань. \emph{Ред.}
}, яке фабричні інспектори Хоуелл і Савндер дають
у «Reports etc. for 31 st October 1848». Див. також петицію проти
«shift system», подану королеві духівництвом Ashton’a й околиць на весні
1849~\abbr{р.}
} Так ось
у чім була річ; це було фабрикантське видання десятигодинного
закону! Це були ті самі фритредери, сповнені благодаті й любови
до людства, що підчас аґітації проти хлібних законів цілих десять
років до останнього шага обчислювали робітникам, що за вільного
довозу хліба, при тих засобах, що їх має англійська промисловість,
цілком досить було б десяти годин праці, щоб збагатити
капіталістів\footnote{
Порівн., наприклад, «The Factory Question and the Ten Hours
Bill. By R.~H.~Greg. 1837».
}.

Дворічний бунт капіталу увінчався нарешті присудом однієї
з чотирьох вищих судових установ Англії, Court of Exchequer,
який в одному з випадків, що дійшов до нього, 8 лютого 1850~\abbr{р.}
вирішив, що хоч фабриканти й чинили проти змісту закону
1844~\abbr{р.}, але самий цей закон містить у собі деякі слова, що роблять
його безглуздим. «Цей вирок знищив закон про десятигодинну
працю»\footnote{
\emph{F.~Engels}: «Die englische Zehnstundenbill» (у видаваній мною
«Neue Rheinische Zeitung». Politish-ökonomische Revue, Aprilheft
1850», p. 13). Той самий «високий» суд так само винайшов підчас американської
громадянської війни словесну зачіпку, яка перетворювала закон
проти озброєння піратських кораблів у його пряму протилежність.
}. Маса фабрикантів, що досі боялись застосовувати
систему змін для підлітків і робітниць, ухопилися за неї тепер
обома руками\footnote{
«Reports etc. for 30 th April 1850».
}.

Але за цією, здавалось, остаточною перемогою капіталу
настав зараз же поворот. Робітники досі ставили пасивний, хоч
і впертий і день-у-день відновлюваний опір. Тепер вони почали
голосно протестувати на загрозливих мітинґах у Ланкашірі і
Йоркшірі. Значить, так званий десятигодинний закон — це лише
ошуканство, парляментське шахрайство, а на ділі він ніколи не
існував! Фабричні інспектори пильно попереджали уряд, що
\parbreak{}  %% абзац продовжується на наступній сторінці

\input{_0230.tex}
\input{_0231.tex}

  \parcont{}  %% абзац починається на попередній сторінці
\index{ii}{0232}  %% посилання на сторінку оригінального видання
періоду обороту на робочу силу вже витрачено 500\pound{ ф. стерл}. Ці 500\pound{ ф. стерл.},
що первісно були частиною всього авансованого капіталу, перестали бути
капіталом. Їх витрачено на заробітну плату. Робітники, з свого боку,
виплачують ними, купуючи собі засоби існування; отже, вони споживають
засоби існування вартістю на 500\pound{ ф. стерл}. Отож знищено товарову
масу на таку саму суму вартости (те, що робітник зберігає, напр., як
гроші тощо, теж не є капітал). Цю товарову масу спожито непродуктивно
для робітника, оскільки вона тільки підтримує в стані працездатности
його робочу склу, тобто неодмінно потрібне знаряддя капіталіста. Але,
подруге, для капіталіста ці 500\pound{ ф. стерл.} перетворено на робочу силу тієї
самої вартости (зглядно ціни). Капіталіст продуктивно споживає робочу
силу в процесі праці. Наприкінці 5 тижня є новоспродукована вартість
в 1000\pound{ ф. стерл}. Половина її, 500\pound{ ф. стерл.}, є репродукована вартість змінного
капіталу, витраченого на оплату робочої сили. Друга половина її,
500\pound{ ф. стерл.} є новоспродукована додаткова вартість. Але ту п’ятитижневу
робочу силу, що в наслідок обміну на неї частина капіталу перетворилась
на змінний капітал, так само витрачено, зужито, хоч і продуктивно.
Праця, що діяла вчора, не є та сама праця, що діє сьогодні. Її
вартість плюс утворена нею додаткова вартість існує тепер як вартість
продукту, речі, відмінної від самої робочої сили. Однак, у наслідок того,
що продукт перетворюється на гроші, частина його вартости, що дорівнює
вартості авансованого змінного капіталу, знову може бути
обмінена на робочу силу, а тому й знову функціонувати як змінний капітал.
Та обставина, що на капітальну вартість, не лише репродуковану,
а й перетворену вже на грошову форму, вживатиметься тих самих робітників,
тобто тих самих носіїв робочої сили, не має жодного значення.
Можливо, що протягом другого періоду обороту капіталіст вживатиме
нових робітників замість старих.

Отже, у дійсності протягом 10 п’ятитижневих періодів обороту на
заробітну плату послідовно витрачається капітал в 5000\pound{ ф. стерл.}, а
не 500\pound{ ф. стерл.}, і цю заробітну плату знову витрачають робітники
на засоби існування. Авансований таким чином капітал в 5000\pound{ ф. стерл.} зужито. Він уже не існує. З другого боку, в процес продукції
послідовно вводиться робочу силу вартістю не в 500, а в 5000\pound{ ф. стерл.},
і вона репродукує не лише свою власну вартість, рівну 5000\pound{ ф. стерл.},
але продукує надлишок, додаткову вартість в 5000\pound{ ф. стерл}. Змінний
капітал в 500\pound{ ф. стерл.}, що авансується на другий період обороту, не тотожній
з капіталом в 500\pound{ ф. стерл.}, що його авансовано в перший період
обороту. Цей останній зужито, витрачено на заробітну плату. Але його
\so{заміщено} новим змінним капіталом в 500\pound{ ф. стерл.}, що його в перший
період обороту спродуковано в формі товару і зворотно перетворено
на грошову форму. Отже, цей новий грошовий капітал в 500\pound{ ф. стерл.} є
грошова форма товарової маси, новоспродукованої в перший період обороту.
Та обставина, що в руках капіталіста знову є тотожня грошова
сума в 500\pound{ ф. стерл.}, тобто, лишаючи осторонь додаткову вартість, саме
стільки грошового капіталу, скільки він первісно авансував, замасковує
\parbreak{}  %% абзац продовжується на наступній сторінці

\parcont{}  %% абзац починається на попередній сторінці
\index{ii}{0233}  %% посилання на сторінку оригінального видання
ту обставину, що капіталіст оперує новоспродукованим капіталом (щодо
інших складових частин вартости товароваго капіталу, які заміщують
сталі частини капіталу, то їхня вартість не продукується наново, а змінюється
лише форма, що в ній існує ця вартість). Візьмімо третій період
обороту. Тут очевидно, що капітал в 500\pound{ ф. стерл.}, авансований втретє,
не є старий, а новоспродукований капітал, бо він є грошова форма
товарової маси, спродукованої в другий, а не в перший період обороту,
тобто грошова форма тієї частини цієї товарової маси, вартість якої
дорівнює вартості авансованого змінного капіталу. Масу товарів, спродуковану
в перший період обороту, продано. Частину її вартости, що дорівнює
змінній частині авансованого капіталу, обмінено на нову робочу
силу для другого періоду обороту; ця частина випродукувала нову масу
товарів, що її теж продано; саме частина вартости цієї нової маси
товарів становить капітал в 500\pound{ ф. стерл.}, авансовуваний у третій період
обороту.

І це повторюється протягом 10 періодів обороту. Протягом їх новоспродуковані
маси товарів (вартість яких, оскільки вона заміщує змінний
капітал, теж знову продукується, а не просто з’являється, як з’являється
вартість сталої обігової частини капіталу), що п’ять тижнів подається
на ринок, щоб знову вводити робочу силу в процес продукції.

Отже, десятиразовим оборотом авансованого змінного капіталу в 500\pound{ ф.
стерл.} досягається не те, що цей капітал в 500\pound{ ф. стерл.} можна продуктивно
спожити десять разів, або, що змінний капітал, достатній для
п’ятьох тижнів, можна застосувати протягом 50 тижнів. Навпаки, за 50
тижнів застосовується 500\pound{ ф. стерл}. X 10 змінного капіталу, і капіталу в
500\pound{ ф. стерл.} завжди вистачає тільки на 5 тижнів, а по п’ятьох тижнях
його доводиться заміщувати новоспродукованим капіталом в 500\pound{ ф. стерл}.
Це стосується до капіталу \emph{А} цілком так само, як і до капіталу \emph{В}. Але
відси починається ріжниця.

На кінець першого п’ятитижневого періоду від капіталіста \emph{В}, як і
від капіталіста \emph{А}, авансовано й витрачено змінний капітал в 500\pound{ ф. стерл}.
І~\emph{В} і \emph{А} перетворили його вартість на робочу силу й замістили цю вартість
тією частиною новоспродукованої цією робочою силою вартости продукту,
яка дорівнює вартості авансованого змінного капіталу в 500\pound{ ф. стерл}. Для
\emph{В}, як і для \emph{А}, робоча сила не лише замістила вартість витраченого змінного
капіталу в 500\pound{ ф. стерл.} новою вартістю на таку саму суму, а й додала
до неї додаткову вартість, — за нашим припущенням вартість такої
самої величини.

Але та новоутворена вартість, що заміщує авансований змінний капітал
і додає до його вартости додаткову вартість, перебуває у \emph{В} не в тій
формі, що в ній вона може знову функціонувати як продуктивний капітал,
зглядно як змінний капітал. Для \emph{А} вона перебуває саме в такій формі.
І до кінця року \emph{В} володіє змінним капіталом, витрачуваним протягом
перших 5 тижнів і потім послідовно витрачуваним що п’ять тижнів, — хоч
його й заміщується новоспродукованою вартістю плюс додаткова вартість,
— не в тій формі, що в ній він знову може функціонувати
\parbreak{}  %% абзац продовжується на наступній сторінці

\input{_0234.tex}
\input{_0235c.tex}
\input{_0236.tex}
\parcont{}  %% абзац починається на попередній сторінці
\index{ii}{0237}  %% посилання на сторінку оригінального видання
застосованих у \emph{В}, і на заміщення цього всього подається на ринок еквівалент
в формі грошей; але протягом цього року на ринок не подається
жодного продукту, щоб замістити взяті з ринку речові елементи продуктивного
капіталу. Коли ми уявимо собі не капіталістичне суспільство,
а комуністичне, то, насамперед, зовсім відпадає грошовий капітал, а значить,
і всі ті маскування оборудок, які постають через грошовий капітал.
Справа сходить просто на те, що суспільство мусить наперед обчислити,
скільки праці, засобів продукції та засобів існування воно може без якої-будь
шкоди витрачати на такі галузі продукції, що, як от, напр., будування
залізниць, довгий час, рік або й більше, не дають ні засобів
продукції, ні засобів існування, ні взагалі будь-якого корисного ефекту,
але звичайно відбирають від цілої річної продукції працю, засоби продукції
і засоби існування. Навпаки, в капіталістичному суспільстві, де
суспільний розум завжди виявляє себе тільки post festum\footnote*{
Post festum — дослівно: „після свята“, коли справу вже вакінчено. \emph{Ред.}
}, можуть і мусять
завжди поставати великі порушення. З одного боку, тиск на грошовий
ринок, тимчасом як гарний стан грошового ринку, навпаки, і собі покликає
до життя багато таких підприємств, отже, призводить саме до таких обставин,
що потім зумовлюють тиск на грошовий ринок. Грошовий ринок
зазнає тиску, бо при цьому треба постійно авансувати великий грошовий
капітал на довгий час. Ми вже зовсім не кажемо про те, що промисловці
й торговці кидають на залізничні спекуляції тощо грошовий капітал,
потрібний їм для провадження власних підприємств, і заміщують його позиками
на грошовому ринку. — З другого боку, зазнає тиску продуктивний капітал,
що є в розпорядженні суспільства. А що елементи продуктивною капіталу
постійно вилучається з ринку і натомість на ринок подається лише
грошовий еквівалент, то більшає виплатоспроможний попит, який, із свого
боку, не має в собі жодних елементів подання. Відси підвищення цін
і на засоби існування, і на продукційні матеріяли. До цього долучається
ще й те, що під такий час звичайно розвивається шахрайство і переміщується
чимало капіталу. Зграя спекулянтів, постачальників, інженерів,
адвокатів тощо збагачується. Вони спричиняють на ринку великий попит
на речі споживання, поряд цього підвищується заробітна плата. Щодо
попиту на харчові засоби, то він звичайно підганяє й сільське господарство.
А що цих харчових засобів не можна збільшити одразу, протягом
року, то більшає довіз їх, як і взагалі довіз екзотичних харчових
засобів (кави, цукру, вина тощо) та речей розкошів. Звідси надмірний
довіз і спекуляція в цій галузі імпортної торговлі. З другого боку, в
тих галузях промисловости де продукцію можна швидко збільшити (власне
мануфактура, гірництво тощо), підвищення цін призводить до раптового
поширення, що по ньому скоро настає крах. Такий самий вплив
справляється на робочий ринок, щоб притягти до нових галузей підприємств
великі маси лятентного відносного надміру людности і навіть робітників,
уже зайнятих. Взагалі такі підприємства великого маштабу, як
от залізниці, відтягують від робочого ринку певну кількість сил, що
\parbreak{}  %% абзац продовжується на наступній сторінці

\parcont{}  %% абзац починається на попередній сторінці
\index{ii}{0238}  %% посилання на сторінку оригінального видання
можуть вийти лише з певних галузей, як, напр., сільське господарство тощо,
де працюють виключно дужі парубки. Це діється й після того, як нові підприємства
стали вже постійною галуззю продукції і, значить, після того,
як уже утворилась потрібна для них бродяча робітнича кляса. Напр.,
коли залізниця раптом почне будуватись у ширшому від пересічного
маштабі. Тоді вбирається частину резервної армії робітників, що її тиск
тримав заробітну плату на порівняно низькому рівні. Тоді заробітна плата
скрізь підвищується, навіть у тих частинах робочого ринку, де робітники
й раніш легко знаходили собі працю. Це триває доти, доки неминучий
крах знову звільняє резервну армію робітників, і заробітну плату
знову знижується до її мінімуму й нижче.\footnote{
В рукопису тут вставлено таку замітку, щоб пізніш її розвинути: „Суперечність
в капіталістичному способі продукції: робітники як покупці товару,
важать для ринку. Але як продавців свого товару — робочої сили капіталістичне
суспільство має тенденцію обмежувати їх мінімумом ціни. Дальша суперечність:
ті епохи, коли капіталістична продукція напружує всі свої сили, регулярно з’являються
як епохи перепродукції, бо продуктивні сили ніколи не можна застосувати
так, щоб у наслідок цього можна було не лише випродукувати, а й зреалізувати
більше вартости; але продаж товарів, реалізація товарового капіталу, отже,
і додаткової вартости, обмежена не просто споживними потребами суспільства
взагалі, з споживними потребами такого суспільства, що його переважна
більшість завжди бідна й мусить завжди лишатися бідною. Однак це стосується
лише до наступного відділу.“ \emph{Ф.~Е.}
}

Оскільки більший або менший протяг періоду обороту залежить від
робочого періоду у власному значенні, тобто від періоду, потрібного на
те, щоб виготувати продукт для ринку, він ґрунтується на кожного
разу даних речових умовах продукції різних капіталовкладень, на
умовах, що в хліборобстві мають більше характер природних умов продукції,
а в мануфактурі і в більшій частині видобувної промисловости
змінюються разом із суспільним розвитком самого продукційного процесу.

Оскільки протяг робочого періоду ґрунтується на величині поставок
(на кількісному розмірі, що в ньому продукт звичайно подається на ринок
як товар), він має умовний характер. Але сама ця умовність має за
матеріяльну базу розміри продукції, а тому вона є випадкова лише остільки,
оскільки ми розглядаємо її ізольовано.

Нарешті, оскільки протяг періоду обороту залежить від протягу періоду
циркуляції, він почасти зумовлюється постійною зміною ринкових
коньюнктур, більшою або меншою легкістю продажу і неминучою, відси
посталою, потребою подавати частину продукту на ближчий або дальший
ринок. Лишаючи осторонь розмір попиту взагалі, рух цін відіграє
тут головну ролю, оскільки при зниженні цін продаж навмисно обмежується,
тимчасом як продукція розвивається далі; навпаки буває при
підвищенні цін, коли продукція та продаж не відстають одне від одного,
або коли продаж може відбуватися заздалегідь. Однак, за власне матеріяльну
базу треба вважати справжнє віддалення місця продукції від
ринку збуту.


\index{ii}{0239}  %% посилання на сторінку оригінального видання
Напр., англійську бавовняну тканину або пряжу продається до Індії.
Припустімо, що купець-експортер платить англійському фабрикантові
бавовни (купець-експортер охоче робить це лише при доброму стані грошового
ринку. А якщо сам фабрикант поповнює свій грошовий капітал
за допомогою кредитових операцій, то справа вже кепська). Експортер
продає потім свій бавовняний товар на індійському ринку, відки йому
повертається авансований ним капітал. До цього повороту справа стоїть
цілком так само, як і в тому випадку, коли протяг робочого періоду
примушує авансувати новий грошовий капітал, щоб підтримувати провадження
процесу продукції в даному розмірі. Грошовий капітал, що ним
фабрикант платить своїм робітникам, а також відновлює всі інші елементи
свого капіталу, не є грошова форма спродукованої ним пряжі.
Це може статись лише тоді, коли вартість цієї пряжі повернеться в Англію
як гроші або продукт. Як і раніше, ці гроші є додатковий грошовий
капітал. Ріжниця лише в тому, що замість фабриканта їх авансує
купець, що, можливо, й сам здобув їх за допомогою кредитових операцій.
Так само, перш ніж ці гроші подається на ринок, або одночасно з цим,
на англійський ринок не подано додаткового продукту, що його можна
купити на ці гроші і ввести в сферу продуктивного або особистого споживання.
Коли такий стан триває довго і в широкому маштабі, то він
мусить зумовити такі самі наслідки, які раніше зумовлювало подовження
робочого періоду.

Можливо, що в самій Індії пряжу знову таки продається на кредит.
На цей кредит в Індії купують продукт і замість грошей за пряжу висилають
в Англію або переказують вексель на відповідну суму. Коли такий
стан триватиме довший час, то він справить тиск на індійський грошовий
ринок, а цей тиск відіб’ється в Англії так, що може спричинити
тут кризу. З свого боку криза, навіть коли вона сполучена з вивозом
благородних металів до Індії, спричиняє в цій країні нову кризу в наслідок
банкрутства англійських торгових домів та їхніх індійських філій,
що мали кредит в індійських банках. Так постає одночасна криза і на
тому ринку, що проти нього торговельний баланс, і на тому, що на
користь йому торговельний баланс. Це явище може бути ще складніше.
Напр., Англія надіслала в Індію срібні зливки, але англійські
кредитори Індії ставлять там тепер свої вимоги, і Індія муситиме скоро
по цьому надіслати свої срібні зливки назад в Англію.

Можливо, що вивізна торговля до Індії та довізна торговля з Індії
приблизно урівноважуються, хоч остання (за винятком особливих обставин,
як подорожчення бавовни тощо) в своїх розмірах визначається й
стимулюється першою. Торговельний баланс між Англією та Індією може
здаватись урівноваженим або виявляти лише незначні коливання в той
або інший бік. Але скоро криза вибухає в Англії, то виявляється, ще
на індійських складах лежать непродані бавовняні товари (отже, що вони
не перетворилися з товарового капіталу на грошовий капітал — перепродукція
на цьому боці), і що, з другого боку, в Англії не тільки лежать
непродані запаси англійських продуктів, але що більшу частину проданих
\parbreak{}  %% абзац продовжується на наступній сторінці

\input{_0240.tex}
\parcont{}  %% абзац починається на попередній сторінці
\index{ii}{0241}  %% посилання на сторінку оригінального видання
свого функціонування само підприємство через капіталізацію певної частини
додаткової вартости. Для капіталіста \emph{В} це не можливо. Частина
капіталу, що про неї мовиться, мусить складати в нього частину первісно
авансованого капіталу. В обох випадках ця частина капіталу фігуруватиме
в книгах капіталіста як авансований капітал — і ним вона є в
дійсності — бо, згідно з нашим припущенням, вона становить частину
продуктивного капіталу, доконечного для провадження підприємства
в даному маштабі. Але величезна ріжниця в тому, з якого фонду
її авансується. У \emph{В} вона дійсно є частина первісного авансованого
капіталу або капіталу, що його треба мати в розпорядженні.
Навпаки, в \emph{А} вона є частина додаткової вартости, застосованої як
капітал. Цей останній випадок показує нам як не лише акумульований
капітал, а й частина первісно авансованого капіталу може бути просто
капіталізованою додатковою вартістю.

Скоро сюди долучається розвиток кредиту, відношення первісно авансованого
капіталу й капіталізованої додаткової вартости заплутується
ще більше. Напр., \emph{А} позичає в банкіра \emph{С} частину продуктивного капіталу,
що з ним, він починає або продовжує справу протягом року. З
самого початку він не має власного капіталу, достатнього для провадження
справи. Банкір \emph{С} позичає йому суму, що складається виключно з
додаткової вартости, покладеної до нього підприємцями \emph{D}, \emph{E}, \emph{F} і~\abbr{т. ін.}
З погляду А тут ще не йдеться про акумульований капітал. А в дійсності
для \emph{D}, \emph{E}, \emph{F} і~\abbr{т. ін.} \emph{А} є не що інше, як аґент, що капіталізує привласнену
ними додаткову вартість.

В книзі 1, розділ 22, ми бачили, що акумуляція, перетворення додаткової
вартости на капітал, своїм реальним змістом є процес репродукції
в поширеному маштабі, все одно, чи виявляється таке поширення
екстенсивно у вигляді долучення нових фабрик до старих, чи в інтенсивному
поширенні попереднього маштабу підприємства.

Розмір продукції може поширюватись малими дозами, оскільки
частину додаткової вартости застосовується на такі поліпшення, що або
тільки підвищують продуктивну силу вживаної праці, або разом з тим
дають змогу і визискувати її інтенсивніше. Або ж, коли робочий день не
обмежено законом, досить додаткової витрати обігового капіталу (на матеріяли
продукції та заробітну плату), щоб поширити розміри підприємства,
не збільшуючи основного капіталу, що його денний протяг вживання
таким чином лише подовжується, тимчасом як період обороту його
відповідно скорочується. Або, за сприятливих ринкових коньюнктур, капіталізована
додаткова вартість може дати змогу спекулювати на сировинному
матеріялі, отже, переводити такі операції, що для них не вистачило
б первісно авансованого капіталу, й~\abbr{т. ін.}

А проте, очевидно, що там, де порівняно велике число періодів обороту
зумовлює частішу реалізацію додаткової вартости протягом року,
будуть наставати періоди, коли не можна буде ні подовжувати робочий
день, ні заводити частинні поліпшення; тимчасом як, з другого боку, пропорційне
поширення цілого підприємства, зумовлене почасти загальним
\parbreak{}  %% абзац продовжується на наступній сторінці

\parcont{}  %% абзац починається на попередній сторінці
\index{ii}{0242}  %% посилання на сторінку оригінального видання
характером підприємства, напр., будівель, почасти поширенням фонду
робочої сили, як у сільському господарстві, можливе лише в певних
більш-менш вузьких межах, і для цього треба додаткового капіталу
такого розміру, що його може дати лише багаторічна акумуляція додаткової
вартости.

Отже, поряд справжньої акумуляції або перетворення додаткової
вартости на продуктивний капітал (і відповідної репродукції в поширеному
розмірі) відбувається акумуляція грошей, нагромадження частини додаткової
вартости як лятентного грошового капіталу, який лише пізніше,
досягши певних розмірів, має функціонувати як додатковий активний
капітал.

Так стоїть справа з погляду поодинокого капіталіста. Однак, з розвитком
капіталістичної продукції розвивається одночасно кредитова система.
Грошовий капітал, що його капіталіст ще не може застосувати в своєму
власному підприємстві, застосовує інший і платить за це йому проценти. Він
функціонує для свого власника як грошовий капітал в особливому
значенні, як особливий ґатунок капіталу, відмінний від продуктивного
капіталу. Але він діє як капітал в руках другого. Очевидно, що при
частішій реалізації додаткової вартости і при збільшенні маштабу, що
в ньому її продукується, зростає пропорція, що в ній новий грошовий
капітал, або гроші як капітал, подається на грошовий ринок, а відси
знову вбирається — принаймні більшу частину його — для поширення
продукції.

Найпростіша форма, що в ній може виявлятися цей додатковий лятентний
грошовий капітал, є форма скарбу. Можливо, що цей скарб є
додаткове золото або срібло, одержане безпосередньо або посередньо
в обміні з країнами, що продукують благородні металі. І тільки таким
способом в країні абсолютно зростає грошовий скарб. З другого боку,
можливо — і так здебільша буває, — що цей скарб є не що інше, як
гроші, вилучені з циркуляції всередині країни, що набрали форму скарбу
в руках поодиноких капіталістів. Можливо далі, що цей лятентний грошовий
капітал складається просто з знаків вартости — кредитові гроші
ми тут ще лишаємо осторонь — або з простих, потверджених леґальними
документами вимог (юридичних титулів) капіталістів до третіх осіб. В
усіх цих випадках, хоч яка буде форма буття цього додаткового грошового
капіталу, він, оскільки він є капітал in spe\footnote*{
In spe — досл.; „в надії, в перспективі“, тобто потенціяльно. \emph{Ред.}
}, репрезентує не
що інше, як додаткові та в запасі тримані юридичні титули капіталістів на
майбутню додаткову річну продукцію суспільства.

„Таким чином, маса справді акумульованого багатства, розглядувана
з кількісного боку,\dots{} надзвичайно мала порівняно з продуктивними
силами суспільства, що йому воно належить, хоч на якому щаблі цивілізації
стояло б те суспільство; або навіть порівняно з дійсним споживанням
цього самого суспільства протягом лише небагатьох років; остільки
мала, що головну увагу законодавців та політико-економів треба було б
\parbreak{}  %% абзац продовжується на наступній сторінці

\parcont{}  %% абзац починається на попередній сторінці
\index{ii}{0243}  %% посилання на сторінку оригінального видання
спрямувати на продуктивні сили та на їхній майбутній вільний розвиток,
а не на саме лише акумульоване багатство, що впадає на очі, як це
було до цього часу. Куди більша частина так званого акумульованого
багатства є лише номінальна й складається не з справжніх речей, кораблів,
будинків, бавовняних товарів, меліорацій, а з простих юридичних
титулів, з вимог на майбутні річні продуктивні сили суспільства, з юридичних
титулів, що утворились і увічнились в наслідок засобів або
інституцій незабезпечености\dots{} Вживання таких предметів (нагромаджених
фізичних речей, або справжнього багатства) як простого
засобу, для присвоювання їхніми власниками багатства, яке лише мають
утворити майбутні продуктивні сили суспільства, таке вживання їм поступінно
відібралось би природними законами розподілу, не вживаючи
сили; за допомогою кооперованої праці (Cooperative labour) його
відібралось би їм протягом небагатьох років“. (William Thompson, „An
Inguiry into the principles of the Distribution of Wealth. London 1850, p. 453.
Ця книга вийшла першим виданням 1824 року).

„Мало хто думає, а більшість навіть і гадки не має, яка надто
незначна й масою своєю і силою свого впливу дійсна акумуляція суспільства
порівняно з продуктивними силами людства і навіть порівняно
з звичайним споживанням одного покоління протягом небагатьох лише
років. Причина очевидна, але вплив дуже шкідливий. Багатство, споживане
щороку, зникає разом із споживанням його; воно лише одну мить
навіч перед нами і справляє вражіння лише, поки з нього користуються
або поки його споживають. Але тільки повільно споживана частина
багатства, меблі, машини, будівлі, стоять перед нашими очима з нашого
дитинства й до старости, як довговічні пам’ятники людської праці.
Маючи цю сталу, довговічну, лише повільно споживану частину суспільного
багатства — землю та сировинний матеріял, що до них прикладається
працю, знаряддя, що ними працюють, будівлі, що дають притулок підчас
праці, — маючи все це, власники цих речей в своїх інтересах захоплюють
річні продуктивні сили всіх дійсно продуктивних робітників суспільства,
хоч би які незначні були ці речі порівняно з постійно відновлюваними
продуктами цієї праці. Людність Брітанії та Ірляндії дорівнює 20 мільйонам;
пересічне споживання кожної людини, — чоловіка, жінки, дитини —
становить, мабуть, щось 20\pound{ ф. стерл.}, увесь щорічно споживаний продукт
праці становить багатство приблизно в 400 мільйонів ф. стерл. За оцінкою,
загальна сума акумульованого капіталу в цих країнах не перевищує
1200 мільйонів, або потроєного річного продукту праці; поділивши
нарівно, маємо 60\pound{ ф. стерл.} на душу; тут для нас радше має вагу відношення,
ніж більш або менш точні абсолютні підсумки сум цієї оцінки.
Процентів з цілого цього капіталу було б досить для того, щоб утримувати
всю людність при її теперішньому рівні життя приблизно два
місяці на рік, а всього акумульованого капіталу (коли б знайшлися для
нього покупці) вистачило б на утримання цієї людности протягом цілих
трьох років без якоїбудь роботи! Але потім, опинившись без будівель,
одягу й харчу, люди мусіли б загинути з голоду, або зробитись рабами
\parbreak{}  %% абзац продовжується на наступній сторінці

\parcont{}  %% абзац починається на попередній сторінці
\index{ii}{0244}  %% посилання на сторінку оригінального видання
тих, хто їх утримував протягом трьох років. Як 3 роки стосуються
до протягу життя одного здорового покоління, прим., до 40 років,
так стосується величина й значення дійсного багатства, акумульований
капітал, навіть найбагатшої країни, до її продуктивної сили, до продуктивних
сил одного лише покоління людей; не до того, що вони могли б
випродукувати за розумного ладу однакової для всіх забезпечености, а
особливо при кооперованій праці, а до того, що вони дійсно абсолютно
продукують за недосконального ладу незабезпечености, що приводить до
збентеження\elli{!..} І для того, щоб зберегти й увічнити в сучасному стані вимушеного
розподілу цю на позір величезну масу наявного капіталу, або радше,
щоб зберегти й увічнити здобуту за її допомогою владу й монополію
над продуктом річної праці, мусить увічнитись весь цей страшенний
механізм, порочність, злочинність і злидні незабезпечености. Нічого не
можна акумулювати, поки не задовольниться неодмінні потреби, а великий
потік людських нахилів прямує до задоволення; звідси порівняно незначний
розмір дійсного багатства суспільства в кожний даний момент.
Це — вічний кругобіг продукції та споживання. При такій величезній масі
річної продукції та споживання навряд чи можна було б обійтися без
пригорщі дійсної акумуляції; і все ж головну увагу звернуто не на
масу продуктивних сил, а на цю пригорщ акумуляції. Але небагато
людей захопили цю пригорщ і перетворили її на знаряддя, щоб
привласнювати рік-у-рік відновлювані продукти праці великої маси людей\dots{}
Звідси надзвичайна важливість такого знаряддя для цих небагатьох\dots{}
Близько третини національного річного продукту відбирається тепер від
продуцентів під назвою громадських податків, і споживають її непродуктивно
люди, що не дають за те жодного еквіваленту, тобто такого, що
мав би значення еквіваленту для продуцентів\dots{} Маса людей з подивом
дивиться на акумульовані багатства, особливо коли зосереджені вони в
руках небагатьох осіб. Але щорічно продуковані маси продуктів, як
вічні та незчисленні хвилі могутнього потоку, ринуть далі й зникають у
забутному океані споживанння. І однак це вічне споживання зумовлює
не лише всі втіхи, а й існування цілого людського роду. Кількість та
розподіл цього річного продукту насамперед повинні бути за об’єкт дослідження.
Дійсна акумуляція має цілком другорядне значення та й його
вона набирає майже виключно в наслідок свого впливу на розподіл річного
продукту\dots{} Дійсну акумуляцію та розподіл завжди розглядається тут
(у досліді Томпсона) у зв'язку з продуктивними силами і як їм підпорядковану.
Майже в усіх інших системах продуктивні сили розглядалось
в зв’язку та підпорядковано акумуляції та увічненню наявного
способу розподілу. Порівняно з збереженням цього наявного способу
розподілу вважається за неварті уваги завжди відновлювані злидні або
добробут цілого людського роду. Увічнення здобутків насильства, обману
й випадковости назвали забезпеченістю, і щоб зберегти цю вигадану
забезпеченість, без жалю приносять у жертву всі продуктивні сили
людського роду“. (Там же, стор. 440--443).


\index{ii}{0245}  %% посилання на сторінку оригінального видання
Можливі лише два нормальні випадки репродукції, якщо залишити
осторонь ті порушення, що перешкоджають навіть репродукції в попередньому
маштабі.

Або відбувається репродукція в простому маштабі.

Або відбувається капіталізація додаткової вартости, акумуляція.

\subsection{Проста репродукція}

При простій репродукції додаткова вартість, продукована й реалізовувана
щорічно або — при кількох оборотах — періодично протягом року,
споживається особисто, тобто непродуктивно, її власником, капіталістом.

Та обставина, що вартість продукту складається почасти з додаткової
вартости, почасти з тієї частини вартости, яка складається з репродукованого
в ньому змінного капіталу плюс зужиткований на його продукцію
сталий капітал, — ця обставина абсолютно нічого не змінює ні в кількості,
ні в вартості цілого продукту, що постійно надходить в циркуляцію, як
товаровий капітал, і так само постійно вилучається з неї для продуктивного
або особистого споживання, тобто для того, щоб служити засобом
продукції або засобом споживання. Якщо сталий капітал залишити осторонь,
то ця обставина впливає тільки на розподіл річного продукту між
робітниками й капіталістами.

Тому, навіть коли припустити просту репродукцію, частина додаткової
вартости має постійно перебувати в формі грошей, а не в формі
продукту, бо інакше її не можна перетворити з грошей на продукт для
споживання. Це перетворення додаткової вартости з її первісної товарової
форми на гроші треба тут дослідити далі. Для спрощення справи
візьмімо проблему в її найпростішій формі, а саме припустімо циркуляцію
виключно металевих грошей, грошей, що являють дійсний грошовий еквівалент.

Згідно з законами простої товарової циркуляції (кн. І, розд. III), маси
наявних у країні металевих грошей має вистачити не лише для
циркуляції товарів. Її має вистачити для того, щоб вирівнювати коливання
грошового обігу, що випливають почасти з флюктуацій\footnote*{
Флюктуація — від лат. слова „fluctus“, гра хвиль, хвилювання, почережне
піднесення й спад. \emph{Ред.}
} в швидкості
циркуляції, почасти з змін товарових цін, почасти з різних та
змінних відношень, що в них функціонують гроші як засіб виплати або
як власне засіб циркуляції. Відношення, що в ньому наявна маса грошей
розподіляється на скарб і на гроші в циркуляції, раз-у-раз змінюється, але
маса грошей завжди дорівнює сумі грошей, наявних у формі скарбу та
в формі грошей в циркуляції. Ця маса грошей (маса благородного металю)
є поступінно нагромаджуваний скарб суспільства. Оскільки частина цього
скарбу зужитковується через зношування, її треба щорічно знову заміщувати,
як і всякий інший продукт. Це в дійсності і відбувається через
\parbreak{}  %% абзац продовжується на наступній сторінці

\parcont{}  %% абзац починається на попередній сторінці
\index{ii}{0246}  %% посилання на сторінку оригінального видання
безпосередній або посередній обмін частини річного продукту країни на
продукт країн, що продукують золото й срібло. Однак такий інтернаціональний
характер оборудки замасковує її простоту. Тому, щоб звести
проблему до найпростішого та найвиразнішого виразу, треба припустити,
що продукція золота й срібла відбувається в самій країні, отже, що продукція
золота й срібла становить частину сукупної суспільної продукції
кожної країни.

Лишаючи осторонь золото й срібло, продуковані для речей розкошів,
мінімум щорічної продукції їх мусить дорівнювати зношуванню грошового
металю, зумовленому річною грошовою циркуляцією. Далі: коли
зростає сума вартости маси товарів, яка щорічно продукується й циркулює,
то мусить зростати й річна продукція золота й срібла, оскільки
виросла сума вартости товарів, що циркулюють, і маса грошей, потрібних
для їхньої циркуляції (та для утворення відповідного скарбу), не компенсується
більшою швидкістю грошового обігу та поширенішою функцією
грошей як засобу виплати, тобто частішими взаємними вирівнюваннями
купівель і продажів без посередництва дійсних грошей.

Отже, частину суспільної робочої сили та частину суспільних засобів
продукції треба щороку витрачати на продукцію золота й срібла.

Капіталісти, які провадять продукцію золота й срібла, провадять її, —
як ми тут, за умов простої репродукції, припускаємо — лише в межах
пересічного річного зношування та зумовленого ким пересічного річного
споживання золота й срібла; свою додаткову вартість, що її вони, згідно
з нашим припущенням, споживають щорічно, нічого не капіталізуючи з
неї, вони пускають у циркуляцію безпосередньо в грошовій формі, яка
для них є натуральна форма, а не перетворена форма продукту, як в інших
галузях продукції.

Далі: щодо заробітної плати — грошової форми, що в ній авансується
змінний капітал — то тут її так само заміщується не через продаж
продукту, не через перетворення його на гроші, а самим продуктом, що
натуральна форма його з самого початку є грошова форма.

Нарешті, так само стоїть справа і з тією частиною продукту благородного
металю, яка дорівнює вартості періодично споживаного сталого
капіталу, так сталого обігового, як і сталого основного, споживаного протягом
року.

Розгляньмо кругобіг, зглядно оборот, капіталу, вкладеного в продукцію
благородних металів, насамперед у формі $Г — Т\dots{} П\dots{} Г'$. Оскільки
$Т$ в $Г — Т$ складається не лише з робочої сили та засобів продуції, а також
із основного капіталу, що з нього в $П$ споживається тільки частину
його вартости, то очевидно, що $Г'$ — продукт — є грошова сума, яка
дорівнює змінному капіталові, витраченому на заробітну плату, плюс обіговий
сталий капітал, витрачений на засоби продукції, плюс частина
вартости, яка відповідає зношуванню основного капіталу, плюс додаткова
вартість. Коли б, при незмінній загальній вартості золота, ця сума була
менша, то вкладення капіталу в золоті копальні було б непродуктивне
або, — коли б це явище набрало загального характеру, — то вартість золота
\parbreak{}  %% абзац продовжується на наступній сторінці

\input{_0247c.tex}
\parcont{}  %% абзац починається на попередній сторінці
\index{ii}{0248}  %% посилання на сторінку оригінального видання
досить, щоб постійно оплачувати робочу силу. Тут, у продукції грошей
вистачить такої самої суми; але зворотно приплилі 100\pound{ ф. стерл.}, що ними
кожні 5 тижнів оплачується робочу силу, є не перетворена форма продукту
цієї робочої сили, а частина цього самого постійно відновлюваного
продукту. Золотопромисловець платить своїм робітникам безпосередньо
частиною золота, що вони сами його випродукували. Тому 1000\pound{ ф. стерл.},
щорічно витрачувані таким чином на робочу силу й подавані робітниками
в циркуляцію, не повертаються через циркуляцію до свого вихідного
пункту.

Далі, щодо основного капіталу, то при першому заснуванні підприємства
треба витратити порівняно великий грошовий капітал, що його,
отже, пускається в циркуляцію. Як кожний основний капітал, він повертається
назад лише частинами протягом кількох років. Але він повертається
назад як безпосередня частина продукту, золота, не через продаж
продукту, не через перетворення його таким способом на золото. Отже,
він поступінно набирає своєї грошової форми не через вилучення грошей
з циркуляції, а через нагромадження відповідної частини продукту.
Відновлений таким чином грошовий капітал не є грошова сума, поступінно
вилучувана з циркуляції на покриття грошової суми, первісно кинутої
в циркуляцію на придбання основного капіталу. Це — додаткова
маса грошей.

Нарешті, щодо додаткової вартости, то вона так само дорівнює тій
частині нового продукту — золота, яку в кожний новий період обороту
пускається в циркуляцію, щоб, згідно з нашим припущенням, витратити
Її непродуктивно, на оплату засобів існування та речей розкошів.

Але згідно з нашим припущенням, вся ця річна продукція золота —
що нею постійно вилучається з ринку робочу силу й матеріяли продукції,
але не вилучається з нього грошей, а постійно подається додаткові гроші
— вся ця річна продукція золота тільки заміщує гроші, зношувані протягом
року, отже, лише підтримує в суспільстві сповна ту кількість грошей,
яка постійно, хоч і в змінних долях, існує в двох формах — в формі
скарбу і в формі грошей, що перебувають у циркуляції.

Згідно з законом товарової циркуляції загальна маса грошей мусить
дорівнювати масі грошей, потрібних для циркуляції, плюс кількість грошей,
що перебуває в формі скарбу, а ця остання кількість більшає або
меншає залежно від скорочення або поширення циркуляції; вона ж служить
також і для утворення потрібного резервного фонду засобів виплати.
Вартість товарів мусить сплачуватись грішми, оскільки виплати
взаємно не урівноважуються. Та обставина, що частина цієї вартости
складається з додаткової вартости, тобто нічого не коштувала продавцеві
товарів, абсолютно нічого не змінює в справі. Припустімо, що всі
продуценти є самостійні власники їхніх засобів продукції, отже, що циркуляція
відбувається між самими безпосередніми продуцентами. Коли залишити
осторонь сталу частину їхнього капіталу, то їхній річний додатковий
продукт за аналогією з капіталістичним станом, можна було б поділити
на дві частини: одну — \emph{а}, що тільки заміщує потрібні засоби
\parbreak{}  %% абзац продовжується на наступній сторінці

\parcont{}  %% абзац починається на попередній сторінці
\index{ii}{0249}  %% посилання на сторінку оригінального видання
їхнього існування, другу — b, що її вони почасти витрачають на речі
розкошів, а почасти застосовують на поширення продукції; а — в такому
разі репрезентує змінний капітал, b — додаткову вартість. Але такий поділ
не мав би жодного впливу на величину тієї маси грошей, яка потрібна
для циркуляції цілого їхнього продукту. За інших незмінних умов, вартість
товарової маси, що циркулює, була б та сама, а значить, і маса
потрібних для цього грошей була б та сама. Крім того, при однаковому
поділі періодів обороту продуценти мусили б мати такі самі грошові запаси,
тобто постійно мати в грошовій формі таку саму частину свого
капіталу, бо, згідно з нашим припущенням, їхня продукція, як і раніш,
була б товаровою продукцією. Отже, та обставина, що частина товарової
вартости складається з додаткової вартости, абсолютно не змінює маси
грошей доконечних для провадження підприємства.

Один з супротивників Тука, що тримається формули $Г — Т — Г$, запитує
його, як капіталістові вдається постійно вилучати з циркуляції більше
грошей, ніж він подає туди. Це цілком зрозуміло. Тут ідеться не про
утворення додаткової вартости. Останнє, являючи єдину таємницю, з
капіталістичного погляду само собою зрозуміле. Застосована бо сума вартости
не була б капіталом, коли б вона не збагачувалась додатковою
вартістю. А що згідно з припущенням вона є капітал, то додаткова вартість
сама собою зрозуміла.

Отже, питання не в тім, відки береться додаткова вартість, а в тім,
відки беруться гроші, що на них вона перетворюється.

Але для буржуазної економії існування додаткової вартости зрозуміле
само собою. Отже, її не лише припускається, але разом з нею припускається
й те, що частина товарової маси, пущеної в циркуляцію, складається
з додаткового продукту, отже, репрезентує таку вартість, що її капіталіст
не кинув у циркуляцію, кидаючи туди свій капітал; що, отже, капіталіст,
разом з своїм продуктом кидає в циркуляцію певний надлишок
порівняно з своїм капіталом, а потім знову вилучає з неї цей надлишок.

Товаровий капітал, що його капіталіст подає в циркуляцію, має більшу
вартість (звідки це постає, не пояснюється або не розуміється, але з
погляду буржуазної економії c’est un fait\footnote*{
Це — факт. \emph{Ред.}
}, ніж продуктивний капітал,
що його він вилучив з циркуляції в формі робочої сили плюс засоби
продукції. Тому при цьому припущенні ясно, чому не лише капіталіст
$А$, але й $В$, $С$, $D$ і~\abbr{т. ін.} можуть постійно вилучати з циркуляції через
обмін своїх товарів більшу вартість, ніж вартість їхнього первісно авансованого
капіталу, що його потім знову й знову авансується. $А$, $В$, $С$,
$D$ і~\abbr{т. ін.} завжди подають в циркуляцію в формі товарового капіталу, —
а ця операція так само багатобічна, як і самостійно діющі капітали, —
більшу товарову вартість, ніж та, що її вони вилучають з циркуляції в
формі продуктивного капіталу. Отже, їм постійно доводиться розподіляти
між собою суму вартости (тобто кожному доводиться вилучати для себе
з циркуляції продуктивний капітал), що дорівнює сумі вартости їхніх
\parbreak{}  %% абзац продовжується на наступній сторінці


\index{iii1}{0250}  %% посилання на сторінку оригінального видання
Частина старого капіталу мусила б лишитись без діла при
всяких обставинах, лишитись без діла щодо своєї властивості як
капіталу — функціонувати і зростати в своїй вартості. Яка саме
частина лишилася б без діла, це вирішила б конкурентна боротьба.
Поки все йде добре, конкуренція, як це виявилось при
вирівненні загальної норми зиску, діє, як практичний братерський
союз класу капіталістів, так що вони спільно ділять між собою загальну
здобич пропорціонально до величини частки, вкладеної
кожним з них. Але як тільки справа йде вже не про розподіл
зиску, а про розподіл збитку, то кожний з них намагається
якомога зменшити свою участь в ньому і перекласти його на
шию іншому. Для всього класу збиток є неминучий. Але скільки
з нього припаде на кожного окремого капіталіста, наскільки
взагалі кожний з них повинен взяти участь в ньому, це стає
тоді питанням сили й хитрості, і конкуренція перетворюється
тоді в боротьбу ворогуючих братів. Протилежність між інтересами
кожного окремого капіталіста і інтересами класу капіталістів виявляється
при цьому цілком так само, як перед тим за допомогою
конкуренції проявлялась на практиці тотожність цих інтересів.

Яким же чином міг би бути знов усунений цей конфлікт
і як могли б відновитись відносини, відповідні „здоровому“
рухові капіталістичного виробництва? Спосіб усунення міститься
вже в простому констатуванні конфлікту, про усунення якого
йде мова. Він полягає в залишенні без діла і навіть частковому
знищенні капіталу, рівного своєю вартістю всьому додатковому
капіталові $ΔК$ або принаймні частині його. Хоча — як
це вже випливає з викладу конфлікту — розподіл цього збитку
ні в якому разі не поширюється рівномірно на поодинокі окремі
капітали, а вирішується в конкурентній боротьбі, в якій збиток
розподіляється дуже нерівно і в дуже різних формах, залежно
від особливих переваг або особливих завойованих уже позицій,
так що один капітал лишається лежати без діла, другий знищується,
третій має тільки відносний збиток або зазнає тільки
тимчасового знецінення і т. д.

Але при всяких обставинах рівновага відновилась би в наслідок
бездіяльності і навіть знищення капіталу в більшому чи
меншому розмірі. Це почасти поширилося б на матеріальну
субстанцію капіталу; тобто частина засобів виробництва, основний
і обіговий капітал, не функціонувала б, не діяла б як капітал;
частина підприємств, що вже почали функціонувати, припинила
б роботу. Хоча в цьому відношенні час робить своє
і погіршує всі засоби виробництва (за винятком землі), але тут
в наслідок припинення функціонування мало б місце значно
сильніше справжнє руйнування засобів виробництва. Головний
результат у цьому відношенні був би, однак, у тому, що ці засоби
виробництва перестали б діяти як засоби виробництва, — в зруйнуванні
їх функції як засобів виробництва на коротший чи довший
час.


\index{iii1}{0251}  %% посилання на сторінку оригінального видання
Найбільш руйнівного впливу, і при тому найгострішого
характеру, зазнав би капітал, оскільки він має властивість
вартості, зазнали б капітальні \emph{вартості}. Частина капітальної
вартості, яка перебуває просто у формі посвідок на майбутню
участь в додатковій вартості, в зиску, і яка в дійсності становить
тільки боргові зобов’язання в різних формах на виробництво,
відразу знецінюється з падінням доходів, на які вона розрахована.
Частина готівки золота й срібла лежить без діла, не функціонує
як капітал. Частина товарів, що перебувають на ринку, може
здійснити свій процес циркуляції і репродукції тільки шляхом
надзвичайного зниження своїх цін, отже, шляхом знецінення
того капіталу, який вона представляє. Цілком так само більше
чи менше знецінюються елементи основного капіталу. До цього
долучається ще й те, що певні припущені відношення цін
обумовлюють процес репродукції, і тому цей останній в наслідок
загального падіння цін приходить до застою і розладу. Цей
розлад і застій паралізує функцію грошей як платіжного засобу,
яка розвивається одночасно з розвитком капіталу і грунтується
на згаданих припущених відношеннях цін; він розриває у сотнях
місць ланцюг платіжних зобов’язань на певні строки і ще більше
загострюється в наслідок зумовленого цим краху (Zusammenbrechen)
кредитної системи, що розвинулась одночасно з капіталом,
і таким чином веде до сильних і гострих криз, до раптових
насильних знецінень і дійсного застою й розладу\footnote*{
В першому німецькому виданні тут стоїть: „застою й занепаду (Sturz)“;
виправлено на підставі рукопису Маркса. \emph{Примітка ред. нім. вид, ІМЕЛ.}
} процесу
репродукції, і тим самим до дійсного зменшення репродукції.

Але одночасно діяли б і інші фактори. Застій виробництва
позбавив би роботи частину робітничого класу і цим поставив би
заняту частину його в такі відносини, при яких вона мусила б
згоджуватись на зниження заробітної плати навіть нижче пересічного
рівня; обставина, яка дає для капіталу такий самий результат,
як коли б при пересічній заробітній платі була підвищена
відносна чи абсолютна додаткова вартість. Період процвітання
сприяв би шлюбам серед робітників і зменшив би смертність їх
дітей, обставини, які — хоч би й яке вони означали дійсне збільшення
населення — не означають збільшення дійсно працюючого
населення, але на відношення робітників до капіталу впливають
цілком так само, як коли б збільшилося число дійсно функціонуючих
робітників. З другого боку, падіння цін і конкурентна
боротьба спонукали б кожного капіталіста підвищувати індивідуальну
вартість свого сукупного продукту понад його загальну
вартість за допомогою застосування нових машин, нових поліпшених
методів праці, нових комбінацій, тобто підвищувати продуктивну
силу даної кількості праці, знижувати відношення
змінного капіталу до сталого, і тим самим звільняти робітників,
\parbreak{}  %% абзац продовжується на наступній сторінці


\index{ii}{0252}  %% посилання на сторінку оригінального видання
Загальну відповідь уже дано: коли має циркулювати маса товарів в
1000\pound{ ф. стерл.}, × X, то величина грошової суми, потрібної для цієї циркуляції,
абсолютно не змінюється від того, чи є в вартості цієї маси товарів
додаткова вартість, чи немає, чи випродукувано цю товарову масу
капіталістично, чи ні. Отже, самої проблеми не існує. За інших
даних умов, швидкости грошової циркуляції та ін., для циркуляції товарової
вартости в 1000\pound{ ф. стерл.} × Х, потрібна певна сума грошей, яка
зовсім не залежить від тієї обставини, чи багато, чи мало з цієї вартосги
припадає безпосереднім продуцентам цих товарів. Оскільки тут і існує
проблема, вона збігається з загальною проблемою: відки береться сума
грошей, потрібна для циркуляції товарів у даній країні.

А проте, з погляду капіталістичної продукції, існує, звичайно, подоба
якоїсь особливої проблеми. А саме за вихідний пункт, відки гроші пускається
в циркуляцію, тут виступає капіталіст. Гроші, що їх витрачає
робітник на оплату засобів свого існування, існують спочатку як грошова
форма змінного капіталу, і тому капіталіст їх спочатку пускає в
циркуляцію як купівельний або виплатний засіб за робочу силу. Крім
того, капіталіст пускає в циркуляцію гроші, що спочатку становили для
нього грошову форму його сталого — основного й поточного — капіталу;
він витрачає їх як купівельний або виплатний засіб на засоби праці та
матеріяли продукції. Але поза цим капіталіст уже не виступає як вихідний
пункт грошової маси, що перебуває в циркуляції.

Але взагалі існують тільки два вихідні пункти: капіталіст і робітник.
Треті особи всіх категорій або мусять одержувати гроші від цих двох
кляс за якібудь послуги, або оскільки вони одержують гроші без якихбудь
послуг з їхнього боку, вони є співвласники додаткової вартости в
формі ренти, проценту й т. ін. Те, що додаткова вартість не лишається
цілком в кишені промислового капіталіста, й що він мусить поділитися
нею з іншими особами, не має жодного чинення до нашого питання.
Питання в тому, яким чином він перетворює на гроші свою додаткову
вартість, а не в тому, як розподіляються потім здобуті за неї гроші.
Отже, в даному разі ми все ще повинні розглядати капіталіста як єдиного
власника додаткової вартости. Щождо робітника, то вже сказано, що
він є тільки вторинний вихідний пункт, але капіталіст є первинний
вихідний пункт тих грошей, що їх пускає в циркуляцію робітник. Гроші,
спочатку авансовані як змінний капітал, пророблюють уже свій другий
обіг, коли робітник витрачає їх на оплату засобів існування.

Отже, кляса капіталістів лишається єдиним вихідним пунктом грошової
циркуляції. Коли їй треба на оплату засобів продукції 400\pound{ ф. стерл.}
і на оплату робочої сили 100\pound{ ф. стерл.}, то вона пускає в циркуляцію
500\pound{ ф. стерл}. Але додаткова вартість, що міститься в продукті, при нормі
додаткової вартости в 100\%, дорівнює вартості в 100\pound{ ф. стерл}. Як
же вона може постійно вилучати з циркуляції 600\pound{ ф. стерл.}, коли
вона постійно пускає в неї лише 500\pound{ ф. стерл.}? З нічого нічого й не
буде. Ціла кляса капіталістів не може вилучати з циркуляції нічого такого,
чого раніш не було пущено в неї.

\index{ii}{0253}  %% посилання на сторінку оригінального видання
Тут ми лишаємо осторонь ту обставину, що грошової суми в 400\pound{ ф. стерл.} при десятиразовому обороті, може, буде досить для циркуляції
засобів продукції вартістю в 4000\pound{ ф. стерл.} і праці вартістю в 1000\pound{ ф.
стерл.}, а решти 100\pound{ ф. стерл.} так само буде досить для циркуляції додаткової
вартости в 1000\pound{ ф. стерл}. Це відношення грошової суми до товарової
вартості, що циркулює за її допомогою, не має ніякого чинення до
справи. Проблема лишається та сама. Коли б та сама монета не циркулювала
декілька разів, то довелось би пустити в циркуляцію 5000\pound{ ф. стерл.}
як капітал і 1000\pound{ ф. стерл.} були б потрібні для перетворення додаткової
вартости на гроші. Постає питання, відки беруться ці гроші, хоч то
1000\pound{ ф. стерл.}, хоч 100\pound{ ф. стерл}. В усякому разі вони є надлишок понад
грошовий капітал, пущений у циркуляцію.

Справді, хоч як це здається парадоксальним на перший погляд, кляса
капіталістів сама пускає в циркуляцію ті гроші, які служать для реалізації
додаткової вартости, що міститься в товарах. Але nota bene\footnote*{
Добре зауважте. \emph{Ред.}
} — кляса
капіталістів пускає їх в циркуляцію не як авансовані гроші, отже, не як
капітал. Вона витрачає їх як купівельний засіб для свого особистого
споживання. Отже, кляса капіталістів не авансує цих грошей, хоч вона
є вихідний пункт їхньої циркуляції.

Візьмімо поодинокого капіталіста, що починає справу, приміром,
фармера. Протягом першого року він авансує грошовий капітал, скажімо,
в 5000\pound{ ф. стерл.}, щоб оплатити засоби продукції (4000\pound{ ф. стерл.}) і робочу
силу (1000\pound{ ф. стерл.}). Норма додаткової вартости хай буде 100\%, привлащувана
ним додаткова вартість \deq{} 1000\pound{ ф. стерл}. Вищезазначені 5000\pound{ ф.
стерл.} являють собою всі гроші, що їх він авансує як грошовий капітал.
Однак ця людина мусить також жити, але до кінця року не одержить
вона жодних грошей. Її споживання становить 1000\pound{ ф. стерл}. Вона мусить
мати ці гроші. Правда, вона каже, що мусить авансувати собі ці 1000\pound{ ф. стерл.}
протягом першого року. Однак це авансування — воно має тут лише
суб’єктивне значення — сходить лише на те, що протягом першого року
вона мусить покривати своє особисте споживання з власної кишені, а не
з дармової продукції своїх робітників. Вона не авансує цих грошей як
капітал. Вона витрачає їх, платить їх як еквівалент за ті засоби існування,
що вона споживає. Цю вартість вона витрачає як гроші, подає в
циркуляцію та вилучає з неї як товарові вартості. Ці товарові вартості
вона спожила. Отже, немає тепер будь-якого відношення її до їхньої
вартости. Гроші, що ними вона заплатила за неї, існують тепер як елемент
грошей, що циркулюють. Але вартість цих грошей вона вилучила
в продуктах із циркуляції, а разом з продуктами, що ними вона жила,
знищено й їхню вартість. Вартість ця зникла. Але ось наприкінці року
ця людина пускає в циркуляцію товарову вартість в 6000\pound{ ф. стерл.} і продає її.
В наслідок цього до неї повертається: 1) авансований нею грошовий
капітал в 5000\pound{ ф. стерл.}, 2) перетворена на гроші додаткова вартість в
1000\pound{ ф. стерл}. Вона авансувала 5000\pound{ ф. стерл.} як капітал, пустила їх в
\parbreak{}  %% абзац продовжується на наступній сторінці

\input{_0254.tex}
\input{_0255.tex}
\input{_0256.tex}
\input{_0257.tex}
\input{_0258c.tex}
\input{_0259.tex}
\parcont{}  %% абзац починається на попередній сторінці
\index{iii1}{0260}  %% посилання на сторінку оригінального видання
час, який виграє суспільство, не цікавить капіталістичне виробництво.
Розвиток продуктивної сили для нього важливий лиш
остільки, оскільки він збільшує додатковий робочий час робітничого
класу, а не оскільки він взагалі зменшує робочий час
матеріального виробництва; таким чином капіталістичне виробництво
рухається в суперечностях.

Ми бачили, що зростаюче нагромадження капіталу включає
зростаючу концентрацію його. Таким чином зростає влада капіталу,
персоніфіковане в капіталісті усамостійнення суспільних
умов виробництва проти дійсних виробників. Капітал дедалі більше
виявляє себе як суспільна сила, яка функціонує через капіталіста
і яка не стоїть уже в ніякому відношенні до того, що
може створити праця окремого індивіда, — але як відчужена,
усамостійнена суспільна сила, що як річ, і за допомогою цієї
речі як влада капіталіста, протистоїть суспільству. Суперечність
між загальною суспільною силою, в яку перетворюється капітал,
і приватною владою окремих капіталістів над цими суспільними
умовами виробництва розвивається в дедалі більш кричущу суперечність
і включає в собі розв’язання цього відношення,
оскільки воно разом з тим передбачає вироблення умов виробництва
у загальні, колективні, суспільні умови виробництва.
Це вироблення визначається розвитком продуктивних сил при
капіталістичному виробництві і тим способом, яким відбувається
цей розвиток.

\pfbreak{}

Жоден капіталіст не застосовує добровільно нового способу
виробництва, хоч би наскільки він був продуктивніший і хоч би
наскільки він збільшував норму додаткової вартості, якщо він
зменшує норму зиску. Але кожен такий новий спосіб виробництва
здешевлює товари. Тому капіталіст спочатку продає
їх вище їх ціни виробництва, може, вище їх вартості. Він кладе
собі в кишеню ріжницю між їх витратами виробництва і ринковою
ціною всіх інших товарів, вироблених при вищих витратах
виробництва. Він може це робити тому, що пересічний робочий
час, суспільно потрібний для виробництва цих товарів, є більший,
ніж робочий час, потрібний при новому способі виробництва.
Його методи виробництва стоять вище пересічних суспільних.
Але конкуренція робить їх загальними і підпорядковує їх загальному
законові. Тоді починається зниження норми зиску, — спочатку,
може, в цій сфері виробництва, а потім вона вирівнюється
з іншими, — яке, отже, цілком незалежне від волі капіталістів.

З приводу цього треба ще зауважити, що цей самий закон
панує і в тих сферах виробництва, продукт яких ні безпосередньо,
ні посередньо не входить у споживання робітника або
в умови виробництва його засобів існування; отже, і в тих сферах
виробництва, в яких ніяке здешевлення товарів не може збільшити
відносну додаткову вартість, здешевити робочу силу.
\parbreak{}  %% абзац продовжується на наступній сторінці

\input{_0261.tex}

\index{ii}{0262}  %% посилання на сторінку оригінального видання
Але в наслідок додаткового продуктивного капіталу в циркуляцію
подається, як продукт його, додаткову товарову масу. Разом з цією
додатковою товаровою масою подається в циркуляцію частину додаткових
грошей, потрібних для реалізації її — а саме подається остільки, оскільки
вартість цієї товарової маси дорівнює вартості продуктивного капіталу,
зужиткованого на її продукцію. Цю додаткову масу грошей авансується
прямо як додатковий грошовий капітал, і тому він зворотно припливає
до капіталіста в наслідок обороту його капіталу. Тут перед нами знову
постає те саме питання, що й раніш. Звідки беруться додаткові гроші на
реалізацію додаткової вартости, що є тепер у товаровій формі в цій
додатковій масі товарів?

Загальна відповідь знову та сама. Сума цін товарової маси, яка циркулює,
збільшилась не тому, що ціна даної товарової маси підвищилась,
а тому, що маса товарів, які тепер циркулюють, більша за масу товарів,
що циркулювали раніше, і при цьому ця ріжниця не вирівнюється зниженням
цін. Додаткові гроші, потрібні для циркуляції цієї більшої товарової
маси більшої вартости, треба здобути або посиленою економією на
масі грошей, що циркулюють, — чи то через взаємне вирівнювання платежів
тощо, чи то засобами, які прискорюють обіг тієї самої монети, —
або їх треба здобути через перетворення грошей з форми скарбу на
обігову форму грошей. Останнє включає не лише те, що бездіяльний
грошовий капітал починає функціонувати як купівельний засіб
або як засіб виплати; або і не лише те, що грошовий капітал, який уже
функціонує як резервний фонд, виконуючи для свого власника функцію
резервного фонду, активно циркулює для суспільства (як от банкові
вклади, що їх завжди дається в позику), отже, виконує двоїсту функцію;
це перетворення включає й те, що заощаджується стагнаційні монетні
резервні фонди.

„Щоб гроші постійно обігали як монети, монети мусять постійно
осідати як гроші. Постійний обіг монет зумовлено тим, що їх постійно
затримується більшими або меншими кількостями як монетні резервні
фонди, що всюди утворюються в межах циркуляції й зумовлюють її, —
монетні резервні фонди, що їх утворення, розподіл, розпад і нове утворення
завжди чергуються, резервні фонди; що буття їх постійно зникає,
що процес їх зникання ніколи не припиняється. Це безперестанне перетворення
монет на гроші й грошей на монети А.~Сміс висловив таким
чином, що кожен товаровласник поряд з тим особливим товаром, що
його він продає, завжди мусить мати в запасі певну суму загального
товару, що на нього він купує. Ми бачили, що в циркуляції $Т — Г — Т$
другий член $Г — Т п$остійно розпадається на ряд актів купівлі, які відбуваються
не одноразово, а послідовно в часі, так що одна частина Г
обігає як монета, тимчасом як друга перебуває в стані спокою як гроші.
Тут гроші справді є лише монети, що їхнє функціонування відкладено,
і окремі складові частини монетної маси, що обігає, завжди з’являються,
чергуючися, то в одній, то в другій формі. Тому, це перше перетворення
засобу циркуляції на гроші являє собою лише технічний момент самого
\parbreak{}  %% абзац продовжується на наступній сторінці

\input{_0263.tex}
\input{_0264.tex}
\parcont{}  %% абзац починається на попередній сторінці
\index{ii}{0265}  %% посилання на сторінку оригінального видання
І це повторюється постійно. Отже, сума в 100\pound{ ф. стерл.} × х ніколи не
може дати робітничій клясі змоги купити частину продукту, яка репрезентує
сталий капітал, не кажучи вже про ту частину, яка репрезентує додаткову
вартість кляси капіталістів. Робітники на 100\pound{ ф. стерл.} × х завжди можуть
купити тільки ту частину вартости суспільного продукту, яка дорівнює
тій частині вартості, що репрезентує вартість авансованого змінного
капіталу.

Лишаючи осторонь той випадок, коли ця всебічна грошова акумуляція
не виражає нічого іншого, крім розподілу в тому або іншому відношенні
додатково довезеного благородного металю між різними поодинокими
капіталістами, — отже, лишаючи осторонь цей випадок, яким чином може
акумулювати гроші ціла кляса капіталістів?

Всі вони мусили б продавати частину свого продукту, нічого не купуючи
натомість. Немає нічого таємничого в тому, що всі вони мають
певний грошовий фонд, який вони пускають в циркуляцію як засіб циркуляції
для свого споживання, при цьому до кожного з них зворотно припливає
з циркуляції певна частина цього фонду. Але в такому разі такий фонд
існує саме як фонд циркуляції, що утворився в наслідок перетворення
на гроші додаткової вартости, але зовсім не як лятентний грошовий
капітал.

Коли розглядати справу так, як вона відбувається в дійсності, то лятентний
грошовий капітал, що його нагромаджується для пізнішого вжитку,
складається з:

1) Депозитів у банках; при цьому сума грошей, що нею в дійсності
порядкують банки, є порівняно незначна. Грошовий капітал тут нагромаджується
лише номінально, але що в дійстності нагромаджується тут,
так це грошові вимоги, які лише тому можна перетворити на гроші
(якщо тільки можна перетворити), що установлюється рівновага між
зворотними грошовими вимогами і вкладами в банк. А те, що є в банку
як гроші, являє лише порівняно невелику суму.

2) З державних паперів. Вони взагалі не є капітал, а лише боргові
вимоги на частину річного продукту нації.

3) З акцій. Оскільки це не шахрайство, вони є титули власности на
дійсний капітал, належний товариству, і посвідки на одержання додаткової
вартости, яку щорічно дає цей капітал.

В усіх цих випадках немає жодного нагромадження грошей: те, що
на одному боці виступає як нагромадження грошового капіталу, виступає
на другому боці як постійне справжнє витрачання грошей. Хоч витрачає
гроші та особа, якій вони належать, хоч інша, її винуватець, це зовсім
не змінює справи.

На основі капіталістичного способу продукції утворення скарбу, як
таке, ніколи не є мета, а результат або застою в циркуляції — і тоді,
більші, ніж звичайно, маси грошей набирають форми скарбу, — або нагромаджень,
зумовлених оборотом, або нарешті: утворення скарбу є лише
утворення грошового капіталу — покищо в лятентній формі, — призначеного
функціонувати як продуктивний капітал.


\index{ii}{0266}  %% посилання на сторінку оригінального видання
Тому, коли з одного боку, частину реалізованої в грошах додаткової
вартости вилучається з циркуляції і нагромаджується як скарб, то одночасно
другу частину додаткової вартости постійно перетворюється на
продуктивний капітал. За тим винятком, коли між клясою капіталістів
розподіляється додатковий благородний металь, нагромадження в грошовій
формі ніколи не відбувається одночасно в усіх пунктах.

Щодо тієї частини річного продукту, яка репрезентує додаткову
вартість у товаровій формі, то для неї має силу цілком те саме, що й
для другої частини річного продукту. Для її циркуляції потрібна певна
сума грошей. Ця сума грошей так само належить клясі капіталістів, як
і щороку продукована маса товарів, яка репрезентує додаткову вартість.
Її спочатку подає в циркуляцію сама кляса капіталістів. Вона завжди
знову розподіляється між ними через самий процес циркуляції. Як і взагалі
при циркуляції монет, одна частина цієї суми завжди затримується
раз в тому, раз в іншому пункті, тимчасом як друга частина безупинно
циркулює. Справа зовсім не змінюється від того, що частину цього нагромадження
робиться навмисно, щоб утворити грошовий капітал.

Ми тут залишили осторонь ті пригоди в циркуляції, що в наслідок
їх один капіталіст захоплює частину додаткової вартости, ба навіть
капіталу другого капіталіста, і коли, отже, постає однобічна акумуляція
й централізація так грошового, як і продуктивного капіталу. Так, напр.,
частина здобутої додаткової вартости, нагромаджувана як грошовий
капітал капіталістом $А$, може бути частиною додаткової вартости капіталіста
$В$, яка не повертається до нього.
\label{original-266}


\index{ii}{0267}  %% посилання на сторінку оригінального видання

\chapteronly{Репродукція й циркуляція сукупного суспільного капіталу}

\addcontentsline{toc}{part}%
  {\protect\partnumberline{\thepart}Репродукція й циркуляція}%
\addtocontents{toc}{\protect\vspace{-0.7em}}
\addcontentsline{toc}{part}%
  {\protect\partnumberlineadd{сукупного суспільного капіталу}}%

\markboth{Відділ \thepart. Репродукція й циркуляція сукупного \dots{} капіталу}{\rightmark}

\section[Вступ]{Вступ\footnotemark{}}
\footnotetext{З рукопису II} % текст примітки прямо під заголовком

\subsection{Предмет досліду}

\label{original-267}
Безпосередній процес продукції капіталу є процес праці й процес
зростання його вартости, процес, що наслідок його є товаровий продукт,
а визначальний мотив — продукція додаткової вартости.

Процес репродукції капіталу охоплює так цей безпосередній процес
продукції, як і обидві фази власне процесу циркуляції, тобто він охоплює
ввесь кругобіг, що як періодичний процес, — процес знову та знов
повторюваний через певні періоди — становить оборот капіталу.

Хоч розглядатимемо ми кругобіг у формі $Г\dots{} Г'$, хоч у формі $П\dots{} П$,
безпосередній процес продукції \emph{П} завжди становить лише одну ланку
цього кругобігу. В одній формі він виступає як посередня ланка процесу
циркуляції, в другій формі процес циркуляції виступає як посередня
ланка для нього. Постійне відновлення цього процесу, постійну повторювану
появу капіталу в формі продуктивного капіталу, в обох випадках
зумовлено його перетвореннями в процесі циркуляції. З другого боку,
постійно поновлюваний процес продукції є умова перетворень, що їх
знову та знов пророблює капітал у сфері циркуляції — є умова його
поперемінної появи то як грошового капіталу, то як товарового
капіталу.

Однак кожний поодинокий капітал становить лише усамостійнену, так
би мовити, обдаровану індивідуальним життям, частину сукупного суспільного
капіталу, так само, як кожен поодинокий капіталіст є лише індивідуальний
елемент кляси капіталістів. Рух суспільного капіталу складається
з сукупности рухів його усамостійнених уламків, з сукупности оборотів
індивідуальних капіталів. Як метаморфоза поодинокого товару є ланка
в ряді метаморфоз товарового світу — товарової циркуляції, — так
\parbreak{}  %% абзац продовжується на наступній сторінці

\parcont{}  %% абзац починається на попередній сторінці
\index{i}{0268}  %% посилання на сторінку оригінального видання
треба керувати, процесу, який, з одного боку, є суспільний процес
праці для виготовлення продукту, з другого — процес зростання
капіталу, то своєю формою це керування є деспотичне. З розвитком
кооперації у великому маштабі деспотизм цей розвиває властиві
йому своєрідні форми. Подібно до того, як капіталіст спочатку
звільняється від ручної праці, скоро тільки його капітал
досягає тієї мінімальної величини, за якої тільки й починається
капіталістична продукція у власному значенні, так само й тепер
він знову таки віддає саму цю функцію безпосереднього й невпинного
нагляду над поодинокими робітниками та групами робітників
осібному ґатункові найманих робітників. Як армія потребує
військових обер-офіцерів та унтер-офіцерів, так і маса робітників,
що працюють разом під командою того самого капіталу,
потребує промислових обер-офіцерів (директорів, managers)
та унтер-офіцерів (доглядачів за працею, foremen, overlookers,
contremaîtres), які підчас процесу праці командують іменем
капіталу. Праця нагляду стає їхньою постійною виключною
функцією. При порівнянні способу продукції незалежних селян
або самостійних ремісників із плянтаторським господарством,
що ґрунтується на рабстві, політико-економ залічує цю працю
нагляду до faux frais продукції\footnoteA{
Змалювавши «superintendence of labour»\footnote*{
нагляд над працею. \emph{Ред.}
} як головну характеристичну
рису рабської продукції в південних штатах Північної Америки,
професор Кернс каже далі: «Селянин-власник (на півночі), що привласнює
собі ввесь продукт своєї землі, не потребує іншої спонуки до праці.
Нагляд тут цілком зайвий». («The peasant proprietor appropriating the
whole produce of his soil, needs no other stimulus to exertion. Superintendence
is here completely dispensed with»). (\emph{Cairnes}: «The Slave Power»,
London 1862, p. 48, 49).
}. Навпаки, розглядаючи капіталістичний
спосіб продукції, він ідентифікує функцію керування,
оскільки вона виникає з природи спільного процесу праці, з
тією самою функцією, оскільки її зумовлює капіталістичний і
тому антагоністичний характер цього процесу\footnote{
Cep Джемс Стюарт, що взагалі дуже добре розбирається в характеристичних
суспільних одмінах різних способів продукції, зауважує:
«Чи не тому великі підприємства у промисловості руйнують приватні
підприємства, що вони більше наближаються до простоти рабського режиму?»
(«Why do large undertakings in the manufacturing way ruin private
industry, but by coming nearer to the simplicity of slaves?»). («Principles
of Political Economy», London 1767, vol. I, p. 167, 168).
}. Капіталіст не
тому капіталіст, що він промисловий керівник, навпаки, він стає
промисловим командиром через те, що він є капіталіст. Вища
команда у промисловості стає атрибутом капіталу так само, як
за февдальних часів вища команда на війні і в суді була атрибутом
земельної власности\footnoteA{
Тим то Оґюст Конт та його школа могли б доводити вічну доконечність
февдальних панів таким самим способом, як це вони робили
щодо панів капіталу.}.

Робітник є власник своєї робочої сили лише доти, доки він
як продавець її торгується з капіталістом, а продавати він може
лише те, що він має, свою індивідуальну, поодиноку робочу силу.
\parbreak{}  %% абзац продовжується на наступній сторінці

\input{_0269.tex}
\input{_0270.tex}
\parcont{}  %% абзац починається на попередній сторінці
\index{i}{0271}  %% посилання на сторінку оригінального видання
характеристичної форми якоїсь осібної епохи розвитку капіталістичного
способу продукції. Щонайбільше, вона є приблизно
така на початках мануфактури\footnote{
«Хіба поєднання вправности, працьовитости та змагання багатьох,
що виконують ту саму працю, не є спосіб посувати наперед цю працю?
І хіба Англія могла б якимось іншим способом довести свою вовняну мануфактуру
до такої досконалости?» («Whether the united skill, industry
and emulation of many together on the same work be not the way to advance
it? And whether it had been otherwise possible for England, to have
carried on her Woollen Manufacture to so great perfection?»). (\emph{Berkeley}: «The
Querist», London 1750, p. 56, § 521).
}, зорганізованої ще на ремісничий
штиб, та в тих великих рільничих господарствах, які відповідають
епосі мануфактури й посутньо відрізняються від селянського
господарства тільки масою одночасно вживаних робітників
та розміром сконцентрованих засобів продукції. Проста кооперація
все ще є переважна форма по таких галузях продукції, де
капітал оперує у великому маштабі, а поділ праці й машини не
відіграють значної ролі.

Кооперація лишається основною формою капіталістичного способу
продукції, хоч проста її форма сама з’являється як осібна
форма поряд інших розвиненіших її форм.

\section{Поділ праці та мануфактура}
\subsection{Двояке походження мануфактури}

Кооперація, що ґрунтується на поділі праці, утворює собі
свою клясичну форму в мануфактурі. Як характеристична форма
капіталістичного процесу продукції вона домінує протягом мануфактурного
періоду у власному значенні, який триває приблизно
від середини XVI віку до останньої третини XVIII.

Мануфактура виникає двояким способом.

Або робітників різнорідних самостійних реместв, що через
їхні руки мусить переходити продукт аж до останньої стадії його
виготовлення, згуртовують в одній майстерні під командою того
самого капіталіста. Приміром, карета була продуктом спільної
праці великого числа незалежних ремісників, як от стельмаха,
римаря, кравця, слюсаря, мідяра, токаря, позументаря, скляра,
маляра, лаківника, позолотника і~\abbr{т. ін.} Каретна мануфактура
сполучає всіх цих різних ремісників в одній майстерні, де вони
одночасно спільно працюють. Правда, карету не можна золотити
раніш, ніж її зроблено. Але якщо одночасно роблять багато карет,
то одну якусь частину можна завжди золотити, тимчасом як
інша частина пробігає ранішу фазу продукційного процесу. До
цього часу ми все ще стоїмо на ґрунті простої кооперації, яка
находить готовим свій людський та речовий матеріял. Алеж
дуже скоро настає ґрунтовна зміна. Кравець, слюсар, мідяр і~\abbr{т. ін.}, що працює тільки коло карет, утрачає крок за кроком
\parbreak{}  %% абзац продовжується на наступній сторінці


\index{iii2}{0272}  %% посилання на сторінку оригінального видання
Коли звести заробітну плату до її загальної основи, тобто до тієї частини
продукту власної праці, що входить в особисте споживання робітника: коли
звільнити цю частину від капіталістичних обмежень і поширити споживання до
такого розміру, який, з одного боку, допускається наявною продуктивною силою
суспільства (тобто суспільною продуктивною силою його власної праці як дійсно
суспільної) і якого, з другого боку, потребує цілковитий розвиток індивідуальности; коли звести далі
додаткову працю, й додатковий продукт до тих розмірів,
які при даних суспільних умовах продукції потрібні, з одного боку, для створення
страхового і резервного фонду, з другого боку, для безупинного поширення
репродукції в тій мірі, що визначається суспільною потребою; коли включити
нарешті, в № 1, в потрібну працю, і в № 2, в додаткову працю, ту кількість
праці, що її мусять завжди виконувати працездатні члени суспільства на ще
непрацездатних або вже непрацездатних членів суспільства; отже, коли таким
чином усунути всі специфічно капіталістичні риси так у заробітній платі, як і в
додатковій вартості, так у потрібній, як і в додатковій праці, — тоді перед
нами залишаться вже не ці форми, а лише їхні основи, спільні всім суспільним
способам продукції.

Проте, треба сказати, що таке підведення було властиве і колишнім панівним
способам продукції, наприклад, февдальному. Продукційні відносини, що
цілком не відповідали йому, стояли цілком поза ним, підводились під февдальні
відносини, наприклад в, Англії tenures in common socage\footnote*{
Володіння на основі панщини. \emph{Прим. Ред.}
} (протилежно до tenures
оn knight’s service)\footnote*{
Володіння на основі рабськоі праці. \emph{Прим. Ред.}
}, які мали в собі виключно грошові зобов’язання
лише з назви були февдальними.

\section{Розподільчі відносини й продукційні відносини}

Вартість, новостворювана щорічно нововитрачуваною працею, — отже, і та
частина річного продукту, що в ній визначається ця вартість і яка може бути
вилучена, виділена з сукупного продукту, — розпадається, отже на три частини,
що набувають трьох різних форм доходу, форм, які виражають одну частину
цієї вартости, як належну або взагалі припалу посідачеві робочої
сили, другу — посідачеві капіталу, третю — посідачеві земельної власности.
Отже, це є відносини або форми розподілу, бо вони визначають ті відносини,
що в них сукупна новоспродукована вартість розподіляється між посідачами
різних чинників продукції.

Згідно з звичайним поглядом ці відносини розподілу виступають як природні
відносини, відносини, що виникають з природи всякої суспільної продукції, з
законів людської продукції взагалі. Хоч і немає можливости заперечувати, що
докапіталістичні суспільства виявляють інші способи розподілу, проте, ці останні
тлумачаться, як нерозвинені, недосконалі й замасковані, що не досягли свого
найчистішого виразу і своєї найвищої форми, своєрідно забарвлені різностаті
цих природних розподільчих відносин.

В такому уявленні правильне лише одно: коли дано суспільну продукцію,
хоч би якого роду (наприклад, коли дано суспільну продукцію природно вирослої
індійської громади або більш штучно розвиненого перуанського комунізму),
то завжди можна відрізнити ту частину праці, що її продукт безпосередньо
особисто споживається продуцентами та їхніми    родинами, — лишаючи
осторонь працю, що припадає продуктивному споживанню, — від тієї
\parbreak{}  %% абзац продовжується на наступній сторінці


\index{ii}{0273}  %% посилання на сторінку оригінального видання
На основі суспільної продукції треба визначити маштаб, що в ньому
такі операції, які на довгий час відтягують робочу силу й засоби продукції,
не даючи протягом цього часу жодного продукту як корисного
наслідку, можуть провадитись без шкоди для тих галузей продукції, які
постійно або кілька разів на рік не лише відтягують робочу силу й засоби
продукції, а й дають засоби, існування й засоби продукції. За суспільної
продукції, так само, як і за капіталістичної продукції, робітники
в галузях підприємств з короткими робочими періодами, як і раніше, лише
на короткий час відтягуватимуть продукти, не даючи натомість нового
продукту, тимчасом як галузі підприємств з довгими робочими періодами,
перше ніж вони сами почнуть давати продукти, постійно відтягують
продукти на довгий час. Отже, ця обставина випливає з речових
умов відповідного процесу праці, а не з його суспільної форми. За суспільної
продукції грошовий капітал відпадає. Суспільство розподіляє робочу
силу й засоби продукції між різними галузями праці. Продуценти
можуть, правда, одержувати паперові посвідки, що ними вони вилучають
з суспільних споживних запасів ту кількість продуктів, яка відповідає їхньому
робочому часові. Ці посвідки — зовсім не гроші. Вони не циркулюють.

Тепер ми бачимо, що, оскільки потреба в грошовому капіталі випливає
з протягу робочого періоду, її зумовлено двома обставинами: п оперше,
тією, що гроші взагалі є та форма, що в ній мусить виступити
кожен індивідуальний капітал (кредит ми лишаємо осторонь) для того,
щоб перетворитись на продуктивний капітал. Це випливає з суті капіталістичної
продукції, взагалі товарової продукції. — Подруге, величину
потрібного грошового авансування зумовлює та обставина, що протягом
порівняно довгого часу суспільству постійно відбирається робочу силу
й засоби продукції, при чому протягом цього часу йому не повертається
жодного продукту, що його можна було б перетворити на гроші.
Першої обставини, а саме того, що авансовуваний капітал треба авансувати
в грошовій формі, не знищує форма самих цих грошей, тобто те,
що вони є або металеві, або кредитові гроші, або знаки вартости й~\abbr{т. ін.} На другу обставину жодного впливу не справляє те, за допомогою
яких грошових засобів або за допомогою якої форми продукції
відтягають працю, засоби існування та засоби продукції, не подаючи
натомість у циркуляцію жодного еквіваленту.

\input{_0274.tex}
\input{_0275.tex}
\parcont{}  %% абзац починається на попередній сторінці
\index{i}{0276}  %% посилання на сторінку оригінального видання
у знаряддях, які раніше служили для різних цілей. Пізнання
тих особливих труднощів, які викликає незмінена форма знарядь,
вказує напрям, у якому має йти зміна їхньої форми. Диференціяція
інструментів праці, в наслідок якої інструменти того самого роду
набувають особливих тривалих форм для кожного окремого
корисного вжитку, і їхня спеціялізація, в наслідок якої кожний
такий окремий інструмент функціонує в повному своєму обсязі
лише в руках спеціяльного частинного робітника, — це характеризує мануфактуру.
В самому лише Бірмінґемі продукують
якихось 500 відмін молотків, і з них кожний не тільки служить
для якогось осібного продукційного процесу, а ще й певне число
таких відмін служить часто лише для різних операцій у тому
самому процесі. Мануфактурний період спрощує, поліпшує та
робить різноманітнішими знаряддя праці, пристосовуючи їх до
виключних окремих функцій частинних робітників\footnote{
Щодо природних органів рослин і тварин Дарвін у своїй епохальній праці
«Постання родів» зауважує ось що: «Доки той самий орган має виконувати різні
праці, доти причину його змінливости, мабуть, можна знайти в тому, що природний
добір менш старанно зберігає або пригнічує кожний дрібний відхил у
формі, аніж тоді, коли той самий орган було б призначено виключно лише для
якогось одного окремого завдання. Так, ножі, які призначено на те, щоб різати
всяку всячину, можуть бути взагалі більш-менш однакової форми, тимчасом як
інструмент, призначений лише для якогось одного вжитку, мусить для кожного
іншого вжитку мати й іншу форму».
}. Тим самим він утворює одну з матеріяльних умов для вживання машин,
які складаються з комбінації простих інструментів.

Частинний робітник та його інструмент становлять прості
елементи мануфактури. Звернімось тепер до її цілого механізму.

\subsection{Дві основні форми мануфактури: гетерогенна мануфактура
й органічна мануфактура}

Організація мануфактури має дві основні форми, які, хоч
випадково й сплітаються одна з одною, становлять, однак, два
посутньо відмінні роди і відіграють цілком різні ролі при пізнішому перетворенні
мануфактури на машинову велику промисловість. Цей двоякий характер мануфактури
випливає з природи самого продукту. Цей останній або утворюється через просту
механічну сполуку самостійних частинних продуктів, абож
завдячує свою готову форму послідовному рядові зв’язаних між
собою процесів та маніпуляцій.

Льокомотив, приміром, складається більш ніж із \num{5.000} самостійних частин. Однак
його не можна вважати за приклад першого роду мануфактури у власному значенні,
бо він є витвір великої промисловости. Зате добрим прикладом може бути годинник,
що на ньому й Вільям Петті унаочнює мануфактурний поділ праці. З індивідуального
витвору нюрнберзького ремісника годинник перетворився на суспільний продукт
безлічі частинних робітників
таких, як от: Rohwerkmacher, виготівник годинникарських
\parbreak{}  %% абзац продовжується на наступній сторінці


\index{iii2}{0277}  %% посилання на сторінку оригінального видання

\section{Кляси.}

Власники самої тільки робочої сили, власники капіталу й земельні власники
що їхніми відповідними джерелами доходів є заробітна плата, зиск і земельна рента,
отже, наймані робітники, капіталісти й земельні власники становлять три великі
кляси сучасного суспільства, яке ґрунтується на капіталістичному способі
продукції.

В Англії сучасне суспільство своєю економічною структурою досягло безперечно
найвищого клясичного розвитку. Проте і тут це клясове розчленування не
виступає ще в цілком чистому вигляді. Навіть і тут середні й переходові ступені
всюди затемнюють межові лінії (правда в селі геть менше, ніж у містах).
А втім це не має значіння для нашого досліду. Ми вже бачили, що постійна
тенденція і закон розвитку капіталістичного способу продукції є в тому, що
засоби продукції дедалі більше відокремлюються від праці, і розпорошені засоби
продукції дедалі більше концентруються в значних масах, що, отже, праця
перетворюється на найману працю, а засоби продукції на капітал. І цій тенденції
відповідає на другому боці самостійне відокремлювання земельної власности
від капіталу й праці\footnote{
F. List слушно зауважує: «Перевага самодостатнього господарства у великих маєтках
свідчить тільки про брак цивілізації, засобів комунікації, тубільних промислів та багатих міст. Тому
ми й знаходимо його всюди в Росії, у Польщі, Угорщині, Мекленбурзі. Давніш воно панувало і в Англії;
з розвитком торговлі й промислу на його місці став поділ на господарства середньої величини та
здавання в оренду» (Die Ackerverfassung, die Zwergwirtschaft und die Auswanderung, 1842, p. 10).
}, або перетворення всякої земельної власности на форму
земельної власности, відповідну капіталістичному способові продукції.

Найближче питання, на яке треба відповісти, таке: що утворює клясу?
причому відповідь ця випливає сама собою з відповіді на інше питання: що
робить з найманих робітників, капіталістів і землевласників утворювачів трьох
великих суспільних кляс?

На перший погляд, це є тотожність доходів і джерел доходу. Перед нами
три великі суспільні групи, що їх члени, індивідууми, які утворюють ці групи,
живуть відповідно з заробітної плати, зиску й земельної ренти, використовуючи
свою робочу силу, свій капітал і свою земельну власність.

Але з цього погляду лікарі і урядовці, наприклад, становили б теж дві
кляси, бо вони належать до двох різних суспільних груп, причому члени кожної
з цих двох груп одержують свої доходи з того самого джерела. Те саме мало б
силу щодо безконечної роздрібнености інтересів і становищ, що до неї призводить
суспільний розподіл праці так серед робітників, як і серед капіталістів та
земельних власників, — напр., розчленовуючи останніх на посідачів виноградників,
орної землі, лісів, копалень, риболовель.

\begin{center}
[Тут рукопис уривається].
\end{center}

\parbreak{}  %% абзац продовжується на наступній сторінці


  \addtocontents{toc}{\protect\newpage}
  
\index{iii1}{0047}  %% посилання на сторінку оригінального видання

\chapter{Перетворення додаткової вартості в зиск і норми додаткової вартості в норму зиску}

\section{Витрати виробництва (kostpreis) і зиск}

В першій книзі були досліджені ті явища, які представляє капіталістичний
\emph{процес виробництва}, взятий сам по собі, як безпосередній
процес виробництва, при чому ще залишались осторонь
усі вторинні впливи чужих йому обставин. Але цей безпосередній
процес виробництва ще не вичерпує життьового шляху капіталу.
В дійсному світі він доповнюється \emph{процесом циркуляції},
який становив предмет досліджень другої книги. Там, саме в
третьому відділі, при розгляді процесу циркуляції як опосереднення
суспільного процесу репродукції, виявилось, що капіталістичний
процес виробництва, розглядуваний у цілому, є єдність
процесу виробництва і циркуляції. Завдання цієї третьої книги
не може полягати в тому, щоб дати загальні міркування про цю
єдність. Навпаки, тут треба знайти і описати ті конкретні форми,
які виростають з \emph{процесу руху капіталу}, \emph{розглядуваного як
ціле}. В своєму дійсному русі капітали протистоять один одному
в таких конкретних формах, для яких форма капіталу в безпосередньому
процесі виробництва, як і його форма в процесі циркуляції,
виступають тільки як особливі моменти. Отже, ті форми
капіталу, які ми описуємо в цій книзі, крок за кроком наближаються
до тієї форми, в якій вони виступають на поверхні
суспільства, в діянні різних капіталів один на одного, в конкуренції
і в звичайній свідомості самих діячів виробництва.

\pfbreak

Вартість кожного капіталістично виробленого товару $Т$ зображується
у формулі: $Т = c + v + m$. Якщо ми від цієї вартості
\parbreak{}  %% абзац продовжується на наступній сторінці


\index{iii2}{0048}  %% посилання на сторінку оригінального видання
[Поки стан справ є такий, що зворотний приплив зроблених авансувань
відбувається реґулярно, отже й кредит лишається незахитаним, пошир та
скорочення циркуляції реґулюється просто потребами промисловців та купців. Що
золото, принаймні в Англії, не має ваги для гуртової торговлі, а циркуляцію
золота — якщо не вважати на сезонні коливання — можна розглядати для довшого
часу як досить сталу величину, то й становить циркуляція банкнот Англійського
банку досить точне мірило ступеня цих змін. За тихих часів по кризі розмір
циркуляції є найменший, з новим оживленням попиту постає й більша потреба
на засоби циркуляції, ця потреба зростає з розвитком розцвіту; найвищої точки
кількість засобів циркуляції доходить в період надмірного напруження та надмірної
спекуляції, — тоді вибухає криза й за ніч зникають з ринку банкноти,
що їх ще вчора було багато, а з ними зникають і дисконтери векселів, і ті хто
дають гроші під цінні папери, і купці товарів. Англійський банк має допомагати,
— але й його сили незабаром вичерпані, банковий акт 1844 року змушує
його обмежувати циркуляції своїх банкнот саме тоді, коли весь світ криком
вимагає банкнот, коли державці товарів не можуть їх продавати, а проте повинні
платити та готові на всякі жертви, аби тільки одержати банкноти. «Підчас
переляку», каже вищезгаданий банкір Wright (1. с. № 2930), «країна потребує
удвоє більшої циркуляції, ніж за звичайних часів, бо банкіри й інші скупчують
собі про запас засоби циркуляції».

Скоро вибухав криза, справа вже тільки в платіжних засобах. А що
в надході цих платіжних засобів кожен залежить від іншого та ніхто не знає,
чи той інший в стані буде платити в реченець, то й настає справжня гонитва
за тими платіжними засобами, що є на ринку, тобто за банкнотами. Кожен
скупчує тих банкнот як скарб, скільки тільки він їх може одержати, і таким
чином банкноти зникають з циркуляції того самого дня, коли їх потребують
найбільше. Samuel Gurney (C. D. 1848/57, № 1116) визначає число банкнот,
прихованих під замок в момент паніки, в жовтні 1847 р. на суму 4—5 мільйонів
ф. ст. — Ф. Е.]

Щодо цього особливо цікаві свідчення перед банковою комісією 1857 року
спільника Gurney’ового, вже згаданого Chapman’а. Я подаю тут головний
зміст їх у зв’язному викладі, хоч в них розглядаються деякі пункти, що їх ми
дослідимо тільки пізніше. Пан Chapman дає таке свідчення.

«4963. Я не вагаючися скажу, що я не вважаю за доладне, коли грошовий
ринок має бути під владою будь-якого індивідуального капіталіста (а їх
в Лондоні є досить), що в стані утворювати величезну недостачу грошей та скруту
тоді, коли циркуляція є саме дуже низька... Це можливо... є не один капіталіст,
що може витягти з циркуляції банкнот на 1 чи 2 міл. ф. ст., якщо він
може тим досягти певної мети». 4995. Якийсь великий спекулянт може продати
консолів на 1 чи 2 міл. й таким способом забрати гроші з ринку. Дещо подібне
сталося зовсім недавно, «і це утворює незвичайно гостру скруту».

4967. Певна річ, банкноти тоді є непродуктивні. «Але нема чого тим
журитися, якщо таким способом можна осягнути великої мети; його велика
мета — збити ціни на фонди, утворити грошову скруту, а зробити це — цілком
в його силі». Приклад: одного ранку був великий попит на гроші на фондовій
біржі; ніхто не знав причини; хтось запропонував Chapman’oвi, щоб останній
позичив йому 50.000 ф. ст. з 7\%. Chapman здивувався, бо в нього рівень
проценту був значно нижчий; він згодився. Скоро по тому той чоловік прийшов
знову, взяв" знову 50.000 ф. ст. з 7 \sfrac{1}{2}\%, потім 100.000 ф. ст. з 8\% і хотів
взяти ще більшу суму з 8 \sfrac{1}{2}\%. Але тоді самого Chapman’a охопила тривога.
Потім виявилося, що раптом забрано з ринку значну суму грошей. Однак, каже
Chapman, «я проте визичив значну суму з 8\%; йти далі я боявся; я не знав,
що з того вийде».


\index{iii2}{0049}  %% посилання на сторінку оригінального видання
Не треба ніколи забувати, що хоч 19—20 мільйонів банкнотами, як
кажуть, перебуває в руках публіки досить стало, проте, та частина цих банкнот,
що є дійсно в циркуляції, з одного боку, і та частина їх, що лежить в банках
незайнята як запас, з другого боку, раз-у-раз та значно змінюються одна
проти однієї. Якщо цей запас великий, отже, рівень дійсної циркуляції низький,
то з погляду грошового ринку це значить, що сфера циркуляції є повна (the
circulation is full, money is plentifull); коли запас малий, отже коли рівень
дійсної циркуляції високий, то грошовий ринок зве його низьким (the circulation
is low, money is scarce); саме та частина являє низьку суму, що представляє
позичковий незайнятий капітал. Дійсний, від фаз промислового циклу незалежний,
пошир або скорочення циркуляції — так що однак та сума, що її потребує
публіка, лишається однаковою — буває лише з технічних причин, напр., коли настає
реченець платежа податків або процентів на державний борг. При платежі податків
банкноти та золото припливають до Англійського банку понад звичайну міру,
фактично скорочуючи циркуляцію, не зважаючи на потреби останньої. Навпаки
буває, коли виплачується дивіденди на державний борг. В першому випадку
роблять позики в банку на те, щоб добути засоби циркуляції. В останньому
випадку спадає рівень проценту в приватних банках з причини тимчасового
зросту їхніх резервів. Де не має нічого до діла з абсолютною масою засобів
циркуляції, а тільки має до діла з тією банковою фірмою, що пускає ці засоби
в циркуляцію і що з погляду її той процес видається вивласненням
позичкового капіталу, що й дає їй тому змогу ховати собі до кишені зиск
від того.

В одному випадку відбувається лише часове переміщення засобів циркуляції,
що його Англійський банк вирівнює тим способом, що незадовго перед реченцем
платежа чвертьрічних податків або виплати так само чвертьрічних дивідендів
він видає короткотермінові позики за низькі проценти; отож ці отак понад міру
видані банкноти спершу заповнюють ті прогалини, що їх викликав платіж
податків, тимчасом як їх зворотний платіж до банку зараз же по тому усовує
той надмір банкнот, що до його призводить виплата дивідендів публіці.

В другому випадку низький або високий рівень циркуляції завжди становить
лише інший розподіл тієї самої маси засобів циркуляції на активну циркуляцію
та вклади, тобто знаряддя позик.

З другого боку, коли, напр., через приплив золота до Англійського банку
більшає число банкнот, виданих за те золото, то ці останні допомагають
дисконтові поза банком та припливають назад на оплату позик, так що абсолютна
маса банкнот в циркуляції збільшується лише на короткий час.

Якщо циркуляція повна з причини поширу справ (що можливе й при порівняно
низьких цінах), то рівень проценту може бути, відносно високий з причини
попиту на позичковий капітал, що зумовлюється зростом зиску та збільшенням
змоги нових приміщень капіталу. Коли рівень циркуляції є низький в наслідок
скорочення справ або й великої поточности кредиту, то рівень проценту може бути
низький і при високих цінах (див. Hubbard).

Абсолютний розмір циркуляції впливає на рівень проценту, визначаючи
його, тільки підчас пригнічення. Тут попит на поширену циркуляцію означає
або лише попит на засоби до утворення скарбів (якщо не вважати на зменшену
швидкість, з якою гроші обертаються та з якою ті ж самі монети раз-у-раз
перетворюються на позичковий капітал) в наслідок відсутности кредиту, як от
в 1847 році, коли припинення банкового акту не викликало жодного поширу
циркуляції, але його вистачило, щоб нагромаджені скарбом банкноти знову
витягти на світ денний та кинути їх до циркуляції. Або ж у певних обставинах
дійсно може бути потрібно більше засобів циркуляції, як от в 1857 році, коли
циркуляція по припиненні банкового акту дійсно зросла на деякий час.


\index{iii2}{0050}  %% посилання на сторінку оригінального видання
В інших випадках абсолютний розмір циркуляції не впливає на рівень
проценту, поперше, тому, що — припускаючи економію та швидкість циркуляції
як сталі — той розмір циркуляції визначається цінами товарів та кількістю операцій
(при чому, здебільша, один момент паралізує вплив другого) та, насамкінець,
станом кредиту, тимчасом коли навпаки сам той розмір ніяк не визначає
цих факторів; і, подруге, тому, що товарові ціни та процент не мають між собою
ніякого неминучого зв’язку.

За тих часів, коли мав силу Bank Restriction Act (1797—1820 p. p.),
був надмір засобів циркуляції, рівень проценту був завжди далеко вищий, ніж
тоді, коли відновили платежі готівкою. Він швидко впав пізніше, коли обмежили
видання банкнот та підвищились вексельні курси. В 1822,1823, 1832 роках
загальний розмір циркуляції був низький, рівень проценту теж низький. В 1824,
1825, 1836 роках розмір циркуляції був високий, рівень проценту піднісся.
Улітку 1830 року циркуляція була висока, рівень проценту низький. Від часу
відкриття нових покладів золота розмір циркуляції грошей поширився по цілій
Европі, рівень проценту підвищився. Отже рівень проценту не залежить від
кількости грошей, що перебувають в циркуляції.

Ріжниця між випуском засобів циркуляції та визичанням капіталу найкраще
виявляється в дійсному процесі репродукції. Розглядаючи його, ми бачили (Книга II,
відділ III), як обмінюється різні складові частини продукції. Напр., змінний капітал
речово складається з життьових засобів робітників, з частини їхнього власного
продукту. Але його виплачують їм частинами у грошах. Ці гроші мусить
авансувати капіталіст, і від організації кредитової справи дуже залежить, чи
зможе він ближчого тижня знову виплатити новий змінний капітал старими
грішми, що він їх платив минулого тижня. Те саме бачимо ми в актах обміну
між різними складовими частинами сукупного суспільного капіталу, напр., між
засобами спожитку та засобами продукції тих засобів спожитку. Гроші, як ми
бачили, мусять для циркуляції авансуватися однією або обома особами, що обмінюються.
Потім гроші лишаються в циркуляції, але по закінченні обміну вони
раз-у-раз вертаються назад до того, хто їх авансував, бо він авансував їх зверх
свого дійсно занятого промислового капіталу (див. Книга II, 20 розділ). За розвинутої
кредитової справи, коли гроші концентруються в руках банків, ці останні,
принаймні, номінально, являють ту установу, яка авансує гроші. Це авансування
стосується тільки до тих грошей, що перебувають в циркуляції. Це — авансування
засобів циркуляції, а не авансування капіталів, що їх циркуляція обумовлюється
цим авансуванням.

Chapman: «5062. Може надійти час, коли банкноти в руках публіки становитимуть
дуже велику суму, а проте їх не можна добути». Гроші є й підчас
паніки; але кожен стережеться перетворювати їх на позичковий капітал, на
позичкові гроші; кожен міцно тримає їх для дійсної платіжної потреби.

«5099. Чи посилають банки сільських округ свої надміри вільних грошей
до вас та до інших лондонських фірм? — Так, — 5100. З другого боку, чи дисконтують
у вас фабричні округи Ланкашайру та Іїоркшайру векселі для своїх
промислових цілей? — Так. — 5101. Отже, цим способом зайві гроші однієї частини
країни стають пожиточні для потреб другої частини країни? — Цілком слушно».

Chapman каже, що звичай банків уживати свій надмірний грошовий капітал
на короткий час на купівлю консолів та посвідок державної скарбниці, цей звичай
за останній час дуже обмежився від того часу, коли стало звичаєм визичати ці
гроші at call (з дня на день, маючи змогу кожного часу вимагати їх назад).
Сам він вважає купівлю таких паперів для свого підприємства за незвичайно
недоцільну. Тому він приміщує гроші в добрі векселі, що для частини їх щодня
надходить реченець, так що він завжди знає, на скільки вільних грошей вік
має рахувати кожного дня. (5001—5005).
\parbreak{}  %% абзац продовжується на наступній сторінці

\input{_0051.tex}

\index{iii1}{0052}  %% посилання на сторінку оригінального видання
У цій формулі частина капіталу, витрачена на працю, відрізняється
від частини капіталу, витраченої на засоби виробництва,
наприклад, на бавовну або вугілля, тільки тим, що вона
служить для оплати речево відмінного елементу виробництва,
але ніяк не тим, що в процесі творення вартості товару, а тому
і в процесі зростання вартості капіталу вона відіграє функціонально
відмінну роль. У витратах виробництва товару ціна засобів
виробництва повертається назад такою, якою вона вже
фігурувала при авансуванні капіталу, і саме тому, що ці засоби
виробництва були доцільно використані. Цілком так само у витратах
виробництва товару ціна або заробітна плата за 666 \sfrac{2}{3}
робочих днів, витрачених на його виробництво, повертається
назад такою, якою вона вже фігурувала при авансуванні капіталу,
і так само якраз тому, що ця маса праці витрачена в доцільній
формі. Ми бачимо тільки готові, наявні вартості, — ті
частини вартості авансованого капіталу, які входять в утворення
вартості продукту, — але не бачимо елементу, який ств'орює
нову вартість. Ріжниця між сталим і змінним капіталом зникла.
Всі витрати виробництва у 500 фунтів стерлінгів набувають
тепер двоякого значення: поперше, вони є та складова частина
товарної вартості в 600 фунтів стерлінгів, яка заміщає капітал
у 500 фунтів стерлінгів, витрачений на виробництво товару;
і, подруге, сама ця складова частина вартості товару існує лише
тому, що вона раніш існувала як витрати виробництва застосованих
елементів виробництва, засобів виробництва і праці, тобто
як авансований капітал. Капітальна вартість повертається назад
як витрати виробництва товару тому і остільки, що і оскільки
її було витрачено як капітальну вартість.

Та обставина, що різні складові частини вартості авансованого
капіталу витрачені на речево різні елементи виробництва,
на засоби праці, сировинні й допоміжні матеріали і працю,
зумовлює тільки те, що на витрати виробництва товару доводиться
знову купити ці речево різні елементи виробництва.
Навпаки, щодо утворення самих витрат виробництва, то тут має
значення тільки одна ріжниця, ріжниця між основним і обіговим
капіталом. В нашому прикладі 20 фунтів стерлінгів були зараховані
на зношування засобів праці (400 с = 20 фунтам стерлінгів
на зношування засобів праці + 380 фунтів стерлінгів на
матеріали виробництва). Якщо вартість цих засобів праці перед
виробництвом товару була = 1200 фунтам стерлінгів, то після
його виробництва вона існує в двох виглядах: 20 фунтів стерлінгів
як частина товарної вартості, 1200—20, або 1180 фунтів стерлінгів,
як решта вартості засобів праці, які перебувають, як і раніш,
у володінні капіталіста, або як елемент вартості не його
товарного капіталу, а його продуктивного капіталу. В протилежність
до засобів праці, матеріали виробництва і заробітна плата
цілком витрачаються на виробництво товару, а тому і вся їх
вартість входить у вартість виробленого товару. Ми бачили,
\parbreak{}  %% абзац продовжується на наступній сторінці

\input{_0053_0054.tex}

\index{iii1}{0055}  %% посилання на сторінку оригінального видання
Тепер капіталістові ясно, що цей приріст вартості виникає
з продуктивних процесів, пророблених з капіталом, що він,
отже, виникає з самого капіталу; бо після процесу виробництва
він є, а перед процесом виробництва його не було. Насамперед,
щодо капіталу, витраченого на виробництво, то здається,
що додаткова вартість виникає рівномірно з різних елементів
його вартості, які існують у вигляді засобів виробництва і праці.
Адже ці елементи рівномірно входять в утворення витрат виробництва.
Вони рівномірно додають до вартості продукту свої
вартості, що наявні як авансування капіталу, і не відрізняються
один від одного як стала і змінна величини вартості. Це стає
цілком очевидним, коли ми на один момент припустимо, що
весь витрачений капітал складається або виключно з заробітної
плати, або виключно з вартості засобів виробництва. Ми мали б
тоді в першому випадку замість товарної вартості 400 с + 100 v +
100 m товарну вартість 500 v + 100 m. Витрачений на заробітну
плату капітал у 500 фунтів стерлінгів є вартість усієї праці,
вжитої на виробництво товарної вартості в 600 фунтів стерлінгів,
J. саме тому становить витрати виробництва всього продукту. Але
утворення цих витрат виробництва, в наслідок чого вартість витраченого
капіталу знову з’являється як складова частина вартості
продукту, є єдиний відомий нам процес в утворенні цієї товарної
вартості. Як виникає та її складова частина у 100 фунтів стерлінгів,
яка становить собою додаткову вартість, ми не знаємо.
Цілком те саме було б у другому випадку, де товарна вартість
була б = 500 с + 100 m. В обох випадках ми знаємо, що додаткова
вартість виникає з даної вартості тому, що ця вартість авансована
в формі продуктивного капіталу, однаково, чи то в формі праці,
чи в формі засобів виробництва. Але, з другого боку, авансована
капітальна вартість з тієї тільки причини, що вона витрачена
і тому становить витрати виробництва товару, не може
утворити додаткової вартості. Бо саме остільки, оскільки вона
становить витрати виробництва товару, вона утворює не додаткову
вартість, а тільки еквівалент, вартість, яка заміщає витрачений
капітал. Отже, оскільки вона утворює додаткову вартість,
вона утворює її не в наслідок своєї специфічної властивості як
витрачений капітал, а як авансований і тому застосований капітал
взагалі. Тому додаткова вартість в однаковій мірі виникає як
з тієї частини авансованого капіталу, що входить у витрати
виробництва товару, так і з тієї частини його, що не входить
у витрати виробництва; одним словом — в однаковій мірі з основних
і обігових складових частин застосованого капіталу. Весь
капітал речево служить як продуктотворець — засоби праці так
само, як і матеріали виробництва та праця. Весь капітал речево
входить у дійсний процес праці, хоч тільки частина його входить
у процес зростання вартості. Може, саме це і є причиною
того, що він тільки частиною бере участь в утворенні витрат виробництва,
але цілком — в утворенні додаткової вартості. Як би там
\parbreak{}  %% абзац продовжується на наступній сторінці

\input{_0056.tex}
\input{_0057.tex}

\index{iii2}{0058}  %% посилання на сторінку оригінального видання
Проте банки мають ще й інші засоби утворювати капітал. За тим самим
Newmarch’oм провінціяльні банки, як уже вище згадано, мають звичай відправляти
свої зайві фонди (тобто банкноти Англійського банку) лондонським billbrokers'aм,
що шлють їм натомість дисконтовані векселі. Цими векселями банк
обслуговує своїх клієнтів, бо він має за правило не видавати векселів, одержаних
від своїх місцевих клієнтів, щоб комерційні операції цих клієнтів не стали відомі
в їхній власній місцевості. Оці, одержані з Лондону, векселі служать не тільки
до того, щоб видавати їх клієнтам, які мають робити платежі безпосередньо
в Лондоні, якщо вони не вважатимуть за краще доручити банкові зробити
власний переказ на Лондон; ці векселі служать ще й до сплочування платежів
у провінції, бо передатний напис банкіра забезпечує їм місцевий кредит. Таким
чином, напр., в Ланкашайрі, витиснули ті векселі з циркуляції всі власні банкноти
місцевих банків та чималу частину банкнот Англійського банку (ibidem,
1568—74).

Отже, ми бачимо, як банки утворюють кредит та капітал: 1) виданням
власних банкнот; 2) виданням переказів на Лондон реченцем до 21 дня, переказів,
що їх однак в момент їхнього видання одразу оплачується банкам готівкою;
3) платежем дисконтованими векселями, що їхня кредитоздібність перед
усім та найголовніше, принаймні для відповідної місцевої округи, забезпечується
передатним написом банку.

Сила Англійського банку виявляється в реґулюванні ним ринкової норми
рівня проценту. Підчас нормального перебігу справ може трапитися, що Англійському
банкові не сила буде припинити помірний відплив золота з свого металевого
скарбу, підвищенням норми дисконту\footnote{
На загальних зборах акційний Union Bank of London 17 січня 1894 президент пан Ritchie
оповідав, що Англійський банк підвищив в 1893 році дисконт від 2 \sfrac{1}{2} (липень) до 3 та 4\% в серпні,
та, згубивши проте протягом чотирьох тижнів повних 4 \sfrac{1}{2} міл. ф. ст. золотом, до 5\%, після чого
золото почало припливати назад і банкову норму дисконту знизили в вересні до 4\%, а в жовтні до 3\%.
Але на ринку цю банкову норму не визнали. «Коли банкова норма була 5\%, то ринкова норма була З \sfrac{1}{2},
а норма для грошей була 2 \sfrac{1}{2}\%; коли банкова норма впала до 4\%, то норма дисконту була 2 \sfrac{3}{8}\%, а
грошова норма 1 \sfrac{3}{4}\%; коли банкова норма була 3\%, то норма дисконту була 1 \sfrac{1}{2}, а грошова норма
трохи нижча». (Daily News 18 січня 1894 р.) — Ф. Е).
}, бо потребу на платіжні засоби
задовольняють приватні й акційні банки та bill-brokers’и, що за останні тридцять
років набули значної сили на полі капіталу. Тоді має він уживати інших
засобів. Але для критичних моментів все ще має силу те, про що банкір Glyn
(з фірми Glyn, Mills, Currie and C°) свідчив перед С. D. 1848/57: «1709. Підчас
великої скрути в країні Англійський банк диктує рівень проценту — 1710. Підчас
надзвичайної скрути..., коли приватні банкірі або brokers’и порівняно обмежують
дисконтові операції, ці операції випадають Англійському банкові, й тоді він має
силу усталювати ринкову норму рівня проценту».

Звичайно, як офіційна установа, що має державну охорону та державні
привилеї, не сміє банк використовувати немилосердно цю свою силу, так як
можуть собі дозволити це приватні підприємства. Тому й Hubbard ось що каже
перед банковою комісією В. А. 1857; «2844 [питання]: А хіба не правда, що
коли норма дисконту є найвища, то Англійський банк обслуговує найдешевше,
а коли вона найнижча, тоді brokers’и обслуговують найдешевше? — [Hubbard]:
Так завжди буває, бо Англійський банк ніколи не знижує норми проценту так
низько, як його конкуренти, а коли норма є найвища, ніколи не підносить її
цілком так високо, як вони».

А проте серйозною подією в комерційному житті буває, коли банк підчас
скрути починає — уживаючи ходячого вислову — наганяти рівень проценту, тобто
ще вище підносити рівень проценту, що вже піднявся вище від пересічного. «Скоро
Англійський банк починає наганяти рівень проценту, припиняються всі закупи
для вивозу закордон... експортери чекають, поки спад цін дійде найнижчої точки
\parbreak{}  %% абзац продовжується на наступній сторінці


\index{iii1}{0059}  %% посилання на сторінку оригінального видання
Безглузде уявлення, ніби витрати виробництва товару становлять
його дійсну вартість, а додаткова вартість виникає
з продажу товару вище його вартості, що, отже, товари продаються
по їх вартостях, якщо їх продажна ціна дорівнює витратам
їх виробництва, тобто дорівнює ціні спожитих на них
засобів виробництва плюс заробітна плата, — це уявлення Прудон
з звичним шахрайством, яке чваниться вченістю, просурмив
як нововідкриту таємницю соціалізму. Це зведення вартості
товарів до витрат їх виробництва становить по суті
основу його народного банку. Раніше ми з’ясували, що різні складові
частини вартості продукту можна представити в пропорціональних
частинах самого продукту. Якщо, наприклад (книга І,
розд. VIІ, 2, стор. 229 \footnote*{
Стор. 153—154 рос. вид. 1935 р. \emph{Ред. укр. перекладу}.
}), вартість 20 фунтів пряжі становить
30 шилінгів — а саме 24 шилінги засоби виробництва, 3 шилінги
робоча сила і 3 шилінги додаткова вартість, — то цю додаткову
вартість можна представити як \sfrac{1}{10} продукту = 2 фунтам пряжі.
Тепер, якщо ці 20 фунтів пряжі продаються по витратах їх
виробництва, за 27 шилінгів, то покупець дістає даром 2 фунти
пряжі, або товар продано на \sfrac{1}{10}  нижче його вартості; але робітник
так само, як і раніш, дав свою додаткову працю — тільки
для покупця пряжі, а не для капіталістичного виробника пряжі.
Було б цілком помилково припускати, що коли б усі товари
продавались по витратах їх виробництва, то результат фактично
був би той самий, як коли б усі товари продавались вище витрат
їх виробництва, але по їх вартостях. Бо навіть коли припустити,
що вартість робочої сили, довжина робочого дня
і ступінь експлуатації праці повсюди однакові, то все ж маси
додаткової вартості, які містяться у вартостях різних видів
товару, аж ніяк не рівні, залежно від різного органічного складу
капіталів, авансованих на їх виробництво\footnote{
„Вироблювані різними капіталами маси вартості і додаткової вартості, при
даній вартості і однаковому ступені експлуатації робочої сили, прямо пропорціональні
до величин змінних складових частин цих капіталів, тобто їх складових
частин, перетворених у живу робочу силу“ (книга 1, розд. ІХ, стор. 321
[стор. 227 рос. вид. 1935 р.]).
}.

\section{Норма зиску}

Загальна формула капіталу є $Г — Т — Г'$; тобто певна сума
вартості кидається в циркуляцію для того, щоб витягти з неї
більшу суму вартості. Процес, який породжує цю більшу суму
вартості, є капіталістичне виробництво; процес, який реалізує
її, є циркуляція капіталу. Капіталіст виробляє товар не ради
самого товару, не ради його споживної вартості або свого особистого
споживання. Продукт, який в дійсності цікавить капіталіста,
\index{iii1}{0060}  %% посилання на сторінку оригінального видання
це не сам відчутний продукт, а надлишок вартості продукту
понад вартість спожитого на нього капіталу. Капіталіст
авансує весь капітал, не звертаючи уваги на ті різні ролі, що їх
відіграють складові частини капіталу у виробництві додаткової
вартості. Він однаково авансує всі ці складові частини капіталу
не тільки для того, щоб репродукувати авансований капітал, але
і для того, щоб виробити певний надлишок вартості понад
цей капітал. Він може перетворити вартість змінного капіталу,
який він авансує, у вищу вартість тільки через обмін його на
живу працю, через експлуатацію живої праці. Але він може експлуатувати
працю тільки в тому разі, коли він одночасно авансує
умови для здійснення цієї праці — засоби праці і предмет
праці, машини і сировинний матеріал, тобто коли він ту суму
вартості, якою він володіє, перетворює в форму умов виробництва;
як і взагалі, він тільки тому є капіталіст, тільки тому взагалі
може взятися до процесу експлуатації праці, що він як власник
умов праці протистоїть робітникові як володільцеві тільки робочої
сили. Вже раніше, в першій книзі, було показано, що саме
те, що цими засобами виробництва володіють не-робітники, перетворює
робітників у найманих робітників, а не-робітників — у капіталістів.

Капіталістові байдуже, чи розглядати справу так, що він
авансує сталий капітал для того, щоб здобути бариш із змінного,
чи так, що він авансує змінний капітал для того, щоб збільшити
вартість сталого; чи так, що він витрачає гроші на заробітну
плату для того, щоб надати машинам і сировинному матеріалові
вищу вартість, чи так, що він авансує гроші на машини та сировинний
матеріал для того, щоб мати можливість експлуатувати працю.
Хоч додаткову вартість утворює лише змінна частина капіталу,
проте вона утворює її тільки при тій умові, що авансуються
й інші частини, виробничі умови праці. Через те що капіталіст
може експлуатувати працю тільки за допомогою авансування сталого
капіталу, що він може збільшити вартість сталого капіталу
тільки за допомогою авансування змінного, то в його уявленні
ці капітали збігаються, і це тим більше, що дійсний рівень його
баришу визначається відношенням не до змінного капіталу, а
до всього капіталу, не нормою додаткової вартості, а нормою
зиску, яка, як ми побачимо, може лишатись однаковою і все ж
виражати різні норми додаткової вартості.

До витрат виготовлення (Kosten) продукту належать усі
складові частини його вартості, які капіталіст оплатив або еквівалент
яких він кинув у виробництво. Ці витрати мусять бути
заміщені для того, щоб капітал просто зберігся або репродукувався
в своїй первісній величині.

Вартість, яка міститься в товарі, дорівнює тому робочому
часові, якого коштує його виготовлення, а сума цієї праці складається
з оплаченої і неоплаченої праці. Навпаки, для капіталіста
витрати виготовлення товару складаються тільки з тієї
\parbreak{}  %% абзац продовжується на наступній сторінці

\index{franko}{0061}

\setcounter{chapter}{23}
\section[Початок і історичний розвиток
капіталістичної продукції в Англії]{Початок і історичний розвиток капіталістичної продукції в Англії\footnotemarkZ{}}
\markboth{Початок і історичний розвиток
капіталістичної продукції в Англії}{Фрагмент «Капіталу» у~перекладі Івана~Франка}

\footnotetextZ{
Вперше надруковано в журн. «Культура», 1926, № 4--9, с. 61--87.

\nopagebreak[4]
Подається за автографом перекладача: відділ рукописних фондів і текстології Інституту літератури ім. Т.~Г.~Шевченка НАН України. — Ф. 3. — Од. зб. 448. — 14 арк. Кінець автографа не зберігся. 

\nopagebreak[4]
Переклад зроблено з другого німецького видання: \textgerman{Das Kapital. Kritik der politischen Oekonomie. Von Karl Marx. Erster Band. Zweite verbesserte Aufgabe. Hamburg. Verlag von Otto Meissner, 1872.} Про це є згадка І. Франка на початку тексту перекладу «Гл[яди] К. Marx. Das Kapital, 2 вид. з р. 1872, стор. 742--794».}

\subsection{Первісне нагромадженє капіталу}

Ми бачили, що гроші стают капіталом тоді, коли служат
до купованя робучої сили. Ми бачили, що капітал
раз-ураз намагає — творити надзвишку вартости, а надзвишка —
вбільшує капітал. Між тим, щоб капітал міг нагромаджуватись,
мусит уже вперед витворюватись надзвишка;
щоб могла витворюватись надзвишка, мусит істнувати капіталістична
продукція, а щоб тота істнувала, мусит уже
вперед більша маса капіталу бути нагромаджена в руках
поєдинчих богатирів. Здаєсь затим, що весь той процес
полягає на якімось „первіснім“ нагромадженю, котре мало
місце перед капіталістичною продукцією, котре, значит, не
було випливом капіталістичної продукції, а єї жерелом.

\index{franko}{0062}
Тото первісне нагромадженє капіталу („previous accumulation“,
як каже А.~Сміт) грає в суспільній економії
майже таку саму ролю, як „гріхопаденіє“ в теольоґії. Адам
зїв яблоко і через те стягнув гріх на рід людський. Початок
гріха обяснений казкою про давнину. Колись-колись
в давнину були з одного боку пильні вибранці, а з другого —
ліниві нероби. Через те сталося, що перші нагромадили
богацтво, а другі зійшли на таке, що остаточно не мали
вже що продавати крім себе самих. І від того гріхопаденія
почалася бідність великої маси, котра ще й доси, хоть і як
тяжко працює, не має що продавати крім себе самих, —
і богацтво деяких, що й доси змагаєся, хоть самі вони
давно перестали працювати\footnote{
Такі безглузді дитиньства плете ще д. Тйер (звісний французький
муж стану) дотепним колись Французам с повагою великого мудрця —
для оборони святої власности. Ну і справді, — скоро діло йде о власність,
то святий обовязок кождого — міцно стояти на становищи букваря,
ще й других переконувати, що те становище для всякого „віка
і возраста“ єдино відповідне і належне.
}. В правдивій історії грали, як
звісно, завойованя, гнет, рабунки, вбійства, — одним словом,
усілякі насиля велику ролю. Але в сумирній політичній
економії з давен-давна — все іділлія. Право і „праця“, се
здавна були єдині способи до збогаченя, тілько, розумієся,
завсігди с тим застереженєм, що аж „сего року воно щось
не так“. Але на ділі способи первісного нагромадженя капіталу
були всякі, які хочете, — тілько не іділлічні.

Гроші і товар не є зразу капіталом, так само, як не
є ним зразу средства продукційні і знадоби до житя. Вони
мусят бути перемінені в капітал. Але та переміна може настати
тілько серед певних обставин, котрі зводятся ось на
що: двоякі дуже відмінні посідачі товарів мусят стати супротів
себе і зіткнутися с собою, — з одного боку властивці
грошей, средств продукційних і знадіб до житя, котрим
о то йде, щоб свою суму вартостей побільшити купівлею
чужої робучої сили; а з другого боку вільні робітники, продавці
власної робучої сили і, значит, продавці \so{праці}.
Вільні вони мусят бути в двоякім значіню, т. є. щоб ані
самі вони беспосередно не були средствами продукційними,
як невольники, кріпаки і т. д., ані шоб вони самі не посідали
средств продукційних, як ґазди-селяне, дрібні властивці
ґрунтові і т. д. Такий розділ товарів між дві крайности
— се основні вимінки для капіталістичної продукції.
Без відділеня робітників від власности не може настати
капіталістична продукція. Але скоро вона раз настала, то
не тілько підтримує те відділенє, але й сама доводит до
него раз-ураз наново і раз-ураз на більший розмір. Коли
затим спитаємо: де є жерело капіталістичного ладу? то
\parbreak{}

\parcont{}
\index{franko}{0063}
відповідь на те дуже проста: жерело капіталістичного ладу,
се не що їнше, як той процес \so{відділюваня робітника
від власности, від средств продукційних}.
Сей процес з одного боку перемінює суспільну істенину
(средства продукційні і знадоби до житя) в капітал, а з другого
боку перемінює беспосередних витвірців в наємних
робітників. Так назване „первісне нагромадженє капіталу“,
се затим не що їнше, як історичний процес відділюваня
продуцента від средств продукційних. Він і справді „первісний“,
бо становит вступ до історії капіталу і капіталістичної
продукції.

На перший погляд видно, що той процес роскладовий
обнимає собою цілий ряд історичних процесів і то ряд двоякий:
з одного боку нищенє тих відносин, котрі робітника
робили власністю третих осіб, їх средством продукційним,
— з другого боку вивласнюванє беспосередних витвірців,
витисканє їх с посіданя средств продукційних. Процес роскладовий,
се затим на ділі ціла історія розвитку новійшої
буржоазної суспільности. Се булаб зовсім не трудна історія,
колиб буржоазні історики не були єї нам вказали виключно
в рожевім світлі еманціпації робітників, а булиб звернули
увагу й на то, якими способами в тій історії визискуванє
феодальне перемінилося в визискуванє капіталістичне. Початок
розвитку становила неволя робітника. В дальшім тягу
того розвитку неволя осталась, тілько в зміненій формі.
Але ми ту не будем вдаватися в розбір середновікових рухів.
Хоть капіталістична продукція вже в \RNum{14} і \RNum{15} віці розпочалася
в деяких місцях над Середземним морем, то прецінь
ера капіталістична починаєся аж від \RNum{16} віку. Там,
де вона росцвитає, давно вже знесено панщину і середновікове
міщанство також як раз хилится до впадку.

Епохи в історії того роскладового процесу становят ті
хвилі, коли великі маси людей нараз і силою відривано від
усіх средств до житя та праці і як свобідних і голих пролєтаріїв
перто на робучий торг. Вивласнюванє робітників
з ґрунту і посідлости становит основу цілого процесу. Тож
і ми насамперед мусимо переглянути історію того вивласненя.
В різних краях вона проявляєся в різних окрасках
і переходит різні фази в неоднакім порядку. Тілько в Англії,
котру ми проте беремо за примір, вона має клясичну форму\footnote{
В Італії, де капіталістична продукція розвилась найраньше, найраньше
також увільнено кріпаків. Тількож при тім увільненю вони не
одержали права на ґрунти, хотьби й за сплатою індемнізації, так що
„воля“ перемінила італіянських кріпаків відразу в голих пролєтаріїв, котрі
крім того по містах, стоячих ще переважно від римських часів, найшли
вже готових нових панів.
}.


\index{iii1}{0064}  %% посилання на сторінку оригінального видання
Величина вартості всього капіталу сама по собі не стоїть
у будьякому внутрішньому відношенні до величини додаткової
вартості, принаймні не стоїть безпосередньо. Щодо своїх речових
елементів весь капітал мінус змінний капітал, отже, сталий капітал,
складається з речових умов здійснення праці, з засобів праці
і матеріалу праці. Для того, щоб певна кількість праці реалізувалась
у товарах і, отже, утворила вартість, потрібна певна
кількість матеріалу праці і засобів праці. Залежно від особливого
характеру додаваної праці існує певне технічне відношення
між масою праці і масою засобів виробництва, до яких повинна
бути додана ця жива праця. Отже, остільки існує також певне
відношення між масою додаткової вартості або додаткової праці
і масою засобів виробництва. Якщо, наприклад, час, необхідний
для виробництва заробітної плати, становить 6 годин на день,
то робітник мусить працювати 12 годин, щоб дати 6 годин додаткової
праці, щоб створити додаткову вартість у 100\%. Він
споживає за ці 12 годин удвоє більше засобів виробництва, ніж
за ці 6 годин. Але від цього додаткова вартість, яку він додає
за 6 годин, зовсім не стає в будьяке безпосереднє відношення
до вартості засобів виробництва, спожитих за ці 6 чи навіть
за ці 12 годин. Ця вартість тут не має ніякого значення; ідеться
тільки про технічно необхідну масу. Чи сировинний матеріал або
засоби праці дешеві, чи дорогі, це не має ніякого значення;
аби тільки вони мали потрібну споживну вартість і були наявні
в технічно встановленій пропорції до тієї живої праці, яку треба
поглинути. Однак, якщо я знаю, що за одну годину перепрядається
$х$ фунтів бавовни, які коштують $а$ шилінгів, то я, звичайно,
знаю і те, що за 12 годин перепрядається 12 $х$ фунтів
бавовни = 12 $а$ шилінгам, і тоді я можу обчислити відношення
додаткової вартості до вартості цих 12, так само як і до вартості
цих 6. Але відношення живої праці до \emph{вартості} засобів
виробництва тут привходить лиш остільки, оскільки $а$ шилінгів
служать назвою для $х$ фунтів бавовни; бо певна кількість бавовни
має певну ціну, а тому й навпаки, певна ціна може служити
показником певної кількості бавовни, поки ціна бавовни
не зміниться. Якщо я знаю, що для того, щоб привласнити 6 годин
додаткової праці, я повинен примушувати працювати 12 годин,
отже, мушу мати напоготові бавовни на 12 годин, і якщо я знаю
ціну цієї потрібної для 12 годин кількості бавовни, то посередньо
існує відношення між ціною бавовни (як показником необхідної
кількості) і додатковою вартістю. Навпаки, з ціни сировинного
матеріалу я ніколи не можу зробити висновок про масу сировинного
матеріалу, яка може бути перепрядена, наприклад, за
одну годину, а не за 6. Отже, немає ніякого внутрішнього, необхідного
відношення між вартістю сталого капіталу, — а тому
і між вартістю всього капіталу ($= с + v$) і додатковою вартістю.

Якщо норма додаткової вартості відома і величина додаткової
вартості дана, то норма зиску виражає не що інше, як
\parbreak{}  %% абзац продовжується на наступній сторінці

\index{franko}{0065}

Перший крок перевороту, що поклав основу капталістичній продукції, припадає в послідній третині \RNum{15} і
в першій чверти \RNum{16} віку. Тоді скасовано феодальне дворацтво, котре, як справедливо замічає Джемс
Стюерт, „залякало  всі хати і двори безхосенно“. Через те викинено масу голих пролєтаріїв на
робучий торг. Хоть королівська власть, що й сама виросла з буржуазного розвитку, намагаючи до
неограниченого панованя, силою скасувала те великопанське дворацтво, то прецінь вона не була єдиною
причиною нового перевороту. Ні, в упертім опорі протів королівства та
парляменту витворили великі пани-феодали далеко більшу масу пролєтаріяту, прогонюючи силою
хліборобів з ґрунту і посідлости, хоть хлібороби мали до тих ґрунтів більше право, ніж вони, і
забираючи для себе громадські ґрунти. Беспосередний товчок до того в Англії дав іменно росцвіт
фляндрійської вовняної мануфактури і звязане з ним підскоченє цін вовни. Стара феодальна шляхта
вигибла в великих феодальних війнах, а нова шляхта — се були діти свого часу, для котрих гроші були
силою понад всі сили. З вірного поля пасовиська для овець! — се став тепер їх загальний оклик.
Гаррізен в своїй „Description of England. Prefixed to Holinshed’s Chronicles“ описує, як
вивласнюванє дрібних ґаздів руйнує край. „Але що нашим великим самозванцям до того?“ Мешканя ґаздів
та коттеджі робітників валят вони силою або прогнавши людей лишают пустками. „Коли перездримо
давнійші інвентарі кождої домінії, то побачимо, що незлічимі хати та дрібні ґаздівства пощезали, що
ґрунт годує далеко меньше люда, що богато міст підупало, хоть деякі нові підносятся\dots{} Мож би
чимало наросповідатися про місточка та села, зруйновані для того, щоб було місце на толоки для
овець; тілько самотні панські двори стоят серед тих толок“. Правда, наріканя тих старих літописів
усе пересаджені, але вони досадно малюют те вражінє, яке на самих сучасників робив переворот
обставин продукційних. Порівнанє між письмами канцлєрів Фортеске і Томаса Моруса вказує наглядно
пропасть між \RNum{15} а \RNum{16} віком. „Із золотого віку — каже справедливо Зорнтон — попали англійські
робітники без ніяких перехідних ступнів прямо в зелізну“.

Праводавство злякалось сего перевороту. Воно не стояло ще на такім високім ступни цівілізації, де
„богацтво народне“, т. є. богацтво капіталістів і безграничне висисанє та зубожінє маси люду
становит верх премудрости
політичної. В своїй історії Генріха VII каже Бекон: „В тім часі (1489) посипалися скарги на то, що
вірне поле перемінюєсь в пасовиська, котрих лехко може дозирати кілька пастухів. Ґрунти, що вперед
виарендовувались на кілька літ, на доживотну або щорічну умову, тепер зіллято разом
\index{franko}{0066}
с панськими. Се підкопало добробуток люду, а через те й міста, церкви, десятини\dots{} Щоб зарадити
тому лиху, проявили король і парлямент дивну на ті часи мудрість\dots{} Вони видали право протів того
обезлюднюючого край загарбуваня громадських ґрунтів (depopulating inclosures) і невідлучної
від него обезлюднюючої ґосподарки толочної (depopulating pasture[s])“. Оден акт Генріха VII з р.~
1489 заказує руйнувати хліборобські хати, до котрих належит що найменьше 20 екрів ґрунту. Генріх
VIII відновив той самий указ. Говорится там між їншим, що „многі аренди і огромні отари, особливо
овець, нагромаджуются в немногих руках, через що дохід
з ґрунту дуже вбільшився, а рільництво дуже підупало, церкви і хати повалено, дивовижні маси народа
стали неспосібні вдержувати себе і свої родини“. Указ наказує затим відбудовувати повалені хутори,
означує, кілько має бути вірного поля в стосунку до овечих толок і т. д. Їнший акт з р.~1533
жалуєсь, що деякі властивці мают по \num{24000} овець, і ограничує їх число на 2000\footnote{
В своїй „Утопії“ говорит Томас Морус про дивовижний край, де
„вівці їдят людей“.
}. Наріканя народа і
праводавство протів вивласнюваня дрібних арендаторів та хліборобів, що почалось від Генріха VII і
трівало зо 150 літ
— не помогли нічо. Чому не помогли, пояснює нам Бекон, сам того не знаючи. „Акт Генріха~VII, — каже
він в своїх „Essays, civil and moral“, Sect. 20, — був глибоко і дивно обдуманий. Він утворив
сільскі ґаздівства і хліборобські доми певного нормального розміру, т. є. вдержав для них таку
пропорцію ґрунту, котра давала їм змогу плодити на світ підданих доста заможних і не придавлених
нуждою, так що плуг був в руках властивців, а не наємників\footnote{
Бекон пояснює далі звязок між свобідним, заможним селянством
а доброю інфантерією. „Се була дивно важна річ для сили і мужности
королівства — мати аренди достаточного розміру, щоб дільних мужів
забеспечити від нужди і велику часть ґрунту краєвого запевнити в посіданє джоменам, т. є. людім
середної заможности між шляхтою а халупниками (cottagers) та наймитами. Бо се загальна думка
найліпших знавців воєнного діла\dots{} що головна сила армії, се інфантерія або піхота. Але щоб
витворити добру інфантерію, тре людей вихованих не в притиску ані в нужді, але свобідно і в певній
заможности. Коли затим яка держава вросте переважно в шляхту та делікатне панство, а хлібороби та
ратаї зійдут на простих зарібників та наймитів або халупників, т. є. жебраків з власною хатою, то
така держава може мати добру кінницю, але доброї піхоти не буде мати. Се видно в Італії і Франції і
деяких других заграничних краях, де справді все або шляхта або нужденні зарібники\dots{} Дійшло там до
того, що ті краї мусят уживати наємного зброду Швейцарів та др. для своєї піхоти: відти то й пішло,
що ті держави мают богато людий, а мало вояків“. („The Reign of Henry VII“ і т.~д.).
}. А між
\parbreak{}


\index{iii1}{0067}  %% посилання на сторінку оригінального видання
Ми зберігаємо позначення, вжиті в першій і другій книгах.
Весь капітал $К$ поділяється на сталий капітал $с$ і змінний капітал
$v$ і виробляє додаткову вартість $m$. Відношення цієї додаткової
вартості до авансованого змінного капіталу, отже $\frac{m}{v}$, ми
називаємо нормою додаткової вартості і позначаємо її через $m'$.
Отже, $\frac{m}{v} = m'$, і тому $m = m'v$. Якщо цю додаткову вартість
віднести не до змінного капіталу, а до всього капіталу, то вона
зветься зиском $(р)$, а відношення додаткової вартості $m$ до
всього капіталу $К$, отже $\frac{m}{K}$, зветься нормою зиску $р'$. Звідси ми
маємо:\[
р' = \frac{m}{K} = \frac{m}{с + v},
\]
якщо ми замість m підставимо його знайдену вище величину
$m'v$, то матимем:\[
р' = m'\frac{v}{К} = m'\frac{v}{c + v},
\]
рівняння, яке можна виразити також у пропорції:\[
р':m' = v:К,
\]
норма зиску відноситься до норми додаткової вартості, як змінний
капітал до всього капіталу.

З цієї пропорції випливає, що $р'$, норма зиску, завжди менша
від $m'$, норми додаткової вартості, бо $v$, змінний капітал, завжди
менший від $К$, суми $v + c$, змінного і сталого капіталу; за винятком
єдиного, практично неможливого випадку, коли $v = K$, отже,
коли капіталістом зовсім не авансувався б сталий капітал, засоби
виробництва, а тільки заробітна плата.

В нашому дослідженні треба, однак, звернути увагу ще на
ряд інших факторів, які визначально впливають на величину $с$,
$v$ і $m$ і тому мають бути коротко згадані.

Поперше, \emph{вартість грошей}. Її ми можемо повсюди приймати
за сталу.

Подруге, \emph{оборот}. Цей фактор ми покищо лишаємо осторонь,
бо його вплив на норму зиску розглядається окремо в одному
з дальших розділів. [Тут ми, забігаючи наперед, згадаємо тільки
про той один пункт, що формула $р' = m'\frac{v}{К}$ є строго правильна
лиш для \emph{одного} періоду обороту змінного капіталу, і що ми,
однак, можемо її зробити правильною для річного обороту, поставивши
замість $m'$, простої норми додаткової вартості, $m'n$,
річну норму додаткової вартості; при чому n є число оборотів
змінного капіталу протягом одного року (див. книгу II, розд.
XVI, 1) — Ф. Е]


\index{iii1}{0068}  %% посилання на сторінку оригінального видання
Потретє, треба взяти до уваги \emph{продуктивність праці}, вплив
якої на норму додаткової вартості докладно розглянуто в книзі І,
відділ IV. Але вона може також справляти і безпосередній
вплив на норму зиску, принаймні окремого капіталу, коли, як
де розвинено в книзі І, розділ X, цей окремий
капітал працює з продуктивністю більшою, ніж суспільнопересічна,
дає свої продукти по вартості нижчій, ніж суспільно-пересічна
вартість таких самих товарів, і таким чином
реалізує надзиск. Але цей випадок лишається тут поза нашим
розглядом, бо і в цьому відділі ми все ще виходимо з припущення,
що товари виробляються при суспільно-нормальних
умовах і продаються по їх вартостях. Отже, в кожному окремому
випадку ми виходимо з припущення, що продуктивність
праці лишається сталою. Справді, вартісний склад капіталу,
вкладеного в певну галузь промисловості, тобто певне відношення
змінного капіталу до сталого капіталу, кожного разу
виражає певний ступінь продуктивності праці. Отже, як тільки
це відношення зазнає зміни з іншої причини, а не в наслідок
простої зміни вартості матеріальних складових частин сталого
капіталу або зміни заробітної плати, то й продуктивність праці
мусить зазнати зміни, і тому ми досить часто бачитимем, що
зміни, які відбуваються з факторами $с$, $v$ і $m$, включають також
і зміни в продуктивності праці.

Те саме стосується і до трьох інших факторів: \emph{довжини робочого
дня, інтенсивності праці і заробітної плати}. Їх вплив
на масу і норму додаткової вартості докладно досліджено в першій
книзі. Отже, зрозуміло, що коли ми задля спрощення завжди
виходимо з припущення, що ці три фактори лишаються сталими,
то все ж ті зміни, які відбуваються з $v$ і $m$, можуть також включати
в собі зміну в величині цих трьох визначаючих їх моментів.
Тут слід тільки коротко нагадати про те, що заробітна плата
діє на величину додаткової вартості і висоту норми додаткової
вартості зворотно до того, як діє довжина робочого дня і інтенсивність
праці; що підвищення заробітної плати зменшує додаткову
вартість, тимчасом як здовження робочого дня і підвищення
інтенсивності праці збільшують її.

Якщо припустимо, наприклад, що капітал у 100 з 20 робітниками
виробляє при десятигодинній праці та загальній тижневій
заробітній платі в 20 додаткову вартість у 20, то ми
матимем:\[
80 с + 20 v + 20 m; m' = 100\%, р' = 20\%.
\]
Припустімо, що робочий день здовжується, без підвищення
заробітної плати, до 15 годин; в наслідок цього вся нововироблена
20 робітниками вартість підвищується з 40 до 60 (10 : 15 = 40 : 60);
\parbreak{}  %% абзац продовжується на наступній сторінці

\input{_0069c.tex}
\input{_0070c.tex}

\index{iii1}{0071}  %% посилання на сторінку оригінального видання
Якщо ми тепер визначимо відношення $К$ і $К_1$, а також
$v$ і $v_1$, припустимо, наприклад, що значення дробу $\frac{К_1}{К} = Е$, а дробу
$\frac{v_1}{v} = е$, то $К_1 = Е К$, а $v_1 = е v$. Підставивши тепер у попередньому
рівнянні для $р'_1$, $К_1$ і $v_1$ здобуті таким чином значення,
ми матимем:\[
р'_1 = m'\frac{еv}{ЕК}.
\]

Але з обох попередніх рівнянь ми можемо вивести ще й другу
формулу, перетворивши їх у пропорцію:\[
р': р'_1 = m'\frac{v}{К} : m' \frac{v_1}{K_1} = \frac{v}{К} : \frac{v_1}{К_1}.
\]
Через те, що величина дробу не змінюється, коли чисельник
і знаменник помножити або поділити на те саме число, ми можемо
$\frac{v}{К}$ і $\frac{v_1}{К_1}$ звести до процентних чисел, тобто припустити, що
$К$ і $К_1 = 100$. Тоді ми матимем $\frac{v}{К} = \frac{v}{100}$ і $\frac{v_1}{К_1} = \frac{v_1}{100}$, і можемо відкинути
у наведеній пропорції знаменники; ми одержуємо:\[
р' : р'_1 = v : v_1\text{; або:}
\]
При двох довільно взятих капіталах, які функціонують з однаковою
нормою додаткової вартості, норми зиску відносяться
одна до одної як змінні частини капіталу, обчислені у процентах
до своїх відповідних цілих капіталів.

Ці дві формули охоплюють усі випадки змін $\frac{v}{К}$.

Раніш ніж дослідити ці випадки кожний окремо, зробимо
ще одно зауваження. Через те, що $К$ є сума $c$ і $v$, сталого і змінного
капіталу, і через те що норма додаткової вартості, як
і норма зиску, звичайно виражається у процентах, то взагалі
зручно суму $c + v$ теж припустити рівною сотні, тобто $c$ і $v$
виражати в процентах. Для визначення, — правда, не маси, а
норми зиску, — однаково, чи ми скажемо: капітал у 15000,
з них 12000 сталого і 3000 змінного капіталу, виробляє додаткову
вартість у 3000, чи зведемо цей капітал до процентів:
\begin{align*}
15000 K &= 12000c + 3000 v (+ 3000 m) \\
100 K &= 80 c + 20 v (+ 20 m).
\end{align*}

В обох випадках норма додаткової вартості $m' = 100\%$, норма
зиску = 20\%.
Те саме, коли ми порівнюємо один з одним два капітали,
наприклад, з попереднім якийсь інший капітал:
\begin{align*}
12000 K &= 10800 c + 1200 v (+ 1200 m) \\
100 K &= 90 c + 10 v (+ 10 m),
\end{align*}


\input{_0072c.tex}

\index{ii}{0073}  %% посилання на сторінку оригінального видання
Процес циркуляції промислового капіталу, що становить лише частину
процесу його індивідуального кругобігу, визначається раніш розвиненими
загальними законами (книга I, розділ III), оскільки він являє лише ряд
актів у межах загальної товарової циркуляції. Та сама маса грошей, напр.,
500\pound{ ф. стерл.}, втягує по черзі в циркуляцію то більше промислових капіталів
(або також індивідуальних капіталів в їхній формі товарових капіталів), що
більша обігова швидкість грошей, що швидше, отже, кожний поодинокий
капітал перебігає ряд своїх товарових або грошових метаморфоз. Тому
та сама кількість капітальної вартости потребує для своєї циркуляції то
менше грошей, що більше гроші функціонують як засіб виплати, отже,
що більше, прим., при заміщенні товарового капіталу засобами його продукції,
доводиться оплачувати лише різність і що коротші строки виплати,
прим., при виплаті заробітної плати. З другого боку, коли швидкість
циркуляції та всі інші обставини дано як незмінні, то кількість грошей,
що мусить циркулювати як грошовий капітал, визначається сумою товарових
цін (ціна, помножена на кількість товарів), або, коли дано кількість
і вартість товарів, — вартісно самих грошей.

Але закони загальної товарової циркуляції мають силу лише остільки,
оскільки процес циркуляції капіталу утворює ряд актів простої циркуляції,
і не мають сили, коли ці акти становлять функціонально визначені
етапи в кругобігу індивідуальних промислових капіталів.

Щоб пояснити це, найкраще розглядати процес циркуляції в його
безперервному зв’язку, яким він з’являється в обох формах:
\[
\text{\phantom{I}II) } П\dots{} Т' 
\left\{\begin{array}{cc}
Т' & — \\
—  & Г'\\
т & —
\end{array}
\right.
\left\{\begin{array}{l}
Г — Т\splitfracm{Р}{Зп}\dots{} \samewidth{$П \dots{} Г'$}{$П (П')\hfill{}$} \\
~ \\
г — т
\end{array}
\right.
\]
\[
\text{III) } \samewidth{$П\dots{} Т'$}{\hfill{}$Т'$}
\left\{\begin{array}{cc}
Т' & — \\
—  & Г'\\
т & —
\end{array}
\right.
\left\{\begin{array}{l}
Г — Т\splitfracm{Р}{Зп}\dots{} П \dots{} Г' \\
~ \\
г — т
\end{array}
\right.
\]

Як ряд актів циркуляції взагалі, процес циркуляції (хоч є він $Т — Г — Т$,
хоч $Г — Т — Г$) являє лише два протилежні ряди товарових метаморфоз,
що з них кожна окрема метаморфоза знову таки має собі й протилежну
метаморфозу на боці чужого товару або чужих грошей, що протистоять
даному товарові.

Те, що з боку товаровласника $Т — Г$, з боку покупця є $Г — Т$; перша
метаморфоза товару в $Т — Г$ є друга метаморфоза товару, що виступає
як $Г$; в формулі $Г — Т$ справа стоїть навпаки. Отже, все, що сказано
про те, як метаморфоза товару на одній стадії переплітається з метаморфозою
товару на другій стадії, має силу для циркуляції капіталу,
оскільки капіталіст виконує функції покупця і продавця товарів, і
оскільки в наслідок цього його капітал функціонує як гроші проти
чужих товарів або як товар проти чужих грошей. Але це переплітання
метаморфоз не є разом з тим вираз для переплітання метаморфоз капіталів.

\parcont{}  %% абзац починається на попередній сторінці
\index{i}{0074}  %% посилання на сторінку оригінального видання
зовсім не потрібно, щоб одночасно підносились або падали ціни
всіх товарів. Досить підвищення цін певного числа головних
товарів в одному випадку або зниження їхніх цін у другому,
щоб підвищити або понизити належну до реалізації суму цін
усіх товарів, що циркулюють, отже, і щоб пустити в циркуляцію
більше або менше грошей. Чи зміна цін товарів одбиває дійсну
зміну вартостей чи просто коливання ринкових цін, вплив на
масу засобів циркуляції лишається той самий.

Припустімо, що дано якесь число продажів, або частинних
метаморфоз, не зв’язаних між собою, що відбуваються одночасно,
отже, просторово одна побіч однієї, приміром, 1 квартера пшениці,
20 метрів полотна, 1 біблії, 4 ґальонів горілки-житнівки. Коли
ціна кожного товару є 2\pound{ фунти стерлінґів}, отже, належна до реалізації
сума цін є 8\pound{ фунтів стерлінґів}, то в циркуляцію мусить
увійти маса грошей в 8\pound{ фунтів стерлінґів}. Навпаки, коли ці самі
товари є члени відомого нам уже ряду метаморфоз: 1 квартер
пшениці — 2\pound{ фунти стерлінґів} — 20 метрів полотна — 2\pound{ фунти
стерлінґів} — 1 біблія — 2\pound{ фунти стерлінґів} — 4 ґальони горілки —
2\pound{ фунти стерлінґів}, то 2\pound{ фунти стерлінґів} спричинюють послідовно
циркуляцію різних товарів, реалізуючи послідовно їхні
ціни, отже, і суму цін в 8\pound{ фунтів стерлінґів}, щоб спочити, кінець-кінцем,
у руках гуральника. Вони роблять чотири обіги. Ця
кількаразова зміна місць тієї самої монети репрезентує подвійну
зміну форми товару, його рух через дві протилежні стадії циркуляції
і сплетіння метаморфоз різних товарів\footnote{
«Лише продукти пускають їх (гроші) в рух і примушують їх
циркулювати\dots{} Швидкість їхнього (грошей) руху заступає їхню кількість.
Коли виникає потреба в них, вони тільки переходять із рук до рук, не
зупиняючись ані на хвилину». («Ce sont les productions qui (l’argent
mettent en mouvement et le font circuler\dots{} La célérité de son mouvement
(sc. de l’argent) supplée à sa quantité. Lorsqu’il en est besoin, il ne fait
que glisser d’une main dans l’autre sans s’arrêter un instant»). (\emph{Le~Trosne}:
«De l’Intérêt Social», Physiocrates, éd. Daire. Paris 1846, p. 915 sq.).
}. Протилежні й
що одна одну доповнюють фази, які перебігає цей процес, не можуть
відбуватися одна поруч однієї просторово, а наступають
одна по одній лише в часі. Тому переміжки часу становлять міру
тривання цього процесу, або швидкість обігу грошей вимірюється
числом обігів тієї самої монети за даний час. Нехай процес
циркуляції зазначених чотирьох товарів триває, приміром, один
день. Тоді сума цін, що має бути зреалізована, становитиме
8\pound{ фунтів стерлінґів}, число обігів тієї самої монети за день — 4
і кількість грошей, що циркулюють — 2\pound{ фунти стерлінґів}, або
для даного переміжку часу процесу циркуляції:
$\frac{\text{Сума цін товарів}}{\text{Число обігів однойменних монет}} \deq{}$
масі грошей, що функціонують
як засіб циркуляції. Цей закон має загальне значення.
Процес циркуляції якоїсь країни за якийсь даний відтинок часу
охоплює, правда, з одного боку, багато ізольованих, що відбуваються
одночасно й просторово один поруч одного, продажів
\parbreak{}  %% абзац продовжується на наступній сторінці


\index{iii2}{0075}  %% посилання на сторінку оригінального видання
\emph{Повосьме.} Відпливи металу, здебільша, є симптом зміни в стані закордонної
торговлі, а ця зміна, своєю чергою, є ознака того, що знову достигають
умови для кризи\footnote{
За Newmarch’oм відплив золота закордон може поставати з причин троякого роду, а саме:
1) від суто-комерційних причин, тобто тоді, коли довіз був більший, ніж вивіз, як то було між 1836
та 1844 роками та знову в 1847 роді, коли був головне значний довіз збіжжя; 2) від того, що треба
добувати засоби для приміщення англійського капіталу закордоном, як то було в 1857 роді, коли
будували
залізниці в Індії; та 3) від того, що остаточно витрачається кошти закордоном, як от в 1853
та 1854 роках на військові справи на Сході.
}.

\emph{Подев’яте.} Платіжний баланс може бути сприятливий для Азії й несприятливий
для Европи й Америки\footnote{
1918. Newmarch «Якщо ви візьмете Індію та Китай разом, якщо ви візьмете на увагу
обороти між Індією та Австралією та ще важливіші обороти між Китаєм та Сполученими Штатами — а в
цих випадках торговля є трибічна й вирівнюються рахунки за нашим посередництвом\dots{} тоді слушно,
що торговельний балянс був несприятливий не тільки для Англії, але й для Франції та Сполучених
Штатів». (В. А 1857).
}.

\pfbreak

Довіз благородного металу відбувається переважно при двох моментах.
З одного боку, за тієї першої фази низького рівня проценту, що настає по
кризі та є вияв обмеження продукції; а потім за другої фази, коли рівень проценту
підноситься, але ще не досягнув своєї середньої висоти. Це — фаза, коли
зворотні припливи капіталів відбуваються легко, комерційний кредит великий,
а тому й попит на позичковий капітал зростає непропорційно поширові продукції.
В обох фазах, коли позичкового капіталу є порівняно багато, надмірний
приплив капіталу, що існує в формі золота та срібла, отже, в такій формі,
в якій він насамперед може функціонувати лише як позичковий капітал, — мусить
значно виливати на рівень проценту, а тому й на загальний розвиток справ.

З другого боку: відплив, невпинний значний вивіз благородного металу
настає тоді, коли вже постають труднощі щодо зворотного припливу капіталів,
коли ринки переповнені, а подоба розцвіту зберігається тільки за допомогою
кредиту; отже, коли вже є дуже збільшений попит на позичковий капітал, а тому
й рівень проценту вже досяг, принаймні, своєї середньої величини. Серед цих
обставин, що відбиваються у відпливі саме благородного металу, значно більшає
вплив невпинного витягування капіталу в такій формі, в якій він існує безпосередньо
як грошовий позичковий капітал. Це мусить безпосередньо впливати
на рівень проценту. Але замість обмежувати кредитові операції, це піднесення
рівня проценту поширює їх та приводить до надмірного напруження всіх їхніх
допоміжних засобів. Тому цей період передує крахові.

Newmarch’а питають (В. А. 1857): «1520. Отже, число векселів у циркуляції
зростає разом з рівнем проценту? — Здається, так. — 1522. За спокійних
звичайних часів головна книга є дійсне знаряддя обміну; але коли постають
труднощі, коли, напр., серед таких обставин, що мною вище були наведені, підвищується
норма банкового дисконту\dots{} тоді операції сходять цілком сами собою
до виписування векселів; ці векселі не тільки придатніші до того, щоб бути за
законний доказ зробленої справи, але вони й зручніші для дальших закупів
і насамперед їх можна уживати, як засіб кредиту, щоб визичати капітал», —
До цього долучається те, що, коли банк серед до певної міри загрозливих обставин
підвищує свою норму дисконту, а в наслідок цього одночасно стає ймовірним,
що банк обмежить реченець векселів, котрі йому доводиться дисконтувати,
— постає загальне побоювання, що це буде розвиватися crescendo\footnote{
Crescendo. від лат. cresco, — росту, чим раз більшаючи, зростаючи. \emph{Пр. Ред.}
}. Отже,
\parbreak{}  %% абзац продовжується на наступній сторінці

\parcont{}  %% абзац починається на попередній сторінці
\index{ii}{0076}  %% посилання на сторінку оригінального видання
праця стає найманою працею; тому капіталістична продукція (а значить,
і товарова продукція) з’являється в цілому своєму об’ємі лише тоді,
коли й безпосередній сільський продуцент є найманий робітник. В відношенні
між капіталістом і найманим робітником грошове відношення, відношення покупця і продавця, стає
відношенням іманентним самій продукції. Але це відношення в основі своїй ґрунтується на суспільному
характері продукції, а не способу обміну; цей останній, навпаки, випливає з першого.
А проте, буржуазному світоглядові, де всю увагу звертається на практичні
операції, саме й відповідає погляд, що не в характері способу продукції
треба вбачати основу відповідного йому способу обміну, а навпаки\footnote{
До цього місця, рукопис V, — Все, що далі до кінця розділу, це замітка, яка є в зшитку з 1877 або
1878 року серед витягів з книжок.
}.

\pfbreak{}

\label{original-76}
Капіталіст кидає в циркуляцію менше вартости в грошовій формі,
ніж бере з неї, бо він кидає в неї більше вартости в товаровій формі,
ніж узяв звідти в товаровій формі. Оскільки він функціонує лише як персоніфікація
капіталу, як промисловий капіталіст, остільки його подання
товарових вартостей завжди більше, ніж його попит на товарові вартості.
Коли б його подання й попит взаємно покривались, то з цього погляду
це значило б, що його капітал не зростає вартістю; капітал не функціонував
би як продуктивний капітал; продуктивний капітал перетворився б
на товаровий капітал, не запліднений додатковою вартістю; підчас продукційного
процесу він не видобував би з робочої сили жодної додаткової
вартости в товаровій формі, отже, зовсім не функціонував би як
капітал; капіталіст дійсно мусить „продавати дорожче, ніж купив“, але
це вдається йому лише тому, що він за допомогою капіталістичного
продукційкого процесу перетворив куплений ним дешевший товар, — бо він
є товар меншої вартости, — на товар більшої вартости, тобто на дорожчий.
Він продає дорожче не тому, що продає свій товар вище понад його
вартість, а тому, що продає товар, який має вартість вищу, ніж сума
вартостей складових елементів його продукції.

Норма, що за нею капіталіст збільшує вартість свого капіталу, то більша,
що більша різність між його поданням і попитом, тобто що більший
надлишок тієї товарової вартости, яку він подає, проти тієї товарової
вартости, на яку він ставить попит. Його мета не та, щоб попит і
подання навзаєм покривались, а щоб вони якомога більше не покривались,
щоб його подання перекривало його попит.

Те, що має силу для поодинокого капіталіста, має силу й для кляси
капіталістів.

Оскільки капіталіст є лише персоніфікація промислового капіталу,
остільки його власний попит є лише попит на засоби продукції та
робочу силу. Його попит на $Зп$, розглядуваний щодо вартости,
менший, ніж його авансований капітал; він купує засоби продукції
за меншу вартість, ніж вартість його капіталу, а тому й за
\parbreak{}  %% абзац продовжується на наступній сторінці

\parcont{}  %% абзац починається на попередній сторінці
\index{ii}{0077}  %% посилання на сторінку оригінального видання
значно меншу вартість, ніж вартість того товарового капіталу, що
його він подає.

Щодо його попиту на робочу силу, то він, щодо вартости, визначається
відношенням його змінного капіталу до його цілого капіталу,
тобто дорівнює v: С, і тому зростає в капіталістичній продукції
відносно менше, ніж його попит на засоби продукції. Капіталіст
у дедалі більшій мірі більше купує Зп, ніж купує Р.

Що робітник перетворює свою заробітну плату переважно на засоби
існування, а найбільшу частину її — на доконечні засоби існування, то попит
капіталіста на робочу силу посередньо є разом з тим попит на засоби споживання,
що ввіходять у споживання робітничої кляси. Але цей попит
дорівнює v, і він на жоден атом не є більший від v (коли робітник зоощаджує
з своєї заробітної плати — кредитові відносини ми повинні лишити
тут осторонь — то це значить, що він частину своєї заробітної плати
перетворює на скарб і pro tanto\footnote*{
Остільки, відповідно до цього. \emph{Ред.}
} вже не виступає як особа, що ставить
попит, не як покупець). Максимальна межа попиту капіталіста дорівнює
C \deq{} $c \dplus{} v$, а його подання дорівнює $c \dplus{} v \dplus{} m$; отже, коли будова
його товарового капіталу є $80c \dplus{} 20v \dplus{} 20m$, то попит його дорівнює
80с \dplus{} 20v, отже, розглядуваний щодо його вартости, він на \sfrac{1}{5} менший
від його подання. Що більше відсоткове відношення спродукованої від нього
маси m (норма зиску),\footnote*{Норма зиску це є відношення маси додаткової вартости до цілого авансованого
капіталу. Про це дивись: Маркс, „Капітал“, т. III, ч. І, розд. II. —
\emph{Ред.}}
) то менший стає його попит порівняно з його
поданням. Хоч попит капіталіста на робочу силу, а тому, посередньо, і
на доконечні засоби існування, з розвитком продукції дедалі меншає порівняно
з його попитом на засоби продукції, все ж, з другого боку, не
треба забувати, що його попит на Зп завжди менший, ніж його капітал,
обчислюючи з дня на день. Отже, попит його на засоби продукції
мусить завжди бути меншої вартости, ніж товаровий продукт капіталіста,
що постачає йому ці засоби продукції та працює з однаковим
капіталом і за однакових інших обставин. Та обставина, що є
багато капіталістів, а не один, справи аж ніяк не змінює. Припустімо,
що його капітал дорівнює 1000\pound{ ф. стерл.}, стала частина його дорівнює
800\pound{ ф. стерл}. Тоді попит його до всіх капіталістів дорівнює 800\pound{ ф. стерл.},
а всі вони разом на кожні 1000\pound{ ф. стерл.} (хоч скільки з цієї суми припадає
на кожного з них зокрема і хоч яку частину цілого капіталу його
становить сума, яка припадає кожному) постачають, за однакової норми
зиску, засобів продукції вартістю в 1200\pound{ ф. стерл.}; отже, його попит покриває
лише \sfrac{2}{3} їхнього подання, тимчасом як увесь його власний попит,
розглядуваний щодо величини його вартости, дорівнює лише \sfrac{4}{5}
його власного подання.

Тепер ми ще мусимо, забігаючи наперед, розглянути, між іншим, оборот
капіталу\footnote*{Про поняття „оборот капіталу“ див. далі розділ VII. — \emph{Ред.}}. Припустімо, що ввесь капітал даного капіталіста дорівнює
\parbreak{}  %% абзац продовжується на наступній сторінці


\index{iii1}{0078}  %% посилання на сторінку оригінального видання
b) Норма зиску лишається незмінною тільки в тому разі, коли
$е = Е$, тобто коли дріб $\frac{v}{К}$ при позірній зміні зберігає те саме
значення, тобто якщо чисельник і знаменник помножуються або
діляться на те саме число. $80 с + 20 v + 20 m$ і $160 с + 40 v + 40 m$,
очевидно, мають однакову норму зиску в 20\%, бо ті лишається
= 100\%, а $\frac{v}{К} = \frac{20}{100} = \frac{40}{200}$ в обох прикладах
представляє ту саму величину.

c) Норма зиску підвищується, якщо $е$ більше за $Е$, тобто
якщо змінний капітал зростає в більшій пропорції, ніж весь капітал.
Якщо $80с + 20v + 20m$ стає $120 с + 40 v + 40 m$, то норма
зиску підвищується від 20\% до 25\%, бо, при незмінному $m'$,
$\frac{v}{К} = \frac{20}{100}$ підвищилось до \frac{40}{160}, з \sfrac{1}{5}
до \sfrac{1}{4}.

При зміні $v$ і $К$ в одному напрямі ми можемо цю зміну величин
розглядати так, ніби обидві величини змінюються до певної
межі в однаковій пропорції, так що до цієї межі $\frac{v}{К}$ лишається
незмінним. Поза цією межею стала б змінюватись тільки
одна з двох величин, і ми таким чином звели б цей складніший
випадок до одного з попередніх простіших.

Якщо, наприклад, $80 с + 20 v + 20 m$ переходить у
$100 c + 30 v + 30 m$, то при цій зміні до $100с + 25v + 25m$ відношення
$v$ до $с$, отже, і до $К$, лишається незмінним. Отже, до цього
пункту і норма зиску лишається незачепленою. Тому ми можемо
взяти тепер за вихідний пункт $100 с + 25v + 25m$; ми бачимо,
що v підвищилось на 5, до $30v$, а в наслідок цього $К$ підвищилось
від 125 до 130, і маємо таким чином перед собою другий
випадок, випадок простої зміни $v$ та спричиненої цим
зміни $К$. Норма зиску, яка спочатку була 20\%, в наслідок такої
додачі в $5v$, при попередній нормі додаткової вартості, підвищується
до $23\sfrac{1}{13}\%$.

Таке саме зведення до простішого випадку може мати місце
навіть тоді, коли $v$ і $К$ змінюють свою величину в протилежному
напрямі. Коли б ми знову виходили, наприклад, з
$80с + 20v + 20m$ і від цього перейшли б до форми: $110с + 10v + 10m$, то
при зміні до $40с + 10v + 10m$ норма зиску лишилася б така
сама, як і спочатку, а саме 20\%. В наслідок додачі $70с$ до цієї
перехідної форми норма зиску знижується до $8\sfrac{1}{3}\%$. Отже, цей
випадок ми знову звели до випадку зміни однієї тільки змінної,
а саме $с$.

Отже, одночасна зміна $v$, $с$ і $К$ не дає нових точок зору і в
кінцевому рахунку завжди веде назад до випадку, коли змінюється
тільки один фактор.

Навіть єдиний випадок, який ще лишається, уже фактично
вичерпаний, а саме той випадок, коли $v$ і $К$ чисельно зберігають
\parbreak{}  %% абзац продовжується на наступній сторінці

\input{_0079.tex}

\index{iii1}{0080}  %% посилання на сторінку оригінального видання
\begin{center}
  \textbf{II. $m'$ змінюється}
\end{center}

Загальну формулу норм зиску при різних нормах додаткової
вартості, однаково, чи  \frac{v}{K} лишається незмінним, чи теж змінюється,
ми одержимо, коли рівняння:\[p' = m' \frac{v}{К}\]

перетворимо в інше:
\[p'\textsubscript{1} = m'\textsubscript{1} \frac{v\textsubscript{1}}{K\textsubscript{1}},\]

де $р'\textsubscript{1}, m'\textsubscript{1}, v\textsubscript{1}, і К\textsubscript{1}$ означають змінені величини $р', m', v$ і $К$.
Тоді ми маємо: \[p': p'\textsubscript{1} = m' \frac{v}{K}: m'\textsubscript{1} \frac{v\textsubscript{1}}{K\textsubscript{1}},\]

і звідси:\[p'\textsubscript{1} = \frac{m'\textsubscript{1}}{m'} × \frac{v\textsubscript{1}}{v} × \frac{K}{K\textsubscript{1}} × p'\].

\begin{center}
\textbf{1. $m'$ змінюється, $\frac{v}{K}$ не змінюється}
\end{center}

В цьому випадку ми маємо рівняння:

\[p' = m'\frac{v}{K}; p'\textsubscript{1} = m'\textsubscript{1} \frac{v}{K},\]

в обох рівняннях \frac{v}{K} має однакову величину. Тому одержуємо
відношення:

\[р': р'\textsubscript{1} = m': m'\textsubscript{1}.\]

Норми зиску двох капіталів однакового складу відносяться
одна до одної, як відповідні норми додаткової вартості. Через
те що в дробу $\frac{v}{K}$ важливі не абсолютні величини $v$ і $К$, а тільки
відношення між ними, то це стосується й до всіх капіталів однакового
складу, яка б не була їх абсолютна величина.

\begin{center}
$80с + 20v + 20m; K = 100, m' = 100\%, p' = 20\%$

$160c + 40v + 20m; K = 200, m' = 50\%, p' = 10\%$

 $100\%: 50\% = 20\%: 10\%.$
\end{center}

Якщо абсолютні величини $v$ і $К$ в обох випадках однакові,
то норми зиску відносяться одна до одної, крім того, як маси
додаткової вартості:

$p': p'\textsubscript{1} = m'v: m'1v = m: m\textsubscript{1}.$


\index{iii1}{0081}  %% посилання на сторінку оригінального видання
Наприклад:

80 с + 20 v + 20 m; m' = 100\%, p' = 20\%
80 с + 20 v + 10 m; m' = 50\%, p' = 10\%
20\% : 10\% = 100 × 20 : 50 × 20 = 20 m : 10 m.

Тепер ясно, що при капіталах однакового абсолютного чи
процентного складу норми додаткової вартості можуть бути
різні тільки в тому випадку, коли різні або заробітна плата,
або довжина робочого дня, або інтенсивність праці. В трьох
випадках:

I.  80 с + 20 v + 10 m; m' = 50\%, p' = 10\%,
II. 80 с + 20 v + 20 m; m' = 100\%, p' = 20\%,
III. 80 с + 20 v + 40 m; m' = 200\%, p' = 40\%,

вся нововироблена вартість буде в І 30 (20 v + 10 m), в II — 40,
в III — 60. Це може статись трояким способом.

Поперше, якщо заробітні плати різні, отже, якщо 20 v в кожному
окремому випадку виражає різне число робітників. Припустім,
що в І занято 15 робітників 10 годин при заробітній
платі в 1  1/3  фунтів стерлінгів і що вони виробляють вартість
у 30 фунтів стерлінгів, з яких 20 фунтів стерлінгів заміщають
заробітну плату, а 10 фунтів стерлінгів лишаються для додаткової
вартості. Якщо заробітна плата падає до 1 фунта стерлінгів,
то можуть бути заняті 20 робітників 10 годин; тоді вони
виробляють вартість у 40 фунтів стерлінгів, з яких 20 фунтів
стерлінгів для заробітної плати і 20 фунтів стерлінгів додаткової
вартості. Якщо заробітна плата падає ще далі, до 2/3 фунтів
стерлінгів, то можуть бути заняті 30 робітників по 10 годин,
які виробляють вартість у 60 фунтів стерлінгів, що з них після
відрахування 20 фунтів стерлінгів для заробітної плати залишиться
ще 40 фунтів стерлінгів для додаткової вартості.

Цей випадок: незмінний процентний склад капіталу, незмінний
робочий день, незмінна інтенсивність праці, зміна норми
додаткової вартості, спричинена зміною заробітної плати — є
єдиний випадок, на якому справджується положення Рікардо:
„profits would be high or low, exactly in proportion as wages
would be, low or high“ [„зиск буде високий чи низький точно
в такій пропорції, в якій заробітна плата буде низька чи висока“]
(„Principles of Political Economy“, розд. І, відділ III, стор. 18.
„Works of D. Ricardo“, вид. Mac Culloch, 1852).

Або, подруге, якщо інтенсивність праці різна. Тоді, наприклад,
20 робітників при однакових засобах праці за 10 робочих
годин на день виробляють у І — 30, у II — 40, у III — 60 штук
певного товару, кожна штука якого, крім вартості спожитих
на неї засобів виробництва, представляє нову вартість в 1 фунт
стерлінгів. Через те що в кожному випадку 20 штук, = 20
фунтам стерлінгів, заміщають заробітну плату, то для додаткової
\index{iii1}{0082}  %% посилання на сторінку оригінального видання
вартості лишаються в І — 10 штук = 10 фунтам стерлінгів,
в II — 20 штук = 20 фунтам стерлінгів, в III — 40 штук = 40 фунтам
стерлінгів.

Або, потретє, робочий день — різної довжини. Якщо 20 робітників
при однаковій інтенсивності працюють у І — дев’ять,
у II — дванадцять, у III — вісімнадцять годин на день, то весь їх
продукт 30 : 40 : 60 відноситься як 9 : 12 : 18, і тому що заробітна
плата в кожному випадку = 20, то знову лишається 10, відповідно
20 і 40 для додаткової вартості.

Отже, підвищення або зниження заробітної плати діє в зворотному
напрямі, підвищення або зниження інтенсивності праці
і здовження або скорочення робочого дня діє в тому самому
напрямі на висоту норми додаткової вартості, а тому, при незмінному
v/K, і на норму зиску.

2. m' і v змінюються, К не змінюється

В цьому випадку має силу пропорція:

p': p'1 = m' v/K : m'1 v1/K = m'v : m'1v1 = m : m1.

Норми зиску відносяться одна до одної, як відповідні маси
додаткової вартості.

Зміна норми додаткової вартості при незмінній величині змінного
капіталу означала зміну у величині й розподілі нововиробленої
вартості. Одночасна зміна v і m' так само завжди включає
інший розподіл, але не завжди зміну величини нововиробленої
вартості. Можливі три випадки:

a) Зміни v і m' відбуваються в протилежному напрямі, але
на однакову величину; наприклад:

80 с + 20 v + 10 m; m' = 50\%, p' = 10\%
90 с + 10 v + 20 m; m' = 200\%, p' = 20\%.

Нововироблена вартість в обох випадках однакова, отже, однакова
й кількість витраченої праці; 20 v + 10 m = 10 v + 20 m = 30.
Ріжниця тільки в тому, що в першому випадку 20 сплачується
як заробітна плата, а 10 лишається для додаткової вартості,
тимчасом як у другому випадку заробітна плата становить
тільки 10, а тому додаткова вартість — 20. Це єдиний випадок,
коли при одночасній зміні v і m' число робітників, інтенсивність
праці і довжина робочого дня лишаються незміненими.

b) Зміни m' і v відбуваються так само в протилежному напрямі,
але не на ту саму величину. Тоді перевага буде або на
стороні зміни v, або на стороні зміни m'.

I. 80 с + 20 v + 20 m; m' = 100\%, p' = 20\%
II.72 с + 28 v + 20 m; m' = 71 3/7\%, p' = 20\%
III. 84 с + 16 v + 20 m; m' = 125\%, p' = 20\%.


\index{iii1}{0083}  %% посилання на сторінку оригінального видання
В І за нововироблену вартість у 40 сплачується $20 v$, в II
за нововироблену вартість у 48 сплачується $28 v$, в III за нововироблену
вартість у 36 сплачується $16 v$. Як нововироблена
вартість, так і заробітна плата змінились; але зміна нововиробленої
вартості означає зміну кількості витраченої праці, отже,
або числа робітників, або тривалості праці, або інтенсивності
праці, або кількох з цих трьох факторів.

с) Зміни $m'$ і $v$ відбуваються в тому самому напрямі; тоді
одна підсилює вплив другої.

\begin{gather*}
90 с + 10 v + 10 m; m' = 100\%, p' = 10\% \\
80 с + 20 v + 30 m; m' = 150\%, p' = 30\% \\
92 с + \phantom{$0$}8 v + \phantom{$0$}6 m; m' = \phantom{$1$}75\%, p' = \hphantom{$1$}6\%
\end{gather*}

\noindent{}І тут усі три нововироблені вартості різні, а саме 20, 50 і 14;
і ця ріжниця в величині витрачуваної в кожному випадку кількості
праці знову зводиться до ріжниці числа робітників, тривалості
праці, інтенсивності праці, або кількох, а то й усіх цих факторів.

\subsection[m', v і К змінюються]{m', v і К змінюються\footnotemarkZ{}}
\footnotetextZ{
В першому німецькому виданні: 3. Примітка ред. нім. вид. ІМЕЛ.
}

\noindent{}Цей випадок не дає нових точок зору і розв’язується за допомогою
загальної формули, даної в рубриці: II. $m'$ змінюється.

\pfbreak{}

Отже, вплив зміни величини норми додаткової вартості на
норму зиску дає такі випадки:

1. $р'$ збільшується або зменшується в тій самій пропорції, як
i $m'$, якщо $\frac{v}{K}$  лишається незмінним.

\begin{gather*}
80 с + 20 v + 20 m; m' = 100\%, p' = 20\% \\
80 с + 20 v + 10 m; m' = \phantom{1}50\%, p' = 10\% \\
100\% : 50\% = 20\% : 10\%.
\end{gather*}

2. $р'$ підвищується або падає в більшій пропорції, ніж $m'$,
якщо $\frac{v}{K}$ рухається в тому самому напрямі, що й $m'$, тобто
збільшується чи зменшується, коли збільшується чи зменшується
$m'$.

\begin{gather*}
80 с + 20 v + 10 m; m' = 50\phantom{\sfrac{2}{3}}\%, p' = 10\% \\
70 с + 30 v + 20 m; m' = 66\sfrac{2}{3}\%, p' = 20\% \\
50\% : 66 2/3\% < 10\% : 20\%.\text{\footnotemarkZ{}}
\end{gather*}

\footnotetextZ{
Знак $<$ означає тут, що збільшення з 50 до 66\sfrac{2}{3} є порівняно менше, ніж
збільшення з 10 до 20. Знак $>$ у дальшій формулі означає зворотне. Примітка
ред. нім. вид. ІМЕЛ.
}



\index{iii1}{0084}  %% посилання на сторінку оригінального видання
3. р' підвищується або падає в меншій пропорції, ніж m',
якщо v/K змінюється в напрямі, протилежному до зміни m', але
в меншій пропорції:\footnote{
с + 20 v + 20 m; m' = 100\%, p' = 20\%
90 с + 10 v + 15 m; m' = 150\%, p' = 15\%

m' підвищилось з 100\% до 150\%, р' зменшилось від 20\% до 15\%.
} с + 20 v + 10 m; m' = 50\%, p' = 10\%
90 с + 10 v + 15 m; m' = 150\%, p' = 15\%
50\% : 150\% > 10\% : 15\%.

4. р' підвищується, хоч m' падає, або падає, хоч m' підвищується,
якщо v/K змінюється в напрямі, протилежному до зміни
m', і в більшій пропорції, ніж m'.

5. Нарешті, р' лишається незмінним, хоч m' підвищується або
падає, якщо v/K змінює свою величину в протилежному напрямі,
але точно в тій самій пропорції, що й m'.

Тільки цей останній випадок потребує ще деякого пояснення.
Як ми бачили вище при змінах v/K, що одна й та сама норма
додаткової вартості може виражатися в найрізніших нормах
зиску, так ми бачимо тут, що в основі однієї і тієї самої норми
зиску можуть лежати дуже різні норми додаткової вартості.
Але в той час, як при незмінному m' першої-ліпшої зміни у відношенні
v до К досить було для того, щоб викликати відмінність
в нормі зиску, — при зміні величини m' мусить настати точно
відповідна зворотна зміна величини v/K для того, щоб норма
зиску лишилась та сама. Для одного й того ж капіталу або для
двох капіталів у тій самій країні це можливе тільки в дуже
виняткових випадках. Візьмімо, наприклад, капітал\footnote{
с + 16 v + 24 m; K = 96, m' = 150\%, p' = 25\%.

Отже, для того, щоб р' було, як і раніш, = 20\%, весь капітал
мусив би зрости до 120, отже, сталий — до 104:
} с + 20 v + 20 m; K = 100, m' = 100\%, p' = 20\%

і припустімо, що заробітна плата впала настільки, що тепер за
16 v можна було б мати те саме число робітників, як раніш за
20 v. Тоді ми, при інших незмінних умовах і звільненні 4 v,
маємо

104 с + 16 v + 24 m; K = 120, m' = 150\%, p' = 20\%


\index{iii1}{0085}  %% посилання на сторінку оригінального видання
Це було б можливе тільки тоді, коли б одночасно з зниженням
заробітної плати настала зміна в продуктивності праці, яка
вимагає цього зміненого складу капіталу; абож коли б грошова вартість
сталого капіталу підвищилась з 80 до 104; коротко кажучи —
такий випадковий збіг обставин, який буває тільки в виняткових
випадках. В дійсності така зміна $m'$, що не зумовлює одночасно
зміни $v$, а тому й зміни $\frac{v}{K}$, мислима тільки при цілком певних
обставинах, а саме в таких галузях промисловості, де застосовується
тільки основний капітал і праця, а предмет праці дається
природою.

Але при порівнянні норм зиску двох країн справа стоїть
інакше. Тут одна й та сама норма зиску в дійсності виражає
здебільшого різні норми додаткової вартості.

Отже, з усіх п’яти випадків випливає, що ростуща норма
зиску може відповідати падаючій або ростущій нормі додаткової
вартості, падаюча норма зиску — ростущій або падаючій
нормі додаткової вартості, незмінна норма зиску — ростущій або
падаючій нормі додаткової вартості. Що ростуща, падаюча або
незмінна норма зиску може також відповідати незмінній нормі
додаткової вартості, це ми бачили під рубрикою І.

\pfbreak

Отже, норма зиску визначається двома головними факторами:
нормою додаткової вартості і вартісним складом капіталу.
Вплив обох цих факторів можна коротко резюмувати, — при чому
склад ми можемо виразити в процентах, бо тут не має значення,
з якої з двох частин капіталу походить зміна, — таким чином:

Норми зиску двох капіталів або одного й того ж капіталу
в двох послідовних різних його станах

\emph{є рівні:}

1) при однаковому процентному складі капіталів і однаковій
нормі додаткової вартості;

2) при неоднаковому процентному складі і неоднаковій нормі
додаткової вартості, якщо добутки з норм додаткової вартості
і взятих у процентах змінних частин капіталу ($m'$ на $v$) є рівні,
тобто якщо \emph{маси} додаткової вартості ($m = m'v$), взяті в процентному
відношенні до всього капіталу, є рівні; інакше кажучи,
якщо в обох випадках множники $m'$ і $v$ стоять у зворотному
відношенні один до одного.

\emph{Вони нерівні:}

1) при однаковому процентному складі, якщо норми додаткової
вартості нерівні, при чому норми зиску відносяться одна до
одної, як норми додаткової вартості;
\parbreak{}  %% абзац продовжується на наступній сторінці

\parcont{}  %% абзац починається на попередній сторінці
\index{iii1}{0086}  %% посилання на сторінку оригінального видання
2) при однаковій нормі додаткової вартості і неоднаковому
процентному складі, при чому норми зиску відносяться одна до
одної, як змінні частини капіталу;

3) при неоднаковій нормі додаткової вартості і неоднаковому
процентному складі, при чому норми зиску відносяться одна до
одної, як добутки $m'v$, тобто як маси додаткової вартості,
взяті в процентному відношенні до всього капіталу.\footnote{
В рукопису є ще дуже докладні обчислення щодо різності між нормою
додаткової вартості і нормою зиску $(m' — р')$; ця різність має різні цікаві особливості,
рух її показує випадки, коли обидві норми віддаляються одна від
одної або наближаються одна до одної. Ці рухи можуть бути зображені у формі
кривих. Я відмовляюсь від відтворення цього матеріалу, бо для ближчих цілей
цієї книги він менш важливий; тут досить просто звернути на це увагу тих
читачів, які захочуть далі простежити це питання. — Ф. Е.
}

\section{Вплив обороту на норму зиску}

[Вплив обороту на виробництво додаткової вартості, отже
й зиску, з’ясовано в другій книзі. Його можна коротко зрезюмувати
таким чином, що в наслідок того, що на оборот потрібен
певний час, на виробництво не може бути застосований одночасно
весь капітал; що, отже, частина капіталу постійно лежить без
діла, чи то в формі грошового капіталу, запасних сировинних
матеріалів, готового, але ще не проданого товарного капіталу,
чи в формі боргових вимог, для яких ще не настав строк платежу;
що капітал, який діє в активному виробництві, тобто при
створенні і привласненні додаткової вартості, постійно зменшується
на цю частину, при чому в такій самій пропорції постійно
зменшується створювана і привласнювана додаткова вартість.
Чим коротший час обороту, тим меншою порівняно з усім
капіталом стає ця частина капіталу, яка лежить без діла; і тим
більшою, отже, стає, при інших незмінних умовах, привласнювана
додаткова вартість.

Уже в другій книзі докладно розвинуто, як скорочення часу
обороту або одного з двох його підрозділів, часу виробництва
і часу циркуляції, підвищує масу вироблюваної додаткової вартості.
Але через те що норма зиску виражає тільки відношення
виробленої маси додаткової вартості до всього капіталу, занятого
в її виробництві, то очевидно, що всяке таке скорочення підвищує
норму зиску. Те, що раніше, в другому відділі другої
книги, розвинуто щодо додаткової вартості, в такій самій мірі
стосується і до зиску та норми зиску і не потребує тут повторення.
Ми хочемо відзначити лиш декілька головних моментів.

Головний засіб скорочення часу виробництва є підвищення
продуктивності праці, що звичайно називають прогресом промисловості.
Якщо цим одночасно не викликається значне збільшення
\index{iii1}{0087}  %% посилання на сторінку оригінального видання
загальних витрат капіталу в наслідок застосування дорогих
машин і т. д., а тому й зниження норми зиску, обчислюваної
на весь капітал, то ця остання мусить підвищитись. І це, безперечно,
має місце при багатьох з найновіших успіхів металургії
і хемічної промисловості. Нововідкриті способи виготовлення
заліза й сталі — Бессемера, Сіменса, Гількріста-Томаса та інших —
скорочують до мінімуму, при відносно незначних витратах,
надзвичайно довгочасні раніш процеси. Виготовлення алізарину
або красильної речовини крапу з кам’яновугільного дьогтю дає
за кілька тижнів, і до того ж при фабричних приладдях, які
вже раніш уживалися для виготовлення фарб з кам’яновугільного
дьогтю, такий самий результат, який раніше вимагав цілих
років; один рік був потрібний для росту крапу, а потім ще
кілька років коріння лишали достигати, раніше ніж уживати
його для фарбування.

Головний засіб скорочення часу циркуляції є поліпшені
шляхи сполучення. І в цьому останні п’ятдесят років зробили
революцію, яку можна порівняти тільки з промисловою революцією
останньої половини минулого століття. На суходолі макадамізовані\footnote*{
Макадамізування — спосіб брукування шляхів за системою Мак-Адама,
при якому скальне каміння укочується круглими котками. Ред. укр. перекладу,
} шляхи відтиснені на задній план залізницею, на
морі повільне і нерегулярне вітрильне сполучення — швидким
і регулярним пароплавним сполученням, і вся земна куля обвивається
телеграфними дротами. Власне кажучи, тільки Суецький
канал і відкрив Східну Азію і Австралію для пароплавного сполучення.
Час циркуляції для товарів, що посилалися до Східної
Азії, який ще в 1847 році становив щонайменше дванадцять
місяців, тепер можна звести майже
до стількох же тижнів. Два великі огнища криз 1825—1857 рр.,
Америка і Індія, в наслідок цього перевороту в засобах сполучення
наблизились до європейських промислових країн на
70—90\% і тим самим утратили більшу частину своєї здатності
до вибухів. Час обороту всієї світової торгівлі скоротився
в такій самій мірі, а дієздатність капіталу, який бере в ній
участь, підвищилась більше, ніж удвоє чи утроє. Що це не
лишилось без впливу на норму зиску, зрозуміло само собою.

Щоб представити в чистому вигляді вплив обороту всього
капіталу на норму, зиску ми мусимо при порівнянні двох
капіталів припустити, що всі інші обставини однакові. Отже,
крім норми додаткової вартості і робочого дня, нехай буде
однаковий і процентний склад капіталів. Візьмім тепер капітал
$А$ з складом $80с + 20v = 100К$, що обертається двічі на рік
при нормі додаткової вартості в 100\%. Тоді річний продукт
буде:

$160 с + 40 v + 40 m$. Але для визначення норми зиску ми обчисляємо
ці $40 m$ не на капітальну вартість у 200, що обернулась,
\index{iii1}{0088}  %% посилання на сторінку оригінального видання
а на авансовану капітальну вартість у 100, і таким чином
одержуємо: $р' = 40\%$.

Порівняймо з цим капітал $В = 160с + 40v=200K$, який
функціонує при такій самій нормі додаткової вартості в 100\%,
але обертається тільки один раз на рік. Тоді річний продукт
буде, як і вище:

$160с + 40 v + 40 m$. Але на цей раз ці $40 m$ слід обчислити на
авансований капітал у 200; це дає для норми зиску тільки 20\%,
отже, тільки половину норми для $А$.

Звідси випливає: при капіталах однакового процентного
складу, при однаковій нормі додаткової вартості і однаковому
робочому дні, норми зиску двох капіталів стоять у зворотному
відношенні до часу їх оборотів. Якщо в двох порівнюваних випадках
неоднаковий склад, або норма додаткової вартості, або
робочий день, або заробітна плата, то цим, звичайно, будуть
породжені й дальші ріжниці в нормі зиску; але вони незалежні
від обороту і тому нас не цікавлять тут; вони вже розглянуті
в розділі III.

Безпосередній вплив скороченого часу обороту на виробництво
додаткової вартості, отже й зиску, полягає в підвищеній
діяльності, яка таким способом надається змінній частині капіталу,
про що див. книгу II, розділ XVI: Оборот змінного
капіталу. Там виявилось, що змінний капітал у 500, який обертається
десять разів на рік, привласнює за цей час стільки ж додаткової
вартості, як і змінний капітал у 5000, який при однаковій
нормі вартості і однаковій заробітній платі обертається
тільки один раз на рік.

Візьмемо капітал І, що складається з 10 000 основного капіталу,
— річне зношування якого становить $10\% = 1000,$ — 500
обігового сталого і 500 змінного капіталу. При нормі додаткової
вартості в 100\% цей змінний капітал обертається десять
разів на рік. Задля спрощення ми припускаємо в усіх дальших
прикладах, що обіговий сталий капітал обертається за той
самий час, як і змінний, що й на практиці здебільшого приблизно
так і буває. Тоді продукт одного такого періоду обороту буде:\[
100 с \text{(зношування)} + 500 с + 500 v + 500 m = 1600,\]
а продукт цілого року з десятьма такими оборотами:
\begin{center}
$1000 с \text{(зношування)} + 5000 с + 5000 v + 5000 m = 16 000$,
 $К = 11 000; m = 5000, р' = \frac{5000}{11000} + 45 \sfrac{5}{11}\%$.
\end{center}

Візьмемо тепер капітал II: основний капітал — 9000, його річне
зношування — 1000, обіговий сталий капітал — 1000, змінний
капітал — 1000, норма додаткової вартості — 100\%, число річних
оборотів змінного капіталу — 5. Отже, продукт кожного періоду
обороту змінного капіталу буде:\[
200c \text{(зношування)} + 1000 c + 1000 v + 1000 m = 3200,\]

а весь річний продукт при п’яти оборотах:

\begin{center}
$1000с\text{(зношування)} + 5000 с + 5000 v + 5000 m = 16 000$,

$К = 11 000, m = 5000, р' = \frac{5000}{11000} = 45 \sfrac{5}{11}\%$.
\end{center}

Візьмімо далі капітал III, в якому зовсім немає основного капіталу,
але є 6000 обігового сталого і 5000 змінного капіталу. При нормі додаткової
вартості в 100\% він обертається один раз на рік. Тоді весь продукт за рік буде:
\begin{center}
$6000c + 5000 v + 5000 m = 16 000,$

$К = 11000, m = 5000, р' = \frac{5000}{11000} = 45\sfrac{5}{11}\%.$
\end{center}
Отже, в усіх трьох випадках ми маємо однакову річну масу
додаткової вартості = 5000, а через те що весь капітал в усіх
трьох випадках теж однаковий, а саме = 11 000, то маємо
й однакову норму зиску в 45\sfrac{5}{11}\%.

Навпаки, якщо при капіталі І ми мали б не 10, а тільки
5 річних оборотів змінної частини, то справа стояла б інакше.
Тоді продукт одного обороту був би:\[
200 с \text{(зношування)} + 500 с + 500 v + 500 m = 1700.\]

Або річний продукт:

\index{iii1}{0089}  %% посилання на сторінку оригінального видання
\begin{center}
$1000с \text{(зношування)} + 2500 с + 2500 v + 2500 m = 8500,$

$К = 11 000, m = 2500, р' = \frac{2500}{11000} = 22 \sfrac{8}{11}\%.$
\end{center}
Норма зиску знизилася б наполовину, бо час обороту подвоївся.

Отже, маса додаткової вартості, привласнювана протягом року,
дорівнює масі додаткової вартості, привласнюваній за один період
обороту \emph{змінного} капіталу, помноженій на число таких оборотів
за рік. Якщо привласнювану за рік додаткову вартість або зиск
ми назвемо $М$, привласнювану за один період обороту додаткову
вартість — $m$, число річних оборотів змінного капіталу — $n$, то
$М = mn$, а річна норма додаткової вартості $М' = m'n$, як це
вже показано в книзі II, розд. XVI, 1.

Само собою зрозуміло, що формула норми зиску $р' = m' \frac{v}{K} =
m' \frac{v}{c+v}$ правильна тільки тоді, коли v чисельника однакове
з $v$ знаменника. У знаменнику $v$ є вся та частина всього капіталу,
яка пересічно застосована як змінний капітал на заробітну
плату; $v$ чисельника насамперед визначається тільки тим, що
воно виробило і привласнило певну кількість додаткової вартості
\index{iii1}{0090}  %% посилання на сторінку оригінального видання
 $= m$, відношення якої до нього, $m/v$, є норма додаткової
вартості $m'$. Тільки таким шляхом рівняння $р' = \frac{m}{c + v}$ перетворилось
в друге: $р' = m' \frac{v}{c + v}$. Тепер $v$ чисельника ближче визначається
тим, що воно мусить бути рівне $v$ знаменника, тобто
всій змінній частині капіталу $К$. Інакше кажучи, рівняння
$р' = m/K$ можна тільки тоді без помилки перетворити в друге рівняння
$р' = m' \frac{v}{c + v}$, коли $m$ означає додаткову вартість, вироблену
за один період обороту змінного капіталу. Якщо $m$ охоплює
тільки частину цієї додаткової вартості, то хоч $m — m'v$ є
правильне рівняння, але це $v$ тут менше, ніж $v$ в $K = с + v$, бо
воно менше, ніж весь змінний капітал, витрачений на заробітну
плату. Якщо ж $m$ охоплює більше, ніж додаткову вартість від
одного обороту $v$, то частина цього $v$ або навіть все $v$ функціонує
двічі: спочатку в першому, потім у другому або в другому
й дальших оборотах; отже, це $v$, яке виробляє додаткову
вартість і яке становить суму всієї виплаченої заробітної плати,
є більше, ніж $v$ в $c + v$, і тому обчислення стає неправильним.

Для того, щоб формула річної норми зиску стала цілком
правильною, ми повинні замість простої норми додаткової вартості
поставити річну норму додаткової вартості, тобто замість
$m'$ поставити $М'$, або $m'n$. Інакше кажучи, ми повинні помножити
$m'$, норму додаткової вартості — або, що зводиться до
того самого, вміщену в $К$ змінну частину капіталу $v$, — на $n$,
число оборотів цього змінного капіталу за рік, і таким чином
ми одержуємо: $р' = m'n \frac{v}{K}$, формулу для обчислення річної
норми зиску.

Але яка є величина змінного капіталу в певному підприємстві,
цього в більшості випадків не знає і сам капіталіст.
У восьмому розділі другої книги ми бачили і побачимо ще
далі, що єдина ріжниця в капіталі капіталіста, яка нав’язується
йому як істотна, є ріжниця основного й обігового капіталу.
З каси, в якій знаходиться частина обігового капіталу, яку він
має в своїх руках у грошовий формі, — оскільки вона не лежить
у банку, — він бере гроші для заробітної плати, з тієї самої
каси він бере гроші для сировинних і допоміжних матеріалів
і записує ті і другі на той самий рахунок каси. А коли б йому
й довелося вести окремий рахунок виплачуваної заробітної
плати, то цей рахунок в кінці року, правда, показав би виплачену
на заробітну плату суму, тобто $vn$, але не показав би
самого змінного капіталу $v$. Щоб визначити цей останній, капіталістові
\index{iii1}{0091}  %% посилання на сторінку оригінального видання
довелося б зробити окреме обчислення, приклад якого
ми хочемо тут навести.

Для цього ми візьмемо бавовнопрядільну фабрику на 10 000
мюльних веретен, описану в книзі І, стор. 227 \footnote*{
Стор. 152 рос. вид. 1935 р. Ред. укр. перекладу.
}, і припустимо
при цьому, що дані, взяті для одного тижня квітня 1871 року,
зберігають своє значення для цілого року. Основний капітал, вміщений
у машинах, становив 10 000 фунтів стерлінгів. Обіговий
капітал не був указаний; припустімо, що він становив 2 500 фунтів
стерлінгів, — досить висока сума, що виправдується, однак, тим
припущенням, яке ми тут весь час мусимо робити, а саме, що
не відбувається ніяких кредитних операцій, отже, що немає
тривалого чи тимчасового користування чужим капіталом. Тижневий
продукт щодо своєї вартості складався з 20 фунтів стерлінгів
на зношування машин, 358 фунтів стерлінгів авансованого
обігового сталого капіталу (плата за найом — 6 фунтів стерлінгів,
бавовна — 342 фунти стерлінгів, вугілля, газ, мастило — 10 фунтів
стерлінгів), 52 фунтів стерлінгів витраченого на заробітну
плату змінного капіталу і 80 фунтів стерлінгів додаткової вартості,
отже:\[
20с \text{(зношування)} + 358 с + 52 v + 80 m = 510.\]

Отже, щотижневе авансування обігового капіталу становило
$358 с + 52 v = 410$, і його процентний склад $= 87,3 с + 12,7 v$. При
обчисленні на весь обіговий капітал у 2500 фунтів стерлінгів
це дає 2182 фунти стерлінгів сталого і 318 фунтів стерлінгів
змінного капіталу. Через те, що вся витрата на заробітну плату
становила на рік 52 рази по 52 фунти стерлінгів, отже, 2704 фунти
стерлінгів, то виходить, що змінний капітал у 318 фунтів стерлінгів
обернувся за рік майже точно $8 1/2$ разів. Норма додаткової
вартості була  $\sfrac{80}{52} = 153 \sfrac{11}{13}\%$. За цими елементами ми обчисляємо
норму зиску, підставивши в формулу $р' = m'n \frac{v}{K}$ значення:
$m' = 153 \sfrac{11}{13}\%, n = 8 \sfrac{1}{2}, v = 318, K = 12500;$ отже:\[
р' = 153 \sfrac{11}{13} × 8 \sfrac{1}{2} × \frac{318}{12 500} = 33,27\%\]

Для перевірки цього ми скористуємось простою формулою
$р' = \frac{m}{K}$. Вся додаткова вартість, або зиск, становить за рік
52 фунти стерлінгів $× 80 = 4160$ фунтів стерлінгів; поділене на
весь капітал у 12 500 фунтів стерлінгів, це дає майже стільки ж, як
вище, 33,28\%, ненормально високу норму зиску, яка пояснюється
тільки надзвичайно сприятливими умовами даного моменту (дуже
дешеві ціни на бавовну поряд з дуже високими цінами на пряжу)
і в дійсності існувала, без сумніву, не на протязі всього року.



\index{iii1}{0092}  %% посилання на сторінку оригінального видання
В формулі $р' = m'n \frac{v}{K}$, як сказано, $m'n$ є те, що в другій
книзі названо річною нормою додаткової вартості. У вищенаведеному
випадку вона становить 153 \sfrac{11}{13}\% × 8 \sfrac{1}{2}, або точно
1307 \sfrac{9}{13}\%. Отже, якщо якийсь бравий чоловік сплеснув руками
з приводу потворності річної норми додаткової вартості в 1000\%,
наведеної в одному прикладі в другій книзі, то він, може, заспокоїться
на факті річної норми додаткової вартості понад
1300\%, який наведено йому тут з живої практики Манчестера.
В часи найвищого розквіту, яких ми, правда, давно вже не переживали,
така норма аж ніяк не є рідкість.

До речі сказати, ми маємо тут приклад дійсного складу капіталу
в сучасній великій промисловості. Весь капітал поділяється на
12182 фунти стерлінгів сталого і 318 фунтів стерлінгів змінного
капіталу, разом 12500 фунтів стерлінгів. Або в процентах:
$97 \sfrac{1}{2}c+ 2 \sfrac{1}{2}v = 100K$. Тільки сорокова частина всього капіталу,
але, повторно обертаючись більше ніж вісім разів на рік, служить
для виплати заробітної плати.

Через те що, звичайно, тільки небагатьом капіталістам спадає
на думку робити такі обчислення щодо свого власного підприємства,
то статистика майже абсолютно мовчить про відношення
сталої частини всього суспільного капіталу до змінної
частини. Тільки американський перепис дає те, що можливе при
сучасних відносинах: суму заробітної плати, виплаченої в кожній
галузі підприємств, і одержаних зисків. Хоч і які підозрілі ці
дані, — бо вони основані тільки на неперевірених повідомленнях
самих промисловців, — проте вони надзвичайно цінні і становлять
усе, що ми маємо про цей предмет. В Европі ми занадто делікатні,
щоб вимагати від наших великих промисловців подібних
викрить. — \emph{Ф. Е.}]

\section{Економія в застосуванні сталого капіталу}

\subsection{Загальні положення}

Збільшення абсолютної додаткової вартості, або здовження
додаткової праці, отже й робочого дня, при незмінній величині
змінного капіталу, тобто при вживанні того самого числа робітників
за ту саму номінально заробітну плату, — при чому байдуже,
чи оплачується надурочний час чи ні, — відносно знижує
вартість сталого капіталу порівняно з вартістю всього капіталу
і змінного капіталу і підвищує цим норму зиску, знов таки
незалежно від зростання й маси додаткової вартості і можливого
підвищення норми додаткової вартості. Розмір основної
частини сталого капіталу, фабричних будівель, машин тощо лишається
той самий, однаково, чи працюють за його допомогою 16,
чи 12 годин. Здовження робочого дня не вимагає ніяких нових
затрат на цю найдорожчу частину сталого капіталу. До цього долучається
\index{iii1}{0093}  %% посилання на сторінку оригінального видання
ще й те, що вартість основного капіталу таким чином репродукується
за коротший ряд періодів обороту, отже, скорочується
час, на який він мусить бути авансований, щоб одержати
певний зиск. Тому здовження робочого дня збільшує зиск навіть
тоді, коли надурочний час оплачується, а до певної міри навіть
тоді, коли він оплачується вище, ніж нормальні робочі години.
Тому постійно зростаюча при сучасній промисловій системі необхідність
збільшення основного капіталу була для ненаситно
жадливих до зиску капіталістів головним стимулом до здовження
робочого дня.\footnote{
„Через те що на всіх фабриках дуже висока сума основного капіталу
вкладена в будівлі і машини, зиск буде тим більший, чим більше число годин,
протягом яких ці машини можуть бути в роботі“ („Rep. of Insp. of Fact., 31.
October 1858“, стор. 8).
}

Інші умови маємо при сталому робочому дні. В цьому випадку
для того, щоб експлуатувати більшу масу праці, треба
або збільшити число робітників, і разом з тим до певної міри
масу основного капіталу, будівель, машин і т. д. (бо ми тут
залишаємо осторонь відрахування з заробітної плати або зниження
заробітної плати нижче її нормальної висоти). Абож, якщо
збільшується інтенсивність праці чи підвищується продуктивність
праці, якщо взагалі виробляється більше відносної додаткової
вартості, то в тих галузях промисловості, які застосовують
сировинний матеріал, зростає маса обігової частини
сталого капіталу, бо за даний період часу переробляється
більше сировинного матеріалу і т. д.; і, подруге, зростає кількість
машин, які приводяться в рух тим самим числом робітників,
отже, і відповідна частина сталого капіталу. Зростання додаткової
вартості супроводиться, отже, зростанням сталого капіталу, зростаюча
експлуатація праці — подорожчанням тих умов виробництва,
за допомогою яких експлуатується праця, тобто більшими
витратами капіталу. Отже, через це норма зиску з одного
боку зменшується, тимчасом як з другого боку вона підвищується.

Цілий ряд поточних затрат лишається майже або цілком
однаковий як при довшому, так і при коротшому робочому дні.
Витрати нагляду менші при 500 робітниках і 18-годинному робочому
дні, ніж при 750 робітниках і 12-годинному робочому дні.
„Витрати ведення фабрики при десятигодинній праці майже однаково
високі, як і при дванадцятигодинній“ („Rep. of Insp. of
Fact., Oct. 1848“, стор. 37). Державні та комунальні податки,
страхування від огню, заробітна плата різних постійних службовців,
зневартнення машин і різні інші затрати фабрики лишаються
незмінними при довгому чи короткому робочому дні;
в міру того, як скорочується виробництво, вони підвищуються
коштом зиску („Rep. of Insp. of Fact., Oct. 1862“, стор. 19).

Період часу, протягом якого репродукується вартість машин
і інших складових частин основного капіталу, на практиці визначається
\index{iii1}{0094}  %% посилання на сторінку оригінального видання
не тим часом, протягом якого вони просто існують,
а загальною тривалістю процесу праці, на протязі якого вони
функціонують і використовуються. Якщо робітники мусять працювати
18 годин замість 12, то це становить за тиждень на три
дні більше, тиждень перетворюється в півтора тижня, два
роки — в три. Отже, якщо надурочний час не оплачується, то
робітники, крім нормального часу додаткової праці, дають задарма
на кожні два тижні третій, на кожні два роки третій.
І таким чином репродукція вартості машин прискорюється на 50\%
і закінчується за \sfrac{2}{3} часу, необхідного при звичайних умовах.

У цьому дослідженні, так само як і в дослідженні коливань
ціни сировинного матеріалу (в розд. VI), ми, щоб уникнути
зайвих ускладнень, виходимо з припущення, що масу і норму
додаткової вартості дано.

Як уже зазначено при розгляді кооперації, поділу праці
і ролі машин, економія в умовах виробництва, яка характеризує
виробництво у великому масштабі, в істотному виникає з того,
що ці умови функціонують як умови суспільної, суспільно-комбінованої
праці, отже, як суспільні умови праці. Вони
споживаються у процесі виробництва спільно, колективним робітником,
замість споживатись у роздрібненій формі масою
незв’язаних між собою робітників або в кращому разі робітниками,
в незначній мірі безпосередньо зв’язаними відносинами співробітництва.
На великій фабриці з одним або двома центральними
двигунами витрати на ці двигуни зростають не в тій самій пропорції,
в якій зростає кількість їх кінських сил, і отже можлива сфера
їх діяння; витрати на передатні механізми зростають не в тій самій
пропорції, в якій зростає маса робочих машин, яким вони передають
рух; самий корпус робочої машини дорожчає не в тій
пропорції, в якій збільшується число знарядь, якими вона діє
як своїми органами, і т. д. Далі, концентрація засобів виробництва
дає заощадження на будівлях усякого роду, не тільки
на власне майстернях, але й на складських приміщеннях і т. д.
Так само стоїть справа з видатками на опалення, освітлення
і т. д. Інші умови виробництва лишаються ті самі, все одно,
багато чи мало людей використовує їх.

Але вся ця економія, яка виникає з концентрації засобів виробництва
та їх масового застосування, передбачає, як істотну
умову, скупчення й спільну діяльність робітників, тобто суспільну
комбінацію праці. Отже, вона виникає з суспільного характеру
праці цілком так само, як додаткова вартість виникає
з додаткової праці кожного окремого робітника, розглядуваного
ізольовано. Навіть постійні поліпшення, які тут можливі й потрібні,
виникають виключно з суспільних дослідів і спостережень,
що їх дає і уможливлює виробництво комбінованого у
великому масштабі колективного робітника.

Те саме стосується і другої великої галузі економії в умовах
виробництва. Ми маємо на увазі зворотне перетворення
\parbreak{}  %% абзац продовжується на наступній сторінці

\input{_0095_0096.tex}
\input{_0097_0098_0099.tex}
\parcont{}  %% абзац починається на попередній сторінці
\index{ii}{0100}  %% посилання на сторінку оригінального видання
і безперервність процесу циркуляції, а тому й процесу репродукції, що
має в собі і процес циркуляції.

Треба згадати, що $Т' — Г'$ може вже відбутись для продуцента $Т$,
хоч $Т$ все ще перебуває на ринку. Коли б сам продуцент захотів тримати
свій власний товар у себе на складах, доки його продасться остаточному
споживачеві, то він мусів би пустити в рух подвійний капітал:
один — як продуцент товару, другий — як купець. Для самого товару,
хоч розглядати його як поодинокий товар, хоч як складову частину
суспільного капіталу, справа зовсім не змінюється від того, чи витрати
на утворення запасу припадають на продуцента товару, чи на ряд купців,
від $А$ до $Z$.

Оскільки товаровий запас є не щось інше, як товарова форма запасу,
що при даному маштабі суспільної продукції, коли б він не існував у
формі товарового запасу, існував би або як продуктивний запас (лятентний
фонд продукції), або як споживний фонд (резерв засобів споживання),
остільки й витрати, що їх потребує зберігання запасу, отже, витрати на
утворення запасу, — тобто вжита для цього зречевлена або жива праця —
є лише витрати зберігання хоч суспільного продукційного фонду,
хоч суспільного фонду споживання. Підвищення вартости товару,
зумовлене ними, розподіляє ці витрати лише pro rata між різними
товарами, бо вони різні для різних сортів товару. Як і раніш,
витрати на утворення запасу лишаються одбавою із суспільного багатства,
хоч вони є умова його існування.

Лише оскільки товаровий запас є умова циркуляції товарів і навіть
форма, що доконечно постала в товаровій циркуляції, оскільки, отже,
цей позірний застій є форма самого руху, цілком так само, як утворення
грошового резерву є умова грошової циркуляції, — лише остільки
він є нормальний. Навпаки, скоро товари, що затрималися в резервуарах
циркуляції, не звільняють місця для наступної хвилі продукції,
скоро, отже, резервуари переповнюються, товаровий запас збільшується
в наслідок застою в циркуляції, цілком так само, як зростають
скарби, коли затримується грошова циркуляція. При цьому
байдуже, чи цей застій постає в амбарах промислових капіталістів, чи
на складах купця. Товаровий запас тоді є вже не умова безперервного
продажу, а наслідок того, що товари не сила продати. Витрати лишаються
ті самі, але що вони тепер випливають виключно з форми, а саме з
доконечности перетворити товари на гроші, та з труднощів у цій метаморфозі,
то вони не входять у вартість товару, а становлять одбаву, втрату
вартости при реалізації вартости. Що нормальна і анормальна форма
запасу не відрізняється щодо форми, і обидві являють застій циркуляції,
то явища можна сплутати, та й самі агенти продукції допускаються
цієї помилки тим легше, що для продуцента процес циркуляції
його капіталу може перебігати, хоч процес циркуляції його товарів,
які перейшли в руки купців, зупинився. Коли більшають розміри
продукції та споживання, то, за інших незмінних обставин, збільшується
й товаровий запас. Він відновлюється й поглинається так само швидко,
\parbreak{}  %% абзац продовжується на наступній сторінці

\input{_0101.tex}
\parcont{}  %% абзац починається на попередній сторінці
\index{iii2}{0102}  %% посилання на сторінку оригінального видання
марнотратникам-вельможам, вони шукали й знаходили поза межами своєї країни
бланковий вексельний кредит, тобто такий кредит, що собі за основу не мав
ніякісінької товарової торговлі; кредит, що його закордонні трасати терпляче
акцептували доти, доки ще надходили римеси, утворені цим вексельовим шахрайством.
За це вони дуже тяжкого лиха зазнали з причини банкрутства такого
банкіра, як Тапер, та інших вельми поважних варшавських банкірів» (І. G. Büsch,
Teoretisch-praktische Darstellung der Handlung etc 3. Auflage. Hamburg 1808. Band
II, p. 232, 233.)

\subsubsection{Користь для церкви від заборони проценту}

«Проценти брати церква забороняла; але не забороняла продавати власність,
щоб зарадити собі в нужді; навіть не забороняла віддавати цю власність
на певний час, аж до оплати боргу, грошовому позикодавцеві, щоб він міг собі
мати в тій власності забезпечення, і також, щоб протягом того часу, поки та
власність перебуває в його руках, міг він користуючися мати винагородження
за визичені гроші. Сама церква або приналежні до неї комуни й ріа corpora\footnote*{
Ріа corpora — дослівно «благочестиві, побожні тіла», тобто вірні, в церковних громадах,
об’єднані. Прим. Ред.
} добували
собі від того значну користь, особливо підчас хрестових походів. Таким
чином значна частина національного доходу опинилася з цієї причини в володінні
так званої «мертвої руки», особливо тому, що єврей не міг лихварювати таким
способом, бо володіння такою нерушною заставою не сила було затаїти... Без
заборони проценту церкви й манастирі ніколи б не мали змоги стати такими
багатими». (1. с., р. 55.)



\index{iii2}{0103}  %% посилання на сторінку оригінального видання
\chapter{Перетворення надзиску на земельну ренту}

\section{Вступ}

Аналіза земельної власности в її різних історичних формах лежить поза
межами цієї праці. Ми спиняємось на ній лише остільки, оскільки частина додаткової
вартости, випродукуваної капіталом, припадає земельному власникові.
Отже, ми припускаємо, що в хліборобстві, цілком так само, як в мануфактурі
панує капіталістичний спосіб продукції, тобто що сільське господарство провадять
капіталісти, які відрізняються від решти капіталістів передусім лише тим
елементом, до якого прикладається їхній капітал та наймана праця, яку цей
капітал пускає в рух. На наш погляд фармер продукує пшеницю і т. ін. так
само, як фабрикант — пряжу або машини. Та передумова, що капіталістичний спосіб
продукції опанував сільське господарство, має в собі й те, що цей спосіб продукції
опановує всі сфери продукції й буржуазного суспільства, що, отже,
є наявні і його цілком розвинуті умови, як от вільна конкуренція капіталів,
змога переносити їх з однієї сфери продукції до іншої, однакова висота пересічного
зиску і т. ін. Та форма земельної власности, що її ми розглядаємо,
становить специфічно історичну її форму, \emph{перетворену} — в наслідок впливу
капіталу та капіталістичного способу продукції — форму або февдальної земельної
власности, або дрібно-селянського хліборобства, що провадиться як ділянка
для прохарчування, хліборобства, що в ньому \emph{володіння} землею для безпосередного
продуцента є одна з умов продукції, а його, того продуцента \emph{власність}
на землю є найвигідніша умова розцвіту \emph{його} способу продукції. Якщо
капіталістичний спосіб продукції взагалі має собі за передумову експропріацію
умов праці в робітників, то в хліборобстві він має собі за передумову
експропріацію землі в сільських робітників та підпорядкування їх капіталістові,
що провадить хліборобство за-для зиску. Отже, для нашої аналізи цілком байдуже,
коли нам заперечуватимуть, нагадуючи, що були або ще й досі є й інші
форми земельної власности та хліборобства. Це заперечення може вразити тільки
тих економістів, що розглядають капіталістичний спосіб продукції в сільському
господарстві та відповідну йому форму земельної власности не як історичні, а як
вічні категорії.

Для нас розгляд новітньої форми земельної власности потрібний тому, що
взагалі справа йде про розгляд тих певних відносин продукції й обміну, що
\parbreak{}  %% абзац продовжується на наступній сторінці


\index{ii}{0104}  %% посилання на сторінку оригінального видання
\chapter{Оборот капіталу}

\section{Час обороту й число оборотів}

Ми бачили: сукупний час циркуляції даного\footnote*{
Термін „сукупний час циркуляції“ тут Маркс вживає в тому самому розумінні,
в якому він далі в цьому ж розділі вживає термін „час обороту“, тимчасом
як взагалі він з цій книзі термін „час циркуляції“ вживає в тому самому
розумінні, що і „час обігу“, тобто в розумінні того часу, що протягом його капітал
перебуває в сфері циркуляції. (Дивись розділ V). \emph{Ред.}
} капіталу дорівнює сумі
часу його обігу та часу його продукції. Це є відтинок часу від моменту
авансування капітальної вартости в певній формі до моменту, коли капітальна
вартість, що процесує, повертається в тій самій формі.

Мета, що визначає капіталістичну продукцію, завжди є зростання
авансованої вартости, чи авансовано цю вартість в її самостійній формі,
тобто в грошовій формі, чи в формі товару, так що його форма вартости
має лише ідеальну самостійність у ціні авансованих товарів.
В обох випадках ця капітальна вартість перебігає протягом свого кругобігу
різні форми існування. Її тотожність з самою собою констатується
в книгах капіталіста або в формі рахункових грошей.

Хоч візьмемо ми форму $Г\dots{} Г'$, хоч форму $П\dots{} П$, обидві форми
значать: 1) що авансована вартість функціонувала як капітальна вартість
і зросла своєю вартістю; 2) що по закінченні процесу вона повернулась
до тієї форми, в якій почала його. Зростання авансованої вартости Г і
разом з тим поворот капіталу до цієї форми (до грошової форми) виразно
помітно в $Г\dots{} Г'$. Але те саме відбувається і в другій формі. Бо
вихідний пункт для П є наявність елементів продукції, товарів даної
вартости. Ця форма має в собі зростання цієї вартости (Т' і $Г'$) і поворот
до первісної форми, бо в другому П авансована вартість
знову має форму елементів продукції, що в ній її первісно авансовано.

Раніше ми бачили: „Якщо продукція має капіталістичну форму, то
і репродукція має ту саму форму. Як процес праці за капіталістичного
способу продукції є лише засіб для процесу зростання вартости, так
\parbreak{}  %% абзац продовжується на наступній сторінці

\parcont{}  %% абзац починається на попередній сторінці
\index{ii}{0105}  %% посилання на сторінку оригінального видання
само й репродукція є лише засіб репродукувати авансовану вартість
як капітал, тобто як вартість, що зростає сама з себе“. (Книга І,
розд. XXI).

Три форми: І) $Г\dots{} Г'$, II) $П\dots{} П$ і III) $Т'\dots{} Т'$ відрізняються між
собою ось чим: в формі II (П\dots{} П) відновлення процесу, процесу репродукції,
виражено як дійсне, а в формі І лише як можливе. Але обидві ці
форми відрізняються від форми III тим, що авансована капітальна вартість
— хоч її авансовано як гроші, хоч в вигляді речових елементів продукції
— становить вихідний пункт, а тому й пункт повороту. В $Г\dots{} Г' п$оворот
є $Г' \deq{} Г \dplus{} г$. Коли процес відновлюється знову в тих самих розмірах,
то Г знову становить вихідний пункт, а г не входить в цей процес і лише
показує нам, що Г зросло своєю вартістю як капітал і тому створило додаткову
вартість г, але відштовхнуло її від себе. В формі $П\dots{} П$ капітальна
вартість, авансована в формі П, елементів продукції, знову таки
становить вихідний пункт. Ця форма має в собі й зростання цієї вартости.
Коли відбувається проста репродукція, то та сама капітальна вартість
в тій самій формі П знову починає свій процес. Коли відбуваєтьсяакумуляція,
то тепер процес починає $П'$ (що величиною вартости дорівнює
$Г' \deq{} Т'$), як збільшена капітальна вартість. Але процес починається знову
авансованою капітальною вартістю в початковій формі, хоч і капітальною
вартістю більшою, ніж раніш. Навпаки, в формі III капітальна вартість
починає процес не як авансована, але як уже виросла, як усе багатство,
що перебуває в формі товарів, і що лише деяка частина його являє
авансовану капітальну вартість. Остання форма важлива для третього
відділу, де рух поодиноких капіталів береться в зв’язку з рухом сукупного
суспільного капіталу. Але, навпаки, з неї не можна користатись,
коли досліджується оборот капіталу, що завжди починається авансуванням
капітальної вартости, чи то у формі грошей, чи то у формі товару, і
який завжди зумовлює, що капітальна вартість, яка чинить оборот, повертається
в тій формі, що в ній її авансовано. З кругобігів І і II
треба триматися першого, коли мають на увазі переважно той вплив, що
його справляє оборот на утворення додаткової вартости; другого — коли
мають на увазі вплив обороту на утворення продукту.

Як мало економісти відрізняли різні форми кругобігів, так само мало
вони розглядали ці різні форми кругобігів відокремлено щодо обороту
капіталу. Звичайно береться форму $Г\dots{} Г'$, бо вона панує над поодиноким
капіталістом і служить йому в його розрахунках навіть тоді, коли
гроші становлять вихідний пункт лише в формі рахункових грошей. Інші
беруть за вихідний пункт витрати в формі елементів продукції, поки
не настане поворот, при цьому про форму повороту — чи буде цей поворот
в товарі, чи в грошах — у них немає й мови. Напр.: „Економічний
цикл\dots{} тобто ввесь перебіг продукції від часу, коли зроблено витрати,
до часу, коли настає поворот. В сільському господарстві час засіву є
початок економічного циклу, а жнива — закінчення“. (Economic Cycle\dots{}
the whole course of production, from the time that outlays are made till
returns are received. In agriculture seedtime is its commencement, and
\parbreak{}  %% абзац продовжується на наступній сторінці

\parcont{}  %% абзац починається на попередній сторінці
\index{iii1}{0106}  %% посилання на сторінку оригінального видання
сухот, в Блекберні й Скіптоні — 167, в Конглетоні й Бредфорді —
168, в Лейстері — 171, в Ліку — 182, в Маккльсфільді — 184,
в Больтоні — 190, в Ноттінгемі — 192, в Ронделі — 193, в Дербі —
198, в Сальфорді і Аштоні-на-Лайні — 203, в Лідсі — 218, в Престоні — 220 і в Манчестері — 263
(стор. 24). Нижченаведена таблиця дає ще разючіший приклад. Вона наводить випадки смерті
внаслідок хвороб легенів окремо для обох статей між 15 і
25 роками, обчислені на кожні 10 0000 мешканців. Вибрано такі
округи, де тільки жінки зайняті в промисловості, провадженій
у закритих приміщеннях, а чоловіки — в усяких можливих галузях праці.

\begin{table}[ht]
  \small
  \begin{tabular}{c c c c}
    \toprule
    Округи &
    Головна промисловість &
    \multicolumn{2}{c}{\makecell{Число випадків смерті від \\легеневих захворувань між 15 і 25\\
роками на 100 000 жителів}}\\
    \cmidrule(rl){3-4}
    & & Чоловіки & Жінки \\
    \midrule
Berkhampstead    & \makecell{Плетіння з соломи,\\ працюють жінки} & 219 & 578 \\
Leighton Buzzard & \makecell{Плетіння з соломи,\\ працюють жінки} & 309 & 554 \\
Newport Pagnell  & \makecell{Плетіння мережива\\ жінками}         & 301 & 617 \\
Towcester        & \makecell{Плетіння мережива\\ жінками}                         & 239 & 577 \\
Yeovil           & \makecell{Виробництво рукавичок,\\ здебільшого працюють жінки} & 280 & 409 \\
Leek             & \makecell{Шовкова промисловість,\\ переважно жінки}            & 437 & 856 \\
Congleton        & \makecell{Шовкова промисловість,\\ переважно жінки}            & 566 & 790 \\
Macclesfield     & \makecell{Шовкова промисловість,\\ переважно жінки}            & 593 & 890 \\
\makecell{Здорова сільська\\ місцевість} &   Землеробство                         & 331 & 333 \\
  \end{tabular}
\end{table}

В округах шовкової промисловості, де участь чоловіків
у фабричній праці більша, більша також і смертність серед них.
Норма смертності від сухот і т. п. як чоловіків, так і жінок
виявляє тут, як сказано в звіті, „обурливі (atrocious) санітарні
умови, за яких провадиться значна частина нашої шовкової
промисловості“. І це, якраз, та сама шовкова промисловість,
фабриканти якої, посилаючись на винятково сприятливі санітарні умови свого виробництва, вимагали і
почасти добилися
винятково довгого робочого часу для дітей, молодших 13 років.
(книга І, розд. VIII, 6, стор. 306\footnote*{Стор. 214 рос. вид. 1935 р. Ред. укр. перекладу.}).

„Без сумніву, жодна з досліджених досі галузей промисловості
не дає сумнішої картини, ніж та, що її дає доктор Сміт
щодо кравецтва\dots{} Майстерні, каже він, дуже неоднакові щодо

\parbreak{}  %% абзац продовжується на наступній сторінці

\parcont{}  %% абзац починається на попередній сторінці
\index{iii1}{0107}  %% посилання на сторінку оригінального видання
санітарного стану; але майже всі вони переповнені, погано провітрюються і в високій мірі
несприятливі для здоров’я\dots{} В таких
кімнатах, крім усього, неодмінно жарко; а коли запалюють газ,
як це роблять удень під час туману або зимою вечорами, то температура підвищується до 80 і навіть до
90 градусів (за Фаренгейтом = 27—33° Цельсія) і викликає надзвичайне пітніння робітників
і згущення пари на шибках, так що вода безупинно стікає або
крапає з вікна в стелі, і робітники змушені держати відчиненими
кілька вікон, хоча вони при цьому неминуче простуджуються. —
Становище в 16 найзначніших майстернях лондонського Вестенду
він описує так: найбільший кубічний простір, який припадає
в цих погано провітрюваних кімнатах на одного робітника, становить 270 кубічних футів; найменший —
105 футів, пересічно —
всього тільки 156 футів на людину. В одній майстерні, яка обведена з усіх боків галереєю і має
освітлення тільки згори,
занято від 92 до 100 осіб; горить багато газових ріжків; клозети
збудовані безпосередньо коло майстерні, і на кожну людину
припадає не більше, як 150 кубічних футів простору. В другій
майстерні, в освітленому згори дворі, яку можна назвати тільки
собачою конурою і яку можна провітрювати тільки через маленьке вікно в даху, працює 5 чи 6 осіб, при
чому на кожну
з них припадає 112 кубічних футів“. І „в цих жахливих (atrocious) майстернях, які описує доктор
Сміт, кравці працюють
звичайно 12—13 годин на день, а іноді праця триває 14—16 годин“ (стор. 25, 26, 28).

\begin{table}[ht]
  \footnotesize
  \begin{tabular}{ c c c c c}
  \toprule
Число занятих людей & \makecell{Галузь промисловості\\ і місцевість} & \multicolumn{3}{c}{\makecell{Норма смертності на 100 000 осіб\\ віком}}\\
\cmidrule(rl){3-5}
& & 25\textendash{}35 р. & 35\textendash{}45 р. & 45\textendash{}55 р.\\
\midrule

958265                          & Землеробство, Англія та Уельс & 743 & 805 & 1145\\
\makecell{22 301 чоловіків і\\ 12 377 жінок} & Кравці, Лондон                & 958 & 1262 & 2093\\
13 803                          & Складачі й друкарі, Лондон    & 894 & 1747 & 2367\\

  \end{tabular}
\end{table}
(стор. 30). Треба відзначити — і це дійсно відзначено складачем
цього звіту, завідувачем медичного відділу, Джоном Сімоном, —
що для віку в 25—35 років смертність кравців, складачів і друкарів Лондона показана применшеною, бо
в обох цих галузях
промисловості лондонські майстри одержують з села велике
число молодих людей (мабуть, до 30 років), що працюють як
учні і „improvers“, тобто для дальшого удосконалення. Вони
збільшують число занятих осіб, на яке треба обчисляти норми
смертності промислового населення Лондона; але вони не збільшують в такій самій мірі число смертей у
Лондоні, бо їх перебування в Лондоні тільки тимчасове; коли вони захворіють на протязі цього часу,
то вертаються додому на село, і смерть
їх, якщо вони умирають, реєструється там. Ця обставина ще
в більшій мірі стосується до молодшого віку, і в наслідок цього
\parbreak{}  %% абзац продовжується на наступній сторінці

\input{_0108.tex}
\input{_0109.tex}
\parcont{}  %% абзац починається на попередній сторінці
\index{i}{0110}  %% посилання на сторінку оригінального видання
товарів, покупець або продавець, а саме, в обох рядах оборудок
виступаю проти одного контраґента лише як покупець, а
проти другого лише як продавець, проти одного — лише як
гроші, проти другого — лише як товар; ані проти одного, ані
проти другого я не виступаю як капітал або як капіталіст або
як представник чогось такого, що було б більше, ніж гроші або
товар, або могло б заподіяти інший вплив, крім того, що його
можуть справляти гроші або товар. Для мене купівля в \emph{А} і продаж
\emph{В} становлять послідовний ряд. Але зв’язок поміж цими
обома актами існує лише для мене. А немає жодного діла до моєї
оборудки з \emph{В}, а \emph{В} — до моєї оборудки з \emph{А}. Коли б я захотів
пояснити їм особливу заслугу, яку я маю перед ними, обертаючи
послідовність ряду, то вони довели б мені, що я помиляюсь щодо
самого порядку послідовности, і що вся операція почалася не
від купівлі й кінчається не продажем, а, навпаки, почалася від
продажу й завершується купівлею. Справді, мій перший акт,
купівля, з погляду \emph{А} є продаж, а мій другий акт, продаж, з
погляду \emph{В} — купівля. Не задовольнившися цим, \emph{А} й \emph{В} заявляють,
що цілий цей порядок послідовности був зайвий фокус-покус.
\emph{А} продасть товар безпосередньо \emph{В}, а \emph{В} купить його безпосередньо
в \emph{А}. Разом з тим вся операція стискується в однобічний
акт звичайної товарової циркуляції, — просто продаж з погляду
\emph{А} і просто купівлю з погляду \emph{В}. Отже, обернувши порядок послідовности,
ми не вийшли поза сферу простої товарової циркуляції,
а тому ми мусимо розглянути, чи допускає вона з своєї природи
зростання вартостей, що входять у неї, тобто чи допускає вона
творення додаткової вартости.

Візьмімо процес циркуляції у формі, в якій він виявляється
як простий обмін товарів. Це завжди буває тоді, коли обидва
посідачі товарів купують один в одного товари і в термін платежу
вирівнюють балянс своїх взаємних грошових зобов’язань. Гроші
служать тут за рахункові гроші, щоб виразити вартості товарів
у їхніх цінах, але вони не виступають проти самих товарів речово.
Ясна річ, що, оскільки йдеться про споживну вартість,
виграти можуть обидва обмінювані. Обидва відчужують товари,
які є некорисні для них як споживні вартості, і одержують товари,
що їх вони потребують для споживання. І користь од цього може
бути не лише ця одна. \emph{А}, що продає вино й купує збіжжя, продукує,
може, більше вина, ніж його зміг би випродукувати за той
самий робочий час рільник \emph{В}, а рільник \emph{В} за той самий робочий
час продукує більше збіжжя, ніж його зміг би випродукувати
винар \emph{А}. Отже, \emph{А} дістає за таку саму мінову вартість більше
збіжжя, а \emph{В} — більше вина, ніж дістав би відповідно кожний
із них без обміну, коли б вони мусили продукувати сами для себе
вино і збіжжя. Таким чином щодо споживної вартости можна
сказати, що «обмін є оборудка, в якій виграють обидві сторони»\footnote{
«Обмін є дивна оборудка, в якій виграють обидва контраґенти —
завжди (!)» («L’échange est une transaction admirable, dans la quelle les
deux contractants gagnent — toujours (!)»). (\emph{Destutt de Tracy}: «Traité de
la Volonté et de ses effets», Paris 1826, p. 68). Ta сама книга появилася пізніше
під назвою «Traité d’Economie Politique».
}.
\index{i}{0111}  %% посилання на сторінку оригінального видання
Інша справа з міновою вартістю. «Людина, що має багато вина,
а не має збіжжя, веде торг з людиною, що в неї багато збіжжя,
але немає вина, і вони обмінюють пшеницю вартістю в 50 на
вино вартістю в 50. Цей обмін не є збільшення мінової вартости
ні для одного, ані для другого, бо вже перед обміном кожний
з них мав вартість, рівну тій, що її він здобув собі за допомогою
цієї операції»\footnote{
\emph{Mercier de la Rivière}: «L’Ordre naturel et essentiel», Physiocrates, éd.
Daire, IІ.~Partie, p. 544.
}. Справа зовсім не змінюється, коли між товарами
виступають гроші як засіб циркуляції і акт купівлі почуттєво
відокремлюється від акту продажу\footnote{
«Само по собі цілком байдуже, чи є одна з цих двох вартостей
гроші, чи обидві вони є звичайні товари» («Que l’une de ces deux valeurs
soit argent, ou qu’elles soient toutes deux marchandises usuelles, rien de
plus indifférent en soi»). (Mercier de la Rivière: «L’Ordre naturel et essentiel».
Physiocrates, éd. Daire, II.~Partie, p. 543).
}. Вартість товарів є виражена
в їхніх цінах раніш, ніж вони вступають до циркуляції, отже,
вона є передумова циркуляції, а не її результат\footnote{
«Не контраґенти визначають вартість, її визначено ще до оборудки»
(«Ce ne sont pas les contractants, qui prononcent sur la valeur; elle est
décidée avant la convention»). (Le Trosne: «De l’Intérêt Social», Physiocrates,
éd. Daire, Paris 1846, p. 906).
}.

Розглядаючи справу абстрактно, тобто залишаючи осторонь
обставини, які не випливають з іманентних законів простої товарової
циркуляції, ми побачимо, що, крім заміни однієї споживної
вартости на іншу, в ній відбувається лише метаморфоза, проста
зміна форми товару. Та сама вартість, тобто та сама кількість
упредметненої суспільної праці, лишається в руках того самого
посідача товарів спочатку в формі товару, потім у формі грошей,
на які товар перетворився, нарешті, у формі товару, на який
знову перетворилися ці гроші. Ця зміна форми не містить у собі
жодної зміни величини вартости. Зміна, якої зазнає в цьому процесі
сама вартість товару, обмежується на зміні її грошової
форми. Спочатку вона існує як ціна подаваного на продаж товару,
потім як грошова сума, що була вже однак виражена в ціні, і,
нарешті, як ціна еквівалентного товару. Ця зміна форм сама по
собі так само мало містить у собі зміну величини вартости, як
ось розмін п’ятифунтової банкноти на соверени, півсоверени й
шилінґи. Отже, оскільки циркуляція товару зумовлює лише
зміну форми його вартости, вона зумовлює, — якщо явище відбувається
в чистій формі, — обмін еквівалентів. Тому навіть вульґарна
політична економія, хоч і як мало вона тямить, що таке
вартість, кожного разу, коли вона на свій штиб хоче розглянути
явище в його чистій формі, припускає, що попит і подання урівноважуються,
тобто, що вплив їхній взагалі припиняється. Отже,
коли щодо споживної вартости обидва контраґенти можуть виграти,
то на міновій вартості вони не можуть обидва виграти. Навпаки,
\parbreak{}  %% абзац продовжується на наступній сторінці

\parcont{}  %% абзац починається на попередній сторінці
\index{ii}{0112}  %% посилання на сторінку оригінального видання
відбирає їм його в другому випадку. Однак та обставина, що засоби праці льокально прикріплені,
пустили своє коріння в землю, надає цій частині основного капіталу особливої ролі в економії націй.
Їх не можна відіслати за кордон, вони не можуть циркулювати як товари на світовому ринку. Титули
власности на цей основний капітал можуть змінюватись, їх можна купувати й продавати, і остільки вони
можуть ідеально
циркулювати. Ці титули власности можуть навіть циркулювати на закордонних ринках, напр., в формі
акцій. Але від зміни осіб, що є власники такого виду основного капіталу, не змінюється відношення
між нерухомою, матеріяльно фіксованою частиною багатства даної країни і рухомою частиною того таки
багатства \footnote{До цього місця рукопис IV.~Відси рукопис II.~Ф.~Е.}).

Своєрідна циркуляція основного капіталу зумовлює своєрідний оборот. Та частина вартости, що її
втрачається в її натуральній формі в наслідок зношування, циркулює, як частина вартости продукту.
Продукт через свою циркуляцію перетворюється з товару на гроші, отже, на гроші перетворюється й та
частина вартости засобів праці, що її продукт несе в циркуляцію, а саме: ця частина вартости падає
краплями як гроші з процесу циркуляції, в тій самій пропорції, що в ній даний засіб праці перестає
бути носієм вартости в продукційному процесі. Отже, вартість цього засобу праці набирає тепер
двоїстого існування. Частина її лишається зв’язана з його споживною або натуральною формою, належною
продукційному процесові, а друга частина відокремлюється від неї як гроші. В перебігу свого
функціонування та частина вартости засобів праці, що існує в натуральній формі, постійно меншає,
тимчасом як перетворена на гроші частина вартости постійно більшає, поки, нарешті, засоби праці
одживуть свій вік, і вся їхня вартість, відокремившись від мертвого тіла, перетвориться на гроші.
Тут виявляється своєрідність в обороті цього елемента продуктивного капіталу. Його вартість
перетворюється на гроші рівнобіжно з тим, як на грошову лялечку перетворюється той товар, що є носій
його вартости. Але його зворотне перетворення з грошової форми на споживну форму відділяється від
зворотного перетворення товару на інші елементи продукції цього товару і визначається періодом його
власної репродукції, тобто часом, що протягом його засоби праці одживають свій вік, і треба їх
замінити на нові екземпляри такого самого роду. Коли час функціонування якоїсь машини, напр.,
вартістю в \num{10.000}\pound{ ф. стерл.}, дорівнює, припустімо, 10 рокам, то час обороту вартости, первісно
авансованої на неї, дорівнює 10 рокам. Поки не мине цей час, її не треба поновлювати, і вона
функціонує далі в своїй натуральній формі. Тимчасом її вартість частинами циркулює як частина
вартости товарів, що до їх безперервної продукції вона придається, — і таким чином поступінно
перетворюється на гроші, поки, нарешті, по десятьох роках, вона цілком перетвориться на гроші, а з
грошей знову на машину, вивершуючи, отже, свій оборот. До цього моменту
\parbreak{}  %% абзац продовжується на наступній сторінці

\input{_0113.tex}
\input{_0114.tex}
\parcont{}  %% абзац починається на попередній сторінці
\index{ii}{0115}  %% посилання на сторінку оригінального видання
й поновлювати зворотною купівлею, зворотним перетворенням з грошової форми на елементи продукції.
Одним заходом їх вилучається з ринку меншими масами, ніж елементи основного капіталу, але тим
частіше доводиться їх вилучати з ринку, а тому авансування витраченого на них капіталу поновлюється
через коротші періоди. Це постійне поновлення упосереднюється постійним збутом того продукту, що в
ньому циркулює вся їхня вартість. Нарешті, вони безупинно пророблюють увесь кругобіг метаморфоз не
лише своєю вартістю, але й у своїй речовій формі; з товару вони постійно перетворюються знову на
елементи продукції цього самого товару.

Разом із своєю власною вартістю робоча сила постійно долучає до продукту додаткову вартість,
втілення неоплаченої праці. Отже, готовий продукт так само подає її постійно в циркуляцію, і вона
разом з ним перетворюється на гроші так само, як і інші елементи вартости продукту. Однак, тут, де
йдеться насамперед про оборот капітальної вартости, а не додаткової вартости, що обертається разом з
нею, — тут ми лишаємо це покищо осторонь.

З наведеного вище випливає ось що:

1) Визначеності форми основного й поточного капіталу походять лише з ріжниці в обороті капітальної
вартости, що функціонує в процесі продукції, або продуктивного капіталу. Ця ріжниця в обороті
походить і собі з ріжниці в способі, що ним різні складові частини продуктивного капіталу переносять
свою вартість на продукт, а не з їхньої різної участи в утворенні вартости продукту або не з
характеристичної ролі їх у процесі зростання вартости. Нарешті, ріжниця в передачі вартости
продуктові, — а тому й різні способи, що ними ця вартість вводиться через продукт у циркуляцію і в
наслідок його метаморфоз поновлюється в своїй первісній натуральній формі, — ця ріжниця походить з
відмінности тих речових форм, що в них існує продуктивний капітал, і що з них одна частина під час
утворення окремого продукту споживається цілком, а другу зужитковується лише поступінно. Отже, лише
продуктивний капітал може розподілятись на основний і поточний. Навпаки, цієї протилежности не існує
для обох інших способів буття промислового капіталу, отже, ні для товарового капіталу, ні для
грошового капіталу; не існує її також як і протилежности цих обох форм проти продуктивного капіталу.
Вона існує лише для продуктивного капіталу і в межах його. Грошовий капітал і товаровий капітал
можуть скільки завгодно функціонувати як капітал і можуть хоч як швидко циркулювати, але зробитись
поточним капіталом протилежно до основного вони можуть лише тоді, коли перетворяться на поточні
складові частини продуктивного капіталу. Але через те, що ці обидві форми капіталу перебувають у
сфері циркуляції, то, як ми побачимо, економія від часів А.~Сміса не могла стриматися від спокуси
сплутати їх з поточною частиною продуктивного капіталу, об’єднуючи їх в категорію обіговий капітал.
А справді грошовий капітал і товаровий капітал є капітал циркуляції протилежно до продуктивного, але
не обіговий капітал протилежно до основного.

\input{_0116_0117.tex}
\input{_0118.tex}
\input{_0119.tex}
\input{_0120_0121.tex}
\input{_0122.tex}
\input{_0123.tex}
\parcont{}  %% абзац починається на попередній сторінці
\index{iii1}{0124}  %% посилання на сторінку оригінального видання
перетворена в гроші; друга частина існує як гроші в будьякій
формі і мусить бути знову перетворена в умови виробництва;
нарешті, третя частина перебуває у сфері виробництва, почасти
у первісній формі засобів виробництва, сировинних матеріалів,
допоміжних матеріалів, куплених на ринку півфабрикатів,
машин та іншого основного капіталу, почасти як продукт, який
ще тільки виготовляється. Як діє тут підвищення вартості або
зниження вартості, це в великій мірі залежить від того відношення,
в якому стоять одні до одних ці складові частини. Щоб
спростити питання, залишмо спочатку осторонь весь основний
капітал і розгляньмо тільки ту частину сталого капіталу, яка
складається з сировинних матеріалів, допоміжних матеріалів,
півфабрикатів і товарів, які ще тільки виготовляються або вже
є готові на ринку.

Якщо підвищується ціна сировинного матеріалу, наприклад,
бавовни, то підвищується й ціна бавовняних товарів — півфабрикатів,
як от пряжа, і готових товарів, як от тканини і т. д., —
сфабрикованих з дешевшої бавовни; так само підвищується
і вартість як ще непереробленої бавовни, яка є на складі, так
і тієї, що перебуває ще в процесі оброблення. Ця остання,
через те що вона в наслідок зворотного впливу стає виразом
більшої кількості робочого часу, додає до продукту, в який
вона входить як складова частина, більшу вартість, ніж та, яку
вона первісно мала сама і яку капіталіст заплатив за неї.

Отже, якщо підвищення цін сировинного матеріалу супроводиться
наявністю на ринку значної маси готового товару, —
все одно, на якому ступені готовості, — то підвищується вартість
цього товару і разом з тим відбувається підвищення вартості
наявного капіталу. Те саме стосується і до запасів сировинного
матеріалу і т. д., які перебувають в руках виробників.
Це підвищення вартості може відшкодувати або й більш ніж
відшкодувати окремих капіталістів або навіть і цілу окрему
сферу виробництва капіталу за падіння норми зиску, яке виникає
з підвищення ціни сировинного матеріалу. Не входячи тут
у деталі впливу конкуренції, можна, однак, ради повноти відзначити,
що 1) коли запаси сировинного матеріалу, які перебувають
на складах, значні, то вони протидіють підвищенню цін,
що виникає в місці виробництва сировинного матеріалу; 2) коли
півфабрикати або готові товари, які перебувають на ринку, дуже
тиснуть на ринок, то вони заважають ціні готових товарів і півфабрикатів
зростати відповідно до ціни їх сировинного матеріалу.

Зворотне маємо при падінні цін сировинного матеріалу, яке
при інших однакових умовах підвищує норму зиску. Товари, які
перебувають на ринку, речі, які ще тільки виготовляються,
запаси сировинного матеріалу знецінюються і цим самим протидіють
одночасному підвищенню норми зиску.

Чим менші запаси, які перебувають у сфері виробництва
і на ринку, наприклад наприкінці операційного року, коли сировинний
\index{iii1}{0125}  %% посилання на сторінку оригінального видання
матеріал знову постачається великими масами, як от у землеробстві після жнив, — тим
виразніше виступає вплив зміни цін сировинного матеріалу.

В усьому нашому дослідженні ми виходимо з того припущення, що підвищення або зниження цін є вираз
дійсних коливань вартості. Але через те що тут мова йде про той вплив, який ці коливання цін
справляють на норму зиску, то в дійсності не має значення, яка є причина цих коливань; отже,
розвинуте тут має силу також і тоді, коли ціни підвищуються і падають не в наслідок коливань
вартості, а в наслідок діяння системи кредиту, конкуренції і т. д.

Через те що норма зиску дорівнює відношенню надлишку вартості продукту до вартості всього
авансованого капіталу, то підвищення норми зиску, що походить із зниження вартості авансованого
капіталу, може бути сполучене з втратою капітальної вартості; так само зниження норми зиску, що
походить з підвищення вартості авансованого капіталу, може бути сполучене з виграшем.

Щодо другої частини сталого капіталу, машин і взагалі основного капіталу, то підвищення вартості,
які тут відбуваються і стосуються саме до будівель, землі і т. д., не можуть бути розглянуті до
викладу вчення про земельну ренту і тому, вони не належать сюди. Але для зниження вартості цієї
частини капіталу загальне значення мають:

1. Постійні поліпшення, які позбавляють наявні машини, фабричне устаткування і т. д. частини їх
споживної вартості,
а тому і їх вартості. Цей процес діє з особливою силою в перший період введення нових машин, раніше
ніж вони досягають певної міри зрілості, і коли вони через це постійно стають застарілими раніше,
ніж встигають репродукувати свою вартість. Це одна з причин звичайного в такі епохи безмірного
здовження робочого часу, праці вдень і вночі почережно змінами, для того, щоб протягом коротшого
часу репродукувати вартість машин, не відраховуючи при цьому занадто багато на їх зношування. Якщо
ж, навпаки, короткий період діяльності
машин (короткий строк їх життя в зв’язку з можливими поліпшеннями) не буде таким способом
скомпенсовано, то внаслідок їх морального зношування вони переносять на продукт занадто велику
частину своєї вартості, так що не можуть конкурувати навіть з ручною працею\footnote{Приклади, між іншим,
у Беббеджа. Звичайний засіб - зниження заробітної плати - застосовується і тут, і таким чином це постйно
знецінення діє цілком інакше, ніж це уявляє собі в своєму гармонійному мозку пан Кері.
}.

Якщо машини, устаткування будівель, взагалі основний капітал досяг певної зрілості, так що протягом
довшого часу
він, принаймні в своїй основній конструкції, лишається незмінним, то подібне ж зниження вартості
настає в наслідок поліпшень
\index{iii1}{0126}  %% посилання на сторінку оригінального видання
у методах репродукції цього основного капіталу. Вартість
машин і т. д. знижується тепер не тому, що вони швидко
витісняються або до певної міри знецінюються новими продуктивнішими
машинами і т. д., а тому, що вони тепер можуть
бути дешевше репродуковані. Це одна з причин, чому великі підприємства
часто процвітають тільки в других руках, після того
як збанкрутує перший власник, а другий, що дешево купив
підприємство, таким чином уже з самого початку починає своє
виробництво з меншими витратами капіталу.

В землеробстві особливо впадає в очі, що ті самі причини,
які підвищують або знижують ціну продукту, підвищують або
знижують також і вартість капіталу, бо цей останній у значній
частині сам складається з цього продукту — хліба, худоби і т. ін.
(Рікардо).

\pfbreak

Тепер треба було б згадати ще про змінний капітал.

Якщо вартість робочої сили підвищується внаслідок підвищення
вартості потрібних для її репродукції засобів існування,
або, навпаки, знижується в наслідок зниження вартості цих засобів
існування, — а підвищення вартості і зниження вартості
змінного капіталу не виражає нічого іншого, крім цих обох випадків, — то при незмінній довжині
робочого дня цьому підвищенню вартості відповідає падіння додаткової вартості, а цьому
зниженню вартості — зростання додаткової вартості. Але в той
самий час з цим можуть бути зв’язані й інші обставини — звільнення і зв’язування капіталу — які не
були ще досліджені і які
треба тепер коротко розглянути.

Якщо заробітна плата знижується внаслідок падіння вартості робочої сили (з чим може бути зв’язане
навіть підвищення
реальної ціни праці), то таким чином звільняється частина капіталу, яка досі витрачалась на
заробітну плату. Відбувається
звільнення змінного капіталу. На нововкладуваний капітал це
справляє тільки той вплив, що він працює з підвищеною нормою додаткової вартості. Та сама кількість
праці приводиться
в рух за допомогою меншої кількості грошей, ніж раніше, і таким чином неоплачена частина праці
збільшується коштом
оплаченої. Але для капіталу, який був вкладений уже раніше,
не тільки підвищується норма додаткової вартості, але, крім
того, звільняється частина капіталу, яка досі витрачалась на
заробітну плату. Досі вона була зв’язана і становила постійну
частину, яка відділялась від виручки за продукт і мусила витрачатись на заробітну плату,
функціонувати як змінний капітал,
якщо підприємство мало й далі провадитися в попередніх розмірах. Тепер ця частина стає вільною, і
може, отже, бути використана як нове капіталовкладення, чи для розширення того самого підприємства,
чи для функціонування в іншій сфері
виробництва.

\parcont{}  %% абзац починається на попередній сторінці
\index{ii}{0127}  %% посилання на сторінку оригінального видання
мости, тунелі, віадуки тощо, являє приклад того, що можна назвати
віковим зношуванням. А швидше й помітніше зневартнення, відшкодуване
протягом коротких переміжків часу ремонтом і заміщенням, є подібне
до періодичних неправильностей. У витрати на річний ремонт заводиться
й полагодження тієї випадкової шкоди, що її зазнають час від часу
зовнішні частини навіть довготриваліших споруд; але й незалежно від
такого ремонту, час не минає для них безслідно, і хоч як далекий той момент,
коли стан цих будов потребуватиме перебудувати їх наново, а все ж
мусить він надійти. В усякому разі щодо фінансової та економічної сторони
цей момент може бути дуже віддалений, щоб його брати на увагу
в практичних обчисленнях“ (Lardner, 1. c., 38, 39).

Це має силу до всіх таких споруд вікової тривалости, що в них,
отже, не доводиться поступінно, рівнобіжно з їхнім зношенням, заміщувати
авансований на них капітал, а доводиться переносити на ціну продукту
лише щорічні пересічні витрати на підтримання і ремонт.

Хоча — як ми бачили — більшість грошей, які щороку або навіть
через коротший час повертаються на заміщення зношуваного основного
капіталу, знову перетворюються на натуральну форму цього капіталу,
проте, кожному поодинокому капіталістові потрібен фонд амортизації для тієї
частини основного капіталу, що для неї лише по багатьох роках надходить час
репродукції, і її треба тоді цілком заміщувати. Значна складова частина основного
капіталу вже в наслідок своїх властивостей виключає часткову репродукцію.
Крім того, там, де частинна репродукція відбувається таким способом, що
через короткі переміжки до зневартненого складу додається нозий, то, щоб це
заміщення було можливе, потрібне попереднє грошове нагромадження
в більших або менших розмірах, залежно від специфічного характеру
даної галузі продукції. Для цього досить не якої завгодно суми грошей,
а грошової суми певних розмірів.

Коли ми розглянемо що справу, припускаючи лише просту грошову
циркуляцію, лишаючи цілком осторонь кредитову систему, що про неї
мова буде далі, то механізм руху такий: в першій книзі (розділ II, 3 а)
показано, що коли одна частина наявних у суспільстві грошей завжди
лежить без діла як скарб, а друга функціонує як засіб циркуляції, зглядно
як безпосередній резервний фонд для грошей, що безпосередньо циркулюють,
то постійно змінюється пропорція, що в ній уся маса грошей
розподіляється на скарб і на засоби циркуляції. В нашому прикладі
гроші — що їх досить великий капіталіст повинен нагромадити як скарб
чималих розмірів, — підчас закупу основного капіталу разом пускається
в циркуляцію. Потім вони знову сами собою розпадаються в суспільстві
на засоби циркуляції та скарб. За допомогою амортизаційного фонду,
куди, як до свого вихідного пункту, повертається вартість основного капіталу
в міру його зношування, частина грошей, що циркулюють, знову
утворює скарб — на більший або менший час — в руках того самого капіталіста,
що від нього підчас закупу основного капіталу віддалився його
скарб, перетворившись на засіб циркуляції. Отже, ми маємо повсякчас
змінний розподіл наявного в суспільстві скарбу, що по черзі функціонує
\parbreak{}  %% абзац продовжується на наступній сторінці

\parcont{}  %% абзац починається на попередній сторінці
\index{iii1}{0128}  %% посилання на сторінку оригінального видання
400 фунтів стерлінгів. Далі, через те що сталий капітал вартістю в 2000 фунтів стерлінгів потребує
для свого функціонування 500 робітників, то 400 робітників можуть привести в рух тільки сталий
капітал вартістю в 1600 фунтів стерлінгів. Отже,
для того, щоб виробництво і далі провадилося в попередніх
розмірах і щоб \sfrac{1}{5} машин не стояла без діла, змінний капітал
мусить бути підвищений на 100 фунтів стерлінгів, щоб, як і раніш, вживати 500 робітників; а цього
можна досягти тільки за
допомогою того, що вільний досі капітал зв’язується, при чому
та частина нагромадження, яка повинна була б служити для розширення виробництва, тепер служить
тільки для поповнення, або ж
до попереднього капіталу додається та частина, яка призначена
була для витрачання як дохід. Із збільшеною на 100 фунтів
стерлінгів витратою змінного капіталу тепер виробляється на
100 фунтів стерлінгів менше додаткової вартості. Щоб привести
в рух те саме число робітників, потрібно більше капіталу, і разом
з тим зменшується додаткова вартість, яку дає кожний окремий робітник.

Вигоди, які випливають із звільнення, і втрати, які випливають із зв’язування змінного капіталу,
існують тільки для капіталу, який уже вкладений і який, отже, репродукується при даних відношеннях.
Для нововкладуваного капіталу вигоди на
одному боці, втрати на другому зводяться до підвищення або
зниження норми додаткової вартості і відповідної, хоч і зовсім
не пропорціональної, зміни норми зиску.

\pfbreak

Щойно досліджене звільнення і зв’язування змінного капіталу є наслідок зниження вартості або
підвищення вартості
елементів змінного капіталу, тобто витрат репродукції робочої
сили. Але змінний капітал може звільнятися й тоді, коли внаслідок розвитку продуктивної сили, при
незмінній нормі заробітної плати, потрібно менше робітників для того, щоб привести
в рух ту саму масу сталого капіталу. Так само, навпаки, зв’язування додаткового змінного капіталу
може мати місце, якщо
в наслідок зниження продуктивної сили праці потрібно більше
робітників для тієї самої маси сталого капіталу. Якщо ж, з другого боку, частина капіталу, який
раніш застосовувався як змінний капітал, застосовується тепер у формі сталого капіталу, отже, якщо
відбувається тільки зміна розподілу між складовими
частинами того самого капіталу, то, хоч це і справляє вплив
на норму додаткової вартості й зиску, але не належить до розглядуваної тут рубрики зв’язування і
звільнення капіталу.

Сталий капітал, як ми вже бачили, також може зв’язуватись
або звільнятись в наслідок підвищення вартості або зниження
вартості тих елементів, з яких він складається. Залишаючи це
осторонь, зв’язування його можливе (без перетворення будь-якої
частини змінного капіталу в сталий) тільки тоді, коли збільшується
\index{iii1}{0129}  %% посилання на сторінку оригінального видання
продуктивна сила праці, отже, коли та сама маса праці
створює більше продукту і тому приводить в рух більше сталого капіталу. Те саме при певних
обставинах може мати місце
тоді, коли продуктивна сила зменшується, як, наприклад, у землеробстві, так що та сама кількість
праці потребує для створення
того самого продукту більше засобів виробництва, наприклад,
більше насіння або добрива, дренування і т. д. Без знецінення
сталий капітал може звільнятись тоді, коли в наслідок удосконалень, застосовування сил природи і т.
ін. сталий капітал меншої вартості стає спроможним технічно виконувати ту саму службу, яку раніше
виконував капітал вищої вартості.

У книзі II ми бачили, що після того як товари перетворені
в гроші, продані, певна частина цих грошей знову мусить бути
перетворена в речові елементи сталого капіталу і саме в тих
пропорціях, яких вимагає певний технічний характер кожної
даної сфери виробництва. Щодо цього в усіх галузях — залишаючи осторонь заробітну плату, отже,
змінний капітал — найважливішим елементом є сировинний матеріал, включаючи й допоміжні матеріали,
які особливо важать у тих галузях виробництва, в які не входить сировинний матеріал у власному
значенні, як от у копальнях і добувній промисловості взагалі. Та частина ціни, яка мусить замістити
зношування машин, поки
машини ще взагалі здатні функціонувати, входить в обрахунок
більше ідеально; не має особливого значення, коли саме ця
частина буде оплачена й заміщена грішми, сьогодні чи завтра,
чи в якийсь інший період часу обороту капіталу. Інакше стоїть
справа з сировинним матеріалом. Якщо ціна сировинного матеріалу підвищується, то може стати
неможливим, після відрахування заробітної плати, цілком замістити ціну його з вартості товару. Тому
сильні коливання цін викликають перерви, великі
колізії і навіть катастрофи в процесі репродукції. Продукти
землеробства у власному розумінні слова, сировинні матеріали,
які походять з органічної природи, особливо підпадають таким
коливанням вартості в наслідок мінливих врожаїв і т. д. — кредитну систему ми тут ще цілком
залишаємо осторонь. Та сама
кількість праці може тут в наслідок непіддатних контролеві природних умов, сприятливості чи
несприятливості діб року та ін.,
виражатися в дуже різних кількостях споживних вартостей,
і тому певна кількість цих споживних вартостей матиме дуже
різну ціну. Якщо вартість x представлена в 100 фунтах товару $a$, то ціна одного фунта $a = \frac{x}{100}$; якщо
ж вона представлена в 1000 фунтах $а$, то ціна одного фунта $a = \frac{x}{1000}$ і т. д. Такий, отже, є один
елемент цих коливань ціни сировинного матеріалу.
Другий елемент, про який тут згадується тільки для повноти, — бо конкуренція, як і кредитна система,
лежить тут поки що поза
межами нашого розгляду, — є такий: з самої природи речей
рослинні й тваринні речовини, ріст і виробництво яких підлягають
\parbreak{}  %% абзац продовжується на наступній сторінці

\parcont{}  %% абзац починається на попередній сторінці
\index{ii}{0130}  %% посилання на сторінку оригінального видання
і вартостетворчий, і треба буде його замінити. Отже, авансована капітальна
вартість повинна зробити деякий цикл оборотів, в даному разі, приміром, цикл
у десять річних оборотів — і визначається цей цикл часом існування,
а тому й часом репродукції або часом обороту застосованого основного
капіталу.

Отже, в тій самій мірі, в якій з розвитком капіталістичного способу
продукції збільшуються розміри вартости і протяг життя застосовуваного
основного капіталу, — в тій самій мірі розвивається життя промисловости
й промислового капіталу в кожній особливій галузі приміщення в багаторічне
життя, скажімо, пересічно в десятирічне. Якщо, з одного боку,
розвиток основного капіталу подовжує це життя, то, з другого боку,
його скорочують постійні перевороти в засобах продукції, перевороти,
що з розвитком капіталістичної продукції так само набирають дедалі
більшої сили. Відси випливає й зміна засобів продукції та потреба постійно
їх заміщувати, бо вони зазнають морального зношування за довгий
час до того, як фізично доживуть свого віку. Можна припустити, що
для вирішальніших галузей промисловости цей цикл життя становить тепер
пересічно десять років. Однак, тут має значення не певне число. В усякому
разі ясно: цим багаторічним циклом взаємно зв’язаних оборотів, що
в них капітал є зв’язаний своєю основною складовою частиною, дається
матеріяльна основа періодичних криз, при чому підприємство послідовно
переживає періоди послаблення, середньої жвавости, раптового розмаху,
кризи. Правда, періоди, коли капітал вкладається, дуже різні й зовсім не
збігаються один з одним. Проте, криза завжди становить вихідний пункт
для нових великих капіталовкладань. Отже, розглядаючи справу з суспільного
погляду — вона також дає більш або менш нову матеріяльну
основу для наступного циклу оборотів\footnote{
„Міська продукція зв’язана з оборотом, що охоплює кілька днів, а сільська,
навпаки, з оборотом, що охоплює кілька років“. Adam G. Müller: „Die Elemente
der Staatskunst“. Berlin. 1809, II, ct. 178. Таке наївне уявлення романтики
про промисловість і хліборобство.
}.

5) Щодо способу обчислення обороту, дамо слово одному американському
економістові. „В деяких галузях підприємств ввесь авансований
капітал обертається або циркулює кілька разів протягом року; в інших
одна частина обертається більш як один раз на рік, а друга не так швидко.
Капіталіст повинен обчислювати свій зиск, зважаючи на той пересічний
період, що потрібен для цілого його капіталу, щоб перейти через його
руки або обернутись один раз. Припустімо, що людина вклала в певне
підприємство половину свого капіталу на будівлі й машини, що їх відновлюється
один раз на десять років; четверту частину — на знаряддя і т. ін.,
що їх відновлюється раз на два роки, і остання четверта частина, витрачена
на заробітну плату й сировинний матеріял, обертається двічі на рік.
Хай ввесь її капітал буде 50.000 долярів. Тоді її річні витрати будуть
такі:

\parcont{}  %% абзац починається на попередній сторінці
\index{iii1}{0131}  %% посилання на сторінку оригінального видання
його репродукцію, і таким чином відновлюється монополія тих
країн — джерел його постачання, які виробляють при найсприятливіших
умовах, — відновлюється, може, з певними обмеженнями,
але все ж відновлюється. Правда, репродукція сировинних
матеріалів в наслідок даного поштовху відбувається в розширеному
масштабі, особливо в країнах, які в більшій чи меншій мірі
володіють монополією цього виробництва. Але та база, на якій в наслідок
збільшення кількості машин і т. д. відбувається виробництво
і яка тепер після кількох коливань має стати новою нормальною
базою, новим вихідним пунктом, дуже розширилася в наслідок
процесів, що відбувались протягом останнього циклу обороту.
При цьому, однак, в частині другорядних джерел постачання
сировинного матеріалу репродукція, яка щойно збільшилась, знову
значно гальмується. Так, наприклад, з таблиць експорту ясно
видно, як протягом останніх 30 років (до 1865 року) зростало
індійське виробництво бавовни, коли наставала недостача в американському
виробництві, і потім раптом знову починалося
більш-менш тривале скорочення. Протягом часу подорожчання
сировинного матеріалу промислові капіталісти об’єднуються,
утворюють асоціації, щоб регулювати виробництво. Так було,
наприклад, в Манчестері після підвищення цін на бавовну в
1848 році, так само як і в виробництві льону в Ірландії. Але як
тільки безпосередній привід мине і знову суверенно запанує загальний
принцип конкуренції „купувати на найдешевшому ринку“
(замість того, щоб намагатися, як це робили згадані асоціації,
підвищити виробничу здатність відповідних країн — джерел постачання
сировинного матеріалу, незалежно від безпосередньої
ціни даного моменту, по якій ці країни можуть у даний час постачати
продукт), — отже, як тільки знову суверенно запанує
принцип конкуренції, регулювати подання знову полишається
„ціні“. Всяка думка про спільний, рішучий і передбачливий контроль
над виробництвом сировинного матеріалу — контроль,
який загалом і в цілому ніяк несполучний з законами капіталістичного
виробництва і тому завжди лишається благочестивим
побажанням або обмежується винятковими спільними кроками
в моменти великої безпосередньої небезпеки й безпорадності —
поступається місцем вірі в те, що попит і подання взаємно
регулюватимуть одно одне.\footnote{
Після того, як це було написано (1865 р.), конкуренція на світовому ринку
значно посилилася в наслідок швидкого розвитку промисловості в усіх культурних
країнах, 'особливо в Америці і Німеччині. Той факт, що швидко й колосально
зростаючі сучасні продуктивні сили з кожним днем все більше переростають
закони капіталістичного товарообміну, в межах яких вони повинні
рухатись, — цей факт нині все більше й більше проникає навіть у свідомість
самих капіталістів. Це виявляється особливо в двох симптомах. Поперше, в новій
загальній манії охоронних мит, яка відрізняється від старої системи охоронних
мит особливо тим, що вона найбільше захищає якраз товари, придатні до
експорту. Подруге, в картелях (trusts) фабрикантів цілих великих сфер виробництва
для регулювання виробництва і разом з тим цін і зисків. Само собою
зрозуміло, що ці експерименти здійснимі тільки при відносно сприятливій
економічній погоді. Перша ж буря повинна їх зруйнувати і довести, що, хоч
виробництво і потребує регулювання, але, без сумніву, не капіталістичний клас
покликаний здійснити його. Покищо ці картелі мають лиш одну мету —
дбати про те, щоб дрібні капіталісти пожиралися великими ще швидше, ніж
досі. — Ф. Е.
} Суєвірство капіталістів тут таке
грубе, що навіть фабричні інспектори в своїх звітах знов і знов
з приводу цього здивовано розводять руками. Чергування сприятливих
\index{iii1}{0132}  %% посилання на сторінку оригінального видання
і несприятливих років, звичайно, знов таки приводить до
здешевлення сировинного матеріалу. Незалежно від того безпосереднього
впливу, який ця обставина справляє на розширення
попиту, сюди долучається ще як стимул вищезгаданий
вплив на норму зиску. І зазначений вище процес ступневого
випереджання виробництва сировинних матеріалів виробництвом
машин і т. д. повторюється тоді в ширшому масштабі. Дійсне
поліпшення сировинного матеріалу, так щоб він постачався не
тільки в потрібній кількості, але й потрібної якості, наприклад,
бавовна американської якості з Індії, вимагало б тривалого, регулярно
зростаючого і постійного попиту з боку Европи (цілком
залишаючи осторонь ті економічні умови, в які поставлений індійський
виробник на своїй батьківщині). Але при таких умовах
сфера виробництва сировинних матеріалів змінюється тільки
стрибками, то раптом розширюється, то знову дуже скорочується.
Все це, як і дух капіталістичного виробництва взагалі, можна
дуже добре вивчати на бавовняному голоді 1861—1865 років, до
якого долучалася ще й та обставина, що часами зовсім не було
сировинного матеріалу, одного з найістотніших елементів репродукції.
Ціна може підвищуватись навіть і тоді, коли подання
цілком достатнє, але достатнє при тяжчих умовах. Або може
мати місце справжня недостача сировинного матеріалу. Під час
бавовняної кризи спочатку мала місце така недостача сировинного
матеріалу.

Отже, чим більше ми наближаємось в історії виробництва
до безпосередньої сучасності, тим регулярніше ми знаходимо,
особливо у вирішальних галузях виробництва, постійне повторення
чергувань відносного подорожчання і виникаючого з нього
пізнішого знецінення сировинних матеріалів органічного походження.
Ілюстрації до вищесказаного дано в наведених нижче
прикладах, взятих із звітів фабричних інспекторів.

Мораль історії, яку можна здобути також з дослідження землеробства
взагалі, полягає в тому, що капіталістична система
протидіє раціональному землеробству, або що раціональне землеробство
несполучне з капіталістичною системою (хоч ця остання
і сприяє його технічному розвиткові) і потребує або руки самостійно
працюючого дрібного селянина, або контролю асоційованих
виробників.

\pfbreak

Тепер ми наводимо щойно згадані ілюстрації з англійських
фабричних звітів.



\index{iii1}{0133}  %% посилання на сторінку оригінального видання
„Стан справ покращав; але цикл сприятливих і несприятливих
періодів скорочується із збільшенням кількості машин, і
коли при цьому збільшується попит на сировинний матеріал,
то частіше повторюються також і коливання в стані справ...
В даний момент не тільки відновилось довір’я після паніки
1857 року, але й сама паніка, здається, майже цілком забута.
Чи це покращання буде тривалим, чи ні, це в дуже значній
мірі залежить від ціни сировинних матеріалів. Я бачу вже ознаки
того, що в деяких випадках уже досягнуто максимуму, після
якого фабрикація ставатиме все менш зисковною, поки, нарешті,
вона й зовсім перестане давати зиск. Якщо ми візьмемо, наприклад,
зисковні роки у підприємствах чесаної вовни, 1849 і 1850,
то ми побачимо, що ціна англійської чесаної вовни стояла на
рівні 13 пенсів, австралійської — від 14 до 17 пенсів за фунт,
і що на протязі десяти років, з 1841 до 1850, пересічна ціна
англійської вовни ніколи не перевищувала 14 пенсів, а австралійської
— 17 пенсів за фунт. Але на початку нещасливого
1857 року австралійська вовна стояла на рівні 23 пенсів; у грудні,
в найлихіший час паніки, вона впала до 18 пенсів, але на протязі
1858 року знову підвищилась до теперішньої ціни в 21 пенс.
Англійська вовна в 1857 році так само почала з 20 пенсів, у квітні
й вересні вона підвищилась до 21 пенса, в січні 1858 року впала
до 14 пенсів, а потім підвищилась до 17 пенсів, так що тепер
ціна її за фунт на 3 пенси вища, ніж пересічна ціна протягом
наведених 10 років... Це свідчить, на мою думку, про те, що або
банкрутства 1857 року, викликані подібними цінами, забуті, або
що вовни виробляється ледве-ледве стільки, скільки можуть
перепрясти наявні веретена, абож має місце тривале підвищення
ціни тканин... Але в моїй дотеперішній практиці я бачив,
як на протязі неймовірно короткого часу не тільки збільшилась
кількість веретен і ткацьких верстатів, але й швидкість
їх роботи; далі, що майже в тій самій мірі підвищився наш вивіз
вовни до Франції, тимчасом як пересічний вік овець, вирощуваних
як усередині країни, так і за кордоном, стає все нижчим,
бо населення швидко збільшується, і вівчарі бажають якомога
швидше перетворити своїх овець у гроші. Тому я часто переживав
тяжке почуття, коли бачив людей, які, не знаючи цього, вкладали
свою долю і свій капітал у підприємства, успіх яких залежить від
подання такого продукту, який може збільшуватись тільки
згідно з певними органічними законами... Стан попиту й подання
всіх сировинних матеріалів... пояснює, як видно, багато коливань
у бавовняній промисловості, а також стан англійського вовняного
ринку восени 1857 року і викликану ним промислову кризу“\footnote{Само собою зрозуміло, що ми не \emph{пояснюємо}, разом з паном Бекером,
вовняну кризу 1857 року невідповідністю між цінами сировинного матеріалу
і фабрикату. Ця невідповідність сама була тільки симптомом, а криза була
загальною. — \emph{Ф. Е.}}
(\emph{R. Baker} в „Rep. of. Insp. of Fact., Oct. 1858“, стор. 56—61).


\index{i}{0134}  %% посилання на сторінку оригінального видання
Отже, ми бачимо: чи з’являється споживна вартість як сировинний
матеріял, чи як засіб праці або продукт — це геть чисто
залежить від певної її функції у процесі праці, від місця, яке
вона займає в ньому, і зі зміною цього місця змінюються й ті
її визначення.

Тому, увіходячи як засоби продукції в нові процеси праці,
продукти втрачають характер продуктів. Вони функціонують
ще тільки як предметні фактори живої праці. Прядун ставиться
до веретена лише як до засобу, яким він пряде, до льону — лише
як до предмету, що його він пряде. Звичайно, не можна прясти
без матеріялу для прядіння й без веретен. Тому наявність цих
продуктів доводиться припустити вже з самого початку прядіння.
Але для самого цього процесу той факт, що льон і веретена є
продукти минулої праці, не має ніякого значення цілком так
само, як для акту харчування не має ніякого значення той факт,
що хліб є продукт минулої праці рільника, мірошника, пекаря
й~\abbr{т. ін.} Навпаки, якщо засоби продукції й виявляють у процесі
праці свій характер продуктів минулої праці, то лише через їхні
вади. Ніж, що не ріже, пряжа, що раз-у-раз рветься, і~\abbr{т. ін.}
живо нагадують ножівника $А$ і прядуна $В$. На вдалому продукті
не помітно слідів минулої праці, що надала йому його споживних
властивостей.

Машина, що не функціонує в процесі праці, є некорисна.
Окрім того, вона зазнає руйнаційного впливу природного обміну
речовин. Залізо ржавіє, дерево гниє. Пряжа, що її не тчуть або
не плетуть, є зіпсована бавовна. Жива праця мусить охопити
ці речі, з мертвих зробити їх живими, перетворити їх з лише
можливих у дійсні й діяльні споживні вартості. Охоплені вогнем
праці, яка асимілює їх як своє тіло, надхнені в процесі праці на виконання
функцій, що відповідають їхній ідеї і призначенню, вони
хоч і будуть спожиті, але спожиті доцільно, як елементи утворення
нових споживних вартостей, нових продуктів, які здатні
увійти як засоби існування в особисте споживання або як засоби
продукції в новий процес праці.

Отже, коли наявні продукти є не лише результат, але й умови
існування процесу праці, то, з другого боку, вкладання їх у
процес праці, отже, їхній контакт з живою працею, є єдиний
засіб для того, щоб зберегти й реалізувати ці продукти минулої
праці як споживні вартості.

Праця споживає свої речові елементи, свій предмет і свої
засоби, з’їдає їх, отже, це є процес споживання. Це продуктивне
споживання відрізняється тим від особистого споживання, що в
останньому продукти споживаються як засоби існування живого
індивіда, а в першому — як засоби існування праці, робочої
сили індивіда, яка виявляється в діяльності. Отже, продукт особистого
споживання є сам споживач, а результат продуктивного
споживання є відмінний від споживача продукт.

Оскільки засоби й предмет праці сами вже є продукти, праця
споживає продукти, щоб утворювати продукти, або застосовує
\parbreak{}  %% абзац продовжується на наступній сторінці

\parcont{}  %% абзац починається на попередній сторінці
\index{ii}{0135}  %% посилання на сторінку оригінального видання
продуктивного капіталу та їхній вплив на характер обороту. Ба навіть
він одразу наводить, як приклад, купецький капітал у такому питанні,
де йдеться виключно про ріжниці частин продуктивного капіталу
в процесі утворення продукту й вартости — ріжниці, що й собі утворюють
ріжниці в обороті й репродукції капіталу.

Він каже далі: „Капітал, застосовуваний таким способом, не дає своєму
власникові доходу або зиску, поки він лишається в його посіданні або
зберігає ту саму форму“\footnote*{
„The capital employed in this manner yields no revenue or profit to its employer,
while it either remains in his possession or continues in the same shape“.
}. — Капітал, застосовуваний таким способом!
Але ж А.~Сміс каже про капітал, вкладений у сільське господарство або
промисловість, і далі каже нам, що приміщений таким способом капітал
розподіляється на основний та обіговий. Отже, приміщення капіталу цим
способом само собою не може зробити його ні основним, ні обіговим.

Але, може, він хотів сказати, що капітал, застосований для того, щоб
продукувати товари й продавати ці товари з зиском, мусить, по перетворенні
на товари, продаватись і через продаж, поперше, переходити з
власности продавця у власність покупця, а подруге, зміняти свою натуральну
форму товару на грошову форму, і тому капітал не є корисний
для свого власника, поки він лишається в його посіданні або зберігає —
для нього — ту саму форму? Однак тоді справа сходить ось на що: та
сама капітальна вартість, яка раніш функціонувала в формі продуктивного
капіталу, в формі належній до продукційного процесу, функціонує
тепер як товаровий капітал і грошовий капітал, — в формах капіталу, належних
до процесу циркуляції, і тому вона вже не є ні основний, ні поточний
капітал. І це має силу так само для тих елементів вартости, що
долучаються сировинними та допоміжними матеріялами, отже, поточним
капіталом, як і для тих, що долучаються в наслідок зношування засобів
праці, отже, основним капіталом. Таким чином, ми тут ні на крок не
наблизились до висвітлення ріжниці між основним і поточним капіталом.

Далі: „Товари торговця не дають йому жодного доходу або зиску,
поки він не продасть їх за гроші, і гроші так само мало дають йому,
поки він знову не обміняє їх на товари. Його капітал безупинно одходить
від нього в одній формі й повертається до нього в другій і тільки
за допомогою такої циркуляції або послідовних актів обміну може дати
йому будь-який зиск. Тому такі капітали можна назвати у власному значенні
слова обіговими капіталами“\footnote*{
„The goods of the merchant yield him no revenue or profit tilt he sells them
for money, and the money yields him as little till it is again exchanged for goods.
His capital is continually going from him in one shape, and returning to him in
another, and it is only by means of such circulation, or successive exchanges, that
it can yield him any profit. Such capitals, therefore, may very properly be called
circulating capitals“.
}.

Те, що А.~Сміс визначає тут як обіговий капітал, я хочу назвати
капіталом циркуляції (Zirkulationskapital). Це капітал в формі,
належній до процесу циркуляції, капітал, що змінює форму за допомогою
обміну (зміни речовин і зміни власника), отже, товаровий капітал і грошовий
\parbreak{}  %% абзац продовжується на наступній сторінці

\parcont{}  %% абзац починається на попередній сторінці
\index{ii}{0136}  %% посилання на сторінку оригінального видання
капітал протилежно до його форми, належної до процесу продукції,
тобто протилежно до форми продуктивного капіталу. Це не різні
відміни, що на них поділяє промисловий капіталіст свій капітал, а різні
форми, що їх поступінно завжди набирає й скидає та сама авансована
капітальна вартість протягом свого curriculum vitae\footnote*{
Curriculum vitae (лат.) — перебіг життя. — \emph{Ред.}
}. А.~Сміс сплутує це —
роблячи великий крок назад порівняно з фізіократами — з тими відмінностями
форми, що постають у межах циркуляції капітальної вартости, в її кругобігу
через ряд її послідовних форм тоді, коли капітальна вартість перебуває
в формі продуктивного капіталу; і постають вони саме в наслідок
різних способів, що ними різні елементи продуктивного капіталу беруть
участь в процесі утворення вартости й переносять свою вартість на продукт.
Ми розглянемо далі наслідки цього основного сплутування капіталу
продуктивного й капіталу, що перебуває в сфері циркуляції (товарового
капіталу й грошового капіталу), з одного боку, і основного та поточного
капіталу, з другого. Капітальна вартість, авансована на основний капітал,
так само циркулює разом з продуктом, як і вартість, авансована на поточний,
капітал, і через циркуляцію товарового капіталу перша так само перетворюється
на грошовий капітал, як і друга. Ріжниця виникає лише з того,
що вартість, авансована на основний капітал, циркулює частинами, а тому
й мусить вона також частинами, протягом довших або коротших періодів,
заміщуватись, репродукуватися в натуральній формі.

Що А.~Сміс розуміє тут під обіговим капіталом не що інше, як капітал
циркуляції, тобто капітальну вартість в її формах, належних до процесу
циркуляції (товаровий капітал і грошовий капітал), це доводить приклад,
обраний ним особливо невлучно. Він бере як приклад відміну капіталу, що
зовсім не належить до процесу продукції, а існує лише в сфері циркуляції,
складається лише з капіталу циркуляції: він бере купецький капітал.

Як безглуздо починати прикладом, де капітал взагалі фігурує не як
продуктивний капітал, він сам каже про це зараз же далі: „Капітал торговця
складається цілком з обігового капіталу“. („The capital of a merchant
is altogether a circulating capital“). Але ріжниця між обіговим і основним
капіталом постає, як нам далі скажуть, з посутніх ріжниць в середині
самого продуктивного капіталу. З одного боку, А.~Сміс має на
увазі визначену в фізіократів ріжницю, з другого боку, — відмінності форми;
що їх пророблює капітальна вартість у процесі свого кругобігу. І те
й друге сплутує він в одну строкату купу.

Але як може утворюватись зиск в наслідок зміни форми грошей і
товару, в наслідок простого перетворення вартости з однієї з цих форм
на другу, це лишається цілком незрозуміло. Та й не можна зовсім цього
пояснити, бо він починає тут з купецького капіталу, що функціонує
лише в сфері циркуляції. Ми ще повернемось до цього, а покищо послухаймо,
що каже А.~Сміс про основний капітал:

„Подруге, його (капітал) можна застосовувати на поліпшення грунту,
на закуп корисних машин і знарядь праці та подібні речі, що дають
\parbreak{}  %% абзац продовжується на наступній сторінці

\parcont{}  %% абзац починається на попередній сторінці
\index{ii}{0137}  %% посилання на сторінку оригінального видання
дохід або зиск, не змінюючи власника, не циркулюючи далі. Тому такі
капітали можна назвати основними капіталами у власному значенні цього
слова. Різні підприємства потребують поділу вкладеного в них капіталу
на основний та обіговий в дуже різних пропорціях\dots{} Кожен ремісник
або фабрикант мусить деяку частину свого капіталу зв’язати в формі
засобів праці своєї галузі. Ця частина однак іноді дуже мала, іноді дуже
велика\dots{} Куди більша частина капіталу всіх цих ремісників (кравців, шевців,
ткачів) перебуває однак в циркуляції, то як заробітна плата їхніх
робітників, то як ціна їхнього сировинного матеріялу, і її треба оплатити
з зиском в ціні їхніх продуктів“\footnote*{
„Secondly, it (capital) may be employed in the improvement of land, in the
purchase of useful machines and instruments of trade, or in such like things as
yield a revenue or profit without changing masters, or circulating any further. Such
capitals therefore, may very properly be called fixed capitals. Different occupations
require very different proportions between the fixed and circulating capitals employed
in them\dots{} Some part of the capital of every master artificer or manufacturer
must be fixed in the instruments of his trade. This part, however, is very small in
some, and very great in others.. The far greater part of the capital of all such master
artificers however is circulated, either in the wages of their workmen, or in the
price of their materials, and to be repaid with a profit by the price of the work“.
}.

Не кажучи вже про дитяче визначення джерела зиску, хибність і заплутаність
видно ось з чого: для фабриканта-машинобудівника, напр.,
машина є продукт, що циркулює як товаровий капітал, або, кажучи словами
А.~Сміса: „is parted with, changes masters, circulates further“ (відокремлюється,
змінює власника, циркулює далі). Отже, машина згідно з
його власним визначенням була б не основним, а обіговим капіталом. Ця
плутанина знову таки постає тому, що Сміс сплутує ріжницю між основним
і поточним капіталом, яка постає в наслідок неоднакових способів
циркуляції різних елементів продуктивного капіталу, з відмінностями
форми, що їх перебігає той самий капітал, оскільки він функціонує
в продукційному процесі як продуктивний капітал, а в сфері
циркуляції, навпаки, як капітал циркуляції, тобто як товаровий капітал
або як грошовий капітал. Тому ті самі речі, залежно від того місця, що
його вони мають у життьовому процесі капіталу, можуть, за А.~Смісом,
функціонувати і як основний капітал (як засоби праці, елементи продуктивного
капіталу) і як „обіговий“ капітал, товаровий капітал (як продукт,
виштовхнутий з сфери продукції в сферу циркуляції).

Але А.~Сміс сплутує разом з тим самі основи цього розподілу й суперечить
тому, з чого він почав кількома рядками вище цілий свій
дослід. Це саме сталось у реченні: „Є два способи застосувати капітал
так, щоб він давав своєму власникові дохід або зиск“, а саме — застосувати
його або як обіговий або як основний капітал. Тут мають на увазі,
очевидно, різні способи застосування різних і незалежних один від одного
капіталів, як, напр., капітали, що їх можна вкласти або в промисловість,
або в хліборобство. Але далі ми читаємо: „Різні підприємства
потребують поділу вкладеного в них капіталу на основний та обіговий
в дуже різних пропорціях“. Тепер основний та обіговий капітал є вже
\parbreak{}  %% абзац продовжується на наступній сторінці

\input{_0138.tex}

\index{iii1}{0139}  %% посилання на сторінку оригінального видання
\subsubsection{1861—1864 рр. Американська громадянська війна. Cotton Famine [бавовняний
голод]. Найбільший приклад перерви в процесі виробництва в наслідок
недостачі й дорожнечі сировинного матеріалу}

1860 рік. Квітень. „Щодо стану справ, то я радий можливості
повідомити вас, що, не зважаючи на високу ціну сировинних
матеріалів, всі галузі текстильної промисловості, за винятком
шовку, працювали протягом останнього півроку дуже добре...
В деяких бавовняних округах робітників шукали шляхом оголошень,
і робітники йшли туди з Норфолька та інших землеробських
графств... Як видно, в усіх галузях промисловості панує велика недостача
сировинного матеріалу. Тільки... ця недостача тримає нас
у певних межах. В бавовняній промисловості число новозбудованих
фабрик, розширення наявних фабрик і попит на робітників,
мабуть, ніколи ще не досягали такого високого рівня, як
тепер. Скрізь і всюди шукають сировинного матеріалу“ („Rep.
of Insp. of Fact., April 1860“ [стор. 57]).

1860 рік. Жовтень. „Стан справ у бавовняних, шерстяних
і льонопрядільних округах був добрий; в Ірландії він, як кажуть,
вже більше року навіть „дуже добрий“, і був би ще кращий,
коли б не висока ціна на сировинний матеріал. Прядільники
льону, здається, з більшим нетерпінням, ніж будьколи,
чекають відкриття індійських джерел постачання за допомогою
залізниць і відповідного розвитку індійського землеробства, щоб,
нарешті... добитися відповідного їх потребам подання льону“
(„Rep. of Insp. of Fact., Oct. 1860“, стор. 37).

1861 рік. Квітень. „Стан справ у даний момент пригнічений...
деякі бавовняні фабрики працюють неповний час і багато шовкових
фабрик працюють тільки частково. Сировинний матеріал
дорогий. Майже в усіх галузях текстильної промисловості
ціна його вища, ніж та, при якій він міг би бути перероблений
для маси споживачів“ („Rep. of Insp. of Fact., April 1861“, стор. 33).

Тепер виявилось, що в 1860 році в бавовняній промисловості
була перепродукція; наслідки цього давалися взнаки ще протягом
ближчих років. „Потрібно було від двох до трьох років,
поки світовий ринок поглинув перепродукцію 1860 року“ („Rep.
of Insp. of Fact., October 1863“, стор. 127). „Пригнічений стан
ринків бавовняних фабрикатів у Східній Азії, на початку 1860 року,
справив відповідний зворотний вплив на стан справ у Блекберні,
де пересічно 30 000 механічних ткацьких верстатів майже виключно
заняті у виробництві тканин для цього ринку. В наслідок
цього попит на працю був уже тут обмеженим багато місяців
перед тим, як став відчутним вплив бавовняної блокади...
На щастя, це уберегло багатьох фабрикантів від краху. Запаси,
поки їх тримали на складах, підвищились у своїй вартості, і таким
чином уникнуто було того жахливого знецінення, яке
інакше при такій кризі було б неминучим“ („Rep. of Insp. of
Fact., Oct. 1862“, стор. 28, 29 [30]).


\index{iii1}{0140}  %% посилання на сторінку оригінального видання
1861 рік. Жовтень. „Стан справ з деякого часу був дуже
пригнічений... Немає нічого неймовірного, що протягом зимніх
місяців багато фабрик дуже скоротять робочий час. Це, зрештою,
можна було передбачити... цілком незалежно від тих причин,
які припинили наш звичайний довіз бавовни з Америки і наш вивіз,
скорочення робочого часу протягом наступної зими стало б
необхідним в наслідок сильного збільшення виробництва за
останні три роки і в наслідок порушень на індійському й китайському
ринках“ („Rep. of Insp. of Fact., Oct. 1861“, стор. 19).

\subsubsection{Бавовняні відпади. Ост-індська бавовна (Surat). Вплив на заробітну плату
робітників. Поліпшення машин. Заміна бавовни крохмальним борошном і
мінералами. Вплив цього шліхтування крохмальним борошном на робітників.
Прядільники тонких нумерів пряжі. Ошуканство фабрикантів}

„Один фабрикант пише мені таке: „Щодо оцінки споживання
бавовни на одно веретено, то ви, мабуть, не досить берете до
уваги той факт, що коли бавовна дорога, кожен прядільник
звичайної пряжі (скажімо, до № 40, переважно № 12—32) пряде
такі тонкі нумери, які тільки може, тобто він прястиме № 16
замість попереднього № 12 або № 22 замість № 16 і т. д.,
і ткач, який тче з цієї тонкої пряжі, доведе свій ситець до
звичайної ваги, додаючи до нього відповідно більше шліхти.
Цим способом користуються тепер в справді ганебних розмірах.
Я чув з надійного джерела, що є звичайний Shirting [тканина
для сорочок] для експорту вагою в 8 фунтів штука, з яких
2 \sfrac{3}{4} фунти є шліхта. В тканинах інших сортів часто є до 50\%
шліхти, так що фабрикант аж ніяк не бреше, коли вихваляється,
що він багатіє, продаючи фунт своєї тканини дешевше, ніж він
заплатив за пряжу, з якої ця тканина зроблена“ („Rep. of Insp.
of Fact., April 1864“, стор. 27).

„Мені також казали, що ткачі приписують свою підвищену
захворюваність шліхті, яку застосовують для основи, випряденої
з ост-індської бавовни, і яка вже не складається, як раніше,
з чистого борошна. Однак, цей сурогат борошна дає, як кажуть,
ту велику вигоду, що він значно збільшує вагу тканини, так
що з 15 фунтів пряжі стає 20 фунтів тканини“ („Rep, of Insp.
of Fact., Oct. 1863“, стор. 63. Цим сурогатом був перемолотий
тальк, називаний China clay [китайською глиною], або гіпс, називаний
French chalk [французькою крейдою].) — „Заробіток ткачів
(тобто тут робітників) дуже зменшується в наслідок застосовування
сурогатів борошна для шліхтування основи. Ця шліхта
робить пряжу важчою, але також твердою і ламкою. Кожна
нитка основи проходить у ткацькому верстаті через так званий
реміз, міцні нитки якого тримають основу в правильному положенні;
твердо нашліхтована основа спричинює постійні розриви
ниток у ремізі; при кожному розриві ткач втрачає п’ять хвилин
на виправлення; тепер ткачеві доводиться виправляти такі
пошкодження принаймні в 10 разів частіше, ніж раніш, і верстат,
\index{iii1}{0141}  %% посилання на сторінку оригінального видання
розуміється, дає протягом робочих годин настільки ж менше
тканини“ (там же, стор. 42, 43).

„В Аштоні, Стелібріджі, Мослеї, Ольдгемі і т. д. робочий
час скорочений на цілу третину, і з кожним тижнем робочі години
скорочуються ще більше... Одночасно з цим скороченням
робочого часу в багатьох галузях відбувається також зниження
заробітної плати“ (стор. 13). — На початку 1861 року стався
страйк механічних ткачів у деяких частинах Ланкашіра. Деякі
фабриканти заявили про зниження заробітної плати на 5—7 \sfrac{1}{2}\%;
робітники настоювали на тому, щоб рівень заробітної плати лишити
незмінним, а робочий день скоротити: Фабриканти на це не
згодились, і почався страйк. Через місяць робітники мусили поступитися.
Але тепер вони одержали і те і друге: „Крім зниження
заробітної плати, на що робітники кінець-кінцем згодились,
вони на багатьох фабриках працюють тепер неповний час“.
(„Rep. of Insp. of Fact., April 1861“, стор. 23).

1862 рік. Квітень. „Страждання робітників від часу мого
останнього звіту значно збільшились; але ще ніколи в історії
промисловості такі раптові і такі тяжкі страждання не переносилися
з такою мовчазною покірливістю і таким терпеливим
самовладанням“ („Rep. of Insp. of Fact., April 1862“, стор. 10). —
„Відносне число цілком безробітних робітників в даний момент,
здається, не дуже перевищує число безробітних 1848 року,
коли панувала звичайна паніка, яка, однак, була досить значною,
щоб спонукати занепокоєних фабрикантів складати такі
самі статистичні відомості про бавовняну промисловість, які
тепер публікують щотижня... В травні 1848 року з усіх бавовняних
робітників Манчестера 15\% було без роботи, 12\% працювало
неповний час, тоді як понад 70\% працювало повний час. 28 травня
1862 року без роботи було 15\%, 35\% працювало неповний час,
49\% — повний час... В сусідніх місцевостях, наприклад, в Стокпорті,
процент тих, що працюють неповний час, і тих, що зовсім
не працюють, вищий, процент тих, що працюють повний час,
нижчий“, бо тут випрядаються грубіші нумери, ніж у Манчестері
(стор. 16).

1862 рік. Жовтень. „За останніми офіціальними статистичними
даними, в 1861 році в Сполученому Королівстві було 2887 бавовняних
фабрик, з них 2109 в моїй окрузі (Ланкашір і Чешір). Я, звичайно,
знав, що дуже значна частина з цих 2109 фабрик моєї округи
були дрібні підприємства, які вживали небагато робітників. Я
був, однак, дуже здивований, коли виявив, як багато таких підприємств.
В 392, або 19\%, рушійна сила, пара або вода, менша за
10 кінських сил; в 345, або 16\%, між 10 і 20 кінськими силами;
в 1372 — 20 кінських сил і більше... Дуже значна частина цих
дрібних фабрикантів — більше ніж третина загального числа їх —
не дуже давно самі були робітниками; це — люди, які не мають
у своєму розпорядженні капіталу... Центр ваги падає, отже, на
інші \sfrac{2}{3}“ („Rep. of Insp. of Fact., Oct. 1862“, стор. 18, 19).


\index{iii1}{0142}  %% посилання на сторінку оригінального видання
За даними того самого звіту, з бавовняних робітників Ланкашіра
й Чешіра тоді працювали повний час 40 146 робітників,
або 11,3\%, неповний робочий час — 134 767 робітників, або 38\%,
зовсім без роботи було 179 721 робітник, або 50,7\%. Коли виключити
звідси дані про Манчестер і Больтон, де випрядаються
головним чином тонкі нумери, — галузь, що порівняно мало потерпіла
від недостачі бавовни, — то справа виявиться ще несприятливішою, 'а
саме: таких, що працюють повний час — 8,5\%,
неповний час — 38\%, безробітних — 53,5\% (стор. 19, 20).

„Для робітників становить істотну ріжницю, чи переробляють
вони добру чи погану бавовну. В перші місяці року, коли фабриканти
намагались тримати свої фабрики в русі тим, що вживали
всяку бавовну, яку тільки можна було купити по помірних цінах,
багато поганої бавовни потрапило на ті фабрики, де раніше звичайно
застосовували добру; ріжниця в заробітній платі робітників
була така велика, що відбулося багато страйків, бо робітники
при старій відштучній платі тепер не могли вже добути
собі зносного щоденного заробітку... В деяких випадках ріжниця
в наслідок застосовування поганої бавовни становила навіть при
повному робочому часі половину всього заробітку“ (стор. 27).

1863 рік. Квітень. „На протязі цього року зможуть бути заняті
повний час трохи більше половини бавовняних робітників“
(„Rep. of Insp. of Fact., April 1863“, стор. 14).

„Дуже серйозна невигода при застосуванні ост-індської бавовни,
яку тепер фабрики мусять споживати, полягає в тому,
що швидкість машин при цьому мусить бути дуже уповільнена.
Протягом останніх років було вжито всіх заходів для збільшення
цієї швидкості, так щоб ті самі машини виконували більше
роботи. Але зменшена швидкість зачіпає робітника в такій самій
мірі, як фабриканта, бо більшість робітників одержують відштучну
плату — прядільники стільки то за фунт випряденої
пряжі, ткачі стільки то за витканий кусок; і навіть у інших
робітників, які одержують тижневу плату, заробітна плата повинна
знизитися в наслідок зменшення виробництва. На підставі
моїх досліджень... і переданих мені даних про заробіток бавовняних
робітників на протязі цього року... виявляється зменшення
заробітної плати пересічно на 20\%, в деяких випадках
на 50\% порівняно з висотою заробітної плати 1861 року“
(стор. 13). — „Зароблена сума залежить... від того, який матеріал
переробляється... Становище робітників, щодо суми заробленої
плати, тепер (жовтень 1863 року) багато краще, ніж минулого
року в цей час. Машини поліпшено, сировинний матеріал
знають краще, і робітники легше справляються з тими труднощами,
з якими їм доводилося боротись спочатку. Минулої весни
я був у Престоні в одній швацькій школі“ [благодійна установа
для безробітних]; „дві молоді дівчини, які за день перед тим
були послані до ткацької фабрики, де, за заявою фабриканта, вони
могли б заробити 4 шилінга на тиждень, просили, щоб їх знову
\parbreak{}  %% абзац продовжується на наступній сторінці

\input{_0143_0144_0145.tex}
\parcont{}  %% абзац починається на попередній сторінці
\index{i}{0146}  %% посилання на сторінку оригінального видання
упредметненої праці, отже, їх не береться до рахуби й не входять
вони у продукт утворення вартости\footnote{
Це одна з обставин, які удорожчують продукцію, основану на
рабстві. Робітник, як влучно висловлювалися за старовини, відрізняється
тут лише як instrumentum vocale\footnote*{
знаряддя, обдароване мовою. \emph{Ред.}
} від тварини як instrumentum semivocale\footnote*{
знаряддя, обдарованого голосом. \emph{Ред.}
} і від мертвого знаряддя праці як від instrumentum mutum\footnote*{
знаряддя німого. \emph{Ред.}
}. Але сам робітник дає відчути тварині і знаряддю праці, що він їм не
рівня, а що він людина. Збиткуючися з них і con amore\footnote*{
з насолодою. \emph{Ред.}
} руйнуючи їх, він з самозадоволенням переконує себе самого в своїй відмінності
від них. Тому за цього способу продукції вважається за економічний принцип
вживати лише найгрубіших, найтяжчих знарядь праці, бо саме через
цю грубість і незграбність їх важко знівечити. Тому в рабовласницьких
державах, які лежать над Мехіканською затокою, до вибуху громадянської
війни, вживали плугів старокитайської конструкції, що рили землю,
як свиня або кріт, але не робили борозни, не повертали її. Порівн.
J.~С.~Cairns: «The Slave Power», London 1862, p. 46 і далі. У своїй праці
«Sea Bord Slave States» (p. 46, 47) Олмстед оповідає, між іншим, таке:
«Мені тут показували знаряддя, що їх у нас жодна людина із здоровим
розумом ніколи не дала б найманому робітникові, бо вони обтяжали б
його; на мою думку, їхня надзвичайна вага й незграбність збільшують
працю щонайменше на 10\% порівняно з тим знаряддям, що його звичайно
вживають у нас. І я певен, що за недбалого й грубого поводження рабів
із знаряддям праці було б неекономно дати їм легше й не таке грубе знаряддя.
А ті знаряддя, що їх ми з користю завжди даємо нашим робітникам,
не збереглися б жодного дня на хлібних полях Вірґінії, хоч ґрунт там
і легший і не такий кам’янистий, як наш. Так само, коли я спитав, чому
на всіх фармах замість коней вживають мулів, то перший арґумент, звичайно,
найдовідніший, був той, що коні не могли б витримати поводження
з боку негрів; у наслідок такого поводження коні завжди швидко нівечаться
або калічіють, тоді як мули витримують биття і брак харчів, не
зазнаючи від цього жодної матеріяльної шкоди, не перестуджуються й не
хоріють, навіть коли нехтувати ними й обтяжати їх працею. Алеж мені
досить підійти до вікна кімнати, де я пишу, щоб побачити завжди таке
поводження з худобою, за яке північний фармер негайно прогнав би погонича».
(«І am here shown tools that no man in his senses, with us, would
allow a labourer for whom he was paying wages, to be encumbered with:
and the excessive weight and clumsiness of which, I would judge, would
make work at least ten per cent greater than with those ordinarily used
with us. And I am assured that, in the careless and clumsy way they must
be used by the slaves, anything ligther or less rude could not be furnished
them with good economy, and that such tools as we constantly give our
bourers, and find our profit in giving them, would not last out a day in
Virginia cornfield — much lighter and more free from stones though it be
than ours. So, too, when I ask why mules are so universally substituted for
horses on the farm, the first reason given, and confessedly the most conclusive
one, is that horses cannot bear the treatment that they always must get
from negroes; horses are always soon foundered or crippled by them while
mules will bear cudgelling, and lose a meal or two now and then, and not
be materially injured, and they do not take cold or get sick, if neglected
or overworked. But I do not need to go further than to the window of the room
in which I am writing, to see at almost any time, treatment of cattle
that would insure the immediate discharge of the driver by almost any farmer
owning them in the North»).
}.

Ми бачимо, що встановлена вже раніш аналізою товару ріжниця
між працею, оскільки вона утворює споживну вартість, і
\parbreak{}  %% абзац продовжується на наступній сторінці

\parcont{}  %% абзац починається на попередній сторінці
\index{ii}{0147}  %% посилання на сторінку оригінального видання
вони не становлять жодного елементу продуктивного капіталу, хоч яке
буде їхнє остаточне призначення, тобто чи кінець-кінцем входять вони
відповідно до свого призначення (своєї споживної вартости) в сферу особистого,
чи продуктивного споживання. В пункті 2 ці продукти є засоби
харчування, в пункті 4 — всі інші готові продукти, що, отже, знову таки
складаються з готових засобів праці або готових засобів споживання
(інші, ніж засоби харчування, зазначені в пункті 2).

Що А.~Сміс при цьому каже і про торговця, це виявляє його плутанину.
Оскільки продуцент продав торговцеві свій продукт, то цей останній
вже взагалі не становить жодної форми його капіталу. З погляду
суспільства це, правда, все ще товаровий капітал, хоч він перебуває в
інших руках, ніж руки його продуцента; але саме тому, що це капітал
товаровий, він не може бути ні основним, ні обіговим капіталом.

В кожній продукції, що не має на меті безпосередньо задовольняти
власні потреби, продукт мусить циркулювати як товар, тобто його
треба продати не для того, щоб одержати таким чином зиск, а для того,
щоб взагалі продуцент міг існувати. За капіталістичної продукції до
цього долучається та обставина, що під час продажу товару реалізується
й додаткову вартість, що міститься в ньому. Продукт виходить з процесу
продукції як товар, а тому він не є ні основний, ні обіговий елемент
цього процесу.

А проте, А.~Сміс тут сам себе збиває. Всі готові продукти, хоча
яка буде їхня речова форма або їхня споживна вартість, їхній корисний
ефект, є тут товаровий капітал, тобто капітал в формі, належній до процесу
циркуляції. Перебуваючи в цій формі, вони зовсім не становлять
складових частин продуктивного капіталу їхнього власника; це ні в якому
разі не заважає тому, що скоро тільки їх продано, вони \so{стають} в
руках покупця складовими частинами продуктивного капіталу, все одно —
обігового, чи основного. Тут виявляється, що ті самі речі, що деякий
час виступали на ринку як товаровий капітал протилежно до продуктивного,
пізніше, скоро тільки їх взято з ринку, можуть функціонувати або
не функціонувати, як поточна або основна складова частина продуктивного
капіталу.

Продукт бавовнопрядника — пряжа — є товарова форма його капіталу,
товаровий капітал для нього. Пряжа не може функціонувати знову, як
складова частина його продуктивного капіталу, ні як матеріял праці, ні
як засіб праці. Але в руках ткача, що її купив, вона входить в його
продуктивний капітал, як одна з поточних складових частин його. Але
для прядуна пряжа є носій вартости частини його капіталу — так основного,
як і поточного (додаткову вартість ми лишаємо осторонь). Так
машина, як продукт фабриканта машин, є товарова форма його капіталу,
товаровий капітал для нього, і доки вона зберігає цю форму, вона не є
ні поточний, ні основний капітал. Продана одному з фабрикантів, що
вживають її, вона стає основною складовою частиною продуктивного капіталу.
Навіть тоді, коли продукт має таку споживну форму, що почасти
може, як засіб продукції, ввійти знову в той самий процес, що з нього
\parbreak{}  %% абзац продовжується на наступній сторінці

\parcont{}  %% абзац починається на попередній сторінці
\index{iii1}{0148}  %% посилання на сторінку оригінального видання
сприятливого вибору ринку, може бути дуже різна залежно від
більшої чи меншої дешевини сировинного матеріалу, закупівлі
його з більшим чи меншим знанням справи; залежно від того,
наскільки застосовувані машини є продуктивні, доцільні й дешеві;
залежно від більшої чи меншої досконалості загальної
організації різних ступенів процесу виробництва, від того, наскільки
усунено марнування матеріалу, наскільки просто й доцільно
організовано управління й нагляд і т. п. Коротко кажучи,
якщо додаткова вартість для певного змінного капіталу є дана,
то та сама додаткова вартість може виражатися в більшій чи
меншій нормі зиску, отже, може давати більшу чи меншу масу
зиску залежно від особистої ділової спритності самого капіталіста
або його наглядачів і прикажчиків. Припустім, що та сама додаткова
вартість в 1000 фунтів стерлінгів, продукт 1000 фунтів
стерлінгів заробітної плати, в підприємстві $A$ припадає на
9000 фунтів стерлінгів, а в іншому підприємстві $В$ — на 11000
фунтів стерлінгів сталого капіталу. У випадку $А$ ми маємо
$р' = \frac{1000}{10000} = 10\%$. У випадку $В$ ми маємо $р' = \frac{1000}{12000} = 8\sfrac{1}{3}\%$.
Весь капітал виробляє в $А$ порівняно більше зиску, ніж у $В$, бо
там норма зиску вища, ніж тут, хоч в обох випадках авансований
змінний капітал = 1000 і здобута з нього додаткова
вартість також = 1000, отже в обох випадках має місце однакова
експлуатація однакового числа робітників. Ця ріжниця
виразу однієї і тієї ж маси додаткової вартості, або ріжниця
норм зиску, а тому й самих зисків, при однаковій експлуатації
праці, може походити і з інших джерел; але вона може також
походити цілком і виключно з ріжниці в діловій вправності, з
якою ведуться обидва підприємства. І ця обставина приводить
капіталіста до ілюзії — переконує його, — що його зиск завдячує
своє існування не експлуатації праці, а, принаймні почасти і
іншим, від цієї експлуатації праці незалежним, обставинам, особливо
його індивідуальній діяльності.

\pfbreak

З викладеного в цьому першому відділі видно помилковість
того погляду (Родбертуса), згідно з яким (відмінно від
земельної ренти, де, наприклад, площа землі лишається незмінною,
в той час як рента зростає) зміна величини капіталу не впливає
на відношення між зиском і капіталом, а тому й на норму
зиску, бо тоді, коли зростає маса зиску, зростає і маса капіталу,
на яку цей зиск обчислюється, і навпаки.

Це правильно тільки в двох випадках. Поперше, тоді, коли
при незмінності всіх інших умов, отже, особливо норми додаткової
вартості, настає зміна вартості товару, який є грошовим
товаром. (Те саме має місце при самій тільки номінальній
зміні вартості, підвищенні чи падінні знаків вартості при інших
однакових умовах). Припустім, що весь капітал = 100 фунтам
стерлінгів, зиск = 20 фунтам стерлінгів, отже, норма зиску = 20\%.
\parbreak{}  %% абзац продовжується на наступній сторінці

\parcont{}  %% абзац починається на попередній сторінці
\index{ii}{0149}  %% посилання на сторінку оригінального видання
товарів, що їх робітник купує на свою заробітну плату, в формі засобів
існування. В цій формі капітальна вартість, витрачена на заробітну плату,
належить, за Смісом, до обігового капіталу. Але те, що вводиться в
продукційний процес, є робоча сила, є сам робітник, а не засоби існування,
що ними робітник підтримує своє життя. А проте, ми бачили
(кн. І, розд. XXI), що, з суспільного погляду, репродукція самого робітника
його особистим споживанням теж належить до процесу репродукції
суспільного капіталу. Але цього не можна сказати про поодинокий,
замкнений в собі продукційний процес, що його ми досліджуємо тут.
Надбані й корисні вмілості, acquired and useful abilities (ст. 187), що їх
Сміс подає під рубрикою основного капіталу, в дійсності становлять
складові частини поточного капіталу, оскільки це є abilities найманого
робітника й оскільки робітник продає свою працю разом з своїми abilities.

Велика помилка Сміса в тому, що він усе суспільне багатство поділяє
на: 1) фонд безпосереднього споживання, 2) основний капітал,
3) обіговий капітал. Згідно з цим усе багатство треба було б розподілити
на 1) фонд споживання, що зовсім не становить частини діющого
суспільного капіталу, хоч поодинокі частини його завжди можуть функціонувати
як капітал, і 2) капітал. Одна частина багатства функціонує
таким чином як капітал, друга частина — як некапітал або як фонд
споживання. І для всякого капіталу тут виявляється неминучість бути
або основним або поточним, так само, як кожен з ссавців неминуче мусить
бути або самцем або самицею. Ми однак бачили, що протилежність
між основним і обіговим капіталом має силу тільки для елементів \so{продуктивного
капіталу}, і що, значить, поряд них є ще дуже значна
маса капіталу — товаровий капітал і грошовий капітал — яка перебуває в
такій формі, що в ній \so{не може} вона бути ні основним, ні поточним
капіталом.

А що за винятком тієї частини продуктів, яку — в натуральній формі —
безпосередньо, без продажу й купівлі, знову вживають самі поодинокі
капіталістичні продуценти як засоби продукції, вся маса продуктів суспільної
продукції — на капіталістичній основі — циркулює на ринку як
товаровий капітал, то очевидно, що з товарового капіталу треба вилучити
так основні й поточні елементи продуктивного капіталу, як і всі
елементи споживного фонду. Фактично це значить, що засоби продукції,
як і засоби споживання, на основі капіталістичної продукції виступають
спочатку як товаровий капітал, хоч вони й мали б призначення в
дальшому служити як засіб споживання або як засоби продукції; так само
навіть робоча сила перебуває на ринку як товар, хоч і не як товаровий
капітал.

Відси така нова плутанина в А.~Сміса. Він каже:

„З цих чотирьох частин (cilculating капіталу, тобто капіталу в його
належних до процесу циркуляції формах товарового капіталу й грошового
капіталу — дві частини, що перетворюються на чотири частини через
те, що Сміс відрізняє складові частини товарового капіталу знову на
основі речових ознак) три: харчові засоби, матеріяли та готові вироби,
\parbreak{}  %% абзац продовжується на наступній сторінці

\parcont{}  %% абзац починається на попередній сторінці
\index{ii}{0150}  %% посилання на сторінку оригінального видання
щороку, або в інші, більш-менш короткі періоди реґулярно вилучається
з нього й приміщується або в основний капітал, або в фонд, призначений
для безпосереднього споживання. Кожний основний капітал первісно
постав з обігового й йому потрібна повсякчасна підтримка від цього
останнього. Всі корисні машини та знаряддя праці первісно постали з
обігового капіталу, який дав матеріяли, що з них їх зроблено, і утримання
робітникам, що їх зробили. Вони потребують також, щоб капітал,
зазначеного виду, підтримував їх завжди справними“\footnote*{
Of these four parts three — provisions, materials, and finished work, are either
annually or in a longer or shorter period, regularly withdrawn from it, and placed
either in the fixed capital, or in the stock reserved for immediate consumption.
Every fixed capital is both originally derived from, and requires to be continually
supported by a circulating capital. All useful machines and instruments of trade are
originally derived from a circulating capital, which furnishes the materials of which
they are made and the maintenance of the workmen who make them. They require,
too, a capital of the same kind to keep them in constant repair“ (p. 188).
}.

З винятком частини продукту, що її продуценти завжди безпосередньо
знову зуживають як засоби продукції, для капіталістичної продукції
має силу таке загальне правило: всі продукти подається як товари на ринок,
вони циркулюють для капіталіста як товарова форма його капіталу, як товаровий
капітал незалежно від того, чи мусять і чи можуть ці продукти
своєю натуральною формою, своєю споживною вартістю, функціонувати як
елементи продуктивного капіталу (продукційного процесу), тобто як засоби
продукції, а тому і як основні або поточні елементи продуктивного капіталу,
або чи можуть вони служити лише як засоби особистого, а не продуктивного
споживання. Всі продукти як товари подається на ринок;
тому всі засоби продукції та споживання, всі елементи продуктивного та
особистого споживання треба знову вилучити з ринку купівлею. Ця тривіяльність
(truism), звичайно, правильна. Отже, це однаково має силу й
щодо основних і щодо поточних елементів продуктивного капіталу, і
для засобів праці і для матеріялів праці в усіх їхніх формах. (При цьому
ще забувають, що елементи продуктивного капіталу дані самою природою,
отже, вони не є продукти). Машину купується на ринку так само,
як і бавовну. Але відси ні в якому разі не випливає, що кожний основний
капітал первісно походить із поточного капіталу — це випливає лише
з Смісового сплутування капіталу циркуляції з обіговим або поточним
капіталом, тобто неосновним капіталом. І до того ж Сміс сам себе збиває.
Машини як товар, за його словами, становлять частину зазначеного
в пункті 4 обігового капіталу. Їхнє походження з обігового капіталу значить,
отже, лише те, що вони функціонували як товаровий капітал раніш,
ніж почали функціонувати як машини, але — що речово вони походять
з самих себе; цілком так само, як бавовна, як поточний елемент капіталу
прядуна, походить з бавовни, що циркулювала на ринку. Але коли
Сміс в дальшому викладі висновує основний капітал з обігового на
тій підставі, що для машинобудівництва потрібні праця й сировинні матеріяли,
то, поперше, для цього потрібні також засоби праці, тобто основний
капітал і, подруге, щоб виготувати сировинні матеріяли, теж
\parbreak{}  %% абзац продовжується на наступній сторінці


  \cleardoublepage
  \defaultfontfeatures{ 
  Path = fonts/ ,
  Scale=MatchLowercase,
}

% 
% Text fonts
%

\setmainfont{Alegreya}[
  Extension=.otf,
  ItalicFont=*-Italic,
  BoldFont=*-Bold,
  BoldItalicFont=*-BoldItalic,
  BoldFeatures={SmallCapsFont=*SC-Bold},
  SmallCapsFont=*SC-Regular,
  SmallCapsFeatures={%
    LetterSpace=10,
    WordSpace=2.5
  },
  SlantedFont = Alegreya,
  SlantedFeatures={FakeSlant=0.13},
  Scale = 0.93925
]
\setsansfont{AlegreyaSans}[
  Extension=.otf,
  UprightFont=*-Regular,
  ItalicFont=*-Italic,
  BoldFont=*-Bold,
  BoldFeatures={SmallCapsFont=*SC-Bold},
  BoldItalicFont=*-BoldItalic,
  SmallCapsFont=*SC-Regular,
  SmallCapsFeatures={%
    LetterSpace=10,
    WordSpace=3
  },
]
\newfontfamily{\letterspacefont}{AlegreyaSans}[
  Extension=.otf,
  UprightFont=*-ExtraBold,
  ItalicFont=*-Italic,
  BoldFont=*-Black,
  BoldItalicFont=*-BoldItalic,
  LetterSpace=15,
  WordSpace=4
]
\newfontfamily{\tablefont}{AlegreyaSans}[
  Extension=.otf,
  UprightFont=*-Regular,
  ItalicFont=*-Italic,
  BoldFont=*-Bold,
  BoldItalicFont=*-BoldItalic,
  Numbers={Monospaced,Lining}
]
\newfontfamily{\greekfont}{Alegreya}[
  Script=Latin,
  Extension=.otf,
]

% 
% Math fonts
%

\usepackage{unicode-math}

\setmathfont{STIX2Math}[% operators
  Extension = .otf ,
  StylisticSet = 01 ,
  Scale=MatchLowercase,
]

\setmathfont{Alegreya.otf}[% numbers
  range = {up},
  Script=Latin,
  script-features={},
  sscript-features={}
]

\setmathfont{Alegreya-Italic.otf}[% italic letters
  range = {it},
  Script=Latin,
  script-features={},
  sscript-features={}
]

%% Alllow cyrilic letters in math
\DeclareSymbolFont{cyrletters}{\encodingdefault}{\familydefault}{m}{it}
%% All letters
\newcommand{\makecyrmathletter}[1]{%
  \begingroup\lccode`a=#1\lowercase{\endgroup
  \Umathcode`a}="0 \csname symcyrletters\endcsname\space #1
}
\count255="409
\loop\ifnum\count255<"44F
  \advance\count255 by 1
  \makecyrmathletter{\count255}
\repeat

%% Fake slant fot г, д, п, т

\DeclareSymbolFont{cyrletterssl}{\encodingdefault}{\familydefault}{m}{sl}
\newcommand{\makecyrmathlettersl}[1]{%
  \begingroup\lccode`a=#1\lowercase{\endgroup
  \Umathcode`a}="0 \csname symcyrletterssl\endcsname\space #1
}
\makecyrmathlettersl{"433} % г
\makecyrmathlettersl{"434} % д
\makecyrmathlettersl{"43F} % п
\makecyrmathlettersl{"442} % т

\end{document}
