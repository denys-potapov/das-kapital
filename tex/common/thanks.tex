\nonumsection{Андрію Річицькому}{}{}

Це видання стало можливим лише завдяки наполегливій праці Андрія Річицького. Андрій Річицький (справжнє ім'я — Пісоцький Анатолій Андрійович) — видатний український вчений, перекладач та політичний діяч. В 1920-х роках він почав працювати над першим і єдиним українським перекладом з~німецької Марксового «Капіталу». Встиг завершити переклад першого тому. Репресований в 1934 році, реабілітований в 1990 році. 

\nonumsection{Подяки}{}{}
Над виданням в рамках спільної ініціативи маркс.укр працювали:
\begin{itemize}[nosep]
\item \textbf{Ірина Зробок}
\item \textbf{Ернест Гук}
\item \textbf{Антон Потапов}
\item Юрий Латыш
\item Taras Bilous
\item Ivanna Kutsil
\item Max Starchevsky
\item Dan Bogynski
\item Dmytro Zhelaha
\item Богдан Бернадський
\end{itemize}
\noindent{}а також Nina Garbo, Андрій Андросович, Liza Walther, Сергій Зінченко,
Mike A. Liakh, Stas Sergienko, Volodymyr Boiko, Andriy Panchenkov, Денис Кучеренко, Настя Авдоніна, Oleg Kavaler, Volodymyr Shostak, Ілля Токар, Vika Khomovska, Maxim Sokhatsky, Mariana Potapova, Danylo Yankovskyi, Anna Potasheva, Yaroslav Kovalchuk, Денис Панкратов, Роман Козлов, \textenglish{Yevhenii Mo\-nas\-tyr\-skyi}, Andrew Zukkermann, Олександр Брайко, Anton Potapenko, Oleksandr Lapchuk, Ліда Криштоп, Anton Stepankovsky, Anton Pechenkin, Wowhura Wowhura, Ann Kurovska, Надія Йовченко, Ksena Meyta, Taras Salamaniuk, Сергей Алушкин, Kirill Kramskiy, Eugenia Virlich, Valeriy Kuropyatnik, Наталка Чех, Artem Tidva, Snizhana Umanets, Liuba Kuibida, Andriy Pogasiy, Yulia Dukach, Дмитро Вершинін, Леонид Бегунов-Новиков, Галина Новосад, Artem Borysov, Oleksandr Nykolyak, Петро Садовий, Mikhail Khokhlovych, Oleksii Parfeniuk, Hlafira Titarenko, Володимир Гунько, Денис Яшный, Картина Мира, Валентин Германович Дупак, Classic Starr, Olena Martynchuk, Zakhar Popovych, Polina Vlasenko, Vasya Opechenik та Денис Потапов.

\smallskip
\noindent{}Особлива подяка \textbf{Миколі Климчуку} за те,
що видання вийшло охайним і гармонійним. 

\cleardoublepage