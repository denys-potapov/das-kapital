\parcont{}
\index{franko}{0080}
насилу обрабованих з ґрунту, хат і майна, насилу пороблених
злодіями та волоцюгами, в ті тверді рами карности,
конечної при сістемі наємної праці.

\subsection{Устави для знищеня робучої плати}

Не досить того, що знадоби продукції розділюются:
на однім боці сам капітал (в руках властивців богатирів),
а на другім боці сама праця, т. є. люде, котрі нічо не мают
на продаж крім своєї праці. Не досить ще присилувати
тих людей до того, щоб добровільно себе самих запродували.
В дальшім ході капіталістичної продукції виростає
вже верства робітників, котра з вихованя, традиції, привички
признає вимоги того способу продукованя природними законами,
чимось таким, що й бути інакше не може. Впорядкованє
видосконаленого капіталістичного процесу продукційного
перемагає всі запори; ненастанне повставанє релятівного
перелюдненя\footnote*{
Звісно, що перелюдненєм звеся то, коли денебудь є забагато
людей, т. є. властиво більш людей, ніж може вижити. А релятівне перелюдненє
значит, що тілько в певнім місци і серед певних обставин є для
певного діла забогато людей, так, що всі вони не можут приміститися,
і одна часть з них дармує. Кождий пійме, що вже сама проява такого
релятівного перелюдненя є знаком нездорових економічних обставин.
Між тим, як побачимо далі, ціла капіталістична продукція нерозлучно
звязана с релятівним перелюдненєм, котре змоглося в краях промислових
особливо від заведеня парових машин, через що мілійони рук робітницьких
стратили роботу (\emph{Прим. перев.}).
} вдержує довіз робучих рук і попит
за працею, значит, і робучу плату на такій висоті, яка кориснійша
для підростаючого капіталу; німий примус економічних
обставин довершує панованя капіталіста над робітником.
Позаекономічна, беспосередна сила входит все ще
в уживанє, але вже лиш виїмково. При звичайнім ході діла
досить є — лишити робітника під властю „природних законів
продукції“, т. є. лишити го в залежности від капіталу,
витвореній і навіки забеспеченій самими вимінками
продукційними. Але сего не мож зробити в тій історичній
хвили, коли капітал і етична продукція інощо зароджуєсь.
Підростаюча буржоазія потребує і уживає власти державної,
щоб „реґулювати“ робучу плату, т. є. втискати єї в такі
границі, які найкориснійші для баришництва, продовжувати
день робучий і вдержувати самого робітника в „належитій“
степени залежности. Се також дуже важний причинок до
т. зв. первісного нагромадженя капіталу.

Верства наємних робітників, що повстала в послідній
половині \RNum{14} віку, становила тоді і в слідуючих столітях
тілько дуже незначну часть людности, котрої становище
\parbreak{}
