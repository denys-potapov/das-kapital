\parcont{}
\index{franko}{0081}
притім міцно обезпечували самостійні ґаздівства по селах,
а цехові звязки по містах. По селах і містах не було великої
суспільної ріжниці між майстрами а робітниками.
Підчиненє праці під капітал було тілько формальне, т. є.
продукція сама не мала ще на собі окремої капіталістичної
ціхи. Попит за наємною працею змагався прото дуже швидко
за кождим нагромадженєм капіталу, — між тим рук готових
найматися до праці прибувало дуже поволи. Велика
часть витворів суспільних, що пізнійше стала фондом вбільшуючим
капітал, тоді переходила ще в руки робітника для
єго власного зужитку.

Праводавство про наємну працю, згори вже вицілене
на визискуванє робітника і в своїм розвитку йому завсігди
однаково неприхильне, почалося в Англії від виданя „Устави
робітницької“ (Statute of Labourers) Едвардом III, 1349.
Рівночасно видано в Франції Указ 1350 р. в імени короля
Жана. Англійські і французькі устави виходят рівнобіжно
і зовсім однакі що до змісту.

Устава робітницька зістала видана за про голосні наріканя
послів. „Давнійше“, каже наївно оден Торі, „жадали
бідні такої великої плати за роботу, що промисл і богацтво
були загрожені. Тепер плата така низька, що знов грозит
промисловії й богацтву і то може ще небеспечнійше ніж
тоді“. Установлено правну тарифу платну для міст і сіл,
за роботу (в рукоп. „робуту“) на дни й від штуки. Сільскі робітники
повинні винайматися на рік, міські „с прилюдного
торгу“. Під карою тюрми заборонено платити висшу плату
від означеної в уставі; а хто бере більшу плату, того кара
виносит більше, ніж сама плата. Так само ще в розд. 18
і 19 устави о учениках ремісницьких, виданої за Єлисавети,
грозится карою 10 день тюрми тому, хто платит більше,
а 21 день тюрми тому, хто бере більшу плату від правом
приписаної. Устава з р. 1360 заострила кари і навіть дала
майстрам право силувати робітників мусом до праці за таку
плату, яка означена в тарифі. Всякі звязки, угоди, присяги
і т. д., котрими взаїмно сполучилися теслі з мулярами,
узнані неважними. Стоваришеня робітницькі караются як
тяжка провина від \RNum{14} віку до 1825, в котрім скасовано
устави протів стоваришень. Дух „Робітницької устави“ з р.
1349 і єї потомків просвічує ясно й с тих устав протів стоваришень.
Се тота сама засада: держава приписує, кілько
мож найбільше платити робітникови, але хрань боже, щоб
хоть натякнула на те, кілько мож йому найменьше платити!

В \RNum{16} віці, як звісно, положінє робітників дуже погіршилося.
Правда, грішми плачено більше, тількож що ціна
грошей стала меньша, а ціна товарів без міри більша. На
ділі затим і плата вменьшилася. А прецінь устави для єї
зниженя трівают далі порівно з обрізуванєм вух та пятнованєм
\index{franko}{0082}
тих, „котрих ніхто не хоче взяти на службу. Єлисаветина
5 устава про учеників ремісницьких, уст. 3 надає
мировим судям власть становити де в яких реміслах плату
і змінювати ї відповідно до пори року і ціни товарів. Яков
I ростягнув ту саму реґуляцію робітницької плати на ткачів,
прядільників і на всі можливі розряди робітників\footnote{
З одної примітки до устави 2 за Якова І, розд. 6 видно, що
деякі суконники позваляли собі самі яко мирові судьї урядово діктувати
платну тарифу в своїх варстатах. — В Німеччині, а іменно по 30-літній
війні, виходит богато устав для знижуваня робучої плати. „Помічникам
на безлюдних ґрунтах дуже прикро давалась чути недостача слуг і робітників.
Всім мужикам-ґаздам заказано приймати в комірне мужчин та
женщин вільного стану; про всіх таких комірників повинно доноситися
урядови, а той запирає їх в тюрму, скоро не хотят стати слугами, хоть би
й без того мали яке їнше вдержанє, хоть би працювали у  мужиків за поденщину
або навіть торгували грішми та збіжєм. (Цісарські прівілєї та
ухвали для Шльонська, І, стор. 125). Через цілих сто літ роздаются в приписах
князів та поміщиків раз-відразу гіркі наріканя на злосливих
і здуфалих слуг, що не хотят піддатися важким условинам, не хотят вдоволюватися
платою правом приписаною. Виходят накази, щоб поєдинчпй
поміщик не смів своїм слугам платити більше, ніж кілько весь краєвий
збір покладе в таксу. А прецінь условини служби по війні нераз ще
бувают ліпші, ніж були 100 літ опісля. В р. 1052 діставали ще слуги на
Шльонську по два рази до тижня мясо; а ще в нашім столітю іменно
там були такі округи, де слуги діставали мясо хіба три рази до року.
І поденщина (плата за день роботи) по 30-літній війні була більша ніж
в слідуючих столітях“ (Ґустав Фрейтаґ).
}, Джордж
II ростягнув устави протів робітницьких товариств на всі
мануфактури. В властивій порі мануфактуровій капіталістична
продукція була вже досить сильною, щоб правну
реґуляцію робучої плати зробити непотрібною, а то й неможливою,
але все такі ще на всякий злучай не закидувано
того перестарілого оружя. Ще 8 устава Джорджа II заказує
давати кравецьким челядникам в Льондоні і околици більше
понад 2\shil{ шіллінґи} і півосьма пенса денної плати, окрім хіба
в разах загальної жалоби. Ще 13 уст. Джорджа III, розд.
68 повіряє мировим судям реґульованє робучої плати у виробників
шовку. Ще 1796 тре було двох декретів висших
судів для рішеня, чи накази мирових судьїв що до робучої
плати мают вагу і для нерільничих робітників. Ще 1799.
потвердила ухвала парляменту, що плата копальників шотляндських
уреґульована уставою Єлисавети і двома шотляндськими
актами з р. 1661 і 1671. А який між тим переворот
доконався у всіх обставинах, доказала подія нечувана
в англійській палаті панів. Ту, де від звиш 400 літ
фабриковано устави виключно о тім, понад яку міру не
може ніяк переступити робуча плата, — ту поставив 1799
\parbreak{}
