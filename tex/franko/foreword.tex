
% ЧОГО vtu ХОЧЕМО? 
% Вперше без підпису надруковано польськаю мовою в газеті «Ргаса», 1879, 18 серпня, під назвою «Czego my chcemy?». В перекладі українською мовою вперше надруковано у вид.: Франко І. Твор и. В 20-ти т., т, 19, с. 215-217. ПодаЕться за першодруком. 
% ВЛАСНІСТЬ ГРУНТОВА І Уі ІСТОРІЯ 
% Вперше надруковано в кн.: Лавле Е. де. Власність грунтова і її історія. Переклав Іван Франко. Львів, 1879 («Дрібна бібліотека», VI), с. 34. (Передмова до книги). ПодаЕться за першодруком. С. 28. лавеле  Еміль де (1822-1892) — бельгійський бур-жуаsний iсторик i •    економlст. Б ю к е р Карл (1847-1930) — німецький буржуазний еконо• міст, історик народного господарства і статистик. 
% ДОПОВНЕННЯ ДО «ОСНОВ СУСПІЛЬНОУ 
% ЕкономІт" 
% Вперше надруковано в журн. «Культура», 1926, Ns 4-9, с. 56-57, та в кн.: Іван Франко, К., 1926, с. 164-166. Це герша передмова до перекладу XXIV розділу «Капіталу» К. Маркса, доданого І. Франком до написаного ним підручника «Ос• нови суспільної економії», який мав вийти у світ в кінці 1879 — на початку 1880 р., але не був надрукований; рукопис його загуб- лено. ПодаЕться за автографом, який зберігся в архіві І. Франка ((р. 3, Nё 448). 
% ГДРУГА ЛЕРЕДМОВА ДО ПЕРЕКЛАДУ 
% 24-го РОЗДІЛУ ПРАцІ К. МАРКСА «КАП[ТАЛв. т. 1] 
% Вперше надруковано в ж,урн. «Культура», 1926, Ns 4-9, с. 57-58, та в кн.: Іван Франко. К., 1926, с. 167---168. Передмову до українського перекладу 24-го розділу «Капі- талу» К. Маркса, який І. Франко мав намір видати окремим ви- пуском «Дрібної бібліотеки» (див. коментар до перекладу в цьому томі)подаеться написано, ймовірно, на початку 1880 року.  за автографом, який э6ерігаЕться в архіві І.Франка, ф. 3, Ns 448. С. 32. ...щоби сама гграця стала товаром...- Тут франківський виклад змісту першого тому «Капіталу» К. Марк- са неточний. К. Маркс мав на уваэі не працю, а робоцу силу. С. 33. Текст перекладу подаЕться у розділі «3 наукових пере- кладів» (с. 581--609). 
% 616 



\section*{Доповненя до „Основ суспільної економії“\protect\footnotemarkZ{}}
\nonumsectioncft{Доповненя до „Основ суспільної економії“}{.~}{Іван Франко}

\footnotetextZ{Вперше надруковано в журн. «Культура», 1926, № 4--9, с. 56--57, та в кн.: Іван Франко, К., 1926, с.~164--166.

Подається за автографом: відділ рукописних фондів і текстології Інституту літератури ім. Т.~Г.~Шевченка НАН України. — Ф. 3. — Од. зб. 448. — 14 арк. 
}

\noindent{}В самім початку „Основ суспільної економії“ сказано було, що економія, се наука абстрактна, т. є. що ціль єї не є виключно — розслідити закони економічні \emph{теперішної} суспільности, але \emph{загальні} закони праці людської. А позаяк с переміною суспільного ладу в протягу віків і закони ті проявляются щораз то в інших формах, випливаючих конечно з даного ладу, то наука економічна не може ніякої с тих форм вважати сталою і незмінною. Не може, значит, і нинішних форм уважати сталими, а мусит шукати таких форм, котрі \emph{після нашого теперішного знаня} булиб відповіднійші для суспільної праці і суспільного добробутку, ніж нинішні форми.

С тої то причини в сістематичнім викладі основ сусп. економії ми не могли давати надто широкого місця вислідам про \emph{нинішний} лад, а ограничились тілько головним єго нарисом. При викладі абстрактної теорії праці се була конечна річ, — але прецінь ніхто не заперечит, що на практиці для кождого дуже важне — знати передовсім докладно теперішний лад, єго почин і розвиток. Таке знанє вже тим корисне, що замісць теоретичних засад подає масу фактів, котрі самі прут розум до таких а таких виводів, між тим коли ті самі виводи, подані без підставних фактів, усякому можут видатися хиткими та схопленими з воздуха мріями. Для того то думаєм ми, що поповнимо подекуди конечний недостаток теоретичного викладу, подаючи в „Доповнених“ обширнійший огляд деяких питань, не порушених або з боку ткнених в самім викладі.

Одна з найважнійших недостач усякого чисто теоретичного викладу та, що приходится виключати з него всякі ширші \emph{історичні} перегляди. Правда, се не є недостача конечна, бо остаточно мож би бути вірним теорії, подаючи перегляд розвитку та впадку всіх економічних порядків від почину цівілізації аж до тепер. Але не кажучи вже о тім, що для такої загальної історії економічного розвитку призбирано доси дуже ще мало матеріялу, — в нашім підручнику такий виклад був би неможливий вже й за недостачею місця. А говорити обширно про розвиток одного — ниніншого — ладу, не казавши нічо про розвиток їнчих, се значилоб вважати сей лад чимось важнійшим від прочих, між тим коли в історії, як і в зрості кождого орґанізму, кожда фаза розвитку для вислідника рівноважна.

Але вважаючи потрібним познайомити наших читателів з історичним розвитком сучасного, капіталістичного ладу, ми робимо се в „Доповненях“. А для своєї ціли ми не можем найти кращого провідника над Карля Маркса, котрий в однім розділі своєї книжки „Das Kapital“ списав короткий, хоть яркий перегляд того, як розвивалася капіталістична продукція. С тим розділом ми й хочемо познакомити наших читателів.


\section*{[Друга передмова до перекладу 24-го розділу праці К.~Маркса «Капітал», т. І]\protect\footnotemarkZ{}}
\nonumsectioncft{[Друга передмова до перекладу 24-го розділу праці К.~Маркса «Капітал», т. І]}{.~}{Іван Франко}

\footnotetextZ{Вперше надруковано в журн. «Культура», 1926, № 4--9, с. 57--58, та в кн.: Іван Франко, К., 1926, с.~167--168.

Подається за автографом: відділ рукописних фондів і текстології Інституту літератури ім. Т.~Г.~Шевченка НАН України. — Ф. 3. — Од. зб. 448. — 14 арк. 
}

\noindent{}В першій части своєї великої економічної праці про „Капітал“ стараєсь Карль Маркс вияснити передовсім, \emph{як повстає капітал}? В тій ціли виказує він поперед усего, що єдиним жерелом усякої вартости є праця людська, котра з матеріалів сирих, даних природою, і при помочи сил природи витворює предмети вжиточні для чоловіка. Коли предмети такі витворюются не для власного вжитку самого витвірця, а для заміни за їнші, тоді вони звутся товарами. Капіталістична продукція полягає на витворюваню товарів, але не всяка продукція, де витворюются товарі, є вже капіталістична. До того потрібно ще одної дуже важної вимінки: \emph{щоби сама праця стала товаром}, т. є. щоб на торзі за певний товар (гроші) мож було заміняти (купити) працю людську.

Звичайно під назвою капіталу у нас розуміются беззглядно гроші. Се по части хибно. Гроші, як бачимо, тоді тілько стают капіталом, коли за них купуєся на торзі робуча сила.

Але праця людська, се не є звичайний товар. Се товар живий, котрий має тоту властивість, що \emph{надає вартість} другим предметам, і надає єї більше, ніж кілько сам коштує. Торгова ціна праці, так як і ціна кождого товару, означена звичайними економічними правилами, с котрих найважнійше — кошт витвореня товару, т. є. в тім разі — кошт удержаня робітника і єго робучої сили. Таку ціну платит капіталіст робітникови за єго працю. Між тим робітник в тім часі, на котрий нанявся, витворює далеко більше, ніж кілько виносит єго плата. Він витворив \emph{надзвишку вартости} понад вартість своєї плати, — тота надзвишка, се зиск капіталіста, — вона побільшує єго капітал. Значит, уся капіталістична продукція полягає на твореню надзвишки, котра задармо дістаєсь капіталістови. Цілий розвиток економічний капіталістичної продукції полягає на тім, що капіталісти всіми силами старалися до крайної можности вбільшити тоту надвишку. Вбільшити єї мож було двома способами: або продовжуючи день робучий (надвишка абсолютна), або приневолюючи робітників в коротшім часі працювати з більшою натугою (релятівна надвишка). Оба ті способи витрібували капіталісти, і то перший з них (продовженє робучого дня) до такої крайности, що аж уряд, затрівожений робітницькими розрухами, мусів вдатися в те діло і ограничити стало довготу робучого дня. Від тоді капіталістична продукція і доси пре в другий бік, — стараєсь той означений правно день робучий як найдоскональше використати, раз-ураз заводячи нові машини, котрі до крайности упрощуют і прискорюют продукцію, а до обслуги вимагают як найменшого числа рук.

Се головні думки, виведені Марксом з безмірної маси фактів, нагромаджених в єго книжці. При кінци книжки розбирає він ще одно важне питане: Яким способом почалася тота капіталістична продукція? Як і на якім ґрунті та при якій управі виріс той дивний порядок, оснований на щоденнім хитрім визиськуваню, на крайній бідносте незлічимих мас народа, а крайнім богацтві немногих щасливців? Сесь важний розділ Марксової книжки — прекрасний культурно-історичний очерк — зрозумілий буде і окремо від цілої книжки і ми хочемо познакомити з ним нашу громаду, як для самої єго великої стійности наукової, так і для того, щоб заохотити всіх, хто тілько владає німецькою мовою, до читаня цілої Марксової книжки. Звичайно говорится про дуже трудний і незрозумілий спосіб писаня у Маркса. Се мож би сказати хіба про перший розділ єго книжки, — а о?кілько такий суд справедливий що до прочих розділів, най посвідчит тота часть, котра отсе переведена.