\parcont{}  %% абзац починається на попередній сторінці
\index{i}{*0087}  %% посилання на сторінку оригінального видання
й заміною його на інший, вищий. І цю цінність має, дійсно,
книга Марксова».

Змалювавши так влучно те, що він зве моєю справжньою методою,
і так прихильно, оскільки справа йде про моє особисте
застосовування її, що ж таке змалював пан автор, як не діялектичну
методу?

Безперечно, спосіб викладу мусить формально відрізнятися
від способу дослідження. Дослідження має в подробицях засвоїти
матеріял, проаналізувати різні форми його розвитку і простежити
внутрішній їхній зв’язок. Тільки після того, як цю роботу виконано,
можна відповідним чином викласти дійсний рух. Якщо
цього досягнуто й життя матеріялу ідеально відбито, то може здатися,
що ми маємо діло з апріорною конструкцією.

Моя діялектична метода не тільки відрізняється від геґелівської
своєю основою, але й цілком їй протилежна. Для Геґеля процес
мислення, що його він під назвою ідеї навіть перетворює на самостійний
суб’єкт, є деміург\footnote*{
творець. \emph{Ред.}
} дійсности, яка є лише зовнішнє його
виявлення. А в мене, навпаки, ідеальне є не що інше, як матеріяльне,
пересаджене в людську голову й перетворене в ній.

Містифікаторський бік геґелівської діялектики я скритикував майже
30 років тому, в той час, коли вона була ще в моді. Але саме
тоді, коли я розробляв перший том «Капіталу», докучливому,
чванькуватому, дрібноголовому епігонству, що тепер має перше
слово в освіченій Німеччині, подобалося поводитися з Геґелем
так, як, за часів Лессінґа, бравий Мойсей Мендельсон поводився
\parbreak{}  %% абзац продовжується на наступній сторінці
