
\index{i}{*0097}  %% посилання на сторінку оригінального видання
\nonumsection{Передмова до четвертого видання}{.~}{Фрідріх Енґельс}
Четверте видання вимагало від мене по змозі остаточно встановити
так текст, як і примітки. Як я виконав це завдання, про
це скажу коротко ось що.

Порівнявши ще раз французьке видання й рукописні помітки
Маркса, я взяв із нього ще кілька додатків для німецького тексту.
Вони є на стор.~80 (стор.~88 третього видання)
(стор.~\pageref{original-71}, \pageref{original-72} цього
укр. видання), стор.~458--460 (стор. 509--510 третього видання)
(стор.~\pageref{original-414}--\pageref{original-416} цього укр. видання),
стор.~547--551 (стор. 600 третього видання)
(стор.~\pageref{original-496}--\pageref{original-499} цього укр. видання),
стор.~591--593 (стор.~644 третього видання)
(стор.~\pageref{original-538}--\pageref{original-540} цього укр. видання) і на стор.~596 (стор.~648 третього видання)
(стор.~\pageref{footnote-79} цього укр. видання) в \ref{footnote-79} примітці. Так само, за прикладом французького
й англійського видань, я вмістив у тексті (стор.~461--467 четвертого видання)
(стор.~\pageref{original-416}--\pageref{original-422} цього укр. видання)
\index{i}{*0098}  %% посилання на сторінку оригінального видання
довгу примітку про гірничих робітників (стор.~509--515 третього
видання). Інші ж незначні зміни є суто технічного характеру.

Далі я поробив ще деякі пояснювальні додаткові примітки,
а саме там, де, як мені здавалося, цього вимагали змінені історичні
обставини. Всі ці додаткові примітки подано у прямих дужках
і позначено моїми ініціялами.

Цілковита перевірка численних цитат стала доконечною для
англійського видання, що вийшло за цей час. Для цього видання
наймолодша дочка Марксова, Елеонора, взяла на себе працю
порівняти з ориґіналами всі наведені цитати, так що цитати
з англійських джерел, які становлять переважну частину цитат,
там подаються не у зворотному перекладі з німецької мови, а
англійською мовою тексту самого ориґіналу. Отже, я повинен був
узяти до уваги цей текст для четвертого видання. При цьому виявились
деякі маленькі неточності. Неправильні посилання на сторінки,
що сталися почасти підчас переписування із зшитків, а почасти
в наслідок друкарських помилок, які назбиралися протягом
трьох видань. Неправильно поставлено лапки або павзи, як це
неминуче буває за масового цитування із зшитків з витягами.
Місцями вжито не зовсім влучно обране для перекладу слово.
Окремі місця цитовано з давніх паризьких зшитків 1843--1845~\abbr{рр.},
коли Маркс іще не розумів англійської мови й англійських економістів
читав у французькому перекладі; там, де через подвійний
переклад сталась легка зміна у відтінку розуміння цитат, приміром
цитати з Steuart’a, Ure’a та ін., — тепер можна було використати
англійський текст. Такі самі й інші дрібні неточності й недогляди.
Коли ми порівняємо тепер четверте видання з попередніми, то
переконаємося, що ввесь цей морочливий процес перевірки ані
трохи не змінив у книзі нічогісінько такого, що варто було б
зауважити. Лише однієї цитати не можна було знайти, а саме
з Richard’a Jones’a (стор.~562 четвертого видання, примітка \ref{footnote-47})
(стор.~\pageref{footnote-47} цього українського видання); Маркс, мабуть, помилився
в заголовку книги. Всі інші цитати зберігають свою повну
доказову силу або зміцнюють її в теперішній точній формі.

Але тут я мушу повернутися до однієї старої історії.

Мені особисто відомий один лише випадок, коли піддано під
сумнів правильність Марксової цитати. Але що цей сумнів висувалось
і після смерти Маркса, то я не можу його тут поминути.

В берлінській «Concordia», органі спілки німецьких фабрикантів,
з’явилася 7 березня 1872~\abbr{р.} анонімна стаття: «Як цитує
Карл Маркс». Тут із превеликим витраченням морального обурення
й непарляментськими висловами твердиться, нібито цитату
з Ґледстонової бюджетової промови від 16 квітня 1863~\abbr{р.}
(вміщену у відозві з приводу заснування (Inauguraladresse) інтернаціональної
асоціяції робітників у 1864~\abbr{р.} і повторену в першому
томі «Капіталу», стор.~617 четвертого видання, стор.~671 третього
видання, стор.~\pageref{footnote-105} цього українського видання) підроблено.
У стенографічному (quasi-офіціальному) звіті Hansard’a,
мовляв, немає ані словечка з речення: «Це приголомшливе збільшення
\index{i}{*0099}  %% посилання на сторінку оригінального видання
багатства й сили\dots{} обмежується цілком і виключно на
заможних клясах». «Але цього речення немає ніде в Ґледстоновій
промові. В ній сказано прямо протилежне. [Жирним
шрифтом] \emph{Маркс формально й матеріяльно прибрехав це речення}!»

Маркс, якому надіслано було це число «Concordia» в наступному
травні, відповів анонімові в «Volksstaat’i» з 1 червня. Через
те, що він уже не пригадував собі, з якого газетного звіту він цитував,
він обмежився на тому, що покликавсь насамперед на рівнозначну
цитату з двох англійських творів і, крім того, процитував
звіт «Times’а», що згідно з ним Ґледстон каже: «Такий є
стан нашої країни з погляду багатства. Щодо мене, то я мушу
признатися, що я майже з занепокоєнням і жалем поглядав би на
це приголомшливе збільшення багатства й сили, коли був би певний,
що воно обмежується на дійсно заможних клясах. Воно
взагалі не стосується до становища робітничої людности. Цей
зріст багатства, який я описую тут на підставі даних, на мою
думку, цілком точних, обмежується виключно на заможних
клясах»\footnote{«That is the state of the case as regards the wealth of this country.
I must say for one, I should look almost with apprehension and with pain
upon this intoxicating augmentation of wealth and power, if it were my
belief that it was confined to classes who are in easy circumstances. This
takes no cognizance at alle of the condition of the labouring population.
The augmentation I have described and which is founded, I think, upon
accurate returns, is an augmentation entirely confined to classes of property».}.

Отож Ґледстон каже тут, що йому, мовляв, було б жалко,
коли б так було, але це \emph{справді} так: це приголомшливе збільшення
сили й багатства обмежується цілком і виключно на заможних
клясах. А щодо quasi-офіційного «Hansard’a», то Маркс каже
далі: «Ґледстон був такий мудрий, що в своєму пізніше виправленому
виданні попідчищував тут місця, справді компромітовні
в устах англійського канцлера скарбу; а втім, це традиційна
англійська парляментська звичка, а зовсім не винахід Ляскера
проти Бебеля».

Анонім дедалі більше сердиться. У своїй відповіді в «Concordia»
4 липня він, відкидаючи джерела з другої руки, соромливо
зазначає про «звичай» цитувати парляментські промови з стенографічних
звітів; алеж, мовляв, і звіт «Times’a» (де є це «прибріхане»
речення) і звіт «Hansard’a» (де його немає) «матеріяльно
цілком збігаються»; у звіті «Times’a» так само міститься «прямо
протилежне до того славетного місця з inaugural-адреси»; при
цьому цей добродій старанно замовчує, що звіт «Times’a» поруч
із цією вигаданою «протилежністю» виразно містить саме «те
славетне місце»! А все ж анонім почуває, що він уклепався, і що
його може врятувати хіба лише новий викрут. Отже, переповнюючи
свою статтю, що, як це тільки но доведено, аж рябіє від
«нахабної брехливости», навчальними лайками — такими, як
«mala fides», «безчесність», «брехливі дані», «та брехлива цитата»,
«нахабна брехливість», «цитата, цілковито підроблена»,
\parbreak{}  %% абзац продовжується на наступній сторінці
