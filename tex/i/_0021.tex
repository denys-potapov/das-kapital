\parcont{}  %% абзац починається на попередній сторінці
\index{i}{0021}  %% посилання на сторінку оригінального видання
вони у виразі вартості вбачали лише кількісне відношення. В дійсності ж еквівалентна форма товару не
має жодного кількісного визначення вартости.

Перша особливість, що впадає на очі при розгляді еквівалентної форми, ось яка: споживна вартість
стає формою виявлення своєї протилежности, вартости.

Натуральна форма товару стає формою вартости. Але, nota bene\footnote*{
зауважте добре. \emph{Ред.}
}, це qui pro quo\footnote*{
Одне замість одного, заміна одного другим. \emph{Ред.}
} для товару \emph{В}
(сурдута або пшениці, заліза і~\abbr{т. ін.}) відбувається лише в межах вартостевого відношення, в яке стає
до нього будь-який інший товар \emph{А} (полотно тощо), — лише в межах цього відношення. А що жоден товар
не може відноситися до самого себе як до еквіваленту, отже, і не може зробити свою власну природну
шкуру виразом своєї власної вартости, то й мусить він відноситись до іншого товару як до
еквіваленту, або природну шкуру іншого товару зробити своєю власною формою вартости.

Нехай нам це унаочнить приклад міри, якою виміряється товарові тіла як такі, тобто як споживні
вартості. Голова цукру через те, що є вона тіло, є важка й тому має вагу, але в жодній голові цукру
не можна побачити або відчути її вагу. Візьмімо тепер різні кусні заліза, що їхню вагу наперед
визначено. Тілесна форма заліза, розглянута сама по собі, так само мало є форма виявлення ваги, як і
тілесна форма голови цукру. А все ж, щоб виразити голову цукру як вагу, ми ставимо її у вагове
відношення до заліза. В цьому відношенні залізо фігурує як тіло, яке нічого не репрезентує крім
ваги. Тому кількості заліза служать за міру ваги цукру й супроти тіла цукру репрезентують лише форму
ваги, форму виявлення ваги. Цю ролю відіграє залізо лише в межах цього відношення, в яке стає до
нього цукор або якесь інше тіло, що його вагу треба знайти. Коли б обидві речі не мали ваги, вони не
могли б увійти в це відношення, і тому одна не могла б бути за вираз ваги другої. Коли ми покладемо
обидві речі на шальки терезів, то ми дійсно побачимо, що як вага вони є те саме, і тому, взяті в
певній пропорції, мають ту саму вагу. Як тіло-залізо, являючи міру ваги, репрезентує проти голови
цукру лише вагу, так і в нашому виразі вартости тіло-сурдут репрезентує проти полотна тільки
вартість.

Однак аналогії тут кінець. У виразі ваги голови цукру залізо репрезентує спільну обом тілам природну
властивість, їхню вагу, тимчасом як сурдут у виразі вартости полотна репрезентує надприродну
властивість обох речей: їхню вартість, щось суто суспільне.

А що відносна форма вартости якогось товару, наприклад,
полотна, виражає його вартостеве буття як щось цілком відмінне від його тіла і його властивостей,
наприклад, як щось рівне сурдутові, то вже самий цей вираз говорить за те, що в ньому криється
\parbreak{}  %% абзац продовжується на наступній сторінці
