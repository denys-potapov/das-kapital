\parcont{}  %% абзац починається на попередній сторінці
\index{i}{0037}  %% посилання на сторінку оригінального видання
обмінюють продукти. Власний суспільний рух цих осіб має в
їхніх очах форму руху речей, під контролем якого вони стоять,
замість самим його контролювати. Треба цілком розвиненої товарової
продукції, раніш ніж із самого досвіду виростає науковий
погляд, що приватні праці, виконувані незалежно одна від однієї,
але всебічно залежні одна від однієї як природно вирослі члени
суспільного поділу праці, постійно зводиться до їхньої суспільної
пропорційної міри; і це тому, що робочий час, суспільно-доконечний
для продукування продуктів, як реґулятивний природний
закон, силоміць пробиває собі шлях серед випадкових і завжди
хитких мінових відношень, так, як, приміром, закон тяжіння,
коли комусь на голову валиться будинок\footnote{
«Що слід сказати про закон, який може здійснюватися лише
через періодичні революції? Це саме закон природи, побудований на
несвідомості тих, що підпадають під його силу». (\emph{Friedrich Engels}: «Umrisse
zu einer Kritik der Nationaloekonomie» in Deutsch-französische
Jahrbücher, herausgegeben von Arnold Ruge und Karl Marks. Paris 1884).
}. Тому визначення
величини вартости робочим часом є таємниця, захована під видимими
рухами відносних вартостей товарів. Відкриття цієї таємниці
усуває позірну випадковість у визначенні величин вартости
продуктів праці, але ні в якому разі не усуває речової форми визначення
величин їхньої вартости\footnote*{
У французькому виданні це речення подано так: «Відкриття цієї
таємниці, ясно показуючи, що величина вартости визначається не випадково,
як це здається, не усуває форми, що подає цю величину як кількісне
відношення між речами, між самими продуктами праці». («Le Capital
etc.», ch. I, p. 30).
}.

Розмірковування над формами людського життя, отже, і наукова
аналіза їх, іде взагалі шляхом, протилежним до дійсного
розвитку. Воно починається post festum\footnote*{
Post festum — дослівно: після свята, тут у розумінні: потім,
опісля. \emph{Ред.}
}, отже, від готових результатів
процесу розвитку. Форми, що штампують продукти
праці на товари і тому є за передумову циркуляції товарів, уже
мають тривалість природних форм суспільного життя, раніш ніж
люди намагаються усвідомити собі не історичний характер цих
форм, які їм, скорше, здаються незмінними, а їхній зміст. Так,
лише аналіза товарових цін привела до визначення величини вартости,
лише спільний вираз товарів у грошах привів до фіксування
їхнього характеру, як вартостей. Але саме ця готова форма
товарового світу — грошова форма — прикриває речовим серпанком
суспільний характер приватних праць, атому й суспільні
відносини приватних робітників, замість розкривати їх. Коли я
кажу: сурдут, чоботи й~\abbr{т. ін.} відносяться до полотна як загальне
втілення абстрактної людської праці, то безглуздість такого
вислову відразу впадає в очі. А коли продуценти сурдутів, чобіт
і~\abbr{т. ін.} ставлять ці товари у відношення до полотна — або, що не
змінює справи, до золота й срібла — як до загального еквіваленту,
то відношення їхніх приватних праць до сукупної суспільної
праці з’являється їм саме в цій безглуздій формі.
