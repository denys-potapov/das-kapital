
\index{i}{0068}  %% посилання на сторінку оригінального видання
Обидві одна одній протилежні фази руху товарової метаморфози
становлять кругобіг: товарова форма, скидання товарової
форми, поворот до товарової форми. У всякому разі сам товар
тут протилежно визначений. При вихідному пункті він є неспоживна
вартість, а при кінцевому пункті він є споживна вартість
для свого посідача. Так само й гроші спочатку з’являються як
твердий кристаль вартости, на який перетворюється товар, щоб
потім розпуститись у просту еквівалентну форму товару.

Дві метаморфози, що становлять кругобіг якогось товару,
становлять разом з тим протилежні одна одній частинні метаморфози
двох інших товарів. Той самий товар (полотно) починає ряд
своїх власних метаморфоз і завершує повну метаморфозу другого
товару (пшениці). Підчас свого першого перетворення, продажу,
він відіграє ці дві ролі своєю власною особою. Навпаки, як золота
лялечка, в образі якої він сам проходить шлях усякого товарового
тіла, він разом з тим вивершує першу метаморфозу якогось третього
товару. Отже, кругобіг, що його описує ряд метаморфоз
кожного товару, нерозв’язно сплітається з кругобігами інших
товарів. Процес у цілому становить циркуляцію товарів.

Циркуляція товару не лише формально, але й по суті відрізняється
від безпосереднього обміну продуктів. Щоб цього переконатися,
досить тільки кинути оком на щойно розглянутий процес.
Ткач безперечно обміняв полотно на біблію, власний товар на
чужий. Але це явище існує лише для нього. Продавець біблії,
який вважає гаряче за щось краще від холодного, і в думці не
мав обміняти біблію на полотно, так само як і ткач нічого не
знає про те, що на його полотно обміняно пшеницю, і~\abbr{т. ін.} Товар
особи \emph{В} заміняє товар особи \emph{А}, алеж \emph{А} й \emph{В} не обмінюють взаємно
своїх товарів. В дійсності може трапитись, що \emph{А} й \emph{В} взаємно
купують один в одного, але такі окремі випадки зовсім не викликаються
загальними відносинами циркуляції товарів. З одного
боку, тут видно, як обмін товарів ламає індивідуальні й локальні
межі безпосереднього обміну продуктів і розвиває обмін
речовин людської праці. З другого боку, розвивається ціле коло
суспільних зв’язків, що не залежать від діячів циркуляції,
і тому лежать поза їхнім контролем. Ткач може продати полотно
лише тому, що селянин уже продав пшеницю; охочий до горілки
може продати біблію лише тому, що ткач продав полотно; фабрикант
горілки може продати своє гаряче питво лише тому, що інший
уже продав питво життя вічного і~\abbr{т. д.}

Тим то процес циркуляції й не закінчується, як за безпосереднього
обміну продуктами, після того, як споживні вартості
перемінили місця або посідачів. Гроші не зникають з тієї причини,
що вони кінець-кінцем випали з ряду метаморфоз якогось
товару. Вони знову й знов осідають у тих пунктах циркуляції,
що їх звільняють товари. Приміром, у повній метаморфозі полотна:
полотно — гроші — біблія — спершу випадає полотно з
циркуляції, гроші заступають його місце, потім з циркуляції
випадає біблія і знову гроші заступають її місце. Заміна одного
\parbreak{}  %% абзац продовжується на наступній сторінці
