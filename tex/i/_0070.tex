\parcont{}  %% абзац починається на попередній сторінці
\index{i}{0070}  %% посилання на сторінку оригінального видання
суперечність набирає в протилежностях товарової метаморфози
своїх розвинених форм руху. Тому ці форми включають можливість
— але лише можливість — криз. Розвиток цієї можливости
на дійсність вимагає цілої низки відносин, які з погляду простої
товарової циркуляції ще зовсім не існують\footnote{
Порівн. мої уваги про Джемса Мілла: «Zur Kritik der Politischen
Oekonomie», S. 74--76. («До критики політичної економії», ДВУ, 1926~\abbr{р.},
стор. 110--111). Два пункти характеристичні тут для методи економічної
апологетики. Поперше, утотожнювання циркуляції товарів з безпосереднім
обміном продуктів через просте абстрагування від їхніх ріжниць.
Подруге, спроба заперечити суперечності капіталістичного процесу продукції;
цього намагаються досягнути таким способом, що відносини між
капіталістичними аґентами продукції зводять до простих відносин, які
виникають із циркуляції товарів. Але продукція товарів і циркуляція
товарів — це явища, що належать до якнайрізніших способів продукції,
хоч і в неоднаковому обсязі й мірі. Отже, ми нічого не знаємо про differentia
specifica\footnote*{
характеристичні особливості. \emph{Ред.}
} тих способів продукції, а тому й не можемо цінувати їх,
коли нам відомі лише спільні їм абстрактні категорії товарової циркуляції.
В жодній науці, крім політичної економії, не панує такого великого
пишання елементарними банальностями. Наприклад, Ж.~Б.~Сей зважується
висловлювати свої міркування про кризи, знаючи тільки, що товар
є продукт.
}.

Як посередник циркуляції товарів гроші набирають функції
засобу циркуляції.

\subsubsection{Обіг грошей}

Зміна форм, що в них відбувається обмін речовин продуктів
праці, $Т — Г — Т$, зумовлює те, що та сама вартість як товар
становить вихідний пункт процесу і повертається назад до цього
самого пункту знову таки як товар. Тому цей рух товарів є кругобіг.
З другого боку, ця сама форма виключає кругобіг грошей.
Її результат є невпинне відпалювання грошей від їхнього вихідного
пункту, а не поворот до нього. Доки продавець зберігає
в своїх руках перетворену форму свого товару, гроші, доти
товар перебуває у стадії першої метаморфози, або пройшов лише
першу половину своєї циркуляції. Коли процес, продаж задля
купівлі, довершено, то й гроші вже віддалились від рук свого
первісного посідача. Певна річ, коли ткач, купивши біблію,
знову продає полотно, то й гроші знов повертаються до його рук.
Але вони повертаються не в наслідок циркуляції перших
20 метрів полотна, яка, навпаки, віддалила їх від рук ткача
до рук продавця біблії. Вони повертаються назад лише через
відновлення або повторення того самого процесу циркуляції
для нового товару; кінцевий результат тут такий самий, як
і там. Тому формою руху, що її безпосередньо надає грошам
циркуляція товарів, є постійне їхнє віддалення від вихідного
пункту, їхній перехід із рук одного посідача товарів до рук
іншого, або обіг їхній (currency, cours de la monnaie).

Обіг грошей є постійне, монотонне повторювання того самого
процесу. Товар перебуває завжди на боці продавця, гроші завжди
\parbreak{}  %% абзац продовжується на наступній сторінці
