
\index{i}{0118}  %% посилання на сторінку оригінального видання
Ми вже показали, що додаткова вартість не може виникнути
з циркуляції, отже, при її утворенні за спиною циркуляції мусить
зчинитися щось таке, чого в ній самій не можна помітити\footnote{
«Серед звичайних умов ринку обмін не створює зиску. Коли його
не було раніш, то його не створиться й після цієї оборудки». («Profit, in
the usual condition of the market, is not made by exchanging. Had it not
existed before, neither could it after that transaction»). (\emph{Ramsay}: «An
Essay on the Distribution of Wealth», Edinburgh 1836, p. 184).
}. Але
чи може додаткова вартість виникнути звідкись інакше, опріч
циркуляції? Циркуляція є сума всіх товарових взаємовідносин
посідачів товарів. Поза нею посідач товарів стоїть лише у відношенні
до свого власного товару. Щождо вартости товару, то це
відношення обмежується на тім, що в товарі міститься певна
кількість власної праці товаропосідача, вимірюваної за певними
суспільними законами. Ця кількість праці виражається у величині
вартости його товару, а що величина вартости виражається
в рахункових грошах, то ця кількість праці виражається, приміром,
у ціні в 10\pound{ фунтів стерлінґів}. Але його праця не виражається
у вартости товару і в лишкові понад власну вартість товару,
не виражається в ціні в 10, яка одночасно є ціна в 11, не виражається
у вартості, яка більша за себе саму. Посідач товарів
може своєю працею створювати вартості, але він не може створювати
вартості, що самозростають. Він може підвищити вартість
товару, додаючи новою працею до наявної вартости нову вартість,
приміром, виготовляючи із шкури чоботи. Той самий матеріял
має тепер більшу вартість, бо в ньому міститься більша кількість
праці. Тому чоботи мають більшу вартість, аніж шкура, алеж вартість
шкури лишилась такою, якою вона була. Вона не зросла,
не прилучила до себе додаткової вартости підчас продукції чобіт.
Отже, неможливо, щоб товаропродуцент поза сферою циркуляції,
не стикаючися з іншими посідачами товарів, збільшував вартість,
а тому й неможливо, щоб він поза сферою циркуляції перетворював
гроші або товар на капітал.

Отже, капітал не може виникнути з циркуляції і так само не
може виникнути поза циркуляцією. Він мусить виникнути одночасно
в циркуляції і не в ній.

Таким чином виявився подвійний результат.

Перетворення грошей на капітал слід розвинути на основі
законів, іманентних товаровій циркуляції, так, щоб обмін еквівалентів
правив за вихідний пункт\footnote{
Після поданих пояснень читач розуміє, що це значить тільки ось
що: утворення капіталу мусить бути можливе й тоді, коли ціни товарів
дорівнюють їхнім вартостям. Утворення капіталу не можна пояснити відхиленням
товарових цін від товарових вартостей. Коли ціни дійсно
відхиляються від вартостей, то треба їх спочатку звести на останні, тобто
залишити цю обставину як випадкову осторонь, щоб мати перед собою в
чистій формі явище утворення капіталу на основі товарового обміну, і
щоб, спостерігаючи його, не заплутатись через побічні обставини, що
ускладнюють самий процес і є чужі для нього. Нарешті, відомо, що ця
редукція ніяким чином не є лише наукова процедура. Постійні коливання
ринкових цін, їхнє піднесення та зниження, урівноважуються,
взаємно
касуються й сами собою зводяться на пересічну ціну як свою внутрішню
норму (Regel). Ця остання є провідна зірка, приміром, для купця або
промисловця в кожному підприємстві, яке функціонує довший час. Отже,
купець або промисловець знає, що, коли розглядати довший період у
цілості, товари дійсно продається не нижче і не вище, а за їхніми пересічними
цінами. Отже, коли б безстороннє мислення було взагалі в його інтересах,
то він мусив би поставити перед собою проблему утворення капіталу
ось як: як може постати капітал за реґулювання цін пересічною ціною,
тобто в останній інстанції вартістю товару? Я кажу «в останній інстанції»,
бо пересічні ціни не збігаються безпосередньо з величиною вартости товарів,
як то гадають А.~Сміс, Рікардо й інші.
}. Наш посідач грошей, який
\index{i}{0119}  %% посилання на сторінку оригінального видання
є покищо тільки гусінню капіталіста, мусить купувати товари
за їхньою вартістю, за їхньою вартістю продавати, і, все ж таки,
наприкінці процесу витягати більше вартости, ніж він авансував.
Його перетворення з гусені на метелика, з лише посідача
грошей на дійсного капіталіста, мусить відбуватися в сфері
циркуляції, і в той самий час воно мусить відбуватись не
в сфері циркуляції. Такі умови цієї проблеми. Ніс Rhodus,
hic salta!\footnote*{
Дослівно: Тут Родос, тут стрибай. — Античне прислів’я з байок
Езопа: відповідь родосців одному хвалькові, що вихвалявся своїми
стрибками. Вживається в значенні: отут покажи свою вмілість. \emph{Ред.}
}

\subsection{Купівля і продаж робочої сили}

Зміна вартости грошей, що повинні перетворитися на капітал,
не може відбутися в самих цих грошах, бо як засіб купівлі і як
засіб виплати вони реалізують ціну товарів, які за них купують
або за які ними платять, а коли вони залишаються в своїй власній
формі, то вони тверднуть у, так би мовити, скам’янілу, незмінну
величину вартости\footnote{
«У формі грошей\dots{} капітал не продукує зиску» («In the form of
money\dots{} capital is productive of no profit»). (\emph{Ricardo}: «Principles
of Political Economy», 3 rd ed. London 1821, p. 267).
}. Так само не може виникнути ця зміна з
другого акту циркуляції, з перепродажу товару, бо цей акт лише
перетворює товар із натуральної форми назад у грошову форму.
Отже, зміна мусить статися з товаром, що купується в першому
акті $Г — Т$, але не з його вартістю, бо обмінюється еквіваленти,
товар оплачується за його вартістю. Отже, зміна може виникнути
лише з його споживної вартости, як такої, тобто з його споживання.
Щоб здобувати вартість з споживання якогось товару, нашому
посідачеві грошей мусило б пощастити відкрити в межах сфери
циркуляції, на ринку, такий товар, що сама його споживна вартість
мала б своєрідну властивість бути за джерело вартости,
отже, такий товар, що його фактичне споживання саме було б
упредметненням праці, а тому й творенням вартости. І посідач
грошей находить на ринку такий специфічний товар: це здатність
до праці, або робоча сила.

Під робочою силою або здатністю до праці ми розуміємо
сукупність фізичних та інтелектуальних здібностей, які існують
в організмі, живій особистості людини, і що їх вона пускає в рух
щоразу, коли продукує якібудь споживні вартості.
\parbreak{}
