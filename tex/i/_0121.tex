
\index{i}{0121}  %% посилання на сторінку оригінального видання
Для того, щоб хтось продавав відмінні від своєї робочої сили
товари, він, певна річ, мусить мати засоби продукції, приміром,
сировинний матеріял, знаряддя праці тощо. Він не може зробити
чобіт, не маючи шкури. Крім того, він потребує ще засобів існування.
Ніхто, навіть музикант майбутнього, не може жити з
продуктів будучини, отже, і з споживних вартостей, що продукція
їх ще не закінчена; від першого дня своєї появи на земній
кулі людина мусить день-у-день споживати, мусить споживати
раніше, ніж почне продукувати, і підчас самої продукції. Коли
продукти продукується як товари, то вони мусять бути продані
після того, як їх випродуковано, і лише після продажу вони
можуть задовольнити потреби продуцента. До часу, потрібного
на продукцію, долучається ще час, потрібний на продаж.

Отже, для перетворення грошей на капітал посідач грошей
мусить знайти на товаровому ринку вільного робітника, вільного
в двоякому розумінні: щоб він, з одного боку, як вільна особа
порядкував своєю робочою силою як своїм товаром, і щоб, з
другого боку, в нього не було на продаж ніяких інших товарів,
щоб був він гольцем голий, позбавлений усіх речей, потрібних
для здійснення його робочої сили.

Питання, чому цей вільний робітник протистоїть у сфері
циркуляції посідачеві грошей, не інтересує останнього, бо для
нього робочий ринок є лише осібний відділ товарового ринку.
І нас воно покищо так само мало інтересує. Ми теоретично тримаємося
цього факту, так само як посідач грошей тримається
його практично. В усякому разі ясне одне: природа не продукує
на одному боці посідачів грошей або товаропосідачів, а на другому
— посідачів лише своєї власної робочої сили. Це відношення
не є природне, не базується на законах природи, і так само воно
не є таке суспільне відношення, яке було б спільне всім періодам
історії. Воно, очевидно, саме є результат попереднього історичного
розвитку, продукт багатьох економічних переворотів, продукт
загину цілого ряду давніших формацій суспільної продукції.
І ті економічні категорії, що ми їх розглядали раніш, теж
мають на собі сліди своєї історії. Буття продукту як товару
потребує певних історичних умов. Щоб стати товаром, продукт
мусить продукуватися не як безпосередній засіб існування для
самого продуцента. Коли б ми продовжували наші дослідження,
коли б ми запитали: за яких обставин усі або принаймні більшість
продуктів набирають форми товару, то виявилось би, що це
стається лише на основі цілком специфічного способу продукції,
а саме капіталістичного способу продукції. Однак таке дослідження
було б далеке від аналізи товару. Продукція товарів і циркуляція
товарів можуть існувати, хоч куди значніша маса продуктів,
призначена безпосередньо для власної потреби, не перетворюється
на товари, отже, коли мінова вартість ще далеко не
опанувала суспільний процес продукції у всій його ширині та
глибині. Поява продукту як товару має за свою передумову
\parbreak{}  %% абзац продовжується на наступній сторінці
