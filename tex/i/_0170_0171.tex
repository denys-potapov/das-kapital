\parcont{}  %% абзац починається на попередній сторінці
\index{i}{0170}  %% посилання на сторінку оригінального видання
15\%, річний оборот товарів фабрики мусить становити вартість у \num{115.000}\pound{ фунтів стерлінґів}\dots{} Кожна з
23 половин робочої години продукує щодня \sfrac{5}{115}, або \sfrac{1}{23} цих \num{115.000}\pound{ фунтів стерлінґів}. З цих \sfrac{23}{23},
що становлять цілість цих \num{115.000}\pound{ фунтів стерлінґів} (constituting the whole \num{115.000}\pound{ £}), \sfrac{20}{23}, тобто
\num{100.000}\pound{ фунтів стерлінґів} з цих \num{115.000} повертають назад лише капітал, \sfrac{1}{23}, або \num{5.000}\pound{ фунтів
стерлінґів}, із гуртового прибутку в \num{115.000} (!) покривають зужитковання фабрики і машин. Решта ж,
\sfrac{2}{23}, тобто обидві останні півгодини кожного дня, продукують чистий прибуток
у 10\%. Тому, коли б фабрика за незмінних цін могла працювати 13 годин замість 11\sfrac{1}{2}, то за
збільшення обігового
капіталу приблизно на \num{2.600}\pound{ фунтів стерлінґів} чистий прибуток більше ніж подвоївся б. З другого
боку, коли б робочі години скоротити на 1 годину на день, то чистий прибуток зник би, а коли б
скоротити на 1\sfrac{1}{2} години, то зник би й гуртовий прибуток»\footnote{
Сеніор, там же, стор. 12, 13. Ми не спиняємось на курйозах, що для нашої мети не мають ваги,
наприклад, на твердженні, що фабриканти зараховують до прибутку, брутто або нетто, брудного або
чистого, покриття зужиткованих машин і~\abbr{т. ін.}, тобто складову частину капіталу. Ми не
спиняємося й на тому, правильні чи неправильні ці числові дані. ІЦо вони
варті не більш, як ця так звана «аналіза», це довів Леонард Горнер в «А Letter to Mr.~Senior etc.
London. 1837». Леонард Горнер, один із Factory Inquiry Commissioners\footnote*{
членів комісії для розсліду відносин по фабриках. \emph{Ред.}
} 1833 p. і фабричний
інспектор, у дійсності
фабричний цензор до 1859~\abbr{р.}, здобув собі невмирущі заслуги перед англійською робітничою клясою. Все
своє життя він боровся не лише з розлютованими фабрикантами, але й з міністрами, для яких
незрівнянно важливіше було рахувати «голоси» фабрикантів у палаті громад, ніж
робочі години «рук» на фабриці.

Додаток до 32 примітки. Незалежно від фалшивости змісту сам виклад Сеніора є плутаний. Сказати він,
власне, хотів ось що. Фабрикант щоденно вживає робітників протягом 11\sfrac{1}{2} годин, або \sfrac{23}{2} години.
Цілорічна праця так само, як і поодинокий робочий день, складається з 11\sfrac{1}{2},
або \sfrac{23}{2} години (помножених на число робочих днів протягом року). За такої передумови \sfrac{23}{2} робочої
години продукують річний продукт у \num{115.000}\pound{ фунтів стерлінґів}; \sfrac{1}{2} робочої години продукує \sfrac{1}{23} ×
\num{115.000}\pound{ фунтів стерлінґів}; \sfrac{20}{2} робочої години продукує \sfrac{20}{23} × \num{115.000}\pound{ фунтів стерлінґів} \deq{} \num{100.000}\pound{ фунтам стерлінґів}, тобто вони лише повертають
авансований капітал. Лишаються \sfrac{3}{2} робочої години, що продукують \sfrac{3}{23} × \num{115.000}\pound{ фунтів стерлінґів} \deq{}
\num{15.000}\pound{ фунтів стерлінґів}, тобто гуртовий прибуток. З цих \sfrac{3}{2} робочої години \sfrac{1}{2} робочої години
продукує \sfrac{1}{23} × \num{115.000}\pound{ фунтів стерлінґів} \deq{} \num{5.000}\pound{ фунтів стерлінґів}, тобто продукує
лише покриття зужитковання фабрики й машин. Останні дві половини робочої години, тобто остання
робоча година, продукує \sfrac{2}{23} × \num{115.000}\pound{ фунтів стерлінґів} \deq{} \num{10.000}\pound{ фунтів стерлінґів}, тобто чистий
зиск.
У тексті Сеніор перетворює останні \sfrac{2}{23} продукту на частину самого робочого
дня.
}.

І це пан професор називає «аналізою»! Коли він повірив лементові фабрикантів, що робітники кращий
час дня витрачають
на продукцію, отже, на репродукцію або на покриття вартости будівель, машин, бавовни, вугілля й~\abbr{т.
ін.}, то всяка аналіза
була зайва. Він мав просто відповісти: Мої панове! Коли ви примусите працювати 10 годин замість 11\sfrac{1}{2}, то за інших незмінних
\index{i}{0171}  %% посилання на сторінку оригінального видання
умов щоденне споживання бавовни, машин і~\abbr{т. ін.} зменшиться на 1\sfrac{1}{2} години. Отже, ви виграєте
рівно стільки, скільки втрачаєте. На майбутнє ваші робітники будуть витрачати на репродукцію
або покриття авансованої вартости капіталу на 1\sfrac{1}{2} години менше. А коли б він не повірив їм на
слово і, як тямуща у справі людина, визнав би за потрібне аналізу, то мусив би насамперед попрохати
панів фабрикантів, щоб у питанні, яке стосується
виключно відношення чистого прибутку до величини робочого дня, вони не плутали в одну купу машини й
фабричні будівлі,
сировинний матеріял і працю, а були ласкаві поставити сталий капітал, вміщений у фабричних будівлях,
машинах, сировинному матеріялі тощо, по один бік, а капітал, авансований на заробітну
плату, — по другий бік. Коли б тоді виявилося, приміром, що за обчисленням фабрикантів робітник за
\sfrac{2}{2} робочої години, або за 1 годину, репродукує або покриває заробітну плату, то нашому
аналітикові треба було б далі так казати:

За вашими даними, робітник за передостанню годину продукує свою заробітну плату, а за останню — вашу
додаткову вартість або чистий прибуток. А що протягом однакового часу він продукує
однакові вартості, то продукт передостанньої години має таку
саму вартість, що й продукт останньої. Далі, він продукує вартість лише остільки, оскільки він
витрачає працю, і кількість
його праці вимірюється його робочим часом. Останній, за вашими даними, становить 11\sfrac{1}{2} годин на
день. Одну частину з цих 11\sfrac{1}{2} годин
він витрачає на продукцію, або на покриття своєї заробітної плати, другу частину — на продукцію
вашого чистого прибутку. Чогось іншого він не робить протягом дня. А що, за вашими даними, його
заробітна плата й постачена від нього додаткова вартість є рівновеликі вартості, то він, очевидно,
продукує свою заробітну плату за 5\sfrac{3}{4} годин, а ваш чистий прибуток за другі
5\sfrac{3}{4} годин. А що, далі, вартість пряжі, спродукованої за дві години, дорівнює сумі вартости його
заробітної плати плюс ваш
чистий прибуток, то ця вартість пряжі мусить вимірятися 11\sfrac{1}{2} робочими
годинами, продукт передостанньої години мусить вимірятися 5\sfrac{3}{4} робочими годинами, продукт останньої
години — так
само. Ми дійшли тут до дражливого пункту. Отже, увага! Передостання робоча година є така сама
звичайна робоча година, як і
перша. Ni plus, ni moins\footnote*{
Не більше й не менше. \emph{Ред.}
}. То ж яким чином прядун міг би за одну робочу годину спродукувати
вартість пряжі, що репрезентує 5\sfrac{3}{4} робочих годин? В дійсності він і не робить такого дива. Та
споживна вартість, яку він продукує за одну робочу годину, є певна кількість пряжі. Вартість цієї
пряжі вимірюється 5\sfrac{3}{4} робочими годинами, з яких 4\sfrac{3}{4} вже без його допомоги містяться в спожитих
протягом години засобах продукції, в бавовні, в машинах і~\abbr{т. ін.}, а \sfrac{4}{4}, або 1 годину, додав він
сам. Отже, що його заробітну плату продукується за 5\sfrac{3}{4} годин, а пряжа, сродукована за одну годину,
так само містить у собі 5\sfrac{3}{4} робочих годин, то в тому зовсім немає
\parbreak{}  %% абзац продовжується на наступній сторінці
