\parcont{}  %% абзац починається на попередній сторінці
\index{i}{0179}  %% посилання на сторінку оригінального видання
вільний чи невільний, мусить до робочого часу, доконечного для
утримання себе самого, додавати надлишок робочого часу, щоб
продукувати засоби існування для власника засобів продукції\footnote{
«Ті, що працюють\dots{} у дійсності годують і пенсіонерів, яких називають
багатими, і самих себе» («Those who labour\dots{} in reality feed both
the pensioners called the rich, and themselves»). (\emph{Edmund Burke}: «Thoughts and
Details on Scarcity». London, 1800, p. 2)
},
хоч буде цей власник атенський \textgreek{χαλος χαγανος}\footnote*{
аристократ. \emph{Ред.}
}, етруський теократ,
civis romanus\footnote*{римський громадянин. \emph{Ред.}},
норманський барон, американський рабовласник,
волоський боярин, сучасний лендлорд або капіталіст\footnote{
У своїй «Römische Geschichte» Нібур дуже наївно зауважує: «Нічого
таїти, що такі витвори, як етруські, що будять у нас подив навіть у
своїх руїнах, у маленьких (!) державах мають собі за передумову існування
панів і рабів». Багато глибше висловився Сісмонді, що «брюссельські
мережива» мають собі за передумову існування панів наймачів і найманих
робітників.
}.
Та проте ясно, то коли в якійсь суспільній економічній формації
має перевагу не мінова вартість, а споживна вартість продукту,
то додаткова праця обмежується на вужчому або ширшому колі
потреб, але з самого характеру продукції не випливає безмежна
потреба додаткової праці. Тому ми находимо жахливу надмірну
працю в старовинному світі там, де йдеться про здобуття мінової
вартости в її самостійній грошовій формі — у продукції золота
й срібла. Поневільна, що тягне за собою смерть робітника, праця
є тут офіціяльна форма надмірної праці. Досить прочитати лише
Діодора Сіцілійського\footnote{
«Не можна без жалю до їхньої злиденної долі дивитися на цих нещасних
(що працюють на копальнях золота між Єгиптом, Етіопією й
Арабією), які навіть не мають змоги подбати про чистоту свого тіла або
покрити свою голизну. Бо тут немає поблажливости, немає жалю до
хорих, покалічених, дідусів, до жіночої слабости. Всі мусять, приневолені
ударами київ, працювати й працювати аж доки смерть покладе кінець
їхнім мукам і злидням». (\emph{Diodorus Siculus}: «Historische Bibliothek», Buch З,
kар. 13).
}. Однак це є винятки у старовинному
світі. Але скоро тільки народи, що в них продукція рухається
ще в низьких формах рабської праці, панщини й~\abbr{т. ін.}, втягуються
у світовий ринок, опанований капіталістичним способом продукції,
в наслідок чого переважним інтересом для них стає продаж виробів
своєї продукції за кордон, — до варварських страхіть рабства,
кріпацтва й~\abbr{т. ін.} прищеплюється цивілізоване страхіття надмірної
праці. Тому праця негрів у південних штатах Американського
союзу мала помірно-патріархальний характер доти, доки продукцію
зверталося головним чином на задоволення власних потреб.
Але в міру того як експорт бавовни стає життєвим інтересом цих
держав, в міру цього й надмірна праця негра, а в деяких місцях
споживання його життя протягом сімох робочих років, стає складовою
частиною байдужно обрахованої системи. Тут ішлося вже
не про те, щоб видушити з нього певну масу корисних продуктів.
Тут уже йшлося про продукцію самої додаткової вартости. Те ж
саме було з панщиною, приміром, у дунайських князівствах.
