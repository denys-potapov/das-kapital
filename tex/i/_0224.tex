\parcont{}  %% абзац починається на попередній сторінці
\index{i}{0224}  %% посилання на сторінку оригінального видання
до в’язниць, а організації зруйновано, захитала в робітничої
кляси Англії довіру до власних сил. Незабаром після цього паризьке
червневе повстання і його криваве придушення об’єднало
під загальним прапором, так на європейському континенті, як
і в Англії, всі фракції панівних кляс — землевласників і капіталістів,
біржових вовків і крамарів, протекціоністів і фритредерів,
уряд і опозицію, попів і вільнодумців, молодих повій і
старих черниць, — що їхнім спільним гаслом було: рятувати власність,
релігію, родину й суспільство. Робітничу клясу повсюди
оголошено злочинною, її проклято й поставлено під «loi des
suspects»\footnote*{
закон про підозрілих. \emph{Ред.}
}. Отже, фабрикантам не треба було далі церемонитися.
Вони підняли одвертий бунт не лише проти десятигодинного закону,
а й проти цілого законодавства, яке, починаючи від 1833~\abbr{р.}, силкувалось
до певної міри загнуздати «вільне» висисання робочої
сили. Це був Proslavery Rebellion\footnote*{
бунт на захист рабства. \emph{Ред.}
} у мініятюрі, що його більш
ніж два роки провадилося з цинічною нещадністю, з терористичною
енерґією, і то дешевше, що бунтівник-капіталіст нічим
не ризикував, хібащо шкурою свого робітника.

Щоб зрозуміти дальший виклад, треба собі пригадати, що
фабричні закони з 1833, 1844 і 1847~\abbr{рр.} всі три мають правну
силу, принаймні остільки, оскільки один не вносить до другого
якихось виправлень; що жоден з них не обмежує робочого дня
робітників-чоловіків старших за 18 років і що від 1833~\abbr{р.} п’ятнадцятигодинний
період від пів на шосту годину зранку до пів
на дев’яту ввечері лишався законним «днем», у межах якого
на приписаних законом умовах треба було підліткам і жінкам
виконувати спершу дванадцятигодинну, а пізніш десятигодинну
працю.

Фабриканти подекуди почали з того, що звільняли частину,
іноді половину занятих у них підлітків і робітниць, відновлюючи
зате майже забуту нічну працю дорослих робітників-чоловіків.
Закон про десятигодинний робочий день, вигукували вони, не
дає їм іншої альтернативи!\footnote{
«Reports etc. for 31 st. October 1848», p. 133, 134.
}

Другий крок стосувався встановлених законом перерв на їжу.
Послухаймо, що кажуть фабричні інспектори: «Від часу обмеження
робочого дня на 10 годин фабриканти запевняють, хоч на практиці
вони й не доводять ще свого погляду до останніх висновків,
що коли праця триває, приміром, від 9 години ранку до 7 години
вечора, то вони в достатній мірі виконують приписи закону, даючи
на їжу одну годину перед 9 годиною ранку й півгодини після
7 години вечора, отже, разом 1\sfrac{1}{2} години. В деяких випадках
вони дозволяють тепер півгодини або цілу годину на обід, але одночасно
настоюють на тому, що вони зовсім не зобов’язані давати
якусь частину з цих 1\sfrac{1}{2} години протягом десятигодинного робочого
дня\footnote{
«Reports, etc. for 30 th April 1848», p. 47.
}. Отже, пани фабриканти запевняли, що дріб’язково
\parbreak{}  %% абзац продовжується на наступній сторінці
