\parcont{}  %% абзац починається на попередній сторінці
\index{i}{0254}  %% посилання на сторінку оригінального видання
вартістю в 1\shil{ шилінґ}, то він продає його на 3\pens{ пенси} вище від його
індивідуальної вартости й таким чином реалізує понаддодаткову
вартість (Extramehrwert) у 3\pens{ пенси.} Але, з другого боку, дванадцятигодинний
робочий день виражається тепер для нього у
24 штуках товару замість 12, як це було раніш. Отже, для
того, щоб продати продукт одного робочого дня, він потребує
подвійного збуту або удвоє більшого ринку. За інших незмінних
обставин його товари завойовують більший ринок лише через
зниження їхніх цін. Тому капіталіст продаватиме їх понад їхню
індивідуальну вартість, але нижче від їхньої суспільної вартости,
приміром, по 10\pens{ пенсів} за штуку. Таким чином із кожної окремої
штуки він усе ще видобуде понаддодаткову вартість в 1\pens{ пенс.} Це
підвищення додаткової вартости відбувається для нього однаково,
належить чи не належить його товар до кола доконечних
засобів існування, отже однаково, входить чи не входить він як
визначальний момент у загальну вартість робочої сили. Отже,
незалежно від цієї останньої обставини для кожного поодинокого
капіталіста існує мотив здешевлювати товари через підвищення
продуктивної сили праці.

Однак навіть у цьому випадку підвищена продукція додаткової
вартости постає із скорочення доконечного робочого часу та
з відповідного здовження додаткової праці\footnoteA{
«Зиск, що дістає якась людина, залежить не від того, що вона
панує над продуктом праці інших людей, а від того, що вона панує над
самою працею. Коли вона може продати свої продукти за вищу ціну, тимчасом
як заробітна плата її робітників лишається незмінна, вона, очевидно,
матиме користь\dots{} Тоді меншої частини того, що вона продукує, вистачить
на те, щоб пустити в рух цю працю, отже, більша частина лишається їй
самій». («А man’s profit does not depend upon his command of the produce
of other men’s labour, but upon his command of labour itself. If he can sell
his goods at a higher price, while his workmen's wages remain unaltered,
he is clearly benefited\dots{} A smaller proportion of what he produces is sufficient
to put that labour into motion, and a larger proportion consequently
remains for himself»). («Outlines of Political Economy», London 1832,
p. 49, 50).
}.  Доконечний робочий
час становив 10 годин, або денна вартість робочої сили становила
5\shil{ шилінґів}, додаткова праця становила 2 години, а тому
денно продукована додаткова вартість становила 1\shil{ шилінґ.} Але
наш капіталіст продукує тепер 24 штуки товару, які він продає
по 10\pens{ пенсів} за штуку, або всі разом за 20\shil{ шилінґів.} А що вартість
засобів продукції дорівнює 12\shil{ шилінґам}, то 14\sfrac{2}{5} штуки товару
покривають лише авансований сталий капітал. Дванадцятигодинний
робочий день виражається в 9\sfrac{3}{5} штуках, що ще залишаються.
А що ціна робочої сили дорівнює 5\shil{ шилінґам}, то доконечний робочий
час виражається в 6 штуках продукту, а додаткова праця
у 3\sfrac{3}{5} штуках. Відношення доконечної праці до додаткової, яке за
пересічних суспільних умов становило $5:1$, тепер становить
уже лише $5:3$. До того самого результату дійдемо й таким способом.
Вартість продукту дванадцятигодинного робочого дня
дорівнює 20\shil{ шилінґам.} Із них 12 належать до вартости засобів
продукції — вартости, що лише знову з’являється. Отже, лишаються
\index{i}{0255}  %% посилання на сторінку оригінального видання
8\shil{ шилінґів} як грошовий вираз вартости, що в ній утілюється робочий день.
Цей грошовий вираз вищий, ніж грошовий
вираз пересічної суспільної праці того самого роду, що з неї
12 годин праці виражаються лише в 6\shil{ шилінґах.} Праця виняткової продуктивної сили
функціонує як складна, тобто піднесена
до ступеня (potenzierte) праця, або творить протягом однакового
часу вищі вартості, аніж пересічна суспільна праця того самого
роду. Але наш капіталіст, як і раніш, платить за денну вартість
робочої сили лише 5\shil{ шилінґів.} Отже, робітник замість колишніх
10 потребує тепер лише 7\sfrac{1}{2} годин для репродукції цієї вартости.
Тому його додаткова праця зростає на 2\sfrac{1}{2} години, а спродукована
ним додаткова вартість — з 1 до 3\shil{ шилінґів.} Отже, капіталіст,
що вживає поліпшеного способу продукції, присвоює собі як додаткову працю більшу
частину робочого дня, ніж усі інші капіталісти
з тієї самої галузі промисловости. Він робить індивідуально те,
що при продукції відносної додаткової вартости капітал робить
взагалі і в цілому. Але, з другого боку, та понаддодаткова вартість
зникає, скоро тільки новий спосіб продукції узагальнюється, і
разом з тим зникає ріжниця між індивідуальною вартістю дешевше
продукованих товарів та їхньою суспільною вартістю. Той самий
закон визначення вартости робочим часом, який капіталістові
з новою методою продукції дається взнаки в такій формі, що він
мусить продавати свій товар нижче від його суспільної вартости, — цей самий
закон як примусовий закон конкуренції спонукає його конкурентів вводити в себе
новий спосіб продукції\footnote{
«Якщо мій сусіда, продукуючи багато з невеликою кількістю праці,
може продавати дешево, то і я мушу старатися продавати так само дешево,
як він. Так, усякий винахід, усяка метода чи машина — все, що дає змогу
обходитися з меншою кількістю рук і тим то продукувати дешевше, доконечно
викликає в інших змагання або застосувати той самий винахід,
ту саму методу чи машину, абож винайти щось їм подібне, так, щоб
усі були в однакових умовах, і ніхто не міг би продавати дешевше за свого
сусіди». («If my neighbour by doing much with little labour, can sell
cheap, I must contrive to sell as cheap as he. So that every art, trade, or
engine, doing work with labour of fewer hands, and consequently cheaper,
begets in others a kind of necessity and emulation, either of using the same
art, trade, or engine or of inventing something like it, that every man
may be upon the square, that no man may be able to undersell his neighbour»).
(«The Advantages of the East-India Trade to England», London 1720, p. 67).
}. Отже, цілий цей процес зачіпає загальну норму додаткової вартости
кінець-кінцем лише тоді, коли підвищення продуктивної
сили праці охоплює такі галузі продукції, отже, понижує ціни
на такі товари, що увіходять у коло доконечних засобів існування,
і тому становлять елементи вартости робочої сили.

Вартість товарів стоїть у зворотному відношенні до продуктивної сили праці,
так само й вартість робочої сили, бо вона
визначається вартістю товарів. Навпаки, відносна додаткова
вартість стоїть у простому відношенні до продуктивної сили праці.
Вона зростає із зростом і падає із спадом продуктивної сили праці.
Пересічний суспільний дванадцятигодинний робочий день, —
\parbreak{}  %% абзац продовжується на наступній сторінці
