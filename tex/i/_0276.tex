\parcont{}  %% абзац починається на попередній сторінці
\index{i}{0276}  %% посилання на сторінку оригінального видання
у знаряддях, які раніше служили для різних цілей. Пізнання
тих особливих труднощів, які викликає незмінена форма знарядь,
вказує напрям, у якому має йти зміна їхньої форми. Диференціяція
інструментів праці, в наслідок якої інструменти того самого роду
набувають особливих тривалих форм для кожного окремого
корисного вжитку, і їхня спеціялізація, в наслідок якої кожний
такий окремий інструмент функціонує в повному своєму обсязі
лише в руках спеціяльного частинного робітника, — це характеризує мануфактуру.
В самому лише Бірмінґемі продукують
якихось 500 відмін молотків, і з них кожний не тільки служить
для якогось осібного продукційного процесу, а ще й певне число
таких відмін служить часто лише для різних операцій у тому
самому процесі. Мануфактурний період спрощує, поліпшує та
робить різноманітнішими знаряддя праці, пристосовуючи їх до
виключних окремих функцій частинних робітників\footnote{
Щодо природних органів рослин і тварин Дарвін у своїй епохальній праці
«Постання родів» зауважує ось що: «Доки той самий орган має виконувати різні
праці, доти причину його змінливости, мабуть, можна знайти в тому, що природний
добір менш старанно зберігає або пригнічує кожний дрібний відхил у
формі, аніж тоді, коли той самий орган було б призначено виключно лише для
якогось одного окремого завдання. Так, ножі, які призначено на те, щоб різати
всяку всячину, можуть бути взагалі більш-менш однакової форми, тимчасом як
інструмент, призначений лише для якогось одного вжитку, мусить для кожного
іншого вжитку мати й іншу форму».
}. Тим самим він утворює одну з матеріяльних умов для вживання машин,
які складаються з комбінації простих інструментів.

Частинний робітник та його інструмент становлять прості
елементи мануфактури. Звернімось тепер до її цілого механізму.

\subsection[%
Дві основні форми мануфактури: гетерогенна мануфактура
й~органічна мануфактура
]{Дві основні форми мануфактури: гетерогенна мануфактура~й~органічна мануфактура}

Організація мануфактури має дві основні форми, які, хоч
випадково й сплітаються одна з одною, становлять, однак, два
посутньо відмінні роди і відіграють цілком різні ролі при пізнішому перетворенні
мануфактури на машинову велику промисловість. Цей двоякий характер мануфактури
випливає з природи самого продукту. Цей останній або утворюється через просту
механічну сполуку самостійних частинних продуктів, абож
завдячує свою готову форму послідовному рядові зв’язаних між
собою процесів та маніпуляцій.

Льокомотив, приміром, складається більш ніж із \num{5.000} самостійних частин. Однак
його не можна вважати за приклад першого роду мануфактури у власному значенні,
бо він є витвір великої промисловости. Зате добрим прикладом може бути годинник,
що на ньому й Вільям Петті унаочнює мануфактурний поділ праці. З індивідуального
витвору нюрнберзького ремісника годинник перетворився на суспільний продукт
безлічі частинних робітників
таких, як от: Rohwerkmacher, виготівник годинникарських
\parbreak{}  %% абзац продовжується на наступній сторінці
