\parcont{}  %% абзац починається на попередній сторінці
\index{i}{0297}  %% посилання на сторінку оригінального видання
заслуговує присуду на смерть, і вбивати її, якщо вона його не
заслуговує. Поділ праці — це вбивство народу»\footnote{
«То subdivide a man is to execute him, if he deserves the sentence,
to assassinate him, if he does not\dots{} the subdivision of labour is the assassination
of a people» (\emph{D.~Urquhart}: «Familiar Words», London 1855,
p. 119). Геґель мав дуже єретичні погляди на поділ праці. «Під освіченою
людиною можна розуміти насамперед таку, яка вміє робити все
те, що роблять інші» — каже він у своїй філософії права.
}.

Кооперація, що ґрунтується на поділі праці, тобто мануфактура,
на своїх початках є спонтанейне утворення. Але скоро
тільки вона набуває певної стійкости й досить широкої бази,
вона стає свідомою, пляномірною та систематичною формою капіталістичного
способу продукції. Історія власне мануфактури
показує, яким чином характеристичний для неї поділ праці
набуває відповідних форм насамперед емпірично, нібито поза
спиною осіб-діячів, а потім, подібно до цехового ремества, силкується
знайдену вже форму зберегти як традицію і в поодиноких
випадках зберігає її протягом цілих віків. Якщо й змінюється
ця форма, то, за винятком другорядних моментів, завжди лише в
наслідок революції у знаряддях праці. Сучасна мануфактура, —
я не кажу тут про велику промисловість, що ґрунтується на
машинах, — або находить, як от, наприклад, мануфактура одягу,
свої disjecta membra poetae\footnote*{
поодинокі члени
} уже готовими по тих великих містах,
де вона постає, та має тільки позбирати ці розкидані члени,
абож принцип поділу праці є цілком ясний, так що різні
операції ремісничої продукції (наприклад, оправляння книжок)
просто присвоюється у виключну функцію окремим робітникам.
У таких випадках не потрібно й тижня досвіду, щоб знайти
кількісну пропорцію між числом рук, потрібних для кожної
функції\footnote{
Наївну віру у винахідливий геній, що його поодинокий капіталіст
виявляє в поділі праці a priori, можна знайти хіба лише в німецьких
професорів, таких, як от, наприклад, пан Рошер, що призначає «різні
заробітні плати» («diverse Arbeitslöhne») капіталістові в подяку за те,
мовляв, що з його юпітерської голови вискакує поділ праці в закінченому
вигляді. Більше або менше застосування поділу праці залежить від
довжини гаманця, а не від величини генія.
}.

Мануфактурний поділ праці через розчленування ремісничої
діяльности, спеціялізацію знарядь праці, утворення частинних
робітників, згрупування та скомбінування їх в один цілий механізм
утворює якісне розчленування й кількісну пропорційність
суспільних процесів продукції, отже, утворює певну організацію
суспільної праці та розвиває разом з тим нову суспільну продуктивну
силу праці. Як специфічно капіталістична форма суспільного
процесу продукції, — а на тих основах, на яких його,
знайдено, він може розвиватись не інакше, а тільки в капіталістичній
формі, — він є лише осібна метода створювати відносну додаткову
вартість, тобто коштом робітника підвищувати самозростання
капіталу, що називають суспільним багатством, «Wealth
\parbreak{}  %% абзац продовжується на наступній сторінці
