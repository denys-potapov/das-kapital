\parcont{}  %% абзац починається на попередній сторінці
\index{i}{0311}  %% посилання на сторінку оригінального видання
однорідних робочих машин, як от у ткацтві, хоч на комбінації
різнорідних, як от у прядінні, сама по собі становить великий
автомат, скоро тільки її пускає в рух якийсь перший саморушний
мотор. Однак цілу систему може пускати в рух, наприклад,
парова машина, тимчасом як окремі виконавчі машини або потребують
ще робітників для деяких операцій, як це було перед заведенням
selfacting mule\footnote*{
саморушних мюлей. \emph{Ред.}
} при праці на мюлях і є ще й досі при тонкопрядінні;
або певні частини машини для виконання їхніх операцій
мусять бути керовані подібно до знаряддя робітником, як це було
при будуванні машин перед перетворенням slide rest (токарського
апарату) на selfactor\footnote*{
саморушний механізм. \emph{Ред.}
}. Скоро тільки робоча машина без допомоги
людини виконує всі рухи, потрібні, щоб обробити сировинний
матеріял, потребуючи лише догляду людини, ми маємо автоматичну
систему машин, яка, однак, здатна до постійного вдосконалення
в деталях. Так, наприклад, апарат, що автоматично
спиняє прядільну машину, як тільки урветься одним-одна нитка,
і selfacting stop, що спиняє поліпшений паровий ткацький верстат,
як тільки на цівці ткацького човника виходить уся нитка, — це
цілком новітні винаходи. За приклад так безперервности продукції,
як і проведення автоматичного принципу може бути сучасна
паперова фабрика. Взагалі на паперовому виробництві можна з користю
детально простудіювати так ріжницю між різними способами
продукції, що базуються на різних засобах продукції, як
і зв’язок суспільних продукційних відносин із цими способами
продукції: колишнє німецьке виробництво паперу дає зразотремісничої
продукції, Голляндія в XVII і Франція у XVIII столітті
дають зразок мануфактури у власному значенні слова, а сучасна
Англія дає зразок автоматичної фабрикації в цій галузі промисловости;
крім того, в Китаї та Індії існують ще дві різні староазійські
форми такої самої промисловости.

Розчленована система робочих машин, які свій рух дістають лише
за допомогою передатної машини від одного центрального автомата,
є найрозвиненіша форма машинового виробництва. Замість
поодинокої машини виступає тут механічна почвара, що її тіло
заповнює цілі фабричні будинки, а демонічна сила її, спочатку
замаскована майже урочисто-розміреним рухом її велетенських
членів, вибухає в гарячково-божевільному вировому танку її
численних робочих органів у власному значенні слова.

Мюлі, парові машини й~\abbr{т. ін.} появилися раніш, ніж робітники,
що їх виключною працею було робити парові машини, мюлі й
т. iн., цілком так само, як людина носила одяг раніше, ніж з’явилися
кравці. Однак винаходи Вокансона, Аркрайта, Ватта
й інших можна було запровадити в життя лише тому, що ці винахідники
застали значне число вправних робітників-механіків,
підготовлених мануфактурним періодом. Одна частина цих робітників
складалася з самостійних ремісників різних професій, друга
\parbreak{}  %% абзац продовжується на наступній сторінці
