\parcont{}  %% абзац починається на попередній сторінці
\index{i}{0312}  %% посилання на сторінку оригінального видання
частина була з’єднана в мануфактурах, де, як уже раніш згадувано,
поділ праці панував з особливою точністю. Із збільшенням числа
винаходів і зростом попиту на нововинайдені машини щораз
більше розвивався, з одного боку, розклад фабрикації машин на
різноманітні самостійні галузі, а з другого боку — поділ праці
всередині машинобудівельних мануфактур. Отже, ми тут вбачаємо
в мануфактурі безпосередню технічну основу великої промисловости.
Мануфактура продукувала машини, за допомогою яких
велика промисловість знищила ремісниче та мануфактурне виробництво
в тих галузях продукції, які вона насамперед охопила.
Отже, машинове виробництво виросло стихійно на невідповідній
йому матеріяльній основі. На певному ступені розвитку машинове
виробництво само мусило зробити переворот у цій основі, яку воно
спочатку застало готовою і потім далі виробляло в її старій формі,
та створити для себе нову базу, відповідну його власному способові
продукції. Як поодинока машина лишається карликовою, поки
її пускає в рух лише людина, як система машин не могла вільно
розвиватися, поки на місце рушійних сил, які вона застала, —
худоби, вітру, а то й води — виступила парова машина, так само
й велика промисловість була паралізована в цілому своєму розвитку
доти, доки характеристичний для неї засіб продукції,
сама машина, завдячувала своє існування особистій силі та особистій
вправності, а значить, залежала від розвитку мускулів,
гостроти зору та віртуозности рук, що з ними частинний робітник
у мануфактурі й ремісник поза нею орудували своїм карликовим
інструментом. Залишаючи осторонь подорожчання машин у
наслідок такого способу виникнення їх, — обставина, що, як свідомий
мотив, панує над капіталом, — поширення промисловости,
яка провадилася вже машиновим способом, та проходження машин
у нові галузі продукції лишалися таким чином цілком залежними
від зросту тієї категорії робітників, яка через напівмистецький
характер своєї праці могла збільшуватися тільки поступінно,
а не скоками. Але на якомусь певному ступені розвитку велика
промисловість стає і технічно в суперечність із своєю ремісничою
та мануфактурною основою. Збільшення розміру рухових
машин, передатного механізму та виконавчих машин, збільшення
складности, різноманітности і точної правильности складових
частин виконавчої машини в міру того, як вона відривається від
того ремісничого зразка, що спочатку цілком визначає її будову, і
дістає вільну форму, яку визначає тільки її механічне завдання\footnote{
Механічний ткацький варстат у своїй першій формі складається
переважно з дерева, поліпшений, сучасний — із заліза. До якої міри стара
форма засобу продукції напочатку опановує його нову форму, показує,
між іншим, найповерховіше порівняння сучасного парового ткацького
варстату з давнім, або сучасних роздмухових пристроїв на ливарнях
з першим безпорадним механічним відродженням звичайного ковальського
міха, і, може, ще влучніше, ніж усе інше, перший льокомотив, що його
пробували збудувати ще перед винаходом теперішніх льокомотивів: у
нього було дійсно таки дві ноги, які він навпереміну підносив, як кінь.
Тільки з дальшим розвитком механіки та з нагромадженням практичного
досвіду форма машини починає цілком визначатися принципами механіки
і тим то цілком емансипується від давньої форми того знаряддя, яке розвивається
на машину.
},
\index{i}{0313}  %% посилання на сторінку оригінального видання
розвиток автоматичної системи та щораз неминучіше вживання
матеріялу, який важко переробити, наприклад, заліза замість
дерева, — розв’язання всіх цих завдань, що виникали стихійно,
натрапляло всюди на особисті перешкоди, які навіть комбінований
робітничий персонал мануфактури міг усунути лише
до певного ступеня, але не по суті. Таких машин, як от, наприклад,
сучасний друкарський прес, сучасний паровий ткацький
варстат та сучасна чухральна машина, не могла дати
мануфактура.

Переворот у способі продукції в одній сфері промисловості
зумовлює переворот у способі продукції і в інших сферах. Це
має силу насамперед для таких галузей промисловости, які,
хоч суспільним поділом праці так ізольовані, що кожна з них
продукує якийсь самостійний товар, проте переплітаються одна
з однією як фази одного якогось цілого процесу. Так, машинове
прядіння зробило доконечним машинове ткання, а одне й друге
разом — механічно-хемічну революцію в білінні, друкуванні
та фарбуванні. Таким саме чином, з другого боку, революція в
прядінні бавовни зумовила винахід gin’a, машини до відділювання
бавовняних волокон від насіння, через що лише й зробилася
можлива продукція бавовни в потрібному тепер великому маштабі\footnote{
Cottongin\footnote*{
Машина, що вибирає зерно з бавовни. \emph{Ред.}
}, винайдений одним янкі, Елія Вайтнеєм, до найновіших
часів у головному зазнав менше змін, ніж якабудь інша машина XVIII віку.
Лише останніми десятиліттями (перед 1867~\abbr{р.}) другий американець, пан
Імрі з Альбані, в Нью-Йорку, за допомогою простого й доцільного
поліпшення зробив машину Вайтнея антикварною річчю.
}.
А революція у способі продукції промисловости і рільництва
зробила доконечною й революцію в загальних умовах суспільного
процесу продукції, тобто в засобах комунікації і транспорту.
Як засоби комунікації й транспорту суспільства, що його
pivot\footnote*{
Точка, що довкола неї все обертається, стрижень. \emph{Ред.}
}, уживаючи вислову Фур’є, було дрібне рільництво з його
домашньою допомічною промисловістю та міське ремество, уже
ніяк не могли задовольняти потреб продукції мануфактурного
періоду з його поширеним поділом суспільної праці, з його концентрацією
засобів праці та робітників і з його колоніальними
ринками, — а тому й дійсно зазнали перевороту, — так само й
засоби транспорту й комунікації, що перейшли від мануфактурного
періоду, перетворились незабаром у нестерпні пута для
великої промисловости з її гарячковим темпом продукції, з її масовими
розмірами, з її постійним перекидуванням мас капіталу й
робітників з однієї сфери продукції до іншої та з її новостворюваними
світовими ринковими зв’язками. Тим то, не кажучи вже
\parbreak{}  %% абзац продовжується на наступній сторінці
