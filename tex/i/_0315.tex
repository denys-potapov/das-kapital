
\index{i}{0315}  %% посилання на сторінку оригінального видання
Якщо ми тепер розглянемо в машині, що її вживають на будування
машин, ту частину, яка становить виконавчу машину у
власному значенні слова, то перед нами знову з’явиться ремісничий
інструмент, тільки в циклопічному розмірі. Приміром, виконавчий
механізм бормашини — це велетенське свердло, яке рухає
парова машина й без якого, навпаки, не можна було б продукувати
циліндрів великих парових машин та гідравлічних пресів.
Механічний токарський варстат — це циклопічне відродження
звичайного ножаного токарського варстату; гембелівниця — це
залізний тесляр, що обробляє залізо тим самим знаряддям, яким
тесляр обробляє деревину; знаряддя, яке на лондонських корабельнях
ріже залізо на облицювальні пластини, — це велетенська
бритва; знаряддя різальної машини, що ріже залізо, як ножиці
кравця ріжуть сукно, — це велетенські ножиці, а паровий молот
оперує головкою звичайного молотка, але такої ваги, що його
не міг би піднести й сам Тор\footnote{
У Лондоні одна з таких кувальних машин paddle-wheelshafts\footnote*{
валів до лопатевих коліс. \emph{Ред.}
}
має назву Тор\footnote*{
Ім’я скандінавського бога блискавки, що його уявляли з великим
молотом у руці. \emph{Ред.}
}. Вона виковує вал вагою 16\sfrac{1}{2} тонн з такою самою легкістю,
як коваль підкову.
}. Наприклад, один з таких парових
молотів, які є винаходом Несміса, важить більше ніж 6 тонн
та падає простовисно з височини 7 футів на ковадло вагою в
36 тонн. Він, граючись, перетворює на порох ґранітну брилу та
й не менш здатний загнати гвіздок у м’яке дерево за допомогою
кількох послідовних легесеньких ударів\footnote{
Ті деревообробні машини, що їх можна вживати і в продукції
невеликого маштабу, є здебільша винахід американців.
}.

У машиновому механізмі знаряддя праці набуває такого матеріяльного
способу існування, який зумовлює заміну людської
сили силами природи та емпіричної рутини свідомим застосуванням
природознавства. В мануфактурі розчленування суспільного
процесу праці є суто суб’єктивне; це комбінація частинних робітників;
у системі машин велика промисловість має цілком об’єктивний
продукційний організм, що його робітник находить як
уже готову матеріяльну умову продукції. У простій кооперації і
навіть у кооперації, що поділом праці вже специфікована, витиснення
ізольованого робітника колективним робітником все ще
видається явищем більш-менш випадковим. Машини, з деякими
винятками, що про них треба нам згадати пізніше, функціонують
лише в руках безпосередньо усуспільненої або спільної праці.
Отже, кооперативний характер процесу праці стає тепер технічною
доконечністю, подиктованою самою природою засобу праці.

\subsection{Вартість, переношувана машиною на продукт}

Ми бачили, що продуктивні сили, які виникають із кооперації
й поділу праці, нічого не коштують капіталові. Вони є природні
\parbreak{}  %% абзац продовжується на наступній сторінці
