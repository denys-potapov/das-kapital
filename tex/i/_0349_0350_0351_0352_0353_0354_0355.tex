\parcont{}  %% абзац починається на попередній сторінці
\index{i}{0349}  %% посилання на сторінку оригінального видання
підручних на фабриці, то її можна заміняти почасти машинами\footnote{
Приклад: різні механічні апарати, які позаводжувано від часів
закону 1844~\abbr{р.} по вовняних фабриках для заміни дитячої праці. Скоро
тільки дітям самих панів фабрикантів доведеться проходити «їхню школу»
як підручним на фабриці, то ця майже незаймана сфера механіки одразу
розвинеться до дивовижних розмірів. «Ледве чи є ще така небезпечна
машина, як от selfacting-mule. Більша частина нещасливих випадків
трапляється з малими дітьми й саме через те, що вони підлазять під мюлі
тоді, як вони в русі, щоб замести долівку. Багатьох «minders» (робітників
при мюлях) притягли (фабричні інспектори) до судової відповідальности
та засудили на грошові кари за ці провини, але без будь-якої загальної
користи. Коли б будівники машин винайшли хоч одну машину
для замітання долівки, вживання якої звільнило б цих малих дітей від
потреби лазити під машини, то це було б щасливим додатком до наших
охоронних заходів». («Reports of Insp. of Factories for 31 st October
1866», p. 63).
},
а почасти вона дозволяє — через те, що вона зовсім проста —
хутко й постійно зміняти людей, обтяжених цими муками.

Хоч машина технічно знищує стару систему поділу праці,
все ж таки остання животіє на фабриці й далі, спочатку за звичкою,
як традиція мануфактури, а потім капітал систематично
репродукує та закріпляє її в ще огидливішій формі як засіб
експлуатації робочої сили. Довічна спеціальність орудувати
частинним знаряддям стає довічною спеціяльністю служити частинній
машині. Машинами зловживають, щоб самого робітника
від дитячих років перетворювати на частину частинної машини\footnote{
Після цього можна оцінити неймовірну вигадку Прудона, який
«конструює» машини не як синтезу засобів праці, а як синтезу частинних
праць для самих робітників. [Крім того, він робить остільки ж історичне,
як і дивовижне відкриття, що «машиновий період відзначається специфічним
характером, а саме найманою працею»]\footnote*{
Подане у прямих дужках ми беремо з французького видання. \emph{Ред.}
}.
}.
Таким чином не тільки значно зменшуються витрати, потрібні
для його власної репродукції, але й одночасно завершується його
безпорадна залежність від фабрики, як цілости, отже, від капіталіста.
Тут, як і всюди, треба розрізняти збільшення продуктивности,
що його зумовлює розвиток суспільного процесу продукції,
від того збільшення продуктивности, що його зумовлює
капіталістичний визиск цього процесу.

В мануфактурі і в реместві знаряддя служить робітникові,
на фабриці робітник служить машині. Там рух знаряддя праці
виходить від нього, тут — він має йти за його рухом. У мануфактурі
робітники є члени живого механізму. На фабриці існує
мертвий механізм незалежно від них, а їх додають до нього як
живі додатки. «Сумна одноманітність безконечної муки праці, з
якою той самий механічний процес знову й знову повторюється, є
подібна до муки Сізіфової; тягар праці немов скеля знову й знову
спадає на знесилених робітників»\footnote{
«\emph{F.~Engels}: «Die Lage der arbeitenden Klasse in England», Leipzig
1845, S. 217 (\emph{Ф.~Енґельс}: «Становище робітничої кляси в Англії»,
Партвидав «Пролетар», 1932~\abbr{р.}, стор. 187). Навіть цілком ординарний,
«оптимістичний фритредер, пан Молінарі, зауважує: «Людина, доглядаючи
щодня 15 годин за одноманітним рухом машини, виснажується
швидше, ніж коли вона протягом того самого часу працює фізично. Ця праця
догляду, яка, може, могла б бути за корисну гімнастику для розуму,
коли б не тривала надто довго, своєю надмірністю руйнує кінець-кінцем
і розум і саме тіло». («Un homme s’use plus vite en surveillant quinze
heures par jour l’évolution uniforme d’un mécanisme, qu’en exerçant dans
le même espace de temps, sa force physique. Ce travail de surveillance,
qui servirait peut-être d’utile gymnastique à l’intelligence, s’il n’etait pas
trop prolongé, détruit à la longue, par son excès, et l’intelligence et le corps
même»). (\emph{G. de Molinari}: «Etudes Economiques», Paris 1846, p. 49).
}. Виснажаючи до крайности
\index{i}{0350}  %% посилання на сторінку оригінального видання
нервову систему, машинова праця пригнічує багатобічну гру
мускулів і відбирає всяку змогу вільної фізичної й інтелектуальної
діяльности\footnote{
\emph{F.~Engels}, там же, стор. 216. (Партвидав «Пролетар», 1932~\abbr{р.},
стор. 186).
}. Навіть полегшення праці робиться засобом тортур,
бо машина не визволяє робітника від праці, а відбирає його
праці зміст. Всякій капіталістичній продукції, оскільки вона є
не тільки процес праці, але й процес зростання вартости капіталу,
є спільне те, що не робітник вживає умов праці, а, навпаки,
умови праці вживають робітника, але тільки при машиновій
системі це перекручення набуває технічно-очевидної реальности.
В наслідок свого перетворення на автомат засіб праці підчас
самого процесу праці протистоїть робітникові як капітал,
як мертва праця, що опановує й висисає живу робочу силу. Відокремлення
інтелектуальних сил процесу продукції від ручної
праці та перетворення їх у владу капіталу над працею завершується,
як ми вже раніш казали, у великій промисловості, побудованій
на основі машин. Частинна вправність індивідуального,
спустошеного машинового робітника зникає як незначна другорядна
річ перед наукою, перед велетенськими силами природи
і перед суспільною масовою працею, що втілені в системі машин
та разом з нею становлять владу «хазяїна» (master). Тому цей хазяїн,
у мозку якого машини нероздільно зрослися з його монополією
на них, у випадках колізії вигукує зневажливо до «рук»:
«Хай фабричні робітники в своїх власних інтересах запам’ятають,
що їхня праця в дійсності є дуже низький сорт навченої
праці; що жодної іншої праці не можна легше вивчити та що,
зважаючи на її якість, жодної праці не оплачується ліпше; що
жодної іншої праці не можна придбати за такий короткий час та
в такому великому розмірі, сяк-так привчивши найменш досвідчених
осіб. Машини хазяїна відіграють у дійсності далеко важливішу
ролю в справі продукції, ніж праця і вправність робітника,
яких можна навчитися за шість місяців і яких може навчитися
кожен сільський наймит»\footnote{
«The factory operatives should keep in wholesome remembrance
the fact that theirs is really a low species of skilled labour; and that there
is none which is more easily acquired or of its quality more amply remunerated,
or which, by a short training of the least expert can be more quickly
as well as abundantly acquired\dots{} The master’s machinery really plays
a far more important part in the business of production than the labour
and the skill of the operative, which six months education can teach, and
a common labourer саn learn». («The Master Spinners’and Manufacturer,
Defence Fund. Report of the Committee», Manchester 1854, p. 17). Пізніш
побачимо, що цей «хазяїн» співає іншої пісеньки, коли йому загрожує
небезпека втратити свої «живі» автомати.
}.

\index{i}{0351}  %% посилання на сторінку оригінального видання
Технічне підпорядковання робітника одноманітній ході засобу
праці і своєрідний склад робочого тіла з індивідів обох статей
і якнайрізнішого віку створюють казармову дисципліну, яка
розвивається на вивершений фабричний режим та цілком розвиває
раніш уже згадану працю нагляду, отже, разом з тим і
поділ робітників на ручних робітників та наглядачів за працею,
на рядових промислових солдатів, та промислових унтер-офіцерів.
«Головні труднощі в автоматичній фабриці\dots{} були\dots{} в дисципліні,
доконечній, щоб примусити людей одмовитися від їхньої
звички до нереґулярности в роботі та пристосувати їх до незмінної
реґулярности великого автомату. Алеж винайти та з успіхом
провести в життя дисциплінарний кодекс, що відповідав би
потребам та швидкості автоматичної системи — ця робота, гідна
Геркулеса, була благородним ділом Аркрайта! Навіть за наших
днів, коли цю систему зорганізовано в цілій її повноті, майже
неможливо знайти серед робітників, що вилюдніли вже в дозрілих
людей\dots{} корисних помічників для автоматичної системи»\footnote{
\emph{Ure}: «Philosophy of Manufacture», стор. 15. Хто знає біографію
Аркрайта, тому ніколи не спаде на думку назвати цього геніяльного
голяра «благородним». Поміж усіх великих винахідників XVIII віку
він безперечно був найбільший крадій чужих винаходів та наймерзенніший
суб’єкт.
}.
Фабричний кодекс, що в ньому капітал формулює свою автократію
над своїми робітниками приватноправним шляхом та самовладно,
без поділу влади, взагалі такого любого буржуазії, і
без ще улюбленішої репрезентативної системи, — цей кодекс є
лише капіталістична карикатура того суспільного реґулювання
процесу праці, яке стає потрібне при кооперації у великому маштабі
та при вживанні спільних засобів праці, особливо машин.
Місце батога в наглядача за рабами заступає карна книга наглядача.
Всі кари природно сходять до грошової кари й відраховань
із заробітної плати, а законодавча бистродумність фабричних
Лікурґів робить порушування їхніх законів, коли можливо,
ще прибутковішим для них, ніж додержування їх\footnote{
«Рабство, в кайданах якого буржуазія тримає пролетаріят, ніде
так ясно не виявляється, як у фабричній системі. Тут настає кінець
усякій свободі, і юридично, і фактично. Вранці о пів на шосту робітник
мусить бути на фабриці; якщо він спізниться на декілька хвилин, — на
нього накладають кари; якщо він спізниться на 10 хвилин — його зовсім
не впускають до фабрики до кінця сніданку, і він втрачає плату за
чверть дня. Він мусить на команду їсти, пити та спати\dots{} Деспотичний
дзвінок підіймає його з ліжка, відриває його від сніданку та обіду. А як
воно на фабриці? Тут фабрикант — абсолютний законодавець. Він видає
фабричні правила, як йому забажається; змінює та робить додатки
до свого кодексу, які йому захочеться; і хоч які безглузді ці зміни та
додатки, суди все ж таки кажуть робітникові: через те, що ви з доброї
волі згодилися на цей контракт, то мусите тепер його виконувати\dots{} Ці
робітники засуджені від дев’ятого року життя аж до самої смерти жити
під моральною та фізичною палицею». (\emph{F.~Engels}: «Die Lage der
arbeitenden Klasse in England», Leipzig 1845. S. 217 ff. — \emph{Ф.~Енґельс}:
«Становище робітничої кляси в Англії», Партвидав, 1932~\abbr{р.}, стор. 187 і
далі). Що «кажуть суди», це я поясню на двох прикладах. Один випадок
трапився в Шеффілді наприкінці 1866~\abbr{р.} Там один робітник найнявся
на два роки на металеву фабрику. Посварившися з фабрикантом,
він залишив фабрику й заявив, що ні в якому разі не працюватиме
більше на нього. Його оскаржили за зламання контракту й засудили
на два місяці в’язниці. (Коли фабрикант ламає контракт, то його можна
оскаржити лише перед цивільним судом і ризикує він лише грошовою
карою). Після того, як він одсидів ці два місяці, той самий фабрикант
закликає його на основі старого контракту повернутися до фабрики.
Робітник відмовляється. Він, мовляв, одбув уже кару за зламання контракту.
Фабрикант позиває його знову, суд знову засуджує його, дарма
що один із суддів, містер Ші, прилюдно назвав правничою потворністю,
що людину ціле життя можна періодично знову й знову карати за ту саму
провину або злочин. А цей присуд виніс не «Great Unpaid» (сільські
мирові судді), провінціяльні Dogberries, а один із найвищих судів у
Лондоні. [До четвертого видання. Тепер це скасовано. Тепер в Англії,
за винятком деяких випадків, наприклад, на громадських газівнях, робітник
за зламання контракту відповідає нарівні з підприємцем, і його
можна оскаржити лише перед цивільним судом. — \emph{Ф.~Е.}]. — Другий
випадок був у Wiltshire наприкінці листопада 1863~\abbr{р.} Якихось З0 робітниць
при парових ткацьких варстатах, працюючи в якогось Геррупа,
фабриканта сукна в Leoner’s Mill, Westbury, Leigh, улаштували страйк,
бо цей самий Герруп мав приємну звичку стягати з заробітної плати за
спізнення вранці, а саме 6\pens{ пенсів} за 2 хвилини, 1\shil{ шилінґ} за 3 хвилини й
1\shil{ шилінґ} 6\pens{ пенсів} за 10 хвилин. При 9\shil{ шилінґах} за годину це становить
4\pound{ фунти стерлінґів} 10\shil{ шилінґів} на день, тимчасом як їхня пересічна річна
плата ніколи не перевищує 10--12\shil{ шилінґів} на тиждень. Герруп доручив
також одному підліткові повідомляти трубою про фабричні години,
а цей іноді робив це перед шостою годиною вранці; якщо ж руки не з’являлися
саме тоді, коли він кінчав, то брама замикалась, а на тих, що залишалися
за брамою, накладали грошову кару; а що на фабриці не було
годинника, то нещасні руки попадали під владу молодого вартового,
інспірованого від Геррупа. Руки, що почали «страйк», матері родин
та дівчата, заявили, що знову стануть до праці, якщо вартового замінять
годинником та заведуть раціональніший карний тариф. Герруп
оскаржив 19 жінок та дівчат перед судом за зламання контракту. Серед
голосного обурення авдиторії кожну з них засудили до 6\pens{ пенсів} грошової
кари та до 2\shil{ шилінґів} 6\pens{ пенсів} судових витрат. Народна маса провела
Геррупа з суду з шиканням. Одна з улюблених операцій фабрикантів
— це карати робітників відрахуванням із заробітної плати за
кепську якість постачуваного їм матеріялу. Ця метода викликала в 1866~\abbr{р.}
загальний страйк в англійських ганчарняних округах. Звіти «Children’s
Employment Commission» (1863--1866) наводять випадки, коли робітник,
замість одержувати плату, ставав через свою працю та за допомогою
карного регламенту ще й винуватцем своїх ясновельможних «хазяїнів».
Повчальні риси бистродумности фабричних автократів у справі відрахувань
з заробітної плати дала також найновіша бавовняна криза.
«Я сам, — каже фабричний інспектор Р.~Бекер, — мусив нещодавно розпочати
судовий процес проти одного бавовняного фабриканта, бо він у цей тяжкий
та лютий час стягав з кількох «молодих» (понад тринадцять років)
робітників, що в нього працювали, по 10\pens{ пенсів} за лікарську посвідку
про вік, яка коштує йому тільки 6\pens{ пенсів} і за яку закон дозволяє стягати
лише 3\pens{ пенси}, а звичай — нічого\dots{} Другий фабрикант, щоб досягти
тієї самої мети без конфлікту з законом, накладає на кожну бідну дитину,
що на нього працює, данину в 1\shil{ шилінґ} за вивчення вмілости та
таємниці прядіння, скоро тільки лікарська посвідка визнає її за дозрілу
виконувати цю працю. Отже, десь на споді існують течії, що їх треба знати,
щоб зрозуміти такі надзвичайні явища, як страйки за таких часів, як
наші» (мовиться про страйк механічних ткачів на фабриці в Darwen у
червні 1863~\abbr{р.}) «Reports of Insp. of Fact, for 30 th April 1863», p. 50,
51. (Фабричні звіти завжди йдуть далі, ніж їхні офіціяльні дати).
}.

\index{i}{0352}  %% посилання на сторінку оригінального видання
Ми відзначаємо тут лише матеріяльні умови, за яких виконується
фабрична праця. Всі чуттьові органи однаково страждають
від штучно підвищеної температури, від повітря, заповненого
відпадками сировинного матеріялу, від оглушливого шуму
й~\abbr{т. д.}, не кажучи вже про небезпеку для життя серед густо
поставлених машин, які з реґулярністю пір року подають свої
\index{i}{0353}  %% посилання на сторінку оригінального видання
промислові бюлетені про вбитих та покалічених\footnoteA{
Закони для охорони від небезпечних машин дали добрі наслідки.
«Але\dots{} тепер існують нові джерела нещасливих випадків, які ще не існували
перед 20 роками, а саме збільшена швидкість машин. Колеса, вали,
веретена і ткацькі варстати женуть тепер із збільшеною силою, яка щораз
зростає: пучки мусять швидше та певніш хапати увірвану нитку, бо
досить вагання або необережности і будуть жертви\dots{} Багато нещасливих
випадків зумовлюються старанням робітників швидше скінчити свою
працю. Треба собі пригадати, що для фабрикантів дуже важливо тримати
свої машини в безнастанному русі, тобто безупинно продукувати пряжу й
тканини. Кожна зупинка на одну хвилину — це втрата не тільки на рушійній
силі, але й на продукції. Тому наглядачі за працею, заінтересовані
в кількості продукції, підганяють робітників тримати машини в русі,
а це не менш важливо і для робітників, яким платять від ваги або від
штуки. Тому, хоч у більшості фабрик формально й заборонено чистити
машини підчас їхнього руху, на практиці це загальне явище. Сама ця
обставина спричинила за останні шість місяців 906 нещасливих випадків\dots{}
Хоч чищення відбувається день-у-день, але все ж ґрунтовне чищення
машин призначають здебільша на суботу, і воно відбувається здебільша
підчас руху машин\dots{} За цю операцію не платять, і через те робітники
силкуються якомога швидше її скінчити. Тим то число нещасливих
випадків у п’ятницю й особливо в суботу далеко більше, ніж іншими
днями тижня. У п’ятницю число нещасливих випадків перевищує пересічне
число за перші чотири дні тижня приблизно на 12\%, у суботу ж
число нещасливих випадків перевищує пересічне число за попередні
п’ять день на 25\%; але коли взяти на увагу, що фабричний день у суботу
налічує лише 7\sfrac{1}{2} годин, а інші дні тижня 10\sfrac{1}{2}, то це
перевищення становитиме більше ніж 65\%». («Reports of Insp. of Fact, for
31st October 1866». London 1867, p. 9, 15, 16, 17).
}. Заощадження
суспільних засобів продукції, що вперше визріває у фабричній
системі, немов у теплиці, перетворюється в руках капіталу разом
з тим на систематичне грабування життєвих умов робітника
підчас його праці, на грабування простору, повітря, світла
та засобів охорони робітника від небезпечних для життя або антигігієнічних
умов продукційного процесу; про влаштування якихось
вигід для робітника нічого й казати\footnote{
У першому відділі третьої книги я розповім про похід англійських
фабрикантів, що стосується до недавнього часу, проти тих статтей фабричного
закону, які мають захищати члени «рук» від небезпечних для життя
машин. Тут досить однієї цитати з офіційного звіту фабричного інспектора
Леонарда Горнера: «Я чув, як фабриканти з непробачливою легкістю
говорили про деякі нещасливі випадки, наприклад, втрата одного пальця —
це, мовляв, дрібничка. Життя й надії робітника так дуже залежать від
його пальців, що така втрата є для нього надзвичайно серйозна подія.
Слухаючи таке безглузде базікання, я питав їх: «Припустіть, що вам потрібен
один додатковий робітник і до вас з’явилося двоє, обидва з усякого
іншого погляду однаково дужі, але один не має великого або вказівного
пальця, то котрого з них ви вибрали б?» І хвилини не вагаючись, вони
висловилися за того, що має всі пальці\dots{} Ці пани фабриканти мають
фалшиві упередження проти того, що вони називають псевдофілантропічним
законодавством». («Reports of Insp. of Fact. for 31 st October
1855»). Ці пани — меткі людці, і не дурно мріють вони про бунт рабовласників!
}. Чи не правду казав Фур’є, називавши фабрики «пом’якшеною каторгою»\footnote{
На фабриках, здавна підлеглих фабричному законові з його примусовим
обмеженням робочого часу та іншими його постановами, деякі
давніші лиха зникли. Саме поліпшення машин вимагає на якомусь певному
пункті «поліпшеної конструкції фабричних будівель», що йде на
користь робітникам. (Порівн. «Reports etc. for 31 st October 1863», p. 109).
}.

\index{i}{0354}  %% посилання на сторінку оригінального видання
\subsection{Боротьба між робітником і машиною}

Боротьба між капіталістом і найманим робітником починається
разом із виникненням самого капіталістичного відношення. Вона
лютує протягом цілого мануфактурного періоду\footnote{
Див. між іншим: \emph{John Houghton}: «Husbandry and Trade improved»,
London 1727. «The Advantages of the East India Trade», 1720.
\emph{John Belters}: «Proposals for raising a Colledge of Industry», London
1696. «Хазяїни та робітники, на жаль, перебувають у постійній війні
між собою. Незмінна мета хазяїнів — діставати працю для себе якомога
дешевше, і вони не вагаються пускатись на всякі хитрощі, щоб досягти
цієї мети, тимчасом як робітники з такою самою впертістю використовують
усяку нагоду, щоб примусити своїх хазяїнів виконати їхні підвищені
вимоги». («The masters and their workmen are unhappily in a perpetual
state of war with each other. The invariable object of the former is to
get their work done as cheap as possibly: and they do not fail to employ
every artifice to this purpose, whilst the latter are equally attentive to
every occasion of distressing their masters into a compliance with higher
demands»). «An Inquiry into the causes of the Present High Prices of
Provisions», 1767, p.61, 62. (Автор — панотець Натаніел Ферстер, що цілком
стоїть на боці робітників).
}. Але лише із заведенням машин робітник починає боротися проти самого
засобу праці, цієї матеріяльної форми існування капіталу. Він
повстає проти цієї певної форми засобу продукції як матеріяльної
основи капіталістичного способу продукції.

Мало не ціла Европа пережила у XVII віці повстання робітників
проти так званої Bandmühle (або інакше Schnurmühle
або Mühlenstuhl»)\footnote*{
стьожковий млин. \emph{Ред.}
}, проти машини для ткання стьожок та брузументу\footnote{
Bandmühle винайдено в Німеччині. Італійський абат Лянчелотті
у своїй праці, що появилась у Венеції 1636~\abbr{р.}, оповідає: «Антон Міллер
з Данціґу якихось 50 років тому (Лянчелотті писав 1579~\abbr{р.}) бачив у Данціґу
дуже мудру машину, яка виготовляла одночасно 4--6 тканин; міська
рада, турбуючися про те, що цей винахід може поробити масу робітників
старцями, затаїла цей винахід, а винахідника наказала потайки задушити
або втопити». В Ляйдені таку саму машину вперше вжито 1629~\abbr{р.} Заколоти
серед брузументників примусили маґістрат спочатку заборонити її;
генеральні штати своїми різними постановами з 1623, 1639~\abbr{рр.} і~\abbr{т. д.}
мали обмежити її вживання; нарешті, її дозволено під певними умовами
постановою 15 грудня 1661~\abbr{р.} «У цьому місті, — каже Боксхерн («Inst.
Роl. 1663») про заведення стьожкової машини в Ляйдені, — якихось 20 років
тому винайдено ткацький варстат, на якому один робітник міг легше й
більше виробляти тканин, аніж багато робітників за той самий час без
варстату. Але це спричинилося до скарг та заколотів серед ткачів, і магістрат,
нарешті, заборонив уживати машини» («In hac urbe ante hos viginti
circiter annos instrumentum quidam invenerunt textorium, quo solus quis
plus panni et facilius conficere poterat, quam plures aequali tempore. Hinc
turbae ortae et querulae textorum, tandemque usus hujus instrumenti a
magistratu prohibitus est»). (\emph{Boxhorn}: «Institutiones politicae», Leyden
1663). Ту саму машину 1676~\abbr{р.} заборонено в Кельні, тимчасом як введення
її в Англії в той самий час спричинилось до заколотів серед робітників.
Королівським едиктом з 19 лютого 1685~\abbr{р.} заборонено вживати її по всій
Німеччині. В Гамбурзі з наказу маґістрату її прилюдно спалили. Карл VI
відновив 9 лютого 1719~\abbr{р.} едикт з 1685~\abbr{р.}, а в саксонському курфюрстві
загальний вжиток її дозволено лише 1765~\abbr{р.} Ця машина, що наробила в
світі стільки шуму, була в дійсності попередницею прядільних і ткальних
машин, отже, і промислової революції XVIII віку. Вона робила цілком
недосвідченого у ткацтві підлітка здатним пускати в рух цілий
варстат з усіма його човниками, через саме лише совання рушієм туди
й назад; у своїй поліпшеній формі вона давала воднораз 40--50 сувоїв
тканини.
}. Наприкінці першої третини XVII віку чернь підчас
заколотів знищила вітряну лісопильню, поставлену якимось голляндцем
коло Лондону. Ще на початку XVII віку тартаки,
\index{i}{0355}  %% посилання на сторінку оригінального видання
гнані водою, лише насилу перемагали в Англії народній опір,
що його підтримував парлямент. Коли 1758~\abbr{р.} Іврі збудував
першу машину стригти овець, гнану водою, її спалили кілька сот
людей, які опинилися без праці. Проти scribbling mills\footnote*{
машин для першого, грубого чесання вовни. \emph{Ред.}
} та чухральних машин Аркрайта \num{50.000} робітників, які досі жили з
чухрання вовни, звернулися з петицією до парляменту. Масове
руйнування машин в англійських мануфактурних округах протягом
перших 15 років XIX віку, зумовлене вживанням парових
варстатів і відоме під назвою руху луддітів, дало антиякобінському
урядові Сідмавта, Castlereagh’a та інших привід до
якнайреакційніших насильницьких кроків. Треба часу й досвіду,
щоб робітник навчився відрізняти машину від капіталістичного
вживання її, а тому й переносити свої напади з самих матеріяльних
засобів продукції на суспільну форму експлуатації їх\footnote{
У старомодних мануфактурах ще й за наших часів повторюються
інколи грубі форми обурення робітників проти машин. Так, наприклад,
у виробництві напилків у Шеффілді 1865~\abbr{р.}
}.

Боротьба за заробітну плату в мануфактурі припускає наявність
мануфактури й зовсім не скерована проти її існування. Якщо
хто і боровся проти утворення мануфактур, то робили це не наймані
робітники, а цехові майстри й упривілейовані міста. Тому письменники
мануфактурного періоду розуміють поділ праці переважно
як засіб заміняти робітників у можливості, а не в дійсності витискувати
з мануфактур робітників\footnote*{
У французькому виданні це речення подано так: «Письменники
мануфактурного періоду в поділі праці вбачають можливий засіб поповнювати
недостачу в робітниках, а не витискувати з мануфактур робітників,
що вже працюють». \emph{Ред.}
}. Ця ріжниця сама собою
\parbreak{}  %% абзац продовжується на наступній сторінці
