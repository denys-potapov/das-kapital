\parcont{}  %% абзац починається на попередній сторінці
\index{i}{0371}  %% посилання на сторінку оригінального видання
телеграфію, фотографію, пароплавство та залізниці. Перепис
1861~\abbr{р.} (для Англії та Велзу) подає для газової промисловости
(газові заводи, продукція механічних апаратів, аґенти газових
компаній і~\abbr{т. ін.}) \num{15.211} осіб, для телеграфії — \num{2.399}, для фотографії
— \num{2.366}, для служби на пароплавах — \num{3.570} та для залізниць —
\num{70.599}, куди входять приблизно \num{28.000} більш-менш постійно
занятих «ненавчених» землекопів і цілий адміністративний і комерційний
персонал. Отже, загальне число індивідів у цих п’ятьох
нових галузях промисловости — \num{94.145}.

Нарешті, надзвичайно підвищена продуктивна сила в сферах
великої промисловости, супроводжувана, як ми це спостерігаємо,
інтенсивним та екстенсивним збільшенням визиску робочої
сили по всіх інших сферах продукції, дає змогу непродуктивно
вживати щораз більшу й більшу частину робітничої кляси й таким
чином репродукувати щораз більшими масами стародавніх домашніх
рабів під назвою «кляси слуг», як от слуг, покоївок, льокаїв
і~\abbr{т. ін.} За переписом 1861~\abbr{р.} вся людність Англії й Велзу налічувала
\num{20.066.244} особи, з того \num{9.776.259} чоловіків та \num{10.289.965} жінок.
Якщо від цього відлічити всіх тих, що застарі або замолоді
для праці, всіх «непродуктивних» жінок, підлітків і дітей,
далі «ідеологічні» професії, як от урядовців, попів, юристів,
військових тощо, потім усіх тих, що їхнє виключне заняття є
споживання чужої праці в формі земельної ренти, процентів і~\abbr{т. ін.}, насамкінець, павперів, волоцюг, злочинців і~\abbr{т. ін.}, то залишається
приблизно 8 мільйонів осіб обох статей та найрізнішого
віку, залічуючи сюди й усіх капіталістів, що так або інакше
функціонують у продукції, торговлі, фінансах тощо. З цих 8 мільйонів
припадає на:

\bigskip
\begin{small}
\noindent\begin{tabularx}{\textwidth}{@{}Xr@{}l}

\makehangcell{Рільничих робітників (залічуючи сюди пастухів та
наймитів і наймичок, що~живуть у фармерів)\dotfill{}} & \num{1.908.261} & ~осіб \\

\makehangcell{Всіх, що працюють на бавовняних, вовняних, напіввовняних,
лляних, конопляних, шовковихі джутових фабриках, на механічних в’язальнях
панчіх та коло фабрикації мережива\dotfill{}} & \num{642.607} & \footnote{
З того чоловіків, старших від 13 років, лише \num{177.596}.
} \dittomark \\

Всіх, що працюють по копальнях та руднях\dotfill & \num{565.835} & \\

\makehangcell{Всіх, що працюють на металюрґійних заводах
(домни, вальцювальні тощо) та металевих мануфактурах усякого
роду\dotfill{}} & \num{396.998} & \footnote{
З того жінок \num{30.501}.
} \dittomark \\

Клясу слуг\dotfill & \num{1.208.648} & \footnote{
З того чоловіків \num{137.447}. З цього числа в \num{1.208.648} виключено ввесь
персонал, що служить не у приватних осіб.

Додаток до другого видання. Від 1861~\abbr{р.} до 1870~\abbr{р.} число слуг-чоловіків
майже подвоїлося. Воно зросло до \num{267.671}. 1847~\abbr{р.} сторожів дичини
було \num{2.694} (в аристократичних мисливських парках), а 1869~\abbr{р.} — \num{4.291}.
— Молодих дівчат, що служать у лондонських дрібних буржуа, народньою
мовою називають «little slaveys» — маленькі
рабині\footnote*{
Тут у власному Марксовому примірнику I німецького видання є
така цитата з «Evening Star» від 11 вересня 1868~\abbr{р.}: «Як виснажують
надмірною працею молодих служниць, це — ганьба для їхніх господинь.
Випадково я знайомий з багатьма з цих «рабинь», як їх дехто називає,
і співчуваю їм від усього серця. Вони мусять рано вставати та працювати
до самісінької ночі. Вони сплять у підвальних комірках із нечистю або
по горищах із пацюками. Вони харчуються покидьками. Їх лають і шельмують,
їх переслідують брутальні хазяйські сини, їх мучать 4 або 5 дітей;
під дощ їх ганяють по пиво, інколи їх б’ють розгнівані господині. Тижнями
їм не дозволяють піти до церкви. Їм платять дуже мало; якщо вони
захоріють, їх відсилають до їхніх родичів, коли в них є родичі, абож до
шпиталю, або до притулку для бідних. Не диво, що вони мають острах
і огиду до пристойної праці і готові «піти світ за очі, хоч к чорту», і це
вони, ці бідолашні маленькі рабині, залюбки й роблять. Я бачив, як вони
плакали, оповідаючи про свої страждання, побої, голод і холод, про те,
як їх прогнали з їхнього «місця», коли вони захоріли, як жили вони
тоді з продажу свого одягу, і як, нарешті, коли все було продано, вони
утопли в мерзоті, дедалі більше занепадаючи. На жаль, лише дехто їм
співчуває». \emph{Ред.}
}.
} \dittomark \\
\end{tabularx}
\end{small}
\bigskip

%% посилання на сторінку оригінального видання
\noindent{}\index{i}{0372}Якщо додамо всіх тих, що працюють по текстильних фабриках,
до персоналу копалень та рудень, то матимемо \num{1.208.442};
якщо ж число перших додамо до персоналу всіх металюрґійних
заводів і мануфактур, то матимемо загальне число \num{1.039.605} —
в обох випадках менше, ніж число сучасних хатніх рабів. От
який величний результат капіталістичної експлуатації машин!

\subsection{Відштовхування і притягування робітників із розвитком
машинового виробництва. Кризи в бавовняній промисловості}

Всі серйозні представники політичної економії визнають,
що заведення в життя машин впливає неначе чума на робітників
у тих традиційних ремествах і мануфактурах, з якими машина
насамперед починає конкурувати. Майже всі вони бідкаються
над рабством фабричного робітника. Але який той великий козир,
що ним усі вони козиряють? Це те, що машини після всіх страхіть
періоду заведення їх у життя та розвитку їх, кінець-кінцем,
не зменшують, а збільшують число рабів праці! Так, політична
економія захоплюється огидною теоремою — огидною для всякого
«філантропа», що вірує у вічну природну доконечність капіталістичного
способу продукції, — теоремою, що навіть фабрика,
яка вже основана на машиновому виробництві, після певного
періоду зросту, після коротшого або довшого «переходового
часу» починає мучити більше число робітників, ніж те, яке вона
первісно викинула на брук!\footnote{
Ґаніль, навпаки, вважає за остаточний результат машинового виробництва
абсолютне зменшення числа рабів праці, що їхнім коштом годується
потім збільшене число «gens honnêtes»\footnote*{
порядних людей. \emph{Ред.}
}, що розвивають свою відому
«perfectibilité perfectible»\footnote*{
здібну вдосконалюватися здібність до вдосконалення. \emph{Ред.}
}, [що її так надхненно висміяв Фур’є]\footnote*{
Заведений у прямі дужки кінець речення беремо з французького
видання. \emph{Ред.}
}. Хоч і як мало він розуміє рух продукції, а все ж він принаймні почуває,
що машини дуже фатальна інституція, скоро заведення їх перетворює
занятих робітників на павперів, тимчасом як розвиток їх покликає до
життя більше рабів праці, ніж вони були вбили. Кретинізм його власного
погляду можна висловити лише його власними словами: «Les classes
condamnées à produire et à consommer diminuent, et les classes qui dirigent
le travail, qui soulagent, consolent et éclairent toute la population,
se multiplient\dots{} et s’approprient tous les bienfaits qui résultent de la diminution
des frais du travail, de l’abondance des productions et du bon marché
des consommations. Dans cette direction, l’espèce humaine s’élève aux plus
hautes conceptions du génie, pénètre dans les profondeurs mystérieuses de la
religion, établit les principes salutaires de la morale (яка є в тому, щоб «s’approprier
tous les bienfaits etc.»), les lois tutélaires de la liberté (liberté pour
«les classes condamnées à produire»?) et du pouvoir, de l’obéissance et de
la justice, du devoir et de l’humanité». («Кляси, засуджені на продукцію
та споживання, зменшуються, а кляси, що керують працею, дають поміч,
утіху та освіту цілому народові, збільшуються\dots{} та присвоюють собі
всі блага, що є результат зменшення витрат праці, рясности продуктів
та подешевшання предметів споживання. В цьому напрямі людський
рід підноситься до найвищих концепцій генія, доходить таємних глибин
релігії, встановлює спасенні принципи моралі (яка є в тому, щоб «присвоювати
собі всі блага й~\abbr{т. ін.}»), закони до охорони волі (волі для «кляс,
засуджених на продукцію»?) та влади, покірливости та справедливости,
обов’язку та гуманности»). Ці теревені маємо в «Des Systèmes d’Economie
Politique etc. Par M.~Ch.~Ganilh». 2 ème ed. Paris 1821, vol. I, p. 224.
Порівн. там же, стор. 212.
}

Правда, з деяких прикладів, як от на англійських фабриках
\index{i}{0373}  %% посилання на сторінку оригінального видання
суканої шерсти та шовкових фабриках, виявилося, що на певному
ступені розвитку надзвичайне поширення фабричних галузей
може сполучатися не тільки з відносним, але й з абсолютним
зменшенням числа вживаних робітників. 1860~\abbr{р.}, коли з наказу
парляменту розпочато спеціяльний перепис усіх фабрик Об’єднаного
Королівства, налічувалося 652 фабрики в тій частині
фабричних округ Ланкашіру, Чешіру та Йоркшіру, яку доручено
фабричному інспекторові Р.~Бекерові; з цих фабрик 570 мали:
парових ткацьких варстатів — \num{85.622}, веретен (за винятком веретен
на сукання) — \num{6.819.146}, кінських сил у парових машинах —
\num{27.439}, у водяних колесах — \num{1.390}, занятих осіб — \num{94.119}.
Навпаки, 1865~\abbr{р.} на цих самих фабриках було: ткацьких варстатів
— \num{95.163}, веретен — \num{7.025.031}, кінських сил у парових машинах
— \num{28.925}, у водяних колесах — \num{1.445}, занятих осіб —
\num{88.913}. Отже, від 1860~\abbr{р.} до 1865~\abbr{р.} зріст цих фабрик становив
у парових ткацьких варстатах 11\%, у веретенах — 3\%, у парових
кінських силах — 5\%, тимчасом як число занятих осіб за той
самий період зменшилося на 5,5\%\footnote{
«Reports of Insp. of Fact. for 31 st October 1865», p. 58 і далі.
Але одночасно було дано вже й матеріяльну базу для вживання чимраз
більшого числа робітників: було засновано 110 нових фабрик з \num{11.625} паровими
ткацькими варстатами, \num{628.756} веретенами, \num{2.695} паровими й водяними
кінськими силами (Там же).
}. Між 1852 і 1862~\abbr{рр.} сталося
значне збільшення англійської вовняної фабрикації, тимчасом як
число вживаних робітників лишилося майже без змін. «Це показує,
в якій великій мірі новозаведені машини витиснули працю
\parbreak{}  %% абзац продовжується на наступній сторінці
