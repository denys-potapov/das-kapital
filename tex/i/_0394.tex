\parcont{}  %% абзац починається на попередній сторінці
\index{i}{0394}  %% посилання на сторінку оригінального видання
якимось чарівним способом витворювати такі величезні маси продуктів
і «звільняти» такі величезні маси робітників.

Виробленням «wearing apparel» займаються ті мануфактури,
які репродукували в себе тільки той поділ праці, що його membra
disjecta\footnote*{
поодинокі члени. \emph{Ред.}
} вони находили готовими: крім того, дрібні ремісники-майстри,
що однак працюють не на індивідуальних споживачів,
як раніш, а на мануфактури й крамниці, так що часто цілі міста
й околиці заняті в таких галузях як своїм фахом, як от шевство
тощо; нарешті, у найбільшому розмірі цим займаються так звані
домашні робітники, які являють собою зовнішні відділи мануфактур,
крамниць, а то й майстерень дрібних майстрів\footnote{
Англійські millinery і dressmaking провадиться здебільша в
будинках підприємців, почасти найманими робітницями, що мешкають
у цих будинках, почасти поденницями, що мешкають осторонь.
}. Маси матеріялу праці, сировинного матеріялу, півфабрикатів тощо
постачає велика промисловість, маса дешевого людського матеріялу
(taillable à merci et miséricorde\footnote*{
відданого на ласку та гнів. \emph{Ред.}
}) складається із «звільнених»
великою промисловістю й рільництвом. Мануфактури
цієї галузі завдячували своє походження переважно потребі капіталістів
мати під рукою готову армію, що відповідала б кожному
рухові попиту\footnote{
Комісар Вайт відвідав одну мануфактуру військового одягу,
де працювало \num{1.000}--\num{1.200} осіб, майже всі жіночої статі, одну мануфактуру
чобіт з \num{1.300} особами, де майже половина були діти й підлітки, і~\abbr{т. ін.} («Children’s Employment Commission. 2nd Report», р. XVII,
n. 319).
}. Однак ці мануфактури поряд себе дозволяли й далі
існувати розпорошеному ремісничому і домашньому виробництву
як своїй широкій основі. Значну продукцію додаткової вартости
в цих галузях праці, разом із проґресивним здешевленням їхніх
продуктів, зумовило й зумовлюється, головним чином, мінімальною
заробітною платою, ледве достатньою для мізерного животіння
та сполученою з максимально можливим для людини робочим
часом. Саме дешевина людського поту й людської крови, перетворених
на товари, постійно поширювала й день-у-день далі поширює
ринок збуту, а для Англії особливо і колоніяльний ринок, де
до того ще й переважають англійські звички й англійський смак.
Нарешті, настав поворотний пункт. Основи старої методи, простої
брутальної експлуатації робочого матеріялу, що більше або
менше супроводилася систематично розвинутим поділом праці,
було вже недосить для ринку, що зростав, та для конкуренції
між капіталістами, що зростала ще швидше. Пробив час машини.
Такою вирішально революційною машиною, що рівномірно охопила
всю безліч галузей цієї сфери продукції, як от виробництво
модного вбрання, кравецтво, шевство, швацтво, капелюшництво
й~\abbr{т. ін.}, була швацька машина.

Її безпосереднє діяння на робітників приблизно таке, як і
всякої машини, що в періоді великої промисловости завойовує
\parbreak{}  %% абзац продовжується на наступній сторінці
