\parcont{}  %% абзац починається на попередній сторінці
\index{i}{0410}  %% посилання на сторінку оригінального видання
з такими ферментами перевороту та їхньою метою, знищенням
старого поділу праці. Однак розвиток суперечностей певної історичної
форми продукції — це єдиний історичний шлях її розкладу
й утворення нової. «Ne sutor ultra crepidam!»\footnote*{
Шевче, тримайся колодок своїх. \emph{Ред.}
} — це nec plus ultra\footnote*{
найвищий ступінь, апогей. \emph{Ред.}
} ремісничої премудрости, стало страшенною дурістю від
того моменту, коли годинникар Ватт вигадав парову машину,
голяр Аркрайт — прядільну машину, ювелірний робітник Фултон
— пароплав\footnote{
Джон Беллерс, справжній феномен в історії політичної економії,
ще наприкінці XVII віку з повною ясністю розумів доконечність знищити
теперішнє виховання й поділ праці, що породжують гіпертрофію й атрофію
на обох полюсах суспільства, хоч і в протилежному напрямі. Між
іншим, він чудово каже: «Вчитись у лінощах — це лише трохи щось ліпше,
ніж учитись лінощів\dots{} Фізична праця — це первісна божа установа\dots{}
Праця так само потрібна для здоров’я тіла, як харч для його життя;
бо ті неприємності, що їх людина уникає через лінощі, впадуть на неї
через недугу\dots{} Праця додає олії до лямпи життя, думання запалює її\dots{}
Дурненька дитяча праця (пророчий закид проти Базедових і сучасних
тупих наслідувачів їх) лишає дитячий розум дурненьким». («An idle
learning being little better than the Learning of Idlenes\dots{} Bodily Labour,
it’s a primitive institution of God\dots{} Labour being as proper for the
bodies health, as eating is for its living; for what pains a man saves by
Ease, he will find in Disease\dots{} Labour adds oyl to the lamp of life when
thinking inflames it\dots{} A childish silly employ, leaves the children’s minds
silly»). («Proposals for raising a Colledge of Industry of all useful Trades
and Husbandry», London 1696, p. 12, 14, 18).
}.

Оскільки фабричне законодавство реґулює працю по фабриках,
мануфактурах тощо, воно спочатку здається тільки втручанням
у експлуататорські права капіталу. Навпаки, всяке реґулювання
так званої домашньої праці\footnote{
Вона, зрештою, здебільша має характер домашньої праці і в
дрібних майстернях, як ми це бачили в мануфактурі мережива і в плетінні
з соломи і як це можна було б докладніше показати особливо на металевих
мануфактурах у Шеффілді, Бермінґемі й~\abbr{т. ін.}
} виявляється відразу ж
як прямий замах на patria potestas, тобто, висловлюючись сучасною
мовою, на батьківський авторитет, крок, що перед ним делікатний
і чутливий англійський парлямент з удаваним жахом
подавався назад. Однак сила фактів примусила, нарешті, визнати,
що велика промисловість руйнує разом з економічною основою
старої родини й відповідної їй родинної праці й самі старі родинні
відносини. Неминуче треба було оголосити право дітей. «На лихо,
— читаємо в кінцевому звіті «Children’s Employment Commission»
з 1866~\abbr{р.}, — з усіх виказів свідків ясно, що ні від кого
не треба так дуже боронити дітей обох статей, як від їхніх власних
батьків». Система безмірної експлуатації дитячої праці взагалі
й домашньої праці зокрема тим «підтримується, що батьки нестримно
й безконтрольно використовують самовільну й нещадну
владу над своїми молодими й тендітними нащадками\dots{} Не можна
давати батькам абсолютної влади робити з своїх дітей просто
машини, щоб добувати з них стільки та стільки тижневого заробітку\dots{}
\index{i}{0411}  %% посилання на сторінку оригінального видання
Діти й підлітки мають право на те, щоб закон захищав їх
проти зловживання батьківської влади, яке передчасно нищить
їхню фізичну силу й понижує їхній моральний та інтелектуальний
рівень»\footnote{
«Children’s Employment Commission. 5th Report», p. XXV,
n. 162 і 2nd Report, p. XXXVIII, n. 285, 289, p. XXXV, n. 191.
}. Однак не зловживання батьківською владою
створило цю безпосередню або посередню експлуатацію недозрілих
робочих сил капіталом; навпаки, капіталістичний спосіб
експлуатації, знищивши економічну основу, що відповідала батьківській
владі, викликав зловживання цією владою. Хоч і яким
страшним і огидливим з’являється розклад старої родини всередині
капіталістичної системи, а все ж велика промисловість,
призначаючи жінкам, підліткам і дітям обох статей вирішальну
ролю в суспільно-організованому процесі продукції поза сферою
хатнього господарства, створює нову економічну основу для
вищої форми родини й відносин поміж обома статями. Певна
річ, однаково абсурдно вважати за абсолютну форму родини її
християнсько-германську форму, як і староримську, або старогрецьку,
або східню, які, зрештою, становлять історичний ряд
розвитку. Так само ясно, що склад комбінованого робочого персоналу
з індивідів обох статей і найрізнішого віку, хоч він у своїй
грубій, стихійно виниклій капіталістичній формі, де робітник
існує для процесу продукції, а не процес продукції для робітника,
є отруйне джерело морального зіпсуття й рабства, — за відповідних
умов мусить, навпаки, перетворитись на джерело гуманного
розвитку\footnote{
«Фабрична праця могла б бути так само чиста і приємна, як домашня
праця, а то, може, навіть і більше» («Factory labour may be as
pure and as excellent as domestic labour, and perhaps more so»). («Reports
of Insp. of Fact, for 31 st October 1865», p. 127).
}.

Доконечність перетворити фабричний закон із виняткового
закону для пряділень і ткалень, цих перших витворів машинового
виробництва, на загальний закон усієї суспільної продукції, випливає,
як ми вже бачили, з історичного ходу розвитку великої промисловости,
на задньому пляні якої зазнають цілковитого перевороту
традиційні форми мануфактури, ремества й домашньої
праці: мануфактура постійно перетворюється на фабрику, ремество
постійно перетворюється на мануфактуру, і, нарешті,
сфери ремества й домашньої праці в дивовижно короткий час
перетворюються на злиденні трущоби, де необмежено панує найшаленіша,
потворна капіталістична експлуатація. Дві обставини
відіграють кінець-кінцем вирішальну ролю: поперше, спостереження,
яке постійно повторюється, що капітал, коли він підпадає
під державний контроль лише на поодиноких пунктах
суспільної периферії, тим безмірніше відшкодовує себе в інших
пунктах;\footnote{
Там же, стор. 27, 32.
} подруге, волання самих капіталістів про рівність
умов конкуренції, тобто про рівні межі експлуатації праці\footnote{
Масові приклади цього в «Reports of Insp. of Fact.».
}.
\parbreak{}  %% абзац продовжується на наступній сторінці
