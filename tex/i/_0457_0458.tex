\parcont{}  %% абзац починається на попередній сторінці
\index{i}{0457}  %% посилання на сторінку оригінального видання
собі зиск простим шахрайством: купівлею товарів нижче й продажем
їх понад їхню вартість. Тому він не доходить до зрозуміння
того, що коли б дійсно існувала така річ, як вартість праці, і
коли б він дійсно сплатив ту вартість, то не існувало б жодного
капіталу, його гроші не перетворювались би на капітал.

До того ж дійсний рух заробітної плати показує явища, які,
як здається, доводять, що оплачується не вартість робочої
сили, а вартість її функції, вартість самої праці. Ці явища можна
звести до двох великих кляс. Поперше: зміна заробітної плати
із зміною довжини робочого дня. З таким самим правом можна б
зробити висновок, що оплачується не вартість машини, а вартість
її операцій, бо дорожче коштує найняти машину на тиждень,
аніж на день. Подруге, індивідуальна ріжниця в заробітних
платах різних робітників, що виконують ту саму функцію. Цю індивідуальну
ріжницю ми бачимо, — однак без того, щоб вона давала
нагоду до ілюзій, — і за системи рабства, де саму робочу силу
продають ділком явно й вільно, без якихбудь прикрас. Тільки за
системи рабства вигода від робочої сили, вищої за пересічну якість,
або шкода від робочої сили, нижчої за пересічну якість, припадає
рабовласникові, а за системи найманої праці — самому
робітникові, бо в останньому випадку він сам продає свою робочу
силу, в першому її продає якась третя особа.

А втім, для такої форми виявлення, як «вартість та ціна праці»
або «заробітна плата», на відміну від того посутнього відношення,
що виявляється, тобто на відміну від вартости й ціни робочої
сили, має силу те саме, що й для всіх форм виявлення та захованої
за ними їхньої основи. Перші репродукуються безпосередньо
сами собою, як найпоширеніші форми мислення, другу мусить
відкрити тільки наука. Клясична політична економія доходить
близько до правдивого стану речей, однак не формулює його
свідомо. Їй і не сила зробити це, доки вона має на собі свою буржуазну
шкуру.

\section{Почасова плата}

Сама заробітна плата знов таки набирає дуже різноманітних
форм — обставина, що про неї не можна дізнатись з економічних
підручників, які у своїй грубій заінтересованості матерією нехтують
усякі ріжниці форм. Однак з’ясування всіх цих форм
належить до спеціяльної науки про заробітну плату, отже, не
може бути завданням цього твору. А все ж дві панівні основні
форми тут треба коротко розвинути.

Як ми пригадуємо, продаж робочої сили відбувається завжди
на певні періоди часу. Тому перетворена форма, в якій безпосередньо
виражається денна вартість, тижнева вартість і~\abbr{т. д.} робочої
сили, є форма «почасової плати», отже, поденна плата і~\abbr{т. д.}

Насамперед треба тут зауважити, що з’ясовані в п’ятнадцятому
розділі закони про зміну величини ціни робочої сили та додаткової
\index{i}{0458}  %% посилання на сторінку оригінального видання
вартости перетворюються через просту зміну форми
на закони заробітної плати. Так само ріжниця між міновою вартістю
робочої сили й масою засобів існування, на які обмінюється
ця вартість, з’являється тепер як ріжниця між номінальною та
реальною заробітною платою. Марно було б повторювати щодо
форми виявлення те, що вже розвинуто щодо посутньої форми.
Тому ми обмежуємось тими небагатьма пунктами, що характеризують
почасову плату.

Та грошова сума\footnote{
Вартість самих грошей тут завжди припускається за сталу.
}, що її робітник одержує за свою денну,
тижневу й так далі працю, становить суму його номінальної
заробітної плати, або заробітної плати, оцінюваної за її вартістю.
Але ясно, що, залежно від довжини робочого дня, отже,
залежно від кількости праці, яку він дає за день, та сама денна,
тижнева й так далі заробітна плата може репрезентувати дуже
різну ціну праці, тобто дуже різні грошові суми за ту саму кількість
праці\footnote{
«Ціна праці є сума, виплачена за певну кількість праці» — («The
price of labour is the sum paid for a given quantity of labour»).
(\emph{Sir Edward West}: «Price of Corn and Wages of Labour», London 1826,
p. 67). Вест є автор анонімного твору, епохального в історії політичної економії:
«Essay on the Application of Capital to Land. By a Fellow of the
University College of Oxford», London 1815.
}. Отже, при почасовій платі треба знову таки відрізняти
цілу суму заробітної плати, поденної, потижневої і так
далі, від ціни праці. Як же знайти цю ціну, тобто грошову вартість
даної кількости праці? Пересічну ціну праці ми добудемо,
поділивши пересічну денну вартість робочої сили на число годин
пересічного робочого дня. Якщо, приміром, денна вартість
робочої сили становить 3\shil{ шилінґи}, вартість, спродуковану протягом
6 робочих годин, а робочий день має 12 годин, то ціна однієї
робочої години дорівнює 3\shil{ шилінґам}/12 \deq{} 3\pens{ пенсам.} Знайдена
таким чином ціна робочої години служить за одиницю міри для
ціни праці.

Звідси випливає, що денна, тижнева і так далі заробітна плата
може лишатися однакова, хоч ціна праці постійно падатиме.
Якщо, наприклад, звичайний робочий день має 10 годин, а денна
вартість робочої сили рівна 3\shil{ шилінґам}, то ціна робочої години
становить 3\sfrac{3}{5}\pens{ пенсів}; вона спадає до 3\pens{ пенсів}, якщо робочий день
зростає до 12 годин, та до \sfrac{22}{5}\pens{ пенсів}, коли він зростає до 15 годин.
Денна або тижнева заробітна плата, не зважаючи на це,
лишаються незмінні. Навпаки, денна або тижнева заробітна
плата може підвищитися, хоч ціна праці лишатиметься стала або
навіть падатиме. Коли, приміром, робочий день має десять годин,
а денна вартість робочої сили становить 3\shil{ шилінґи}, то ціна однієї
робочої години дорівнює \sfrac{33}{5}\pens{ пенсів.} Якщо робітник у наслідок
збільшення роботи працює при незмінній ціні праці 12 годин,
то його денна плата зростає до 3\shil{ шилінґів} 7\sfrac{1}{5}\pens{ пенсів} без зміни
ціни праці. Той самий результат міг би постати, коли б замість
\parbreak{}  %% абзац продовжується на наступній сторінці
