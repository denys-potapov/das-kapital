\parcont{}  %% абзац починається на попередній сторінці
\index{i}{0486}  %% посилання на сторінку оригінального видання
продукційного процесу від того, що худоба сама споживає
те, що їсть. Постійне зберігання і репродукція робітничої кляси
лишається постійною умовою репродукції капіталу. Виконання
цієї умови капіталіст може спокійно полишити інстинктові
робітників до самозбереження й розмножування. Капіталіст
дбає лише про те, щоб якомога обмежити їхнє особисте споживання
на найдоконечнішому, і, як небо від землі, він далекий від
тієї південно-американської грубости, з якою робітників примушують
їсти поживніший харч замість менш поживного\footnote{
«Робітники в копальнях Південної Америки, що їхня щоденна
праця (найтяжча, мабуть, у світі) є в тому, щоб витягати на своїх плечах
на поверхню землі вантаж руди в 180--200 фунтів з глибини 450 футів,
харчуються лише хлібом та бобами; вони воліли б харчуватися
самим хлібом, але їхні пани, відкривши, що на самому хлібі вони не
можуть працювати так дуже, поводяться з ними, як із кіньми, і примушують
їх їсти боби; а боби далеко багатші на кісткову золу, ніж хліб».
(\emph{Liebig}: «Die Chemie in ihrer Anwendung auf Agrikultur und Physiologie»,
7 Auflage, 1862, част. 1, стор. 194, примітка).
}.

Тому капіталіст і його ідеолог, політико-економ, розглядають
як продуктивне споживання лише ту частину особистого споживання
робітника, що потрібна для увіковічнення робітничої
кляси, отже, що дійсно мусить бути спожита для того, щоб капітал
міг споживати робочу силу; а те, що робітник споживає поверх
того для своєї насолоди, є непродуктивне споживання\footnote{
\emph{James Mill}: «Eléments d’Economie Politique», Paris 1823, стор. 238
і далі.
}.
Коли б акумуляція капіталу спричинила підвищення заробітної
плати, а тому й збільшення засобів споживання робітника без
збільшеного споживання робочої сили капіталом, то додатковий
капітал був би спожитий непродуктивно\footnote{
«Коли б ціна на працю піднеслася так високо, що, не зважаючи
на приріст капіталу, не можна було б уживати більше праці, то я сказав
би, що такий приріст капіталу споживається непродуктивно» (\emph{Ricardo}:
«Principles of Political Economy», 3rd ed., London 1821, p. 163).
}. Справді, особисте
споживання робітника є для нього самого непродуктивне, бо воно
репродукує лише індивіда, що має потреби; воно є продуктивне
для капіталіста і для держави, бо воно є продукування сили,
що продукує чуже багатство\footnote{
«Єдине продуктивне споживання у власному значенні слова є
споживання або руйнування багатства (він має на думці споживання
засобів продукції) капіталістом з метою репродукції\dots{} Робітник\dots{} є
продуктивний споживач для особи, що вживає його, і для держави, але,
точно кажучи, не для себе самого». (\emph{Malthus}: «Definitions in Political
Economy», London 1853, p. 30).
}.

Отже, з суспільного погляду робітнича кляса, навіть поза
безпосереднім процесом праці, є так само приналежність капіталу,
як і мертве знаряддя праці. Навіть її особисте споживання
є в певних межах лише момент процесу репродукції капіталу.
Але цей процес, постійно віддаляючи продукт праці робітничої
кляси від її полюса до протилежного полюса капіталу, дбає
про те, щоб ці самосвідомі знаряддя продукції не втекли. Особисте
споживання робітників дбає, з одного боку, про їхнє власне збереження
\index{i}{0487}  %% посилання на сторінку оригінального видання
й репродукцію, а з другого боку, знищуючи засоби
існування, воно дбає про те, щоб вони постійно знову й знов
з’являлися на ринку праці. Римський раб був прикований
кайданами, а найманий робітник прив’язаний незримими нитками
до свого власника. Видимість його незалежности підтримує
постійна зміна індивідуальних панів-наймачів і юридична фікція
контракту.

Колись капітал, де це йому здавалося потрібним, здійснював
своє право власности на вільного робітника за допомогою примусового
закону. Так, наприклад, до 1815~\abbr{р.} еміґрацію машинобудівельних
робітників в Англії було заборонено під загрозою
тяжкої кари.

Репродукція робітничої кляси включає також передачу і
нагромаджування вправности від одного покоління до другого\footnote{
«Єдина річ, про яку можна сказати, що її нагромаджують і заздалегідь
підготовляють, — це вправність робітника\dots{} Акумуляція і нагромадження
вправної праці, ця найважливіша операція, провадиться щодо
великої маси робітників без жодного капіталу». (\emph{Hodgskin}: «Labour
Defended etc.» p. 13).
}.
До якої міри капіталіст вважає існування такої вправної
робітничої кляси за одну з належних йому умов продукції, розглядає
її в дійсності як реальне існування свого змінного капіталу,
виявляється тоді, коли криза загрожує йому її втратою.
Як відомо, в наслідок американської громадянської війни й
бавовняного голоду, що її супроводив, було викинуто на брук
більшість бавовняних робітників у Ланкашірі й~\abbr{т. ін.} З надр
самої робітничої кляси, як і з інших верств суспільства, залунав
заклик до державної допомоги та добровільних національних
пожертов, щоб уможливити еміґрацію «зайвих» робітників до
англійських колоній або до Сполучених штатів. Тоді «Times»
(24 березня 1863~\abbr{р.}) опублікував листа Едмунда Потера, колишнього
президента менчестерської торговельної палати. В Палаті
громад його лист цілком справедливо названо «маніфестом
фабрикантів»\footnote{
«Цей лист можна розглядати як маніфест фабрикантів» («Thal
letter might be looked upon as the manifesto of the manufacturers»).
(\emph{Ferrand}: «Подання з приводу бавовняного голоду, засідання Палати
громад з 27 квітня 1863~\abbr{р.}»).
}. Ми подаємо тут із нього деякі характеристичні
місця, де без прикрас говориться про право власности капіталу
на робочу силу.

«Бавовняним робітникам можуть сказати, що їх забагато на
ринку праці\dots{} що їх, може, треба б зменшити на одну третину,
і тоді настане нормальний попит на останні дві третини\dots{} Громадська
думка наполягає на еміґрації. Хазяїн (тобто бавовняний
фабрикант) не може добровільно згодитися на зменшення подання
праці; він вважає, що це було б так само несправедливо, як
і неправильно\dots{} Якщо еміґрацію підтримують із громадських
фондів, то він має право вимагати, щоб його вислухали, а може
і протестувати». Цей Потер пояснює далі, яка корисна бавовняна
промисловість, як «вона, безперечно, відтягла людність з Ірляндії
\index{i}{0488}  %% посилання на сторінку оригінального видання
та з англійських рільничих округ», яка вона величезна розміром,
як вона 1860~\abbr{р.} дала \sfrac{5}{13} усієї англійської експортної
торговлі, як вона через декілька років знову зросте через поширення
ринку, особливо індійського, і через примусовий достатній
«довіз бавовни по 6\pens{ пенсів} за фунт». Він каже далі: «Час — один,
два, може, три роки — випродукує потрібну кількість\dots{} І я хотів би
тоді поставити питання, чи варта ця промисловість того, щоб її
підтримувати, чи варто тримати машини (а саме живі робочі
машини) в порядку і чи не найбільша дурість думати про те, щоб
відмовитися від них? Я думаю, що так. Я визнаю, що робітники
не є власність («І allow that the workers are not a property»),
не власність Ланкашіру й хазяїнів; але вони — сила обох; вони
є духовна й навчена сила, що її не можна замістити протягом
життя однієї ґенерації; навпаки, інші машини, коло яких вони
працюють («the mere machinery which they work»), можна здебільшого
з користю замінити й поліпшити за дванадцять місяців\footnote{
Пригадаймо собі, що той самий капітал співає іншої пісеньки за
звичайних обставин, коли йдеться про зниження заробітної плати. Тоді
«хазяїни», як один, заявляють (див. четвертий відділ, примітку 188):
«Хай фабричні робітники в своїх власних інтересах запам’ятають, що їхня
праця в дійсності є дуже низький сорт навченої праці; що жодної іншої
праці не можна легше вивчити та що, зважаючи на її якість, жодної
праці не оплачується ліпше; що жодної іншої праці не можна придбати
за такий короткий час та в такому великому розмірі, сяк-так привчивши
найменш досвідчених осіб. Машини хазяїна (що їх, як ми тепер
чуємо, можна з користю замінити й поліпшити за дванадцять місяців)
відіграють у дійсності далеко важливішу ролю в справі продукції, ніж
праця і вправність робітника (яких тепер не можна замінити за тридцять
років), яких можна навчитися за шість місяців і яких може навчитися
кожен сільський наймит».
}.
Заохочуйте до еміґрації робочої сили, або дозволяйте (!)
її, алеж, що тоді станеться з капіталістом? («Encourage
or allow the working power to emigrate, and what of the capitalist?»
Цей крик серця нагадує придворного маршала Кальба)\dots{}
Зберіть вершки робітників — і основний капітал буде в значній
мірі зневартнений, а оборотний капітал не зможе боротися при
недостатньому поданні праці нижчого сорту\dots{} Нам кажуть,
що робітники сами бажають еміґрації. Це дуже природна річ,
що вони це роблять\dots{} Зменшуйте, придушуйте бавовняне виробництво,
відбираючи в нього робочі сили (by taking away its working
power), зменшуючи, приміром, на третину або на 5 мільйонів
видатки в заробітній платі, але що тоді станеться з найближчою
над робітниками вищою клясою, — з дрібними крамарями?..
Що станеться з земельними рентами, з платою за наймання котеджів?..
З дрібним фармером, кращим домовласником і землевласником?
А тепер скажіть, чи може бути якийсь плян самозгубніший
для всіх кляс країни, ніж оцей плян ослабити націю
експортом її найкращих фабричних робітників і зневартненням
частини її найпродуктивнішого капіталу й багатства?» «Я раджу
позику в 5--6 мільйонів, розподілену на 2 або 3 роки, що нею
порядкуватимуть приставлені до адміністрації опіки над бідними
\parbreak{}  %% абзац продовжується на наступній сторінці
