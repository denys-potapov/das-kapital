\parcont{}  %% абзац починається на попередній сторінці
\index{i}{0504}  %% посилання на сторінку оригінального видання
(«не має жодної дати»). Лише остільки його власна минуща
доконечність криється в минущій доконечності капіталістичного
способу продукції. Але остільки ж рушійним мотивом його
діяльности є не споживна вартість і не споживання, а мінова
вартість та її збільшення. Як фанатик зростання вартости, він
нещадно примушує людство до продукції задля продукції, отже,
до розвитку суспільних продуктивних сил і до створення тих
матеріяльних умов продукції, які тільки й можуть становити
реальну базу вищої суспільної форми, що її основний принцип
є повний і вільний розвиток кожного індивіда. Лише як персоніфікація
капіталу капіталіст є респектабельний. У цій ролі
він так само, як і збирач скарбів, пройнятий жагою абсолютного
збагачування. Але те, що в збирача скарбів становить індивідуальну
манію, у капіталіста є діяння суспільного механізму,
в якому він є лише одне колесо. Крім того, розвиток капіталістичної
продукції робить доконечним невпинне збільшення капіталу,
вкладеного в промислове підприємство, а конкуренція накидає
кожному індивідуальному капіталістові іманентні закони
капіталістичного способу продукції як зовнішні примусові закони.
Конкуренція примушує його невпинно збільшувати свій
капітал, щоб зберегти його, а збільшувати його він може лише
за допомогою проґресивної акумуляції.

Тим то, оскільки вся діяльність капіталіста є лише функція
капіталу, обдарованого в його особі волею і свідомістю, його
власне приватне споживання в його очах має значення грабежу
в акумуляції його капіталу подібно до того, як в італійській
бухгальтерії приватні видатки фігурують на сторінці дебету
капіталіста проти його капіталу. Акумуляція — це завойовання
світу суспільного багатства. Разом з масою експлуатованого людського
матеріялу вона поширює безпосереднє й посереднє панування
капіталіста\footnote{
На прикладі старомодної, хоч і постійно відновлюваної форми
капіталіста, — на прикладі лихваря, Лютер дуже добре унаочнює властолюбство
як елемент жадоби до збагачення. «Поганці могли збагнути
своїм розумом, що лихвар тричі злодій і душогуб. Ми ж, християни, так
шануємо їх, що мало не молимося на них задля їхніх грошей\dots{} Той, хто
висисає в другого його харч, хто грабує і краде, так само є душогубець
(оскільки це від нього залежить), як і той, що голодом мордує когось
та заганяє на той світ. Але лихвар робить усе це, і все ж він спокійно
сидить у своєму кріслі, хоч і мав би висіти на шибениці, де б його шматувало
стільки ворон, скільки він накрав золотих, якби тільки на ньому
було стільки м’яса, щоб усі ті ворони могли пошматувати те м’ясо та
поділити між собою. Малих злодіїв вішають на шибениці\dots{} Малих злодіїв
тримають по в’язницях, а великі ходять собі, пишаючися, в золоті
та шовках\dots{} Отже, немає й більшого ворога людини на землі (крім чорта),
як скнара та лихвар, бо він хоче бути богом над усіма людьми. Турки,
вояки, тирани теж лихі люди, однак вони мусять давати людям жити й
визнають, що вони лихі люди й вороги; вони можуть, і-навіть мусять
іноді змилуватися над деким. Але лихвар і скнара хотів би, щоб увесь
світ пропадав з голоду, спраги, суму й нужди; він хотів би все, що навколо
нього є, мати лише собі, щоб усяк діставав усе від нього, наче від бога,
і був навіки його кріпаком. Він носить пишні мантії, золоті ланцюжки
й персні, пестить свою пику, видає себе за добру побожну людину
й пишається цим\dots{} Лихвар же — страшелезна потвора, як той вовкулак,
що все плюндрує, гірший, ніж Какус, Геріон або Антус. Але він
прибирається й удає із себе побожного, щоб ніхто не бачив, де подіваються
ті воли, що їх він утягує задом у свій барліг. Але Геркулес повинен
чути, як ревуть воли й кричать полонені, повинен шукати Какуса навіть
у скелях і ярах, повинен визволити волів від лиходія. Бо Какус є лиходій,
і той лиходій — побожний лихвар, що все краде, грабує та пожирає.
І однак удає, ніби він нічого лихого не заподіяв, і ніхто не може викрити
його лиходійства, бо волів оін утягнув у свій барліг задом, і вони лишають
такі сліди, ніби їх випустили з барлогу. Так і лихвар хоче обдурити
світ, наче він дає користь і дає світові волів, тимчасом як він
захоплює їх собі й пожирає\dots{} І коли грабіжників, розбійників і напасників
колесують і стинав ть їм голови, то в скільки разів більше слід
би колесувати всіх лихварів, вимотувати з них жили\dots{} проганяти їх,
проклинати їх та стинати їм голови». (\emph{Martin Luther}: «An die Pfarrherrn,
wider den Wucher zu predigen», Wittenberg 1540).
}.

\index{i}{0505}  %% посилання на сторінку оригінального видання
Але первородний гріх діє повсюди. З розвитком капіталістичного
способу продукції, акумуляції й багатства капіталіст
перестає бути простим утіленням капіталу. Він починає відчувати
«людське почуття» до свого власного Адама; до того ж він
стає настільки освіченим, що починає глузувати з фанатичного
аскетизму, як із забобону старомодного збирача скарбів. Тимчасом
як клясичний капіталіст плямує індивідуальне споживання
як прогріх проти своєї функції й як «поздержливість» від
акумуляції, модернізований капіталіст у силі зрозуміти акумуляцію
як «відречення» від особистої насолоди. «Ах, дві душі
живуть у його грудях, одна хоче розлучитися з другою!» («Zwei
Seelen wohnen, ach! in seiner Brust, die eine will sich von der
andern trennen!»).

На історичних початках капіталістичного способу продукції —
а кожний капіталістичний вискочень індивідуально пророблює
цю історичну стадію — жага збагачення й скупість панують як
абсолютні пристрасті. Але проґрес капіталістичної продукції
створює не тільки світ насолод. Разом із спекуляцією і кредитовою
справою він відкриває тисячі джерел раптового збагачення. На
певному щаблі розвитку деякий умовний ступінь марнотратства,
що є разом з тим виставою на показ багатства, а тому й кредитоспроможности,
стає навіть діловою доконечністю для «нещасного»
капіталіста. Розкоші входять у видатки капіталу на представництво.
До того ж капіталіст багатіє не пропорційно до своєї особистої
праці і свого особистого не-споживання, як, приміром,
збирач скарбів: він багатіє в міру того, як висисає чужу робочу
силу і примушує робітника зрікатися всіх життєвих насолод.
Тому, хоч марнотратство капіталіста ніколи не має щирого
характеру марнотратства февдального пана-гуляки, навпаки,
в основі його завжди криється якнайогидливіше скнарство й
найдріб’язковіша ощадність, проте його марнотратство зростає
із зростом його акумуляції, при чому одне одному не перешкоджає.
Разом із тим у благородних грудях капіталіста розвивається
фавстівський конфлікт між жагою акумуляції і жагою насолод.
