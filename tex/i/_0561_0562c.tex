\parcont{}  %% абзац починається на попередній сторінці
\index{i}{0561}  %% посилання на сторінку оригінального видання
рис соціяльного становища країни, — каже Ґледстон, — що одночасно
зі зменшенням споживної сили народу та збільшенням
статків і злиднів робітничої кляси відбувається постійна акумуляція
багатства у вищих клясах і постійний приріст капіталу»\footnote{
Ґледстон у Палаті громад 13 лютого 1843~\abbr{р.}: «It is one of the
most melancholy features in the social state of this country that we see,
beyond the possibility of denial, that while there is at this moment, a decrease
in the consuming powers of the people, an increase of the pressure
of privations and distress; there is at the same time a constant accumulation
of wealth in the upper classes, an increase in the luxuriousness of their
habits, and of their means of enjoyment». («Times», 14 Februar 1843. —
«Hansard», 13 Februar).
}. Так говорив цей єлейний міністер у Палаті громад
13 лютого 1843~\abbr{р.} Двадцять років пізніше, 16 квітня 1863~\abbr{р.},
подаючи на розгляд бюджет, він каже: «Від 1842~\abbr{р.} до 1852~\abbr{р.}
доходи цієї країни, що підпадають оподаткуванню, зросли на
6\%\dots{} За вісім років, від 1853 до 1861~\abbr{р.}, вони збільшились, якщо
взяти за основу доходи 1853~\abbr{р.}, на 20\%. Це такий дивовижний
факт, що він майже неймовірний\dots{} Це приголомшливе збільшення
багатства й сили\dots{} обмежується геть чисто на заможних
клясах, але\dots{} але воно мусить давати посередню користь і робітничій
людності, бо воно здешевлює предмети загального споживання,
— в той час, як багаті стали багатшими, бідні в усякому
разі стали менш бідними. Я не наважуся сказати, що крайності
бідности змінилися»\footnote{
«From 1842 to 1852 the taxable income of the country increased by
6 per cent\dots{} In the 8 years from 1853 to 1861, it had increased from the
basis taken in 1853, 20 per cent! The fact is so astonishing as to be almost
incredible\dots{} this intoxicating augmentation of wealth and power\dots{} entirely
confined to classes of property\dots{} must be of indirect benefit to the
labouring population, because it cheapens the commodities of general consumption
— while the rich have been growing richer the poor have been
growing less poor! at any rate, whether the extremes of poverty are less,
I do not presume to say». Ґледстон у Палаті громад 16 квітня 1863~\abbr{р.}
«Morning Star», 17 квітня.
}. Яка слабенька прикінцева частина цього
періоду!

Якщо робітнича кляса лишилася «бідною», тільки «менш
бідною» порівняно з тим «приголомшливим збільшенням багатства
й сили», яке вона спродукувала для кляси власників, то
відносно вона лишилася так само бідною, як і раніш. Якщо
крайності бідности не зменшилися, то вони збільшилися, бо
збільшилися крайності багатства. Щождо подешевшання засобів
існування, то офіціяльна статистика, наприклад, дані лондонського
сирітського дому (Orphan Asylum), показує подорожчання
на 20\% пересічно за три роки 1860--1862 порівняно з роками
1851--1853. В наступні три роки, 1863--1865, проґресивне
подорожчання м’яса, масла, молока, цукру, соли, вугілля й
сили інших доконечних засобів існування\footnote{
Див. офіціяльні дані в Синій Книзі: «Miscellaneous Statistics
of the United Kingdom». Part VI, London 1866, \stor{}260--273 і далі.
Замість статистики сирітських домів за докази могли б служити й деклямації
міністерських журналів, що обстоюють придане для дітей королівського
дому. Там ніколи не забувають про дорожнечу засобів існування.
}. Наступна бюджетова
\index{i}{0562}  %% посилання на сторінку оригінального видання
промова Ґледстона, з 7 квітня 1864~\abbr{р.}, — це піндарівський
дитирамб на проґрес у збагаченні й на стримуване «злиднями»
щастя народу. Він каже про маси, що стоять «на краю
павперизму», про галузі продукції, «де заробітна плата не
зросла», і наприкінці резюмує щастя робітничої кляси в таких
словах: «Людське життя в дев’ятьох із десятьох випадків це
просто боротьба за
існування»\footnote{\label{footnote-105}«Think of those who
are on the border of that region (pauperism)»,
«wages\dots{} in others not increased\dots{} human life is but, in nine cases out of ten,
a struggle for existence» (Ґледстон у Палаті громад 7 квітня 1864~\abbr{р.}).
«Hansard» дає таку версію цього резюме: «Висловлюючи це в загальнішій
формі: «Що таке людське життя в більшості випадків, як не боротьба
за існування» («Again, and yet more at large, what is human life but,
in the majority of cases, a struggle for existence»). — Постійні кричущі суперечності
в бюджетових промовах Ґледстона з 1863 і 1864~\abbr{рр.} один англійський
письменник характеризує такою цитатою з Мольєра:

\settowidth{\versewidth}{Il change à tous moments d’esprit comme de mode».}
\begin{verse}[\versewidth]
«Voilà l’homme en effet. Il va du blanc au noir. \\
Il condamne au matin ses sentiments du soir. \\
Importun à tout autre, à soi même incommode, \\
Il change à tous moments d’esprit comme de mode». \\
\smallskip
(«Така людина: зараз біле, далі чорне. \\
Що ввечорі хвалила, засуджує уранці. \\
Усім набридла і самій собі як тягар, \\
І щохвилини настрій змінює як моду»). \\
\smallskip
(«Th.~Theory of Exchanges etc.», London 1864, p. 135).
\end{verse}

}. Професор Фавсет, не зв’язаний,
як Ґледстон, офіціяльними міркуваннями, прямо заявляє:
«Я, звичайно, не заперечую, що грошова плата підвищилася із
цим збільшенням капіталу [останніми десятиріччями], але ця
позірна користь у значних розмірах знову пропадає через те,
що багато потрібних засобів існування щораз дорожчає на його
думку, через падіння вартости благородних металів]\dots{} Багаті
швидко стають ще багатшими (the rich grow rapidly richer),
тимчасом як у побуті робітничих кляс не помітно ніякого поліпшення\dots{}
Робітники стають майже рабами крамарів, що в них
вони позаборговувались»\footnote{
\emph{H.~Fawcett}: «The Economie Position of the British Labourer»,
London 1865. p. 67, 82. Щождо дедалі більшої залежности робітників од
крамарів, то це є наслідок щораз частіших коливань і перерв у їхній
занятості.
}.

У розділах про робочий день і машини ми розкрили ті обставини,
в яких брітанська робітнича кляса створила «приголомшливе
збільшення багатства й сили» для маєтних кляс. Однак
нас тоді цікавив переважно робітник підчас його суспільної
функції. Щоб цілком висвітлити закони капіталістичної акумуляції,
треба також на хвилину зупинитись на становищі робітника
поза майстернею, на тому, яке його харчове й житлове становище.
Рамки книги цієї примушують нас насамперед взяти тут на увагу
найгірш оплачувану частину промислового пролетаріяту і рільничих
робітників, тобто більшість робітничої кляси.
