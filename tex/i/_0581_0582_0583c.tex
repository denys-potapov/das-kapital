
\index{i}{0581}  %% посилання на сторінку оригінального видання
Ми вже раніш відзначали становище сільських робітників
наприкінці антиякобінської війни, протягом якої так надзвичайно
позбагачувалися земельні аристократи, фармери, фабриканти,
купці, банкіри, біржові лицарі, постачальники до армії
й т.ін. Номінальна заробітна плата підвищилась почасти в наслідок
знецінення банкнот, почасти в наслідок, незалежного від
цього, зросту цін на найдоконечніші засоби існування. Але дійсний
рух заробітної плати можна сконстатувати дуже простим
способом, не вдаючись у непотрібні тут подробиці. Закон про
бідних і відповідна адміністрація були в 1814~\abbr{р.} ті самі, що і в
1795~\abbr{р.} Пригадаймо собі, як цей закон застосовувано на селі:
у формі милостині парафія доповняла номінальну заробітну
плату до номінальної суми, потрібної для простого животіння
робітника. Відношення між заробітною платою, що її платить
фармер, і тим дефіцитом її, що його поповнює парафія, показує
нам таке: поперше, наскільки заробітна плата впала нижче її
мінімуму, подруге, міру, в якій сільський робітник складався
з найманого робітника й павпера, або міру, в якій його перетворювано
на кріпака його парафії. Ми виберемо графство, яке
репрезентує пересічні умови всіх інших графств. 1795~\abbr{р.} пересічна
тижнева заробітна плата в Northamptonshire становила
7\shil{ шилінґів} 6\pens{ пенсів}, загальна сума річних видатків родини з
6 осіб — 36\pound{ фунтів стерлінґів} 12\shil{ шилінґів} 5\pens{ пенсів}, загальна сума
її доходів — 29\pound{ фунтів стерлінґів} 18\shil{ шилінґів}, дефіцит, поповнюваний
парафією, — 6\pound{ фунтів стерлінґів} 14\shil{ шилінґів} 5\pens{ пенсів.}
У тому самому графстві тижнева заробітна плата становила 1814~\abbr{р.}
12 шил. 2\pens{ пенси}, загальна сума річних видатків родини з 5 осіб —
54\pound{ фунти стерлінґів} 18\shil{ шилінґів} 4\pens{ пенси}, загальна сума її доходів
— 36\pound{ фунтів стерлінґів} 2\shil{ шилінґи}, дефіцит, поповнюваний
парафією, — 18\pound{ фунтів стерлінґів} 6\shil{ шилінґів} 4\pens{ пенси}\footnote{
\emph{Parry}: «The Question of the Necessity of the existing Cornlaws
considered», London 1816, p. 86.
}, в 1795~\abbr{р.}
дефіцит становив менш ніж четвертину заробітної плати, в
1814~\abbr{р.} — більше, ніж половину. Само собою зрозуміло, що за
таких обставин зник 1814~\abbr{р.} і той невеличкий комфорт, що його
бачив Ідн у котеджі сільського робітника\footnote{
Там же, стор. 213.
}. З усіх тварин,
що їх тримає фармер, робітник, цей instrumentum vocale\footnote*{
говорюще знаряддя. \emph{Ред.}
}, лишився
відтепер тією, яку мучать якнайбільше, годують якнайгірше і
з  якою поводяться щонайбрутальніше.

Такий стан речей спокійно тривав далі, доки «бурхливі повстання
1830~\abbr{р.} виявили нам (тобто панівним клясам) при світлі
підпалених скирт хліба, що під поверхнею рільничої Англії
злидні й глухе бунтівниче незадоволення палають так само буйно,
як і під поверхнею промислової Англії»\footnote{
\emph{S.~Laing}: «National Distress», 1844, p. 62.
}. Седлер охристив тоді
в палаті громад сільських робітників «білими рабами» («white
slaves»); з уст якогось єпископа пролунав цей самий епітет у палаті
\index{i}{0582}  %% посилання на сторінку оригінального видання
лордів. Е.~Дж.~Векфілд, найвидатніший економіст того періоду,
каже: «Сільський робітник південної Англії і не раб
і не вільна людина, — він павпер»\footnote{
«England and America», London 1833, vol. I, p. 47.
}.

Час, що безпосередньо передував скасуванню хлібних законів,
по-новому освітлив становище сільських робітників. З одного
боку, в інтересі буржуазних агітаторів було показати, як
мало ті охоронні закони захищають дійсних продуцентів хліба.
З другого боку, промислова буржуазія кипіла гнівом з приводу
того, що земельні аристократи викривали умови фабричної роботи,
з приводу того, що ці наскрізь зіпсовані, безсердечні і
знатні нероби виявляли вдаване співчуття до страждань фабричного
робітника та «дипломатичний запал» до фабричного законодавства.
Є давня англійська приказка, що коли два злодії
чублять один одного, то з цього завжди буде якась користь.
І дійсно, галаслива, пристрасна суперечка поміж двома фракціями
панівної кляси про те, яка з них якнайбезсоромніше експлуатує
робітника, допомогла і справа і зліва вияснити правду. Граф
Шефтсбері, інакше лорд Ешлі, стояв на чолі аристократичного
філантропічного походу проти фабрик. Тим то в 1844 й
1845~\abbr{рр.} він був улюбленою темою для «Morning Chronicle», що
викривав становище рільничих робітників. Ця газета, найзначніший
тодішній ліберальний орган, надіслала до селянських округ
власних комісарів, які зовсім не задовольнилися загальним
описом і статистикою, а опублікували імена так тих робітничих
родин, що їх становище вони дослідили, як і їхніх панів-землевласників.
Нижченаведена таблиця подає заробітну плату,
яку платять у трьох селах, у сусідстві Blanford’a, Wimbourne
і Poole. Села ці — власність містера Дж.~Бенкса і графа Шефтсбері.
Треба зауважити, що цей папа «low church»\footnote*{
низької церкви. \emph{Ред.}
}, цей голова англійських
пієтистів, так само як і згаданий Бенкс, із злиденної
заробітної плати робітників відбирав ще в них значну частину
під приводом плати за квартиру.

\begin{center}
\begin{small}
  \settowidth\rotheadsize{Тижневий дохід}
\begin{tabular}{ccc@{~}cc*{4}{c@{~}c}}

  \toprule

  \rotcell{Дітей} &
    \rotcell{Членів родин} &
    \multicolumn{2}{l}{
      \rotatebox[origin=c]{90}{\parbox[l]{\rotheadsize}{\raggedright Тижнева зароб. плата чоловіків}}
    } &
    \rotcell{
      Тижнева \\ заробітна \\ плата дітей
    } &
    \multicolumn{2}{l}{
      \rotatebox[origin=c]{90}{\parbox[l]{\rotheadsize}{\RaggedRight{}Тижневий дохід цілої родини }}
    } &
    \multicolumn{2}{l}{
      \rotatebox[origin=c]{90}{\parbox[l]{\rotheadsize}{\RaggedRight{}Тижнева квартирна плата }}
    } &
    \multicolumn{2}{l}{
      \rotatebox[origin=c]{90}{\parbox[l]{\rotheadsize}{\RaggedRight{}Загальний тижневий заробіток з відрахуванням квартирної плати }}
    } &
    \multicolumn{2}{l}{
      \rotatebox[origin=c]{90}{\parbox[l]{\rotheadsize}{\RaggedRight{}Тижневий заробіток на людину }}
    }
    \\
  %   \makevertcell{ \\
%
%


  \addlinespace
    \multicolumn{13}{c}{Перше село} \\

  & &
    ш. & п. &
    &
    ш. & п. &
    ш. & п. &
    ш. & п. &
    ш. & п. \\

  2 & 4 &
    8 & 0 & \emptycell{} &
    8 & 0 &
    2 & 0 &
    6 & 0 &
    1 & 6\phantom{\sfrac{1}{3}} \\

  3 & 5 &
    8 & 0 & \emptycell{} &
    8 & 0 &
    1 & 6 &
    6 & 6 &
    1 & 3\sfrac{1}{3} \\

  2 & 4 &
    8 & 0 & \emptycell{} &
    8 & 0 &
    1 & 0 &
    7 & 0 &
    1 & 9\phantom{\sfrac{1}{3}} \\

  2 & 4 &
    8 & 0 & \emptycell{} &
    8 & 0 &
    1 & 0 &
    7 & 0 &
    1 & 9\phantom{\sfrac{1}{3}} \\

  6 & 8 &
    7 & 0 & 1\textendash{}1 ш. 6 п. &
    10 & 6 &
    2 & 0 &
    8 & 6 &
    1 & 0\sfrac{1}{4} \\

  3 & 5 &
    7 & 0 & 1\textendash{}2 ш. 0 п. &
    7 & 6 &
    1 & 4 &
    5 & 8 &
    1 & 1\sfrac{1}{2} \\

  \addlinespace
    \multicolumn{13}{c}{Друге село} \\

  6 & 8 &
    7 & 0 & 1\textendash{}1 ш. 6 п. &
    10 & 0 &
    1 & 6\phantom{\sfrac{1}{2}} &
    8 & 6\phantom{\sfrac{1}{2}} &
    1 & \phantom{0}0\sfrac{3}{4} \\

  6 & 9 &
    7 & 0 & 1\textendash{}1 ш. 6 п. &
    7 & 0 &
    1 & 3\sfrac{1}{2} &
    5 & 8\sfrac{1}{2} &
    0 & \phantom{0}8\sfrac{1}{2} \\

  8 & \hang{r}{1}0 &
    7 & 0 & \emptycell{} &
    7 & 0 &
    1 & 3\sfrac{1}{2} &
    5 & 8\sfrac{1}{2} &
    0 & \phantom{0}7\phantom{\sfrac{1}{2}} \\

  4 & 6 &
    7 & 0 & \emptycell{} &
    7 & 0 &
    1 & 6\sfrac{1}{2} &
    5 & 5\sfrac{1}{2} &
    0 & 11\phantom{\sfrac{1}{2}} \\

  3 & 5 &
    7 & 0 & \emptycell{} &
    7 & 0 &
    1 & 6\sfrac{1}{2} &
    5 & 5\sfrac{1}{2} &
    1 & \phantom{0}1\phantom{\sfrac{1}{2}} \\

  \addlinespace
    \multicolumn{13}{c}{Третє село} \\

  4 & 6 &
    7 & 0 & \emptycell{} &
    7 & 0 &
    1 & \phantom{0}0 &
    6 & 0 &
    1 & 0\phantom{\sfrac{1}{2}} \\

  3 & 5 &
    7 & 0 & 1\textendash{}2 ш. \phantom{0}0 п. &
    \hang{r}{1}1 & 6 &
    0 & 10 &
    \hang{r}{1}0 & 8 &
    2 & 1\sfrac{1}{2} \\

  0 & 2 &
    5 & 0 & 1\textendash{}2 ш. 16 п. &
    5 & 0 &
    1 & \phantom{0}0 &
    4 & 0 &
    2 & 0\hang{l}{\footnotemark{}}\phantom{\sfrac{1}{2}} \\
\end{tabular}
\end{small}
\end{center}
\footnotetext{«London Economist», 29 березня 1845~\abbr{р.}, р. 290.}

\index{i}{0583}  %% посилання на сторінку оригінального видання
\noindent{}Скасування збіжжевих законів дало надзвичайний поштовх
англійському рільництву. Дренажні роботи якнайбільшого маштабу\footnote{
Земельна аристократія сама авансувала собі для цієї мети фонди
з державної скарбниці, звичайно через парлямент, за дуже низький
процент, що його фармери мали виплачувати їй удвоє.
},
нова система годівлі худоби в стайнях і засіву штучних
кормових трав, заведення механічних апаратів до угноювання,
нові способи обробляти глинкуватий ґрунт, збільшене вживання
мінерального добрива, застосування парової машини й
усякого роду нових робочих машин і~\abbr{т. д.}, взагалі інтенсивніша
культура — ось що характеризує цю епоху. Пан Пезі, президент
королівського рільничого товариства, твердить, що (відносні)
господарські витрати через заведення нових машин зменшились
майже удвоє. З другого боку, швидко збільшився позитивний
дохід від землі. Основною умовою нових метод були збільшені
капіталовкладення на акр, отже, і прискорена концентрація
фарм\footnote{
Зменшення числа середніх фармерів ясно видно з рубрик перепису:
«Син фармера, онук, брат, племінник, дочка, онучка, сестра, племінниця»,
одно слово, члени родини самого фармера, що працюють у
нього. Ці рубрики налічували 1851~\abbr{р.} \num{216.851} особу, 1861~\abbr{р.} лише \num{176.151}
особу. Від 1851 до 1871~\abbr{р.} число фарм, менших від 20 акрів, зменшилось
більш, ніж на 900, число фарм від 50 до 77 акрів спало з \num{8.253} до \num{6.370};
те саме й з усіма іншими фермами, меншими від 100 акрів. Навпаки,
протягом тих самих двадцятьох років число великих фарм збільшилось;
число фарм від 300 до 500 акрів збільшилось з \num{7.771} до \num{8.410}, число
фарм, більших за 500 акрів, — з \num{2.755} до \num{3.914}, число фарм, більших
за 1000 акрів, — з 492 до 582.
}. Одночасно й засівна площа збільшилась від 1846~\abbr{р.}
до 1856~\abbr{р.} на \num{464.119} акрів, не кажучи вже про величезні площі
у східніх графствах, які з кролячих загородок і злиденних пасовиськ
немов чарами перетворено в буйні збіжжеві лани. Ми вже
знаємо, що одночасно з цим зменшилось загальне число осіб,
занятих у рільництві. Щождо власне рільників обох статей і
\parbreak{}  %% абзац продовжується на наступній сторінці
