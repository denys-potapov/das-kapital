\parcont{}  %% абзац починається на попередній сторінці
\index{i}{0595}  %% посилання на сторінку оригінального видання
2 цалі завдовжки, 9 футів 5 цалів завширшки, вся площа має
21 фут 2 цалі довжини, 9 футів 5 цалів ширини. Спальня — це
кімната на горищі, стіни звужуються до стелі, як голова цукру,
з фасаду відкривається дахове віконце. Чому він живе тут?
Садок? Надзвичайно маленький. Квартирна плата? Висока —
1\shil{ шилінґ} 3\pens{ пенси} на тиждень. Близько до місця його праці?
Ні, хата стоїть на віддалі 6 миль від місця його праці, так що
йому доводиться день-у-день маршувати по 12 миль. Він живе
тут тому, що тут здавали в найми cot, а він хотів мати cot для
себе самого, хоч і де б то було, хоч і по якій би ціні, хоч і в якому
стані. Нижченаведена таблиця подає нам статистичні відомості
про 12 хат у Langtoft’i з 12 спальнями, що в них живуть 38 дорослих
і 36 дітей:

\begin{table}[H]
\centering
\begin{small}
\caption*{12 хат у Langloft’i}
\settowidth\rotheadsize{Число дітей}
\noindent\begin{tabular}{*{5}{r}@{\hspace{1em}}|@{\hspace{1em}}*{5}{r}}
  \toprule
  \rotcell{Хати} &
  \rotcell{Спальні} &
  \rotcell{Число дорослих} &
  \rotcell{Число дітей} &
  \makecell[r]{Загальне \\ число \\ мешканців} &
  \rotcell{Хати} &
  \rotcell{Спальні} &
  \rotcell{Число дорослих} &
  \rotcell{Число дітей} &
  \makecell[r]{Загальне \\ число \\ мешканців} \\
  \midrule
1  &  1  &  3  &  5  &  8  &  1  &  1  &  3  &  3  &  6\\
1  &  1  &  4  &  3 & 7  &  1  &  1  &  3 & 2  &  5\\
1  &  1  &  4  &  4  &  8  &  1  &  1  &  2  &  \textemdash{} & 2\\
1  &  1  &  5  &  4 & 9  &  1  &  1  &  2 & 3  &  5\\
1  &  1  &  2  &  2  &  4  &  1  &  1  &  3  &  3  &  6\\
1  &  1  &  5  &  3  &  8  &  1  &  1  &  2  &  4  &  6\\
\end{tabular}
\end{small}
\end{table}

\paragraph{Kent}

Kennington був надзвичайно переповнений у 1859~\abbr{р.}, коли
з’явилася дифтерія і парафіяльний лікар організував офіціальний
дослід становища найбіднішої кляси людности. Він виявив, що
в цій місцевості, де потребують багато праці, багато cots зруйновано,
а нових не збудовано. В одній окрузі стояли 4 будинки,
так звані birdcages (пташині клітки), в кожному з них було по
4 кімнати таких розмірів у футах і цалях:

\begin{table}[H]
  \centering
  \begin{tabular}{lr}
    Кухня\dotfill{} & 9,5 × 8,11 × 6,6 \\
    Комірка-полоскальня & 8,6 × 4,6\phantom{0} × 6,6 \\
    Спальня\dotfill{}&8,5 × 5,10 × 6,3 \\
    Спальня\dotfill{}&8,3 × 8,4\phantom{0} × 6,3 \\
  \end{tabular}
\end{table}

\paragraph{Northamptonshire}
Brinworth, Pickford i Floore: В цих селах зимою тиняються
по вулицях 20 — 30 робітників, не находячи праці. Фармери не
завжди як слід обробляли землю під збіжжя й корінняки, і лендлорд
збагнув, що йому корисніше сполучити всі свої оренди
в дві або три. Звідси недостача роботи. Тимчасом як по одному
боці рову поле потребує обробітку, по другому боці ошукані
робітники кидають на нього пожадливі погляди. Воно й не диво,
що, виснажені гарячковою надмірною працею влітку й напівголодні
\index{i}{0596}  %% посилання на сторінку оригінального видання
зимою, робітники кажуть на своєму власному діалекті,
що «the parson and gentlefolks seem frit to death at them»\footnoteA{
«Піп і шляхтич, здається, заприсяглися замордувати їх на
смерть».
}.

У Floore є приклади, що в спальні найменшого розміру живе
подружжя з 4, 5, 6 дітьми, або 3 дорослих з 5 дітьми, або подружжя
з дідом і 6 дітьми, хорими на скарлятину, і~\abbr{т. ін.}; у двох
хатах з двома спальнями — 2 родини, кожна складається з 8
і 9 дорослих.

\paragraph{Wiltshire}

Stratton: Досліджено 31 хату, 8 з них мають лише одну
спальню. Pentill у тій самій парафії. Один cot винаймають за
1\shil{ шилінґ} 3\pens{ пенси} на тиждень; в ньому живе 4 дорослих і 4 дітей;
крім добрих стін, у ньому немає нічого доброго, починаючи від
долівки з погано обтесаного каменю й кінчаючи зігнилою солом’яною
стріхою.

\paragraph{Worcestershire}

Тут хати не так жорстоко поруйновано; однак від 1851~\abbr{р.} до
1861~\abbr{р.} число мешканців на хату збільшилося з 4,2 до 4,6.

Badsey: Тут багато котеджів і садочків. Декотрі фармери
заявляють, що cots є «а great nuisance here, because they bring
the poor» («cots — велике лихо, бо принаджують бідноту»).
Один джентлмен каже: «Бідним від цього зовсім не краще; коли
збудувати 500 cots, їх розхоплять, наче булочки; справді, що
більше їх будують, то більше їх потрібно», отже, на його погляд,
хати покликають до життя мешканців, а мешканці, ясна річ,
натискують на «засоби мешкання». — З приводу цього вислову
д-р Гентер зауважує: «Алеж ці бідняки мусять звідкілясь
приходити, а що в Badsey немає особливої приваби, як от милостиня,
то, певно, мусить існувати відштовхування їх від якогось
ще невигіднішого місця, що й жене їх сюди. Коли б кожний міг
знайти недалечко від місця своєї праці cot і клаптик землі, то
напевне віддав би йому перевагу над Badsey, де йому за свій
клаптик землі доводиться платити удвоє дорожче, ніж фармерові
за свій».

Постійна еміграція до міст, постійне «створення перелюднення»
на селі через концентрацію фарм, перетворення нив на пасовиська,
застосування машин і~\abbr{т. ін.}, ідуть пліч-о-пліч з постійним
виганянням сільської людности через руйнування котеджів.
Що рідше заселена округа, то більше її «відносне перелюднення»,
то більший тиск останнього на засоби заробітку, то більший
абсолютний надмір сільської людности проти її житлових
засобів, отже, то більші по селах місцеве перелюднення й скупченість
людей з її наслідками — пошесними недугами. Скупченість
мас людей у порозкиданих дрібних селах і торгових містечках
відповідає ґвалтовному спустошенню людности на поверхні
\parbreak{}  %% абзац продовжується на наступній сторінці
