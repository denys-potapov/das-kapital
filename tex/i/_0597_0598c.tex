\parcont{}  %% абзац починається на попередній сторінці
\index{i}{0597}  %% посилання на сторінку оригінального видання
землі. Безперервне «створення надміру» сільських робітників,
не зважаючи на зменшення числа їх і одночасне зростання маси
їхнього продукту, є джерело їхнього павперизму. Їхній евентуальний
павперизм є мотив виселення їх і головне джерело їхніх житлових
злиднів, які остаточно ламають силу їхнього опору й роблять
їх справжніми рабами землевласників\footnote{
«Благородне заняття сільського поденника надає гідности навіть
його становищу. Він не раб, а солдат миру, і заслуговує на те, щоб лендлорд,
який захопив собі право примушувати його до такої самої праці,
якої країна вимагає від солдата, дав йому помешкання, відповідне для
жонатої людини. За свою працю він так само, як і солдат, не дістає ринкової
ціни. Як і солдата, його забирають молодим, несвідомим, обізнаним
лише з своєю професією і своєю місцевістю. Ранній шлюб і різні закони
про осілість впливають на нього так само, як вербування рекрутів і військовий
карний кодекс на солдата». («The heavenbom employment of
the hind gives dignity even to his position. Не is not a slave, but a soldier
of peace, and deserves his place in married man’s quarters, to be provided
by the landlord, who has claimed a power of enforced labour similar to that
the country demande of a military soldier. Не no more receives marketprice
for his work than does a soldier. Like the soldier he is caughtyoung
ignorant, knowing only his own trade and his own locality. Early marriage and
the operation of the various laws of settlement affect the one as enlistment
and the Mutiny Act affect the other»). (Dr. \emph{Hunter} y «Public Health. Seventh
Report 1864», London 1865, p. 132). Іноді якогось винятково м’якосердого
лендлорда охоплює сум перед пустелею, що її він сам створив. «Дуже
сумно бути одному в своїх маєтках», казав граф Лейчестерський, коли
його поздоровили з закінченням будови його замку Holkham. «Я оглядаюсь
навколо й не бачу жодного будинку, крім свого власного. Я — велетень
башти велетнів і пожер усіх своїх сусідів».
} і фармерів, так що мінімум
заробітної плати стає для них природним законом. З другого
боку, село, не зважаючи на своє постійне відносне перелюднення,
є разом з тим недосить залюднене. Це виявляється не лише як
місцеве явище в таких пунктах, звідки людність занадто швидко
відпливає до міст, копалень, на будову залізниць тощо; це виявляється
повсюди так підчас жнив, як і на весні й улітку підчас
тих численних моментів, коли дуже старанне й інтенсивне
англійське рільництво потребує додаткових рук. Сільських
робітників завжди занадто багато для середніх і занадто мало для
виняткових або тимчасових потреб рільництва\footnote{
Подібний рух спостерігався останніми десятиліттями у Франції,
в міру того як капіталістична продукція там опановувала рільництво й
гнала «надмір» сільської людности до міст. Так само тут спостерігається
й погіршення житлових та інших умов коло джерела «надмірних рук».
Про своєрідний «prolétariat foncier»\footnote*{
сільський пролетаріят. \emph{Ред.}
}, витворений системою парцель, див.,
між іншим, раніш цитовану працю \emph{Colins’a}: «L’Economie Politique», і
\emph{Karl Marx}: «Der Achtzehnte Brumaire des Louis Bonaparte», 2 Aufl.
Hamburg 1869 p., стор. 91 і далі. (\emph{К.~Маркс}: «Вісімнадцяте Брюмера Люї
Бонапарта», Партвидав «Пролетар» 1932, стор. 100 і далі). В 1846~\abbr{р.}
міська людність Франції становила 24,42\%, сільська — 75,58\%, в 1861~\abbr{р.}
міська — 28,86\% сільська — 71,14\%. Протягом останніх п’яти років
зменшення проценту сільської людности ще більше. Уже в 1846~\abbr{р.} П’ер
Дюпон співає в своєму «Ouvriers»:

\settowidth{\versewidth}{Sous les combles, dans les décombres,}
\begin{verse}[\versewidth]
«Mal vêtus, logés dans des trous,\\
Sous les combles, dans les décombres,\\
Nous vivons avec les hiboux\\
Et les larrons, amis des ombres».\\
\smallskip
(«Усі в дранті, тут на горищах,\\
Серед руїн, в льохах живемо ми,\\
Де сови лиш та злодії нічні\\
Ховаються, охочі до пітьми»).\\
\end{verse}
}. Тому в офіціяльних
документах маємо зареєстровані суперечні скарги з
тих самих місцевостей на одночасну недостачу і надмір робочих
рук. Тимчасова або місцева недостача робочих рук не спричиняє
підвищення заробітної плати, а спричиняє приневолення жінок
і дітей до польових робіт і вживання робітників, щораз молодших
віком. Скоро тільки ця експлуатація жінок і дітей набирав
більших розмірів, вона й собі стає новим способом перетворювати
сільських робітників-чоловіків у зайвих і тримати їхню
заробітну плату на найнижчому рівні. На сході Англії процвітає
\index{i}{0598}  %% посилання на сторінку оригінального видання
чудовий плід цього cercle vicieux\footnote*{
зачарованого кола. \emph{Ред.}
} — так звана Gangsystem
система артілей або ватаг (Gang-oder Bandensystem), про яку
я скажу тут декілька слів\footnote{
Шостий і останній «Report of Children’s Employment Commission»,
опублікований наприкінці березня 1867~\abbr{р.}, каже лише про систему
рільничих артілей.
}.

Система артілей процвітає майже виключно в Lincolnshire
Huntingdonshire, Cambridgeshire, Norfolk, Suffolk і Nott nghamshire,
спорадично — в сусідніх графствах Nothampton, Bedford
і Rutland. Як приклад візьмімо тут Lincolnshire. Значна
частина цього графства — це нова земля, колишнє болото, абож
земля, як і в інших названих східніх графствах, відвойована від
моря. Парова машина наробила чудес при осушуванні. Колишня
драговина й пісковий ґрунт красіють тепер у наряді буйного
збіжжя і дають якнайвищу ренту. Те саме стосується й до штучно
здобутого наносного ґрунту, як от на острові Axholme та інших
парафіях на побережжі Trent’y. В міру того як виникали нові
фарми, не лише не будували нових котеджів, але руйнували і
старі, а робітників постачали з відкритих сел, віддалених за
кілька миль, порозкидуваних здовж сільських доріг, що в’ються
схилами горбів. Лише там людність раніш знаходила собі захист
від тривалих зимових поводей. Робітники, що живуть на фармах
розміром від 400 до \num{1.000} акрів (їх тут звуть «confined labourers»\footnote*{
«прикріплені робітники». \emph{Ред.}
}),
служать виключно для постійних важких польових робіт, виконуваних
кіньми. На кожні 100 акрів (1 акр — 40,49 ара, або
\num{1.584} пруських морґів) пересічно припадає ледве один котедж.
Один фармер, що орендував колишню драговину, свідчить перед
слідчою комісією: «Моя фарма має більш ніж 320 акрів, все це
саме орне поле. Котеджів на ній немає. Тепер у мене живе один
робітник. Чотири робітники, що доглядають моїх коней, живуть
\parbreak{}  %% абзац продовжується на наступній сторінці
