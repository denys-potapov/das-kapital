\parcont{}  %% абзац починається на попередній сторінці
\index{i}{0600}  %% посилання на сторінку оригінального видання
розпуста і найбезсоромніше нахабство панують у ватазі. Здебільша
в шинку розплачується староста ватаги; потім він, похитуючись,
вертається додому на чолі ватаги, піддержуваний справа
й зліва кремезними бабами; позад нього скачуть діти й підлітки
співаючи глумливих і непристойних пісень. По дорозі додому звичайним
явищем є те, що Фур’є називає «фанерогамія»\footnote*{
прилюдне злягання. \emph{Ред.}
}. Часто тринадцятилітні
й чотирнадцятилітні дівчатка вагітніють від своїх
однолітків-хлопців. Відкриті села, що постачають континґент
для ватаг, стають Содомом і Гоморрою\footnote{
«Половину дівчат у Ludford’i зіпсували ватаги» (там же, Appendix,
стор. 6, п. 32).
} і дають удвоє більше
нешлюбних народжень, аніж решта королівства. Ми вже раніше
зазначали, що з морального погляду можуть дати виховані в
такій школі дівчата, ставши молодицями. Їхні діти, якщо опій
не заподіє їм смерти, є природжені рекрути ватаги.

Ватага у своїй щойно описаній клясичній формі називається
публічною, громадською, або бродячою ватагою (public, common
or tramping gang). Бо бувають ще й приватні ватаги (private
gangs). Склад їх такий самий, що й громадських, тільки в них
менше людей, і працюють вони під проводом не старости ватаги,
а якогось старого сільського наймита, що його фармер не може
застосувати якось краще. Циганський гумор тут зникає, але, як
кажуть усі свідки, плата і поводження з дітьми тут гірші.

Система ватаг, що останніми роками щораз більше поширюється,
існує очевидно не заради старости ватаги\footnote{
«Ця система дуже поширилась останніми роками. В деяких місцевостях
її заведено лише недавно, в інших, де вона існує давніше, до ватаг
вербують щораз більше і щораз молодших дітей» (там же, стор. 79,
п. 174).
}. Вона існує
для збагачення великих фармерів\footnote{
«Дрібніші фармери не вживають праці ватаг». «Її не вживають
на поганій землі, а вживають на такій, що дає ренту від 2\pound{ фунтів стерлінґів}
до 2\pound{ фунтів стерлінґів} 10\shil{ шилінґів} з акра» (там же, стор. 17 і 14).
} або лендлордів\footnote{
Одному з цих панів його ренти так припадають до смаку, що він
обурено заявив слідчій комісії, ніби ввесь галас зчинився лише через
назву системи. Коли б замість «ватаги» охристити її «юнацьким промислово-рільничим
кооперативним товариством для самостійного заробітку».
то все було б all right\footnote*{
гаразд. \emph{Ред.}
}.
}. Для фармера
немає дотепнішої методи підтримувати свій робочий персонал
нижче нормального рівня і все ж завжди мати напоготові
додаткові руки для всякої додаткової праці, за якомога менші
гроші видушувати якомога більше праці\footnote{
«Праця ватаг дешевша від усякої іншої праці; ось причина, чому
її уживають», каже один колишній староста ватаги (там же, стор. 17
і 14). «Система ватаг безперечно найдешевша для фармера й так само
безперечно найзгубніша для дітей», каже один фармер (там же, стор. 1о,
п. 3).
} і робити дорослих
робітників-чоловіків «зайвими». Після попередніх пояснень
можна зрозуміти, чому, з одного боку, визнають більше або менше
\parbreak{}  %% абзац продовжується на наступній сторінці
