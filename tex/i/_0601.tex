\parcont{}  %% абзац починається на попередній сторінці
\index{i}{0601}  %% посилання на сторінку оригінального видання
безробіття сільських робітників, а з другого — заявляють, що
система ватаг «доконечна» в наслідок браку робітників-чоловіків
та еміґрації їх до міст\footnote{
«Без сумніву, багато робіт, що їх виконують тепер у ватагах
діти, раніш виконували чоловіки й жінки. Там, де до праці вживають
жінок і дітей, тепер безробітних чоловіків більш, ніж було раніш» (more
men are out of work) (там же, стор. 43, п. 202). Але, з другого боку, між
іншим, читаємо: «Робітниче питання (labour question), у багатьох рільничих
округах, особливо тих, що продукують збіжжя, набуває такого
серйозного характеру в наслідок еміґрації і тієї легкости переселятись
до великих міст, яку дають залізниці, що я [«я» — це сільський аґент
одного великого лендлорда] вважаю дитячу працю за абсолютно доконечну»
(там же, стор. 80, n. 180). The labour question (робітниче питання)
в англійських рільничих округах, на відміну від решти цивілізованого
світу, означає власне the landlords’and farmers’question (лендлордське
й фармерське питання): яким чином, не зважаючи на щораз більший
відплив сільської людности, увічнити на селі достатнє «відносне перелюднення»,
а цим самим і «мінімум заробітної плати» для сільського
робітника?
}. Поле, очищене від бур’яну, і людський
бур’ян Лінколншіру й~\abbr{т. ін.} — це протилежні полюси
капіталістичної продукції\footnote{
Вище цитований мною «Public Health Report», де з приводу
смертности дітей сказано мимохідь і про систему ватаг, лишився невідомий
пресі, а тим то й англійській публіці. Навпаки, останній звіт «Children’s
Employment Commission» дав пресі бажану «сенсаційну» поживу.
Тимчасом як ліберальна преса запитувала, яким чином шляхетні джентлмени
й леді та священики державної церкви, що ними кишіє Лінколншір,
яким чином ці персонажі, що посилали до антиподів свої спеціяльні
«місії для поліпшення звичаїв у дикунів Південного океану», могли допустити,
щоб навіч перед ними зросла така система в їхніх маєтках, —
шляхетна преса обмежувалась міркуваннями про грубу зіпсованість
селян, здатних продавати своїх дітей у таке рабство! Однак, за тих проклятих
обставин, на які «шляхетні» засудили селянина, було б зрозуміло,
коли б він навіть з’їдав своїх власних дітей. З чого дійсно можна
дивуватись, так це з пристойности, яку він ще здебільша зберіг. Автори
офіціяльних звітів доводять, що батьки навіть в округах із системою ватаг
ставляться до цієї системи з огидою. «У зібраних нами свідченнях можна
найти багато доказів того, що батьки в багатьох випадках були б вдячні
за такий примусовий закон, що дав би їм змогу опиратися спокусам і
натискові, яким вони часто підпадають. То парафіяльний урядовець,
то підприємець, загрожуючи їм звільненням, примушує їх посилати дітей
на заробітки замість до школи\dots{} Кожне марне витрачання часу й сил,
усі страждання, що їх спричиняє селянинові і його родині надзвичайна
й некорисна втома, кожний випадок, коли батьки можуть приписати
моральний занепад своєї дитини переповненню котеджів або розкладницькому
впливові системи ватаг, — все це пробуджує у грудях цих трудящих
бідолах почуття, які, певно, легко зрозуміти і які нема потреби описувати
докладніш. Вони свідомі, що їм завдають багато фізичних і моральних
мук обставини, за які вони зовсім не відповідальні, і на які вони,
коли б на те була їхня сила, ніколи не дали б своєї згоди, і проти яких
вони неспроможні боротися» (там же, стор. XX, п. 82 і XXIII, п. 96).
}.

\manualpagebreak{}
\subsubsection{Ірляндія}

Закінчуючи цей відділ, ми мусимо ще на хвилинку спинитися
на Ірляндії. Насамперед подаємо факти, що служать нам
за вихідний пункт.
