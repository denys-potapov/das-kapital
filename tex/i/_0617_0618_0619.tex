\parcont{}  %% абзац починається на попередній сторінці
\index{i}{0617}  %% посилання на сторінку оригінального видання
людини людиною. Однак лицарі промисловости витиснули лицарів
меча лише тим, що вони скористалися з подій, які сталися
без жодної їхньої участи. Вони піднеслися за допомогою таких
самих ницих засобів, якими колись користувалися римські пущені
з рабства, щоб стати владарями своїх патронів.

Вихідним пунктом розвитку, що утворив так найманого робітника,
як і капіталіста, було рабство робітника. Проґрес полягав
у зміні форми цього рабства, у перетворенні февдальної експлуатації
на капіталістичну експлуатацію. Щоб зрозуміти перебіг
цього процесу, нам зовсім не треба заглиблюватись у дуже
далеке минуле. Хоч перші початки капіталістичної продукції
спорадично можна спостерігати ще в XIV і XV століттях по
деяких містах на побережжі Середземного моря, однак капіталістична
ера починається лише з XVI століття. Там, де вона настає,
вже давно знищено кріпацтво і з давнього часу почався
занепад краси й гордощів середньовіччя — вільних міст.

В історії первісної акумуляції епохальне значення мають всі
ті перевороти, які служили за підойму піднесення кляси капіталістів,
що формувалася; але надто важливе значення мають
ті моменти, коли величезні маси людей раптом і силоміць відривано
від засобів їхнього існування й викидувано на ринок праці
як вільних, мов птахи, пролетарів. Експропріяція землі в сільського
продуцента, селянина, становить основу цілого процесу.
[Тому ми повинні розглянути її насамперед]\footnote*{
Заведене у прямі дужки беремо з другого німецького видання. \emph{Ред.}
}. Історія її в
різних країнах набирає різного забарвлення, перебігає різні
фази в різній послідовності і в різні історичні епохи. Клясичну
форму вона має лише в Англії, яку ми тому й беремо як приклад\footnote{
В Італії, де капіталістична продукція розвинулась найраніше,
найраніше відбувся і розклад кріпацьких відносин. Кріпака тут визволено
раніш, ніж він устиг забезпечити собі якебудь право давности на
землю. Тому визволення відразу перетворило його на вільного, мов
птах, пролетаря, який до того ж находить нових панів у тих містах, що
здебільша збереглися ще від римської епохи. Коли революція на світовому
ринку, що почалася з кінця XV століття, знищила торговельну
перевагу північної Італії, то постав рух зворотного напряму. Робітників
масами витискували з міст на село, де відтоді дрібна культура, організована
за типом садівництва, набула нечуваного розквіту.
}.

\subsection{Експропріяція землі в сільської людности}

В Англії кріпацтво зникло фактично наприкінці XIV століття.
Величезна більшість людности\footnote{
«Дрібні землевласники, які обробляли власні поля власною працею
і тішилися скромними достатками\dots{} становили тоді куди більшу
частину нації, ніж тепер\dots{} Не менш, як \num{160.000} землевласників, що разом
із своїми родинами становили більше однієї сьомої частини всієї людности,
жили з того, що обробляли свої дрібні Freehold-дільниці [Freehold —
цілковита власність на землю]. Пересічний дохід цих дрібних землевласників\dots{}
оцінюється в 60--70\pound{ фунтів стерлінґів}. Обчислено, що число
тих, хто обробляв власну землю, було більше, ніж число орендарів чужої
землі». (\emph{Macaulay}: «History of England», 10th ed. London 1854,
vol. I, p. 333--334). — «Ще в останній третині XVII століття \sfrac{4}{5}
англійської людности були рільники» (там же, стор. 413). — Я цитую Маколея,
бо він, як систематичний фальсифікатор історії, по змозі применшує
подібні факти.
} складалась тоді — і ще
більше в XV столітті — з вільних селян, які господарювали
самостійно, хоч за якими февдальними вивісками ховалася їхня
власність. У великих панських маєтках вільний фармер витиснув
\index{i}{0618}  %% посилання на сторінку оригінального видання
управителя (bailiff, Vogt, який раніше сам був кріпаком. Рільничі
наймані робітники складалися почасти з селян, що використовували
свій вільний час, працюючи у великих землевласників,
почасти із самостійної, відносно й абсолютно малочисельної
кляси найманих робітників у власному значенні слова. Останні
фактично разом з тим теж були селянами, що самостійно господарювали,
бо, крім своєї заробітної плати, вони діставали котедж
і 4 або й більш акрів поля. Окрім того, вони спільно з справжніми
селянами користалися з громадської землі, на якій вони пасли
свою худобу й яка разом з тим давала їм паливо — дрова, торф
і~\abbr{т. ін.}\footnote{
Ніколи не слід забувати, що навіть кріпак був не тільки власником
— правда, власником, що мусив платити данину — земельних
парцель, що належали до його дому, але й співвласником громадської
землі. «Селянин є тут (у Шльонську) кріпак» («Le paysan у (en Silésie)
est serf»). Проте ці кріпаки (serfs) є посідачі громадської землі. «Досі
ще не вдалося схилити жителів Шльонську до поділу громадських
земель, тимчасом як у Новій Марці немає вже жодного села, де б цього
поділу не проведено з якнайбільшим успіхом» («On n’a pas pu encore
engager les Silésiens au partage des communes, tandis que dans la nouvells
Marche, il n’y a guère de village où ce partage ne soit exécuté avec le plue
grand succès»). (\emph{Mirabeau}: «De la Monarchie Prussienne», London 1788,
vol. II, p. 125, 126).
} По всіх країнах Европи февдальна продукція характеризується
поділом землі поміж якомога більшим числом васалів.
Сила февдального пана, як і всякого суверена, спиралась не на
розміри його ренти, а на число його підданців, а останнє залежало
від числа селян, що господарювали самостійно\footnote{
Японія з її суто февдальною організацією земельної власности
та з її розвиненим дрібноселянським господарством дає куди вірніший
образ європейського середньовіччя, ніж усі наші книги з історії, здебільшого
подиктовані буржуазними забобонами. Занадто воно вже вигідно
бути «ліберальним» коштом середньовіччя.
}. Тим то,
хоч англійська земля після норманського завоювання була
поділена на величезні баронства, що з них окремі часто охоплювали
900 давніших англосаксонських лордств, проте вона була
вкрита дрібними селянськими господарствами, що лише де-не-де
перемежалися великими панськими маєтками. Та і відносини,
за одночасного розквіту міст, яким відзначається XV століття,
уможливили те народнє багатство, яке так красномовно
змальовує канцлер Фортескю у своїх «Laudibus Legum Angliae»,
але вони виключали капіталістичне багатство.

Пролог до перевороту, що створив основу капіталістичного
способу продукції, відбувся в останній третині XV і в перші
десятиліття XVI століть. Масу вільних, як птиці, пролетарів
було викинуто на робітничий ринок у наслідок розпуску февдальних
\index{i}{0619}  %% посилання на сторінку оригінального видання
дружин, які, як слушно зауважує Джемс Стюарт, «всюди
марно наповняли будинки й двори». Хоч королівська влада,
сама продукт буржуазного розвитку, у своєму прагненні абсолютного
суверенітету силоміць прискорювала розпуск тих дружин,
проте вона зовсім не була його єдиною причиною. Сами
великі февдали, що стояли в якнайгострішій опозиції до королівської
влади й парляменту, створили куди численніший пролетаріят,
силоміць зганяючи селян із тієї землі, на яку селяни
мали таке саме февдальне право, як і сами великі февдали, і
узурпуючи їхні громадські землі. Безпосередній поштовх до цього
в Англії дав головним чином розквіт фляндрійської вовняної
мануфактури й відповідний зріст цін на вовну. Стару февдальну
шляхту поглинули февдальні війни, нова ж була дитиною свого
часу, для якого гроші були силою над силами. Тому гаслом її
стало перетворення орного поля на пасовиська для овець. Геррісон
у своїх «Description of England. Prefixed to Holinshed’s
Chronicles» описує, як експропріація дрібних селян руйнує
країну. «What care our great incroachers!» (Що нашим великим
узурпаторам до того!). Селянські житла й робітничі котеджі
силоміць поруйновано або засуджено на руїну. «Коли порівняти, —
каже Геррісон, — давніші інвентарі кожного лицарського маєтку,
то виявиться, що безліч хат і дрібних селянських господарств
зникло, що земля годує тепер далеко менше людей, що багато
міст занепало, хоч деякі нові міста проквітають\dots{} Я міг би багато
чого розповісти про міста й села, які поруйновано задля
овечих пасовиськ і в яких позалишалися самі тільки панські
замки». Нарікання таких старих хронік завжди трохи прибільшені,
— проте вони точно змальовують вражіння, яке революція
в продукційних відносинах справила на самих сучасників.
Порівнюючи твори канцлера Фортескю й Томаса Мора, ми виразно
бачимо ту безодню, що відділяє XV і XVI століття. Із свого
золотого віку англійська робітнича кляса, як слушно каже Торнтон,
без ніяких переходових ступенів попала в залізний.

Законодавство злякалося цього перевороту. Воно ще не
стояло на тій височині цивілізації, де «Wealth of the Nation»\footnote*{
національне багатство. \emph{Ред.}
},
тобто утворення капіталу й нещадна експлуатація та павперизація
народньої маси, вважається за ultima Thule\footnote*{
вершину. \emph{Ред.}
} всякої державної
мудрости. У своїй історії Генріха VII Бекон каже: «Того
часу (1489) збільшилися нарікання на перетворення орного поля
в пасовиська (для овець тощо), за якими легко може доглядати
кілька чабанів; а фарми, що здавалися в доживотну оренду, на декілька
років або на рік (з чого жила велика частина yeomen’ів\footnote*{
вільних рільників. \emph{Ред.}
})
перетворено на панські маєтки. Це привело до занепаду народу,
а через це й до занепаду міст, церков, десятин\dots{} У лікуванні
цього лиха король і парлямент виявили в той час мудрість, гідну
подиву\dots{} Вони вжили заходів проти цієї узурпації громадських
\parbreak{}  %% абзац продовжується на наступній сторінці
