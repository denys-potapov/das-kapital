\parcont{}  %% абзац починається на попередній сторінці
\index{i}{0637}  %% посилання на сторінку оригінального видання
аніж тоді»\footnote{
Sophisms of Free Trade. By a Barrister», London 1850~\abbr{р.}, p. 206).
Він лукаво додає: «Ми завжди готові були виступити в обороні підприємців.
Невже ж таки нічого не можна зробити для робітників?»
}. Закон установив тариф заробітної плати для міста
й села, для відштучної й поденної праці. Сільські робітники
повинні найматися на рік, міські ж «на вільному ринку». Під
загрозою ув’язнення забороняється платити заробітну плату
вищу, ніж її приписує статут, але тих, що беруть таку вищу
плату, карають важче, ніж тих, що платять її. Так, ще статут
Єлисавети про учнів, в розділах 18 і 19, визначає десятиденне
ув’язнення для того, хто заплатить вищу заробітну плату, і
тритижневе для того, хто її бере. Статут із р. 1360 посилив ці
кари й навіть уповноважував хазяїна шляхом фізичного примусу
приневолювати робітників працювати на основі законного тарифу.
Всі спілки, угоди, присяги й~\abbr{т. ін.}, що ними зобов’язалися поміж
собою мулярі й теслярі, проголошено за недійсні. Об’єднання
робітників розглядається як важкий злочин, починаючи від
XIV віку аж до 1825~\abbr{р.}, коли скасовано закони проти об’єднань.
Дух робітничого статуту з р. 1349 і всіх пізніших яскраво виявляється
в тому, що держава, щоправда, встановлювала максимум
заробітної плати, але аж ніяк не мінімум її.

В XVI столітті становище робітників, як це відомо, дуже
погіршало. Грошова плата підвищилася, але зовсім не пропорційно
до зневартнення грошей і відповідного зросту товарових цін.
Отже, на ділі заробітна плата спала. Проте закони, що мали на
меті зменшення заробітної плати, все ще мали силу; все ще відрізували
вуха й таврували тих, «кого ніхто не хотів брати на
службу». Статут про учнів 5, Єлисавети с. 3, уповноважує мирових
суддів встановлювати певну заробітну плату і змінювати
її відповідно до пори року й зміни товарових цін. Яків І поширив
це реґулювання праці також і на ткачів, прядунів та на всякі
інші категорії робітників\footnote{
З одного параграфу статуту 2, Якова І, с. 6, видно, що деякі
фабриканти сукна, які були разом з тим і мировими суддями, дозволяли
собі офіціяльно визначати тариф заробітної плати у своїх власних майстернях.
У Німеччині дуже часто видавали статути для пониження
заробітної плати, особливо після тридцятирічної війни. «Поміщикам
дуже дошкуляв брак челяді й робітників у збезлюднених місцевостях.
Всім мешканцям села заборонено було винаймати кімнати неодруженим
чоловікам і жінкам; про всіх таких осіб треба було доносити урядові
й замикати їх до в’язниці, якщо вони не хотіли бути слугами, навіть
і тоді, коли вони утримували себе з якоїсь іншої роботи, працюючи на
полі у селян за поденну плату або навіть торгуючи грішми і збіжжям.
(«Kaiserliche Privilegien und Sanctiones für Schlesien», I, 125). Цілих
сто років у наказах князів не вгавають гіркі нарікання поміщиків на
злісну й непокірливу голоту, що не хоче коритися суворому режимові,
не хоче задовольнятися приписаною законом платою; поодиноким поміщикам
заборонено було давати вищу плату, ніж установлює такса, вироблена
властями округи. А проте умови служби були після війни
часом кращі, ніж сто років пізніше; у Шльонську челядь ще в 1652~\abbr{р.}
діставала м’ясо двічі на тиждень, а в нашому столітті там по деяких
округах челядь дістає м’ясо лише тричі на рік. І заробітна плата була
після війни вища, ніж у наступних століттях». (\emph{G.~Freitag}).},
а Ґеорґ II поширив закони проти
робітничих об’єднань на всі мануфактури.

За власне мануфактурного періоду капіталістичний спосіб
продукції настільки зміцнився, що міг зробити законодатне
реґулювання заробітної плати так само можливим, як і зайвим,
а все ж, про всякий випадок, хотілося мати напоготові зброю
з старого арсеналу. Ще акт 8 Ґеорґа II забороняв давати кравцям
підмайстрам Лондону й околиць більш ніж 2\shil{ шилінґи} 7\sfrac{1}{2}\pens{ пенса}
поденної плати, за винятком випадків загального трауру; ще
\parbreak{}
\parcont{}
\index{i}{0638}  %% посилання на сторінку оригінального видання
акт 13 Ґеорґа III с. 68 уповноважував мирових суддів реґулювати
заробітну плату шовкоткачів; ще 1796~\abbr{р.} треба було двох
присудів вищих судових інстанцій, щоб вирішити, чи судові
накази мирових суддів про заробітну плату мають силу і для
нерільничих робітників; ще 1799~\abbr{р.} один парляментський акт
потвердив, що плату копальневих робітників Шотляндії має
реґулювати статут Єлисавети і два шотляндські акти 1661 і 1671~\abbr{рр.}
Але якого радикального перевороту зазнали за той час економічні
обставини, показав один нечуваний в англійській палаті
громад випадок. Тут, де більш ніж протягом 400 років фабриковано
закони про той максимум, що його ні в якому разі не
могла перевищувати заробітна плата, Вайтбред запропонував
у 1796~\abbr{р.} встановити законодатно мінімум заробітної плати для
рільничих робітників. Піт спротивився цьому, згоджуючись
однак, що «становище бідних жахливе» (cruel). Нарешті, в
1813 році закони про реґулювання заробітної плати скасовано.
Вони стали смішною аномалією від того часу, коли капіталіст
почав реґулювати працю на фабриці своїм приватним законодавством,
а плату сільського робітника почали доповнювати до доконечного
мінімуму податком на користь бідних. Але постанови
робочих статутів щодо контрактів між хазяїнами й робітниками,
щодо строків звільнення й~\abbr{т. ін.}, постанови, за якими хазяїна,
що зламав контракт, можна позивати лише до цивільного суду,
а робітника, що зламав контракт, до карного суду, — мають ще
й тепер повну силу.

Жорстокі закони проти об’єднань впали в 1825~\abbr{р.} в наслідок
грізної позиції, що її заняв пролетаріят. А проте вони впали
лише частково. Деякі прекрасні рештки старих статутів зникли
лише в 1859 році. Нарешті, парляментський акт з 29 червня
1871~\abbr{р.} через законодатне визнання тред-юньойонів претендував
усунути останні сліди цього клясового законодавства. Але
інший парляментський акт, виданий того самого дня (An act
to amend the criminal law relating to violence, threats and molestation),
фактично відновлював попередній стан у новій формі.
Цими парляментськими викрутасами всі засоби, що ними могли
користатися робітники підчас страйку або льокавту (страйк
об’єднаних фабрикантів, що одночасно закривають фабрики),
виключено з загального права і підведено під виняткові карні
закони, що їхня інтерпретація належала самим фабрикантам
у їхній ролі мирових суддів. Два роки перед тим та сама палата
\parbreak{}  %% абзац продовжується на наступній сторінці
