\parcont{}  %% абзац починається на попередній сторінці
\index{i}{0639}  %% посилання на сторінку оригінального видання
громад і той самий пан Ґледстон з усім відомою добропорядністю
подали законопроєкта про скасування всіх виняткових карних
законів проти робітничої кляси. Але далі, ніж до другого читання,
не дійшло, і таким чином справу затягувано так довго, доки
нарешті «велика ліберальна партія», об’єднавшися з торі, набралася
сміливости рішуче виступити проти того самого пролетаріяту,
що поставив її до влади. Не задовольнившись цією зрадою
«велика ліберальна партія» дозволила англійським суддям,
що завжди виляли хвостом, вислуговуючись перед панівними
клясами, відкопати знову старі закони проти «конспірації»
і застосовувати їх проти робітничих об’єднань. Як бачимо,
лише проти волі й під натиском мас англійський парлямент відмовився
від законів проти страйків і тред-юньойнів, після того,
як сам він з безсоромним егоїзмом цілих п’ятсот років був за
перманентного тред-юньйона капіталістів.

На самому початку революційної бурі французька буржуазія
зважилася відібрати в робітництва щойно завойоване право
асоціяцій. Декретом з 14 червня 1791~\abbr{р.} вона проголосила всі
робітничі об’єднання за «замах на волю й деклярацію прав людини»,
караний штрафом у 500 фунтів і позбавленням права
активного громадянства на один рік\footnote{
Перша стаття цього закону каже: «Через те, що скасування
всякого роду корпорацій громадян того самого стану й тієї самої професії
є одна з корінних основ французької конституції, забороняється відбудовувати
подібні корпорації хоч під яким приводом і в хоч якій формі»
(«L’anéantissement de toutes espèces de corporations des citoyens du même
état et profession étant l’une des bases fondamentales de la constitution
française, il est déféndu de les rétablir de fait sous quelque prétexte et
sous quelque forme que ce soit»). Стаття четверта каже, що коли «громадяни,
які належать до тієї самої професії, ремества або фаху, змовляться
між собою, укладуть угоду з тією метою, щоб спільно відмовитися
або лише за певну ціну згоджуватися допомагати своїм ремеством і працею,
— всі оті змови й угоди\dots{} будуть оголошені за протиконституційні
й за замах на волю й деклярацію прав людини й~\abbr{т. ін.}» (\dots{} des citoyen
attachés aux mêmes professions, arts et métiers prenaient des délibérations,
faisaient entre eux des conventions tendantes à refuser de concert ou
à n’accorder qu’a un prix déterminé le secours de leur industrie ou de
leurs travaux, les dites délibérations et conventions\dots{} seront déclarées
inconstitutionelles et attentatoires à la liberté et à la déclaration des
droits de l’homme etc.»), отже, за державний злочин, цілком так само,
як у старих робітничих статутах». («Révolutions de Paris». Paris 1791,
vol. VIII, p. 523).
}. Цей закон, що державнополіційними
заходами втиснув конкуренцію між капіталом і
працею у рамки, вигідні для капіталу, пережив революції і зміни
династій. Навіть терористичний уряд і той лишив його непорушним.
І лише нещодавно його викреслено з Code Pénal. Немає
нічого характеристичнішого за привід, що ним мотивують
цей буржуазний державний переворот. «Хоч і бажана річ, —
каже Шапельє, доповідач цього закону, — щоб заробітна плата підвищилась
понад теперішній її рівень, для того, щоб той, хто її
дістає, звільнився від абсолютної, майже рабської залежности,
зумовлюваної недостачею доконечних засобів існування», однак
\parbreak{}  %% абзац продовжується на наступній сторінці
