\parcont{}  %% абзац починається на попередній сторінці
\index{i}{0664}  %% посилання на сторінку оригінального видання
мале, але й неґарантоване»\footnote{
Там же, т. II, стор. 116.
}. «Хоч продукт, призначений до розподілу між робітником і капіталістом,
і великий, але робітник бере собі таку велику частину, що він швидко стає капіталістом\dots{} Навпаки,
небагато людей, навіть коли вони живуть надзвичайно довго, можуть нагромадити великі маси
багатства»\footnote{
Там же, т. І, стор. 131.
}. Робітники ні в якому разі не дозволяють капіталістові здержуватися від оплати їм за
найбільшу частину їхньої праці. Капіталістові ані крихти не допоможе, якщо він навіть
остільки хитрий, що разом із своїм власним капіталом імпортує із Европи й своїх власних найманих
робітників. «Вони незабаром перестають бути найманими робітниками, вони незабаром перетворюються на
незалежних селян, а то навіть і на конкурентів своїх колишніх хазяїнів на самому ринку найманої
праці»\footnote{
Там же, т. II, стор. 5.
}. Уявіть собі, який жах! Чесний капіталіст за свої власні гроші сам імпортував з Европи
своїх власних живих конкурентів! Та це ж світу кінець! Не диво, що Векфілд скаржиться на недостатню
залежність і недостатнє почуття залежности в найманих робітників по колоніях. «У наслідок високої
заробітної плати, — каже його учень Мірвел, — в колоніях є палке жадання дешевої і покірнішої праці,
жадання такої кляси, що їй капіталіст
міг би диктувати свої умови, а не щоб вона йому диктувала їх\dots{} У країнах із старою цивілізацією
робітник, хоч і є вільний, але за силою природного закону залежить від капіталіста; в колоніях ця
залежність мусить бути створена штучними засобами»\footnote{
\emph{Merivale}: «Lectures on Colonization and Colonies», London 1841 and 1842, vol. II, p. 235— 314 і
далі. Навіть лагідненький вульґарний економіст-фрітредер Молінарі каже: «В колоніях, де рабство
скасовано без заміни примусової праці на відповідну кількість вільної праці, ми бачили щось
протилежне тому, що бачимо щодня на власні очі. Ми
бачили, як прості робітники із свого боку експлуатують промислових підприємців, вимагають від них
заробітної плати, значно вищої від тієї законної частки їхнього продукту, що припадає їм.
Плянтатори, не маючи змоги дістати за свій цукор ціну достатню, щоб покрити підвищення заробітної
плати, мусіли покривати це збільшення спочатку із своїх зисків, а пізніш навіть із своїх капіталів.
Багато плянтаторів таким
чином зруйновано, іншим довелося закрити свої підприємства, щоб уникнути
неминучої руїни. Без сумніву, краще якщо загинуть нагромаджені капітали, ніж як загинуть цілі
покоління людей (яка великодушність з боку пана Молінарі!); але чи не краще було б, коли б не
загинули ні ті, ні ці?» («Dans les colonies où l’esclavage a été aboli sans que le travail forcé se
trouvât remplacé par une quantité équivalente de travail libre,
on a vu s’opérer la contre-partie du fait qui se réalise tous les jours sous nos yeux. On a vu les
simples travailleurs exploiter à leur tour les entrepreneurs d’industrie, exiger d’eux des salaires
hors de toute proportion avec la part légitime qui leur revenait dans le produit. Les planteurs, ne
pouvant obtenir de leurs sucres un prix suffisant pour couvrir la hausse du salaire, ont été obligés
de fournir l’excédant, d’abord sur leurs profits, ensuite sur leurs
capitaux mêmes. Une foule de planteurs ont été ruinés de la sorte, d’autres ont fermé leurs ateliers
pour échapper à une ruine imminente\dots{} Sans doute, il vaut mieux voir périr des accumulations de
capitaux, que des générations d'hommes: mais ne vaudrat-il pas mieux que ni les unes ni les autres périssent?» (\emph{Molinari}: «Etudes Economiques», Paris 1846, p. 51, 52). Пане Молінарі,
пане Молінарі! Що це буде з десятьма заповідями, з Мойсеєм та пророками, із законом попиту й
подання, коли в Европі «entrepreneur»\footnote*{
підприємець. \emph{Ред.}
} може скорочувати part légitime\footnote*{
законну пайку. \emph{Ред.}
} робітника, а в Західній
Індії робітник part légitime підприємця. І скажіть, будь ласка, що це таке, ота «part légitime», що
її, як ви сами призналися, капіталіст
в Европі щодня не доплачує? Молінарі страшенно хочеться там, у колоніях, де робітники такі «прості»,
що «експлуатують» капіталістів, поліційними заходами надати належної чинности законові попиту й
подання, що в інших випадках діє автоматично.
}.

\index{i}{0665}  %% посилання на сторінку оригінального видання
Які ж то, на думку Векфілда наслідки цього сумного стану в колоніях?\footnote*{
У другому німецькому виданні це речення зформульовано так: «Який же то результат панівної в
колоніях системи приватної власности, основаної на власній праці, а не на експлуатації чужої
праці?». \emph{Ред.}
} «Варварська система
розпорошености» продуцентів і національного майна\footnote{
\emph{Wakefield}: «England and America», London 1833, vol. II, p. 52.
}. Роздрібнення засобів
продукції поміж численних самостійно господарюючих власників нищить з централізацією капіталу всі
основи комбінованої праці. Кожне розраховане на довгий час підприємство, що поширюється на багато
років і потребує витрати основного капіталу, наражається, переводячи свої справи, на перешкоди. В
Европі капітал не гає ані хвилинки, бо робітнича кляса становить там його живу приналежність, її там
з лишком, і він завжди може нею порядкувати. Але в колоніяльних країнах! Векфілд
оповідає надзвичайно сумну анекдоту. Він мав розмову з кількома капіталістами з Канади й штату
Нью-Йорк, де хвилі еміґрації часто спиняються, лишаючи по собі осад «зайвих» робітників. «Наш
капітал, — зідхає один з персонажів мелодрами, — наш капітал був напоготові для багатьох операцій,
що для свого виконання потребують чималого часу; але хіба ми могли починати такі операції з
робітниками, які — ми знали це — незабаром повернули б нам спину? Коли б ми були певні, що зможемо
вдержати в себе працю цих еміґрантів, ми охоче були б їх негайно найняли, та ще й за високу ціну. Ще
більше: навіть упевнені, що втратимо їх, ми все ж були б їх найняли, коли б були певні, що матимемо
нове подання праці, відповідно до наших потреб»\footnote{Там же, стор. 191, 192.}.

Після того, як Векфілд так пишно змалював контраст між англійським капіталістичним рільництвом з
його «комбінованою» працею і розпорошеним американським селянським господарством, він мимохіть
пробовкнувся й про зворотний бік
медалі. Він змальовує американську народню масу як заможну, незалежну, підприємливу й порівняно
освічену, тимчасом як «англійський рільничий робітник є жалюгідний голодранець (a miserable wretch),
павпер\dots{} У якій іншій країні, крім Північної Америки й деяких нових колоній, заробітна плата за
вільну працю, вживану в рільництві, хоч у якійбудь вартій згадки
\parbreak{}  %% абзац продовжується на наступній сторінці
