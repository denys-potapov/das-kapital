\parcont{}  %% абзац починається на попередній сторінці
\index{ii}{0006}  %% посилання на сторінку оригінального видання
Берліні 1859~\abbr{р.} під тією самою назвою. В ньому викладено на сторінках
1--220 (зшитки I--V), а потім знову на сторінках від 1159 до 1472
(зшитки XIX--XXIII) теми, досліджені в І книзі „Капіталу“, починаючи з
перетворення грошей на капітал і до кінця; це є перша наявна редакція
цих тем. На сторінках від 973 до 1158 (зшитки XVI--XVIII) мовиться
про капітал і зиск, про норму зиску, купецький капітал і грошовий капітал,
— отже, про теми, пізніше розвинуті в рукопису до III книги. Навпаки,
теми, викладені в II книзі, а також дуже багато тем, розглянутих
пізніше в III книзі, тут окремо ще не розроблені. Їх трактується мимохідь,
саме в відділі, що становить головну частину рукопису, сторінки
220--972 (зшитки VI--XV): \emph{Теорії додаткової вартости}. В цьому відділі
подано докладну критичну історію центрального пункту політичної
економії, а саме теорії додаткової вартости, і разом з тим розвинуто в
формі полеміки з попередниками більшість пунктів, досліджених пізніше
окремо та в логічному зв’язку в рукопису, що стосується до II та III
книг. Я маю на думці опублікувати критичну частину цього рукопису,
викинувши з нього багато місць, докладно розглянутих у книгах II і
III, — як IV книгу „Капіталу“\footnote*{
Цей рукопис після смерти \emph{Енгельса} виготовив до друку й видав під назвою
„Теорії додаткової вартости“ \emph{К.~Кавтський}. \emph{Ред.}
}. Хоч який цінний цей рукопис, однак, у
ньому мало з чого можна було скористатися для цього видання II книги.

Дальший датою рукопис є рукопис III книги. Його написано, принаймні
більшу частину, в 1854 і 1865 році. Лише після того, як він
був готовий в основному, Маркс почав обробляти першу книгу надрукованого
1867 року першого тому. Цей рукопис III книги я обробляю
тепер до друку.

З найближчого періоду — по виданні книги І — маємо для II книги зібрання
чотирьох рукописів in folio, що їх перенумерував сам Маркс з
І до IV.~З них рукопис І (150 сторінок), написаний, мабуть, 1865 або
1867~\abbr{р.}, є перше самостійне, але більш-менш уривчасте оброблення
книги II в її теперішній побудові. І в цьому рукопису також не можна
було нічого використати. Рукопис III складається почасти з зібрання
цитат і посилань на Марксові зшитки з виписами — все це стосується переважно
до першого відділу II книги, — а почасти він є оброблення поодиноких
пунктів, а саме критики Смісових засад про основний та обіговий
капітал та про джерело зиску; далі висвітлено відношення норми
додаткової вартости до норми зиску, що стосується до III книги. Посилання
дали мало нового, бо в наслідок пізніших редакцій з них годі
було користатись для II і III книг; отже, їх теж здебільша довелось відкласти.
— Рукопис IV є готове до друку оброблення першого відділу та
першого розділу другого відділу книги II, і його тут у відповідних місцях
використано. Хоч виявилось, що цей рукопис написано раніше, ніж
рукопис II, однак, з нього можна було добре скористатись для відповідної
частини книги, бо він більш закінчений формою; досить було
зробити деякі додатки з рукопису II. — Цей останній рукопис є одним-одне
\index{ii}{0007}  %% посилання на сторінку оригінального видання
більш-менш викінчене оброблення книги II, і датовано його
1870 роком. У примітках для остаточної редакції, що про них мова
буде зараз, зазначено виразно: „В основу треба покласти друге оброблення“.

Після 1870 року знову постала перерва, зумовлена, головним чином,
хоробливим станом Маркса. Своїм звичаєм Маркс присвятив цей час
студіям. Агрономія, американські та особливо російські земельні відносини,
грошовий ринок і банківська справа, нарешті, природничі науки:
геологія і фізіологія, а особливо самостійні математичні роботи становлять
зміст численних Марксових записних зшитків цього часу. На початку
1877 року він почував себе так добре, що знову міг узятись
до своєї справжньої роботи. Кінцем березня 1877~\abbr{р.} датовано посилання
й замітки з чотирьох вищезгаданих рукописів, які являли основу
того нового перероблення II книги, що початок його є в рукопису V
(56 сторінок in folio). Він охоплює перші чотири розділи й ще мало
оброблений; посутні пункти розроблено в примітках під текстом; матеріял
скорше зібрано, ніж просіяно, але це останній викінчений виклад цієї найважливішої
частини першого відділу. — Першу спробу зробити з цього
рукопис, готовий до друку, являє рукопис VI (написаний \emph{після} жовтня
1877 року й до липня 1878 року); в ньому лише 17 сторінок чвертьаркушевих,
які охоплюють більшу частину першого розділу; а другу — і
останню — спробу являє рукопис VII, датований „2 липня 1878~\abbr{р.}“, що
має лише 7 сторінок in folio.

Того часу Маркс, здається, зрозумів, що, як не буде ґрунтовного перевороту
в стані його здоров’я, він ніколи не матиме змоги закінчити оброблення
другої й третьої книги так, щоб це задовольняло його самого. І справді
на рукописах V--VIII відбилась у багатьох місцях надмірно напружена
боротьба з пригнітною недугою. Найскладнішу частину першого відділу
було перероблено наново в рукопису V, решта першого відділу і цілий
другий відділ (за винятком розділу сімнадцятого) не являли жодних
значних теоретичних труднощів; навпаки, третій відділ, — репродукція
та циркуляція суспільного капіталу, — на його думку, конче треба
було переробити. А саме, в рукопису II розглядалось репродукцію
спочатку без зв’язку з грошовою циркуляцією, що її упосереднює, а потім
ще раз у зв’язку з нею. Це треба було усунути і взагалі так переробити
цілий відділ, щоб він відповідав поширеному кругозорові автора.
Так постав рукопис VIII, зшиток, що має лише 70 сторінок чвертьаркушевих;
але скільки Маркс зумів утиснути в ці сторінки, це доводить
порівняння відділу III у друкованому вигляді, по вилученні з нього
місць, узятих з рукопису II.

І цей рукопис подає лише попереднє трактування предмету, при чому
насамперед малось на увазі визначити й розвинути новоздобуті, порівняно
з манускриптом II, погляди, залишаючи осторонь пункти, що про них
не можна було сказати нічого нового. Чималу частину розділу XVII
другого відділу, якій, окрім того, деяким чином стосується третього відділу,
внову перероблено й поширено. Логічна послідовність часто уривається,
\index{ii}{0008}  %% посилання на сторінку оригінального видання
виклад подекуди має прогалини, і особливо наприкінці він цілком
уривчастий. Але те, що Маркс хотів сказати, так або інакше тут сказано.

Такий матеріял для II книги, що з нього я, як сказав Маркс не задовгий
час до своєї смерти своїй дочці Елеонорі, повинен був „дещо зробити“.
Це доручення я зрозумів у найвужчому його значенні; де лише
можна було, я обмежив мою роботу простим вибором між різними редакціями,
і саме так, що в основу завжди покладав останню з даних редакцій,
порівнявши її з попередніми. Справжні, тобто не лише технічні
труднощі являв при цьому тільки перший і третій відділи, але ці труднощі
були не абиякі. Я дбав про те, щоб розв’язати їх виключно в
авторовому дусі.

Цитати в тексті я здебільша перекладав там, де їх наведено на потвердження
фактів, або там, де ориґінал є до послуг кожного, хто хоче
докладно обізнатися з питанням, прим., у цитатах з А.~Сміса. Тільки в
розділі X не можна було зробити цього, бо тут безпосередньо критикується
англійський текст. У цитатах з І тому посилання зроблено на
сторінки другого видання його, останнього, яке вийшло за життя
Маркса.

Для III книги, крім першої обробки в рукопису „Zur Kritik“, згаданих
частин рукопису III і деяких коротеньких приміток, зроблених подекуди
в записних зшитках, маємо лише ось що: зазначений вище рукопис
in folio від 1864--1865~\abbr{р.}, розроблений майже так само повно, як
і рукопис II книги II, і, нарешті, зшиток з 1875~\abbr{р.}: відношення норми
додаткової вартости до норми зиску, викладене математично (в рівнаннях).
Підготовка цієї книги до друку йде швидким темпом. За думкою,
що в мене склалась до цього часу, вона являтиме, головним чином, технічні
труднощі, за винятком, звичайно, деяких дуже важливих відділів.

\pfbreak{}

\vspace*{\fill}
Тут буде до речі розбити те обвинувачення проти Маркса, що його
поширювали спочатку потихеньку й поодинці, а тепер, після смерти його
проголосили за безперечний факт німецькі катедерсоціялісти й державні
соціялісти та їхні прихильники, — обвинувачення, ніби Маркс учинив
пляґіят у Родбертуса. В іншому місці\footnote{
У передмові до „Das Elend der Philosophie. Antwort auf Proudhons Philosophie
des Elends von Karl Marx. Deutsch von E.~Bernstein und K.~Kautsky.
Stuttgart 1885“ (\emph{К.~Маркс}. „Злидні філософії“. Відповідь на „Філософію злиднів“
\emph{Прудона}).
} я вже сказав усе найпосутніше
з цього приводу, але лише тут можу навести рішучі докази.

Обвинувачення це, оскільки я знаю, вперше трапляється в „\textgerman{Emanzipationskampf
des vierten Standes}“ P.~Maєpa, стор. 43: „З цих оголошених
друком праць“ (праць Родбертуса, датованих до останньої половини
тридцятих років) „Маркс, \emph{як це можна довести}, почерпнув більшу частину
своєї критики“. Поки не було дальших доказів, я, звичайно, міг
припускати, що вся „довідність“ цього твердження сходить на те, що
Родбертус упевнив п. Маєра в цьому. — 1879 року виступае на кін сам Родбертус
\index{ii}{0009}  %% посилання на сторінку оригінального видання
і в зв’язку з своєю працею „\textgerman{Zur Erkenntniss unserer staatswirtschaftlichen
Zustände}“ (1842) пише Й.~Целлерові („\textgerman{Tübinger Zeitschrift für
die Gesamte Staatswissenschaft}“, 1879, S. 219) ось що: „Ви побачите, що
в цього“ (з розвинутих тут думок) „непогано скористався\dots{} Маркс, звичайно,
не посилаючись на мене“. Це повторює, не довго думаючи, за ним
і його посмертний видавець, Т.~Козак („Das Kapital von Rodbertus“, Berlin
1884, Einleitung, S.~XV). — Нарешті, у виданих 1881 року Р.~Маєром
„Briefe und sozialpolitischen Aufsätze von Dr.~Rodbertus-Jagetzow“ Родбертус
прямо каже: „Тепер я бачу, як мене \emph{обікрали} Шефле й Маркс, не
посилаючись на мене“ (Brief № 60, S. 134). А в другому місці претенсія
Родбертуса набирає виразнішої форми: „Відхи \emph{виникає додаткова
вартість} капіталіста, це я показав у моєму третьому „Соціяльному листі“
\emph{посутньо так само}, як Маркс, тільки коротше та виразніше“ (Brief
№ 48, S. 111).

Про всі ці обвинувачення в плягіяті Маркс ніколи нічого не знав.
В його примірнику „Emanzipationskampf“ — розрізано тільки частину,
що стосується до Інтернаціоналу, а решту книги розрізав уже я після
його смерти. Тюбінґенської „Zeitschrift“ він ніколи не бачив. „Briefe“ etc.
до Р.~Маєра також були йому невідомі, і мою увагу на місце, що стосується
„обкрадання“, лише 1887 року, ласкаво звернув сам п. д-р Маєр.
Навпаки, лист №48 був Марксові відомий; п. Маєр сам з своєї ласки
подарував ориґінал молодшій дочці Марксовій. Маркс, що до нього звичайно
дійшло потайне шушукання про таємні джерела його критики, які
треба шукати у Родбертуса, показав мені цього листа й додав, що тут
він має, нарешті, автентичне свідчення про те, на що, власне, претендує
сам Родбертус; коли він не каже нічого більш, то для нього, тобто для
Маркса, справа йде на добре; а що Родбертус уважає свій виклад за
коротший та виразніший, то він може дати йому й це задоволення.
Маркс справді гадав, що цим листом Родбертуса вичерпано всю справу.

І так можна було гадати то більше, що, як я добре знаю, вся літературна
діяльність Родбертуса лишалась невідома Марксові до 1859 року,
коли його власна критика політичної економії не лише в основному, але
й у найважливіших подробицях була готова. Свої економічні студії він
почав 1843 року в Парижі, вивчаючи великих англійців і французів; з
німців він знав лише Рав і Ліста, і цього йому було досить. Ні Маркс,
ні я не знали нічогісінько про існування Родбертуса, поки 1848 року
не довелось нам критикувати в „Neue Rheinische Zeitung“ його промову,
як берлінського депутата, і його вчинки, як міністра. Ми були так необізнані,
що запитували райнських депутатів, хто це такий Родбертус,
що так швидко зробився міністром. Але й вони не могли нам нічого
сказати про економічні праці Родбертуса. Навпаки, що Маркс і без допомоги
Родбертуса вже тоді дуже добре знав, не лише звідки, але також
і \emph{як} „виникає додаткова вартість капіталіста“, — це доводять його
„Misère de la Philosophie“ 1847 року і лекції про найману працю та капітал,
прочитані 1847~\abbr{р.} в Брюсселі й опубліковані 1849~\abbr{р.} в „Neue
Rheinische Zeitung“, під № 264--69. Тільки щось 1859~\abbr{р.} Маркс довідався
\index{ii}{*0010}  %% посилання на сторінку оригінального видання
від Ляссаля, що є також економіст Родбертус, і потім знайшов
його „Третій соціяльний лист“ у Британському музеї.

Такі фактичні обставини. А як стоїть справа з тим змістом, що його
ніби „украв“ Маркс у Родбертуса? „Відки виникає додаткова вартість
капіталіста, — каже Родбертус, — це я показав у моєму третьому соціяльному
листі так само, як і Маркс, тільки коротше та виразніше“. Отже,
ось де центральний пункт: теорія додаткової вартости; і справді, не
можна сказати, на що інше міг би Родбертус претендувати з Маркса,
як на свою власність. Таким чином, Родбертус виголошує тут себе за
справжнього автора теорії додаткової вартости, що її викрав у нього Маркс.

Що ж каже нам третій соціяльний лист про постання додаткової
гартости? Він каже просто, що „рента“, — а Родбертус має на увазі тут
разом і земельну ренту і зиск, — постає не з „додатку вартости“ до
вартости товарів, але „в наслідок віднімання вартости, що його зазнає заробітна
плата, або, інакше кажучи, в наслідок того, що заробітна плата
являє лише частину вартости продукту“, а за достатньої продуктивности
праці „немає потреби, щоб вона дорівнювала природній міновій вартості
її продукту для того, щоб від неї лишалася ще частина на покриття
капіталу (!) й на ренту“\footnote*{
Цитовані місця є в Родбєртуса, а саме в „Soziale Briefe an von Kirchmann,
Dritter Brief“, Berlin, 1851, S. 87. (Примітка Кавтського до Volksausgabe. 1926~\abbr{р.}).
\emph{Ред.}
}. При цьому нам не кажуть, що являє собою
ця „природна мінова вартість“ продукту, що при ній нічого не залишається,
на „покриття капіталу“, а, значить, і на покриття сировинного
матеріялу та зношування знарядь праці.

На щастя, нам припадає констатувати, яке враження справило це епохальне
відкриття Родбертусове на Маркса. В рукопису „Zur Kritik“ etc., в зшитку X,
на стор. 445 і далі, читаємо ми: „Відхилення. Пан Родбертус. Нова теорія
земельної ренти“. Тільки з цього погляду розглядається тут третій соціяльний
лист. З Родбертусовою теорією додаткової вартости справу взагалі
закінчено таким іронічним зауваженням: „Пан Родбертус спочатку досліджує
стан речей у країні, де не відокремлено посідання землею від посідання
капіталом, і доходить потім \emph{важливого} висновку, що рента [а її
він розуміє, як цілу додаткову вартість] просто дорівнює неоплаченій
праці або кількості продуктів, що в ній цю працю втілено.“

Капіталістичне людство вже протягом багатьох століть продукувало
додаткову вартість і помалу дійшло того, що почало замислюватись над
її постанням. Перший погляд виник із безпосередньої купецької
практики: додаткова вартість постає з додатку до вартости продукту.
Цей погляд панував серед меркантилістів, але вже Джемс Стюарт побачив,
що при цьому те, що один виграє, другий неминуче втрачає. Не зважаючи
на це, цей погляд тримався ще й далі довгий час, особливо серед
соціялістів; але з клясичної науки його витиснув А.~Сміс.

У „Багатстві народів“, кн. І, розділ VI, він каже: „Скоро капітал
(stock) нагромадився в руках поодиноких осіб, деякі з них звичайно застосують
\index{ii}{*0011}  %% посилання на сторінку оригінального видання
його так, що посадять за роботу старанних людей, дадуть їм
сировинний матеріял і засоби існування для того, щоб здобути \emph{зиск},
продаючи продукти їхньої праці, або того, що \emph{їхня праця додала до
вартости того сировинного матеріялу\dots{} Вартість}, що її робітники
\emph{додають до сировинного матеріялу}, поділяється тут на \emph{дві частини} —
з них однією оплачується \emph{їхню заробітну плату}, а друга становить
\emph{зиск підприємця} на всю суму, що її він авансував на сировинний
матеріял та заробітну плату.“ І трохи далі: „Скоро вся земля в якійбудь
країні стане приватною власністю, землепосідачі, як і інші люди,
воліють за краще жати там, де вони не сіяли, і вимагають земельної
ренти навіть за природні продукти землі\dots{} Робітник\dots{} мусить \emph{відступити}
землепосідачеві \emph{деяку частину} з того, що він зібрав або виробив
своєю \emph{працею}. Ця частина, або — що те саме — ціна цієї частини становить
\emph{земельну ренту}“.

З приводу цього місця Маркс зауважує в зазначеному рукопису
„Zur Kritik etc.“, на стор. 253: „Отже, А.~Сміс розуміє додаткову вартість,
— а саме додаткову працю, надлишок виконаної та зречевленої в
товарі праці \emph{проти} оплаченої праці, тобто проти праці, що одержала
свій еквівалент у заробітній платні, — як \emph{загальну категорію}, при чому
власно зиск і земельна рента являють лише її відгалуження“.

Далі А.~Сміс каже в книзі І, розд. VIII: „Скоро земля стала приватною
власністю, землепосідач вимагає частину майже всіх продуктів, що
їх робітник може виробити або зібрати на ній. Його земельна рента
становить \emph{перше відрахування з продукту праці, прикладеної до землі}.
Але хлібороб рідко коли має засоби утримувати себе до збору врожаю.
Його утримання звичайно авансується йому з капіталу (stock) підприємця,
орендаря, що не мав би жодного інтересу давати йому роботу, коли б
хлібороб \emph{не ділився з ним продуктом своєї праці} або не повертав йому
капіталу разом з зиском. Цей зиск є \emph{друге відрахування} з праці, прикладеної
до землі. Продукт майже всякої праці зазнає такого самого
відрахування для зиску. В усіх галузях промисловости для більшости робітників
потрібен підприємець, який авансував би їм до закінчення праці
сировинний матеріял і заробітну плату та утримання. Цей підприємець
\emph{поділяє} з ними \emph{продукт їхньої праці}, або вартість, що її вони додають
до перероблюваного сировинного матеріялу, і ця частка становить його
зиск“.

Маркс додає до цього (рукопис, ст. 256): „Отже, А.~Сміс по-простому
визначає тут земельну ренту і зиск на капітал як прості відрахування з
продукту робітника або з вартости його продукту, що дорівнює праці,
долученій ним до сировинного матеріялу. Але це відрахування, як довів
раніше сам А.~Сміс, може складатися лише з частини праці, яку робітник
додав до матеріялу понад ту кількість праці, яка оплачує тільки
його заробітну плату або дає еквівалент його заробітної плати, отже, з
додаткової праці, з неоплаченої частини його праці“.

„Відки виникає додаткова вартість капіталіста“ і, крім того, землевласника,
це знав, як бачимо, ще А.~Сміс; Маркс відверто визнає це ще
\parbreak{}  %% абзац продовжується на наступній сторінці
