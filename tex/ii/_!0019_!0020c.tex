\parcont{}  %% абзац починається на попередній сторінці
\index{ii}{*0019}  %% посилання на сторінку оригінального видання
мають тут нагоду довести, що може дати Родбертусова економія. Коли
вони доведуть, як може та мусить утворюватись однакова пересічна
норма зиску не лише без порушення закону вартости, але, навпаки, на
його основі, тоді ми будемо розмовляти з ними далі. А тимчасом чи не
будуть вони ласкаві поспішити? Блискучі досліди цієї II книги й її цілком
нові результати в досі майже недосліджених галузях являють лише
попередні тези до змісту III книги, яка розвиває остаточні висновки Марксового
викладу суспільного процесу репродукції на капіталістичній
основі. Коли вийде ця III книга, менше буде розмов про якогось економіста
Родбертуса.

Друга й третя книги „Капіталу“, як часто казав мені Маркс, повинні
бути присвячені його дружині.
Ф.~Енґельс.

Лондон, день Марксового народження, 5 травня 1885~\abbr{р.}

Друге видання, що його тут подано, є в головному дослівний передрук
першого. Виправлено друкарські помилки, усунуто деякі стилістичні
хиби, викинуто кілька коротких абзаців, де були тільки повторення.

Третя книга, що являла цілком несподівані труднощі, тепер майже
готова в рукопису. В разі, що буду здоровий, її можна буде почати
друкувати ще цієї осени. Ф.~Енґельс.

Лондон, 15 липня 1893~\abbr{р.}

Щоб полегшити орієнтування, подаємо коротеньке зіставлення місць,
узятих з окремих рукописів II--VIII.

Відділ перший

Стор. З з рукопису II. — Стор. 4 з рукопису VII. — Стор. 13--16
з рукопису VI. — Стор. 16--76 з рукопису V.~Стор. — 76--79 замітка, знайдена
між витягами з книжок. — Ст. 79 до кінця — рукопис IV; однак
вставлено: стор. 85--86 місце з рукопису VIII; стор. 89 і 94 замітки з
рукопису II.

Відділ другий

Початок, ст. 104--112 є кінець рукопису IV.~Відси до кінця відділу,
ст. 267, все з рукопису II.

Відділ третій

Розділ 18 (стор. 267--274) з рукопису II.

Розділ 19: I і II (стор. 274--299) з рукопису VIII. — III (стор. 299--300)
з рукопису II.
\index{ii}{*0020}  %% посилання на сторінку оригінального видання
Розділ 20: І (стор. 300--302) з рукопису II, лише прикінцевий
абзгц з рукопису VIII.

II (стор. 300--305) в головному з рукопису II.

Ill, IV, V (стор. 305--325) з рукопису VIII.

VI, VII, VIII, IX (стор. 325--337) з рукопису II.

X, XI, XII (стор. 338--373) з рукопису VIII.

XIII (стор. 373--380) з рукопису II.

Розділ 21: (стор. 380--409) увесь з рукопису VIII.
\parbreak{}  %% абзац продовжується на наступній сторінці
