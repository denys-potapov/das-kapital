\index{ii}{0004}  %% посилання на сторінку оригінального видання
\subsection{Перша стадія. $Г — Т$\footnote{
Відси рукопис VII, розпочатий 2 липня 1878.
}}

$Г — Т$ являє перетворення певної грошової суми на певну суму товарів; для покупця — перетворення його
грошей на товар, для продавців — перетворення їхніх товарів на гроші. Що перетворює цей акт
загальної товарової циркуляції разом з тим на функціонально визначений відділ в самостійному
кругобігу індивідуального капіталу, це — насамперед не форма цього акту, а його речовий зміст,
специфічний характер
споживання тих товарів, що обмінюються місцем з грішми. Це, з одного боку, засоби продукції, а з
другого — робоча сила, речові та особові чинники товарової продукції, що їхній особливий рід,
звичайно, мусить відповідати тому ґатункові предметів, що мають виробляти. Коли ми позначимо робочу
силу Р, засоби продукції Зп, то сума товарів, що їх мають купити є: $Т \deq{} Р \dplus{} Зп$, або коротше Т \splitfrac{Р}{Зп}. Отже, розглядуваний щодо свого змісту акт $Г — Т$ являє собою $Г — Т\splitfrac{Р}{Зп}$; тобто $Г — Т$
розпадається на $Г — Р$ і $Г — Зп$; грошова сума Г розпадається на дві частини, що з них одна купує
робочу силу, а друга засоби продукції. Ці два ряди купівель належать до цілком різних ринків: один
до власне товарового ринку, а другий — до ринку праці.

Але, крім такого якісного розщеплення тієї суми товарів, що на неї перетворюється Г, акт $Г — Т  \splitfrac{Р}{Зп}$ являє собою ще надзвичайно характеристичне кількісне відношення.

Ми знаємо, що вартість, зглядно ціна робочої сили, власникові її, що продає її як товар, сплачується
в формі заробітної плати, тобто як ціну певної кількости праці, що має в собі додаткову працю; так
що, коли, напр., денна вартість робочої сили дорівнює 3 маркам, продуктові п’ятигодинної праці, то в
угоді між покупцем і продавцем ця сума фігурує як ціна або плата, напр., за десятигодинну працю.
Коли таку угоду складено, напр., з 50 робітниками, то вони повинні загалом дати покупцеві протягом
одного дня 500 робочих годин, що з них половина, 250 робочих годин, тобто 25 десятигодинних робочих
днів, є чиста додаткова праця. Кількість, а також і об’єм тих засобів продукції,
що їх треба купити, мусять бути достатні, щоб можна було вжити цю кількість праці.

Отже, $Г — Т \splitfrac{Р}{Зп}$ виражає не лише якісне відношення, не лише те, що певна сума грошей, напр., 422
фунти стерл., перетворюється на відповідні одне одному засоби продукції та робочу силу, але також
виражає
\index{ii}{0005}  %% посилання на сторінку оригінального видання
і кількісне відношення між частиною грошей, витраченою на робочу
силу Р, і частиною, витраченою на засоби продукції Зп, — відношення,
що визначається насамперед сумою надлишкової додаткової праці,
що її має витратити певне число робітників.

Коли, отже, напр., в якійсь пральні тижнева заробітна плата 50 робітників
становить 50 фунтів стерл., то на засоби продукції треба витратити
372 фунт, стерл., припускаючи, що це є вартість засобів продукції,
перетворюваних на пряжу тижневою працею в \num{3.000} годин, що з них
\num{1.500} годин є додаткова праця.

Тут цілком байдуже, до якої міри в різних галузях промисловости
додаткове вживання праці зумовлює додаткову витрату вартости в
формі засобів продукції. Тут має значіння лише те, щоб, за всіх умов,
витрачуваної на засоби продукції частини грошей — засобів продукції,
куплених в акті $Г — Зп$ — було достатньо, отже, заздалегідь обчислено й
подано у відповідній пропорції. Інакше кажучи, маса засобів продукції
мусить бути достатня, щоб увібрати масу праці, щоб через неї перетворитись
на продукт. Коли б у наявності не було достатньої кількости
засобів продукції, то надлишкова праця, що нею порядкує покупець,
не могла б застосуватись; право покупця порядкувати цією
працею не призвело б ні до чого. Коли б у наявності засобів продукції
було більше, ніж праці, що нею можна порядкувати, то ці засоби продукції
не наситилось би працею, не перетворилось би на продукт.

Скоро відбувся акт $Г — Т \splitfrac{Р}{Зп}$,покупець порядкує не лише засобами продукції та робочою силою, що
потрібні для продукції певного корисного
предмету. Він порядкує більшою кількістю робочої сили, що її
можна реалізувати як працю, або більшою кількістю праці, ніж треба
на покриття вартости робочої сили, і разом з тим порядкує засобами
продукції, які потрібні для реалізації або зречевлення цієї суми
праці: отже, порядкує він чинниками для продукції предметів більшої
вартости, ніж вартість елементів їх продукції, або чинниками
продукції такої кількости товарів, що в ній буде й додаткова вартість.
Отже, вартість, що її він авансував у грошовій формі, перебуває тепер
в такій натуральній формі, що в ній вона може реалізуватись як вартість,
яка вилуплює додаткову вартість (у вигляді товарів). Інакше кажучи:
вона перебуває в стані, або в формі \emph{продуктивного капіталу}, який має
здібність функціонувати так, що утворює вартість і додаткову вартість.
Капітал у цій формі ми позначимо П.

Але вартість П дорівнює вартості $Р \dplus{} Зп$, тобто дорівнює Г,
перетвореному на Р і Зп. Г є така сама капітальна вартість,
як і П, тільки форма її буття інша, а саме це є капітальна
вартість у грошовому стані або в грошовій формі — \emph{грошовий капітал}.
Тому $Г — Т \splitfrac{Р}{Зп}$, або в своїй загальній формі $Г — Т$, сума всіх товарокупівель,
\index{ii}{0006}  %% посилання на сторінку оригінального видання
— цей процес загальної товарової циркуляції становить разом з тим,
як стадія в процесі самостійного кругобігу капіталу, перетворення капітальної
вартости з її грошової форми на її продуктивну форму, або
коротше — перетворення \emph{грошового капіталу} на \emph{продуктивний капітал}.
Отже, у тій схемі кругобігу, що її ми насамперед тут розглядаємо,
гроші з’являються як перший носій капітальної вартости, а тому грошовий
капітал — як форма, що в ній авансується капітал.

Як грошовий капітал, він перебуває в такому стані, що може виконувати
функції грошей, в даному разі функції загального купівельного
засобу і загального засобу виплати. (Останнє остільки, оскільки робочу
силу, хоч і раніш куплену, оплачується лише після того, як вона функціонувала.
Оскільки готових продукційних засобів немає на ринку, а
треба їх замовляти, в процесі $Г — Зп г$роші теж правлять за засіб виплати).
Ця здібність випливає не з того, що грошовий капітал є капітал, а з
того, що він — гроші.

З другого боку, капітальна вартість у грошовому стані може виконувати
лише функції грошей і жадних інших. Що перетворює функції
грошей на функції капіталу, це — їхня певна роля в русі капіталу, а тому
і зв’язок стадії, що в ній вони з’являються, з іншими стадіями його
кругобігу. Наприклад, у випадку, який тут насамперед розглядається,
гроші перетворюються на товари, що їхнє сполучення утворює натуральну
форму продуктивного капіталу, отже, форму, що лятентно, в
змозі, вже таїть у собі наслідок капіталістичного процесу продукції.

Частина грошей, що в $Г — Т \splitfrac{Р}{Зп}$ виконує функцію грошового капіталу, вивершивши цю циркуляцію, сама
переходить до такої функції, що в
ній зникає її характер капіталу, але лишається її характер грошей. Циркуляція
грошового капіталу Г розпадається на $Г — Зп$ і $Г — Р$, купівлю
засобів продукції та купівлю робочої сили. Розгляньмо останній процес
сам по собі. $Г — Р$ є купівля робочої сили з боку капіталіста; з боку
робітника, власника робочої сили, це є продаж робочої сили — ми можемо
сказати тут — продаж праці, бо тут наперед припускається форму заробітної
плати. Те, що для покупця є $Г — Т$ (= $Г — Р$), тут, як і в усякій
купівлі, для продавця (робітника) є $Р — Г$ (= $Т — Г$), продаж його робочої
сили. Це — перша стадія циркуляції, або перша метаморфоза товару
(книга І, розд. III, 2а); це — з боку продавця праці — перетворення його
товару на його грошову форму. Одержані таким чином гроші робітник
поступінно витрачає на певну суму товарів, що задовольняють його
потреби, на предмети споживання. Отже, ціла циркуляція його товару
виявляється як $Р — Г — Т$, тобто поперше $Р — Г$ (= $Т — Г$), а подруге $Г — Т$,
отже, у загальній формі простої товарової циркуляції $Т — Г — Т$, де гроші
фігурують як простий минущий засіб циркуляції, як простий посередник
в обміні товару на товар.

$Г — Р$ є характеристичний момент перетворення грошового капіталу
на продуктивний капітал, бо це є істотна умова для того, щоб вартість,
\parbreak{}  %% абзац продовжується на наступній сторінці
