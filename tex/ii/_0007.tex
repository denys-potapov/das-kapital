\parcont{}  %% абзац починається на попередній сторінці
\index{ii}{0007}  %% посилання на сторінку оригінального видання
авансована в грошовій формі, дійсно перетворилась на капітал, на вартість,
що продукує додаткову вартість. $Г — Зп$ доконечне лише для того,
щоб реалізувати кількість праці, куплену в процесі $Г — Р$. Тому $Г — Р$
з цього погляду описано в книзі І, відділ II, перетворення грошей
на капітал. Але тут треба розглянути справу ще з іншого погляду,
а саме спеціяльно щодо грошового капіталу як форми виявлення капіталу.

Акт $Г — Р$ звичайно розглядається як характеристичний для капіталістичного
способу продукції. Але зовсім не з тієї, зазначеної вище причини,
що купівля робочої сили є така купівельна угода, в якій зумовлено
постачання більшої кількости праці, ніж треба на покриття ціни робочої
сили, заробітної плати, отже, постачання додаткової праці, цієї основної
умови для капіталізації авансованої вартости, або, що є те саме, для
продукування додаткової вартости. Його визнається за характеристичний
скорше в наслідок його форми, бо в формі заробітної плати за \emph{гроші}
купується праця, а це вважається за ознаку грошового господарства.

Тут знову ж таки за характеристичне визнається не іраціональність
форми. Навпаки, іраціональности тут не помічають. Іраціональність тут у
тому, що сама праця як елемент, що утворює вартість, не може мати
жадної вартости, отже, і певна кількість праці не може мати жадної
вартости, яка визначалась би в її ціні, в її еквівалентності з певною
сумою грошей. Але ми знаємо, що заробітна плата є лише замаскована
форма, — форма, що в ній, напр., одноденна ціна робочої сили видається
як ціна праці, приведеної до стану поточности цією робочою силою протягом
одного дня, так що вартість, спродукована цією робочою силою
протягом, напр., 6 годин праці, стає виразом вартости дванадцятигодинного
функціонування робочої сили або дванадцятигодинної праці.

$Г — Р$ визнається за характеристичну рису, за ознаку так званого грошового
господарства, бо праця тут з’являється як товар її власника, а тому
гроші з’являються як покупець, отже — в наслідок грошових відносин
(тобто купівлі й продажу людської діяльности). Але гроші давно вже
виступають як покупець так званих послуг, а все ж $Г$ не перетворювалось
на грошовий капітал, і загальний характер господарства не
змінювався.

Для грошей цілком байдуже, на який ґатунок товарів їх перетворюється.
Вони є загальна еквівалентна форма всіх товарів, які вже
своїми цінами показують, що вони ідеально репрезентують певну суму грошей,
чекають свого перетворення на гроші і, лише помінявшись своїм місцем
з грішми, набирають такої форми, що в ній вони можуть перетворюватись на
споживні вартості для їхніх власників. Коли, отже, на ринку вже є робоча
сила як товар її власника, а продаж цього товару відбувається в формі
плати за працю, у вигляді заробітної плати, то купівля й продаж його
не являє собою нічого дивовижнішого порівняно з купівлею та
продажем усякого іншого товару. Характеристичне не в тому, що товар
— робочу силу можна купити, а в тому, що робоча сила являє собою
товар.
