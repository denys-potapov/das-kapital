
\index{ii}{0008}  %% посилання на сторінку оригінального видання
Через $Г — Т\splitfrac{Р}{Зп}$, через перетворення грошового капіталу на
продуктивний капітал, капіталіст досягає сполучення речових і особових
чинників продукції, оскільки ці чинники складаються з товарів. Коли
гроші вперше перетворюються на продуктивний капітал, або вперше
функціонують як грошовий капітал для свого власника, то він раніше,
ніж купити робочу силу, мусить спочатку купити засоби продукції,
приміщення робітні, машини тощо; бо, скоро робоча сила переходить
під його руку, в нього мусять уже бути засоби продукції, щоб її можна
було застосувати як робочу силу.

Так стоїть справа з боку капіталіста.

А з боку робітника так: продуктивне функціонування його робочої
сили стає можливе лише з того моменту, коли її в наслідок її продажу
ставиться в зв’язок з засобами продукції. Отже, до продажу вона існує відокремлено
від засобів продукції, від речових умов її функціонування.
У цьому стані відокремлення її не можна безпосередньо прикласти ні до
продукції споживних вартостей для її власника, ані до продукції товарів,
що, їх продаючи, міг би він жити. Але, скоро тільки в наслідок продажу
її поставлено в зв'язок з засобами продукції, вона стає складовою
частиною продуктивного капіталу, що належить її покупцеві, так само як
і засоби продукції.

Тому хоч в акті $Г — Р$ власник грошей і власник робочої сили
ставляться один до одного лише як покупець і продавець, виступають
один проти одного як власник грошей і власник товарів, отже, в цьому
розумінні між ними прості грошові відносини, — все ж з самого початку
покупець виступає разом з тим як власник засобів продукції, які являють
речові умови продуктивного витрачання робочої сили з боку її власника.
Інакше кажучи: ці засоби продукції виступають проти власника
робочої сили як чужа власність. З другого боку, продавець праці протистоїть
її покупцеві як чужа робоча сила, що мусить перейти під його
руку, мусить бути введена як складова частина в його капітал, щоб цей
останній дійсно міг функціонувати як продуктивний капітал. Отже, класове
відношення між капіталістом і найманим робітником уже є в наявності,
уже наперед дано в той момент, коли обидва вони виступають один
проти одного в акті $Г — Р$ ($Р — Г$ з боку робітника). Цей акт є купівля
й продаж, грошове відношення, але така купівля й продаж, де за покупця
припускається капіталіст, а за продавця — найманий робітник, і це відношення
дано тим, що умови для реалізації робочої сили — засоби існування
й засоби продукції — відокремлені як чужа власність від власника робочої
сили.

Тут нас не обходить, як постає це відокремлення. Воно існує, скоро
тільки відбувається акт $Г — Р$. Нас тут цікавить ось що: коли $Г — Р$ є
функція грошового капіталу, або коли гроші тут — форма існування
капіталу, то зовсім не тому, що гроші виступають тут як засіб виплати
за людську діяльність, яка має корисний ефект, за послугу; отже, ніяк
\parbreak{}  %% абзац продовжується на наступній сторінці
