\parcont{}  %% абзац починається на попередній сторінці
\index{ii}{0015}  %% посилання на сторінку оригінального видання
авансовані гроші функціонують як грошовий капітал, бо в наслідок
циркуляції вони перетворюються на товари специфічної споживної
вартости. Тут товар може функціонувати як капітал лише остільки,
оскільки вже з продукційного процесу раніш, ніж почалась його циркуляція,
він набув характеру капіталу. Протягом процесу прядіння утворили
прядільники вартість пряжі в 128\pound{ ф. стерл}. З них, припустімо, 50 фун.
стерл. становлять для капіталіста просто еквівалент за його витрати на
робочу силу, а 78\pound{ ф. стерл.} — при рівні експлуатації робочої сили в 156\%
— становлять додаткову вартість. Отже, вартість \num{10.000} ф. пряжі містить у
собі, поперше, вартість зужиткованого продуктивного капіталу П, що з неї
стала частина дорівнює 372\pound{ ф. стерл.}, змінна \deq{} 50\pound{ ф. стерл.}, їх
сума \deq{} 422\pound{ ф. стерл.}, — дорівнює \num{8.440} ф. пряжі. Але вартість
продуктивного капіталу П дорівнює Т, вартості його складових елементів,
що на стадії $Г — Т$ протистояли капіталістові як товари в руках їхніх
продавців. — Але, подруге, вартість пряжі містить у собі додаткову вартість,
в 78\pound{ ф. стерл.} \deq{} 1560 ф. пряжі. Отже, $Т$, як вираз вартости \num{10.000} ф.
пряжі, дорівнює $Т \dplus{} ΔТ$, $Т$ плюс приріст $Т$ (= 78\pound{ ф. стерл.}), що його ми й
позначимо $т$, бо він існує в тій самій товаровій формі, в якій тепер існує
первісна вартість $Т$. Вартість \num{10.000} ф. пряжі дорівнює 500\pound{ ф. стерл.}, отже,
вона дорівнює $Т \dplus{} т \deq{} Т'$. Що перетворює $Т$, як вираз \num{10.000} ф. пряжі, на
$Т'$ — це зовсім не абсолютна величина його вартости (500\pound{ ф. стерл.}), бо вона,
як і в усіх інших $Т$, оскільки вони є вираз вартости певної суми якихбудь
інших товарів, визначається кількістю зречевленої в ньому праці.
А перетворює його відносна величина його вартости, величина його
вартости порівняно з вартістю капіталу $П$, зужиткованого на його
продукцію. $Т'$ містить у собі цю останню вартість плюс додаткову
вартість, подану продуктивним капіталом. Його вартість більша, перевищує
цю капітальну вартість на цю додаткову вартість, $т$. \num{10.000} ф.
пряжі — це носії вирослої у своїй вартості, збагаченої додатковою вартістю
капітальної вартости, і вони являють собою таких носіїв, як
продукт капіталістичного продукційного процесу. $Т'$ виражає відношення
вартостей — відношення вартости товарового продукту до
вартости капіталу, витраченого на його продукцію; отже, виражає, що
його вартість складається з капітальної вартости й додаткової вартости.
\num{10.000} ф. пряжі є товаровий капітал, $Т'$, лише як перетворена форма
продуктивного капіталу $П$, отже, лише у зв’язку, що насамперед існує
тільки в кругобігу цього індивідуального капіталу, або лише для того
капіталіста, що своїм капіталом продукував пряжу. Це, так би мовити,
лише внутрішнє, а не зовнішнє відношення, що робить ці \num{10.000} ф.
пряжі як носіїв вартости товаровим капіталом. Капіталістична родинка
цих \num{10.000} ф. пряжі не в абсолютній величині їхньої вартости, а в її відносній
величині, у величині їхньої вартости порівняно з тією, яку мав
продуктивний капітал, що містився в них раніше, ніж він перетворився
на товар. Тому, коли \num{10.000} ф. пряжі продається за їхню вартість, за
500\pound{ ф. стерл.}, то цей акт циркуляції, розглядуваний сам по собі, $= Т — Г$,
є просте перетворення вартости, що лишається незмінною, з товарової форми
\parbreak{}  %% абзац продовжується на наступній сторінці
