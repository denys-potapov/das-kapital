\parcont{}  %% абзац починається на попередній сторінці
\index{ii}{0017}  %% посилання на сторінку оригінального видання
в 1560 ф. пряжі. Коли т, виражене в грошах, ми позначимо г, то
$Т' — Г'$ = (Т + т) — (Г + г), а кругобіг $Г — Т\dots{} П\dots{} Т' — Г'$ в його
розвиненій формі позначиться як $Г — Т \splitfrac{Р}{Зп}\dots{} П$\dots{} (Т + т) — (Г + г).

У першій стадії капіталіст вилучає предмети споживання з власне
ринку товарового і з ринку праці; в третій стадії він подає товари назад, але
тільки на один ринок, на ринок власне товаровий. А коли своїм товаром
він знову вилучає з ринку більше вартости, ніж подав туди раніше, то
лише тому, що він подає туди більшу товарову вартість, ніж раніше
вилучив. Він кинув на ринок вартість Г і вилучив рівновартість Т; він
кидає туди $Т + т$ і вилучає рівновартість $Г + г$. В нашому прикладі Г
було рівне вартості 8.440 ф. пряжі; але він кидає на ринок 10.000 ф.
пряжі, отже, дає йому більшу вартість, ніж сам узяв з нього. З другого
боку, він кинув туди цю вирослу вартість лише тому, що він експлуатацією
робочої сили в продукційному процесі спродукував додаткову
вартість (як складову частину продукту, виражену в додатковому продукті).
Лише як продукт цього процесу ця товарова маса є товаровий капітал,
носій капітальної вартости, що сама з себе зросла в своїй вартості. Актом $Т' —
Г'$ реалізується і авансовану капітальну вартість і додаткову вартість. Реалізація
обох їх відбувається одночасно в ряді продажів або також у продажу
цілої маси товарів одним заходом, що й виражає $Т' — Г'$. Але той самий акт
циркуляції $Т' — Г'$ є неоднаковий щодо капітальної вартости й додаткової вартости,
оскільки для кожної з них він виражає різну стадію їхньої циркуляції,
неоднаковий відділ того ряду метафорфоз, що його вони мають перебігати
протягом циркуляції. Додаткова вартість, т, тільки в продукційному
процесі з’явилася на світ. Отже, вона вперше виступає на товаровому
ринку і саме в товаровій формі; це є її перша форма циркуляції,
а тому й акт $т — г$ є перший акт її циркуляції або її перша метаморфоза,
отже, її ще треба доповнити протилежним актом циркуляції або протилежною
метаморфозою $г — т$\footnote{
Це має силу незалежно від того способу, в який ми відділяємо капітальну
вартість від додаткової вартости. У 10.000 ф. пряжі є 1560 ф. = 78\pound{ ф. стерл.} додаткової
вартости, але в 1 ф. пряжі = 1 шилінґ. міститься також 2,496 унцій = 1,872
пенні додаткової вартости
}.

Інша справа з циркуляцією, що її пророблює капітальна вартість Т
в тому самому акті циркуляції $Т' — Г'$, який для неї є акт $Т — Г$, де
$Т = П$, дорівнює первісно авансованому $Г. П$ервісна авансована
вартість відкрила перший акт своєї циркуляції як Г, як грошовий капітал,
і повертається назад за посередництвом акту $Т — Г$ до тієї самої форми; отже,
вона перебігла обидві протилежні фази циркуляції: 1) $Г — Т$ і 2) $Т — Г$
і перебуває знову в такій формі, що в ній вона знову може почати той
самий процес кругобігу. Те, що для додаткової вартости є перше
перетворення товарової форми на грошову форму, для капітальної вартости
є поворот або зворотне перетворення на її первісну грошову
форму.
