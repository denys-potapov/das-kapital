
\index{ii}{0018}  %% посилання на сторінку оригінального видання
Актом $Г — Т \splitfrac{Р}{Зп} г$рошовий капітал перетворено на рівновартісну
суму товарів, Р і Зп. Ці товари не функціонують більше як товари, як
предмети продажу. Їхня вартість існує тепер у руках їхнього покупця,
капіталіста, як вартість його продуктивного капіталу П. У функції П,
в продуктивному споживанні, вони перетворюються на ґатунок товару,
речево відмінний від засобів продукції, на пряжу, що в ній їхня вартість
не лише зберігається, але й збільшується з 422\pound{ ф. стерл.} до 500\pound{ ф. стерл}.
У наслідок цієї реальної метаморфози товари, вилучені з ринку в першій
стадії $Г — Т$, замінюється на речево й вартісно відмінний товар, що тепер
мусить функціонувати як товар, мусить перетворитись на гроші й піти в
продаж. Тому продукційний процес виступає тут лише як перерва в процесі
циркуляції капітальної вартости, що в ньому до цього часу перейдено лише
першу фазу $Г — Т$. Капітальна вартість перебігає другу й кінцеву фазу $Т — Г
п$ісля того, як Т речево й вартісно зміниться. Але коли розглядати капітальну
вартість саму по собі, то виявиться, що вона в продукційному
процесі зазнала змін тільки в своїй споживній формі. Вона існувала як
422\pound{ ф. стерл.} вартости в Р і Зп, тепер вона існує як 422\pound{ ф. стерл.}
вартости в 8440 ф. пряжі. Отже, коли ми будемо розглядати лише обидві
фази процесу циркуляції капітальної вартости, взятої окремо від її додаткової
вартости, то виявиться, що вона перебігає 1) $Г — Т$ і 2) $Т — Г$,
де друге Т є змінена споживна форма, але має ту саму вартість,
що й перше Т; отже, капітальна вартість перебігає $Г — Т — Г$, форму
циркуляції, що дворазовим переміщенням товару в протилежному
напрямку, перетворенням з грошей на товар, перетворенням з товару на
гроші, неминуче зумовлює поворот вартости, авансованої у формі грошей,
до її грошової форми, тобто зумовлює її зворотне перетворення на гроші.

Той самий акт циркуляції $Т' — Г'$, що для капітальної вартости, авансованої
в грошах, являє другу кінцеву метаморфозу, поворот до грошової
форми, для додаткової вартости, що є в тому самому товаровому капіталі
і разом з ним реалізується через перетворення його в грошову форму,
являє першу метаморфозу, перетворення з товарової форми на грошову
форму, $Т — Г$, першу фазу циркуляції.

Отже, тут треба зауважити дві обставини. Поперше: кінцеве зворотне
перетворення капітальної вартости на її первісну грошову форму є функція
товарового капіталу. Подруге: ця функція має в собі перше перетворення
форми додаткової вартости з її первісної товарової форми на грошову
форму. Таким чином грошова форма відіграє тут подвійну ролю: з одного
боку, вона є форма, в яку повертається вартість, первісно авансована
в грошах, отже, поворот до тієї форми вартости, що з неї почався
процес; з другого боку, вона є перша перетворена форма вартости,
що первісно входить у циркуляцію в товаровій формі. Коли товари,
що з них складається товаровий капітал, продається за їхню вартість,
як це тут припускається, то $Т \dplus{} т п$еретворюється на рівновартісне
$Г \dplus{} г$; в цій формі $Г \dplus{} г$ (422\pound{ ф. стерл.} \dplus{} 78\pound{ ф. стерл.} \deq{} 500 ф.
\parbreak{}  %% абзац продовжується на наступній сторінці
