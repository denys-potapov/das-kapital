\parcont{}  %% абзац починається на попередній сторінці
\index{ii}{0019}  %% посилання на сторінку оригінального видання
стерл.) існує тепер реалізований товаровий капітал у руках капіталіста.
Капітальна вартість і додаткова вартість існують тепер як гроші,
отже, в загальній еквівалентній формі.

Отже, наприкінці процесу капітальна вартість перебуває знову
в тій самій формі, що в ній вона увійшла в нього, і, отже, може знову,
як грошовий капітал, розпочати й перебігати цей процес. Саме тому, що
початкова й кінцева форма процесу являє собою форму грошового капіталу
($Г$), цю форму процесу кругобігу названо нами кругобігом
грошового капіталу. Кінець-кінцем, змінилась не форма, а лише величина
авансованої вартости.

$Г \dplus{} г$ є не що інше, як грошова сума певної величини, в даному разі
500\pound{ ф. стерл}. Але як наслідок кругобігу капіталу, як реалізований
товаровий капітал, ця грошова сума має в собі капітальну вартість
і додаткову вартість, і до того ж вони вже не зрослі одна з однією, як
у пряжі, а лежать тепер поряд. Реалізація їх дала кожній з них самостійну
грошову форму. \sfrac{211}{250} цієї суми є капітальна вартість, 422\pound{ ф.
стерл.}, і \sfrac{39}{250} її є додаткова вартість в 78\pound{ ф. стерл}. Це відокремлення,
спричинене реалізацією товарового капіталу, має не тільки формальний
зміст, про що ми казатимемо зараз; воно набирає важливости в процесі
репродукції капіталу, залежно від того, чи долучається $г$ до $Г$ цілком, чи
почасти, чи зовсім не долучається, отже, залежно від того, чи функціонує
воно далі як складова частина авансованої капітальної вартости, чи ні.
$г$ і $Г$ можуть також перебігати цілком різні циркуляції.

В $Г'$ капітал знову повернувся до своєї первісної форми $Г$, до своєї
грошової форми; але повернувся він у такій формі, що в ній він зреалізований
як капітал.

Поперше, тут є кількісна ріжниця. Було $Г$, 422\pound{ ф. стерл.}; тепер є $Г'$,
500\pound{ ф. стерл.}, і цю ріжницю виражено в $Г\dots{} Г'$, в кількісно різних
крайніх членах кругобігу, що його власний рух позначено лише крапками.
$Г'$ більше за $Г$, $Г'$ мінус $Г \deq{} М$, додатковій вартості. — Але як результат цього
кругобігу $Г\dots{} Г'$ тепер існує лише $Г'$; це є результат, що в ньому погас
процес його утворення. $Г'$ існує тепер самостійно само по собі, незалежно
від руху, що породив його. Рух минув, натомість маємо $Г'$.

Але $Г'$, як $Г \dplus{} г$, 500\pound{ ф. стерл.}, як 422\pound{ ф. стерл.} авансованого капіталу
плюс його приріст в 78\pound{ ф. стерл.}, являє разом з тим якісне відношення,
хоч саме це якісне відношення існує лише як відношення частин тієї
самої суми, отже, як кількісне відношення. $Г$, авансований капітал, що
тепер знову перебуває в своїй первісній формі (422\pound{ ф. стерл.}), існує
тепер як реалізований капітал. Він не лише зберігся, він також реалізувався
як капітал, бо саме як капітал відрізняється він від $г$ (78\pound{ ф. стерл.})
що до нього він стосується, як до \emph{свого} приросту, до \emph{свого} витвору, до
породженого ним самим приросту. Він реалізувався як капітал, тому що
він реалізувався як вартість, що породила вартість. $Г'$ існує як капіталістичне
відношення; $Г$ вже виступає не як просто гроші, а виразно як
\parbreak{}  %% абзац продовжується на наступній сторінці
