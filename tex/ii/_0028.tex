\parcont{}  %% абзац починається на попередній сторінці
\index{ii}{0028}  %% посилання на сторінку оригінального видання
$П\dots{} П'$ (П \dplus{} п), а в формі $Т'\dots{} Т'$ взагалі не видно жадної ріжниці щодо
вартости між двома крайніми членами. Отже, формула $Г\dots{} Г'$ відзначається,
з одного боку, тим, що капітальна вартість є вихідний пункт, а виросла
капітальна вартість є поворотний пункт, так що авансування
капітальної вартости є засіб, а виросла капітальна вартість — є мета
цілої операції; а, з другого боку, відзначається тим, що це відношення
виражається в грошовій формі, самостійній формі вартости, а тому грошовий
капітал — як гроші, що вилуплюють гроші. Утворення додаткової
вартости вартістю виражається не лише як альфа і омега процесу,
але й виразно в засліпній грошовій формі.

4) Що $Г'$, реалізований грошовий капітал як результат $Т' — Г'$, вивершної
та кінцевої фази $Г — Т$, перебуває абсолютно в тій самій формі, що в
ній він почав свій перший кругобіг, то, вийшовши з нього, він може
знову почати такий самий кругобіг, як збільшений (акумульований)
грошовий капітал: $Г' \deq{} Г \dplus{} г$; принаймні форма $Г\dots{} Г'$ не виражає, що
при повторенні кругобігу циркуляція $г$ відокремлюється від циркуляції $Г$. Розглядуваний у своєму одноразовому вигляді, формально, кругобіг
грошового капіталу виражає, отже, лише процес збільшення вартости й
процес акумуляції. Споживання в ньому виражено лише як про-дуктивне споживання
через $Г — Т\splitfrac{Р}{Зп}$, тільки таке споживання
і є в цьому кругобігу індивідуального капіталу. $Г — Р$ є $Р — Г$ або $Т — Г$
з боку робітника; отже, це перша фаза циркуляції, яка упосереднює його
індивідуальне споживання: $Р — Г — Т$ (засоби існування). Друга фаза, $Г — Т$,
вже не входить у кругобіг індивідуального капіталу; але цей кругобіг
призводить до неї, має її за свою передумову, бо робітник, щоб мати
змогу завжди бути на ринку як матеріял експлуатації для капіталіста,
насамперед мусить жити, тобто підтримувати себе особистим споживанням.
Але навіть це споживання припускається тут лише як умова продуктивного
споживання робочої сили капіталом, отже, лише остільки, оскільки
робітник своїм особистим споживанням підтримує себе й репродукує себе
як робочу силу.

Щодо $Зп$, власне товарів, які входять у кругобіг, то вони
становлять лише поживний матеріял для продуктивного споживання. Акт
$Р — Г$ упосереднює особисте споживання робітника, перетворення життьових
засобів на тіло його й кров. Звичайно, для того, щоб капіталіст
функціонував як капіталіст, він теж мусить існувати, а, значить, жити й
споживати. Але для цього в дійсності потрібне лише таке споживання, як і
робітникові, і більшого в цій формі процесу циркуляції не мислиться.
І навіть це тут формально не виражено, бо формула закінчується $Г'$,
тобто наслідком, що зразу ж знову може функціонувати як збільшений
грошовий капітал.

У $Т' — Г'$ безпосередньо міститься продаж $Т$; але $Т' — Г'$, продаж з
одного боку, є $Г — Т$, купівля з другого; а товар остаточно купується
лише заради його споживної вартости, для того, щоб (лишаючи осторонь
\parbreak{}  %% абзац продовжується на наступній сторінці
