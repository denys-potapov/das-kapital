\parcont{}  %% абзац починається на попередній сторінці
\index{ii}{0033}  %% посилання на сторінку оригінального видання
процес циркуляції промислового капіталу, ввесь його рух у межах фази
циркуляції становить лише перерву, а, значить, лише посередню ланку між
продуктивним капіталом, що як перший крайній член розпочинає кругобіг
і як останній закінчує його в тій самій формі, тобто в формі, що в ній
може відновитись. Власне циркуляція виступає лише як посередня ланка
періодично поновлюваної і через це поновлення безупинної репродукції.

Подруге. Ціла циркуляція з'являється тут у формі, цілком протилежній
тій, що її вона має в кругобігу грошового капіталу. Там ця форма,
лишаючи осторонь визначення вартости, була така: $Г — Т — Г$ ($Г — Т.~Т — Г$);
а тут, знову таки, лишаючи осторонь визначення вартости, вона
така: $Т — Г — Т$ ($Т — Г.~Г — Т$), отже, форма простої товарової циркуляції.

\subsection{Проста репродукція}

Розгляньмо насамперед процес $Т' — Г' — Т$, що відбувається в сфері
циркуляції між крайніми членами $П\dots{}П$.

Вихідний пункт цієї циркуляції є товаровий капітал: $Т' \deq{} Т \dplus{} т
\deq{} П \dplus{} т$. Функцію товарового капіталу $Т' — Г'$ (реалізація капітальної вартости,
що міститься в ньому, $\deq{} П$, що існує тепер як складова частина
товару $Т$, а також реалізація додаткової вартости, яка міститься в
ньому, та, маючи вартість $т$, існує як складова частина тієї таки
товарової маси) розглянули ми в першій формі кругобігу. Але там становила
вона другу фазу перерваної циркуляції і кінцеву фазу цілого кругобігу.
Тут вона становить другу фазу кругобігу, але першу фазу циркуляції.
Перший кругобіг закінчується $Г'$, а що $Г'$ так само, як і первісне
$Г$, може знову як грошовий капітал почати другий кругобіг, то спочатку
не було потреби розглядати, чи продовжуватимуть $Г$ і $г$ (додаткова
вартість), що містяться в $Г'$, свій шлях спільно, чи кожне з них перебігатиме
свій відмінний шлях. Це було б потрібно зробити лише тоді,
коли б ми простежили перший кругобіг у його дальшому відновленні.
Але в кругобігу продуктивного капіталу цей пункт мусить бути розв’язаний,
бо від цього розв’язання залежить уже визначення його першого
кругобігу, а також і тому, що $Т' — Г'$ є в ньому перша фаза циркуляції,
що її треба доповнити через $Г — Т$. Від цього розв’язання залежить,
чи позначає формула просту репродукцію, чи репродукцію в поширеному
маштабі. Отже, залежно від цього розв’язання змінюється й характер
кругобігу.

Отже, візьмімо спочатку просту репродукцію продуктивного капіталу
і при цьому, як і в першому розділі, припустімо, що обставини лишаються
незмінні і товари купується й продається за їхньою вартістю. Ціла
додаткова вартість при такому припущенні ввіходить у сферу особистого
споживання капіталіста. Скоро товаровий капітал $Т'$ перетворився на
гроші, то частина грошової суми, що репрезентує капітальну вартість,
циркулює далі в кругобігу промислового капіталу; друга частина, перетворена
на золото додаткова вартість, увіходить у загальну товарову
циркуляцію та являє собою грошову циркуляцію, що виходить від капіталіста,
\index{ii}{0034}  %% посилання на сторінку оригінального видання
проте, таку, що відбувається поза циркуляцією його індивідуального
капіталу.

У нашому прикладі мали ми товаровий капітал $Т'$ в \num{10.000} ф.
пряжі вартістю в 500\pound{ ф. стерл.}; з них 422\pound{ ф. стерл.} є вартість продуктивного
капіталу; як грошова форма 8440 ф. пряжі вони й далі
продовжують циркуляцію капіталу, почату $Т'$, тимчасом як додаткова
вартість в 78\pound{ ф. стерл.}, грошова форма 1560 ф. пряжі, надлишкової
частини товарового продукту, виходить із цієї циркуляції і чинить
свій окремий шлях у межах загальної товарової циркуляції.

\[
 Т'
 \begin{pmatrix}
  T \\
  \dplus{} \\
  т
 \end{pmatrix}
 \begin{array}{cc} - & - \\ - & Г' \\ - & - \end{array}
 \begin{pmatrix}
  Г \\
  \dplus{} \\
  г
 \end{pmatrix}
 \begin{array}{ll} — & Т \splitfrac{Р}{Зп} \\ ~ \\ - & т  \end{array}
\]
$г — т$ є ряд купівель на гроші, що їх капіталіст витрачає або на власне
товари, або на послуги для своєї поважної особи, або для сім’ї. Ці
купівлі розпорошені, відбуваються в різний час. Отже, гроші існують
тимчасово в формі певного грошового запасу або скарбу,
призначеного на поточне споживання, бо гроші, що їхня циркуляція
перервалась, перебувають у формі скарбу. Їхнє функціонування як
засобу циркуляції — а такі вони є і в своїй тимчасовій формі скарбу — не
входить у циркуляцію капіталу в його грошовій формі $Г$. Гроші тут
не авансується, а витрачається.

Ми припускали, що ввесь авансований капітал завжди цілком переходить
з однієї його фази до іншої; так само й тут ми припускаємо, що товаровий
продукт $П$ має в собі всю вартість продуктивного капіталу
$П \deq{} 422$\pound{ ф. стерл.} плюс додаткова вартість \deq{} 78\pound{ ф. стерл.}, утворена
протягом продукційного процесу. В нашому прикладі, де ми маємо справу
з подільним товаровим продуктом, додаткова вартість існує в формі
1560 ф. пряжі, цілком так само, як обчислена на 1 ф. пряжі, вона існує
у формі 2,496 унцій пряжі. Коли б, навпаки, товаровий продукт був, прим., машиною
в 500\pound{ ф. стерл.} і такого самого складу щодо вартости, то хоча б
одна частина вартости цієї машини була рівна 78\pound{ ф. стерл.} додаткової
вартости, все ж ці 78\pound{ ф. стерл.} існували б лише в машині як цілому;
машину не можна поділити на капітальну вартість і додаткову вартість,
не розбиваючи її на куски й не знищуючи таким чином разом з її споживною
вартістю і її вартість. Отже, обидві складові частини вартости
можна лише ідеально уявляти собі як складові частини товарового тіла,
але не можна їх визначати як самостійні елементи товару $Т'$, подібно до
кожного фунту пряжі, що його можна визначити як віддільний, самостійний
товаровий елемент \num{10.000} ф. пряжі. У першому випадку цілий товар, товаровий
капітал, машина мусить бути цілком продана, раніше, ніж $г$ зможе розпочати
свою окрему циркуляцію. Навпаки, коли капіталіст продає 8440 ф.,
то продаж дальших 1560 ф. являє цілком відокремлену циркуляцію
\parbreak{}  %% абзац продовжується на наступній сторінці
