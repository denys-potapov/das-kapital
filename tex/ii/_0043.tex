\parcont{}  %% абзац починається на попередній сторінці
\index{ii}{0043}  %% посилання на сторінку оригінального видання
Власники їх мусять оголосити себе невиплатоспроможними або продавати
за яку завгодно ціну, щоб зробити виплати. Такий продаж не має
жадного чинення до дійсного стану попиту. Він має чинення лише
з \emph{попитом на виплату}, з абсолютною доконечністю перетворити товар на
гроші. Тоді вибухає криза. Вона виявляється ие в безпосередньому
зменшенні споживного попиту, попиту для особистого споживання, але
в скороченні обміну капіталу на капітал, у скороченні процесу репродукції
капіталу.

Коли товари $Зп$ і $Р$, що на них перетворюється $Г$ для того, щоб
виконати свою функцію як грошовий капітал, як капітальна
вартість, призначена для зворотного перетворення на продуктивний
капітал, — коли ці товари треба купувати або оплачувати в різний час,
отже, коли $Г — Т$ репрезентує ряд купівель і виплат, що відбуваються
одна по одній, то частина $Г$ виконує акт $Г — Т$, тимчасом як друга частина
лишається у формі грошей, щоб потім у момент, визначений умовами
самого процесу, придатись для одночасних або послідовних актів $Г — Т$.
Цю частину лише тимчасово вилучено з циркуляції, щоб у певний
момент вона могла перейти до дії, здійснити свою функцію. Але таке
припасання цієї частини є також функція визначувана її циркуляцією
і призначена для циркуляції. Отже, її існування як купівельного й виплатного
запасу, припинення її руху, перерва в її циркуляції теж являє стан,
що в ньому гроші виконують одну із своїх функцій як грошовий капітал; як
грошовий капітал тому, що в цьому разі гроші, які тимчасово перебувають
у стані спокою, є частина грошового капіталу $Г$ (частина від
$Г'$ мінус $г \deq{} Г$), тієї частини вартости товарового капіталу, яка дорівнює $П$,
вартості продуктивного капіталу, що є вихідний пункт кругобігу. З другого
боку, всі гроші, вилучені з циркуляції, перебувають у формі скарбу.
Отже, скарбова форма грошей стає тут функцією грошового капіталу, так
само, як в $Г — Т$ функція грошей, як купівельного або виплатного засобу, стає
функцією грошового капіталу, і це саме тому, що капітальна вартість
існує тут у грошовій формі, грошова форма являє тут стан промислового
капіталу, визначуваний загальним зв’язком його кругобігу на одній з його
стадій. Тут знову має силу те, що грошовий капітал у межах кругобігу
промислового капіталу не виконує жадних інших функцій, крім функцій
грошей, і що ці функції грошей лише в наслідок свого зв’язку
з іншими стадіями цього кругобігу одночасно мають значіння функцій
капіталу.

Що $Г'$ виражає відношення $г$ до $Г$, капіталістичне відношення, це
безпосередньо є функція не грошового капіталу, а товарового капіталу $Т'$,
який знову таки, як відношення $т$ до $Т$, виражає лише результат продукційного
процесу, результат посталого там самозростання капітальної
вартости.

Коли процес циркуляції в своєму перебігу наражається на перешкоди,
так що $Г$ в наслідок зовнішніх обставин, стану ринку і т. ін., мусить
припинити свою функцію $Г — Т$ і тому на більш або менш довгий час
лишатись у своїй грошовій формі, то знову ми маємо стан грошей у
\parbreak{}  %% абзац продовжується на наступній сторінці
