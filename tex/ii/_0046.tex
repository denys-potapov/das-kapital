\parcont{}  %% абзац починається на попередній сторінці
\index{ii}{0046}  %% посилання на сторінку оригінального видання
собі як окремий кругобіг, виражає лише одне, а саме, що $Г$, грошовий
капітал (або промисловий капітал у своєму кругобігу як грошовий
капітал), є гроші, що вилуплюють гроші, вартість, що вилуплює вартість,
— що він породжує додаткову вартість. Навпаки, в кругобігу $П$,
коли вивершено першу стадію продукційного процесу, процес зростання
вартости вже закінчено, а по скінченні другої стадії $Т' — Г'$ (першої
стадії циркуляції) капітальна вартість плюс додаткова вартість існують
уже як реалізований грошовий капітал, як $Г'$, що являло останній крайній
член у першому кругобігу. Що спродуковано додаткову вартість, це в
раніш розглянутій формі $П\dots{}П$ (дивись розгорнуту формулу на стор.~\pageref{original-41})
позначено через $т — г — т$, яке в своїй другій стадії виходить
з меж циркуляції капіталу й являє циркуляцію додаткової вартости як
доходу. Отже, в цій формі, де ввесь рух зображаєтся як $П\dots{} П$ і де,
отже, не відбувається жадної зміни вартости між обома крайніми пунктами,
зростання авансованої вартости, утворення додаткової вартости, зображено
так само, яків $Г\dots{} Г'$; тільки акт $Т' — Г'$, що є остання стадія в $Г\dots{} Г'$
і друга стадія кругобігу, являє першу стадію циркуляції в $П\dots{}П$.

$П'$ в $П\dots{}П'$ виражає не те, що випродуковано додаткову вартість, а
те, що спродуковану додаткову вартість капіталізовано, отже, що капітал
акумульовано, а тому $П'$, протилежно до $П$, складається з первісної капітальної
вартости плюс вартість капіталу, акумульованого в наслідок руху
капітальної вартости.

$Г'$, як просте вивершення $Г\dots{} Г'$, так само і $Т'$, як воно виступає
в усіх цих кругобігах, виражають самі по собі не рух, а його результат:
зростання капітальної вартости, реалізоване в товаровій формі або в грошовій,
а тому капітальну вартість як $Г \dplus{} г$ або як $Т \dplus{} т$, як відношення
капітальної вартости до додаткової вартости, як до свого нащадка. Вони
виражають цей результат як різні форми циркуляції вирослої капітальної
вартости. Але ні в формі $Т'$, ні в формі $Г'$ постале зростання вартости
не є ні функція грошового капіталу, ні функція товарового капіталу. Як
особливі, різні форми, як форми буття капіталу, що відповідають особливим
функціям промислового капіталу, грошовий капітал і товаровий капітал
можуть виконувати відповідно лише функції грошей і функції товару,
і ріжниця між ними — це лише ріжниця між грішми й товаром. Так само
промисловий капітал у своїй формі продуктивного капіталу може складатись
тільки з тих самих елементів, що й кожний інший процес праці,
який утворює продукти: з одного боку, з речевих умов праці (засобів
продукції), з другого боку, з продуктивно (доцільно) діющої робочої
сили. Так само, як промисловий капітал може існувати в сфері продукції
лише в такому складі, що відповідає продукційному процесові взагалі,
а, значить, і некапіталістичному продукційному процесові, так само в сфері
циркуляції він може існувати лише в тих двох формах, що відповідають
цій сфері — у формі товару й грошей. Але, як сума елементів продукції
з самого початку виявляє сеое як продуктивний капітал через те, що робоча
сила є чужа робоча сила, яку капіталіст купив у її власника, так само як
він купує собі засоби продукції в інших товаровласників; отже, як і самий
\parbreak{}  %% абзац продовжується на наступній сторінці
