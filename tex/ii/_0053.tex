\parcont{}  %% абзац починається на попередній сторінці
\index{ii}{0053}  %% посилання на сторінку оригінального видання
капітальної вартости $П$ плюс додаткова вартість $m$, спродукована в
наслідок функціонування $П$.

Тільки в кругобігу самого $Т'$ частина його $Т \deq{} П \deq{}$ капітальній вартості
може й мусить відокремитись від тієї частини $Т'$, що в ній існує додаткова
вартість, від додаткового продукту, що в ньому міститься додаткова
вартість, — може й мусить відокремитись незалежно від того, чи обидва
вони справді подільні, як от пряжа, чи ні, як от машина. Вони стають
подільні кожного разу, скоро $Т'$ перетворюється на $Г'$.

Коли цілий товаровий продукт можна поділити на самостійні однорідні
частинні продукти, як, прим., наші \num{10.000} ф. пряжі, і коли в
наслідок цього акт $Т' — Г'$ можна зобразити як суму послідовно вчинених
продажів, то капітальна вартість може функціонувати в товаровій
формі як $Т$, може відокремитись від $Т'$ раніше, ніж реалізується додаткова
вартість, отже, раніш, ніж реалізовано цілком усе $Т'$.

В \num{10.000} ф. пряжі в 500\pound{ ф. стерл.}, вартість \num{8.440} ф. пряжі \deq{} 422\pound{ф.
стерлін.} \deq{} капітальній вартості, відокремленій від додаткової вартости.
Коли капіталіст продає спочатку \num{8.440} ф. пряжі за 422\pound{ ф. стерл.}, то
ці \num{8.440} ф. пряжі репрезентують $Т$, капітальну вартість у товаровій
формі; додатковий продукт, що є, крім того, в $Т'$, а саме 1560 ф.
пряжі \deq{} додатковій вартості в 78\pound{ ф. стерл.} ввійде в циркуляцію лише
пізніше; капіталіст міг би здійснити $Т — Г — Т\splitfrac{Р}{Зп}$ перед циркуляцією
додаткового продукту $т — г — т$.

Або коли б він спочатку продав 7440 ф. пряжі вартістю в 372\pound{ ф.
стерл.}, а потім 1000 ф. пряжі вартістю в 50\pound{ ф. стер.}, то першою частиною
$Т$ можна було б покрити засоби продукції (сталу частину капіталу
$с$), а другою частиною $Т$ — змінну частину капіталу $v$, робочу
силу, — а далі все відбувалось би, як і раніш.

Але коли відбуваються такі послідовні продажі і коли умови кругобігу
це дозволяють, то капіталіст замість поділити $Т'$ на $c \dplus{} v \dplus{} m$,
може зробити такий поділ і в аліквотних частинах $Т'$.

Наприклад, 7440 ф. пряжі \deq{} 372\pound{ ф. стерл.}, що як частини $Т'$
(\num{10.000} ф. пряжі \deq{} 500\pound{ ф. стерл.}) репрезентують сталу частину капіталу,
своєю чергою можуть бути розкладені на 5535,360 ф. пряжі вартістю
в 276,768\pound{ ф. стерл.}, що покривають лише сталу частину капіталу, вартість
засобів продукції, зужиткованих у виготовленні \num{7.440} ф. пряжі; на 744 ф. пряжі
вартістю в 37,200\pound{ ф. стер.}, що покривають лише змінний капітал; на
1160,640 ф. пряжі вартістю в 58,032\pound{ ф. стерл.}, що, як додатковий продукт,
є носії додаткової вартости. Отже, з проданих \num{7.440} ф. пряжі
капіталіст може покрити вміщену в них капітальну вартість, продавши
6279,360 ф. пряжі за 313,968\pound{ ф. стерл.}, а вартість додаткового продукту
1160,640 ф. пряжі \deq{} 58,032\pound{ ф. стерл.} витратити як дохід.

Так само може він і далі розкласти й відповідно до цього продати
1000 ф. пряжі \deq{} 50\pound{ ф. стерл.} \deq{} змінній капітальний вартості: 744 ф.
пряжі за 37,200\pound{ ф. стерл.} — стала капітальна вартість 1000 ф. пряжі;
\parbreak{}  %% абзац продовжується на наступній сторінці
