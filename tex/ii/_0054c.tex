\parcont{}  %% абзац починається на попередній сторінці
\index{ii}{0054}  %% посилання на сторінку оригінального видання
100 ф. пряжі за 5,000\pound{ ф. стерл.} — змінна частина капіталу тих таки
1000 ф. пряжі; отже, 844 ф. пряжі за 42,200\pound{ ф. стерл.} — покриття
капітальної вартости, яка є в 1000 ф. пряжі; нарешті, 156 ф. пряжі
вартістю в 7,800\pound{ ф. стерл.}, які репрезентують додатковий продукт, що
є в ній, і які можуть бути спожиті як такий.

Нарешті, може він решту 1560 ф. пряжі вартістю в 78\pound{ ф. стерл.},
якщо пощастить її продати, розкласти так, що продаж 1160,640 ф.
пряжі за 58,032\pound{ ф. стерл.} покриватиме вартість засобів продукції, що
містяться в 1560 ф. пряжі, а 156 ф. пряжі вартістю в 7,800\pound{ ф. стерл.} —
змінну капітальну вартість; разом 1316,640 ф. пряжі \deq{} 65,832\pound{ ф. стерл.},
покривають усю капітальну вартість; нарешті, лишається додатковий
продукт 243,360 ф. пряжі \deq{} 12,168\pound{ ф. ст.}, що їх можна витрачати як
дохід.

Так само, як кожен елемент, що існує в пряжі — с, v, m, можна
знову розкласти на ці самі складові частини, так само можна розкласти й
кожен окремий фунт пряжі вартістю в 1\shil{ шилінґ} \deq{} 12\pens{ пенсів}:
\begin{table}[h]
% TODO: нужно поправить центрирование всей таблици
\centering
\setlength{\tabcolsep}{2pt}
\begin{tabularx}{\textwidth}{r c c c c c c}

с \deq{} & 0,744 & ф. & пряжі & \deq{} & 8,928 & \pens{ пенсів} \\
v \deq{} & 0,100 & „ & „ & \deq{} & 1,200 & „ \\
m \deq{} & 0,156 & „ & „ & \deq{} & 1,872 & „ \\
\cmidrule{1-7}
$c \dplus{} v \dplus{} m$  \deq{} & 1,000 & ф. & пряжі & \deq{} & 12,000 &\pens{ пенсів} \\
\end{tabularx}
\end{table}
Коли ми складемо результати трьох зазначених частинних продажів, то
результат буде такий самий, як і пои одночасному продажу \num{10.000} ф.
пряжі.

Сталого капіталу ми маємо:

\begin{table}[h]
  \setlength{\tabcolsep}{2pt}
  \begin{tabularx}{\textwidth}{c c c c c c c c c c}
    При & 1-му & продажу & 5535,360 & ф. & пряжі & \deq{} & 276,768 & ф. & стерл.\\
    „ & 2-му & „ & 744,000 & „ & „ & \deq{} & 37,200 & „ & „\\
    „ & 3-му & „ & 1160,640 & „ & „ & \deq{} & 58,032 & „ & „\\
    \cmidrule{1-10}
    \multicolumn{3}{c}{Разом} & 7440,000 & ф. & пряжі & \deq{} & 372,000 & ф. & стерл.\\
\end{tabularx}
\end{table}

Змінного капіталу:

\begin{table}[h]
  \setlength{\tabcolsep}{2pt}
  \begin{tabularx}{\textwidth}{c c c c c c c c c c}
    При & 1-му & продажу & 744,000 & ф. & пряжі & \deq{} & 37,200 & ф. & стерл.\\
    „ & 2-му & „ & 100,000 & „ & „ & \deq{} & 5,000 & „ & „\\
    „ & 3-му & „ & 156,000 & „ & „ & \deq{} & 7,800 & „ & „\\
    \cmidrule{1-10}

  \multicolumn{3}{c}{Разом} & 1000,000 & ф. & пряжі & \deq{} & 50,000 & ф. & стерл.\\
\end{tabularx}
\end{table}

Додаткової вартости:

\begin{table}[h]
  \setlength{\tabcolsep}{2pt}
  \begin{tabularx}{\textwidth}{c c c c c c c c c c}
    При & 1-му & продажу & 1160,640 & ф. & пряжі & \deq{} & 58,032 & ф. & стерл.\\
    „ & 2-му & „ & 156,000 & „ & „ & \deq{} & 7,800 & „ & „\\
    „ & 3-му & „ & 243,360 & „ & „ & \deq{} & 12,168 & „ & „\\
    \cmidrule{1-10}
    \multicolumn{3}{c}{Разом} & 1560,000 & ф. & пряжі & \deq{} & 78,000 & ф. & стерл.\\
  \end{tabularx}
\end{table}
\parbreak{}  %% абзац продовжується на наступній сторінці
