\parcont{}  %% абзац починається на попередній сторінці
\index{ii}{0055}  %% посилання на сторінку оригінального видання
Загальна сума:
\begin{table}[h]
  \setlength{\tabcolsep}{2pt}
  \begin{tabularx}{\textwidth}{l r c c c r c c}
    Сталий  капітал  & 7440 & ф. & пряжі & = & 372 & ф. & стерл.\\
    Змінний  \ditto{капітал}  & 1000 & \ditto{ф.} & \ditto{пряжі} & = & 50 & \ditto{ф.} & \ditto{стерл.} \\
    Додаткова  вартість  & 1560 & \ditto{ф.} & \ditto{пряжі} & = & 78 & \ditto{ф.} & \ditto{стерл.}\\
    \cmidrule{1-8}
    \multicolumn{1}{c}{Разом}  & 10.000 & ф. & пряжі & = & 500 & ф. & стерл.\\
  \end{tabularx}
\end{table}

$Т' — Г'$ само по собі є не що інше, як продаж 10.000 ф. пряжі.
10.000    ф. пряжі є товар, як і всяка інша пряжа. Для покупця має значення
ціна в 1 шил. за фунт, або 500\pound{ ф. стерл.} за 10.000 ф. пряжі. Коли
підчас торгу він і звертає увагу на склад вартости, то лише маючи
хитрий намір довести, що 1 ф. можна було б продати дешевше, ніж за
1 шил., і що навіть у цьому разі продавець усе ж зробить вигідну оборудку.
Але кількість товару, що його він купує, залежить від його
потреб; наприклад, коли він власник ткацького підприємства, то ця
кількість залежить від складу його власного капіталу, що функціонує в
підприємстві, але не від складу капіталу того прядуна, що в нього він
купує. Відношення, що в них $Т'$ повинне, з одного боку, покрити
зужиткований у процесі його продукції капітал (тобто різні його складові
частини), а, з другого боку, правити за додатковий продукт, призначений
чи то на витрачання додаткової вартости, чи то на акумуляцію капіталу,
існують лише в кругобігу капіталу, що його товарову форму являють
10.000 ф. пряжі. З продажем, як таким, вони не мають нічого спільного.
Тут, крім того припускається, що $Т'$ продається по своїй вартості, тобто
справа сходить лише на перетворення його з товарової форми на грошову.
Для $Т'$, як для функціональної форми в кругобігу цього індивідуального
капіталу, — форми, що з неї треба покрити продуктивний капітал, має,
природно, вирішувальне значення, чи відхиляються та до якої міри відхиляються
одне від одного ціна і вартість при продажу, але тут, розглядаючи
самі лише ріжниці щодо форм, нам не потрібно досліджувати
цього.

У формі І, $Г\dots{} Г'$, процес продукції відбувається посередині між двома
протилежними, що одна одну доповнюють, фазами циркуляції капіталу;
і він закінчується раніше, ніж надійде кінцева фаза $Т' — Г'$. Гроші авансується
як капітал, спочатку на елементи продукції, з них вони перетворюються
на товаровий продукт, і цей товаровий продукт знову перетворюється
на гроші. Це — цілком вивершений цикл оборудок, що результат
його є на все й для кожного придатні гроші. Таким чином відновлення
процесу дано лише в можливості. $Г\dots{} П$\dots{} $Г'$ може бути так само останнім
кругобігом, що вивершує функціонування індивідуального капіталу, який
виходить з підприємства, як і першим кругобігом капіталу, що вперше
вступає у функціонування. Загальний рух тут є $Г\dots{} Г'$, від грошей до більшої
суми грошей.

У формі II, тобто у формі $П\dots{} Т' — Г' — Т\dots{} П$($П'$) увесь процес циркуляції
йде за першим $П$ і поперед другого, але відбувається він зворотним
порядком проти форми І. Перше $П$ є продуктивний капітал, і його функдія
\index{ii}{0056}  %% посилання на сторінку оригінального видання
е його процес продукції, як попередня умова наступного процесу
циркуляції. Навпаки, кінцеве $П$ не є процес продукції; воно є лише
повторне буття промислового капіталу в його формі продуктивного
капіталу. А саме воно е результат перетворення, що сталося в
останній фазі циркуляції — перетворення капітальної вартости на $Р + Зп$,
на суб’єктивні й об’єктивні чинники, які в своєму сполученні становлять
форму буття продуктивного капіталу. Капітал, хоч буде він $П$, хоч $П'$,
кінець-кінцем, є знову наявний у такій формі, що в ній він мусить знову
функціонувати як продуктивний капітал, здійснювати процес продукції.
Загальна форма руху, $П\dots{} П$, є форма репродукції і не показує, як це показує
$Г\dots{} Г'$, що зростання вартости є мета цього процесу. Тому то більше полегшує
вона клясичній економії змогу не звертати уваги на певну капіталістичну
форму продукційного процесу й зображати продукцію як таку
за мету процесу, яка є ніби в тім, щоб яко мога більше й дешевше
продукувати й обмінювати продукт на якнайрізноманітніші інші продукти,
почасти для відновлення продукції $(Г — Т)$, почасти для споживання
% TODO: fix small latters in formulas
$(г — т)$. А що при цьому $Г$ і $г$ з’являються лише як минущий засіб
циркуляції, то особливостей грошей і грошового капіталу не помічається, і
ввесь процес здається простим і природним, тобто має природність
плаского раціоналізму. Так само, розглядаючи товаровий капітал, забувають
іноді зиск, і коли мовиться про кругобіг продукції в цілому, то капітал
фігурує лише як товар; але коли мова йде про складові частини вартости,
то він фігурує як товаровий капітал. Акумуляція зображається, природно,
таким самим способом, як і продукція.

У формі III, $Т' — Г' — Т\dots{} П\dots{} Т'$, кругобіг починають дві фази
процесу циркуляції, і саме таким самим порядком, як і у формі II,
$П\dots{} П$.; потім іде $П$, саме так, як і в формі І, із своєю функцією,
з процесом продукції; разом з результатом цього процесу, $Т'$, кругобіг
закінчується. Так само, як у формі II, він закінчується $П$, як простим
повторним буттям продуктивного капіталу, так само тут він закінчується
$Т'$, повторним буттям товарового капіталу; так само, як у
формі II капітал у своїй кінцевій формі П знову мусить розпочати
процес як процес продукції, так само й тут разом з повторною появою
промислового капіталу в формі товарового капіталу, кругобіг мусить
знову початись фазою циркуляції $Т' — Г'$. Обидві форми кругобігу тут
невивершені, бо їх ще не вивершило $Г'$, перетворена на гроші виросла
капітальна вартість. Обидві, отже, вони мусять провадитись далі, а тому
містять у собі репродукцію. Цілий кругобіг у формі III є $Г\dots{} Т'.

Третя форма від двох попередніх відрізняється тим, що лише в цьому
кругобігу за вихідний пункт зростання вартости є виросла капітальна
вартість, а не первісна капітальна вартість, що лише ще має зростати.
$Т'$ як капіталістичне відношення є тут вихідний пункт і як таке визначально
впливає на ввесь кругобіг, бо вже на першій фазі
своїй воно містить у собі кругобіг капітальної вартости та кругобіг
додаткової вартости, а додаткова вартість, коли й не в кожному окремому
кругобігу, то пересічно, мусить витрачатись почасти як дохід, пророблювати
\index{ii}{0057}  %% посилання на сторінку оригінального видання
циркуляцію $т — г — т$, почасти функціонувати як елемент акумуляції капіталу.

У формі $Т'\dots{} Т'$ споживання сукупного товарового продукту припускається, як умова нормального
перебігу кругобігу самого капіталу. Особисте споживання робітника та особисте споживання тієї
частини додаткової вартости, яку не акумулюється, охоплює все особисте споживання. Отже, споживання
в цілому — і особисте і продуктивне — входить як умова в кругобіг $Т'$. Продуктивне споживання (куди
посутньо входить і особисте споживання робітника, бо робоча сила в певних межах є постійний продукт
особистого споживання робітника) відбувається безпосередньо за допомогою кожного індивідуального
капіталу.
Особисте споживання остільки, оскільки воно потрібне для існування
індивідуального капіталіста — припускається тільки як суспільний акт, ні в якому разі як акт
індивідуального капіталіста.

У формах I і II ввесь рух зображено, як рух авансованої капітальної вартости. У формі III вирослий у
своїй вартості капітал, у вигляді цілого товарового продукту, становить вихідний пункт, і має форму
капіталу, що рухається, товарового капіталу. Лише після його перетворення на гроші, цей рух
розгалужується на рух капіталу й на рух доходу. В цій формі в кругобіг капіталу включено і розподіл
цілого суспільного продукту і особливий розподіл продукту для кожного індивідуального товарового
капіталу — розподіл, з одного боку, на фонд особистого споживання, а з другого — на фонд
репродукції.

$Г\dots{} Г'$ містить у собі поширення кругобігу залежно від величини $г$, яке
входить у поновлений кругобіг.

$П$ в $П\dots{} П$ може почати новий кругобіг з тією самою вартістю, а може і з меншою — і все ж воно може
являти репродукцію в поширених розмірах; так, напр., тоді, коли елементи товару подешевшають у
наслідок підвищеної продуктивности праці. Навпаки, в протилежному випадку, вирослий своєю вартістю
продуктивний капітал може являти репродукцію в речево вужчих розмірах, коли, напр., елементи
продукції дорожчають. Те саме має силу й для $Т'\dots{} Т'$.

В $Т'\dots{} Т'$ капітал у товаровій формі є передумова продукції; і знову таки як передумова повертається
він у межах цього кругобігу в другому $Т$. Якщо це $Т$ ще не спродуковано або не репродуковано, то
кругобіг зупиняється: це $Т$ мусить бути репродуковано здебільша як $Т'$ якогось іншого промислового
капіталу. В цьому кругобігу $Т'$ існує як вихідний пункт, переходовий пункт і кінцевий пункт руху, — і
воно тому завжди є наявне. Воно є постійна умова процесу репродукції.

$Т'\dots{} Т'$ відрізняється від форм I і II ще й іншим моментом. У всіх трьох кругобігів спільне те, що
форма, в якій капітал починає процес свого кругобігу, є також і форма, що в ній він закінчує його, а
тому знову набирає початкової форми, в якій він знову починає той самий кругобіг. Початкова форма $Г,
П, Т'$ є завжди та форма, що в ній авансується капітальна вартість (у III формі з прирослою до неї
додатковою вартістю) — отже, її первісна форма щодо кругобігу; кінцева
\parbreak{}  %% абзац продовжується на наступній сторінці
