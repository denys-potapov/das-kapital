\parcont{}  %% абзац починається на попередній сторінці
\index{ii}{0058}  %% посилання на сторінку оригінального видання
форма $Г'$, $П$, $Т'$ завжди є перетворена форма якоїбудь попередньої функціональної форми кругобігу, що
не є первісна форма.

Так $Г'$ в I є перетворена форма $Т'$, кінцеве $П$ в II — перетворена форма $Г$ (і в I і в II цього
перетворення досягається простим актом товарової циркуляції, в наслідок формального переміщення
товару й грошей); $Т'$ в III є перетворена форма $П$, продуктивного капіталу. Але тут, у III,
перетворення стосується, поперше, не лише до функціональної форми капіталу, але також і до величини
його вартости; подруге, перетворення є результат не просто формального переміщення, властивого
процесові циркуляції, але дійсного перетворення, що його проробили в процесі продукції споживна
форма і вартість товарових складових частин продуктивного капіталу.

Форму початкового члена $Г$, $П$, $Т'$ наперед дається для кожного кругобігу — I, II, III; форму, що знову
повторюється в кінцевому члені, дано, а, значить, і зумовлено рядом метаморфоз самого кругобігу. $Т'$,
як кінцевий пункт кругобігу індивідуального промислового капіталу, має собі за передумову лише
неналежну до циркуляції форму $П$ того самого промислового капіталу, що продукт його є $Т'$; $П$, як
кінцевий пункт в I, як перетворена форма $Т'$ ($Т' — Г'$), має собі за передумову, що $Г$ є в руках
покупця, існує поза кругобігом $Г\dots{} Г'$ і лише в наслідок продажу $Т'$ втягується в цей кругобіг і стає
його кінцевою формою. Таким чином, в II кінцеве $П$ має собі за передумову $Р$ і $Зп$ ($Т$), як наявні поза
кругобігом і введені в кругобіг як його кінцева форма через $Г — Т$. Але коли облишити осторонь
останній крайній член, то ні кругобіг індивідуального грошового капіталу не припускає у своєму
кругобігу буття грошового капіталу взагалі, ні кругобіг індивідуального
продуктивного капіталу не припускає в своєму кругобігу буття продуктивного капіталу. В I може $Г$ бути
першим грошовим капіталом, в II може $П$ бути першим продуктивним капіталом, що виступає на кін
історії, але в III \[
Т' \left\{
\begin{array}{c@{~}c}
Т & — \\
— & Г' \\
т & —
\end{array}
\right.
\left\{
\begin{array}{l}
Г — Т — \splitfrac{Р}{Зп}\dots{} П\dots{} Т'\\
~ \\
 г — т
\end{array}
\right.
\] припускається, що $Т$ двічі існує поза кругобігом. Одного разу в кругобігу $Т' — Г' — Т\splitfrac{Р}{Зп}$. Це $Т$,
оскільки воно складається з $Зп$, є товар у руках продавця; воно саме є товаровий капітал, оскільки
воно є продукт капіталістичного продукційного процесу; а коли навіть і ні, то воно з’являється як
товаровий капітал у руках купця. Другого разу, в $т — г — т$ в другому $т$, яке, щоб його можна було
купити, так само мусить бути наявне як товар. У всякому разі $Р$ і $Зп$, хоч вони є товаровий
\index{ii}{0059}  %% посилання на сторінку оригінального видання
капітал, хоч ні, все ж вони такі самі товари, як $Т'$ і відносяться один до одного як товари. Це
саме має силу й щодо другого $т$ в $т — г — т$. Отже, оскільки $Т' \deq{} Т$ ($Р+Зп$), остільки товари є творчі
елементи самого $Т'$ і остільки воно само мусить замінюватися в циркуляції на такі самі товари;
і в $т — г — т$ друге $т$ теж мусить замінюватись у циркуляції так само на інші товари.

Крім того, на основі капіталістичного способу продукції, як панівного, кожен товар у руках продавця
мусить бути товаровим капіталом. Він і далі лишається таким у руках купця або стає таким в його
руках, коли не був ним раніше. Або — як, напр., довізні товари — він мусить бути товаром, що
замістив первісний товаровий капітал і тому надав йому лише іншу форму буття.

Товарові елементи $Р$ і $Зп$, що з них складається продуктивний капітал $П$, як форма буття $П$ мають не той
самий вигляд, що був у них на тих різних товарових ринках, де їх придбали. Тепер їх сполучено, і в
такому своєму сполученні вони можуть функціонувати як продуктивний капітал.

Той факт, що лише в цій III формі, в межах самого кругобігу, $Т$ з’являється як передумова $Т$, походить
із того, що за вихідний пункт є капітал у товаровій формі. Кругобіг починається перетворенням $Т'$
(оскільки вона функціонує як капітальна вартість — хоч збільшена додатковою вартістю, хоч ні) на
товари, що являють елементи його продукції. Але це перетворення охоплює цілий процес циркуляції $Т —
Г — Т$ ($= Р+Зп$) і є результат його. Отже, тут $Т$ стоїть на обох крайніх пунктах, але другий крайній
пункт, що набуває своєї форми через акт $Г — Т$ іззовні,
з товарового ринку, не є останній пункт кругобігу, а лише останній пункт його двох
перших стадій, що охоплюють процес циркуляції. Його результат є $П$, що його функція, процес
продукції, починається після цього. Лише як результат цього процесу, отже, не як результат процесу
циркуляції, $Т$ з’являється як завершення кругобігу і в тій самій формі, як і початковий пункт $Т'$.
Навпаки, в $Г\dots{} Г'$, $П\dots{} П$, кінцеві крайні пункти $Г'$ і $П$ є безпосередні результати процесу
циркуляції. Отже, тут лише наприкінці кругобігу припускається, що в чужих руках перебуває одного
разу $Г'$, другого разу $П$. Оскільки кругобіг відбувається між крайніми пунктами, остільки ні $Г$ в
першому випадку, ані $П$ в другому, — тобто ні буття $Г$ як чужих грошей, ані буття $П$ як чужого
продукційного процесу — не є передумова цих кругобігів. Навпаки, $Т'\dots{} Т'$ припускає $Т$ ($= Р+Зп$) як
чужі товари,
що перебувають у чужих руках, втягуються в кругобіг через увідний процес циркуляції і перетворюються
на продуктивний капітал, а як результат функціонування цього капіталу $Т'$ тепер знову стає кінцевою
формою кругобігу.

Але саме тому, що кругобіг $Т'\dots{} Т'$ припускає в своїх межах інший промисловий капітал у формі
$Т$ ($=Р+Зп$) (а $Зп$ охоплюють різноманітні інші капітали, напр., у даному разі — машини, вугілля, мастиво
тощо), то він сам призводить до того, що його розглядають не лише як \emph{загальну} форму кругобігу, тобто
не лише як таку суспільну форму,
\parbreak{}  %% абзац продовжується на наступній сторінці
