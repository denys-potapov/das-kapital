
\index{ii}{0066}  %% посилання на сторінку оригінального видання
Для індивідуальних капіталів безперервність репродукції іноді більш
або менш порушується. Поперше, в різні періоди маси вартости часто
розподіляються по різних стадіях і функціональних формах нерівними
порціями. Подруге, ці порції, залежно від характеру вироблюваного товару,
а значить, залежно від особливої сфери продукції, куди вкладено капітал,
можуть розподілятись різно. Потретє, безперервність може більше або
менше порушуватись в тих галузях продукції, які залежать від доби
року, — чи то в наслідок природних умов (хліборобство, ловитва оселедців
тощо), чи то в наслідок умовних обставин, як, напр., при так званих сезонових
роботах. Якнайправильніше та якнайодноманітніше перебігає процес
на фабриці й у гірництві. Але ця відмінність галузей продукції не спричинюється
до жодної відмінности в загальних формах процесу кругобігу.

Капітал як вартість, що самозростає, охоплює не лише клясові
відносини, не лише певний характер суспільства, що ґрунтується на наявності
праці як праці найманій. Він є рух, процес кругобігу через різні
стадії, який знову таки містить у собі три різні форми процесу кругобігу.
Тому його можна зрозуміти лише як рух, а не як річ у стані спокою.
Ті, хто розглядають усамостійнення вартости лише як абстракцію, забувають,
що рух промислового капіталу є ця абстракція іn асtu\footnote*{
В дії, в акції. \emph{Ред.}
}. Вартість
перебігає тут різні форми, різні рухи, що в них вона зберігається й разом
з тим виростає, збільшується. Що ми тут маємо діло насамперед з простою
формою руху, то ми не беремо на увагу тих революцій, що їх
може зазнавати капітальна вартість у процесі свого кругобігу; однак,
зрозуміло, що, не зважаючи на всі революції в вартості, капіталістична
продукція існує й може існувати далі лише доти, доки капітальна вартість
буде зростати, тобто доки вона, як усамостійнена вартість, робить
свій кругобіг, отже, доти, доки революції в вартості так або інакше
переборюються й вирівнюються. Рухи капіталу виступають як дії поодинокого
промислового капіталіста в той спосіб, що він функціонує як
покупець товарів і праці, продавець товарів і продуктивний капіталіст,
і таким чином своєю діяльністю упосереднює кругобіг. Коли суспільна
капітальна вартість зазнає революції щодо вартости, то може статись, що
індивідуальний капітал його підпаде їй і загине, бо не матиме змоги пристосуватись
до умов цього руху вартости. Що гостріші й частіші стають
революції щодо вартости, то більше автоматичний, з силою стихійного
природного процесу діющий рух усамостійненої капітальної вартости,
бере гору над передбачливістю й розрахунками поодинокого капіталіста,
то більше перебіг нормальної продукції підпадає під ненормальну спекуляцію,
то більшою стає небезпека для існування поодиноких капіталів.
Отже, ці періодичні революції в вартості потверджують те, що вони,
здавалось би, повинні збити, а саме усамостійнення, що його вартість
як капітал набуває та через свій рух зберігає й зміцнює.

Це чергування метаморфоз капіталу, що процесує, призводить до
того, що зміна в величині вартости капіталу — зміна, яка постає в кругобігу,
\index{ii}{0067}  %% посилання на сторінку оригінального видання
постійно вирівнюється з первісною вартістю. Коли усамостійнення
вартости проти вартостетворчої сили, робочої сили, починається в акті
$Г — Р$ (купівля робочої сили) і здійснюється в процесі продукції як
експлуатація робочої сили, то це усамостійнення вартости не виявляється
знову в тому кругобігу, де гроші, товар, елементи продукції являють
лише почережні форми капітальної вартости, що процесує, і де попередня
величина вартости вирівнюється з теперішньою зміненою величиною
вартости капіталу.

„Вартість, — каже Бейлі, заперечуючи усамостійнення вартости, яке
характеризує капіталістичний спосіб продукції, і яке він трактує як
ілюзію деяких економістів, — вартість є відношення між одночасно наявними
товарами, бо лише такі товари можна обмінювати один на один“\footnote*{
«Value is a relation between contemporary commodities, because such only
admit of being exchanged with each other».
}.
Він це каже, заперечуючи можливість порівняння товарових вартостей в
різні доби, порівняння, яке — в разі грошову вартість фіксовано для
кожної доби — означає лише порівняння витрати праці, потрібної в різні
доби для продукції товарів однакового сорту. Це випливає з його загального
хибного уявлення, що згідно з ним мінова вартість дорівнює
вартості, а форма вартости є сама вартість; отже, товарові вартості не
можуть порівнюватись, коли вони не функціонують активно як мінові вартости,
тобто коли їх не можна realiter\footnote*{
Realiter — в дійсності, реально. \emph{Ред.}
} обміняти одна на одну. Таким
чином, йому й на думку не спадає, що вартість функціонує як капітальна
вартість або капітал лише остільки, оскільки вона в різних фазах свого
кругобігу — а вони зовсім не є contemporary\footnote*{
Cotemporary — одночасно; є contemporary — є одночасні. \emph{Ред.}
}, а постають одна по
одній — лишається ідентична самій собі і сама з собою порівнюється.

Щоб дослідити формулу кругобігу в чистому вигляді, не досить лише
того припущення, що товари продається за їхньою вартістю, але треба
ще припустити й те, що це відбувається за інших незмінних обставин.
Візьмімо, напр., форму $П\dots{} П$ не вважаючи на всякі революції у техніці
продукційного процесу, що можуть зневартнити продуктивний капітал певного
капіталіста, а також не вважаючи на всякий зворотний вплив, що
його може справити зміна вартости елементів продуктивного капіталу на
вартість наявного товарового капіталу, яка може підвищитись або
знизитись, коли є запас такого капіталу. Хай $Т'$, \num{10.000} ф. пряжі,
продається за їхньою вартістю за 500\pound{ ф. стерл.}; \num{8.440} ф. пряжі \deq{}
422\pound{ ф. стерл.} покривають капітальну вартість, що є в $Т'$. Але коли
вартість бавовни, вугілля і т. ін. підвищилась (ми тут не беремо на увагу
звичайного коливання цін), то можливо цих 422\pound{ ф. стерл.} буде не досить,
щоб повнотою покрити елементи продуктивного капіталу; потрібен додатковий
грошовий капітал — грошовий капітал зв’язується. Навпаки, коли
ті ціни падають, грошовий капітал звільняється. Процес перебігає цілком
нормально тільки тоді, коли відношення вартости лишаються сталі; в дійсності
він перебігає нормально доти, доки перешкоди в повторенні кругобігу
\index{ii}{0068}  %% посилання на сторінку оригінального видання
усовуються; що більші ці перешкоди, то більший грошовий капітал
мусить мати промисловий капіталіст, щоб чекати, поки їх усунеться; а що в
розвитку капіталістичної продукції поширюються розміри кожного індивідуального
продукційного процесу, а разом з тим і мінімальна величина
авансовуваного капіталу, то ця обставина прилучається до ряду
інших, які дедалі більше перетворюють функцію промислового капіталіста
на монополію великих грошових капіталістів, поодиноких або
асоційованих.

До речі треба тут позначити, що коли постає зміна в вартості елементів
продукції, то виявляється ріжниця між формою $Г\dots{} Г'$, з одного
боку, і формою $П\dots{}П$ і $Т'\dots{}Т'$, з другого боку.

В $Г\dots{} Г'$, як формулі нововкладуваного капіталу, який спочатку
виступає як грошовий капітал, в разі що знизиться вартість засобів
продукції, напр., сировинного матеріялу, допоміжних матеріялів і т. ін., для
того, щоб одкрити підприємство певних розмірів, треба меншої витрати
грошового капіталу, ніж до цього зниження, бо розмір продукційного
процесу (за незмінного рівня розвитку продуктивної сили) залежить від
маси й розміру засобів продукції, що їх може опанувати дана кількість
робочої сили; але він не залежить ні від вартости цих засобів продукції,
ні від вартости робочої сили (остання впливає лише на величину
зростання вартости). Навпаки. Коли вартість тих елементів продукції товарів,
що є елементи продуктивного капіталу, підвищується, то треба
більше грошового капіталу, щоб закласти підприємство даних розмірів.
В обох випадках зазнає впливу лише величина нововкладуваного грошового
капіталу; в першому випадку постає надмір грошового капіталу, в
другому — зв’язується грошовий капітал, якщо в даній галузі продукції
звичайним порядком відбувається приріст нових індивідуальних промислових
капіталів.

Кругобіги $П\dots{} П$ і $Т'\dots{} Т'$ виявляються сами як $Г\dots{} Г'$ лише
остільки, оскільки рух $П$ і $Т'$ є разом з тим акумуляція, отже, лише
остільки, оскільки додаткове $г$, гроші, перетворюється на грошовий капітал.
Залишаючи це осторонь, зміна вартости елементів продуктивного капіталу
відбивається на них інакше, ніж на $Г\dots{} Г'$; ми тут знову залишаємо
осторонь зворотний вплив, що його справляє така зміна вартости на складові
частини капіталу, які перебувають у процесі продукції. Безпосередньо
зазнає впливу тут не первісна витрата, а промисловий капітал, що перебуває
у процесі своєї репродукції, а не в своєму першому кругобігу;
отже, впливу зазнає $Т'\dots{} Т \splitfrac{Р}{Зп}$, зворотне перетворення товарового
капіталу на елементи його продукції, оскільки ці останні складаються з товарів.
Коли падає вартість (зглядно ціна), то можливі три випадки: процес
репродукції триває далі в тих самих розмірах; тоді звільняється частина
грошового капіталу, що був досі, і нагромаджується грошовий капітал,
хоч не відбувається ані справжньої акумуляції (продукції в поширених
розмірах), ані підготовчого до неї й рівнобіжного з нею перетворення $г$
\parbreak{}  %% абзац продовжується на наступній сторінці
