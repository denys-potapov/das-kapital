\parcont{}  %% абзац починається на попередній сторінці
\index{ii}{0070}  %% посилання на сторінку оригінального видання
$Г — Т\splitfrac{Р}{Зп}$, що з них складаються Зп, засоби продукції, також буде
чужим діющим товаровим капіталом. Отже, з погляду продавця, тут
відбувається $Т' — Г'$, перетворення товарового капіталу на грошовий. Але
це не має абсолютного значення. Навпаки. В процесі своєї циркуляції,
де промисловий капітал функціонує або як гроші, або як товар, кругобіг
промислового капіталу, — хоч він виступає як грошовий капітал, хоч як
товаровий капітал — перехрещується з циркуляцією товарів, найрізнорідніших
способів суспільної продукції, оскільки вони є разом з тим товарова
продукція. Хоч ці товари є продукт такої продукції, яка ґрунтується
на рабстві, хоч продукт селян (китайці, індійські райоти),
громадської продукції (голляндська Ост-Індія), державної продукції (як
основана на кріпацтві продукція, що трапляється в раніші періоди російської
історії), хоч продукції напівдиких мисливських народів і~\abbr{т. ін.}
— все одно: як товари й гроші протистоять вони грошам і товарам,
що в них виявляється промисловий капітал і входять у кругобіг його так само,
як і в кругобіг додаткової варюсти, що міститься в товаровому капіталі,
оскільки її витрачається як дохід, — отже, входять в обидві галузі
циркуляції товарового капіталу. Характер продукційного процесу, що
з нього вони походять, не має значення; як товари вони функціонують
на ринку, як товари входять вони в кругобіг промислового капіталу, так
само, як і в циркуляцію додаткової вартости, що є в ньому. Отже, всебічний
характер їхнього походження, наявність ринку як світового ринку
— ось що є відзначна риса процесу циркуляції промислового капіталу.
Те, що має силу для чужих товарів, має силу також і для чужих грошей: так
само, як товаровий капітал протистоїть грошам лише як товар, так і ці
гроші протистоять йому лише як гроші; гроші функціонують тут як
світові гроші.

Але тут треба зазначити дві обставини.

Поперше. Товари ($Зп$), скоро закінчився акт $Г — Зп$, перестають бути
товаром і стають одним з способів буття промислового капіталу в його
функціональній формі $П$, продуктивним капіталом. Але разом з тим
зникають сліди їхнього походження; вони існують далі тільки як форми
існування промислового капіталу, вони є в його складі. Однак, щоб
їх замістити, потрібна репродукція їх; в цьому розумінні капіталістичний
спосіб продукції зумовлено способами продукції, що перебувають на
іншій, ніж він, стадії розвитку. Але його тенденція в тому, щоб по змозі
всяку продукцію перетворити на товарову продукцію; його головний засіб
для цього є саме оце втягнення цих способів продукції в його процес
циркуляції; а розвинена товарова продукція вже сама є капіталістична
продукція. Втручання промислового капіталу всюди прискорює це перетворення,
а разом з тим і перетворення всіх безпосередніх продуцентів
на найманих робітників.

Подруге. Товари, що входять у процес циркуляції промислового
капіталу (сюди належать і доконечні засоби існування, що на них перетворюється
\index{ii}{0071}  %% посилання на сторінку оригінального видання
задля репродукції робочої сили змінний капітал, після того як
його виплачено робітникам), хоч яке буде їхнє походження, суспільна
форма продукційного процесу, що з нього вони походять, — протистоять
вже самому промисловому капіталові в формі товарового капіталу, в формі
товарово-торговельного або купецького капіталу, а цей останній з самої
природи своєї охоплює товари всяких способів продукції.

Так само, як капіталістичний спосіб продукції припускає широкі розміри
продукції, так само неминуче припускає він широкі розміри продажу;
отже, продаж купцеві, а не поодинокому споживачеві. Оскільки цей
споживач сам є продуктивний споживач, тобто промисловий капіталіст,
оскільки, отже, промисловий капітал однієї галузі дає засоби продукції
для другої галузі, остільки (в формі замовлення тощо) відбувається
також безпосередній продаж товарів одного промислового капіталіста
багатьом іншим. В цьому розумінні кожен промисловий капіталіст є безпосередній
продавець, сам для себе купець, що ним він є між іншим і
продаючи товар купцеві.

Товарова торговля як функція купецького капіталу припускається
капіталістичною продукцією і чимраз більше розвивається з розвитком
цієї продукції. Отже, принагідно для ілюстрації окремих боків капіталістичного
процесу циркуляції, ми припускаємо наявність товарової торговлі,
а в загальній аналізі капіталістичного процесу циркуляції ми
припускаємо безпосередній продаж без втручання купця, бо це останнє
затемнює різні моменти руху.

Звернімось до Сісмонді, який дещо наївно освітлює цю справу:

„Торговля застосовує чималий капітал, що на перший погляд,
здається, не становить жодної частини того капіталу, що його перебіг
ми розглянули в подробицях. Вартість сукна, нагромадженого в крамницях
торговця сукном, як здається спочатку, є щось цілком відмінне
від тієї частини річної продукції, що її багатий дає бідному як заробітну
плату, щоб примусити його робити. Однак, цей капітал лише заміщує
той, що про нього ми казали. Щоб добре уявити собі розвиток багатства,
ми взяли його в момент його утворення і простежили аж до його
споживання. При цьому, капітал, вкладений, напр., у суконну мануфактуру,
здавався нам завжди тим самим; обмінений на дохід споживача, він
лише поділився на дві частини: одна у формі зиску була доходом фабриканта,
друга у формі заробітної плати — доходом робітників на час,
протягом якого вони виробляли нове сукно.

„Але скоро виявилось, що для всіх було б краще, коли б різні частини
цього капіталу заміщували одна одну, і коли б, якщо сотні тисяч екю
досить для всієї циркуляції між фабрикантом і споживачем, ці сто тисяч
екю рівномірно розподілились між фабрикантом, гуртовим торговцем і
роздрібним торговцем. Перший, маючи лише третину цієї суми, виробляв
стільки, скільки раніше, мавши цілу суму, бо в момент закінчення
своєї продукції, він находив торговця-покупця куди раніше, ніж
він знайшов би споживача. Так само і капітал гуртового торговця
куди швидше заміщено капіталом роздрібного торговця\dots{} Різність
\parbreak{}  %% абзац продовжується на наступній сторінці
