
\index{ii}{0074}  %% посилання на сторінку оригінального видання

Поперше, як ми бачили, $Г — Т$ ($Зп$) може являти переплітання метаморфоз
різних індивідуальних капіталів. Напр., пряжа, товаровий капітал
бавовнопрядника, заміщується почасти вугіллям. Частина його капіталу
перебуває в грошовій формі і з неї перетворюється на товарову форму,
тимчасом як капітал капіталістичного вуглепромисловця перебуває в товаровій
формі й тому перетворюється з неї на грошову форму; той самий акт
циркуляції являє собою тут протилежні метаморфози двох (належних до
різних галузей продукції) промислових капіталів, отже, переплітання ряду
метаморфоз цих капіталів. Однак, як ми бачили, $Зп$, що на нього перетворюється
$Г$, не повинне бути неодмінно товаровим капіталом в категоричному
значенні слова, тобто функціональною формою промислового
капіталу, товаровим капіталом, що його випродукував капіталіст. Воно
завжди є на одному боці $Г — Т$, а на другому $Т — Г$, але не завжди
воно є переплітання метаморфоз капіталу. Далі, $Г — Р$, купівля робочої
сили, ніколи не є переплітання метаморфоз капіталів, бо хоч
робоча сила й є товар робітника, але вона стає капіталом лише
бувши продана капіталістові. З другого боку, $Г'$ в процесі $Т' — Г'$
не повинне бути неодмінно перетвореним товаровим капіталом; воно
може бути перетворенням на гроші товару робочої сили (заробітна
плата), або перетворенням на гроші продукту, спродукованого самостійним
робітником, рабом, кріпаком, громадою.

Але, подруге, для функціонально визначеної ролі, що її відіграє
кожна метаморфоза, яка відбувається в межах процесу циркуляції індивідуального
капіталу, зовсім не треба, щоб ця метаморфоза являла відповідну
протилежну метаморфозу в кругобігу іншого капіталу, — звичайно,
при тому припущенні, що всю продукцію світового ринку провадиться
капіталістично. Наприклад, в кругобігу $П\dots{} П$, $Г'$, яке перетворює $Т'$
на гроші, може бути на боці покупця лише його додатковою вартістю,
перетвореною на гроші (коли товар є предмет споживання); або в
акті $Г' — Т'\splitfrac{Р}{Зп}$ (куди капітал, отже, входить уже акумульований) $Г'$ для
продавця $Зп$ може ввійти в циркуляцію його капіталу лише як покриття
авансованого від нього капіталу або навіть і зовсім знову не входити в
циркуляцію його капіталу, — коли воно відгалужується як витрачання доходу.

Отже, питання про те, як різні складові частини сукупного суспільного
капіталу, — що в ньому поодинокі капітали є лише самостійно
діюіщі складові частини, — навзаєм заміщуються в процесі циркуляції, це
питання — і щодо капіталу, і щодо додаткової, вартости, — не розв’язується
дослідом простих переплітань метаморфоз товарової циркуляції, що в
ній акти циркуляції капіталу такі самі, як і при всякій іншій товаровій
циркуляції, а тут треба досліджувати іншим способом. Досі при цьому
задовольнялись фразами, які, коли їх ближче аналізувати, не мають нічого,
крім невиразних уявлень, що запозичені виключно з дослідження таких
переплітань метаморфоз, які властиві кожній товаровій циркуляції.

\pfbreak
