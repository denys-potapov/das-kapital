
\index{ii}{0075}  %% посилання на сторінку оригінального видання
Одна з найвиразніших особливостей процесу кругобігу промислового
капіталу, а значить, і капіталістичної продукції, є та обставина, що, з
одного боку, елементи утворення продуктивного капіталу мусять походити
з товарового ринку й завжди поновлюватися з нього, завжди купуватись
як товари; з другого боку, що продукт процесу праці виходить з
цього процесу як товар, і завжди знову мусить продаватись як товар.
Порівняймо, напр., сучасного фармера Нижньої Шотляндії з континентальним
дрібним селянином старого часу. Перший продає ввесь свій
продукт і повинен тому замістити на ринку всі його елементи, навіть
зерно для посіву; другий безпосередньо споживає більшу частину свого
продукту, яко мога менше купує і продає і по змозі сам виробляє знаряддя,
одяг тощо.

На цій підставі протиставлялось натуральне господарство, грошове
господарство і кредитове господарство одне одному, як три характеристичні
економічні форми руху суспільної продукції.

Поперше, ці три форми не являють рівнозначних фаз розвитку. Так
зване кредитове господарство саме є лише форма грошового господарства,
оскільки обидва позначення виражають функції обміну або способи
обміну між самими продуцентами. В розвиненій капіталістичній продукції
грошове господарство являє лише основу кредитового господарства.
Таким чином, грошове господарство і кредитове господарство відповідають
лише різним ступеням розвитку капіталістичної продукції, але
вони зовсім не є різні самостійні форми обміну протилежно натуральному
господарству. З таким самим правом можна було б цим двом формам
протиставити, як рівноцінні, різні форми натурального господарства.

Подруге, що в категоріях: грошове господарство, кредитове господарство,
підкреслюється й випинається як відзначну ознаку не господарство,
тобто не самий продукційний процес, а відповідний господарству спосіб
обміну між різними аґентами продукції або продуцентами, то те саме
треба було б зробити й розглядаючи першу категорію. Отже, замість
натурального господарства ми мали б мінове господарство. Цілком замкнене
натуральне господарство, напр., Перуанська держава інків, не підпала
б під жодну з цих категорій.

Потретє: грошове господарство властиве всякій товаровій продукції, і
продукт фігурує як товар в найрізноманітніших суспільних продукційних
організмах. Отже, за характеристичну для капіталістичної продукції рису
були б лише розміри, що в них продукт продукується як предмет торговлі,
як товар, отже, і розміри, що в них елементи його власного утворення
мусять теж як предмети торговлі, як товари ввійти в те господарство,
відки походить продукт.

В дійсності капіталістична продукція, як загальна форма продукції,
є товарова продукція, але такою вона є і дедалі більше такою стає в
своєму розвитку лише тому, що тут сама праця з’являється як товар, бо
робітник продає працю, тобто функцію своєї робочої сили, і до того, як
ми це припускаємо, продає за її вартістю, визначуваною витратами її
репродукції. Продуцент стає промисловим капіталістом в міру того, як
\parbreak{}  %% абзац продовжується на наступній сторінці
