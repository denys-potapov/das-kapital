\parcont{}  %% абзац починається на попередній сторінці
\index{ii}{0076}  %% посилання на сторінку оригінального видання
праця стає найманою працею; тому капіталістична продукція (а значить,
і товарова продукція) з’являється в цілому своєму об’ємі лише тоді,
коли й безпосередній сільський продуцент є найманий робітник. В відношенні
між капіталістом і найманим робітником грошове відношення, відношення покупця і продавця, стає
відношенням іманентним самій продукції. Але це відношення в основі своїй ґрунтується на суспільному
характері продукції, а не способу обміну; цей останній, навпаки, випливає з першого.
А проте, буржуазному світоглядові, де всю увагу звертається на практичні
операції, саме й відповідає погляд, що не в характері способу продукції
треба вбачати основу відповідного йому способу обміну, а навпаки\footnote{
До цього місця, рукопис V, — Все, що далі до кінця розділу, це замітка, яка є в зшитку з 1877 або
1878 року серед витягів з книжок.
}.

\pfbreak{}

Капіталіст кидає в циркуляцію менше вартости в грошовій формі,
ніж бере з неї, бо він кидає в неї більше вартости в товаровій формі,
ніж узяв звідти в товаровій формі. Оскільки він функціонує лише як персоніфікація
капіталу, як промисловий капіталіст, остільки його подання
товарових вартостей завжди більше, ніж його попит на товарові вартості.
Коли б його подання й попит взаємно покривались, то з цього погляду
це значило б, що його капітал не зростає вартістю; капітал не функціонував
би як продуктивний капітал; продуктивний капітал перетворився б
на товаровий капітал, не запліднений додатковою вартістю; підчас продукційного
процесу він не видобував би з робочої сили жодної додаткової
вартости в товаровій формі, отже, зовсім не функціонував би як
капітал; капіталіст дійсно мусить „продавати дорожче, ніж купив“, але
це вдається йому лише тому, що він за допомогою капіталістичного
продукційкого процесу перетворив куплений ним дешевший товар, — бо він
є товар меншої вартости, — на товар більшої вартости, тобто на дорожчий.
Він продає дорожче не тому, що продає свій товар вище понад його
вартість, а тому, що продає товар, який має вартість вищу, ніж сума
вартостей складових елементів його продукції.

Норма, що за нею капіталіст збільшує вартість свого капіталу, то більша,
що більша різність між його поданням і попитом, тобто що більший
надлишок тієї товарової вартости, яку він подає, проти тієї товарової
вартости, на яку він ставить попит. Його мета не та, щоб попит і
подання навзаєм покривались, а щоб вони якомога більше не покривались,
щоб його подання перекривало його попит.

Те, що має силу для поодинокого капіталіста, має силу й для кляси
капіталістів.

Оскільки капіталіст є лише персоніфікація промислового капіталу,
остільки його власний попит є лише попит на засоби продукції та
робочу силу. Його попит на Зп, розглядуваний щодо вартости,
менший, ніж його авансований капітал; він купує засоби продукції
за меншу вартість, ніж вартість його капіталу, а тому й за
\parbreak{}  %% абзац продовжується на наступній сторінці
