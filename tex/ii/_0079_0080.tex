\parcont{}  %% абзац починається на попередній сторінці
\index{ii}{0079}  %% посилання на сторінку оригінального видання
припущенням, що за спонукальний мотив є особисте споживання, а не
саме збагачення, знищується саму основу капіталізму.

Але воно неможливе також і технічно. Капіталіст не лише мусить
утворити запасний капітал, щоб забезпечити себе проти коливань цін і
мати змогу чекати на сприятливу коньюнктуру для купівлі і продажу;
він мусить акумулювати капітал, щоб поширювати продукцію і вводити
технічні поліпшення в свій продуктивний організм.

Щоб акумулювати капітал, він мусить насамперед деяку частину додаткової
вартости, що до нього допливає з циркуляції, вилучати з циркуляції
в грошовій формі і збільшувати її як скарб доти, доки вона
дійде розмірів, потрібних для того, щоб поширити старе підприємство або
відкрити нове поряд старого. Поки триває скарботворення, попит капіталіста
не збільшується; гроші імобілізовано; вони не вилучають з товарового
ринку жодного товарового еквівалента за грошовий еквівалент, вилучений
з ринку за поданий товар.

Кредит ми лишаємо тут осторонь, а до кредитових відносин належить,
напр., те, що капіталіст, у міру нагромадження грошей, кладе їх у банк
на біжучий рахунок за проценти.
\label{original-79-1}

\section[Час обігу]{Час обігу\footnotemark{}}
\footnotetext{Відси рукопис IV.}

\label{original-79-2}
\noindent{}Рух капіталу через сферу продукції та дві фази сфери циркуляції
відбувається, як ми бачили, послідовно в часі. Протяг його перебування в
сфері продукції становить час його продукції, протяг перебування в
сфері циркуляції — час його циркуляції або час його обігу. Ввесь час, що
протягом його капітал робить свій кругобіг, дорівнює сумі часу продукції
та часу обігу.

Час продукції природно охоплює період процесу праці, але цей
останній не охоплює цілого часу продукції. Насамперед пригадаймо, що
одна частина сталого капіталу існує в засобах праці, як от машини,
будівлі тощо, які до останнього дня свого існування придаються в тих
самих, знову й знову повторюваних, процесах праці. Періодична перерва
процесу праці, напр., вночі, хоч і є перерва у функціонуванні цих засобів
праці, але не перерва у перебуванні їх на місці продукції. Вони
належать продукції не тільки, поки функціонують, а й тоді, коли не функціонують.
З другого боку, капіталіст мусить мати напоготові певний запас
сировинного матеріялу та допоміжних матеріялів, щоб процес продукції
протягом більш або менш довгого часу відбувався в заздалегідь визначених
розмірах, незалежно від випадковостей щоденного подання товарів
на ринку. Цей запас сировинного матеріялу тощо споживається продуктивно
лише поступінно. Відси постає ріжниця між його часом продукції\footnote{
Час продукції тут взято в активному значенні: час продукції засобів продукції
є тут не час, що протягом його їх продукується, а час, що протягом його
вони беруть участь у процесі продукції товарового продукту.
}
\index{ii}{0080}  %% посилання на сторінку оригінального видання
та його часом функціонування. Отже, час продукції засобів
продукції взагалі охоплює 1) час, що протягом його вони функціонують
як засоби продукції, тобто придаються в процесі продукції;
2) павзи, що протягом їх продукційний процес, а значить, функціонування
належних йому засобів продукції, переривається; 3) час, що
протягом його вони, хоч і є напоготові як умови процесу, і тому являють
уже продуктивний капітал, але ще не ввійшли в процес продукції.

Розглядувана досі ріжниця завжди є ріжниця між часом перебування
продуктивного капіталу в сфері продукції і часом перебування його в
процесі продукції. Але сам процес продукції може зумовлювати перерви
в процесі праці, а тому й у часі праці, зумовлювати переміжки, коли
предмет праці зазнає впливу фізичних процесів без дальшого прикладання
людської праці. Продукційний процес, а тому й функціонування засобів
продукції в цьому разі триває далі, хоч процес праці, а значить, і функціонування
засобів продукції як засобів праці, перервано. Так буває, напр.,
з зерном, що його висіяно, з вином, що ферментує в льоху, з матеріялом
праці в багатьох мануфактурах, як, напр., на чинбарнях, де цей матеріял
зазнає впливу хемічних процесів. Час продукції тут більший, ніж час
праці. Ріжниця між ними — це лишок часу продукції над часом праці.
Цей надлишок ґрунтується завжди на тому, що продуктивний капітал
перебуває в \emph{лятентному стані} в сфері продукції, не функціонуючи в
самому процесі продукції, або на тому, що він функціонує в процесі
продукції, але не перебуває в процесі праці.

Та частина лятентного продуктивного капіталу, що її наготовлено
лише як умову для продукційного процесу, напр., бавовна, вугілля й~\abbr{т. ін.}, в прядільні, не функціонує ні як продуктотворча, ні як вартостетворча.
Це — бездіяльний капітал, хоч така його бездіяльність становить
умову для безперервного перебігу продукційного процесу. Будівлі, апарати
тощо, потрібні як сховища продуктивного запасу (лятентного капіталу),
є умови продукційного процесу, а тому становлять складові частини авансованого
продуктивного капіталу. Вони виконують свою функцію, як сховища
продуктивних складових частин в попередній стадії. Оскільки на цій стадії
потрібні процеси праці, остільки вони удорожнюють сировинний матеріял
тощо, але вони є продуктивна праця і утворюють додаткову
вартість, бо частину цієї праці, як і всякої іншої найманої праці, не
оплачується. Нормальні перерви цілого продукційного процесу, тобто
переміжки, що в них продуктивний капітал не функціонує, не продукують
ні вартости, ні додаткової вартости. Відси постає намагання примусити
працювати навіть уночі (книга І, розд, VIII, 4). — Переміжки в часі праці,
що їх мусить перейти предмет праці підчас самого продукційного процесу,
не утворюють ні вартости, ні додаткової вартости; але вони розвивають
продукт, становлять частину його життя, процес, що його він
\parbreak{}  %% абзац продовжується на наступній сторінці
