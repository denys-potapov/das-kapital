\parcont{}  %% абзац починається на попередній сторінці
\index{ii}{0081}  %% посилання на сторінку оригінального видання
мусить проробити. Вартість апаратів і~\abbr{т. ін.} переноситься на продукт
відповідно всьому тому часові, що протягом його вони функціонують;
продукт вводиться в цю стадію самою працею, і вживання цих апаратів
є так само умова продукції, як і розпорошування частини
бавовни, яка не входить у продукт, але все ж переносить на нього
свою вартість. Друга частина лятентного капіталу, напр., будівлі, машини
і~\abbr{т. ін.}, тобто засоби праці, що їхню функцію переривається лише
нормальними павзами в продукційному процесі — ненормальні
перерви в наслідок скорочення продукції, криз тощо є чисті
втрати — ця друга частина лятентного капіталу додає до продукту
вартість, хоч не бере участи в утворенні продукту; сукупна вартість, що
її ця частина додає до продукту, визначається середнім часом її тривання;
вона втрачає свою вартість, бо втрачає свою споживну вартість і в той
час, коли вона функціонує, і в той час, коли вона не функціонує.

Нарешті, вартість сталої частини капіталу, яка й далі перебуває в
продукційному процесі, не зважаючи на перерву в процесі праці, знову
з’являється в наслідок продукційного процесу. Самою працею засоби
продукції поставлено тут у такі умови, що в них вони сами собою пророблюють
певні природні процеси, що в наслідок їх постає певний
корисний результат або зміна форми їхньої споживної вартости. Праця
завжди переносить вартість засобів продукції на продукт, оскільки вона
споживає їх справді доцільно, як засоби продукції. Справа ані трохи не
змінюється від того, чи мусить праця, щоб досяглось такого ефекту,
безперервно діяти на предмет праці за допомогою засобів праці, чи вона
мусить лише дати поштовх, поставивши засоби продукції в такі умови,
що в них вони без дальшого впливу праці сами з себе зазнали б
передбаченої зміни в наслідок природних процесів.

Хоч на чому ґрунтується такий надлишок часу продукції над часом
праці, — чи на тому, що засоби продукції становлять тільки лятентний
продуктивний капітал, тобто перебувають на попередньому щаблі до
справжнього процесу продукції, чи на тому, що підчас процесу продукції
їхню власну функцію переривають павзи в продукційному процесі, чи,
нарешті, на тому, що самий процес продукції зумовлює перерви в процесі
праці — в жодному з цих випадків засоби продукції не функціонують як
вбирачі праці. Коли вони не вбирають праці, то не вбирають і додаткової
праці. Тому тут не відбувається ніякого зростання вартости продуктивного
капіталу, поки він перебуває в тій частині часу своєї продукції, що є надлишок
над часом праці, хоч як би неподільно сполучалось здійснення процесу
зростання вартости з цими павзами. Очевидно, що як більше збігаються один
з одним час продукції та час праці, то більша продуктивність і зростання
вартости даного продуктивного капіталу протягом даного часу. Відси
випливає тенденція капіталістичної продукції по змозі зменшити надлишок
часу продукції над часом праці. Але хоч час продукції капіталу й може
відхилитись від його часу праці, а проте, він завжди охоплює цей час,
і надлишок першого над другим є навіть умова продукційного процесу.
Отже, час продукції є завжди той час, що протягом його капітал продукує
\index{ii}{0082}  %% посилання на сторінку оригінального видання
споживні вартості і сам зростає вартістю, а тому функціонує як
продуктивний капітал, хоч цей час продукції має в собі й той час, коли
капітал або перебуває в лятентному стані, або продукує продукти, не
зростаючи своєю вартістю.

В сфері циркуляції капітал перебуває як товаровий капітал і грошовий
капітал. Обидва його процеси циркуляції в тому, що він перетворюється
з товарової форми на грошову та з грошової форми на товарову. Та
обставина, що перетворення товару на гроші є тут разом з тим реалізація
додаткової вартости, вміщеної в товарі, і що перетворення грошей
на товар є разом з тим перетворення або зворотне перетворення капітальної
вартости на форму елементів її продукції, нічого не змінює в
тому, що ці процеси, як процеси циркуляції, є процеси простої метаморфози
товарів.

Час обігу і час продукції навзаєм виключають один одного. Протягом
часу свого обігу капітал не функціонує як продуктивний капітал і тому
не продукує ні товару, ні додаткової вартости. Якщо ми розглядаємо
кругобіг у найпростішій формі, коли вся капітальна вартість кожного разу
одним заходом переходить з однієї фази в іншу, то очевидно, що процес
продукції, а, значить, і самозростання капіталу переривається доти, доки
триває час його обігу, і що залежно від протягу останнього процес продукції
буде відновлюватися швидше або повільніше. Навпаки, коли різні
частини капіталу пророблюють кругобіг одна по одній, так що кругобіг
цілої капітальної вартости здійснюється послідовно в кругобігу її
різних частин, то очевидно, що як довше аліквотні частини капітальної
вартости постійно перебувають у сфері циркуляції, то менша
мусить бути та її частина, яка завжди функціонує в сфері
продукції. Тому збільшення або скорочення часу обігу впливає тут як
негативна межа для скорочення або збільшення часу продукції або тих
розмірів, що в них капітал даної величини функціонує як продуктивний
капітал. Що більше метаморфози циркуляції капіталу є лише ідеальні,
тобто що більше час обігу дорівнює нулеві або до нього наближається,
то більше функціонує капітал, то вища стає його продуктивність і самозростання
його вартости. Коли, напр., капіталіст працює на замовлення, так
що плату він одержує, здаючи продукт, при чому виплату цю робиться
засобами його власної продукції, — то час циркуляції наближається до
нуля.

Отже, час обігу капіталу взагалі обмежує час продукції його, а тому
й процес зростання його вартости. І обмежує його саме пропорційно
до свого протягу. Цей протяг може більшати або меншати дуже неоднаково,
а тому й у дуже неоднаковій мірі обмежувати час
продукції капіталу. Але політична економія бачить тут лише \emph{зовнішній}
вигляд явища, а саме вплив часу обігу на процес зростання капітальної
вартости взагалі, Вона вважає цей неґативний вплив за позитивний, бо
наслідки його позитивні, і то більше тримається цієї позірности, що ця позірність
дає ніби доказ того, що капітал має в собі, незалежне від його
процесу продукції, а, значить, і від експлуатації праці, містичне джерело
\parbreak{}  %% абзац продовжується на наступній сторінці
