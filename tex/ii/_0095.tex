
\index{ii}{0095}  %% посилання на сторінку оригінального видання
Запас у формі продуктивного капіталу існує в формі засобів продукції,
що перебувають уже в процесі продукції або принаймні в руках
продуцента, тобто латентно вже в продукційному процесі. Ми бачили
раніше, що з розвитком продуктивности праці, а, значить, і з розвитком
капіталістичного способу продукції — а він розвиває продуктивну
силу суспільної праці більше, ніж усі попередні способи продукції —
постійно зростає маса засобів продукції, що раз назавжди введені
у формі засобів праці в процес продукції і знову та знову
функціонують у ньому протягом більш або менш довгого часу (будівлі,
машини тощо), і що зростання цих засобів продукції є разом і передумова
й наслідок розвитку суспільної продуктивної сили праці. І що не
лише абсолютний, а ле й відносний зріст багатства в цій формі (порівн.
кн. І, розд. XXIII, 2) характеризує насамперед капіталістичний спосіб
продукції. Речові форми існування сталого капіталу, засоби продукції,
складаються не лише з такого роду засобів праці, а також і з матеріялу
праці на різноманітних щаблях його оброблення та з допоміжних матеріялів.
Разом з маштабом продукції та підвищенням продуктивної сили
праці в наслідок кооперації, поділу праці, заведення машин тощо, зростає
й маса сировинного матеріялу, допоміжних матеріялів і т. ін., що
входять у щоденний процес репродукції. Ці елементи мусять бути напоготові
на місці продукції. Отже, розміри цього запасу, що існує в формі продуктивного
капіталу, абсолютно зростають. Щоб й процес відбувався плавко —
зовсім незалежно від того, чи можна цей запас поновлювати щоденно чи
лише в певні строки, — для цього треба, щоб на місці продукції завжди
був напоготові більший запас сировинного матеріялу і т. ін., ніж його
споживається, напр., щоденно або щотижнево. Безперервність процесу
потребує, щоб наявність умов для нього не залежала ні від можливих
перерв у щоденних закупах, ні від того, що товаровий продукт продається
щоденно або щотижнево і тому лише нерегулярно може перетворюватись
знову на елементи його продукції. А, проте, очевидно, що
продуктивний капітал у дуже різних розмірах може бути в латентному
стані або становити запас. Напр., велика ріжниця, чи мусить прядник
наготувати запас бавовни або вугілля на три місяці, чи на один. Як
бачимо, цей запас може відносно меншати, хоч абсолютно він більшає.

Це залежить від різних умов, але всі вони посутньо сходять на ту
більшу швидкість, регулярність, певність, що з ними завжди можна подати
потрібну масу сировинного матеріялу так, щоб ніколи не було перерви
в продукційному процесі. Що менше виконано ці умови, що менша
певність, регулярність і швидкість подачі, то більша мусить бути у продуцента
лятентиа частина продуктивного капіталу, тобто запас сировинного
матеріялу тощо, який ще чекає на своє перероблення. Ці умови,

1864--1866 р. нечуваний вивіз рижу в Австралію, на Мадагаскар і т. ін. Відси
гострий характер голоду 1866 р „який призвів до того, що в самому лише
яистрикті Орісса вмерло мільйон чоловіка (І. с. 174, 175, 213, 214 і III: Papers relating
to the Famine in Behar; ct. 32, 33, де поміж причин голоду зазначається „вичерпання старих запасів“).
(З рукопису II).
\parbreak{}  %% абзац продовжується на наступній сторінці
