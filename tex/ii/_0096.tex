\parcont{}  %% абзац починається на попередній сторінці
\index{ii}{0096}  %% посилання на сторінку оригінального видання
отже, і розміри запасу в цій формі стоять у зворотному відношенні до
рівня розвитку капіталістичної продукції, а тому і продуктивної сили
суспільної праці.

А, проте, те, що тут видається зменшенням запасу (прим., для Ляльора),
почасти є лише зменшення запасу в формі товарового капіталу або
власно товарового запасу. Отже, це є лише зміна форми того таки запасу.
Коли, напр „кількість вугілля, щоденно продукованого в даній країні,
велика, а, значить, і розміри та енергія вугільної продукції великі, то
прядникові не треба великих складів вугілля, щоб забезпечити безперервність
своєї продукції. Це зайва річ, бо є постійний і правильно
відновлюваний довіз вугілля. ГІодруге, швидкість, що з нею продукт
одного процесу може перейти, як засіб продукції, в другий, залежить
від розвитку засобів транспорту й комунікації. Дешевина транспорту
відіграє при цьому велику ролю. Постійно відновлюваний транспорт,
напр., вугілля з копалень до прядці коштував би дорожче, ніж забезпечення
більшою кількістю вугілля на довший час при порівняно дешевшому
транспорті. Ці дві обставини, розглянуті досі, походять з самого процесу
продукції. Потретє, впливає тут і розвиток кредитової системи. Що менше
залежить прядник при поновленні своїх запасів бавовни, вугілля тощо від
безпосереднього продажу своєї пряжі, — а що розвиненіша кредитова система,
то менша ця безпосередня залежність, — то менша може бути відносна
величина цих запасів, щоб забезпечити безупинну продукцію пряжі
в даному розмірі незалежно від випадковостей продажу пряжі. Але,
почетверте, багато сировинних матеріялів, напівфабрикатів і~\abbr{т. ін.} потребують
для своєї продукції довшого часу, і це особливо має силу для
всіх сировинних матеріялів, що їх постачає хліборобство. Отже, щоб не
було жодної перерви в продукційному процесі, мусить бути наявний
певний запас їх на ввесь той час, поки новий продукт не може заступити
старого. Коли цей запас меншає в руках промислового капіталіста,
то це лише значить, що він збільшується в руках купця в
формі товарового запасу. Розвиток транспортових засобів, дозволяє, наприклад,
швидко перевозити з Ліверпулу до Менчестеру бавовну, що
лежить в імпортовій гавані, так що фабрикант може, по потребі, поновлювати
відносно невеликими порціями свій запас бавовни. Але в такому
разі то більша маса цієї самої бавовни буде лежати як товаровий
запас у руках ліверпулських торговців. Отже, це лише зміна форми запасу,
а цього й недобачають Ляльор та інші. А коли розглядати суспільний
капітал, то тут, як і раніш, та сама маса продукту існує в формі
запасу. З розвитком транспортових засобів для окремої країни меншають
розміри тієї потрібної маси, що її треба наготовити, напр., на рік.
Коли між Америкою та Англією курсує багато пароплавів і вітрильних
суден, то в Англії збільшується число випадків поновлення зипасу бавовни
і, отже, меншає пересічна кількість того запасу бавовни, що мусити
бути в Англії. Так само впливає розвиток світового ринку, а
тому і збільшення кількости джерел, відки добувають той самий
сіродукт. Продукт довозиться частинами з разних країн і в різні терміни
\parbreak{}  %% абзац продовжується на наступній сторінці
