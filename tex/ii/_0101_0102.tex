\parcont{}  %% абзац починається на попередній сторінці
\index{ii}{0101}  %% посилання на сторінку оригінального видання
але його розміри більшають. Отже, об’єм товарового запасу, що
бубнявіє в наслідок застою циркуляції, можна помилково вважати за
ознаку поширення процесу репродукції, і це особливо тоді, коли з розвитком
кредитової системи справжній рух може містифікуватись.

Витрати на утворення запасу складаються: 1) з кількісного зменшення
маси продукту (напр., запасу борошна), 2) з якісного псування, 3) із зречевленої
або живої праці, що потрібна для зберігання запасу.

\subsection{Витрати на транспорт}

Тут нам не треба вдаватись у всі подробиці щодо витрат циркуляції,
як, прим., пакування, сортування тощо. Загальний закон той, що \emph{всі
витрати циркуляції, які випливають лише з перетворення
форми товару, не додають до нього жодної
вартости}. Це — лише витрати на реалізацію вартости або на перетворення
її з однієї форми на іншу. Капітал, витрачений на ці видатки
(разом із працею, що є під його орудою), належить до faux frais капіталістичної
продукції. Покриття цих витрат мусить відбуватись з додаткового
продукту і становить, розглядаючи цілу капіталістичну клясу, одбаву
з додаткової вартости або додаткового продукту цілком так само,
як час, потрібний робітникові на закуп засобів його існування, є для нього
змарнований час. Але витрати на транспорт відіграють дуже важливу
ролю, й тому на них треба тут трохи зупинитись.

У межах кругобігу капіталу й метаморфози товарів, що становить
відділ цього кругобігу, відбувається обмін речовин суспільної праці. Цей
обмін речовин може зумовлювати переміщення продуктів, їхній справжній
рух з одного місця на інше. Але циркуляція товарів може відбуватись і
без їхнього фізичного руху, а транспорт продуктів — без товарової циркуляції,
ба навіть без безпосереднього обміну продуктів. Будинок, що
його А продає В, циркулює як товар, але лишається на тому самому
місці. Рухомі товарові вартості, прим., бавовна або чавун, лишаються
на тому самому товаровому складі в той самий час, як вони перебігають
десятки різних процесів циркуляції, купуються спекулянтами й знову
продаються\footnote{
Шторх зве цю циркуляцію factice.
}. Справді тут рухається лише титул власности на річ, а не
сама річ. З другого боку, напр., у царстві інків, транспортова промисловість
відігравала велику ролю, хоч суспільний продукт не циркулював як
товар і не розподілялось його за допомогою мінової торговлі.

Тому, хоч транспортова промисловість на основі капіталістичної продукції
видається причиною витрат циркуляції, однак ця особлива форма
виявлення їх зовсім не змінює справи.

Маса продуктів не більшає в наслідок перевозу їх. Всі зміни, спричинені
перевозом у природних властивостях продуктів, за деякими винятками,
є не навмисний корисний ефект, а неминуче лихо. Але споживна вартість
речей реалізується лише в їх споживанні, а споживання їх може потребувати
\index{ii}{0102}  %% посилання на сторінку оригінального видання
їх переміщення, тобто додаткового продукційного процесу
транспортової промисловости. Отже, вкладений в неї продуктивний капітал
додає вартість до транспортованого продукту, почасти через перенесення
вартости транспортових засобів, почасти тому, що вартість додається працею
транспорту. Ця остання додана вартість розкладається, як взагалі в
капіталістичній продукції, на покриття заробітної плати й на додаткову
вартість.

В кожному продукційному процесі велику ролю відіграє переміщення
предмету праці й потрібні на це засоби праці й робоча сила — напр.,
бавовну переміщується з чесальної майстерні до прядільні, вугілля підіймається
з шахти на поверхню. Перехід готового продукту як готового
товару з одного місця самостійної продукції на друге, просторово віддалене
від нього, показує нам те саме явище, тільки в ширшому маштабі.
Після перевозу продуктів з одного місця продукції в інше відбувається
перевіз готових продуктів із сфери продукції в сферу споживання.
Продукт тільки тоді готовий для споживання, коли він закінчить це
переміщення.

Як показано раніше, загальний закон товарової продукції такий: продуктивність
праці та утворювання нею вартости перебувають у зворотному
відношенні. Це має силу для транспортової промисловости, як і
для кожної іншої. Що менше мертвої та живої праці треба для перевозу
товару на дану віддаль, то вища продуктивна сила праці, і навпаки\footnote{
Рікардо цитує Сея, який вважає за щастя для торговлі те, що вона удорожнює
продукти й підвищує їхню вартість в наслідок транспортових витрат.
„Торговля, — каже Сей, — дозволяє нам одержувати товар у місці його постання й
перевозити його в інше місце споживання; отже, вона дозволяє нам збільшувати вартість товару на всю
ріжницю між його ціною в першому місці та в другому“.
Рікардо каже з цього приводу: „Правильно. Але як долучається до неї цей додаток
вартости? Через додачу до витрат продукції, поперше, витрат на транспорт,
а подруге, зиску на капітал, авансований торговцем. Товар має більшу вартість
лише з тієї причини, з якої може збільшитись вартість кожного товару, коли на
його продукцію та перевіз, раніш ніж купиться його, витратиться більше праці.
А це не можна вважати за одну з переваг торговлі“. („True, but how is the additional
value given to it? By adding to the cast of production, first, the expences of
conveyance, secondly, the profit on the advances of capital made by the merchant.
The commodity is only more valuable, for the same reason that every other commodity
may become more valuable, because more labour is expended on its production
and conveyance, before it is purchased by the consumer. This must not be
mentioned as one of the advantages of commerce*. (Ricardo. Principles of Pol. Econ.,
3-rd ed., London, 1821, ст. 309, 310).
}.

Абсолютна величина вартости, додавана до товарів транспортом, за
інших незмінних обставин, стоїть у зворотному відношенні до продуктивної
сили транспортової промисловости і в прямому відношенні до віддалей,
що на них товари переміщуються.

Відносна частина вартости, що її, за інших незмінних обставин,
долучають до ціни товару витрати на транспорт, стоїть у прямому відношенні
до просторової величини і ваги товару. Але є багато обставин,
що модифікують справу. Для перевозу, напр., потрібні більші або менші
\parbreak{}  %% абзац продовжується на наступній сторінці
