\parcont{}  %% абзац починається на попередній сторінці
\index{ii}{0112}  %% посилання на сторінку оригінального видання
відбирає їм його в другому випадку. Однак та обставина, що засоби праці льокально прикріплені,
пустили своє коріння в землю, надає цій частині основного капіталу особливої ролі в економії націй.
Їх не можна відіслати за кордон, вони не можуть циркулювати як товари на світовому ринку. Титули
власности на цей основний капітал можуть змінюватись, їх можна купувати й продавати, і остільки вони
можуть ідеально
циркулювати. Ці титули власности можуть навіть циркулювати на закордонних ринках, напр., в формі
акцій. Але від зміни осіб, що є власники такого виду основного капіталу, не змінюється відношення
між нерухомою, матеріяльно фіксованою частиною багатства даної країни і рухомою частиною того таки
багатства\footnote{До цього місця рукопис IV.~Відси рукопис II.~Ф.~Е.}.
\label{original-112}

Своєрідна циркуляція основного капіталу зумовлює своєрідний оборот. Та частина вартости, що її
втрачається в її натуральній формі в наслідок зношування, циркулює, як частина вартости продукту.
Продукт через свою циркуляцію перетворюється з товару на гроші, отже, на гроші перетворюється й та
частина вартости засобів праці, що її продукт несе в циркуляцію, а саме: ця частина вартости падає
краплями як гроші з процесу циркуляції, в тій самій пропорції, що в ній даний засіб праці перестає
бути носієм вартости в продукційному процесі. Отже, вартість цього засобу праці набирає тепер
двоїстого існування. Частина її лишається зв’язана з його споживною або натуральною формою, належною
продукційному процесові, а друга частина відокремлюється від неї як гроші. В перебігу свого
функціонування та частина вартости засобів праці, що існує в натуральній формі, постійно меншає,
тимчасом як перетворена на гроші частина вартости постійно більшає, поки, нарешті, засоби праці
одживуть свій вік, і вся їхня вартість, відокремившись від мертвого тіла, перетвориться на гроші.
Тут виявляється своєрідність в обороті цього елемента продуктивного капіталу. Його вартість
перетворюється на гроші рівнобіжно з тим, як на грошову лялечку перетворюється той товар, що є носій
його вартости. Але його зворотне перетворення з грошової форми на споживну форму відділяється від
зворотного перетворення товару на інші елементи продукції цього товару і визначається періодом його
власної репродукції, тобто часом, що протягом його засоби праці одживають свій вік, і треба їх
замінити на нові екземпляри такого самого роду. Коли час функціонування якоїсь машини, напр.,
вартістю в \num{10.000}\pound{ ф. стерл.}, дорівнює, припустімо, 10 рокам, то час обороту вартости, первісно
авансованої на неї, дорівнює 10 рокам. Поки не мине цей час, її не треба поновлювати, і вона
функціонує далі в своїй натуральній формі. Тимчасом її вартість частинами циркулює як частина
вартости товарів, що до їх безперервної продукції вона придається, — і таким чином поступінно
перетворюється на гроші, поки, нарешті, по десятьох роках, вона цілком перетвориться на гроші, а з
грошей знову на машину, вивершуючи, отже, свій оборот. До цього моменту
\parbreak{}  %% абзац продовжується на наступній сторінці
