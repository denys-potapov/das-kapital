\parcont{}  %% абзац починається на попередній сторінці
\index{ii}{0115}  %% посилання на сторінку оригінального видання
й поновлювати зворотною купівлею, зворотним перетворенням з грошової форми на елементи продукції.
Одним заходом їх вилучається з ринку меншими масами, ніж елементи основного капіталу, але тим
частіше доводиться їх вилучати з ринку, а тому авансування витраченого на них капіталу поновлюється
через коротші періоди. Це постійне поновлення упосереднюється постійним збутом того продукту, що в
ньому циркулює вся їхня вартість. Нарешті, вони безупинно пророблюють увесь кругобіг метаморфоз не
лише своєю вартістю, але й у своїй речовій формі; з товару вони постійно перетворюються знову на
елементи продукції цього самого товару.

Разом із своєю власною вартістю робоча сила постійно долучає до продукту додаткову вартість,
втілення неоплаченої праці. Отже, готовий продукт так само подає її постійно в циркуляцію, і вона
разом з ним перетворюється на гроші так само, як і інші елементи вартости продукту. Однак, тут, де
йдеться насамперед про оборот капітальної вартости, а не додаткової вартости, що обертається разом з
нею, — тут ми лишаємо це покищо осторонь.

З наведеного вище випливає ось що:

1) Визначеності форми основного й поточного капіталу походять лише з ріжниці в обороті капітальної
вартости, що функціонує в процесі продукції, або продуктивного капіталу. Ця ріжниця в обороті
походить і собі з ріжниці в способі, що ним різні складові частини продуктивного капіталу переносять
свою вартість на продукт, а не з їхньої різної участи в утворенні вартости продукту або не з
характеристичної ролі їх у процесі зростання вартости. Нарешті, ріжниця в передачі вартости
продуктові, — а тому й різні способи, що ними ця вартість вводиться через продукт у циркуляцію і в
наслідок його метаморфоз поновлюється в своїй первісній натуральній формі, — ця ріжниця походить з
відмінности тих речових форм, що в них існує продуктивний капітал, і що з них одна частина підчас
утворення окремого продукту споживається цілком, а другу зужитковується лише поступінно. Отже, лише
продуктивний капітал може розподілятись на основний і поточний. Навпаки, цієї протилежности не існує
для обох інших способів буття промислового капіталу, отже, ні для товарового капіталу, ні для
грошового капіталу; не існує її також як і протилежности цих обох форм проти продуктивного капіталу.
Вона існує лише для продуктивного капіталу і в межах його. Грошовий капітал і товаровий капітал
можуть скільки завгодно функціонувати як капітал і можуть хоч як швидко циркулювати, але зробитись
поточним капіталом протилежно до основного вони можуть лише тоді, коли перетворяться на поточні
складові частини продуктивного капіталу. Але через те, що ці обидві форми капіталу перебувають у
сфері циркуляції, то, як ми побачимо, економія від часів А.~Сміса не могла стриматися від спокуси
сплутати їх з поточною частиною продуктивного капіталу, об’єднуючи їх в категорію обіговий капітал.
А справді грошовий капітал і товаровий капітал є капітал циркуляції протилежно до продуктивного, але
не обіговий капітал протилежно до основного.
