
\index{ii}{0116}  %% посилання на сторінку оригінального видання
2) Оборот основної складової частини капіталу, а, значить, і потрібний для цього час обороту,
охоплює кілька оборотів поточної складової частини капіталу. Протягом того самого часу, коли
основний капітал зробив один оборот, поточний капітал робить їх кілька. Одна з складових частин
вартости продуктивного капіталу набуває визначености форми основного капіталу лише остільки,
оскільки засіб продукції, що в ньому вона існує, не зужитковується цілком в той час, що протягом
його продукт виготовляється і з продукційного процесу викидається як товар. Деяка частина йою
вартости мусить лишатися зв’язаною в старій, далі збереженій споживній формі, тимчасом як друга
частина циркулює в наслідок циркуляції готового продукту; навпаки, щодо поточних складових частин
капіталу, то разом з циркуляцією готового продукту циркулює вся їхня вартість.

3) Витрачувану на основний капітал частину вартости продуктивного капіталу авансується цілком одним
заходом на ввесь час функціонування тієї частини засобів продукції, що з неї складається основний
капітал. Отже, капіталіст одним заходом кидає цю вартість в циркуляцію, а вилучається її знову з
циркуляції лише частинами й поступінно через реалізацію тих частин вартости, що їх основний капітал
частинами долучає до товарів. З другого боку, сами засоби продукції, що в них фіксується одна з
складових частин продуктивного капіталу, вилучаються з циркуляції одним заходом, і на ввесь час
свого функціонування їх зв’язується з продукційним процесом. Але протягом цього часу не треба їх
замінювати на нові екземпляри того самого роду, ні репродукувати. Протягом довшого або коротшого
часу вони й далі беруть участь в утворенні товарів, подаваних в циркуляцію, не вилучаючи однак з
циркуляції елементів свого власного поновлення. Отже, протягом цього часу вони також і собі не
потребують поновлення авансування з боку капіталіста. Нарешті, капітальна вартість, витрачена на
основний капітал, перебігає кругобіг своїх форм протягом часу функціонування тих засобів продукції,
що в них вона існує, — перебігає не речово, а лише своєю вартістю, та й то лише частинами й
поступінно. Тобто, частина вартости основного капіталу циркулює безупинно як частина вартости товару
й перетворюється на гроші, не перетворюючись знову з грошей на свою первісну натуральну форму. Це
зворотне перетворення грошей на натуральну форму засобів продукції відбувається лише наприкінці
періоду їх функціонування, коли засоби продукції цілком зужитковано.

4) Елементи поточного капіталу так само постійно повинні бути зафіксовані в процесі продукції — в
разі він має перебігати безупинно — як і елементи основного капіталу. Але фіксовані таким чином
елементи першого постійно поновлюється in natura (засоби продукції замінюється на нові екземпляри
того самого роду, а робочу силу — через постійно поновлюваний закуп), тимчасом як елементи основного
капіталу протягом їхнього існування ні самі не поновлюються, ані доводиться поновлювати їх купівлю.
В продукційному процесі постійно є сировинні й допоміжні матеріяли, але їх завжди замінюється на
нові екземпляри того самого
\parbreak{}  %% абзац продовжується на наступній сторінці
