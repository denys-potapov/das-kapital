\parcont{}  %% абзац починається на попередній сторінці
\index{ii}{0118}  %% посилання на сторінку оригінального видання
зношування в наслідок залізничного руху, скільки від якости дерева, заліза та будівельного
матеріялу, що підпадають під вплив атмосферних чинників. Один суворий зимовий місяць заподіє більше
шкоди залізниці, ніж цілий рік залізничного руху“ (R. Р. Williams. „On the Maintenance of Permanent
Way“. Доповідь в Institute of Civil Engineers, восени 1867 р.).

Нарешті тут, як і всюди в великій промисловості, відіграє ролю моральне зношування; по 10 роках
звичайно можна купити за \num{30.000}\pound{ ф. ст.} стільки ж вагонів і паротягів, скільки раніш коштувало \num{40.000}\pound{ ф. стерл}. Отже, на цей матеріял треба рахувати 25\% зниження з ринкової ціни, хоча б не відбулося
жодного зниження споживної вартости (Lardner,
„Railway Economy“).

„Трубчасті мости в їхній теперішній формі не поновлюються“. (Тепер бо є кращі форми таких мостів).
„Звичайний ремонт, зняття й заміна поодиноких частин недоцільні“ (W. Р. Adams, „Road and Rails“.
London, 1862). В наслідок поступу промисловости в засобах праці здебільша відбуваються постійні
перевороти. Тому їх замінюється не в їх первісній формі, а в формі, що зазнала перевороту. З одного
боку, та обставина,
що маса основного капіталу вкладається в певній натуральній формі і повинна в ній протриматися
протягом певного пересічного життьового часу, становить причину того, що нові машини та ін.
вводиться лише поступінно, а тому ця обставина перешкоджає швидкому й загальному запровадженню
удосконалених засобів праці. З другого боку, конкуренційна боротьба, особливо підчас рішучих
переворотів, примушує заміняти
старі засоби праці ще до їхньої природної смерти на нові. Катастрофи, кризи — ось що, головним
чином, примушує до такого передчасного поновлення технічних знарядь в широкому суспільному маштабі.

Зношування (залишаючи осторонь моральне) є та частина вартости, що її основний капітал в наслідок
свого зуживання поступінно передає продуктові, в тому пересічному розмірі, що в ньому він втрачає
свою споживну вартість.

Почасти це зуживання таке, що основний капітал має певний пересічний час життя; його цілком
авансується на цей час; а як він мине, то треба його цілком замінити. Для живих засобів праці,
напр., коней, час репродукції визначається самими законами природи. Пересічний час життя їхнього як
засобів праці визначається законами природи. Скоро цей час мине, зужиті екземпляри треба замінювати
на нові. Кінь не може замінюватись частинами, а тільки на нового коня.

Щодо інших елементів основного капіталу, то тут можливе періодичне або частинне поновлення. І тут
треба відрізняти частинне або періодичне заміщення від поступінного поширення підприємства.

Основний капітал складається почасти з однорідних складових частин, що неоднаково довго тривають, а
поновлюються частинами в різні переміжки часу. Прим., рейки біля станції, що їх доводиться
поновлювати частіше, ніж на решті залізничної лінії. Так само й злежні, що їх на бельгійських
залізницях 50-ми роками, за Ларднером, доводилось поновлювати щорічно на 8\%, що, отже, цілком
поновлювались протягом 12 років.
\index{ii}{0119}  %% посилання на сторінку оригінального видання
Отже, тут справа така: певну суму авансується на певний рід основного капіталу, напр., на
десять років. Цю витрату роблять одним заходом. Але певну частину цього основного капіталу, що його
вартість увійшла в вартість продукту й разом з ним перетворилась на гроші, щороку заміщується in
natura, тимчасом як друга частина існує і далі в своїй первісній натуральній формі. Ось оця-о
одноразова витрата і лише
частинна репродукція в натуральній формі й відрізняє цей капітал як основний від поточного.

Інші елементи основного капіталу складаються з неоднорідних частин, що зношуються протягом
неоднакового часу, а тому й мусять вони поновлюватись неодноразово. Саме так справа стоїть з
машинами. Те, що ми щойно зазначили щодо різної життьової тривалости різних складових частин
основного капіталу, має тут силу й щодо життьової тривалости різних складових частин тієї самої
машини, що фігурує як елемент цього
основного капіталу.

Щодо поступінного поширення підприємства з перебігом частинного поновлення, то ми зауважуємо таке.
Хоч, як ми бачили, основний капітал in natura й далі діє в продукційному процесі, однак частина його
вартости, залежно від пересічного зношування, циркулює разом з продуктом, перетворюється на гроші й
становить елемент грошового резервного фонду на заміщення капіталу, коли надходить час для його
репродукції in natura. Ця частина вартости основного капіталу, перетворена таким
чином на гроші, може придатися на те, щоб поширити підприємство або вробити поліпшення в машинах, що
збільшать їхню діяльність. Таким чином відбувається через більші або менші переміжки репродукція і
саме — розглядаючи з суспільного погляду — репродукція в поширеному маштабі: екстенсивно — коли
поширюється поле продукції; інтенсивно, коли засоби продукції робляться ефективніші. Ця репродукція
в поширеному маштабі випливає не з акумуляції — перетворення додаткової вартости на капітал, — а із
зворотного перетворення вартости, яка, відгалузившись, відокремившись у грошовій формі від тіла
основного капіталу, перетворилась на новий, або додатковий, або ефективніший, основний капітал того
самого роду. Звичайно, залежить почасти від специфічного характеру даного підприємства, чи може воно
та оскільки так поступінно поширюватись; отже, цей характер також визначає, в яких розмірах треба
нагромаджувати резервний фонд, щоб його можна було таким чином знову вкласти в це підприємство, та в
які переміжки часу це можна зробити. З другого боку, щодо спроможности запроваджувати детальні
поліпшення в наявних машинах, то це залежить, звичайно, від характеру цих поліпшень і конструкції
самої машини. Але до якої значної міри, напр., у залізничних спорудах доводиться з самого початку
звертати увагу на цю обставину, це доводить Адамс: „Вся конструкція повинна будуватись на тому
самому принципі, що панує в вулику: на здібності необмежено поширюватись. Всі надто солідні й
особливо симетричні будови являють зло, коли їх доводиться розбирати в разі поширення“ (р. 123).
