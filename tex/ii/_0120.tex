
\index{ii}{0120}  %% посилання на сторінку оригінального видання
Здебільша це залежить від площі, яка є в розпорядженні. При деяких будівлях можна надбудовувати
горішні поверхи, при інших треба поширювати в боки, тобто треба більше землі. За капіталістичної
продукції, з одного боку, багато засобів витрачається марно, а з другого боку, при поступінному
поширенні підприємства спостерігається багато випадків такого роду недоцільного поширення будівель в
боки (почасти це шкодить робочій силі), бо нічого не робиться за суспільним пляном, а все залежить
від безлічі різних обставин, засобів і т. ін., що з ними має діло капіталіст. А з цього постає
велике марнотратство продуктивних сил.

Таке повторне вкладання грошового резервного фонду частинами (тобто частини основного капіталу,
знову перетвореної на гроші) найлегше робиться в хліборобстві. Просторове обмежене поле продукції
тут якнайбільш здібне поступінно вбирати капітал. Так само стоїть справа й там, де відбувається
природна репродукція, як, напр., у скотарстві.

Основний капітал спричиняє особливі витрати на зберігання. Почасти це зберігання здійснюється самим
процесом праці; основний капітал псується, коли він не функціонує в процесі праці (див. кн. І, розд.
VI і розд. XIII. Зношування машин, що постає від їх невживання). Тому англійський закон вважає
буквально за шкоду (waste), коли орендовані ділянки не обробляється заведеним у країні способом. (W.
A. Holdsworth, Barrister at Law, „The Law of Landlord and Tenant“, London, 1857, p. 96). Це
зберігання, що походить з ужитку в процесі праці, є безплатний природний дар живої праці. Ця
властива праці сила зберігання має двоїстий характер. З одного боку, праця зберігає вартість
матеріялів праці, переносячи їх на продукт; з другого боку, оскільки вона й не переносить на продукт
вартости засобів праці, вона все ж зберігає їхню вартість, зберігаючи їхню споживну вартість тим, що
вони функціонують у процесі продукції.

Однак, для того, щоб основний капітал зберігався в належному стані, потрібні й безпосередні витрати
праці. Машини треба час від часу чистити. Тут справа йде про новододавану працю, що без неї вони
будуть непридатні до вжитку, про безпосереднє зберігання від шкідливих стихійних впливів, завжди
сполучених з продукційним процесом, отже, про зберігання машин у стані працездатности в прямому
значенні цього слова. Само собою зрозуміло, нормальну життьову тривалість основного капіталу
обчислюється, зважаючи на те, що здійсняться умови, в яких він може нормально існувати протягом
цього часу, так само як припускається, що, коли людина живе пересічно 30 років, вона також і
миється. Отже, тут ходить не про те, щоб замінити працю, яка є в машині, а про постійну новододавану
працю, потрібну в наслідок уживання машини. Тут ідеться не про ту працю, що її виконує машина, а про
ту, що прикладається до машини, тим часом як машина є не чинник продукції, а сировинний
матеріял. Капітал, витрачений на цю працю, — хоч і не входить власне в той процес праці, що йому
продукт завдячує своїм походженням, — належить до поточного капіталу. Цю працю доводиться постійно
витрачати на продукцію, а тому й вартість цієї праці завжди мусить покриватись
\parbreak{}  %% абзац продовжується на наступній сторінці
