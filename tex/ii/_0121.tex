\parcont{}  %% абзац починається на попередній сторінці
\index{ii}{0121}  %% посилання на сторінку оригінального видання
вартістю продукту. Витрачений на цю працю капітал належить до тієї частини поточного капіталу, яка
має покрити загальні затрати (Unkosten), і треба її розподілити на новоспродуковану вартість
відповідно до пересічного річного розрахунку. Ми бачили, що у власне промисловості цю працю чищення
робітники виконують безплатно під час павз для відпочинку, і саме через це вони часто виконують її
під час самого процесу продукції; від цього походить більшість нещасних випадків. Цю працю
не оплачується в ціні продукту. Отже, споживач остільки й має її безплатно. З другого боку,
капіталіст ощаджує таким чином витрати на зберігання машин. Робітник платить сам, власного особою, і
це становить одну з тих таємниць самозберігання капіталу, що в дійсності утворюють юридичні права
робітника на машину й перетворюють його навіть з буржуазного погляду на співвласника машини. Однак,
в різних галузях продукції, там, де машини для чищення треба вилучати з продукційного процесу, а
тому й не можна їх чистити між іншим, — як, прим., при чищенні паровозів, — ця праця зберігання
належить до поточних витрат, тобто є елемент поточного капіталу. Після трьох щонайбільше днів праці
паровоза треба подати в депо й там чистити; щоб не зіпсувати казан, промиваючи його, треба спочатку
його охолодити. (R. С. №\num{17823}).

Власне ремонт або полагодження потребують таких витрат капіталу й праці, що їх немає в первісно
авансованому капіталі, а значить, і не можна їх — в усякому разі не завжди можна — замістити й
покрити поступінним заміщенням вартости основного капіталу. Коли, напр., вартість основного капіталу
дорівнює \num{10.000}\pound{ ф. стерл.}, а його загальна життьова тривалість дорівнює 10 рокам, то ці \num{10.000}\pound{ ф.
стерл.}, по 10 роках цілком перетворившись на гроші, заміщують лише вартість первісно вкладеного
капіталу, але вони не заміщують капіталу, зглядно праці, новодолученого під час ремонту. Це є
додаткова складова частина вартости, що її теж авансується не одним заходом, а залежно від потреби,
коли саме надходять різні моменти її авансування, це з самої природи речей залежить від випадку.
Кожен основний капітал потребує таких пізніших,
часткових, додаткових витрат капіталу на засоби праці й робочу силу.

Ушкодження, що їх зазнають поодинокі частини машин і~\abbr{т. ін.}, випадкові своєю природою, а тому так
само випадкові й зумовлювані цим полагодження. Однак з таких ремонтних робіт відзначаються дві
відміни їх, що мають більш-менш сталий характер і припадають на різні періоди життя основного
капіталу — це недуги дитинства й куди численніші недуги віку, що вийшов поза межі середнього віку
життя. Хоч яка досконала конструкція, прим., машини, яка входить у продукційний процес, на практиці,
при її застосуванні, завжди виявляються хиби, що їх треба виправляти добавочною працею. З другого
боку, що більше вона виходить за середній свій вік, отже, що більше стає її нормальне зношування, а
матеріял, що з нього вона складається, зуживається й старіє, то частішого й більшого ремонту треба,
щоб підтримати функціонування машини до скінчення її пересічного життьового періоду; так само як
старій людині, шоб не вмерти передчасно, доводиться більше
\parbreak{}  %% абзац продовжується на наступній сторінці
