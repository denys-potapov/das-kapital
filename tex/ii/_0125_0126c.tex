\parcont{}  %% абзац починається на попередній сторінці
\index{ii}{0125}  %% посилання на сторінку оригінального видання
of Inquiry on Caledonian Railway, передруковано в Money Market Review, 1867)\footnote*{
„Цитоване місце є в нумері з 25 січня 1868 року, і взято його з статті
в „Money Market Rewiew“ — The Caledonian Railway, The Directors Reply, де йде мова про звіт капітана Фіцморіса“. \emph{Ред.}
}.

Практично неможливо й недоцільно розмежовувати заміщення й підтримання основного капіталу в
хліборобстві, принаймні, оскільки воно ще не застосовує сили пари. „Коли є повний, але не надто
великий комплект реманенту (різних хліборобських та інших всякого роду знарядь праці та
господарювання), щорічне зношування та витрати на підтримання реманенту звичайно обчислюється в
15--25\% авансованого капіталу, залежно від різних наявних умов“ (Kirchhof, Handbuch der
landwirtschaftlichen Betriebslehre. Berlin, 1862, p. 137).

Щодо рухомої частини залізниці, то зовсім не можна розмежувати ремонт і заміщення. „Ми підтримуємо
нашу рухому частину у наявних її розмірах. Скільки паровозів є в нас, стільки ми й підтримуємо. Коли
з плином часу паровіз робиться непридатний, так що вигідніше збудувати новий, то ми й будуємо його
на кошти доходів, при чому, звичайно, записуємо на дохід вартість матеріялів, що лишились від старої
машини\dots{} А лишається завжди чимало\dots{} Колеса, осі, казан тощо, коротко кажучи, лишається чимала частина
старого паровозу“. (T.~Gooch, Chairman of Great Western Railway C°, R.~C. № \num{17327}--29). „Ремонтувати
значить відновлювати; для мене немає слова „заміщення“\dots{} Коли залізничне товариство купило вагон
або паротяг, то воно мусить їх так полагодити, щоб вони вічно могли служити (\num{17784}). Ми обчислюємо
витрати на паротяги в 8\sfrac{1}{2}\pens{ пенсів} на англійську милю пробігу. На ці 8\sfrac{1}{2}\pens{ пенсів} ми назавжди
підтримуємо паротяги. Ми поновлюємо наші машини. Коли ви хочете купити машину нову, ви витрачаєте
більше грошей, ніж треба\dots{} В старій машині завжди буде пара коліс, вісь або ще яка придатна
частина, і це дає змогу збудувати дешевше таку саму гарну машину, як і цілком нова (\num{17790}). Тепер я
продукую щотижня новий паротяг, тобто такий самий гарний, як новий, бо в ньому казан, циліндр і рама
нові“ (\num{17823}. Archibald Sturrock, Locomotive Superintendent of Great Northern Railway в R.~C. 1867).

Це стосується й до вагонів: „З плином часу запас паротягів і вагонів постійно поновлюється; одного
разу насаджуються нові колеса, другого разу робиться нову ряму. Частини, що на них ґрунтується рух і
що найбільше зношуються, відновлюються поступінно; таким чином, машини й вагони можуть підлягати
стільком ремонтам, що в багатьох з них не лишиться й сліду старого матеріялу\dots{} Навіть, коли вони
зробляться вже зовсім непридатні для ремонту, з старих вагонів або паротягів перероблюються
поодинокі частини і таким чином вони ніколи не гинуть цілком для залізниці. Тому рухомий капітал
перебуває в стані постійної репродукції:
\index{ii}{0126}  %% посилання на сторінку оригінального видання
те, що для залізничної колії повинно в певний час робити одним заходом, а саме, коли лінію
цілком перекладається наново, — це в рухомій частині робиться поступінно з року на рік. Її існування
вічне, вона завжди омолоджується“ (Lardner, p. 116).

Цей процес, як його описує тут Ларднер щодо залізниць, не підходить до поодинокої фабрики, але
змальовує нам картину постійної, частинної, переплетеної з ремонтом репродукції основного капіталу в
межах якоїсь цілої галузі промисловости, або взагалі в межах сукупної продукції, розглядуваної у
суспільному масштабі.

Наводимо тут ще одну вказівку, що пояснює, в яких широких розмірах спритні управління можуть
орудувати поняттями ремонт і заміщення, щоб здобувані дивіденди. Згідно з вище цитованою доповіддю
Р.~Б.~Вільямса, різні англійські залізничні товариства пересічно за ряд років списували з рахунку
доходів такі суми на ремонт та витрати на підтримання залізничної колії та будівель (на англійську
милю довжини колії щороку):
\begin{table}[H]
\centering
\begin{tabular}{lr@{}r}
London and North Western\dotfill{} & 370 & \pound{ф. стерл.}\\
Midland\dotfill & 225 & \\
London and South Western\dotfill & 257 & \\
Great Northern\dotfill & 360 & \\
Lancashire and Yorkshire  & 377 & \\
South Eastern\dotfill & 263 & \\
Brignton\dotfill & 266 & \\
Manchester and Sheffield & 200 & \\
\end{tabular}
\end{table}

\noindent{}Ці ріжниці лише дуже мало залежать від неоднакового розміру дійсно зроблених витрат: вони походять
майже виключно з неоднаковости в способах обчислення, з того, чи залічується статті видатків на
рахунок капіталу, чи на рахунок доходів. Вільямс прямо каже: «Меншу цифру витрат вибирається тому,
що це потрібно для доброго дивіденда, а більшу цифру подається тому, що є досить високий дохід, який
може витримати це“.

Іноді зношування, отже, і заміщення його стає величиною практично зникомою, так що на увагу береться
лише витрати на ремонт. Те, що Ларднер каже далі про works of art\footnote*{
Works of art — будівельні споруди. \emph{Ред.}
} на залізницях, має силу взагалі
для всіх таких довготривалих споруд, як канали, доки, залізні та кам’яні мости і~\abbr{т. ін.} —
„Зношування, що постає в наслідок повільного впливу часу на солідніших спорудах, діє майже непомітно
протягом невеликих переміжків часу; а коли минає більше часу, прим., століття, то воно мусить
призвести до поновлювання, повного або частинного, навіть для найсолідніших споруд. Це непомітне
зношування, порівняно з помітнішими зношуваннями інших частин залізниці, можна прирівняти до вікових
і періодичних відхилів у русі світових тіл. Вплив часу на масивніші споруди залізниці —
\parbreak{}  %% абзац продовжується на наступній сторінці
