\parcont{}  %% абзац починається на попередній сторінці
\index{ii}{0129}  %% посилання на сторінку оригінального видання
на гроші \sfrac{1}{10} = 1000\pound{ ф. стерл}. Ці 1000\pound{ ф. стерл.} протягом року
перетворилися з грошового капіталу на продуктивний капітал, а потім на
товаровий капітал, а з нього знову на грошовий капітал. Вони повернулись
до своєї первісної грошової форми, як поточний капітал, — коли ми
розглядаємо його в цій формі, при чому байдуже, чи перетвориться наприкінці
року грошовий капітал в 1000\pound{ ф. стерл.} знову на натуральну
форму якоїсь машини, чи ні. Обчислюючи цілий оборот авансованого продуктивного
капіталу, ми фіксуємо тому всі його елементи в грошовій формі,
так що поворот до грошової форми вивершує оборот. Ми завжди припускаємо,
що вартість авансовано в грошах, навіть і при безперервному
процесі продукції, коли ця грошова форма вартости є лише форма
рахункових грошей. Таким способом ми й можемо обчислити пересічну
величину.

3) 3 цього випливає, що хоча б переважна частина авансованого
продуктивного капіталу складалася з основного капіталу, час репродукції
якого, а, значить, і час обороту, охоплює багаторічний цикл, все ж капітальна
вартість, що обертається протягом року, в наслідок повторюваних
протягом року оборотів поточного капіталу, може бути більша, ніж ціла
вартість авансованого капіталу.

Припустімо, що основний капітал = 80.000\pound{ ф. стерл.}, час його репродукції
= 10 рокам, отже, 8.000\pound{ ф. стерл.} щороку повертаються до своєї
грошової форми, або основний капітал робить \sfrac{1}{10} свого обороту. Хай
поточний капітал дорівнює 20.000\pound{ ф. стерл.} і робить на рік п’ять оборотів.
Отже, ввесь капітал тоді дорівнює 100.000\pound{ ф. стерл}. Основний
капітал, що обернувся, дорівнює 8 000\pound{ ф. стерл.}, поточний капітал, що
обернувся, дорівнює 5X20.000 = 100.000\pound{ ф. стерл}. Отже, капітал, що
обернувся протягом року = 108.000\pound{ ф. стерл.}, на 8.000\pound{ ф. стерл.} більший,
ніж авансований капітал. Обернулось 1 + \sfrac{2}{25} капіталу.

4) Отже, оборот вартости авансованого капіталу відділяється
від часу його справжньої репродукції або від часу реального обороту
його складових частин. Припустімо, що капітал в 4.000\pound{ ф. стерл.} обертається
п’ять разів на рік. Тоді капітал, що обернувся, дорівнює
5X4.000 = 20.000\pound{ ф. стерл}. Наприкінці кожного обороту повертається,
щоб знову авансуватись, первісно авансований капітал в 4.000\pound{ ф. стерл}.
Його величина не змінюється від числа тих періодів обороту, що
протягом них він знову функціонує як капітал. (Додаткову вартість
лишаємо осторонь).

Отже, в прикладі 3) згідно з припущенням, наприкінці року до рук
капіталіста повернулось: а) сума вартости з 20.000\pound{ ф. стерл.}, що її він
знову витрачає на поточні складові частини капіталу, і б) сума 8.000\pound{ ф.
стерл.}, що в наслідок зношування відокремилась від вартости авансованого
основного капіталу; разом з тим у продукційному процесі, як і
раніш, лишається той самий основний капітал, але вартість його зменши.
лася до 72.000\pound{ ф. стерл.} замість 80.000\pound{ ф. стерл}. Отже, треба продовжувати
продукційний процес ще дев’ять років, поки авансований основний
капітал доживе свого віку, перестане функціонувати як продуктотворчий
\parbreak{}  %% абзац продовжується на наступній сторінці
