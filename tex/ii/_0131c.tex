
\index{ii}{0131}  %% посилання на сторінку оригінального видання
\begin{table}[H]
\centering
\begin{tabular}{l@{ }l@{ }r@{ }r@{ }l@{ }l}
\num{50.000} $:$ 2 \deq{} \num{25.000}\usd{ дол.} на & 10 & років & \deq{} & \phantom{0}\num{2.500}\usd{ дол.} на 1 рік \\

\num{50.000} $:$ 4 \deq{} \num{12.500} & \phantom{0}2 & & \deq{} & \phantom{0}\num{6.250} \\

\num{50.000} $:$ 4 \deq{} \num{12.500} & \tbfrac{1}{2}& & \deq{} & \num{25.000} \\
\cmidrule{2-5}
& \multicolumn{2}{@{ }r@{ }}{На 1 рік} & \deq{} & \num{33.750} 
\end{tabular}
\end{table}

\noindent{}Отже, пересічний час, що протягом його ввесь капітал обертається
один раз, становить 16 місяців\footnote*{
В обчисленні є помилка. Пересічний час, що протягом його обертається ввесь
капітал, становить не 16 місяців, а 17,76 місяців. \emph{Ред.}
}. Візьмімо другий приклад. Хай чверть
усього капіталу в \num{50.000}\usd{ долярів} обертається протягом 10 років; друга
чверть — протягом року; і решта — половина — двічі на рік. Тоді річні витрати
будуть такі:

\begin{table}[H]
\centering
\noindent\begin{tabular}{r@{ }c@{ }r@{ \deq{} }l}
\num{12.500} & $:$ & 10 & \phantom{0}\num{1.250}\usd{ долярів} \\
\num{12.500} &     &    & \num{12.500}\ditto{\usd{ долярів}} \\
\num{25.000} & ×   & 2  & \num{50.000}\ditto{\usd{ долярів}} \\
\cmidrule(rl){1-3}
\multicolumn{3}{@{ }r@{ \deq{} }}{Протягом 1 року обернулось} & \num{63.750}\ditto{\usd{ долярів}}
\end{tabular}
\end{table}

\noindent{}(Scrope „Роl. Econ.“, edit. Alonzo Potter. New-York, 1841, р. 141, 142).

6) Справжні й позірні відмінності в обороті різних частин капіталу. —
Той самий Скроп каже там само: „Капітал, що його фабрикант, сільський
господар або купець витрачає на видачу заробітної плати, циркулює якнайшвидше,
бо він, коли робітникам платиться раз на тиждень, обертається,
може, раз на тиждень в наслідок щотижневих надходжень за
продані товари або оплачені рахунки. Капітал, вкладений в сировинний
матеріял або готові запаси, циркулює з меншою швидкістю; він може
обернутись два або чотири рази на рік, залежно від того, скільки
часу минає між закупові матеріялів і продажем товарів, — ми припускаємо,
що кредит на закуп і продаж дається на однаковий термін. Капітал, вкладений
в знаряддя й машини, циркулює ще повільніше, бо він протягом
5 або 10 років пересічно, може, зробить один оборот, тобто — його зуживеться
і поновиться, хоч деякі знаряддя вже зуживеться й по
небагатьох операціях. Капітал, вкладений в споруди, напр., в фабрики,
крамниці, склади, амбари, брук, зрошувальні споруди, тощо, як здається,
взагалі не циркулює. А в дійсності й ці споруди, відіграючи
свою ролю в продукції, зношуються цілком так само, як і вище згадані,
і їх треба репродукувати, щоб продуцент міг далі продовжувати
свої операції. Ріжниця лише в тому, що їх зуживається й репродукується
повільніше, ніж інші\dots{} Вкладений в них капітал робить, може,
один оборот протягом 20 або 50 років“.

Скроп сплутує тут ту ріжницю в русі певних частин поточного
капіталу, до якої призводять — щодо поодинокого капіталіста — терміни
виплат і кредитові відносини, з тією ріжницею оборотів, яка випливає
\parbreak{}  %% абзац продовжується на наступній сторінці
