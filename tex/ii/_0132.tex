\parcont{}  %% абзац починається на попередній сторінці
\index{ii}{0132}  %% посилання на сторінку оригінального видання
з природи капіталу. Він каже, що заробітна плата мусить щотижня покриватись
щотижневими надходженнями від оплачуваних продажів або рахунків.
Поперше, тут треба зауважити, що й для заробітної плати є
відмінності залежно від довжини строку виплат, тобто від протягу
того часу, що на нього робітник повинен кредитувати капіталіста; отже,
залежно від того, який строк видачі заробітної плати: щотижневий,
щомісячний, тримісячний, піврічний тощо. Тут має силу вище розглянутий
закон: „Потрібна маса засобів виплати (тобто грошового капіталу, що
його доводиться авансувати воднораз) стоїть у зворотному відношенні
до протягу періодів виплати“. (Кн.~І, розд. III, 3, b.).

Подруге, в тижневий продукт входить не лише вся нова вартість, долучена
підчас його продукції тижневою працею, але також і вартість сировинних
і допоміжних матеріялів, зужиткованих на його продукцію. Разом
з продуктом циркулює й ця, вміщена в ньому, вартість. В наслідок продажу
цього продукту вона набуває грошової форми, і знову її треба
перетворити на ті самі елементи продукції. Це однаково має силу так
щодо робочої сили, як і для сировинних та допоміжних матеріялів. Але
ми вже бачили (розділ VI, II, I), що для безперервности продукції потрібен
запас засобів продукції, різний для різних галузей підприємств,
а в тій самій галузі підприємств знову таки різний для різних складових
частин цього елемента поточного капіталу, напр., для вугілля й бавовни.
Тому, хоч ці матеріяли завжди доводиться заміщувати in natura, не треба
їх постійно знову купувати. Оскільки часто робиться закуп, це залежить
від розмірів наготовленого запасу, від того, на який час вистачить його,
поки його вичерпається. Щодо робочої сили, то тут такого запасу немає.
Зворотне перетворення на гроші для частини капіталу, витраченої на
працю, відбувається рівнобіжно з зворотним перетворенням частини, витраченої
на допоміжний та сировинний матеріял. Але зворотне перетворення
грошей, з одного боку, на робочу силу, а з другого, на сировинний
матеріял відбувається окремо, бо терміни закупу й виплати для цих двох
складових частин різні: одну з них, як продуктивний запас, купується
через довші переміжки, а другу, робочу силу, через коротші, прим.,
щотижня. З другого боку, крім продуктивного запасу, капіталіст мусить
мати запас готових товарів. Лишаючи осторонь труднощі з продажем
тощо, треба, напр., випродукувати певну кількість на замовлення. Коли
продукується останню частину її, вже готова частина лишається на складах
до того часу, поки зробиться все замовлене. Інші відмінності в обороті
поточного капіталу постають тоді, коли окремим елементам його
доводиться довший час лишатись на підготовчій стадії продукційного
процесу (сушіння дерева тощо), ніж іншим.

Кредитова система, що на неї тут посилається Скроп, а також і торговельний
капітал модифікують оборот щодо поодинокого капіталіста.
А в суспільному маштабі вони модифікують його лише остільки, оскільки
вони прискорюють не лише продукцію, а й споживання.
