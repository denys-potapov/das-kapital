
\index{ii}{0133}  %% посилання на сторінку оригінального видання
\section{Теорії про основний та обіговий капітал. Фізіократи~і~Адам Сміс}

У Кене ріжниця між основним і обіговим капіталом з’являється як
ріжниця між avances primitives\footnote*{
Аванси первинні. \emph{Ред.}
} і avances annuelles\footnote*{
Аванси річні. \emph{Ред.}
}. Він правильно визначає
цю ріжницю як ріжницю в межах продуктивного капіталу, тобто
капіталу, вкладеного в безпосередній процес продукції. А що для нього
єдиним справді продуктивним капіталом є капітал, застосовуваний в хліборобстві,
тобто капітал фармера, то й ці ріжниці подає він тільки для
капіталу фармера. Цим самим пояснюється, чому він для однієї частини
капіталу бере річний період обороту, для другої — довший (десятирічний).
В дальшому розвитку свого вчення фізіократи почали мимохідь переносити
ці ріжниці й на інші відміни капіталу, на промисловий капітал взагалі.
Для суспільства ріжниця між авансуваннями щорічними й багаторічними
така важлива, що багато економістів, навіть після Адама Сміса,
повертались до цього визначення.

Ріжниця між обома відмінами авансів постає лише тоді, коли авансовані
гроші перетворено на елементи продуктивного капіталу. Ця ріжниця
існує виключно в рамцях продуктивного капіталу. Тому Кене й не спадає
на думку залічувати гроші до первинних або щорічних авансів. Як аванси
для продукції, тобто як продуктивний капітал, обидві ці категорії протистоять
так само й грошам, як і наявним на ринку товарам. Далі, у Кене
ріжниця між цими двома елементами продуктивного капіталу правильно
сходить на ріжницю між способами, що ними ці елементи входять у вартість
готового продукту, а значить, на ріжницю між способами циркуляції
їхньої вартости разом з вартістю продукту, а тому й на ріжницю
між способами їхнього заміщення або їхньої репродукції, коли вартість
одного елемента щорічно заміщується цілком, а вартість другого — частинами
протягом довших періодів\footnote{
Порівн. Quesnay, Analyse du Tableau Economique. (Physiocrates, éd. Daire,
I.~Partie, Paris. 1846). Ми читаємо там, напр. „Щорічні аванси складаються з витрат,
що їх робиться щороку на обробіток землі; ці аванси треба відрізняти від первинних
авансів, що становлять фонд організації сільського господарства“). Les
avances annuelles consistent dans les dépenses qui se font annuellement pour le
travail de la culture; ces avances doivent être distinguées des avances primitives,
qui forment les fonds de l’établissement de la culture. P. 59). У пізніших фізіократів
аванси часто зветься вже просто капіталом: „Capital ou avances“ Dupont de
Nemours, „Origine et Progrès d’une science nouvelle“, 1767 (Daire, I, p. 291 \footnote*{
Цитоване місце є не в статті „Origine et Progrès“, 1767 (Daire, I, p. 291),
a в статті „Maximes du docteur Quesnay“ (Daire, I, p. 391). \emph{Ред.}
},
далі Le Trosne пише: „В наслідок більшої або меншої довготривалости продуктів
праці, нація має чималий фонд багатств, незалежний від його щорічної
репродукції; фонд, що становить \so{капітал}, нагромаджений протягом довгого
часу, первинно оплачений продуктами, постійно поновлюваний і збільшуваний“
(„Au moyen de la durée plus ou moins grande des ouvrages demain d'œuvre, une
nation possède un fonds considérable de richesses, indépendant de sa reproduction
annuelle, qui forme un \so{capital} accumulé de longue main, et originairement payé
avec des productions, qui s'entretient et s’augmente toujours“ (Daire, I, p. 928).
Тюрґо вже систематично вживає слова капітал замість аванси і ще повніше ототожнює
аванси мануфактуристів з авансами фармерів (Turgot, „Rétlexions sur la
Formation et la Distribution des Richesses“, 1766).
}.

Єдиний успіх, що його зробив А.~Сміс, — це узагальнення зазначених
категорій. Він прикладає їх уже не лише до спеціяльної форми капіталу,
до капіталу фармера, але взагалі до всякої форми продуктивного капіталу.
Відси само собою зрозуміло, що замість ріжниці між однорічним і багаторічним
оборотом, різниці, запозиченої від хліборобства, виступає взагалі
ріжниця різночасних оборотів, так що оборот основного капіталу завжди
\index{ii}{0134}  %% посилання на сторінку оригінального видання
охоплює більше, ніж один оборот капіталу обігового, хоч який буде
протяг цих оборотів обігового капіталу: річний, більш ніж річний або
менш ніж річний. Таким чином, у Сміса avances annuelles перетворюються
на обіговий, a avances primitives — на основний капітал. Але цим узагальненням
категорій і обмежується його крок наперед. Щодо виконання він
лишається далеко позаду Кене.

Вже той грубий емпіричний спосіб, що ним він розпочинає свій дослід,
породжує плутанину: „Є два способи застосувати капітал так, щоб
він давав своєму власникові дохід або зиск“\footnote*{
„There are two different ways in which a capital may be employed so as to
yield a revenue or profit to its employer“. (Wealth of Nations. Book II, ch. 1, p. 189.
Edit. Aberdeen, 1848).
}.

Способи приміщувати вартість так, щоб вона функціонувала як капітал,
щоб давала своєму власникові додаткову вартість, так само різні
і так само різноманітні, як і сфери приміщення капіталу. Це є питання
про різні галузі продукції, куди можна вкласти капітал. Але питання, так
зформульоване, поширюється далі. Воно захоплює й питання про те, як
вартість, навіть, коли вона не вкладена в продуктивний капітал, може для
її власника функціонувати, напр., як процентодайний капітал, купецький
капітал тощо. Отже, тут ми безмежно віддалились від справжнього предмету
аналізи, від питання: як розподіл \so{продуктивного} капіталу на
його різні елементи впливає на його оборот, незалежно від різних сфер
його приміщення.

А.~Сміс безпосередньо по цьому каже: „Насамперед його можна застосувати
в сільському господарстві, мануфактурі або на закуп благ і дальший
продаж з зиском“\footnote*{
„First, it may be employed in raising, manufacturing, or purchasing goods and
selling them again with a profit“.
}. А.~Сміс каже тут лише те, що капітал можна
застосувати в сільському господарстві, мануфактурі й торговлі. Отже, він
каже лише про різні сфери приміщення капіталу, і між іншим про такі, де,
як у торговлі, капітал не ввіходить у безпосередній процес продукції, тобто
не функціонує як продуктивний капітал. Тим самим він покидає той грунт,
що на ньому стояли фізіократи, визначаючи відмінності різних частин
\parbreak{}  %% абзац продовжується на наступній сторінці
