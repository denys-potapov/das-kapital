\parcont{}  %% абзац починається на попередній сторінці
\index{ii}{0138}  %% посилання на сторінку оригінального видання
не різні самостійні вкладання капіталу, а різні частини того самого продуктивного
капіталу, які в різних сферах приміщення становлять різні
частини сукупної вартости цього капіталу. Отже, це — ріжниці, що випливають
з розподілу самого продуктивного капіталу відповідно до
обставин і тому мають силу лише для цього останнього. Але цьому знову
суперечить та обставина, що торговельний капітал, як виключно обіговий,
протиставиться основному, бо сам Сміс каже: „капітал торговця є
цілком обіговий капітал“. А справді це — капітал, що функціонує лише
в межах сфери циркуляції, і як такий він взагалі протистоїть продуктивному
капіталові, вкладеному в процес продукції, але саме тому його не
можна протиставляти як поточну (обігову) складову частину продуктивного
капіталу основній складовій частині продуктивного капіталу.

В прикладах, що їх наводить Сміс, він визначає, як основний капітал,
знаряддя праці; як обіговий капітал — ту частину капіталу, що витрачена
на заробітну плату й сировинний матеріял, зараховуючи сюди й допоміжні
матеріяли, „оплачувані із зиском в ціні продуктів“ („repaid with а
profit by the price of work“).

Отже, насамперед, за вихідний пункт тут є лише різні елементи процесу
праці: робоча сила (праця) й сировинний матеріял на одному боці,
знаряддя праці — на другому. Але все це є складові частини капіталу, бо
в них вкладено суму вартости, що має функціонувати як капітал. Остільки
це є речові елементи, способи буття продуктивного капіталу, тобто
капіталу, що функціонує в продукційному процесі. Чому ж одна частина
зветься основною? Тому що „деякі частини капіталу мусять бути фіксовані
в засобах праці“ („some parts of the capital must be fixed in the
instruments of trade“). Але друга частина теж є фіксована в заробітній
платі й сировинному матеріялі. Тимчасом, машини і „знаряддя праці\dots{} та
подібні речі\dots{} дають дохід або зиск, не змінюючи власника, не циркулюючи
далі. Тому такі капітали можна назвати основними капіталами у
власному значенні цього слова“.

Візьмімо, напр., гірничу справу. Сировинного матеріялу тут зовсім не
застосовується, бо предмет праці, прим., мідь, є продукт природи, що
його треба лише видобути за допомогою праці. Мідь, що її лише треба
видобути, — продукт процесу, що потім циркулює як товар, зглядно як
товаровий капітал, не становить жодного елемента продуктивного капіталу.
Жодна частина його вартости не вкладена в нього. З другого боку,
інші елементи продукційного процесу, робоча сила й допоміжні матеріяли,
як от вугілля, вода і~\abbr{т. ін.} так само не входять речово в продукт.
Вугілля зуживається цілком, і тільки його вартість ввіходить у продукт,
цілком так само, як частина вартости машин тощо ввіходить у продукт.
Нарешті, робітник лишається так само самостійним проти продукту, міді,
як і машина. Тільки вартість, спродукована його працею, є тепер складова
частина вартости міді. Отже, в цьому прикладі жодна з складових
частин продуктивного капіталу не змінює власника (masters) і жодна з
них не циркулює далі, бо жодна з них не ввіходить речово в продукт.
Де ж тут обіговий капітал? Згідно з власним визначенням А. Сміса довелось
\index{ii}{0139}  %% посилання на сторінку оригінального видання
би визнати, що ввесь капітал, застосований на розробку мідних
копалень, є лише основний капітал.

Візьмімо, навпаки, іншу промисловість, яка застосовує сировинний
матеріял, що становить субстанцію продукту, і допоміжні матеріяли, які
своєю речовиною, а не лише вартістю — як от вугілля, що йде на опалення
— увіходять у продукт. Разом з продуктом, напр., пряжею, сировинний
матеріял, що з нього складається продукт, напр., бавовна, змінює
власника й переходить з процесу продукції в процес споживання. Але
поки бавовна функціонує як елемент продуктивного капіталу, власник не
продає її, а обробляє, наказує робити з неї пряжу. Власник не випускає
її з рук, або, вживаючи грубофалшивого тривіяльного вислову Смісового,
власник не здобуває жодного зиску, „коли продукт відокремлюється від
нього, коли змінюється його хазяїн або коли він циркулює“ (by parting
with it, by its changing masters, or by circulating it). Він так само мало
пускає в циркуляцію свої матеріяли, як і свої машини. Вони фіксовані
в продукційному процесі цілком так само, як прядільні машини та фабричні
будівлі. Частина продуктивного капіталу мусить навіть бути завжди
фіксована в формі вугілля, вовни тощо, так само, як і в формі засобів
праці.

Ріжниця лише та, що бавовну, вугілля та ін., потрібні, напр., для
щотижневої продукції пряжі, завжди цілком зужитковується на продукцію
тижневого продукту, і, значить, їх треба замінювати на нові екземпляри
бавовни, вугілля, тощо; отже, ці елементи продуктивного капіталу,
хоч вони лишаються тотожні своїм родом, постійно складаються з нових
екземплярів того самого роду, тимчасом як та сама поодинока прядільна
машина, та сама поодинока фабрична будівля й далі беруть участь
у цілому ряді повторюваних тижневих процесів продукції, не заміщуючись
на інші екземпляри того самого роду. Як елементи продуктивного
капіталу, всі його складові частини завжди фіксовані в процесі
продукції, бо без них він не може відбуватися. І всі елементи продуктивного
капіталу, основні й поточні, як продуктивний капітал, однаково
протистоять капіталові циркуляції, тобто товаровому капіталові й грошовому
капіталові.

Так само стоїть справа й щодо робочої сили. Частина продуктивного
капіталу завжди мусить бути фіксована в ній, і той самий капіталіст
протягом більш або менш довгого часу застосовує ті самі тотожні
поміж себе робочі сили, на зразок того, як застосовує ті самі машини.
Ріжниця між ними й машинами тут не в тому, що машину купується раз
назавжди (хоч цього не буває, коли, напр., за неї сплачують рати), а
робітника не на завжди, а в тому, що праця, яку витрачає робітник,
цілком входить у вартість продукту, тимчасом як вартість машини — лише
частинами.

Сміс плутає різні визначення, коли він про обіговий капітал, протилежно
до основного, каже таке: „Капітал, застосований таким способом,
не дає своєму власникові доходу або зиску, поки лишається в його посіданні
або зберігає ту саму форму“. Він ставить на один рівень ту лише
\index{ii}{0140}  %% посилання на сторінку оригінального видання
формальну метаморфозу товару, що її пророблює продукт, товаровий
капітал, в сфері циркуляції, та що упосереднює переміщення товару,
з рук до рук, і ту речову метаморфозу, що її пророблюють різні елементи
продуктивного капіталу протягом процесу продукції. Перетворення
товару на гроші й грошей на товар, купівлю й продаж, він, не довго
думаючи, сплутує тут з перетворенням елементів продукції на продукт.
Його приклад для обігового капіталу є купецький капітал, що перетворюється
з товару на гроші, з грошей на товар; це зміна форми $Т — Г — Т$,
належна до товарової циркуляції. Але ця зміна форми в межах циркуляції
має для діющого промислового капіталу те значення, що товари, на
які зворотно перетворюються гроші, є елементи продукції, засоби праці
й робоча сила; отже, те значення, що вона упосереднює безперервність
функціонування промислового капіталу, процес продукції, як безперервний
процес або як процес репродукції. Вся ця зміна форм відбувається
в циркуляції; саме вона упосереднює справжній перехід товарів з
одних рук до інших. Навпаки, метаморфози, що їх перебігає продуктивний
капітал в своєму процесі продукції, є метаморфози, належні до
процесу праці, доконечні, щоб перетворити елементи продукції на
згаданий продукт. А. Сміс спиняється на тому, що одна частина засобів
продукції (власне засоби праці) функціонує в процесі праці (що він
неправильно висловлює, кажучи: „дає зиск їхньому власникові“ — yields а
pronfit to their master), не змінюючи своєї натуральної форми, зношується
лише поступінно, тимчасом як друга частина, матеріяли, змінюється, і
саме в наслідок своєї зміни виконує вона своє призначення як засіб продукції.
Але це різне поводження елементів продуктивного капіталу в
процесі праці становить лише вихідний пункт ріжниці між основним і неосновним
капіталом, а не саму цю ріжннцю, і це видно вже з того, що
таке поводження однаково існує за всіх способів продукції, капіталістичних
і некапіталістичних. Але цьому різному речовому поводженнювідповідає
віддача вартости продуктові, а цій віддачі знову таки відповідає
заміщення вартости за допомогою продажу продукту; і лише це
заміщення утворює ту ріжницю. Капітал, отже, є основний не тому, що
його зафіксовано в засобах праці, а тому, що частина його вартости,
вкладеної в засоби праці, лишається фіксована в них, тимчасом як друга
частина циркулює як складова частина вартости продукту.

„Коли він (капітал) застосовується для того, щоб утворити в майбутньому
зиск, то він мусить утворити цей зиск, або перебуваючи у
нього (власника), або одходячи від нього. В першому разі це основний,
в другому — обіговий капітал“\footnote*{
„If it (the stock) is employed in procuring future profit, it must procure this
profit by staying with him (the employer), or by going from him. In the one case
it is a fixed, in the other it is a circulating capital“ (p. 189).
}.

Тут насамперед впадає в очі грубо емпіричне — перейняте з уявлення
звичайного капіталіста — уявлення про зиск, яке цілком суперечить ліпшому
езотеричному поглядові А. Сміса. В ціні продукту заміщується і ціну
матеріялів, і ціну робочої сили, але так само і ту частину вартости знарядь
\index{ii}{0141}  %% посилання на сторінку оригінального видання
праці, що переходить на продукт в наслідок зношування знарядь
праці. З цього заміщення ще зовсім не утворюється зиск. Залежно від
того, чи заміщується в наслідок продажу продукту авансовану на його
продукцію вартість цілком чи частинами, разом чи поступінно, може змінитися
лише спосіб і час заміщення; але це ніяк не може перетворити
спільне обом випадкам заміщення вартости на утворення додаткової вартости.
В основі маємо тут звичайне уявлення, що додаткова вартість —
тому що її реалізується лише через продаж продукту, через його циркуляцію,
— виникає лише з продажу, з циркуляції. А справді різні
способи постання зиску є тут лише неправильні вислови того, що різні
елементи продуктивного капіталу відіграють різну ролю, неоднаково
функціонують в процесі праці як продуктивні елементи. Нарешті, ця
ріжниця висновується в нього не з процесу праці, зглядно процесу зростання
вартости, не з функції самого продуктивного капіталу, а повинна
мати лише суб’єктивне значення для поодинокого капіталіста, що для
нього одна частина капіталу корисна цим, а друга — тим способом.

Навпаки, Кене висновує ці ріжниці з самого процесу репродукції
та його доконечности. Для того, щоб цей процес був безперервний,
вартість річних авансів мусить цілком заміщуватись з вартости річного
продукту, і навпаки — вартість основного капіталу мусить заміщуватись
лише частинами, так що лише протягом ряду років, прим., десятиліття,
її цілком заміщується, а тому й репродукується цілком (заміщується новими
екземплярами того самого роду). А. Сміс, отже, робить великий
крок назад порівняно з Кене.

Таким чином, для визначення основного капіталу А. Смісові не лишається
зовсім нічого іншого, як сказати, що це — засоби праці, які, протилежно
до продуктів, що їх утворенню вони допомагають, не змінюють
своєї форми в процесі продукції та функціонують далі в продукції, поки
зносяться. При цьому забувають, що всі елементи продуктивного капіталу
в своїй натуральній формі (як засоби праці, матеріяли й робоча сила)
постійно протистоять продуктові, і продуктові, що циркулює як товар,
і що ріжниця між частиною, яка складається з матеріялів та робочої
сили, і частиною, яка складається з засобів праці, лише в тому, що
робочу силу завжди купується наново (а не на ввесь час її існування,
як купується засоби праці), тимчасом як у процесі праці функціонують
не ті самі тотожні, а завжди нові екземпляри матеріялів того самого роду.
Разом з тим постає ілюзія, ніби вартість основного капіталу не циркулює,
хоч А. Сміс звичайно зазначав раніше, що зношування основного
капіталу ввіходить як частина в ціну продуктів.

При визначенні обігового капіталу як протилежности до основного,
не пояснюється, що обіговий капітал має цю протилежність лише як та
складова частина продуктивного капіталу, яка мусить цілком заміститись
з вартости продукту й тому мусить цілком брати участь у його
метаморфозах, тимчасом як із основним капіталом цього немає. Замість
пояснити це, А. Сміс сплутує обіговий капітал з тими формами, що їх
набирає капітал, переходячи зі сфери продукції в сферу циркуляції, —
\parbreak{}  %% абзац продовжується на наступній сторінці
