
Але, подруге, при перелічуванні речей, що з них складається основний
і обіговий капітал, виразно виявляється, що А.~Сміс сплутує ріжницю
між основними й поточними складовими частинами капіталу, яка має
силу й рацію лише щодо продуктивного капіталу (капіталу в його
продуктивній формі), з ріжницею між продуктивним капіталом і формами,
властивими капіталові в процесі його циркуляції, тобто формами: товаровий
капітал і грошовий капітал. Він каже в тому самому місці (стор.
187, 188):

„Обіговий капітал складається\dots{} з харчових засобів, матеріялів
та різних готових виробів, що є в руках осіб, які ними торгують, і з
грошей, потрібних для їх циркуляції та розподілу і~\abbr{т. ін.}“\footnote*{
\textenglish{The circulating capital consists\dots{} of the provisions, materials, and finished work
of all kinds that are in the hands of their respective dealers, and of the money
that is necessary for circulating and distributing them etc.}“.
} — А~справді,
коли ми придивимось ближче, то побачимо, що тут, протилежно до вищенаведеного,
обіговий капітал знову таки ототожнюється з товаровим
капіталом і грошовим капіталом, отже, з двома формами капіталу, що зовсім
не належать до процесу продукції, що становлять не обіговий (поточний)
капітал протилежно до основного, а капітал циркуляції протилежно до продуктивного
капіталу. Тільки \so{поряд} капіталу циркуляції фігурують потім
знову складові частини продуктивного капіталу, авансовані на матеріяли
(сировинний матеріял або напівфабрикати) і справді введені в процес продукції.
А.~Сміс каже:
\index{ii}{0146}  %% посилання на сторінку оригінального видання

„Третя й остання з трьох частин, що на них природно поділяється
ввесь капітал суспільства, є обіговий капітал, і він відзначається тим, що
дає дохід лише циркулюючи і змінюючи власника. Він також складається
з чотирьох частин: поперше, з грошей\dots{}“ (Але гроші ніколи не
бувають формою продуктивного капіталу, що функціонує в процесі продукції.
Вони завжди є лише одна з форм, що їх набирає капітал в процесі
своєї циркуляції); „подруге, з запасу харчів, що його має різник,
торговець худобою, фармер\dots{} і що з продажу його вони сподіваються
мати зиск\dots{} нарешті, почетверте, з продуктів, які хоч виготовлені й цілком
викінчені, але все ще перебувають в руках купця або мануфактуриста“.
— І: „потретє, з матеріялів, або зовсім сирових, або більш-менш
оброблених, з одягу, меблів, будівель, яким ще не надано цих трьох
форм, але які лишаються в руках вівчарів, мануфактуристів, дрібних
крамарів, торговців сукном, торговців лісоматеріялами, теслярів, столярів,
мулярів, тощо“\footnote*{
\dots{} The third and last of the three portions into which general stock ot the
society naturally divides itself, is the circulating capital, of which the characteristic
is, that it affords a revenue only by circulating or changing masters. This is
composed likewise of four parts: first, of the money\dots{} secondly, of the stock of
provisions which are in the posession of the butcher, the grazier, the farmer\dots{} and
from the sale of which they expect to derive a profit\dots{} Thirdly, of the materials,
whether altogether rude or more or less manufactured, of clothes, furniture, and
building, which are not yet made up into any of those three shapes, but which remain
in the hands of the growers, the manufacturers, the mercers and drapers, the timbermerchants, the
carpenters and joiners, the brickmakers etc\dots{} Fourthly and lastly, of
the work which is made up and completed, but which is still in the hands of the
merchant or manufacturer“.
}.

В Пунктах 2 і 4 немає нічого, крім продуктів, що як такі виштовхнуті
з процесу продукції й мають бути продані, коротко кажучи,
це — продукти, які функціонують тепер як товари, отже, зглядно як товаровий
капітал, значить, мають таку форму й таке місце в процесі, що
\parbreak{}  %% абзац продовжується на наступній сторінці
