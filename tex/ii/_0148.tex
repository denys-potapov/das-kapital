\parcont{}  %% абзац починається на попередній сторінці
\index{ii}{0148}  %% посилання на сторінку оригінального видання
він вийшов, як, напр., вугілля в продукції вугілля, — навіть і тут саме
частина продукту — вугілля, призначена на продаж, не являє ані поточного,
ані основного капіталу, а є вона товаровий капітал.

З другого боку, продукт може мати таку споживну форму, яка робить
його цілком непридатним для того, щоб становити якийнебудь елемент
продуктивного капіталу, чи то як матеріял праці, чи то як засіб
праці. Так, напр., деякі засоби існування. А проте, для своїх продуцентів
він все ж є товаровий капітал, носій вартости так основного, як поточного
капіталу, — першого або другого залежно від того, чи цілком,
чи частинами треба заміщувати вжитий для його продукції капітал, чи
цілком чи частинами цей капітал переносить свою вартість на продукт.

У Сміса в пункті 3 фігурує сировинний матеріял (сировина, напівфабрикат,
допоміжний матеріял), з одного боку, не як складова частина,
що вже ввійшла в продуктивний капітал, а справді лише як особливий ґатунок
споживних вартостей, що з них взагалі складається суспільний
продукт, як особливий ґатунок товарової маси, поряд з переліченими в
пунктах 2 і 4 іншими речовими складовими частинами, засобами існування
тощо. З другого боку, і самі сирові матеріяли наведено в нього, в
усякому разі, як елементи, що ввійшли в продуктивний капітал, а тому й
як елементи останнього, які є в руках продуцента. Плутанина виявляється в
тому, що, почасти, ці сирові матеріяли розглядається як діющі в руках
продуцента („в руках торговців худобою, мануфактуристів“ тощо), з
другого боку, як перейшлі до рук торговців („дрібних крамарів, торговців
сукном, лісоматеріялами“ і~\abbr{т. ін.}), де вони є тільки товаровий капітал, а
не складові частини продуктивного капіталу.

А.~Сміс, перелічуючи тут елементи обігового капіталу, в дійсності цілком
забуває про ріжницю між основним і поточним капіталом, яка має
силу тільки для продуктивного капіталу. Навпаки, товаровий капітал і
грошовий капітал, тобто обидві форми капіталу, що належать до процесу
циркуляції, він протиставить продуктивному капіталові, та й то несвідомо.
Варто, нарешті, уваги, що А.~Сміс, перелічуючи складові частини обігового
капіталу, забуває про робочу силу. У цьому є дві причини.

Ми щойно бачили, що, коли облишити осторонь грошовий капітал,
обіговий капітал є в нього лише друга назва товарового капіталу. Але
оскільки робоча сила циркулює на ринку, вона не є капітал, не є будьяка
форма товарового капіталу. Вона взагалі не капітал; робітник не є
капіталіст, хоч він і виносить на ринок товар, а саме — свою власну шкуру.
Лише після того, як робочу силу вже продано і введено в продукційний
процес, — отже, лише після того, як вона перестала циркулювати
як товар, — вона стає складовою частиною продуктивного капіталу: змінним
капіталом як джерело додаткової вартости, поточною складовою
частиною продуктивного капіталу щодо обороту витраченої на неї капітальної
вартости Сміс сплутує тут поточний капітал з товаровим капіталом,
а тому й не може він підвести робочу силу під свою рубрику обігового
капіталу. Тому змінний капітал виступає в нього в формі тих
\parbreak{}  %% абзац продовжується на наступній сторінці
