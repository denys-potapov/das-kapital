\parcont{}  %% абзац починається на попередній сторінці
\index{ii}{0149}  %% посилання на сторінку оригінального видання
товарів, що їх робітник купує на свою заробітну плату, в формі засобів
існування. В цій формі капітальна вартість, витрачена на заробітну плату,
належить, за Смісом, до обігового капіталу. Але те, що вводиться в
продукційний процес, є робоча сила, є сам робітник, а не засоби існування,
що ними робітник підтримує своє життя. А проте, ми бачили
(кн. І, розд. XXI), що, з суспільного погляду, репродукція самого робітника
його особистим споживанням теж належить до процесу репродукції
суспільного капіталу. Але цього не можна сказати про поодинокий,
замкнений в собі продукційний процес, що його ми досліджуємо тут.
Надбані й корисні вмілості, acquired and useful abilities (ст. 187), що їх
Сміс подає під рубрикою основного капіталу, в дійсності становлять
складові частини поточного капіталу, оскільки це є abilities найманого
робітника й оскільки робітник продає свою працю разом з своїми abilities.

Велика помилка Сміса в тому, що він усе суспільне багатство поділяє
на: 1) фонд безпосереднього споживання, 2) основний капітал,
3) обіговий капітал. Згідно з цим усе багатство треба було б розподілити
на 1) фонд споживання, що зовсім не становить частини діющого
суспільного капіталу, хоч поодинокі частини його завжди можуть функціонувати
як капітал, і 2) капітал. Одна частина багатства функціонує
таким чином як капітал, друга частина — як некапітал або як фонд
споживання. І для всякого капіталу тут виявляється неминучість бути
або основним або поточним, так само, як кожен з ссавців неминуче мусить
бути або самцем або самицею. Ми однак бачили, що протилежність
між основним і обіговим капіталом має силу тільки для елементів \so{продуктивного
капіталу}, і що, значить, поряд них є ще дуже значна
маса капіталу — товаровий капітал і грошовий капітал — яка перебуває в
такій формі, що в ній \so{не може} вона бути ні основним, ні поточним
капіталом.

А що за винятком тієї частини продуктів, яку — в натуральній формі —
безпосередньо, без продажу й купівлі, знову вживають самі поодинокі
капіталістичні продуценти як засоби продукції, вся маса продуктів суспільної
продукції — на капіталістичній основі — циркулює на ринку як
товаровий капітал, то очевидно, що з товарового капіталу треба вилучити
так основні й поточні елементи продуктивного капіталу, як і всі
елементи споживного фонду. Фактично це значить, що засоби продукції,
як і засоби споживання, на основі капіталістичної продукції виступають
спочатку як товаровий капітал, хоч вони й мали б призначення в
дальшому служити як засіб споживання або як засоби продукції; так само
навіть робоча сила перебуває на ринку як товар, хоч і не як товаровий
капітал.

Відси така нова плутанина в А.~Сміса. Він каже:

„З цих чотирьох частин (cilculating капіталу, тобто капіталу в його
належних до процесу циркуляції формах товарового капіталу й грошового
капіталу — дві частини, що перетворюються на чотири частини через
те, що Сміс відрізняє складові частини товарового капіталу знову на
основі речових ознак) три: харчові засоби, матеріяли та готові вироби,
\parbreak{}  %% абзац продовжується на наступній сторінці
