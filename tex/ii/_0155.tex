\parcont{}  %% абзац починається на попередній сторінці
\index{ii}{0155}  %% посилання на сторінку оригінального видання
далі; для них вирішальне визначення тієї частини капіталу, яку витрачено
на робочу силу, не лише в тому, що вона є обіговий капітал протилежно
до основного, — вони основне визначення обігового капіталу зводять
на те, що його витрачається на засоби існування для робітників.
Відси природно випливає вчення про робочий фонд, складений з доконечних
засобів існування, як про дану величину, яка, з одного боку,
фізично обмежує пайку робітників у суспільному продукті, а, з другого
боку, цілком мусить бути витрачена на закуп робочої сили.

\section{Теорії про основний та обіговий капітал. Рікардо}

Рікардо наводить ріжницю між основним і обіговим капіталом тільки
для того, щоб показати винятки з правила вартости, а саме ті випадки,
коли норма заробітної плати впливає на ціни. Про це ми говоритимемо
лише в III томі.

Однак основна неясність вже з самого початку виявляється в такому
зіставленні. „Ця ріжниця в ступені довготривалости основного капіталу,
і ця змінність відношень, що в них обидві відміни капіталу можуть бути
комбіновані“\footnote{
„This difference in the degree of durability of fixed capital, and this variety
in the proportions in which the two sorts of capital may be combined.“ — Principles,
p. 25.
}.

Коли ми запитаємо, які саме ці дві відміни капіталу, то виявиться ось
що: „Так само різно можуть комбінуватись відношення між капіталом,
призначеним на утримання праці, і капіталом, витраченим на знаряддя,
машини і будівлі“\footnote{
„The proportions, too, in which the capital that is to support labour, and the
capital that is invested in tools, machinery and buildings, may be various by combined“
— 1. c.
}. Отже, основний капітал \deq{} засобам праці, і обіговий
капітал \deq{} капіталові, витраченому на працю. Капітал, призначений на
утримання праці, — уже це є плаский вислів, запозичений в А.~Сміса.
Обіговий капітал сплутується тут, з одного боку, із змінним капіталом,
тобто частиною продуктивного капіталу, витраченою на працю. Але,
з другого боку, тому що цю протилежність узято не з процесу зростання
вартости — сталий і змінний капітал, — а з процесу циркуляції (стара Смісова
плутанина), то постають визначення, подвійно хибні.

Поперше. Ріжниці в ступені довготривали основного капіталу і
ріжниці в складі капіталу, який складається із сталої та змінної частини,
тут розглядається як рівнозначні. Але остання ріжниця визначає ріжницю
в продукції додаткової вартости, навпаки, перша, оскільки зважається на
процес самозростання вартости, стосується лише до того способу, в який
дану вартість переноситься із засобів продукції на продукт; а оскільки
береться до уваги процес циркуляції, вона стосується лише до періоду
\parbreak{}  %% абзац продовжується на наступній сторінці
