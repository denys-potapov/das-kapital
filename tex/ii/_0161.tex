\parcont{}  %% абзац починається на попередній сторінці
\index{ii}{0161}  %% посилання на сторінку оригінального видання
капіталові як сталий капітал — це з погляду процесу зростання вартости.
Або, коли тут мова повинна бути про речову ріжницю, оскільки вона
впливає на процес циркуляції, то справа така: з природи вартости, яка є
не що інше, як зречевлена праця, і з природи діющої робочої сили, яка
є не що інше, як праця, що зречевлює себе, випливає, що робоча сила
протягом періоду її функціонування постійно утворює вартість і долярову
вартість; і що те, що на боці робочої сили виявляється як рух, як
утворення вартости, на боці її продукту виявляється у формі спокою,
як уже утворена вартість. Коли робоча сила вже діяла, то капітал не
складається вже більше з робочої сили на одному боці, із засобів продукції
на другому. Капітальна вартість, витрачена на робочу силу, є тепер
вартість, що її (+ додаткову вартість) долучено до продукту. Щоб
повторити процес, треба продати продукт і на вторговані гроші знову й
знову купувати робочу силу і вводити її в продуктивний капітал. Це
надає тоді частині капіталу, витраченій на робочу силу, так само, як і частинам
його, витраченим на матеріял праці тощо, характер обігового капіталу,
протилежно до того капіталу, що лишається закріплений у засобах праці.

Коли, навпаки, другорядне визначення обігового капіталу, спільне
йому з частиною сталого капіталу (сировинними й допоміжними матеріяламн)
— саме те визначення, що вартість, витрачену на обіговий капітал,
цілком переноситься на продукт, в продукції якого його зуживається, а
не поступінно й частинами, як в основного капіталу, що вартість ця,
отже, мусить цілком заміститися через продаж продукту, — перетворити
на посутню характеристику частини капіталу, витраченої на робочу силу,
то й частина капіталу, витрачена на заробітну плату, речово мусить
складатися не з діющої робочої сили, а з речових елементів, що їх робітник
купує на свою плату, отже, з частини суспільного товарового капіталу,
яка ввіходить у споживання робітника — з засобів існування.
Основний капітал складається при такому погляді на справу з засобів
праці, що зношуються повільніше, а тому й доводиться їх рідше відновлювати,
а капітал, витрачений на робочу силу, з засобів існування, що
їх треба заміщувати швидше.

Однак межі швидшої та повільнішої зношуваности стираються.

„Харч і одяг що їх зуживає робітник, будівлі, де він працює, знаряддя,
що допомагають йому в роботі, всі ці речі з своєї природи минущі.
Але є величезна ріжниця в часі, що протягом його зберігаються
ці різні капітали: парова машина зберігається довший час, ніж корабель,
корабель — довший час, ніж одяг робітника, одяг робітника знову таки
довший час, ніж харч, що його він споживає“\footnote{
The food and clothing consumed the labourer, the buildings in which he
works, the implements with which his labour is assisted, are all of a perishable
nature. There is, however, a vast difference in the time for which these different
capitals will endure: a steam-engine will last longer than a ship, a ship than the
clothing of the labourer, and the clothing of the labourer longer than the food which
he consumes“. (Ricardo, etc., p. 27).
}.
