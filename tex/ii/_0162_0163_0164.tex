
\index{ii}{0162}  %% посилання на сторінку оригінального видання
Рікардо забуває при цьому будинок, де живе робітник, його меблі,
знаряддя його споживання, напр., ножі, виделки, посуд і~\abbr{т. ін.}, що всі
своєю довготривалістю мають той самий характер, як і засоби праці. Ті
самі речі, ті самі кляси речей виступають тут як засоби споживання,
там — як засоби праці.

Ріжниця, як її висловлює Рікардо, ось у чому: „Відповідно до того,
чи зношується капітал швидко й потребує частої репродукції, чи зуживається
його повільно, його клясифікують як обіговий, або як основний
капітал“\footnote{
„According as capital is rapidly perishable and requires to be frequently
reproduced, or is of slow consumption it is classed under the heads of circulating,
or fixed capital“ (Ricardo, 1. c.).
}.

До цього він робить помітку: „Розподіл непосутній, що в ньому, крім
того, немає змоги точно провести розмежувальну лінію“\footnote{
„А division not essential, and in which the line ot demarcation cannot be
accurately drawn“ (Ricardo, 1. c.).
}.

Таким чином ми щасливо дійшли знову до фізіократів, що в них
ріжниця між avances annuelles і avances primitives була ріжницею в часі
споживання, а, значить, і в часі репродукції ужитого капіталу. Тільки те,
що в них виражає важливий для суспільної продукції феномен і в Tableau
économique подано також у зв’язку з процесом циркуляції, тут стає
суб’єктивним і, як каже сам Рікардо, зайвим відрізненням.

Якщо частина капіталу, витрачена на працю, відрізняється від частини
капіталу, витраченої на засоби праці, лише періодом своєї репродукції,
а тому й часом своєї циркуляції; якщо одна частина складається з засобів
існування так само, як друга з засобів праці, так що останні відрізняються
від перших лише ступенем швидкости зношування й при цьому
перші й собі мають різні ступені тривалости, — коли це так, то differentia
specifica\footnote*{
Характеристичні, відзначні риси. \emph{Ред.}
} між капіталом, витраченим на робочу силу, і капіталом, витраченим
на засоби продукції, звичайно, стирається.

Це цілком суперечить Рікардовій теорії вартости так само, як і його
теорії зиску, що фактично є теорія додаткової вартости. Він розглядає
ріжпицю між основним і обіговим капіталом взагалі лише остільки
оскільки різні пропорції обох, при рівновеликих капіталах, впливають
в різних галузях підприємств на закон вартости, а саме, він розглядає,
як, в наслідок цих обставин, підвищення або зниження заробітної плати
впливає на ціни. Але навіть в обмежених рямцях цього досліду він, сплутуючи
основний та обіговий капітал із сталим та змінним, робить величезні
помилки і справді будує свій дослід на цілком хибній основі. А
саме: 1) оскільки частину капітальної вартости, витрачену на робочу
силу, підводиться під рубрику обігового капіталу, неправильно висновується
визначення самого обігового капіталу і особливо ті обставини,
що підводять частину капіталу, витрачену на працю, під цю рубрику; 2)
сплутується те визначення, що, згідно з ним, частина капіталу, витрачена
\index{ii}{0163}  %% посилання на сторінку оригінального видання
на працю, є змінний капітал, і те, що, згідно з ним, вона є обіговий
капітал протилежно до основного.

Вже з самого початку очевидно, що визначення капіталу, витраченого
на робочу силу, як обігового або поточного, є другорядне визначення,
в якому зникають його differentia specifica в продукційному процесі;
бо, з одного боку, при такому визначенні капітали, витрачений на працю
й витрачений на сировинний матеріял і~\abbr{т. ін.}, вважається за рівнозначні;
рубрика, що ототожнює частину сталого капіталу зі змінним, цілком ігнорує
differentia specifica змінного капіталу протилежно до сталого. З другого
боку, частини капіталу, витрачені на працю і на засоби праці, хоч
і протиставиться одна одній, але зовсім не в тому розумінні, що вони
цілком різним способом входять у продукцію вартости, а лише в тому
розумінні, що обидві вони переносять на продукт свою дану вартість
тільки в різні переміжки часу.

В усіх цих випадках ідеться тільки про те, як дана вартість, що її
витрачається на процес продукції товару — хоч то буде заробітна плата,
ціна сировинного матеріялу або ціна засобів праці, — переноситься на
продукт, а, значить, і як вона циркулює за допомогою продукту і в наслідок
продажу його повертається до свого вихідного пункту, або як
покривається її. Єдина ріжниця тут у цьому „\so{як}“, в особливому способі
перенесення, а, значить, і циркуляції цієї вартости.

Чи виплачується кожного разу заздалегідь визначену контрактом ціну
робочої сили грішми, чи засобами існування — це нічого не змінює в її
характері, а саме в тому, що вона є певна дана ціна. А проте, при виплаті
заробітної плати грішми цілком очевидно, що самі гроші не входять
у процес продукції, як входять засоби продукції, що в них не лише
вартість, а й сама речовина входить у продукційний процес. А коли ж,
навпаки, засоби існування, що їх робітник купує на свою заробітну плату,
безпосередньо підводиться як речову форму обігового капіталу під
одну рубрику з сировинними матеріялами тощо й протиставиться засобам
праці, то це надає справі іншого вигляду. Коли вартість цих речей,
тобто засобів продукції, в процесі праці переноситься на продукт, то
вартість тих других речей, тобто засобів існування, знову з’являється в
робочій силі, що їх спожила, і через функціонування робочої сили її знову
таки переноситься на продукт. І в тому, і в другому разі однаково
йдеться про просту повторну появу в продукті вартостей, авансованих
підчас продукції. (Фізіократи брали це серйозно, а тому не визнавали,
що промислова праця створює додаткову вартість). Напр., Вейленд
пише в цитованому вже місці: „Байдуже, в якій саме формі з’являється
знову капітал\dots{} різні відміни харчу, одягу й житла, потрібні для існування
й добробуту людей, також змінюються. Їх зуживається з плином
часу, і вартість їхня з’являється знову і~\abbr{т. ін.}“. (Elements of Political
Economy, ст. 31, 32). Капітальні вартості, авансовані в формі
засобів продукції й засобів існування для продукції тут однаково знову
з’являються в вартості продукту. Так щасливо досягається перетворення
капіталістичного процесу продукції на цілковиту містерію, а походження
\index{ii}{0164}  %% посилання на сторінку оригінального видання
додаткової вартости, що є в продукті, лишається цілком поза
межами поля зору.

Далі, тут завершується властивий буржуазній політичній економії фетишизм,
що перетворює суспільний, економічний характер, накладуваний
на речі суспільним процесом продукції, на природний, з самої речової
природи цих речей посталий характер. Напр., „засоби праці є основний
капітал“ — схоластичне визначення, що призводить до суперечностей і плутанини.
Цілком так само, як при вивченні процесу праці („Капітал“, книга
І, розділ V) показано, що від тієї ролі, яку в кожному окремому
випадку відіграють речові складові частини у певному процесі праці,
від їхньої функції, цілком залежить те, чи будуть вони виступати як
засіб праці, чи як матеріял праці, або як продукт; цілком так само засоби
праці тільки тоді є основний капітал, коли процес продукції є взагалі
капіталістичний продукційний процес і, значить, засоби продукції
взагалі є капітал, коли вони мають економічну визначеність, суспільний
характер капіталу; і, подруге, вони є основний капітал лише тоді, коли
вони свою вартість переносять на продукт певним способом. Коли цього
немає, вони лишаються засобами праці, не являючи основного капіталу.
Так само допоміжні матеріяли, напр., добриво, коли вони передають
свою вартість тим самим особливим способом, що й більша частина засобів
праці, стають основним капіталом, хоч вони й не є засоби праці.
Тут ідеться не про визначення, що під нього можна підводити речі. Тут
ідеться про певні функції, що їх виражається в певних категоріях.

Коли засобам існування самим по собі при всяких обставинах приписується
властивість бути капіталом, витраченим на заробітну плату, то
ця властивість „підтримувати працю“, to support labour (Рікардо, ст. 25),
стає також характеристичною властивістю цього „обігового“ капіталу.
Отже, виходить, що коли б засоби існування не були „капіталом“, то
вони не підтримували б робочої сили; тимчасом характер капіталу надає
їм саме властивости підтримувати \so{капітал} за допомогою чужої праці.

Якщо засоби існування сами по собі є обіговий капітал — після того
як цей останній перетворився на заробітну плату, — то відси випливає
далі, що величина заробітної плати залежить від відношення між числом
робітників і даною масою обігового капіталу — улюблена засада економістів;
тимчасом як у дійсності маса засобів існування, що її робітник
бере з ринку, і маса засобів існування, що її має капіталіст для власного
споживання, залежить від відношення між додатковою вартістю й ціною
праці.

Рікардо, як і Бартон\footnote{
„Observations on the Circumstances which influences the Condition of the
Labouring Classes of Society. London, 1817“. Відповідне, сюди належне, місце
подано в I кн. „Капіталу“, розд. XXIII, 3, примітка 79.
}, сплутують усюди відношення між змінним і сталим
капіталом із відношенням між обіговим і основним капіталом. Ми побачимо
далі, якої хибности набирає в наслідок цього дослід над нормою зиску.

Далі, Рікардо ототожнює ріжницю між основним і обіговим капіталом із
ріжницями, що постають в процесі обороту підо впливом інших причин. Він
\parbreak{}  %% абзац продовжується на наступній сторінці
