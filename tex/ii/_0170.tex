\parcont{}  %% абзац починається на попередній сторінці
\index{ii}{0170}  %% посилання на сторінку оригінального видання
капітал, може збільшуватись, не зумовлюючи відповідного до цього збільшення
величини авансованого капіталу. Капітал тут треба авансувати на
порівняно довший час, і більші маси капіталу тут зв’язано в формі продуктивного
капіталу.

На нерозвиненіших щаблях капіталістичної продукції підприємства,
що потребують довгого робочого періоду, а, значить, великих витрат капіталу
на довший час, — особливо, коли їх можна здійснити лише у великому
маштабі — провадиться або зовсім не капіталістично, а громадським
або державним коштом, як от будування шляхів, каналів тощо (за старих
часів здебільша шляхом примусової праці, оскільки ми маємо
на увазі робочу силу). Або такі продукти, що на виготовлення їх потрібен
порівняно довший робочий період, лише в незначній частині фабрикується
коштом майна самих капіталістів. Напр., коли будується будинок,
то приватна особа, що для неї його будується, дає частинами аванси
підприємцеві-будівникові. Отже, вона на ділі оплачує будинок частинами
у міру того, як посовується продукційний процес. Навпаки, в добу розвиненої
капіталістичної продукції, коли, з одного боку, чималі маси капіталу
сконцентровані в руках поодиноких осіб, а з другого, поряд поодиноких
капіталістів виступає асоційований капіталіст (акційні товариства),
і разом з тим кредитова справа є розвинута, капіталістичний
підприємець-будівник лише винятково будує на замовлення поодиноких
приватних осіб.

Він робить із цього ґешефт: будує для ринку цілий ряд будинків і
міських кварталів, так само як поодинокі капіталісти роблять ґешефти
із того, що будують як підрядчики залізниці.

Який переворот зумовила капіталістична продукція в лондонському
житлобудівництві, про це кажуть нам свідчення одного підприємця-будівника
перед банківською комісією 1875 року. Як він каже, в його
молодості будинки здебільша будувались на замовлення й витрати оплачувались
підприємцеві поступінно, протягом часу будування, по закінченні
окремих стадій будування. Заради спекуляції будували дуже мало;
підприємці вдавались до цього переважно лише для того, щоб давати
реґулярно працю робітникам і, таким чином, тримати їх усіх разом при
собі. Протягом останніх сорока років все це змінилось. На замовлення
будують вже дуже мало. Кому потрібен новий будинок, вишукує
собі один із збудованих на спекуляцію або із тих, що їх ще лише будується.
Підприємець робить тепер не на замовників, а на ринок; як і
всякий інший промисловий підприємець, він мусить мати на ринку
готові товари. Якщо раніше підприємець одночасно будував на спекуляцію,
може, три або чотири будинки, то тепер йому доводиться купувати
чималу ділянку землі (висловлюючись по-континентальному — орендувати
здебільша на 99 років) і збудувати на ньому 100 або 200 будинків
вдаючись таким чином у підприємство, що переважає в двадцять — п’ятдесят
разів його маєтність. Кошти здобуваються під гіпотеки, і гроші
надходять до розпорядку підприємця в міру того, як посувається будування
окоемих будинків. Коли постає криза, що затримує виплату авансів,
\parbreak{}  %% абзац продовжується на наступній сторінці
