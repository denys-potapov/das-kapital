
\index{ii}{0179}  %% посилання на сторінку оригінального видання
Те саме і в скотарстві. Частина стада (запасу худоби) лишається в продукційному процесі, тимчасом як
другу частину його щорічно продається, як продукт. Лише одна частина капіталу обертається тут
щорічно, цілком так само, як це є з основним капіталом, машинами, робочою худобою тощо. Хоч цей
капітал є капітал закріплений у процесі продукції на довший час, і таким чином він уповільнює оборот
цілого капіталу, все ж він не є основний капітал у категоричному значенні.

Те, що зветься тут запасом — певна кількість живого дерева або худоби — відносно перебуває в процесі
продукції (одночасно як засоби праці й матеріял праці); згідно з природними умовами його
репродукції, при правильному господарстві, чимала частина його мусить завжди перебувати в цій формі.

Подібно впливає на оборот друга відміна запасу, що становить лише потенціяльний продуктивний
капітал, але в наслідок природи господарства мусить нагромаджуватись більшими або меншими масами, а
значить, і авансуватись для продукції на довший час, хоча вона лише помалу ввіходить в активний
процес продукції. Сюди належить, напр., добриво, поки його не вивезено на поле, також зерно, сіно
тощо та інші запаси засобів існування, що увіходять у продукцію худоби. „Чимала частина капіталу
продукції (Betriebskapital) є в господарських запасах. Але останні можуть втратити більше або менше
своєї вартости, скоро не додержано належних запобіжних заходів, потрібних, щоб їх зберегти в доброму
стані; можливо навіть, що в наслідок недостатнього догляду частина запасів продукту й зовсім марно
пропаде для господарства. Ось чому тут потрібен особливо пильний догляд за інбарами, клунями,
пашенними коморами й льохами; так само треба добре замикати приміщення, де зберігаються запаси, а
крім того тримати їх чисто, провітрювати і~\abbr{т. ін.}; збіжжя та овочі, що зберігаються в інбарах, треба
час від часу пересипати, картоплю й буряк захищати від холоду, а також від води й огню“ (Kirchhof,
р. 292). „Обчисляючи власні потреби, особливо те, що потрібно для утримання худоби — при цьому
розподіл цей роблять залежно від урожаю та визначеної мети, — треба мати на увазі не лише
задоволення даної потреби, а також подбати про те, щоб лишався й відповідний запас про непередбачені
випадки. Скоро при цьому виявляється, що цих потреб не можна цілком покрити виробами власного
господарства, то треба насамперед подумати про те, чи не можна цю недостачу покрити іншими
продуктами (сурогатами) або принаймні придбати їх дешево, щоб покрити цю недостачу. Коли, напр.,
виявиться, що не вистачає сіна, то його можна замінити корінниками, додавши соломи. Вазалі при цьому
треба завжди зважувати матеріяльну цінність і ринкову ціну різних продуктів і відповідно до цього
визначати споживання; коли, напр., овес відносно дорогий, а ціни на горох і жито порівняно низькі,
то можна вигідно замінити в кормі коням частину овса на горох або жито, а заощаджений таким чином
овес продати“. (Там же, стор. 300).

Вище, розглядаючи питання про утворення запасів, ми вже зауважили,
\parbreak{}  %% абзац продовжується на наступній сторінці
