\parcont{}  %% абзац починається на попередній сторінці
\index{ii}{0184}  %% посилання на сторінку оригінального видання
засобів комунікації. Який-будь осередок продукції, що мав особливі переваги,
в наслідок того, що він містився на великому шляху або каналі,
тепер опиняється близько однісінького залізничного рукава, який функціонує
з порівняно великими перервами, тимчасом як інший осередок,
що був раніш зовсім осторонь від головних шляхів сполучення, тепер
опиняється у вузловому пункті кількох залізниць. Другий осередок розвивається,
перший занепадає. Отже, зміна в засобах транспорту зумовлює
місцеві відмінності в часі обігу товарів, в умовах купівлі, продажу
тощо, або вона інакше розподіляє вже наявні місцеві відмінності. Важливість
цієї обставини для обороту капіталу виявляється в суперечках між
представниками купців та промисловців різних місцевостей з управлінням
залізниць. (Див., напр., вище цитовану Синю книгу Railway Committee).

Тому всі галузі продукції, що, відповідно до природи своїх продуктів,
розраховані переважно на місцевий збут, як напр., броварні, розвиваються
до велетенських розмірів у головних залюднених центрах.
Швидший оборот капіталу почасти урівноважує тут більше подорожчання
деяких умов продукції, місця під будівлю тощо.

Коли, з одного боку, з поступом капіталістичної продукції розвиток
засобів транспорту й комунікації скорочує час обігу для даної кількости
товарів, то той самий поступ і дана з розвитком засобів транспорту
й комунікації можливість, навпаки, зумовлює доконечність роботи
на чимраз віддаленіші ринки, коротко кажучи, на світовий ринок. Маса
товарів, що перебувають у дорозі, відправлених до віддалених пунктів,
надзвичайно зростає, а тому абсолютно й відносно зростає і та частина
суспільного капіталу, яка постійно протягом довшого часу перебуває в
стадії товарового капіталу, перебуває в періоді обігу. Одночасно зростає
в наслідок цього й та частина суспільного багатства, що, замість безпосередньо
служити засобом продукції, витрачається на засоби транспорту
й зв’язку та на основний і обіговий капітал, потрібний для їхньої
роботи.

Відносний протяг подорожі товару від місця продукції до місця
збуту зумовлює ріжницю не лише в першій частині часу обігу, в часі
продажу, а і в другій частині, в зворотному перетворенні грошей
на елементи продуктивного капіталу, в часі купівлі. Напр., товар відправляють
в Індію. Це триває, припустімо, чотири місяці. Хай час продажу
дорівнює нулеві, тобто товар надсилається на замовлення й гроші
виплачується аґентові продуцента підчас здачі товару. Зворотна пересилка
грошей (форма, з якій їх пересилається, тут не має значення) триває
знову таки чотири місяці. Отже, минає загалом вісім місяців, раніш
ніж той самий капітал має змогу знову функціонувати як продуктивний капітал,
— раніш ніж з ним можна знову розпочати ту саму операцію.
Спричинені таким чином відмінності в обороті становлять одну з матеріяльних
основ для різних кредитових строків, подібно до того, як
морська торговля, напр., у Венеції та Ґенуї взагалі становить одно з
джерел кредитової системи у власному розумінні слова. „Криза 1847~\abbr{р.}
дала банкам і торговим підприємствам того часу змогу скоротити індійські
\index{ii}{0185}  %% посилання на сторінку оригінального видання
та китайські узанції\footnote*{
Узанції (Usance) — строки оплати векселів, що визначаються згідно з місцевими
купецькими звичаями. \emph{Ред.}
} (для часу, потрібного на подорож векселів
між цими країнами та Европою) з десятьох місяців по написанні до
шістьох місяців по поданні; тепер, по двадцятьох роках, коли прискорено
зв’язки й заведено телеграф, постала потреба в дальшому скороченні
з шости місяців по поданні до чотирьох місяців по написанні як
перший крок до чотирьох місяців по поданні. Плавба вітрильного судна
з Калькути до Лондону повз ріг Доброї Надії триває пересічно мало
не 90 днів. Узанція в чотири місяці по поданні дорівнювала б приблизно
150 дням плавби. А теперішня узанція в шість місяців по поданні
дорівнює 210 дням дороги“. („London Economist“, 16 червня 1866). —
Навпаки, „Бразільська узанція все ще визначається в два й три місяці
по поданні, векселі з Антверпену (на Лондон) видається на 3 місяці по
написанні й навіть Менчестер і Бредфорд видають векселі на Лондон на
три місяці й довший час. За мовчазною згодою купцеві дається достатню
змогу реалізувати свій товар, якщо й не раніше, то все ж
близько того часу, коли кінчається строк виданих за товар векселів.
Тому узанція індійських векселів не надмірна. Індійські продукти, що
їх продається в Лондоні здебільша строком платежу через три місяці,
не можна реалізувати, коли зарахувати сюди деякий час на продаж, за
значно коротший час, ніж п’ять місяців, тимчасом як ще п’ять місяців
пересічно минає між закупом їх в Індії та здачею на англійські склади.
Ми маємо тут період в десять місяців, тимчасом як строк виданих за
товар векселів не перевищує семи місяців“. (Там же, 30 червня 1866).
„2 липня 1866 п’ять великих лондонських банків, що переважно мають
зв’язок з Індією та Китаєм, а також паризька Comptoir d’Escompte заявили,
що з 1 січня 1867 року їхні філії та агентства на Сході будуть
купувати й продавати лише векселі, видані не більш як на чотири місяці
по поданні“. (Там же 7 липня 1865~\abbr{р.}). Це зниження однак не мало
успіху й довелось його скасувати (з того часу Суецький канал революціонізував
усе це).

Зрозуміло, коли довшає час обігу товарів, то більшає ризик, що
зміняться ціни на ринку продажу, бо довшає період, що протягом нього
можуть змінитись ціни.

Ріжниця в часі обігу — почасти індивідуальна між різними поодинокими
капіталами тієї самої галузі підприємств, почасти між різними галузями
підприємств залежно від різних узанцій, там, де не виплачується
одразу готівкою, — випливає з різних строків виплати при купівлі та
продажу. Ми не будемо тут докладніше зупинятись на цьому пункті,
важливому для кредитової справи.

Від розміру угод на поставки, а він зростає разом із зростом розмірів
і маштабу капіталістичної продукції, залежать також і ріжниці в часі
обороту. Угода на поставку, як оборудка між продавцем і покупцем, є
операція, що належить до ринку, до сфери циркуляції. Ріжниці, що випливають
\index{ii}{0186}  %% посилання на сторінку оригінального видання
відси, в часі обороту, випливають, отже, з сфери циркуляції,
але вони безпосередньо відбиваються на сфері продукції, і до того ж
незалежно від строків виплат і кредитових відносин, а значить, і при виплаті
готівкою. Напр., вугілля, бавовна, пряжа, тощо є продукти подільні.
Кожний день дає певну кількість готового продукту. Але коли прядун
або власник копалень береться поставити таку масу продуктів, що для
неї потрібен, приміром, чотиритижневий або шеститижневий період послідовних
робочих днів, то відносно до протягу часу, що на нього треба
авансувати капітал, це все одно, якби в цьому процесі праці був заведений
безперервний робочий період в чотири або шість тижнів. Тут звичайно
припускається, що всю замовлену масу продуктів треба доставити
одним заходом, або що її оплатиться лише після того, як її всю доставиться.
Отже, кожен день, взятий окремо, дав свою певну кількість готового
продукту. Але ця готова маса завжди є лише частина тієї маси,
що її треба доставити згідно з угодою. В цьому разі, якщо виготовлена
вже частина замовленого товару не перебуває в процесі продукції, то
вона в усякому разі лежить на складі лише як потенціяльний капітал.

Перейдімо тепер до другого відділу часу обігу — до часу купівлі або
до періоду, що протягом його капітал з грошової форми знову перетворюється
на елементи продуктивного капіталу. Протягом цього періоду
він мусить довший або коротший час лежати в стані грошового капіталу,
а значить, певна частина цілого авансованого капіталу має перебувати
безупинно в стані грошового капіталу, хоч ця частина складається з елементів,
що постійно змінюються. В якомубудь певному підприємстві з
усього авансованого капіталу мусить бути в формі грошового капіталу,
прим., 100\pound{ ф. стерл.} × n, і тимчасом як усі складові частини цих 100\pound{ ф.
стерл.} × n безупинно перетворюються на продуктивний капітал, ця сума
все ж так само завжди знову поповнюється припливом із циркуляції, з реалізованого
товарового капіталу. Отже, певна частина вартости авансованого
капіталу постійно перебуває в стані грошового капіталу, отже, в формі,
що належить не до сфери його продукції, а до сфери його циркуляції.

Ми вже бачили, що подовження часу, зумовлене віддаленістю ринку,
подовження, що протягом його капітал є зв’язаний в формі товарового
капіталу, безпосередньо призводить до запізнення зворотного припливу
грошей, отже, затримує перетворення капіталу з грошового капіталу на
продуктивний.

Далі, щодо закупу товарів, ми бачили (розд. VI), як час купівлі,
більша або менша віддаленість від головних джерел придбання сировинного
матеріялу примушує купувати сировинний матеріял на довші періоди
й зберігати його придатним до вжитку у формі продуктивного запасу, лятентного
або потенціяльного продуктивного капіталу; отже, що така
віддаленість, при тому самому зрештою маштабі продукції, збільшує масу
капіталу, що його доводиться авансувати одним заходом, і час, що на
нього доводиться авансувати його.

Подібно впливають в різних галузях підприємства періоди — більш
або менш протяжні — що в них на ринок подається чималі маси сировинного
\index{ii}{0187}  %% посилання на сторінку оригінального видання
матеріялу. Так, напр., у Лондоні що три місяці бувають великі
авкціони шерсти, які реґулюють шерстяний ринок, тимчасом як ринок
бавовни від урожаю до врожаю поновлюється в цілому безупинно, хоч
і не завжди рівномірно. Такі періоди визначають головні строки закупу
цих сировинних матеріялів і особливо впливають на спекулятивні закупи,
що зумовлюють більш або менш протяжні авансування на ці елементи
продукції, — впливають цілком так само, як природа випродукуваних
товарів впливає на спекулятивне, навмисне, довше або коротше затримування
продукту в формі потенціяльного товарового капіталу. „Отже,
сільський господар теж мусить до певної міри бути спекулянтом і тому
утримуватись від продажу своїх продуктів, зважаючи на обставини часу“\dots{}
Далі ідуть деякі загальні правила\dots{} „Тимчасом при збуті продуктів найголовніше
все ж таки залежить від особи, від самого продукту й місцевости.
Коли людина, крім кмітливости та вдачі (!), має достатній капітал для продукції
(Betriebskapital), їй не можна докоряти, якщо при незвичайно низьких цінах
вона залишить лежати свій зібраний хліб ще цілий рік; навпаки, кому
бракує обігового капіталу або взагалі (!) духа спекуляції, той дбатиме
про те, щоб взяти звичайну пересічну ціну і, значить, муситиме продавати,
скоро матиме нагоду до цього. Коли вовну зберігати довше, ніж протягом
одного року, то це майже завжди зробить тільки шкоду; тимчасом як
зернові хліба та олійне насіння можна зберігати кілька років, і при цьому
не псуються їхні властивості й добротність. Ті продукти, що зазнають
звичайно протягом короткого часу великого піднесення та падіння цін як
от, прим., олійне насіння, хміль, ворсувальні шишки тощо, небезпідставно
залишають лежати в ті роки, коли ціни на них нижчі від цін їхньої продукції.
Найменше слід відкладати продаж таких продуктів, що потребують
щоденних витрат на їхнє утримання, як от відгодована худоба, або таких,
що псуються, як от фрукти, картопля і~\abbr{т. ін.} В деяких місцевостях у певну
добу року ціна продукту пересічно є найнижча, а іншого часу, навпаки,
найвища. Так, напр., пересічно ціна на зерно на Мартіна в деяких місцевостях
нижча, ніж між різдвом і великоднем. Далі, в деяких місцевостях
деякі продукти можна добре продати тільки певного часу, як, напр.,
вовну на вовняних ярмарках у таких місцевостях, де, крім ярмарок, звичайно
дуже мало торгують вовною“. (Kirchhof, стор. 302).

Розглядаючи другу половину часу обігу, що протягом його гроші
знову перетворюються на елементи продуктивного капіталу, треба взяти
на увагу не лише це перетворення само собою; не лише час, що протягом
його гроші припливають назад, залежно від віддалености того
ринку, де продається продукт; треба взяти насамперед на увагу й розміри
тієї частини авансованого капіталу, яка постійно мусить перебувати в
грошовій формі, в стані грошового капіталу.

Лишаючи осторонь усяку спекуляцію, розмір закупів тих товарів, які
постійио мають бути наявні як продуктивний запас, залежить від строків
поновлення цього запасу, отже, від обставин, що й собі залежать від
ринкових відносин, і які тому є різні для різних сировинних матеріялів
тощо; отже, тут доводиться час від часу одним заходом авансовувати
\parbreak{}  %% абзац продовжується на наступній сторінці
