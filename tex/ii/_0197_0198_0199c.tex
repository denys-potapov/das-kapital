
\index{ii}{0197}  %% посилання на сторінку оригінального видання
\begin{table}
  \footnotesize
  \caption{Капітал ІI}
  \toprule
  \begin{tabularx}{\textwidth}{r c r c r c c c r}
  \multicolumn{3}{c}{Періоди обороту} & \multicolumn{2}{c}{Робочі періоди} & \multicolumn{2}{c}{Авансовано} & \multicolumn{2}{c}{Періоди циркуляції} \\
\cmidrule{1-3}
\cmidrule(l){4-5}
\cmidrule(l){6-7}
\cmidrule(l){8-9}
І. & Тижні & 4\sfrac{1}{2}—13\sfrac{1}{2} & Тижні & 4\sfrac{1}{2}—9 & 450 & ф. ст. & Тижні & 10--13\sfrac{1}{2}\\
II. & \ditto{Тижні} & 13\sfrac{1}{2}—22\sfrac{1}{2} & \ditto{Тижні} & 13\sfrac{1}{2}—18 & 450 & \ditto{ф.} \ditto{ст.} & \ditto{Тижні} & 19--22\sfrac{1}{2}\\
III. & \ditto{Тижні} & 22\sfrac{1}{2} — З1\sfrac{1}{2} & \ditto{Тижні} & 22\sfrac{1}{2}—27 & 450 & \ditto{ф.} \ditto{ст.} & \ditto{Тижні} & 28--31\sfrac{1}{2}\\
IV. & \ditto{Тижні} & 31\sfrac{1}{2}—40\sfrac{1}{2} & \ditto{Тижні} & 31\sfrac{1}{2}—36   & 450 & \ditto{ф.} \ditto{ст.} & \ditto{Тижні} & 37--40\sfrac{1}{2}\\
V. & \ditto{Тижні} & 40\sfrac{1}{2}—49\sfrac{1}{2} & \ditto{Тижні} & 40\sfrac{1}{2}—45   & 450 & \ditto{ф.} \ditto{ст.} & \ditto{Тижні} & 46--49\sfrac{1}{2}\\
VI. & \ditto{Тижні} & 49\sfrac{1}{2} — [58\sfrac{1}{2}] & \ditto{Тижні} & 49\sfrac{1}{2} — [54] & 450 & \ditto{ф.} \ditto{ст.} & \ditto{Тижні} & [55\footnotemarkZ{} — 58\sfrac{1}{2}]\\
  \end{tabularx}
\end{table}
\footnotetextZ{В нім. тексті тут, очевидно, помилково стоїть „54“. \emph{Ред.}} % текст примітки прямо під заголовком
Протягом 50 тижнів, що їх ми тут беремо за рік, капітал І закінчив
шість повних робочих періодів, отже, випродукував товарів на $450 × 6
\deq{} 2700$\pound{ ф. стерл.}, а капітал II в п’ять повних робочих періодів — на
$450 × 5 \deq{} 2250$\pound{ ф. стерл}. Крім того, капітал II в останні 1\sfrac{1}{2} тижні року
(з середини 50-го до кінця 51-го тижня\footnote*{
Дальше обчислення побудовано на припущенні 51 тижня в році. \emph{Ред.}
} випродукував ще на 150\pound{ ф.
стерл.} — всього продукту за 51 тиждень випродукувано на 5100\pound{ ф. ст}.
Отже, щодо безпосередньої продукції додаткової вартости — а її продукується
лише протягом робочого періоду — цілий капітал в 900\pound{ ф. стерл.} обернувся
б 5\sfrac{2}{3} раза ($900 × 5\sfrac{2}{3} \deq{} 5100$\pound{ ф. стерл.}). Але коли ми розглянемо
справжній оборот, то побачимо, що капітал І обернувся 5\sfrac{2}{3} раза, бо
наприкінці 51 тижня йому треба ще протягом 3 тижнів закінчувати
свій шостий період обороту; $450 × 5\sfrac{2}{3} \deq{} 2550$\pound{ ф. стерл.}; а капітал
II обернувся 5\sfrac{1}{6} раза, бо він проробив тільки 1\sfrac{1}{2} тижні свого
шостого періоду обороту, значить, ще 7\sfrac{1}{2} тижнів його припадуть на
наступний рік; $450 × 5\sfrac{1}{6} \deq{} 2325$\pound{ ф. стерл.}, ввесь дійсний оборот дорівнює
4875\pound{ ф. стерл}.

Розгляньмо капітал І й капітал II, як два цілком самостійні один проти
одного капітали. В своїх рухах вони цілком самостійні; ці рухи доповнюють
один одного тільки тому, що їхні робочі періоди та періоди
циркуляції безпосередньо чергуються один по одному. Їх можна розглядати
як два цілком незалежні капітали, що належать різним капіталістам.

Капітал І проробив п’ять повних періодів обороту і дві третини свого
шостого періоду обороту. Наприкінці року він перебуває в формі товарового
капіталу, що йому треба ще 3 тижні для своєї нормальної реалізації.
Протягом цього часу він не може ввійти в процес продукції.
Він функціонує як товаровий капітал: він циркулює. З свого останнього
періоду обороту він проробив лише \sfrac{2}{3}. Це можна висловити так: він
обернувся лише \sfrac{2}{3} раза, лише \sfrac{2}{3} цілої вартости його зробили повний оборот.
Ми кажемо: 450\pound{ ф. стерл.} пророблюють свій оборот у 9 тижнів, отже,
300\pound{ ф. стерл.} — у 6 тижнів. При такому способі вислову нехтується органічні
відношення між обома специфічно різними складовими частинами часу обороту.
\index{ii}{0198}  %% посилання на сторінку оригінального видання
Точний зміст вислову, що авансований капітал в 450\pound{ ф. стерл.} зробив
5\sfrac{2}{3} обороту лише той, що він зробив п’ять повних оборотів і тільки
\sfrac{2}{3} шостого. Навпаки, в вислові: капітал, що обернувся, дорівнює
авансованому капіталові, взятому 5\sfrac{2}{3} раза, тобто в наведеному вище прикладі
дорівнює 450\pound{ ф. стерл.} × 5\sfrac{2}{3} \deq{} 2550\pound{ ф. стерл.}, — правильне те, що
коли б цей капітал в 450\pound{ ф. стерл.} не доповнювався б другим капіталом
в 450\pound{ ф. стерл.}, то в дійсності одна частина його мусила б бути в процесі
продукції, а друга — в процесі циркуляції. Коли ми хочемо час обороту
виразити в масі капіталу, що обернувся, то можемо виразити його
виключно в масі наявної вартости (в дійсності — в масі готового продукту).
Та обставина, що авансований капітал не перебуває в такому
стані, в якому він знову може почати процес продукції, виражається в
тому, що лише частина його перебуває в стані, придатному для продукції,
або в тому, що капітал, коли він має бути в стані безперервної продукції,
треба поділити на частини, що з них одна постійно була б у періоді
продукції, а друга — постійно в періоді циркуляції, залежно від взаємного
відношення цих періодів. Це той самий закон, що згідно з ним
масу постійно діющого продуктивного капіталу визначається відношенням
часу обігу до часу обороту.

Наприкінці 51-го тижня — а ми його беремо тут як кінець року
150\pound{ ф. стерл.} з капіталу II авансовано на продукцію недоробленого ще
продукту. Ще деяка частина перебуває в формі поточного сталого капіталу
— сировинного матеріялу тощо — тобто в такій формі, що в ній
вона може функціонувати в процесі продукції як продуктивний капітал.
Але третя частина перебуває в грошовій формі, а саме, принаймні, сума
заробітної плати за решту робочого періоду (3 тижні), що оплачується
лише наприкінці кожного тижня. Хоч на початку нового року, отже, нового
циклу оборотів, ця частина капіталу перебуває не в формі продуктивного
капіталу, а в формі грошового капіталу, що в ній вона не може
ввійти в процес продукції, все ж, коли починається новий оборот, поточний,
змінний капітал, тобто жива робоча сила, уже діє в процесі продукції.
Це явище випливає з того, що хоч робочу силу купується на початку
робочого періоду, напр., щотижня, і так само зуживається, але
оплачується її лише наприкінці тижня. Гроші функціонують тут як засіб
виплати. Тому вони, з одного боку, як гроші перебувають ще в руках
капіталіста, тимчасом як, з другого боку, робоча сила, товар, що на
нього їх перетворюється, вже діє в продукційному процесі; отже, та сама
капітальна вартість з’являється тут двічі.

Коли ми розглядаємо лише робочі періоди, то:

\begin{table}[h]
  \setlength{\tabcolsep}{2pt}
  \begin{tabularx}{\textwidth}{c c c c c}
Капітал І випродукував & 450 × 6 & \deq{} & 2700 & ф. стерл.\\

\ditto{Капітал} II \ditto{випродукував} & 450 × 5\sfrac{1}{3} & \deq{} & 2400 & ф. стерл.\\
\cmidrule{1-5}
Отже, разом\dotfill & 900 × 5\sfrac{2}{3} & \deq{} & 5100 & ф. стерл.\\
  \end{tabularx}
\end{table}
Отже, увесь авансований капітал в 900\pound{ ф. стерл.} за рік функціонував
5\sfrac{2}{3} раза як продуктивний капітал. Для продукції додаткової вартости
байдуже, чи функціонують навперемінку 450\pound{ ф. стерл.} ввесь час у процесі
\index{ii}{0199}  %% посилання на сторінку оригінального видання
продукції і 450\pound{ ф. стерл.} ввесь час у процесі циркуляції, чи 900\pound{ ф.
стерл.} функціонують протягом 4\sfrac{1}{2} тижнів у процесі продукції, а протягом
наступних 4\sfrac{1}{2} тижнів — у процесі циркуляції.

Навпаки, коли ми розглядаємо періоди обороту, то:

\begin{table}[h]
  \setlength{\tabcolsep}{2pt}
  \begin{tabularx}{\textwidth}{r c c c c}
    Капітал І & 450 × 5\sfrac{2}{3} & \deq{} & 2550 & ф. стерл.\\

    \ditto{Капітал} II & 450 × 5\sfrac{1}{6} & \deq{} & 2325 & ф. стерл.\\
    \cmidrule{1-5}
    Отже, оборот цілого капіталу & 900 × 5\sfrac{5}{12} & \deq{} & 4875 & ф. стерл.\\

  \end{tabularx}
\end{table}
Бо число оборотів цілого капіталу дорівнює сумі підсумків оборотів капіталів
І і II, поділеній на суму капіталу І і II.

Треба зазначити, що капітали І і II, коли б були вони самостійні
один проти одного, все ж становили б лише різні самостійні частини суспільного
капіталу, авансованого в тій самій сфері продукції. Отже, коли б
суспільний капітал у цій сфері продукції складався лише з І і II, то для
обороту суспільного капіталу в цій сфері мало б силу те саме обчислення,
що тут має силу для обох складових частин, І і II, того самого
приватного капіталу. Йдучи далі, можна зробити таке обчислення для
кожної частини цілого суспільного капіталу, вкладеної в будь-яку особливу
сферу продукції. Нарешті, число оборотів цілого суспільного капіталу
дорівнює сумі капіталу, що обернувся в різних сферах продукції,
поділеній на суму капіталу, авансованого в цих сферах продукції.

Далі треба зауважити, що так само, як тут у тому самому приватному
підприємстві капітали І і II, точно кажучи, мають різні роки обороту
(що цикл оборотів капіталу II починається на 4\sfrac{1}{2} тижні пізніше, ніж
цикл оборотів капіталу І, то рік капіталу І закінчується на 4\sfrac{1}{2} тижні
раніше, ніж рік капіталу II), так і різні приватні капітали в тій самій
сфері продукції починають свою роботу в цілком різні моменти часу,
а тому й закінчують свій річний оборот в різні часи року. Тут досить
зробити таке саме пересічне обчислення, що його ми вище застосували
до капіталів І і II, щоб роки обороту різних самостійних частин суспільного
капіталу звести до одного загального року обороту.

\subsection{Робочий період більший, ніж період циркуляції}

Замість чергуватися один по одному, робочі періоди й періоди обороту
капіталу І і II перехрещуються один з одним. Разом з тим постає
тут звільнення капіталу, чого не було в вище розглянутому випадку.

Але від цього нічого не змінюється в тому, що тепер, як і раніше,
1) число робочих періодів цілого авансованого капіталу дорівнює сумі
вартости річного продукту обох авансованих частин капіталу, поділеній
на весь авансований капітал, і 2) число оборотів цілого капіталу дорівнює
сумі підсумків обох оборотів, поділеній на суму обох авансованих
капіталів. Ми мусимо й тут розглядати обидві частини капіталу так, ніби
вони пророблювали цілком незалежні один від одного рухи обороту.

Отже, ми знову припускаємо, що на процес праці треба щотижня
авансовувати 100\pound{ ф. стерл}. Робочий період триває 6 тижнів, отже, кожного
\parbreak{}  %% абзац продовжується на наступній сторінці
