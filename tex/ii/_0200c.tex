\parcont{}  %% абзац починається на попередній сторінці
\index{ii}{0200}  %% посилання на сторінку оригінального видання
разу він потребує авансування в 600\pound{ ф. стерл.} (капітал І). Період циркуляції
3 тижні; отже, період обороту, як і раніш, 9 тижнів. Капітал II
в 300\pound{ ф. стерл.} ввіходить у роботу протягом тритижневого періоду циркуляції
капіталу І. Коли розглядати їх обидва, як капітали, незалежні
один від одного, то схема річного обороту матиме такий :

  \begin{table}[h]
    \caption*{Таблиця II.}
    \caption*{Капітал І. 600\pound{ ф. стерл.}}
    \begin{tabular}{r@{ } c@{ } c c@{ } c r@{ } c c@{ } c}
      \toprule
      \multicolumn{3}{c}{Періоди обороту} & \multicolumn{2}{c}{Робочі періоди} & \multicolumn{2}{c}{Авансовано} & \multicolumn{2}{c}{Періоди циркуляції}\\
      \cmidrule(r){1-3}
      \cmidrule(r){4-5}
      \cmidrule(r){6-7}
      \cmidrule(r){8-9}
      І.  & Тижні         & 1  \textendash{} 9 & Тижні         & 1 \textendash{} 6       & 600 & ф. стерл.                 & Тижні         & 7\textendash{}9\\
      II. & \ditto{Тижні} & 10 \textendash{} 18 & \ditto{Тижні} & 10 \textendash{} 15    & 600 & \ditto{ф.} \ditto{стерл.} & \ditto{Тижні} & 16\textendash{}18\\
      III.& \ditto{Тижні} & 19 \textendash{} 27 & \ditto{Тижні} & 19 \textendash{} 24    & 600 & \ditto{ф.} \ditto{стерл.} & \ditto{Тижні} & 25\textendash{}27\\
      IV. & \ditto{Тижні} & 28 \textendash{} 36 & \ditto{Тижні} & 28 \textendash{} 33    & 600 & \ditto{ф.} \ditto{стерл.} & \ditto{Тижні} & 34\textendash{}36\\
      V.  & \ditto{Тижні} & 37 \textendash{} 45 & \ditto{Тижні} & 37 \textendash{} 42    & 600 & \ditto{ф.} \ditto{стерл.} & \ditto{Тижні} & 43\textendash{}45\\
      VI.  & \ditto{Тижні} & 46 \textendash{} [54] & \ditto{Тижні} & 46 \textendash{} 51 & 600 & \ditto{ф.} \ditto{стерл.} & \ditto{Тижні} & [52\textendash{}54]\\
    \end{tabular}
    \caption*{Додатковий капітал II. 300\pound{ ф. стерл.}}
    \begin{tabular}{r@{ } c@{ } r c@{ } c r@{ } c c@{ } c}
      \toprule
      \multicolumn{3}{c}{Періоди обороту} & \multicolumn{2}{c}{Робочі періоди} & \multicolumn{2}{c}{Авансовано} & \multicolumn{2}{c}{Періоди циркуляції}\\
      \cmidrule(r){1-3}
      \cmidrule(r){4-5}
      \cmidrule(r){6-7}
      \cmidrule(r){8-9}

      І.  & Тижні         & 7  \textendash{} 15 & Тижні         & 7 \textendash{} 9   & 300 & ф. стерл.                 & Тижні         & 10\textendash{}15\\
      II. & \ditto{Тижні} & 16 \textendash{} 24 & \ditto{Тижні} & 16 \textendash{} 18 & 300 & \ditto{ф.} \ditto{стерл.} & \ditto{Тижні} & 19\textendash{}24\\
      III.& \ditto{Тижні} & 25 \textendash{} 33 & \ditto{Тижні} & 25 \textendash{} 27 & 300 & \ditto{ф.} \ditto{стерл.} & \ditto{Тижні} & 28\textendash{}33\\
      IV. & \ditto{Тижні} & 34 \textendash{} 42 & \ditto{Тижні} & 34 \textendash{} 36 & 300 & \ditto{ф.} \ditto{стерл.} & \ditto{Тижні} & 37\textendash{}42\\
      V.  & \ditto{Тижні} & 43 \textendash{} 51 & \ditto{Тижні} & 43 \textendash{} 45 & 300 & \ditto{ф.} \ditto{стерл.} & \ditto{Тижні} & 45\textendash{}51\\
    \end{tabular}
  \end{table}

Процес продукції відбувається цілий рік безперервно в однакових
розмірах. Обидва капітали І і II лишаються цілком відокремлені. Але
для того, щоб подати їх так відокремленими, нам довелось роз’єднати
їхні справжні схрещування й переплітання, а через це змінити й число
оборотів. А саме, згідно з вище наведеною таблицею, обертається:

\begin{table}[h]
  \begin{tabular}{r@{\hspace{1}} r@{ } c@{ } c@{ } c{\hspace{1}} c@{ }}
    Капітал & І & 600 × 5\sfrac{2}{3} & = & 3400 & ф. стерл.\\

    \ditto{Капітал} & II & 300 × 5 & = & 1500 & ф. стерл.\\
    \midrule
    отже, ввесь капітал & & 900 × 5\sfrac{4}{9} & = & 4900 & ф. стерл.\\
  \end{tabular}
\end{table}
Але це неправильно, бо, як ми побачимо, справжні періоди продукції
та циркуляції не абсолютно збігаються з цими періодами вище наведеної
схеми, де головне було в тому, щоб подати обидва капітали, І і II, незалежними
один від одного.

В дійсності саме капітал II не має ані особливого робочого періоду, ані особливого
періоду циркуляції, відокремлених від цих періодів капіталу І. Робочий
період триває 6 тижнів, період циркуляції 3 тижні. Що капітал II дорівнює
тільки 300\pound{ ф. стерл.}, то він може виповнити лише частину робочого
\parbreak{}  %% абзац продовжується на наступній сторінці
