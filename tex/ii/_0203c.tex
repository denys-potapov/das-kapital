
\index{ii}{0203}  %% посилання на сторінку оригінального видання
Другий період обороту, тижні 8--16, має в собі другий робочий
період, тижні 8--14. З них потреби 8-го й 9-го тижнів покривається
капіталом II.~Наприкінці 9-го тижня повертаються давніші 700\pound{ ф. стерл.};
з них пускається в роботу до кінця робочого періоду (тижні 10--14)
500\pound{ ф. стерл.} 200\pound{ ф. стерл.} лишаються вільні для ближчого наступного
робочого періоду. Другий період обігу триває протягом 15-го й 16 тижнів;
наприкінці 16-го тижня знову повертаються назад 700\pound{ ф. стерл}.
З цього моменту в кожному робочому періоді повторюється те саме
явище. Потреба в капіталі протягом перших двох тижнів покривається
за допомогою 200\pound{ ф. стерл.}, що звільнились наприкінці попереднього
робочого періоду; наприкінці 2-го тижня повертаються назад 700\pound{ ф.
стерл.}; але робочий період налічує ще тільки 5 тижнів, так що на нього
можна авансувати лише 500\pound{ ф. стерл.}; отже, 200\pound{ ф. стерл.} завжди лишаються
вільні для наступного робочого періоду.

Отже, виявляється, що в нашому випадку, де ми припускали, що робочий
період більший, ніж період обігу, наприкінці кожного робочого
періоду при всяких обставинах є звільнений грошовий капітал, такої
саме величини, як капітал II, авансований на період циркуляції. В наших
трьох прикладах капітал II дорівнював: в першому — 300\pound{ ф. стерл.}, в
другому — 400\pound{ ф. стерл.}, в третьому — 200\pound{ ф. стерл.}; відповідно до
цього капітал, що звільнявся наприкінці кожного робочого періоду, був
послідовно 300, 400, 200\pound{ ф. стерл}.

\subsection{Робочий період менший від часу обігу}

Спочатку ми знову припустимо період обороту в 9 тижнів: з них
З тижні становлять робочий період, що для нього є в розпорядженні
капітал  І \deq{} 300\pound{ ф. стерл}. Період обігу хай буде 6 тижнів. Для цих
6 тижнів потрібен додатковий капітал в 600\pound{ ф. стерл.}, який ми знову
можемо розподілити на два капітали по 300\pound{ ф. стерл.}, що з них кожен
заповнює один робочий період. Тоді ми маємо три капітали по 300\pound{ ф.
стерл.}, з них 300\pound{ ф. стерл.} завжди зайнято в продукції, тимчасом як
600\pound{ ф. стерл.} циркулюють.

\begin{table}[H]
\centering
{\bfseries Таблиця III}
\caption*{Капітал І}
\bigskip
  \begin{tabular}{r r@{~}c r@{~}c r@{~}c}
    \toprule
    & \multicolumn{2}{c}{Періоди обороту} & \multicolumn{2}{c}{Робочі періоди}
    & \multicolumn{2}{c}{Період церкуляції}\\
    \cmidrule(lr){2-3}
    \cmidrule(lr){4-5}
    \cmidrule(lr){6-7}

І.  & Тижні & 1\textendash{}9 & Тижні
    & 1\textendash{}3 & Тижні & 4\textendash{}9\\

ІІ. & \ditto{Тижні} & 10\textendash{}18 & \ditto{Тижні} 
    & 10\textendash{}12 & \ditto{Тижні} & 13\textendash{}18\\

III.& \ditto{Тижні} & 19\textendash{}27 & \ditto{Тижні}
    & 19\textendash{}21 & \ditto{Тижні} & 22\textendash{}27\\

IV. & \ditto{Тижні} & 28\textendash{}36 & \ditto{Тижні}
    & 28\textendash{}30 & \ditto{Тижні} & 31\textendash{}36\\
V.  & \ditto{Тижні} & 37\textendash{}45 & \ditto{Тижні} 
    & 37\textendash{}39 & \ditto{Тижні} & 40\textendash{}45\\

VI. & \ditto{Тижні} & \hang{r}{46}\textendash{}\hang{l}{[54]} & \ditto{Тижні}
    & 46\textendash{}48 & \ditto{Тижні} & \hang{r}{49}\textendash{}\hang{l}{[54]}\\
  \end{tabular}
\end{table}
