
\index{ii}{0205}  %% посилання на сторінку оригінального видання
Подаємо лише перші три обороти.
\begin{table}[h]
  \begin{center}
  \caption*{Таблиця IV.}

  \caption*{Капітал І.}
  \begin{tabular}{r@{ } c@{ } r@{\textendash{}} l c@{ } r@{\textendash{}} l c@{ } r@{\textendash{}} l}
  \toprule
  \multicolumn{4}{c}{Періоди обороту} & \multicolumn{3}{c}{Робочі періоди} & \multicolumn{3}{c}{Періоди обігу}\\
  \cmidrule(r){1-4}
  \cmidrule(r){5-7}
  \cmidrule{8-10}
  І.  & Тижні         & 1 & 9   & Тижні         & 1 & 4   & Тижні & 5 & 9\\
  ІІ. & \ditto{Тижні} & 9 & 17 & \ditto{Тижні} & 9. 10 & 12 & \ditto{Тижні} & 13 & 17\\
  III.& \ditto{Тижні} & 17 & 25 & \ditto{Тижні} & 17. 18 & 20 & \ditto{Тижні} & 21 & 25\\
  \end{tabular}

  \caption*{Капітал ІI.}
  \begin{tabular}{r@{ } c@{ } r@{\textendash{}} l c@{ } r@{\textendash{}} l c@{ } r@{\textendash{}} l}
  \toprule
  \multicolumn{4}{c}{Періоди обороту} & \multicolumn{3}{c}{Робочі періоди} & \multicolumn{3}{c}{Періоди обігу}\\
  \cmidrule(r){1-4}
  \cmidrule(r){5-7}
  \cmidrule{8-10}
  І.  & Тижні         & 5 & 13   & Тижні         & 5 & 8   & Тижні & 9 & 13\\
  ІІ. & \ditto{Тижні} & 13 & 21 & \ditto{Тижні} & 13. 14 & 16 & \ditto{Тижні} & 17 & 21\\
  III.& \ditto{Тижні} & 21 & 29 & \ditto{Тижні} & 21. 22 & 24 & \ditto{Тижні} & 25 & 29\\
  \end{tabular}

  \caption*{Капітал ІII.}
  \begin{tabular}{r@{ } c@{ } r@{\textendash{}} l c@{ } c  c@{ } r@{\textendash{}} l}
  \toprule
  \multicolumn{4}{c}{Періоди обороту} & \multicolumn{2}{c}{Робочі періоди} & \multicolumn{3}{c}{Періоди обігу}\\
  \cmidrule(r){1-4}
  \cmidrule(r){5-6}
  \cmidrule{7-9}
  І.  & Тижні         & 9 & 17   & Тижні         & 9 &  Тижні & 10 & 17\\
  ІІ. & \ditto{Тижні} & 17 & 25 & \ditto{Тижні} & 17 &  \ditto{Тижні} & 18 & 25\\
  III.& \ditto{Тижні} & 25 & 33 & \ditto{Тижні} & 25 &  \ditto{Тижні} & 26 & 33\\
  \end{tabular}
\end{center}
\end{table}

Переплітання капіталів тут є остільки, оскільки робочий період капіталу
III, що не має самостійного робочого періоду, бо його вистачає
тільки на один тиждень, збігається з першим робочим тижнем капіталу І.
Але зате наприкінці робочого періоду і капіталу І й капіталу II
звільняється рівна капіталові III сума в 100\pound{ ф. стерл}. А саме, коли капітал
III виповнює перший тиждень другого та всіх наступних робочих періодів
капіталу І, а наприкінці цього першого тижня назад припливає ввесь
капітал І, 400\pound{ ф. стерл.}, то для решти робочого періоду капіталу І лишається
тільки 3 тижні, і відповідна витрата капіталу буде 300\pound{ ф. стерл}. Звільнених
таким чином 100\pound{ ф. стерл.} буде досить для першого тижня безпосередньо
наступного робочого періоду капіталу II; наприкінці цього тижня повертається
назад увесь капітал II в 400\pound{ ф. стерл.}; а що розпочатий робочий
період може ввібрати ще тільки 300\pound{ ф. стерл.}, то наприкінці його лишаються
\index{ii}{0206}  %% посилання на сторінку оригінального видання
знову вільних 100\pound{ ф. стерл.} і так далі. Отже, постає звільнення капіталу
наприкінці робочого періоду, скоро час обігу не є просте кратне
робочого періоду; і цей вільний капітал дорівнює тій частині капіталу,
що повинна виповнювати час, який становить надлишок періоду циркуляції
проти робочого періоду або проти кратного робочих періодів.
В усіх досліджених випадках припускалося, що й робочий період і
час обігу протягом цілого року лишались однакові в кожному з розглянутих
тут підприємств. Таке припущення було потрібне, коли ми хотіли
встановити вплив часу обігу на оборот і на авансування капіталу. Той
факт, що в дійсності це припущення не здійснюється з такою безумовністю,
а іноді й зовсім не здійснюється, зовсім не змінює справи.

В цілому відділі цьому ми розглядали тільки обороти обігового капіталу,
а не основного. Проста причина цього та, що розглядуване питання
не має жодного чинення до основного капіталу. Вживані в процесі продукції
засоби праці тощо становлять основний капітал лише остільки,
оскільки час вживання їх триває довше, ніж період обороту поточного
капіталу; оскільки час, що протягом його ці засоби праці й далі служать
у постійно повторюваних процесах праці, більший, ніж період обороту
поточного капіталу, оскільки, отже, він дорівнює $n$ періодам обороту
поточного капіталу. Чи ввесь час, що його складають ці $n$ періодів обороту
поточного капіталу, буде довший, чи коротший, все одно частину
продуктивного капіталу, авансовану на цей час на основний капітал, не
авансуватиметься знову в межах цього часу. Вона й далі функціонує в
своїй старій споживній формі. Ріжниця лише ось в чому: відповідно до
різного протягу поодинокого \emph{робочого періоду}, в кожному періоді
обороту поточного капіталу основний капітал віддає більшу або меншу
частину своєї початкової вартости продуктові цього робочого періоду, і,
відповідно до довжини часу циркуляції в кожному періоді обороту, ця
перенесена на продукт частина вартости основного капіталу швидше або
повільніше припливає назад в грошовій формі. Природа предмету, що
його ми розглядаємо в цьому розділі — оборот обігової частини продуктивного
капіталу — випливає з самої природи цієї частини капіталу. Поточний
капітал, вжитий в одному робочому періоді, не можна вжити в
новому робочому періоді, поки не закінчить він свого обороту, поки не
перетвориться на товаровий капітал, з нього — на грошовий капітал, а
з цього останнього знову на продуктивний капітал. Отже, для того, щоб
за першим робочим періодом одразу починався другий, треба знову
авансувати капітал і перетворити його на поточні елементи продуктивного
капіталу, та ще й авансувати його в достатніх розмірах, щоб виповнити
прогалини, посталі в наслідок періоду циркуляції поточного
капіталу, авансованого на перший робочий період. Відси постає вплив
протягу робочого періоду поточного капіталу на маштаб процесу праці
в підприємстві й на поділ авансованого капіталу, зглядно й на розмір
додаткового авансування нових частин капіталу. А це саме й є те, що ми
повинні були дослідити в цьому відділі.
