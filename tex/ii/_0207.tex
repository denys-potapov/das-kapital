\index{ii}{0207}  %% посилання на сторінку оригінального видання
\subsection{Висновки}

З попереднього досліду випливає:

А. Різні частки, на які треба поділити капітал, щоб одна з частин
його постійно могла бути в робочому періоді, тимчасом як друга перебуває
в періоді циркуляції, — чергуючись змінюють одна одну, як різні
самостійні приватні капітали, в двох випадках: 1) коли робочий період
дорівнює періодові циркуляції, коли, отже, період обороту розпадається
на два однакові відділи; 2) коли період циркуляції довший, ніж робочий
період, але разом з тим становить просте кратне робочого періоду, так
що один період циркуляції дорівнює n робочим періодам, де n мусить
бути цілим числом. В цих випадках жодна частина послідовно авансованого
капіталу не звільняється.

В. Навпаки, в усіх тих випадках, коли 1) період циркуляції більший,
ніж робочий період, і не становить простого кратного йому і 2) коли
робочий період більший, ніж період циркуляції, то наприкінці кожного
робочого періоду, починаючи з другого обороту, постійно й періодично
звільняється частина цілого поточного капіталу. При цьому, коли робочий
період більший, ніж період циркуляції, то цей звільнений капітал
дорівнює частині цілого капіталу, авансованій на період циркуляції, а
коли період циркуляції більший, ніж робочий період, то цей звільнений
капітал дорівнює частині капіталу, що має поповнювати надлишок періоду
циркуляції проти робочого періоду або проти кратного робочих періодів.

С. З цього випливає, що для сукупного суспільного капіталу, розглядуваного
з погляду його поточної частини, звільнення капіталу
мусить бути загальним правилом, а просте чергування частин капіталу,
що послідовно функціонують у процесі продукції, — винятком. Бо однаковість
робочого періоду й періоду циркуляції або однаковість періоду
циркуляції та простого кратного робочого періоду, така правильна пропорційність
двох складових частин періоду обороту не має жодного
чинення до суті справи і тому взагалі та в цілому може траплятись лише
винятково.

Отже, дуже значна частина суспільного обігового капіталу, що робить
кілька оборотів протягом року, буде протягом річного циклу оборотів
періодично перебувати в формі звільненого капіталу.

Далі зрозуміло, що, припускаючи всі інші умови незмінними, величина
цього звільненого капіталу зростає разом з розміром процесу праці
або з маштабом продукції, отже, взагалі з розвитком капіталістичної
продукції. У випадку, позначеному під В, 2) — тому, що зростає ввесь
авансований капітал; в випадку В, 1) — тому, що з розвитком капіталістичної
продукції зростає протяг періоду циркуляції, а значить, і період
обороту, в тих випадках, коли немає правильного відношення між робочим
періодом і періодом циркуляції.

В першому випадку ми повинні були щотижня витрачати, напр.,
100\pound{ ф. стерл}. Для шеститижневого робочого періоду 600\pound{ ф. стерл.}, для тритижневого
періоду циркуляції 300\pound{ ф. стерл.}, разом 900\pound{ ф. стерл}. Тут буде
\parbreak{}  %% абзац продовжується на наступній сторінці
