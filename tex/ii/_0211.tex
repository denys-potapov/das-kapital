\parcont{}  %% абзац починається на попередній сторінці
\index{ii}{0211}  %% посилання на сторінку оригінального видання
Маркс надавав незаслугованого значення одній — на мою думку — на ділі
мало важливій обставині. Я маю тут на увазі те, що він зве „звільнення“
грошового капіталу. Справжня суть справи, при зроблених вище припущеннях,
така:

Хоч яке буде відношення величин робочого періоду та часу обігу,
отже, відношення величини капіталу І до капіталу II, — по закінченні першого
обороту до капіталіста все одно через реґулярні переміжки, рівні
довжині робочого періоду, повертається в грошовій формі капітал, потрібний
для одного робочого періоду, — отже, все одно повертається сума,
рівна капіталові І.

Коли робочий період дорівнює 5 тижням, час обігу — 4 тижням, капітал
І — 500\pound{ ф. стерл.}, то кожного разу повертається назад грошова
сума 500\pound{ ф. стерл.}: наприкінці 9-го, 14-го, 19-го, 24-го, 29-го і т. д.
тижнів.

Коли робочий період дорівнює б тижням, час обігу — 3 тижням, капітал
І — 600\pound{ ф. стерл.}, то по 600\pound{ ф. стерл.} припливає назад: наприкінці
9-го, 15-го, 21-го, 27-го, 33-го і т. д. тижнів.

Нарешті, коли робочий період = 4 тижням, час обігу = 5 тижням,
капітал І = 400\pound{ ф. стерл.}, то по 400\pound{ ф. стерл.} припливає назад наприкінці
9-го, 13-го, 17-го, 21-го, 25-го і т. д. тижнів.

Чи буде й скільки буде зайвих із цих грошей для поточного робочого
періоду — із цих грошей, що приплили назад, отже, чи звільняться
вони й скільки їх звільниться, це не має значення. Припускається, що
продукція відбувається безперервно в даному маштабі, а щоб це було
можливо, гроші мусять бути наявні, отже, вони мусять припливати назад,
усе одно, чи „звільняються“ вони, чи ні. А коли продукція переривається,
то припиняється й це звільнення.

Інакше кажучи: звільнення грошей, тобто утворення лятентного, лише
потенціяльного капіталу в грошовій формі, звичайно, відбувається;
але воно відбувається при всяких обставинах, а не лише в особливих умовах,
точніше зазначених у тексті; і відбувається воно в ширших розмірах,
ніж зазначено в тексті. Відносно обігового капіталу І промисловий капіталіст
наприкінці кожного обороту перебуває цілком в тому самому
стані, як і на початку заснування підприємства: він знову разом одержав
увесь обіговий капітал, тимчасом як перетворювати його знову на продуктивний
капітал він може лише поступінно.

Головне в тексті, це — доказ того, що, з одного боку, чимала частина
промислового капіталу завжди мусить бути наявна в грошовій формі,
а з другого боку — ще більша частина мусить час від часу набирати
грошової форми. Мої додаткові примітки лише потверджують цей доказ. Ф. Є.].

\subsection{Вплив зміни цін}

Вище ми припускали незмінні ціни, незмінний маштаб продукції, з
одного боку, скорочення або подовження часу циркуляції — з другого.
\parbreak{}  %% абзац продовжується на наступній сторінці
