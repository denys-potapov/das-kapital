\parcont{}  %% абзац починається на попередній сторінці
\index{ii}{0214}  %% посилання на сторінку оригінального видання
постійно витрачається на матеріяли для продукції і \sfrac{1}{5} \deq{} 120\pound{ ф. стерл.}
на заробітну плату. Отже, щотижня 80\pound{ ф. стерл.} на матеріяли для продукції,
20\pound{ ф. стерл.} на заробітну плату. Отже, капітал II \deq{} 300\pound{ ф. стерл.}
так само мусить поділитись на \sfrac{4}{5} \deq{} 240\pound{ ф. стерл.} для продукційних матеріялів
і на \sfrac{1}{5} \deq{} 60\pound{ ф. стерл.} для заробітної плати. Капітал, витрачуваний
на заробітну плату, завжди мусить авансуватись у грошовій формі.
Скоро товаровий продукт вартістю в 600\pound{ ф. стерл.} зворотно перетворюється
на грошову форму, скоро його продано, — 480\pound{ ф. стерл.} з цієї
суми можна перетворити на матеріяли для продукції (на продуктивний
запас), але 120\pound{ ф. стерл.} зберігають свою грошову форму, щоб служити
для виплати заробітної плати протягом 6 тижнів. Ці 120\pound{ ф. стерл.} являють
той мінімум приплилого назад капіталу в 600\pound{ ф. стерл.}, який завжди
мусить поповнюватись і заміщуватись у формі грошового капіталу, а
тому й мусить він завжди бути наявний як діюща в грошовій формі частина
авансованого капіталу.

Коли тепер з тих 300\pound{ ф. стерл.}, що періодично звільняються на З
тижні й так само розпадаються на 240\pound{ ф. стерл.} для продуктивного запасу
й на 60\pound{ ф. стерл.} для заробітної плати, в наслідок скорочення часу
обігу виділюється 100\pound{ ф. стерл.} у формі грошового капіталу, зовсім викидаємся
з механізму обороту, то постає питання: відки береться гроші
для цих 100\pound{ ф. стерл.} грошового капіталу? Лише на п’яту частину вони
складаються з грошового капіталу, що періодично звільняється в межах
оборотів. Але \sfrac{4}{5} \deq{} 80\pound{ ф. стерл.} уже заміщено додатковим продуктивним
запасом тієї самої вартости. Яким же чином цей додатковий продуктивний
запас перетворюється на гроші, і відки береться гроші на це перетворення?
Якщо постало скорочення часу обігу, то з вищезгаданих 600\pound{ ф. стерл.}
на продуктивний запас замість 480\pound{ ф. стерл.} перетворюється лише 400\pound{ ф.
стерл}. Решту 80\pound{ ф. стерл.} зберігається в їхній грошовій формі, і
разом з вищезгаданими 20\pound{ ф. стерл.}, призначеними для заробітної плати,
вони становлять цей виділений капітал в 100\pound{ ф. стерл}. Хоч ці 100\pound{ ф.
стерл.} приходять з циркуляції в наслідок продажу товарового капіталу в
600\pound{ ф. стерл.} і тепер їх вилучається з циркуляції, бо їх не витрачається
знову на заробітну плату й елементи продукції, однак, не треба забувати,
що в грошовій формі вони знову є в тій самій формі, що в ній їх
первісно кинуто в циркуляцію. Спочатку на продукційний запас і на заробітну
плату витрачалось 900\pound{ ф. стерл.} грішми. Щоб подати той
самий процес продукції, треба тепер вже лише 800\pound{ ф. стерл}. Виділені в
наслідок цього в грошовій формі 100\pound{ ф. стерл.} становлять тепер новий
грошовий капітал, що шукає приміщення, нову складову частину грошового
ринку. Щоправда, вони й раніш періодично перебували у формі
звільненого грошового капіталу й додаткового продуктивного капіталу,
але цей лятентний стан сам був умовою провадження процесу продукції,
бо він був умовою його безперервности. Тепер їх уже не треба для
цього, а тому вони становлять новий грошовий капітал і одну з складових
частин грошового ринку, хоч вони зовсім не є ні додатковий елемент
\index{ii}{0215}  %% посилання на сторінку оригінального видання
до вже наявного суспільного грошового запасу (бо вони вже були
на початку заснування підприємства, й воно пустило їх в циркуляцію),
ні новоакумульований скарб.

Тепер ці 100\pound{ ф. стерл.} дійсно вилучається з циркуляції, оскільки вони
є частина авансованого грошового капіталу, що її тепер уже не застосовується
в тому самому підприємстві. Але таке вилучення можливе
тільки тому, що перетворення товарового капіталу на гроші, а цих грошей
— на продуктивний капітал, $Т' — Г — Т$, прискорюється на один тиждень,
отже, прискорюється й обіг діющих у цьому процесі грошей. Їх вилучено
з циркуляції, бо вони більше непотрібні для обороту капіталу X.

Тут припускається, що авансований капітал належить тому, хто його
застосовує. Коли б він був позичений, справа через це ані трохи не змінилась
би. Із скороченням часу обігу, підприємцеві треба було б замість
900\pound{ ф. стерл.} лише 800\pound{ ф. стерл.} позиченого капіталу. 100\pound{ ф. стерл.},
повернені позикодавцеві, становлять, як і раніше, новий грошовий капітал
в 100\pound{ ф. стерл.} тільки вже не в руках X, а в руках Y. Далі, коли
капіталіст X одержував наборг свої продукційні матеріяли вартістю в
480\pound{ ф. стерл.}, так що сам він мав авансувати грішми тільки 120\pound{ ф. стерл.}
на заробітну плату, то тепер він має брати наборг продукційних матеріялів
на 80\pound{ ф. стерл.} менше, отже, ці 80\pound{ ф. стерл.} становлять надлишковий
товаровий капітал для капіталіста, що дає наборг, тимчасом як
для капіталіста X виділилось би 20\pound{ ф. стерл.} грішми.

Додатковий продукційний запас зменшився тепер на \sfrac{1}{3}. Являючи \sfrac{4}{5}
від 300\pound{ ф. стерл.}, він дорівнював додатковому капіталові II=240\pound{ ф.
стерл.}, тепер він дорівнює лише 160\pound{ ф. стерл.}, тобто являє додатковий
запас на 2 тижні замість 3. Тепер він відновлюється що 2 тижні замість
що 3, але також тільки на два тижні замість 3. Закупи, напр., на ринку
бавовни повторюється, таким чином, частіше й меншими пайками. З ринку
береться ту саму кількість бавовни, бо маса продукту лишається та
сама. Але закупи розподіляються інакше в часі й розтягується їх на довший
час. Припустімо, напр., що йдеться про 3 місяці й про 2 місяці; хай
річне споживання бавовни буде 1200 пак. В першому випадку продаватиметься:
\begin{table}[h]
  \begin{center}
    \begin{tabular}{c@{ } c@{ } c@{ } c@{ } c@{ } c@{ }}
      1 & січня  & 300 & пак, на складі лишається                                & 900 & пак \\
      1 & квітня & 300 & \ditto{пак} \ditto{на} \ditto{складі} \ditto{лишається} & 600 & \ditto{пак} \\
      1 & липня  & 300 & \ditto{пак} \ditto{на} \ditto{складі} \ditto{лишається} & 300 & \ditto{пак} \\
      1 & жовтня & 300 & \ditto{пак} \ditto{на} \ditto{складі} \ditto{лишається} & 0   & \ditto{пак} \\
    \end{tabular}
  \end{center}
\end{table}

Навпаки, в другому випадку:

\begin{table}[h]
  \begin{center}
    \begin{tabular}{c@{ } l@{} c@{ } l@{ } c@{ } c@{ } c@{ }}
      1 & січня & продається & 200, & на складі & 1000 & пак\\
      1 & березня & \ditto{продається} & 200 & \ditto{на} \ditto{складі} & 800 & \ditto{пак} \\
      1 & травня & \ditto{продається} & 200 & \ditto{на} \ditto{складі} & 600 & \ditto{пак} \\
      1 & липня & \ditto{продається} & 200 & \ditto{на} \ditto{складі} & 400 & \ditto{пак} \\
      1 & вересня & \ditto{продається} & 200 & \ditto{на} \ditto{складі} & 200 & \ditto{пак} \\
      1 & листопада & \ditto{продається} & 200 & \ditto{на} \ditto{складі} & \phantom{00}0 & \ditto{пак} \\
    \end{tabular}
  \end{center}
\end{table}

Отже, витрачені на бавовну гроші цілком повертаються лише на місяць
пізніше, в листопаді замість жовтня. Отже, коли в наслідок скорочення
\parbreak{}  %% абзац продовжується на наступній сторінці
