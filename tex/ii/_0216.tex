\parcont{}  %% абзац починається на попередній сторінці
\index{ii}{0216}  %% посилання на сторінку оригінального видання
часу обігу, а разом з тим і часу обороту, виділюється в формі грошового
капіталу \sfrac{1}{9} частина авансованого капіталу \deq{} 100\pound{ ф. стерл.} і коли
ці 100\pound{ ф. стерл.} складаються з 20\pound{ ф. стерл.} періодично надлишкового
грошового капіталу, призначеного для виплати щотижневої заробітної
плати, і з 80\pound{ ф. стерл.}, що існують як періодичний надлишковий тижневий
продукційний запас, — то цьому зменшенню у фабриканта надлишкового
продукційного запасу на 80\pound{ ф. стерл.} відповідає збільшення товарового
запасу у торговця бавовною. Та сама бавовна то довше лежить
на його складах як товар, що менше лежить вона на складах у фабриканта
як продукційний запас.

Досі ми припускали, що скорочення часу обігу в підприємстві X випливає
з того, що X швидше продає свої товари або швидше одержує
за них гроші, зглядно, що при кредиті термін виплати скорочується.
Отже, це скорочення часу обігу випливає з швидкого продажу товарів,
швидкого перетворення товарового капіталу на грошовий, з $Т' — Г'$, з
першої фази процесу циркуляції. Воно могло б випливати й з другої фази,
$Г — Т$, а тому й з одночасної зміни, чи то робочого періоду, чи то часу
обігу капіталів $Y$, $Z$, і~\abbr{т. ін.}, що постачають капіталістові X продукційні
елементи його поточного капіталу.

Коли, напр., бавовна, вугілля та ін., в старих умовах транспорту перебувають
8 тижні в дорозі від місця продукції або від складів до місця
підприємства капіталіста X, то мінімуму продукційного запасу X мусить
вистачати принаймні на 3 тижні, поки надійдуть нові запаси. Поки
бавовна та вугілля перебувають в дорозі, вони не можуть служити як
засоби продукції. Вони скоріше становлять тоді предмет праці для транспортової
промисловости й приміщеного в ній капіталу, а також товаровий
капітал для вуглепродуцента або для продавця бавовни, товаровий капітал,
що перебуває в своїй циркуляції. При поліпшеному транспорті час
перевозу скорочується до 2 тижнів. Таким чином, продукційний запас може
перетворитися з тритижневого на двотижневий. Разом з тим звільняється
авансований на це додатковий капітал у 80\pound{ ф. стерл.}, а також 20\pound{ ф. стерл.}, призначені на заробітну плату, бо капітал у 600\pound{ ф. стерл.},
що обернувся, повертається на тиждень раніше.

З другого боку, коли, напр., робочий період капіталу, що постачає
сировинний матеріял, скорочується (приклади про це подано в попередніх
розділах), отже, зростає й можливість відновлювати сировинний матеріял,
то продукційний запас може зменшитись, переміжок від одного періоду
відновлення до другого може скоротитись.

Навпаки, коли час обігу, а тому й період обороту довшає, то потрібне
авансування додаткового капіталу — з кишені самого капіталіста,
коли в нього є додатковий капітал. Але цей капітал є в тій
або іншій формі приміщений, як частина грошового ринку; щоб ним
можна було порядкувати, його треба визволити з старої форми, напр.,
продати акції, взяти вклади, так що й тут постає посередній вплив на
грошовий ринок. Або капіталіст мусить десь позичити додатковий капітал
Щож до частини додаткового капіталу, потрібної для заробітної плати, то
\parbreak{}  %% абзац продовжується на наступній сторінці
