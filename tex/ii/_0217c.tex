\parcont{}  %% абзац починається на попередній сторінці
\index{ii}{0217}  %% посилання на сторінку оригінального видання
в нормальних обставинах її завжди треба авансувати як грошовий капітал,
і цим капіталіст $X$ з свого боку справляє безпосередній тиск на грошовий
ринок. Для тієї частини капіталу, що її треба вкласти в продукційні
матеріяли, авансування в формі грошового капіталу неминуче лише
тоді, коли капіталіст мусить платити готівкою. Коли він може одержати
ці матеріяли на кредит, то це не справить жодного безпосереднього
впливу на грошовий ринок, бо додатковий капітал тоді безпосередньо
авансується, як продукційний запас, а не як з самого початку грошовий
капітал. Оскільки його кредитор подає і собі одержаний від $X$ вексель
безпосередньо на грошовий ринок, дає його дисконтувати і~\abbr{т. ін.}, то це
справить на грошовий ринок посередній вплив, через другу особу. Але
коли він скористається з цього векселя, щоб, напр., покрити борги пізнішого
терміну, то цей додатково авансований капітал не справить ні
безпосереднього, ні посереднього впливу на грошовий ринок.

\textbf{II випадок.} \emph{Змінюються ціни на продукційні матеріяли; всі інші
обставини незмінні.}

Вище ми припускали, що з усього капіталу в 900\pound{ ф. стерл.} \sfrac{4}{5} \deq{} 720\pound{ ф.
стерл.} витрачається на продукційні матеріяли і \sfrac{1}{5} \deq{} 180\pound{ ф. стерл.} на заробітну
плату.

Коли продукційні матеріяли дешевшають наполовину, то на них для
щотижневого робочого періоду треба лише 240\pound{ ф. стерл.} замість 480\pound{ ф.
стерл.} і на додатковий капітал II треба лише 120\pound{ ф. стерл.} замість
240\pound{ ф. стерл}. Отже, капітал І скорочується з 600\pound{ ф. стерл.} до $240 \dplus{}
120 \deq{} 360$\pound{ ф. стерл.} і капітал II з 300\pound{ ф. стерл.} до $120 \dplus{} 60 \deq{} 180$\pound{ ф.
стерл}. Весь капітал з 900\pound{ ф. стерл.} зменшується до $360 \dplus{} 180 \deq{} 540$\pound{ ф.
стерл}. Отже, виділяється 360\pound{ ф. стерл}.

Цей виділений і тепер вільний капітал, отже, грошовий капітал, що
шукає приміщення на грошовому ринку, є не що інше, як частина капіталу
в 900\pound{ ф. стерл.}, первісно авансованого як грошовий капітал, яка
стала надлишковою в наслідок зниження ціни тих елементів продукції,
що на них вона періодично перетворювалася, — стала надлишковою,
якщо підприємство не поширюється, а провадиться далі в попередньому
розмірі. Коли цей спад цін було б зумовлено не випадковими обставинами
(особливо багатим урожаєм, надмірним довозом і~\abbr{т. ін.}), а збільшенням
продуктивної сили в тій галузі, що постачає сировинний матеріял,
то цей грошовий капітал був би абсолютним додатком до грошового
ринку і взагалі до вільного капіталу, що перебуває в формі грошового
капіталу; бо цей капітал вже не становив би інтеграційної складової частини
капіталу вже застосованого.

\textbf{III випадок.} \emph{Зміна ринкової ціни самого продукту.}

Тут разом із зниженням ціни втрачається й частину капіталу, тому її
треба замістити новим авансуванням грошового капіталу. Цю втрату
продавця може виграти покупець. Безпосередньо — коли продукт подешевшав
у своїй ринковій ціні тільки в наслідок випадкових коньюнктур,
і потім знову ціна його підвищується до нормального рівня. Посередньо —
якщо зміну ціни зумовлено зміною вартости, яка впливає й на старий
\parbreak{}  %% абзац продовжується на наступній сторінці
