\parcont{}  %% абзац починається на попередній сторінці
\index{ii}{0221}  %% посилання на сторінку оригінального видання
додаткової вартости, в даному випадку, отже, 500 × \frac{100}{100} \deq{} 500 × 1 \deq{} 500\pound{ ф. стерл}. Коли б
авансований капітал був \deq{} 1500\pound{ ф. стерл.} при незмінній
нормі додаткової вартости, то маса додаткової вартости була б \deq{}
1500 × \frac{100}{100} \deq{} 1500\pound{ ф. стерл}.

Змінний капітал у 500\pound{ ф. стерл.}, що обертається 10 разів на рік, і
що продукує протягом року додаткову вартість в 5000\pound{ ф. стерл.}, отже,
капітал, що для нього річна норма додаткової вартости \deq{} 1000\%, ми
будемо називати капіталом $А$.

Припустімо тепер, що інший змінний капітал $В$ в 5000\pound{ ф. стерл.}
авансується на цілий рік (тобто, тут на 50 тижнів) і тому обертається
лише один раз на рік. Припустімо при цьому далі, що наприкінці року
продукт оплачується в той самий день, як його виготовлено, і, значить,
грошовий капітал, що на нього його перетворюється, повертається в той
самий день. Отже, період циркуляції тут \deq{} 0, період обороту дорівнює
робочому періодові, а саме, одному рокові. Як і в попередньому випадку,
в процесі праці щотижня перебуває змінний капітал в 100\pound{ ф. стерл.},
а тому протягом 50 тижнів — в 5000\pound{ ф. стерл}. Далі, норма додаткової
вартости хай буде та сама \deq{} 100\%, тобто за однакової довжини робочого
дня половина його складається з додаткової праці. Коли ми візьмемо
5 тижнів, то вкладений змінний капітал \deq{} 500\pound{ ф. стерл.}, норма додаткової
вартости \deq{} 100\%, отже, маса додаткової вартости, створена протягом
5 тижнів \deq{} 500\pound{ ф. стерл}. Кількість робочої сили, що її тут експлуатується,
і ступінь її експлуатації, згідно з нашим припущенням, тут
точно такі самі, як і при капіталі $А$.

Вкладений змінний капітал в 100\pound{ ф. стерл.} щотижня створює додаткову
вартість в 100\pound{ ф. стерл.}, тому протягом 50 тижнів вкладений капітал
в 100 × 50 \deq{} 5000\pound{ ф. стерл.} створить додаткову вартість в 5000\pound{ ф. стерл}. Маса щороку створюваної
додаткової вартости буде така сама, як і в попередньому випадку \deq{} 5000\pound{ ф. стерл.}, але річна норма
додаткової
вартости цілком інша. Вона дорівнює спродукованій протягом року
додатковій вартості, поділеній на авансований змінний капітал:
\frac{5000m}{5000v} \deq{} 100\%, тимчасом як раніш для капіталу $А$ вона дорівнювала 1000\%.

При капіталі $А$, як і при капіталі $В$, ми витрачали щотижня 100\pound{ ф. стерл.} змінного капіталу; ступінь
зростання вартости або норма додаткової
вартости цілком та сама, вона дорівнює 100\%; величина змінного
капіталу теж та сама \deq{} 100\pound{ ф. стерл}. Експлуатується цілком таку
саму кількість робочої сили, величина й ступінь експлуатації в обох випадках
однакові, робочі дні однакові і однаково поділяються на доконечну
й додаткову працю. Сума змінного капіталу, застосованого протягом
року, однакова величиною \deq{} 5000\pound{ ф. стерл.}, вона пускає в рух таку
саму масу праці й витягує з робочої сили, пущеної в рух обома рівними
капіталами, однакову масу додаткової вартости, 5000\pound{ ф. стерл}. І,
\parbreak{}  %% абзац продовжується на наступній сторінці
