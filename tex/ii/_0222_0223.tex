\parcont{}  %% абзац починається на попередній сторінці
\index{ii}{0222}  %% посилання на сторінку оригінального видання
все ж в річній нормі додаткової вартости капіталів А і В є ріжниця в
900\%.

Правда, це явище має такий вигляд, ніби норма додаткової вартости
залежить не лише від маси та ступеня експлуатації робочої сили,
пущеної в рух змінним капіталом, але крім того, від якихось незрозумілих
впливів, що походять з процесу циркуляції; і справді, це явище
освітлювали саме таким способом; і хоч не в цій чистій, а в своїй складнішій
та прихованішій формі (в формі річної норми зиску) воно спричинило
початку 20-х років цілковите замішання в школі Рікардо.

Дивне в цьому явищі зникає одразу, скоро ми не лише позірно, а
й справді поставимо капітал А і капітал В в цілком однакові обставини.
Обставини будуть однакові тільки тоді, коли змінний капітал В в цілому
своєму об’ємі витрачається на оплату робочої сили протягом того
самого часу, що й капітал А.

5000\pound{ ф. стерл.} капіталу В витрачається тоді протягом 5 тижнів, по
1000\pound{ ф. стерл.} щотижня; це становить за рік витрату в \num{50.000}\pound{ ф. стерл}.
Додаткова вартість буде тоді, згідно з нашим припущенням, теж = \num{50.000}\pound{ ф. стерл}. Капітал, що обернувся = \num{50.000}\pound{ ф. стерл.}, поділений на авансований
капітал = 5000\pound{ ф. стерл.}, дає число оборотів = 10. Норма додаткової
вартости = \frac{5000 m}{5000 v} = 100\%, помножена на число оборотів = 10, дає річну норму додаткової
вартости = \frac{5000 m}{5000 v} = \frac{10}{1} = 1000\%. Отже, тепер річні норми додаткової вартости однакові для
А і для В, а саме
1000\%, але маси додаткової вартости становлять: для В — \num{50.000}\pound{ ф. стерл.},
для А — 5000\pound{ ф. стерл.}; маси спродукованої додаткової вартости відносяться
тепер, як авансовані капітальні вартості В і А, а саме як
$5000 : 500 = 10 : 1$. Але зате капітал В в той самий час пустив у рух удесятеро
більше робочої сили, ніж капітал А.

Тільки капітал, дійсно застосований у процесі праці, утворює додаткову
вартість і тільки для нього мають силу всі закони, що стосуються
до додаткової вартости, а значить, і той закон, що за даної норми маса
додаткової вартости визначається відносною величиною змінного капіталу.

Самий процес праці вимірюється часом. За даної довжини робочого
дня (як тут, де ми ставимо капітал А і капітал В в однакові обставини,
щоб висвітлити краще різницю в річній нормі додаткової вартости)
робочий тиждень складається з певного числа робочих днів. Або ми
можемо розглядати якийбудь робочий період, напр., в даному разі п’ятитижневий,
як суцільний робочий день, що складається з 300 годин, коли
робочий день = 10 годинам, а тиждень = 6 робочим дням. Але далі ми
мусимо помножити це число на число робітників, що їх одночасно щодня
вживається разом у тому самому процесі праці. Коли це число
було б, напр., 10, то тижневий підсумок був би = 60 × 10 = 600 годинам,
а п’ятитижневий робочий період = 600 × 5 = 3000 годинам. Отже,
при однаковій нормі додаткової вартости і при однаковій довжині робочого
\index{ii}{0223}  %% посилання на сторінку оригінального видання
дня застосовується змінні капітали однакової величини, коли протягом
того самого переміжку часу пускається в рух однакові маси робочої
сили (обчислювані помноженням однієї робочої сили тієї самої
ціни на число цих сил).

Повернімось тепер до наших первісних прикладів. В обох випадках
А і В протягом кожного тижня року застосовується змінні капітали
однакової величини, по 100\pound{ ф. ст.} щотижня. Застосовані, справді діющі
в процесі праці змінні капітали тому однакові, але авансовані змінні
капітали зовсім неоднакові. В прикладі А на кожні 5 тижнів авансовано по
500\pound{ ф. стерл.}, що з них щотижня застосовується 100\pound{ ф. стерл}. В прикладі
В на перший п’ятитижневий період треба авансувати 5000\pound{ ф. стерл.}, але
з них застосовується лише по 100\pound{ ф. стерл.} щотижня, отже, протягом
5 тижнів лише 500\pound{ ф. стерл.} = \sfrac{1}{10} авансованого капіталу. Протягом
другого п’ятитижневого періоду треба авансувати 4500\pound{ ф. стерл.}, але застосовується
тільки 500\pound{ ф. стерл.} і т. далі. Змінний капітал, авансовуваний
на певний період часу, перетворюється на застосовуваний, тобто справді
діющий і чинний змінний капітал лише тією мірою, якою він справді входить
у відділи цього періоду часу, заповнені процесом праці, якою він
дійсно функціонує в процесі праці. В переміжки, що протягом них
частину його авансовано лише для того, щоб її можна було застосувати
пізніше, ця частина мов би зовсім не існує для процесу праці, а тому
не справляє жодного впливу ні на утворення вартости, ні на утворення
додаткової вартости. Так стоїть, приміром, справа з капіталом А в
500\pound{ ф. стерл}. Його авансовано на 5 тижнів, але в процес праці послідовно
входять з нього щотижня лише 100\pound{ ф. стерл}. Протягом першого тижня
застосовується \sfrac{1}{5} його; \sfrac{4}{5} авансовано, але не застосовано, хоч їх і
треба мати в запасі для процесу праці наступних 4 тижнів, і тому їх
доводиться авансувати.

Обставини, що зумовлюють ріжницю у відношенні між авансованим і
застосованим капіталом, впливають на продукцію додаткової вартости —
за даної норми додаткової вартости — лише остільки й лише тим, що
вони роблять різною ту кількість змінного капіталу, яку дійсно можна
застосувати протягом певного періоду часу, напр., протягом одного тижня,
протягом п’ятьох тижнів та ін. Авансований змінний капітал функціонує
як змінний капітал лише остільки й лише протягом того часу, оскільки й
коли його справді застосовується; але не протягом того часу, коли він
лишається авансований як запас, і не застосовується його. Однак усі
обставини, що зумовлюють ріжницю у відношенні між авансованим і застосованим
змінним капіталом, сходять на ріжницю періодів обороту
(визначувану ріжницею або робочого періоду, або періоду циркуляції,
або їх обох). Закон продукції додаткової вартости, є в тому, що, при однаковій
нормі додаткової вартости, однакові маси діющого змінного капіталу
утворюють однакові маси додаткової вартости. Отже, коли з капіталів
А і В за однакові переміжки часу при однаковій нормі додаткової
вартости застосовується однакові маси змінного капіталу, то вони мусять
протягом однакових переміжків часу утворити однакові маси додаткової
\parbreak{}  %% абзац продовжується на наступній сторінці
