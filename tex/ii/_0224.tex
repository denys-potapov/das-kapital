\parcont{}  %% абзац починається на попередній сторінці
\index{ii}{0224}  %% посилання на сторінку оригінального видання
вартости, хоч яке різне буде відношення цього змінного капіталу, застосованого
протягом певного часу, до змінного капіталу, авансованого на
той самий час, і значить, хоч яке різне буде також і відношення утворених
мас додаткової вартости, не до застосованого, а до взагалі авансованого
змінного капіталу. Неоднаковість цього відношення, замість суперечити
розвиненим законам продукції додаткової вартости, навпаки,
потверджує їх і є неминучий наслідок їх.

Розгляньмо перший п’ятитижневий період продукції капіталу \emph{В}. Наприкінці
5-го тижня 500\pound{ ф. стерл.} застосовано й зужито. Новостворена
вартість \deq{} 1000\pound{ ф. стерл.}, отже, $\frac{500m}{500v} \deq{} 100\%$. Цілком так, як при капіталі \emph{А}.
Та обставина, що при капіталі \emph{А} додаткова вартість реалізується разом з
авансованим капіталом, а при \emph{В} — ні, нас тут покищо не обходить, бо
тут покищо йдеться лише про продукцію додаткової вартости і про
відношення її до змінного капіталу, авансованого підчас її продукції.
Навпаки, коли ми обчислимо відношення додаткової вартости \emph{В} не
до тієї частини авансованого капіталу в 5000\pound{ ф. стерл.}, що її застосовано й
тому зужито протягом продукції цієї додаткової вартости, а до самого
цього цілого авансованого капіталу, то матимемо $\frac{500m}{5000v} \deq{} \sfrac{1}{10} \deq{} 10\%$.
Отже, для капіталу \emph{В} 10\%, а для капіталу \emph{А} 100\%, тобто вдесятеро
більше. Коли б тут сказали: така ріжниця в нормі додаткової вартости
для однакових величиною капіталів, що пускають у рух однакову кількість
праці, та ще праці, яка однаковою мірою поділяється на оплачену й
неоплачену, суперечить законам продукції додаткової вартости, — то
відповідь була б проста й випливала б з першого погляду на фактичні
відношення: для \emph{А} виражається справжня норма додаткової вартости,
тобто відношення додаткової вартости, спродукованої протягом 5 тижнів
змінним капіталом в 500 ф., до цього змінного капіталу в 500\pound{ ф. стерл}.
Для \emph{В}, навпаки, обчислення робиться таким способом, що не має жодного
чинення ні до продукції додаткової вартости, ні до відповідного їй
визначення норми додаткової вартости. 500\pound{ ф. стерл.} додаткової вартости,
спродуковані змінним капіталом у 500\pound{ ф. стерл.}, обчислюється
власне не в їхньому відношенні до 500\pound{ ф. стерл.} змінного капіталу, авансованого
протягом продукції цієї додаткової вартости, а в їхньому відношенні
до капіталу в 5000\pound{ ф. стерл.}, що \sfrac{9}{10} його, 4500\pound{ ф. стерл.}, не мають
жодного чинення до продукції цієї додаткової вартости в 500\pound{ ф. стерл.}, а
скорше мають лише поступінно функціонувати протягом наступних 45 тижнів;
отже, вони зовсім не існують для продукції протягом перших 5 тижнів,
що про них тільки й мовиться тут. Отже, в цьому випадку ріжниця
в нормі додаткової вартости капіталів \emph{А} і \emph{В} не становить жодної
проблеми.

Порівняймо тепер річні норми додаткової вартости для капіталів \emph{В} і \emph{А}.
Для капіталу \emph{В} ми маємо $\frac{5000m}{5000v} \deq{} 100\%$;
для капіталу \emph{А} $\frac{5000m}{500v} \deq{} 1000\%$.
\parbreak{}  %% абзац продовжується на наступній сторінці
