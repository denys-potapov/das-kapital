\parcont{}  %% абзац починається на попередній сторінці
\index{ii}{0225}  %% посилання на сторінку оригінального видання
Але відношення норм додаткової вартости таке саме, як і раніш. Там ми мали:
$\frac{\text{Норма додаткової вартости капіталу }В}{\text{Норма додаткової вартости капіталу }А} \deq{}
\frac{10\%}{100\%}
$, а тепер ми маємо:
$\frac{\text{Річна норма додаткової вартости капіталу }В}{\text{Річна норма додаткової вартости капіталу }А} \deq{}
\frac{100\%}{1000\%}
$, але $\frac{10\%}{100\%} \deq{} \frac{100\%}{ 1000\%}$, отже, те саме відношення, як і перше.

Однак, тепер проблема перевернулася. Річна норма капіталу \emph{В}:
$\frac{5000m}{5000v} \deq{} 100\%$, звичайно, не має жодного відхилення — ні навіть
подоби відхилення — від відомих нам законів щодо продукції та відповідної
їй норми додаткової вартости. $5000v$ протягом року авансовано й
продуктивно спожито, вони спродукували $5000m$. Отже, норма додаткової
вартости позначиться наведеним вище дробом $\frac{5000m}{5000v} \deq{} 100\%$. Річна
норма збігається з дійсною нормою додаткової вартости. Отже, на цей
раз не капітал \emph{В}, як перше, а капітал \emph{А} являє аномалію, що її треба
пояснити.

Ми маємо тут норму додаткової вартости $\frac{5000m}{5000v} \deq{} 1000\%$.

Але якщо в першому випадку $500m$, продукт 5 тижнів, обчислювалося
на авансований капітал 5000\pound{ ф. стерл.}, що \sfrac{9}{10} його не застосовувалось
у продукції цієї додаткової вартости, то тепер $5000m$ обчислюється на $500v$,
тобто, тільки на \sfrac{1}{10} змінного капіталу, справді, застосованого в продукції
$5000m$; бо $5000m$ є продукт змінного капіталу в 5000\pound{ ф. стерл.}, продуктивно
зужитого протягом 50 тижнів, а не продукт капіталу в 500\pound{ ф. стерл.},
зужитого протягом лише одного п’ятитижневого періоду. В першому випадку
додаткова вартість, спродукована протягом 5 тижнів, обчислюється
на капітал, авансований на 50 тижнів, отже, на капітал, вдесятеро більший,
ніж той, що його зужитковано протягом 5 тижнів. Тепер додаткова вартість,
спродуковану протягом 50 тижнів, обчислюється на капітал, авансований
на 5 тижнів, отже, на капітал, вдесятеро менший, ніж той, що
його зужитковано протягом 50 тижнів.

Капітал \emph{А} в 500\pound{ ф. стерл.} ніколи не авансується більше, ніж на 5 тижнів.
Наприкінці він припливає назад і, роблячи протягом року десять
оборотів, може повторити цей процес десять разів. Із цього випливають
два висновки.

\so{Поперше}, авансований капітал \emph{А} лише вп’ятеро більший за частину
капіталу, постійно застосовувану в процесі продукції протягом
тижня. Навпаки, капітал \emph{В}, що протягом 50 тижнів обертається лише
один раз, і, значить, його теж треба авансувати на 50 тижнів, в 50 разів
більший за ту частину його, що її постійно можна застосовувати протягом
тижня. Тому оборот змінює відношення між капіталом, авансованим
на процес продукції протягом року, і капіталом, постійно застосовуваним
для певного періоду продукції, напр., одного тижня. Ми маємо
\parbreak{}  %% абзац продовжується на наступній сторінці
