\parcont{}  %% абзац починається на попередній сторінці
\index{ii}{0226}  %% посилання на сторінку оригінального видання
тут перший випадок, де п’ятитижневу додаткову вартість обчислюється
не на капітал, застосований протягом цих 5 тижнів, а на капітал вдесятеро
більший, застосований протягом 50 тижнів.

\so{Подруге}. П’ятитижневий період обороту капіталу $А$, 5 тижнів,
становить тільки \sfrac{1}{10} року, отже, рік охоплює 10 таких періодів обороту,
що в них капітал $А$ в 500\pound{ ф. стерл.} знову й знову застосовується.
Застосований капітал дорівнює тут капіталові, авансованому на 5 тижнів,
помноженому на число періодів обороту протягом року. Капітал, застосований
протягом року $\deq{} 500 × 10 \deq{}$ 5000\pound{ ф. стерл}. Капітал, авансований
протягом року \deq{} \frac{5000}{10} \deq{} 500\pound{ ф. стерл}. Справді, хоч 500\pound{ ф. стерл.} завжди
знову застосовується, але що п’ять тижнів ніколи не авансується більше,
ніж ці самі 500\pound{ ф. стерл}. З другого боку, при капіталі $В$ протягом 5 тижнів
застосовується й авансується на ці 5 тижнів лише 500\pound{ ф. стерл.}; а що
період обороту дорівнює тут 50 тижням, то капітал, застосований протягом
року, дорівнює капіталові, авансованому не на кожні 5 тижнів, а на
всі 50 тижнів. Але щорічно продукована маса додаткової вартости, за
даної норми додаткової вартости, пропорціональна капіталові, застосованому
протягом року, а не капіталові, авансованому протягом року. Отже, для
цього капіталу в 5000\pound{ ф. стерл.}, що обертається один раз, ця маса не
більша, ніж для капіталу в 500\pound{ ф. стерл.}, що обертається десять разів, і
вона лише тому така велика, що капітал, який обертається один раз на
рік, сам удесятеро більший від капіталу, який обертається десять разів
на рік.

Змінний капітал, що обертається протягом року, — отже, частина річного
продукту, або рівна їй частина річних витрат, — є змінний капітал,
справді застосований, продуктивно зужитий протягом року. З цього випливає,
що коли змінний капітал $А$, який обертається протягом року, і
змінний капітал $В$, який обертається протягом року, рівновеликі, і коли
їх застосовується в однакових умовах зростання вартости, отже, коли норма
додаткової вартости однакова для обох, то й щороку продукована
маса додаткової вартости також мусить бути однакова для обох; отже,
однакова мусить бути — тому, що застосовані маси капіталу однакові — і
норма додаткової вартости, обчислена на рік, оскільки її виражається
формулою:\[
\frac{\text{маса додаткової вартости, спродукованої протягом року}}{\text{змінний капітал, який обертається протягом
року}}
\]

\noindent{}Або, виражаючи в загальній формі: хоч яка буде відносна величина
змінних капіталів, що обернулись, норма їхньої додаткової вартости,
спродукованої протягом року, визначається тією нормою додаткової вартости,
за якої відповідні капітали працювали в пересічні періоди (прим.,
у пересічному за тиждень або день).

Такий єдиний висновок, що випливає з законів продукції додаткової
вартости та визначення норми додаткової вартости.
