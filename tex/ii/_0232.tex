\parcont{}  %% абзац починається на попередній сторінці
\index{ii}{0232}  %% посилання на сторінку оригінального видання
періоду обороту на робочу силу вже витрачено 500\pound{ ф. стерл}. Ці 500\pound{ ф. стерл.},
що первісно були частиною всього авансованого капіталу, перестали бути
капіталом. Їх витрачено на заробітну плату. Робітники, з свого боку,
виплачують ними, купуючи собі засоби існування; отже, вони споживають
засоби існування вартістю на 500\pound{ ф. стерл}. Отож знищено товарову
масу на таку саму суму вартости (те, що робітник зберігає, напр., як
гроші тощо, теж не є капітал). Цю товарову масу спожито непродуктивно
для робітника, оскільки вона тільки підтримує в стані працездатности
його робочу склу, тобто неодмінно потрібне знаряддя капіталіста. Але,
подруге, для капіталіста ці 500\pound{ ф. стерл.} перетворено на робочу силу тієї
самої вартости (зглядно ціни). Капіталіст продуктивно споживає робочу
силу в процесі праці. Наприкінці 5 тижня є новоспродукована вартість
в 1000\pound{ ф. стерл}. Половина її, 500\pound{ ф. стерл.}, є репродукована вартість змінного
капіталу, витраченого на оплату робочої сили. Друга половина її,
500\pound{ ф. стерл.} є новоспродукована додаткова вартість. Але ту п’ятитижневу
робочу силу, що в наслідок обміну на неї частина капіталу перетворилась
на змінний капітал, так само витрачено, зужито, хоч і продуктивно.
Праця, що діяла вчора, не є та сама праця, що діє сьогодні. Її
вартість плюс утворена нею додаткова вартість існує тепер як вартість
продукту, речі, відмінної від самої робочої сили. Однак, у наслідок того,
що продукт перетворюється на гроші, частина його вартости, що дорівнює
вартості авансованого змінного капіталу, знову може бути
обмінена на робочу силу, а тому й знову функціонувати як змінний капітал.
Та обставина, що на капітальну вартість, не лише репродуковану,
а й перетворену вже на грошову форму, вживатиметься тих самих робітників,
тобто тих самих носіїв робочої сили, не має жодного значення.
Можливо, що протягом другого періоду обороту капіталіст вживатиме
нових робітників замість старих.

Отже, у дійсності протягом 10 п’ятитижневих періодів обороту на
заробітну плату послідовно витрачається капітал в 5000\pound{ ф. стерл.}, а
не 500\pound{ ф. стерл.}, і цю заробітну плату знову витрачають робітники
на засоби існування. Авансований таким чином капітал в 5000\pound{ ф. стерл.} зужито. Він уже не існує. З другого боку, в процес продукції
послідовно вводиться робочу силу вартістю не в 500, а в 5000\pound{ ф. стерл.},
і вона репродукує не лише свою власну вартість, рівну 5000\pound{ ф. стерл.},
але продукує надлишок, додаткову вартість в 5000\pound{ ф. стерл}. Змінний
капітал в 500\pound{ ф. стерл.}, що авансується на другий період обороту, не тотожній
з капіталом в 500\pound{ ф. стерл.}, що його авансовано в перший період
обороту. Цей останній зужито, витрачено на заробітну плату. Але його
\so{заміщено} новим змінним капіталом в 500\pound{ ф. стерл.}, що його в перший
період обороту спродуковано в формі товару і зворотно перетворено
на грошову форму. Отже, цей новий грошовий капітал в 500\pound{ ф. стерл.} є
грошова форма товарової маси, новоспродукованої в перший період обороту.
Та обставина, що в руках капіталіста знову є тотожня грошова
сума в 500\pound{ ф. стерл.}, тобто, лишаючи осторонь додаткову вартість, саме
стільки грошового капіталу, скільки він первісно авансував, замасковує
\parbreak{}  %% абзац продовжується на наступній сторінці
