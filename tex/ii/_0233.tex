\parcont{}  %% абзац починається на попередній сторінці
\index{ii}{0233}  %% посилання на сторінку оригінального видання
ту обставину, що капіталіст оперує новоспродукованим капіталом (щодо
інших складових частин вартости товароваго капіталу, які заміщують
сталі частини капіталу, то їхня вартість не продукується наново, а змінюється
лише форма, що в ній існує ця вартість). Візьмімо третій період
обороту. Тут очевидно, що капітал в 500\pound{ ф. стерл.}, авансований втретє,
не є старий, а новоспродукований капітал, бо він є грошова форма
товарової маси, спродукованої в другий, а не в перший період обороту,
тобто грошова форма тієї частини цієї товарової маси, вартість якої
дорівнює вартості авансованого змінного капіталу. Масу товарів, спродуковану
в перший період обороту, продано. Частину її вартости, що дорівнює
змінній частині авансованого капіталу, обмінено на нову робочу
силу для другого періоду обороту; ця частина випродукувала нову масу
товарів, що її теж продано; саме частина вартости цієї нової маси
товарів становить капітал в 500\pound{ ф. стерл.}, авансовуваний у третій період
обороту.

І це повторюється протягом 10 періодів обороту. Протягом їх новоспродуковані
маси товарів (вартість яких, оскільки вона заміщує змінний
капітал, теж знову продукується, а не просто з’являється, як з’являється
вартість сталої обігової частини капіталу), що п’ять тижнів подається
на ринок, щоб знову вводити робочу силу в процес продукції.

Отже, десятиразовим оборотом авансованого змінного капіталу в 500\pound{ ф.
стерл.} досягається не те, що цей капітал в 500\pound{ ф. стерл.} можна продуктивно
спожити десять разів, або, що змінний капітал, достатній для
п’ятьох тижнів, можна застосувати протягом 50 тижнів. Навпаки, за 50
тижнів застосовується 500\pound{ ф. стерл}. X 10 змінного капіталу, і капіталу в
500\pound{ ф. стерл.} завжди вистачає тільки на 5 тижнів, а по п’ятьох тижнях
його доводиться заміщувати новоспродукованим капіталом в 500\pound{ ф. стерл}.
Це стосується до капіталу \emph{А} цілком так само, як і до капіталу \emph{В}. Але
відси починається ріжниця.

На кінець першого п’ятитижневого періоду від капіталіста \emph{В}, як і
від капіталіста \emph{А}, авансовано й витрачено змінний капітал в 500\pound{ ф. стерл}.
І \emph{В} і \emph{А} перетворили його вартість на робочу силу й замістили цю вартість
тією частиною новоспродукованої цією робочою силою вартости продукту,
яка дорівнює вартості авансованого змінного капіталу в 500\pound{ ф. стерл}. Для
\emph{В}, як і для \emph{А}, робоча сила не лише замістила вартість витраченого змінного
капіталу в 500\pound{ ф. стерл.} новою вартістю на таку саму суму, а й додала
до неї додаткову вартість, — за нашим припущенням вартість такої
самої величини.

Але та новоутворена вартість, що заміщує авансований змінний капітал
і додає до його вартости додаткову вартість, перебуває у \emph{В} не в тій
формі, що в ній вона може знову функціонувати як продуктивний капітал,
зглядно як змінний капітал. Для \emph{А} вона перебуває саме в такій формі.
І до кінця року \emph{В} володіє змінним капіталом, витрачуваним протягом
перших 5 тижнів і потім послідовно витрачуваним що п’ять тижнів, — хоч
його й заміщується новоспродукованою вартістю плюс додаткова вартість,
— не в тій формі, що в ній він знову може функціонувати
\parbreak{}  %% абзац продовжується на наступній сторінці
