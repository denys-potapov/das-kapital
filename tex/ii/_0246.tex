\parcont{}  %% абзац починається на попередній сторінці
\index{ii}{0246}  %% посилання на сторінку оригінального видання
безпосередній або посередній обмін частини річного продукту країни на
продукт країн, що продукують золото й срібло. Однак такий інтернаціональний
характер оборудки замасковує її простоту. Тому, щоб звести
проблему до найпростішого та найвиразнішого виразу, треба припустити,
що продукція золота й срібла відбувається в самій країні, отже, що продукція
золота й срібла становить частину сукупної суспільної продукції
кожної країни.

Лишаючи осторонь золото й срібло, продуковані для речей розкошів,
мінімум щорічної продукції їх мусить дорівнювати зношуванню грошового
металю, зумовленому річною грошовою циркуляцією. Далі: коли
зростає сума вартости маси товарів, яка щорічно продукується й циркулює,
то мусить зростати й річна продукція золота й срібла, оскільки
виросла сума вартости товарів, що циркулюють, і маса грошей, потрібних
для їхньої циркуляції (та для утворення відповідного скарбу), не компенсується
більшою швидкістю грошового обігу та поширенішою функцією
грошей як засобу виплати, тобто частішими взаємними вирівнюваннями
купівель і продажів без посередництва дійсних грошей.

Отже, частину суспільної робочої сили та частину суспільних засобів
продукції треба щороку витрачати на продукцію золота й срібла.

Капіталісти, які провадять продукцію золота й срібла, провадять її, —
як ми тут, за умов простої репродукції, припускаємо — лише в межах
пересічного річного зношування та зумовленого ким пересічного річного
споживання золота й срібла; свою додаткову вартість, що її вони, згідно
з нашим припущенням, споживають щорічно, нічого не капіталізуючи з
неї, вони пускають у циркуляцію безпосередньо в грошовій формі, яка
для них є натуральна форма, а не перетворена форма продукту, як в інших
галузях продукції.

Далі: щодо заробітної плати — грошової форми, що в ній авансується
змінний капітал — то тут її так само заміщується не через продаж
продукту, не через перетворення його на гроші, а самим продуктом, що
натуральна форма його з самого початку є грошова форма.

Нарешті, так само стоїть справа і з тією частиною продукту благородного
металю, яка дорівнює вартості періодично споживаного сталого
капіталу, так сталого обігового, як і сталого основного, споживаного протягом
року.

Розгляньмо кругобіг, зглядно оборот, капіталу, вкладеного в продукцію
благородних металів, насамперед у формі $Г — Т\dots{} П\dots{} Г'$. Оскільки
$Т$ в $Г — Т$ складається не лише з робочої сили та засобів продуції, а також
із основного капіталу, що з нього в $П$ споживається тільки частину
його вартости, то очевидно, що $Г'$ — продукт — є грошова сума, яка
дорівнює змінному капіталові, витраченому на заробітну плату, плюс обіговий
сталий капітал, витрачений на засоби продукції, плюс частина
вартости, яка відповідає зношуванню основного капіталу, плюс додаткова
вартість. Коли б, при незмінній загальній вартості золота, ця сума була
менша, то вкладення капіталу в золоті копальні було б непродуктивне
або, — коли б це явище набрало загального характеру, — то вартість золота
\parbreak{}  %% абзац продовжується на наступній сторінці
