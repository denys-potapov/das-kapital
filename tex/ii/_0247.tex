\parcont{}  %% абзац починається на попередній сторінці
\index{ii}{0247}  %% посилання на сторінку оригінального видання
в майбутньому зросла б порівняно з товарами, котрих вартість не змінюється;
тобто ціни товарів знизились би, отже, в майбутньому грошова
сума, витрачувана в акті $Г — Т$, зменшилась би.

Якщо розглядати насамперед тільки обігову частину капіталу, авансовуваного
в $Г$, вихідному пунктові $Г — Т\dots{} П\dots{} Г'$, то виявиться, що певну
грошову суму авансується, пускається в циркуляцію на оплату робочої
сили й на закуп матеріялів продукції. Але через кругобіг цього капіталу
її не вилучається знову з циркуляції, щоб знову ж таки подати її
в неї. Продукт є гроші вже в своїй натуральній формі, отже, йому не
доводиться перетворюватись на гроші через обмін, через процес
циркуляції. З процесу продукції в сферу циркуляції він увіходить не в
формі товарового капіталу, що повинен перетворитись на грошовий капітал,
а як грошовий капітал, що повинен перетворитись знову на продуктивний
капітал, тобто знову купувати робочу силу й матеріяли продукції.
Грошову форму обігового капіталу, зужитого на робочу силу й на засоби
продукції, заміщується не через продаж продукту, а натуральною
формою самого продукту; отже, не через зворотне вилучення з циркуляції
вартости продукту в її грошовій формі, а через додаткові новоспродуковані
гроші.

Припустімо, що цей обіговий капітал \deq{} 500\pound{ ф. стерл.}, період обороту
\deq{} 5 тижням, робочий період \deq{} 4 тижням, період циркуляції \deq{} лише 1
тижневі. Гроші з самого початку треба авансувати на 5 тижнів почасти
на продукційний запас, почасти вони мають бути в запасі, щоб можна
було поступінно виплачувати з них заробітну плату. На початку шостого
тижня повертаються назад 400\pound{ ф. стерл.} і звільняються 100\pound{ ф. стерл}. Це
повторюється раз-у-раз. Тут, як і раніш, протягом певного часу обороту
100\pound{ ф. стерл.} завжди будуть в формі вільних грошей. Але вони складаються
з додаткових новоспродукованих грошей, цілком так само, як і останні
400\pound{ ф. стерл}. Ми маємо тут 10 оборотів на рік і спродукований річний
продукт \deq{} 5000\pound{ ф. стерл.} золотом. (Період циркуляції складається тут
не з того часу, що його потребує перетворення товару на гроші, а з
часу, потрібного для перетворення грошей на елементи продукції).

Для всякого іншого капіталу в 500\pound{ ф. стерл.}, що обертається в таких
самих умовах, раз-у-раз відновлювана грошова форма є перетворена форма
спродукованого товарового капіталу, який що чотири тижні подається
в циркуляцію і який у наслідок продажу — отже, в наслідок періодичного
вилучення такої кількости грошей, яка спочатку ввійшла в процес, —
знову й знову набирає цієї грошової форми. Навпаки, тут в кожний період
обороту з самого процесу продукції подається в циркуляцію нову
додаткову суму грошей в 500\pound{ ф. стерл.} для того, щоб постійно вилучати
відти матеріяли продукції та робочу силу. Цих поданих у циркуляцію
грошей не вилучається знову з неї через кругобіг цього капіталу, а їх
дедалі збільшують, додаючи раз-у-раз новоспродуковані маси золота.

Коли ми розглянемо змінну частину цього обігового капіталу й припустимо,
як і перше, що вона дорівнює 100\pound{ ф. стерл.}, то при звичайній
товаровій продукції цих 100\pound{ ф. стерл.} при десятьох оборотах на рік було б
\parbreak{}  %% абзац продовжується на наступній сторінці
