
\index{ii}{0252}  %% посилання на сторінку оригінального видання
Загальну відповідь уже дано: коли має циркулювати маса товарів в
1000\pound{ ф. стерл.}, × X, то величина грошової суми, потрібної для цієї циркуляції,
абсолютно не змінюється від того, чи є в вартості цієї маси товарів
додаткова вартість, чи немає, чи випродукувано цю товарову масу
капіталістично, чи ні. Отже, самої проблеми не існує. За інших
даних умов, швидкости грошової циркуляції та ін., для циркуляції товарової
вартости в 1000\pound{ ф. стерл.} × Х, потрібна певна сума грошей, яка
зовсім не залежить від тієї обставини, чи багато, чи мало з цієї вартосги
припадає безпосереднім продуцентам цих товарів. Оскільки тут і існує
проблема, вона збігається з загальною проблемою: відки береться сума
грошей, потрібна для циркуляції товарів у даній країні.

А проте, з погляду капіталістичної продукції, існує, звичайно, подоба
якоїсь особливої проблеми. А саме за вихідний пункт, відки гроші пускається
в циркуляцію, тут виступає капіталіст. Гроші, що їх витрачає
робітник на оплату засобів свого існування, існують спочатку як грошова
форма змінного капіталу, і тому капіталіст їх спочатку пускає в
циркуляцію як купівельний або виплатний засіб за робочу силу. Крім
того, капіталіст пускає в циркуляцію гроші, що спочатку становили для
нього грошову форму його сталого — основного й поточного — капіталу;
він витрачає їх як купівельний або виплатний засіб на засоби праці та
матеріяли продукції. Але поза цим капіталіст уже не виступає як вихідний
пункт грошової маси, що перебуває в циркуляції.

Але взагалі існують тільки два вихідні пункти: капіталіст і робітник.
Треті особи всіх категорій або мусять одержувати гроші від цих двох
кляс за якібудь послуги, або оскільки вони одержують гроші без якихбудь
послуг з їхнього боку, вони є співвласники додаткової вартости в
формі ренти, проценту й т. ін. Те, що додаткова вартість не лишається
цілком в кишені промислового капіталіста, й що він мусить поділитися
нею з іншими особами, не має жодного чинення до нашого питання.
Питання в тому, яким чином він перетворює на гроші свою додаткову
вартість, а не в тому, як розподіляються потім здобуті за неї гроші.
Отже, в даному разі ми все ще повинні розглядати капіталіста як єдиного
власника додаткової вартости. Щождо робітника, то вже сказано, що
він є тільки вторинний вихідний пункт, але капіталіст є первинний
вихідний пункт тих грошей, що їх пускає в циркуляцію робітник. Гроші,
спочатку авансовані як змінний капітал, пророблюють уже свій другий
обіг, коли робітник витрачає їх на оплату засобів існування.

Отже, кляса капіталістів лишається єдиним вихідним пунктом грошової
циркуляції. Коли їй треба на оплату засобів продукції 400\pound{ ф. стерл.}
і на оплату робочої сили 100\pound{ ф. стерл.}, то вона пускає в циркуляцію
500\pound{ ф. стерл}. Але додаткова вартість, що міститься в продукті, при нормі
додаткової вартости в 100\%, дорівнює вартості в 100\pound{ ф. стерл}. Як
же вона може постійно вилучати з циркуляції 600\pound{ ф. стерл.}, коли
вона постійно пускає в неї лише 500\pound{ ф. стерл.}? З нічого нічого й не
буде. Ціла кляса капіталістів не може вилучати з циркуляції нічого такого,
чого раніш не було пущено в неї.
