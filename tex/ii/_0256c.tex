\parcont{}  %% абзац починається на попередній сторінці
\index{ii}{0256}  %% посилання на сторінку оригінального видання
грошей, що змінними порціями перебуває завжди в руках кляси капіталістів
як грошова форма їхньої додаткової вартости, не є елемент
щорічно продукованого золота, а елемент маси грошей, раніш акумульованих
у країні.

Згідно з нашим припущенням, річної продукції золота в 500\pound{ ф. стерл.}
досить саме лише для того, щоб заміщувати щорічне зношування грошей.
Тому, коли ми матимемо на увазі тільки ці 500\pound{ ф. стерл.} і абстрагуємось
від тієї частини щорічно продукованої маси товарів, що її циркуляцію
обслуговують раніш акумульовані гроші, то додаткова вартість, спродукована
в товаровій формі, уже тому знаходить в циркуляції гроші для
свого перетворення на гроші, що на другому боці щороку продукується
додаткову вартість у формі золота. Це саме має силу й щодо інших частин
продукту-золота в 500\pound{ ф. стерл.}, що заміщують авансований грошовий
капітал.

Тут треба тепер зробити два зауваження.

З наведеного вище випливає, поперше: додаткова вартість, витрачувана
капіталістами у формі грошей, так само, як і змінний та інший
продуктивний капітал, що його вони авансують в формі грошей, в
дійсності є продукт робітників, а саме робітників, що працюють у
золотопромисловості. Вони знову продукують так ту частину продукту-золота,
що її „авансується“ їм як заробітну плату, як і ту частину
продукту-золота, що безпосередньо репрезентує додаткову вартість
капіталістів-продуцентів золота. Нарешті, щодо тієї частини продукту-золота,
яка лише покриває сталу капітальну вартість, авансовану на
продукцію цього продукту-золота, то вона знову з’являється в грошовій
формі (взагалі в продукті) лише в наслідок річної праці робітників.
При заснуванні підприємства капіталіст віддав її спочатку в вигляді
грошей, не новоспродукованих, а тих, що становили частину маси грошей,
яка циркулювала у суспільстві. Навпаки, оскільки її заміщується новим
продуктом, додатковим золотом, вона вже є річний продукт робітника.
Те, що її авансує капіталіст, — це й тут є лише форма, яка випливає з
того, що робітник не є власник своїх власних засобів продукції й не
має в своєму розпорядженні підчас продукції засобів існування, спродукованих
іншими робітниками.

Але, подруге, щодо тієї маси грошей, яка існує незалежно від цього
річного заміщення в 500\pound{ ф. стерл.} і перебуває почасти в формі скарбу,
почасти в формі грошей, що циркулюють, то з нею справа мусить
стояти, — тобто первісно мусила стояти — цілком так само, як щорічно
стоїть справа з цими 500\pound{ ф. стерл}. До цього пункту ми повернемось
наприкінці цього підвідділу. Але перше зробимо ще кілька зауважень.

\pfbreak

Досліджуючи оборот, ми бачим, що за інших незмінних умов, коли
змінюється величина періодів обороту, змінюється й маса грошового
капіталу, потрібного для того, щоб провадити продукцію в тому самому
маштабі. Отже, елястичність грошової циркуляції мусить бути досить
\parbreak{}  %% абзац продовжується на наступній сторінці
