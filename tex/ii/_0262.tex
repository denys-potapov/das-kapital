
\index{ii}{0262}  %% посилання на сторінку оригінального видання
Але в наслідок додаткового продуктивного капіталу в циркуляцію
подається, як продукт його, додаткову товарову масу. Разом з цією
додатковою товаровою масою подається в циркуляцію частину додаткових
грошей, потрібних для реалізації її — а саме подається остільки, оскільки
вартість цієї товарової маси дорівнює вартості продуктивного капіталу,
зужиткованого на її продукцію. Цю додаткову масу грошей авансується
прямо як додатковий грошовий капітал, і тому він зворотно припливає
до капіталіста в наслідок обороту його капіталу. Тут перед нами знову
постає те саме питання, що й раніш. Звідки беруться додаткові гроші на
реалізацію додаткової вартости, що є тепер у товаровій формі в цій
додатковій масі товарів?

Загальна відповідь знову та сама. Сума цін товарової маси, яка циркулює,
збільшилась не тому, що ціна даної товарової маси підвищилась,
а тому, що маса товарів, які тепер циркулюють, більша за масу товарів,
що циркулювали раніше, і при цьому ця ріжниця не вирівнюється зниженням
цін. Додаткові гроші, потрібні для циркуляції цієї більшої товарової
маси більшої вартости, треба здобути або посиленою економією на
масі грошей, що циркулюють, — чи то через взаємне вирівнювання платежів
тощо, чи то засобами, які прискорюють обіг тієї самої монети, —
або їх треба здобути через перетворення грошей з форми скарбу на
обігову форму грошей. Останнє включає не лише те, що бездіяльний
грошовий капітал починає функціонувати як купівельний засіб
або як засіб виплати; або і не лише те, що грошовий капітал, який уже
функціонує як резервний фонд, виконуючи для свого власника функцію
резервного фонду, активно циркулює для суспільства (як от банкові
вклади, що їх завжди дається в позику), отже, виконує двоїсту функцію;
це перетворення включає й те, що заощаджується стагнаційні монетні
резервні фонди.

„Щоб гроші постійно обігали як монети, монети мусять постійно
осідати як гроші. Постійний обіг монет зумовлено тим, що їх постійно
затримується більшими або меншими кількостями як монетні резервні
фонди, що всюди утворюються в межах циркуляції й зумовлюють її, —
монетні резервні фонди, що їх утворення, розподіл, розпад і нове утворення
завжди чергуються, резервні фонди; що буття їх постійно зникає,
що процес їх зникання ніколи не припиняється. Це безперестанне перетворення
монет на гроші й грошей на монети А.~Сміс висловив таким
чином, що кожен товаровласник поряд з тим особливим товаром, що
його він продає, завжди мусить мати в запасі певну суму загального
товару, що на нього він купує. Ми бачили, що в циркуляції $Т — Г — Т$
другий член $Г — Т п$остійно розпадається на ряд актів купівлі, які відбуваються
не одноразово, а послідовно в часі, так що одна частина Г
обігає як монета, тимчасом як друга перебуває в стані спокою як гроші.
Тут гроші справді є лише монети, що їхнє функціонування відкладено,
і окремі складові частини монетної маси, що обігає, завжди з’являються,
чергуючися, то в одній, то в другій формі. Тому, це перше перетворення
засобу циркуляції на гроші являє собою лише технічний момент самого
\parbreak{}  %% абзац продовжується на наступній сторінці
