\parcont{}  %% абзац починається на попередній сторінці
\index{ii}{0263}  %% посилання на сторінку оригінального видання
грошового обігу“. (Карл Маркс, „До критики політичної економії“
1859~\abbr{р.}, стор. 105--106. — Вираз „монета“ протилежно до грошей вжито
тут на позначення грошей у їхній функції як простого засобу циркуляції,
протилежно до інших їхніх функцій).

Коли всіх цих засобів не досить, то доводиться додатково продукувати
золото або — що сходить на те саме — частину додаткового продукту
обмінюється безпосередньо або посередньо на золото, на продукт країн,
що продукують благородні металі.

Вся сума робочої сили та суспільних засобів продукції, витрачуваних
на щорічну продукцію золота й срібла як знаряддя циркуляції, становить
чималу частину faux frais капіталістичного способу продукції і взагалі
способу продукції, що ґрунтується на товаровій продукції. Ця продукція
відтягує від суспільного користання відповідну суму можливих,
додаткових засобів продукції та споживання, тобто справжнього багатства.
Оскільки за незмінного даного маштабу продукції або за даного ступеня
її поширення меншають витрати на цей дорогий механізм циркуляції,
остільки ж підвищується в наслідок цього продуктивна сила суспільної
праці. Отже, оскільки так впливають допоміжні засоби, що розвиваються
разом з кредитовою системою, вони безпосередньо збільшують капіталістичне
багатство, або тим, що більшу частину процесу суспільної продукції
та процесу суспільної праці провадиться в наслідок цього без
якоїбудь інтервенції справжніх грошей, або тим, що підвищується
функціональну спроможність грошової маси, яка справді функціонує.

Цим розв’язується безглузде питання про те, чи можлива була б
капіталістична продукція в її теперішніх розмірах без кредитової системи
(коли навіть розглядати її тільки з цього погляду), тобто при самій
металевій циркуляції. Очевидно, ні. Навпаки, вона була б обмежена розміром
продукції благородних металів. З другого боку, не треба складати собі
містичних уявлень про продуктивну силу кредитової системи, оскільки
вона дає в розпорядження грошовий капітал або пускає його в рух.
Однак дальший розвиток цього сюди не стосується.
\pfbreak
Тепер ми повинні розглянути той випадок, коли відбувається не
справжня акумуляція, тобто безпосереднє поширення розмірів продукції,
а коли частину реалізованої додаткової вартости на більш-менш довгий
час акумулюється як грошовий резервний фонд, щоб пізніше перетворити
його на продуктивний капітал.

Оскільки гроші, що їх акумулюється таким чином, є додаткові гроші,
справа сама собою зрозуміла. Вони можуть бути лише частиною надлишкового
золота, довезеного з країн, що продукують золото. При цьому
треба зазначити, що в країні немає вже того національного продукту,
що за нього довезено це золото. Його віддано за кордон в обмін на
золото.

Навпаки, коли припустити, що в країні лишається, як і раніш, та сама
маса грошей, то нагромаджувані гроші припливають з циркуляції; змінюється
\index{ii}{0264}  %% посилання на сторінку оригінального видання
лише їхня функція. З грошей, що циркулюють, вони перетворюються
на лятентний грошовий капітал, що поступінно утворюється.

Гроші, нагромаджувані при цьому, є грошова форма проданих товарів,
а саме форма тієї частини їхньої вартости, яка репрезентує для їхніх
власників додаткову вартість. (Тут припускається, що кредитова система
не існує). Капіталіст, що нагромадив ці гроші, pro tanto продавав, не
купуючи.

Коли уявити собі цей процес, як окремий випадок, а не як загальний,
то він не потребує жодних пояснень. Частина капіталістів затримує частину
грошей, вторгованих від продажу своїх продуктів, не купуючи на
них продукту на ринку. Навпаки, друга частина капіталістів перетворює
на продукт усі свої гроші за винятком потрібного для продукції грошового
капіталу, що завжди повертається. Частина продукту, що її, як носія
додаткової вартости, подається на ринок, складається з засобів продукції
або з реальних елементів змінного капіталу, з доконечних засобів існування.
Отже, вона може одразу придатись для поширення продукції.
Ми бо зовсім не припускаємо, що одна частина капіталістів нагромаджує
грошовий капітал, у той час, як друга частина цілком споживає всю свою
додаткову вартість; ми лише припускаємо, що одна частина капіталістів
провадить свою акумуляцію у грошовій формі, утворює лятентний грошовий
капітал, тимчасом як друга справді акумулює, тобто поширює розміри
продукції, справді збільшує свій продуктивний капітал. Маси наявних
грошей завжди досить для потреб циркуляції, коли навіть по черзі одна
частина капіталістів акумулює гроші, тимчасом як друга частина поширює
маштаб продукції, і навпаки. Крім того, нагромадження грошей на одному
боці може відбуватись і без наявних грошей, шляхом самого лише нагромадження
боргових вимог.

Але труднощі постають тоді, коли ми припускаємо акумуляцію грошового
капіталу не як окремий випадок, а як загальну акумуляцію грошового
капіталу в кляси капіталістів. Згідно з нашим припущенням —
загальне й виключне панування капіталістичної продукції — поза цією
клясою взагалі немає жодних інших кляс, крім робітничої кляси. Все,
що купує робітнича кляса, дорівнює сумі її заробітної плати, дорівнює
сумі змінного капіталу, авансованого цілою клясою капіталістів. До цих
останніх ці гроші припливають назад тому, що вони продають свій
продукт робітничій клясі. В наслідок цього їхній змінний капітал знову
набирає грошової форми. Припустімо, що сума цього змінного капіталу,
тобто сума змінного капіталу, не просто авансованого протягом року, а
справді застосованого, дорівнює 100\pound{ ф. стерл.} $× х$; для розглядуваного тут
питання не має жодного значення, чи багато чи мало, залежно від швидкости
обороту, треба грошей для того, щоб авансувати протягом року
змінний капітал такої вартости. Цими 100\pound{ ф. стерл.} $× х$ капіталу кляса капіталістів
купує певну масу робочої сили або сплачує заробітну плату
певному числу робітників — перша оборудка. Робітники на цю саму суму
купують у капіталістів деяку кількість товарів, в наслідок цього сума
100\pound{ ф. стерл.} $× х$ зворотно припливає до капіталістів — друга оборудка.
\parbreak{}  %% абзац продовжується на наступній сторінці
