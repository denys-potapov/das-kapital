\parcont{}  %% абзац починається на попередній сторінці
\index{ii}{0283}  %% посилання на сторінку оригінального видання
закуп його робочої сили, та з другої частини, що протягом її він продукує
додаткову вартість (зиск, ренту й т. ін.). — Саме та щоденна праця,
яку витрачається на репродукцію засобів продукції, і вартість якої
розкладається на заробітну плату й додаткову вартість, — саме ця праця
реалізується в нових засобах продукції, які заміщують сталу частину
капіталу, витрачену на продукцію засобів споживання.

Головні труднощі, що з них більшу частину уже розв’язано в попередньому
викладі, постають тоді, коли досліджують не акумуляцію, а просту
репродукцію. Тим то А. Сміс (книга II), як і раніше Кене (Tableau
économique), виходять з простої репродукції, скоро мова йде про рух
річного продукту суспільства та його репродукцію, упосереднену циркуляцією.

\subsubsection{Розклад мінової вартостіі на $v \dplus{} m$ у Сміса}

Догму А. Сміса, ніби ціна або мінова вартість (exchangeable value)
кожного поодинокого товару, — отже, і всіх товарів, сукупність яких
становить річний продукт суспільства (він слушно припускає всюди капіталістичну
продукцію), — складається з трьох складових частин (component
parts) або розкладається на (resolves itself into): заробітну плату,
зиск і ренту, можна звести на те, що товарова вартість — $v \dplus{} m$, тобто
дорівнює вартості авансованого змінного капіталу плюс додаткова вартість.
Це зведення зиску й ренти до того загального й єдиного, що
ми звемо m, ми можемо зробити саме з виразного дозволу А. Сміса, як
це видно з наступних цитат, де ми спочатку не звертаємо уваги на всебічні
пункти, тобто на всі позірні або справжні відхили від догми, що за
нею товарова вартість складається виключно з елементів, які ми позначаємо
як $v \dplus{} m$.

В мануфактурі „вартість, що її робітники долучають до матеріялів,
розкладається на\dots{}\dots{} дві частини, що з них одна оплачує їхню заробітну
плату, а друга — зиск їхньому хазяїнові на ввесь капітал, авансований ним
на матеріял і на заробітну плату“. (Кн. І, розд. 6, стор. 41). — „Хоч
мануфактуристові“ (мануфактурному робітникові) „його заробітну плату
й авансує підприємець, але в дійсності вона нічого не коштує цьому
останньому, бо звичайно вартість цієї заробітної плати, разом з зиском,
повертається (restored) в збільшеній вартості предмету, що на нього застосовано
працю „мануфактуриста“. (Кн. II, розд. З, стор. 221). Частина капіталу
(Stock), витрачена на „утримання продуктивної праці\dots{} після того як вона
служила йому (підприємцеві) в функції капіталу\dots{} становить їх (робітників)
дохід“. (Кн. II, розд. З, стор. 223).

А. Сміс у щойно цитованому розділі виразно каже: „Ввесь річний
продукт землі та праці кожної країни\dots{} сам собою (naturally) розпадається
на дві частини. Одну з цих частин, і часто найбільшу, насамперед
призначається замістити капітал і відновити засоби існування, сировинні
матеріяли й готові продукти, взяті з капіталу. Другу частину призначається
утворити дохід, чи то для власника цього капіталу, як зиск на
його капітал, чи то дохід для когобудь іншого, як ренту з його
\parbreak{}  %% абзац продовжується на наступній сторінці
