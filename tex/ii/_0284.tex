\parcont{}  %% абзац починається на попередній сторінці
\index{ii}{0284}  %% посилання на сторінку оригінального видання
\so{землеволодіння} (стор. 222). Тільки одна частина капіталу, як ми
раніше чули від А. Сміса, становить разом із тим і дохід для когось
іншого, а саме — частина, витрачена на закуп продуктивної праці. Ця частина
— змінний капітал — спочатку виконує в руках підприємця й для підприємця
„функцію капіталу“, а потім „становить дохід“ самих продуктивних
робітників. Капіталіст перетворює частину своєї капітальної вартости на
робочу силу й тим самим перетворює її на змінний капітал; лише в наслідок
цього перетворення функціонує як промисловий капітал не тільки ця частина
капіталу, а й увесь його капітал. Робітник — продавець робочої сили — одержує
у формі заробітної плати вартість робочої сили. В його руках робоча
сила є лише придатний для продажу товар, що з продажу його він живе
й що, отже, становить єдине джерело його доходу. Як змінний капітал
робоча сила функціонує лише в руках її покупця, капіталіста, і саму
закупну ціну її капіталіст авансує лише позірно, бо її вартість уже раніше
дали йому робітники.

Після того як А. Сміс показав нам таким чином, що вартість продукту
в мануфактурі \deq{} $v \dplus{} m$ (де m \deq{} зискові капіталіста), він каже нам,
що в хліборобстві робітники, крім „репродукції вартости, що дорівнює
їхньому власному споживанню й капіталові, який дає їм працю“ (змінному),
„разом із зиском капіталіста“, — крім цього „понад капітал фармера
та весь його зиск, регулярно репродукують також і ренту землевласника“.
(Кн. II, розд. 5, стор. 243). Та обставина, що рента
йде до рук землевласника, зовсім не має значення для розглядуваного
нами питання. Перше ніж перейти до його рук, вона має бути в
руках фармера, тобто в руках промислового капіталіста. Перше ніж зробитись
чиїмось доходом, вона має бути складовою частиною вартости
продукту. Отже, і рента й зиск у самого А. Сміса є лише складові частини
додаткової вартости, що її постійно репродукує продуктивний робітник
разом із його власною заробітною платою, тобто разом із вартістю
змінного капіталу. Отже, рента й зиск є частина додаткової вартости
m, а тому в А. Сміса ціна всіх товарів розкладається на $v \dplus{} m$.

Догма, ніби ціна всіх товарів (отже, і ціна річного товарового продукту)
розкладається на заробітну плату плюс зиск плюс земельна рента, в
усюди помітній езотеричній частині Смісової праці набирає й такої форми,
що вартість всякого товару, отже, і вартість річного товарового продукту
суспільства, \deq{} $v \dplus{} m$, \deq{} капітальній вартості, витраченій на робочу силу
й постійно репродукованій робітниками, плюс додаткова вартість, долучена
робітниками за допомогою їхньої праці.

Цей кінцевий висновок А. Сміса разом з тим відкриває нам — див.
далі — джерело його однобічної аналізи складових частин, що на них
розкладається товарова вартість. До визначення величини кожної поодинокої
з цих складових частин і меж величини суми їхніх вартостей не має
жодного чинення та обставина, що ці складові частини разом із тим
становлять різні джерела доходу для різних кляс, діющих у продукції.
Коли А. Сміс каже: „Заробітна плата, зиск і земельна рента є три первісні
джерела і всякого доходу і всякої мінової вартости; всякий інший
\parbreak{}  %% абзац продовжується на наступній сторінці
