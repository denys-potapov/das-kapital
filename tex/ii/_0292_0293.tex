\parcont{}  %% абзац починається на попередній сторінці
\index{ii}{0292}  %% посилання на сторінку оригінального видання
виплатити, гроші відіграють ролю лише ідеальної міри вартости, і при
цьому ще зовсім не потрібно, щоб вони були в руках капіталіста); подруге,
в процесі продукції, де робоча сила функціонує в руках капіталіста
як капітал, тобто як елемент, що утворює споживну вартість і вартість.
Вона вже дала в товаровій формі той еквівалент, що його треба виплатити
робітникові, дала еквівалент цей раніше, ніж капіталіст виплатить
його в грошовій формі робітникові. Отже, робітник сам утворює виплатний
фонд, що з нього капіталіст оплачує його. Та це ще не все.

Робітник витрачає одержувані гроші на утримання своєї робочої
сили, отже, — коли розглядати клясу капіталістів і клясу робітників у
їхній сукупності, — робітник витрачає ці гроші, щоб зберегти капіталістові
те знаряддя, що за допомогою його лише й може він лишатись
капіталістом.

Отже, постійна купівля й продаж робочої сили увічнює, з одного
боку, робочу силу як елемент капіталу; в наслідок цього капітал з’являється
як творець товарів, предметів споживання, що мають вартість;
далі, в наслідок цього ж ту частину капіталу, яка купує робочу силу, постійно
відновлюється продуктом цієї робочої сили, і значить, сам робітник постійно
утворює той фонд капіталу, що з нього йому платять. З другого
боку, постійний продаж робочої сили стає повсякчас поновлюваним джерелом
засобів існування робітника й таким чином його робоча сила
з’являється як здатність, що через неї він одержує дохід, з якого він
живе. Дохід тут значить не що інше, як зумовлюване постійно повторюваним
продажем товару (робочої сили) привласнення вартостей, при
чому самі ці вартості служать лише для постійної репродукції продаваного
товару. І остільки має А.~Сміс рацію казати, що джерелом доходу
робітника стає та частина вартости утвореного самим робітником продукту, за
яку капіталіст дає йому еквівалент у формі заробітної плати. Але це так
само нічого не змінює в природі або величині цієї частини вартости товару,
як нічого не змінює у вартості засобів продукції та обставина, що
вони функціонують як капітальні вартості, або так само, як нічого не
змінюється в природі й величині прямої лінії від того, чи буде вона правити
за основу трикутника чи за діяметр еліпси. Вартість робочої сили,
як і перше, визначається також незалежно від цієї обставини, як і вартість
засобів продукції. Ця частина вартости товару ані складається з доходу,
як одного з її складових самостійних чинників, ані розкладається на
дохід. Хоч ця нова вартість, постійно репродуковувана робітником, і становить
для нього джерело доходу, однак, навпаки, його дохід не становить
складової частини продукованої ним нової вартости. Величина виплачуваної
йому частини, утворюваної ним нової вартости визначає розмір
вартости його доходу, а не навпаки. Та обставина, що ця частина нової
вартости становить для нього дохід, свідчить лише про те, що з нею робиться,
про спосіб вжитку її, але це так само не має жодного чинення до створення
її, як і до створення всякої іншої вартости. Коли я щотижня
одержую десять талерів, то самий факт цього щотижневого одержання
нічого не змінює ні в природі вартости десятьох талерів, ні в величині
\index{ii}{0293}  %% посилання на сторінку оригінального видання
їхньої вартости. Як вартість кожного іншого, товару, вартість робочої
сили визначається кількістю праці, доконечної для її репродукції; те, що
ця кількість праці визначається вартістю доконечних засобів існування
робітника, отже, дорівнює праці, доконечній для репродукції засобів його
власного існування, є характеристичне для цього товару (робочої сили);
але не характеристичніше за те, що вартість в’ючної худоби визначається
вартістю засобів існування, доконечних для її утримання, отже,
масою людської праці, потрібної для того, щоб випродукувати ці засоби
існування.

Саме ця категорія „доходу“ і спричиняє тут усе лихо в А.~Сміса.
Різні відміни доходів становлять у нього „component parts“, складові
частини щорічно продукованої, новоутворюваної товарової вартости, тимчасом
як, навпаки, ті дві частини, що на них розкладається ця товарова
вартість для капіталіста — еквівалент його змінного капіталу, авансованого
в грошовій формі підчас закупу праці, і друга частина вартости, яка також
належить йому, хоч нічого йому й не коштувала, додаткова вартість —
становлять джерела доходів. Еквівалент змінного капіталу знов авансується
на робочу силу і остільки становить дохід для робітника у формі його
заробітної плати. Друга частина — додаткова вартість — не має заміщувати
капіталістові жодного авансування капіталу, і тому він може витратити її
на засоби споживання (доконечні, а також речі розкошів), може спожити її
як дохід, замість перетворювати її на капітальну вартість будь-якого роду.
Передумова цього доходу є сама товарова вартість, і її складові частини
лише остільки відрізняються для капіталіста, оскільки вони являють або
еквівалент за авансовану ним змінну капітальну вартість, або надлишок
над авансованою ним змінною капітальною вартістю. Обидві частини складаються
не з чого іншого, як з робочої сили, витраченої підчас продукції
товару, пущеної в рух у процесі праці. Вони складаються з витрати,
не з надходження або доходу, а з витрати праці.

Після цього qui pro quo, де дохід стає джерелом товарової вартости
замість товаровій вартості бути джерелом доходу, товарова
вартість виступає тепер „складеною“ з різних відмін доходів. їх
визначається незалежно одну від однієї, і всю вартість товару визначається
доданням величин вартости цих доходів. Але запитаймо тепер,
як визначається вартість кожного з цих доходів, що з них має постати
товарова вартість? Щодо заробітної плати, то її можна визначити, бо вона
є вартість відповідного товару, робочої сили, а цю останню визначається
(як і вартість всякого іншого товару) працею, потрібного на репродукцію
цього товару. Але як визначається додаткову вартість або радше, за А.~Смісом, дві її форми — зиск і земельну ренту? Тут усе сходить на порожню
балаканину. А.~Сміс то подає заробітну плату й додаткову вартість
(зглядно — заробітну плату й зиск) як складові частини, що з них складається
товарова вартість, зглядно ціна, то — і часто майже безпосередньо
по цьому — як частини, що на них „розкладається“ (resolves itself) товарова
ціна; а це значить, навпаки, що товарова вартість є наперед дана,
і що різні частини цієї даної вартости в формі різних доходів дістаються
\parbreak{}  %% абзац продовжується на наступній сторінці
