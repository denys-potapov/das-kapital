\parcont{}  %% абзац починається на попередній сторінці
\index{ii}{0298}  %% посилання на сторінку оригінального видання
мотивом, що спонукав взагалі нашого капіталіста займатись продукцією
товару. Але ні його спочатку добромисний намір здобути додаткову
вартість, ні дальше її витрачання як доходу ним та іншими не справляє
жодного впливу на додаткову вартість як таку. І те і те нічого не змінює
в тому, що вона є застигла неоплачена праця, і так само це ані
трохи не змінює й величини її, визначуваної цілком іншими умовами.

Але коли А.~Сміс хотів уже при розгляді товарової вартости вивчити,
як він це і робить, ролю, яка припадає її різним частинам у сукупному
процесі репродукції, то було очевидно, що коли окремі її частини
функціонують як дохід, то інші так само постійно функціонують як капітал,
а тому, згідно з його логікою, їх також треба було б визнати за
складові частини товарової вартости або за частини, що на них вона
розкладається.

А.~Сміс ототожнює товарову продукцію взагалі з капіталістичною
товаровою продукцією; засоби продукції з самого початку є „капітал“,
праця — з самого початку наймана праця, і тому „число корисних і продуктивних
робітників усюди\dots{} пропорційне величині капіталу, застосованого
на те, щоб дати їм працю“ („to the quantity of capital stock which
is employed in setting them to work“. — „Introduction“, стор. 12). Коротко
кажучи, різні фактори процесу праці — речові і особові — з самого початку
з’являються в характеристичних масках капіталістичного періоду
продукції. Тому аналіза товарової вартости також безпосередньо збігається
з з’ясовуванням того, оскільки ця вартість є, з одного боку, простий
еквівалент витраченого капіталу й оскільки, з другого боку, вона є
„вільна“ вартість, яка не є еквівалент будь-якої авансованої капітальної
вартости, тобто є додаткова вартість. Таким чином, частини товарової
вартости, з цього погляду зіставлювані одна з однією, потай перетворюються
на її самостійні „складові частини“ і, нарешті, на „джерела вартости“.
Дальший висновок той, що товарова вартість складається з доходів
різного роду або — навпереміну — розкладається на доходи різного
роду, так що не доходи складаються з товарової вартости, а товарова
вартість з „доходів“. Але так само, як в природі товарової вартости як
товарової вартости, або грошей як грошей, нічого не змінює та обставина,
що вони функціонують як капітальна вартість, так само нічого не
змінює в товаровій вартості та обставина, що вона пізніше функціонує
як чийсь дохід. Товар, що з ним має справу А.~Сміс, є з самого початку
товаровий капітал (що має в собі, крім зужиткованої в процесі продукції
товару капітальної вартости, ще й додаткову вартість), отже, капіталістично
спродукований товар, результат капіталістичного процесу
продукції. Тому треба було б спочатку аналізувати цей останній, отже,
і включений у ньому процес зростання вартости й утворення вартости.
А що його передумову знову таки являє товарова циркуляція, то для
зображення його треба зробити незалежну від неї попередню аналізу
товару. Навіть коли А.~Сміс „езотерично“ мимохіть натрапляє на правильний
шлях, він розглядає продукцію вартости завжди лише принагідне
при аналізі товару, тобто при аналізі товарового капіталу.
