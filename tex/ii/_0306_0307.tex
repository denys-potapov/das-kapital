\parcont{}  %% абзац починається на попередній сторінці
\index{ii}{0306}  %% посилання на сторінку оригінального видання
кляса капіталістів II знову перетворила свій сталий капітал, рівний 2000,
з форми засобів споживання на форму засобів продукції засобів споживання,
на форму, що в ній він може знову функціонувати як чинник
процесу праці і як стала капітальна вартість для зростання вартости. З
другого боку, в наслідок цього еквівалент робочої сили в І ($1000 Iv$)
і додаткова вартість капіталістів I ($1000 Іm$) реалізувались у засобах
споживання; і те й друге з своєї натуральної форми засобів продукції перетворилися
на таку натуральну форму, що в ній їх можна спожити як дохід.

Але таке взаємне перетворення здійснюється за допомогою грошової
циркуляції, що так само упосереднює його, як і утруднює його розуміння;
однак, вона відіграє вирішально важливу ролю, бо змінна частина капіталу
знову й знов мусить виступати в грошовій формі, як грошовий капітал,
що з грошової форми перетворюється на робочу силу. В усіх галузях
підприємств, що одночасно працюють одне біля одного на периферії
суспільства, все одно, чи належать вони до категорії I чи до II, змінний
капітал доводиться авансувати в грошовій формі. Капіталіст купує
робочу силу раніш ніж вона входить у процес продукції, але оплачує її
лише в строки, визначені умовою, після того, як її вже витрачено на
продукцію споживної вартости. Так само, як і інші частини вартости
продукту, йому належить і та частина її, що є лише еквівалент грошей
витрачених на оплату робочої сили, тобто та частина вартости продукту,
що репрезентує змінну капітальну вартість. Цією частиною вартости продукту
робітник уже дав капіталістові еквівалент своєї заробітної плати. Але
лише зворотне перетворення товару на гроші, продаж товару, відновлює
капіталістові його змінний капітал як грошовий капітал, що його він
знову може авансувати на закуп робочої сили.

Отже, в підрозділі I капіталіст, розглядуваний як збірний капіталіст,
виплатив робітникам 1000\pound{ ф. стерл.} (я кажу ф. стерл. лише для того,
щоб зазначити, що це — вартість у грошовій формі) \deq{} $1000v$ за
ту вартість, яка вже існує як частина v вартости продукту I, тобто спродукованих
робітниками засобів продукції. На ці 1000\pound{ ф. стерл.} робітники
купують у капіталістів II засобів споживання на таку саму вартість і таким
чином перетворюють половину сталого капіталу II на гроші; капіталісти II
із свого боку купують на ці 1000\pound{ ф. стерл.} засоби продукції вартістю
на 1000 у капіталістів I; тим самим змінну капітальну вартість останніх
\deq{} $1000 v$, що існує як частина їхнього продукту в натуральній формі
засобів продукції, знову перетворено на гроші, і тепер в руках капіталістів
І знову може вона функціонувати як грошовий капітал, що перетворюється
на робочу силу, отже, на найпосутніший елемент продуктивного
капіталу. Таким чином, в наслідок реалізації частини їхнього товарового
капіталу, до них зворотно припливає їхній змінний капітал у
грошовій формі.

Щодо грошей, потрібних для обміну частини m товарового капіталу
І на другу половину сталої частини капіталу II, то їх можна авансувати
різними способами. На ділі ця циркуляція охоплює незчисленну силу поодиноких
актів купівлі й продажу, що їх переводять індивідуальні капіталісти
\index{ii}{0307}  %% посилання на сторінку оригінального видання
обох категорій, але при цьому гроші в усякому разі мають походити
від цих капіталістів, бо ми вже закінчили обчислення з тими
грішми, що їх подали в циркуляцію робітники. Тут буває або так, що
капіталіст категорії II із свого грошового капіталу, який існує поряд капіталу
продуктивного, купує засоби продукції в капіталістів категорії І;
або, навпаки, капіталіст категорії І із свого грошового фонду, призначеного
на особисті витрати, не на витрачання як капітал, купує засоби
споживання в капіталістів категорії II.~Як ми вже показали в відділі 1 і
II, ми мусимо в усякому разі припустити, що в руках капіталістів поряд
продуктивного капіталу є певні грошові запаси, — чи то для авансування
капіталу, чи то для витрачання доходу. Припустімо — пропорція не має
жодного значення для нашої цілі, — що капіталісти II авансують половину
грошей на закуп засобів продукції, щоб замістити свій сталий капітал,
а другу половину капіталісти І витрачають на споживання, а саме: підрозділ
II авансує 500\pound{ ф. стерл.} і купує на них у І засоби продукції,
тим самим він заміщує in natura (разом з вищезгаданими 1000\pound{ ф. стерл.},
то походять від робітників) \sfrac{3}{4} свого сталого капіталу; підрозділ І на
одержані таким чином 500\pound{ ф. стерл.} купує у II засоби споживання
і закінчує разом з тим циркуляцію $т — г — т$ для половини тієї частини
свого товарового капіталу, яка складається з $т$, реалізує цей продукт
свій в фонді споживання. В наслідок цього другого процесу 500\pound{ ф. стерл.}
повертаються назад до рук II як грошовий капітал, що його капіталісти
II мають поряд свого продуктивного капіталу. З другого боку, І для
половини тієї частини $т$ свого товарового капіталу, що лежить ще у
нього як продукт, антиципує — раніш, ніж цю частину продано — витрачання
грошей в розмірі 500\pound{ ф. стерл.} на закуп засобів споживання у II.~На ці самі 500\pound{ ф. стерл}. II купує засоби продукції в І і таким чином
заміщує in natura ввесь свій сталий капітал ($1000 \dplus{} 500 \dplus{} 500$ \deq{} 2000),
тимчасом як І реалізував у засобах споживання всю свою додаткову
вартість. В загальному підсумку обмін товарів на суму в 4000\pound{ ф. стерл.}
відбувся б за грошової циркуляції в 2000\pound{ ф. стерл.} і при цьому вона
досягає такої величини лише тому, що, як подано у нас, увесь річний
продукт обмінюється разом, небагатьма великими порціями. Важлива
при цьому лише та обставина, що II підрозділ не лише знову перетворив
на форму засобів продукції свій сталий капітал, репродукований
у формі засобів споживання, але що до нього, крім того, повертаються
ті 500\pound{ ф. стерл.}, що їх він авансував для циркуляції на закуп засобів
продукції; і що І підрозділ так само не лише знову одержав у грошовій
формі, як грошовий капітал, що його можна безпосередньо перетворити
на робочу силу, свій змінний капітал, репродукований ним у формі засобів
продукції, але що до нього, крім того, повертаються знову ті
500\pound{ ф. стерл.}, що їх він витратив на закуп засобів споживання, перед продажем
частини додаткової вартости від свого капіталу, антиципуючи цей
продаж. Але вони повертаються до нього назад не в наслідок витрати,
що вже відбулася, а в наслідок дальшого продажу частини його товарового
продукту, що є носій половини його додаткової вартости.
