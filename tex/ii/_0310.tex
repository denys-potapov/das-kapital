\parcont{}  %% абзац починається на попередній сторінці
\index{ii}{0310}  %% посилання на сторінку оригінального видання
споживання кляси капіталістів, хоч у цьому разі вони часто відрізняються
якістю й вартістю від засобів споживання робітників. Для нашої мети
ми можемо охопити весь цей підвідділ однією рубрикою: \so{доконечні}
засоби споживання, і при цьому тут цілком байдуже, чи є такий продукт,
як наприклад, тютюн, доконечний засіб споживання з фізіологічного
погляду чи ні; досить того, що він є звично доконечний засіб споживання.

b) Засоби споживання \deq{} \so{речі розкошів}, що входять лише в
споживання кляси капіталістів, а значить, їх можна обміняти лише на
витрачувану додаткову вартість, що ніколи не дістається робітникам.
Щодо першої рубрики, то очевидно, що змінний капітал, авансований у
грошовій формі на продукцію належних сюди ґатунків товару, мусить
безпосередньо повернутись до тієї частини кляси капіталістів II (тобто
до капіталістів II~\emph{а}), яка продукує ці доконечні засоби існування.
Капіталісти продають їх своїм власним робітникам на суму змінного капіталу,
сплаченого їм як заробітна плата. Цей зворотний приплив щодо цілого
цього підвідділу \emph{а} кляси капіталістів II відбувається \so{безпосередньо},
хоч які б численні були ті оборудки між капіталістами різних належних
сюди галузей промисловости, через що розподіляється pro rata
цей змінний капітал, який зворотно припливає. Це процеси циркуляції,
що для них засоби циркуляції є безпосередньо ті гроші, що їх витрачають
робітники. Інакше стоїть справа з підвідділом II~\emph{b}. Вся та частина
новоспродукованої вартости, що з нею ми маємо тут справу, II~\emph{b} ($v \dplus{} m$)
існує в натуральній формі речей розкошів, тобто речей, що їх робітнича
кляса так само не може купити, як і товарової вартости I~$v$, яка існує
в формі засобів продукції; хоч і ці речі розкошів і ті засоби продукції
є продукти цих робітників. Отже, зворотний приплив, за допомогою
якого змінний капітал, авансований у цьому підвідділі, повертається до
капіталістичного продуцента в своїй грошовій формі, не може бути
прямим, а тільки посереднім, як у I~$v$.

\label{original-310}
Припустімо, напр., як раніше, для всієї кляси II: $v \deq{} 500$; $m \deq{} 500$;
але змінний капітал і відповідна йому додаткова вартість хай розподіляються
так:

Підвідділ \emph{а}, доконечні засоби існування: $v \deq{} 400$, $m \deq{} 400$; отже,
маса товарів в доконечних засобах споживання вартістю в $400 v \dplus{} 400 \deq{} 800$,
або II~\emph{а} ($400 v \dplus{} 400 m$).

Підвідділ \emph{b}: речі розкошів вартістю в $100 v \dplus{} 100 m \deq{} 200$, або
II~\emph{b} ($100 v \dplus{} 100 m$).

Робітники підвідділу II~\emph{b} в оплату за свою робочу силу одержали
100 грішми, напр., 100\pound{ ф. стерл.}; на них робітники купують у капіталістів
II~\emph{а} засоби споживання на суму 100. Ця кляса капіталістів купує
тоді на 100 товару II~\emph{b}, і в наслідок цього до капіталістів II~\emph{b}
зворотно припливає в грошовій формі їхній змінний капітал.

В руках капіталістів II~\emph{а} в наслідок обміну з їхніми власними робітниками
вже є 400 в грошовій формі; крім того, четверта частина їхнього
продукту, яка репрезентує додаткову вартість, відійшла до робітників
II~\emph{b}, і за неї одержано в товарах розкошів II~\emph{b} ($100 v$).
