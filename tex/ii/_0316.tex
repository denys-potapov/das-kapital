\parcont{}  %% абзац починається на попередній сторінці
\index{ii}{0316}  %% посилання на сторінку оригінального видання
підвищується не тільки споживання доконечних засобів існування; робітнича
кляса (куди тепер увіходить, як її активна частина, вся резервна
армія) на малу часину бере участь у споживанні речей розкошів, що іншим
часом для неї неприступні, і, крім того, вона бере участь у споживанні
тієї категорії доконечних засобів споживання, яка іншим часом, здебільшого,
становить „доконечні“ засоби споживання лише для кляси капіталістів;
це також із свого боку зумовлює підвищення цін.

Було б простою тавтологією сказати, що кризи випливають з недостачі
платоспроможного споживання або платоспроможних споживачів.
Капіталістична система не знає інших видів споживання, крім оплачуваного,
за винятком видів sub forma pauperis\footnote*{
У формі одержання милостині. \emph{Ред.}
} або „шахраїв“. Що
товари несила продати, не значить нічого іншого, як те, що на них не
знаходиться платоспроможних покупців, отже, споживачів (припускаючи,
що товари, кінець-кінцем, купується для продуктивного або особистого
споживання). А коли цій тавтології намагаються надати вигляд
глибшого обґрунтовання, кажучи, що робітнича кляса одержує дуже малу
частину свого власного продукту і що цьому лихові можна запобігти,
коли вона одержуватиме більшу частину свого продукту, тобто, коли її
заробітна плата збільшиться, то треба тільки зауважити, що кожна криза
підготовляється саме таким періодом, коли повсюди підвищується заробітна
плата, і робітнича кляса справді одержує більшу пайку тієї частини
річного продукту, що призначена для споживання. Такий період — з погляду
цих лицарів здорового й „простого“ (!) розуму — мусив би, навпаки,
віддалити кризу. Отже, бачимо, що капіталістична продукція включає
незалежні від доброї або злої волі умови, які допускають відносний
добробут робітничої кляси тільки на малу часину, та й це лише завжди
як буровісник кризи\footnote{
Ad notam (до відома) деяких прихильників Ротбертусової теорії криз. \emph{Ф.~Е.}
}.

Ми бачили раніше, як пропорційне відношення між продукцією
доконечних засобів споживання і продукцією засобів розкошів зумовлює
поділ II ($v \dplus{} m$) між II~\emph{а} і II~\emph{b}, a значить, і поділ II $с$ між (II~\emph{а}) $с$ і (II~\emph{b}) $с$.
Отже, цей поділ стосується до самого кореня характеру й кількісних відношень
продукції і є момент, посутньо визначальний для цілого її ладу.

Проста репродукція по суті має на меті споживання, хоч здобування
додаткової вартости й тут являє рушійний чинник для індивідуальних
капіталістів; але додаткова вартість, — хоч яка буде її відносна величина —
кінець-кінцем, повинна тут служити лише для особистого споживання
капіталіста.

Оскільки проста репродукція є також частина, і до того найзначніша
частина, кожної річної репродукції в поширеному маштабі, цей мотив —
особисте споживання — лишається поряд з мотивом збагачення і протилежно
до нього як такого. В дійсності справа видається заплутанішою,
бо спільники (partners) здобичі — додаткової вартости капіталіста — виступають
як незалежні від нього споживачі.
