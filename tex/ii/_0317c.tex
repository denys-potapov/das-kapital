\index{ii}{0317}  %% посилання на сторінку оригінального видання
\subsection{Упосереднення обмінів грошовою циркуляцією}

Оскільки з’ясовано до цього часу, циркуляція між різними клясами
продуцентів перебігала за такою схемою:

1) Між клясою І і клясою II:
\[
І. 4000с \dplus{} 1000 v \dplus{} 1000 m.
II\dots{} 2000 с \dots{} \dplus{} 500 v \dplus{} 500 m
\]
Отже, закінчилась циркуляція II$с$ \deq{} 2000, і обмінено його на
І ($1000 v \dplus{} 1000 m$).

А що ми лишаємо на деякий час осторонь 4000 І$с$, то лишається ще
циркуляція $v \dplus{} m$ в межах кляси II. Ці II ($v \dplus{} m$) розподіляються між
підклясами II$а$ і II$b$ таким чином:

\[
\text{2) II. }500 v \dplus{} 500 m \deq{} а (400 v \dplus{} 400 m) \dplus{} b (100 v \dplus{} 100 m).
\]

$400 v$ ($а$) циркулюють в межах своєї власної підкляси; робітники,
оплачені цими $400 v$ ($а$), купують на них спродуковані ними самими доконечні
засоби існування в своїх наймачів, у капіталістів II$а$.

А що капіталісти обох підкляс витрачають свою додаткову вартість
по \sfrac{3}{5} на продукти II$a$ (доконечні засоби існування) і по \sfrac{2}{5} на продукти
II$b$ (речі розкошів), то \sfrac{3}{5} додаткової вартости $а$, тобто 240, споживається
в межах самої підкляси II$а$; так само \sfrac{2}{5} додаткової вартости $b$ (що спродукована)
й існує в засобах розкошів — в межах підкляси II$b$.

Лишається, отже, для обміну між II$a$ і II$b$:

на боці II$а$: $160 m$

на боці II$b$: $100 v \dplus{} 60 m$. Ці продукти навзаєм покриваються. Робітники
II$b$ на свої 100, одержані як заробітна плата в грошовій формі,
купують в II$а$ доконечні засоби існування на суму в 100. Капіталісти
II$b$ на суму в \sfrac{2}{5} своєї додаткової вартости \deq{} 60 так само купують
доконечні засоби існування в II$а$. В наслідок цього капіталісти II$а$ одержують
гроші, потрібні для того, щоб, як припущено вище, \sfrac{2}{5} своєї
додаткової вартости \deq{} $160 m$ витратити на речі розкошів, спродуковані
в ІІ$b$ ($100 v$, що є в руках капіталістів II$b$ як продукт, який заміщує
видану заробітну плату, і $60 m$). Отже, маємо для цього таку схему:
\[
3) II а (400 v) \dplus{} (240 m) \dplus{} 160 m
b \dotfill 100 v \dplus{} 60 m \dplus{} (40 m),
\]
де в дужки взято ті величини, які циркулюють і споживаються лише в
межах своєї власної підкляси.

Безпосередній зворотний приплив грошового капіталу, авансованого у
формі змінного капіталу, який відбудеться лише для підрозділу капіталістів
II$а$, де продукується доконечні засоби існування, є лише модифіковане
особливими умовами виявлення того згаданого вище
загального закону, що гроші при нормальному перебігу товарової
циркуляції повертаються назад до тих товаропродуцентів, які авансують
їх на циркуляцію. З цього між іншим випливає, що коли за
товаропродуцентом взагалі стоїть грошовий капіталіст, який знову
\parbreak{}  %% абзац продовжується на наступній сторінці
