
\index{ii}{0326}  %% посилання на сторінку оригінального видання

Труднощі розв’язується досить просто, коли ми візьмемо на увагу,
що ввесь товаровий продукт І своєю натуральною формою складається
з засобів продукції, тобто з речових елементів самого сталого капіталу.
Тут маємо те саме явище, як і в II, тільки в іншому аспекті. В II ввесь товаровий
продукт складався з засобів споживання; тому частину його, вимірювану
заробітною платою плюс додаткова вартість, що містяться в цьому товаровому
продукті, могли спожити сами продуценти цієї частини. Тут, в
І, ввесь товаровий продукт складається з засобів продукції, будівель, машин,
вмістищ, сировинних і допоміжних матеріялів тощо. Тому частина їх
та, що заміщує застосований в даній сфері сталий капітал, може всвоїй натуральній
формі одразу почати функціонувати як складова частина сталого
капіталу. Оскільки вона входить у циркуляцію, вона циркулює в межах
кляси І.~В II підрозділі частину товарового продукту in natura споживають
його власні продуценти особисто, в І, навпаки, частину продукту
in natura споживають капіталістичні продуценти продуктивно.

В частині товарового продукту І \deq{} 4000 с, стала капітальна вартість,
спожита в цьому підрозділі, з’являється знову й саме в такій натуральній
формі, що в ній вона одразу може знову функціонувати як продуктивний
сталий капітал. В II частина з товарового продукту в 3000, що її вартість
дорівнює вартості заробітної плати плюс додаткова вартість (\deq{} 1000),
безпосередньо входить в особисте споживання капіталістів і робітників II;
навпаки, стала капітальна вартість цього товарового продукту (\deq{} 2000)
не може знову ввійти в продуктивне споживання капіталістів II, її треба
замістити через обмін з І.

Протилежно до цього в І підрозділі частина з товарового продукту
в 6000, що її вартість дорівнює заробітній платі плюс додаткова вартість
(\deq{} 2000), не входить і не може в наслідок своєї натуральної форми
ввійти в особисте споживання продуцентів цього продукту. Спочатку
мусить відбутись обмін цієї частини з II.~Навпаки, стала частина вартости
цього продукту \deq{} 4000 перебуває в такій натуральній формі, що в ній
вона — коли розглядати клясу капіталістів І як ціле — безпосередньо може
знову функціонувати як сталий капітал І.~Інакше кажучи: ввесь продукт
підрозділу І складається з споживних вартостей, що в наслідок своєї натуральної
форми — при капіталістичному способі продукції — можуть служити
лише як елементи продуктивного капіталу. Отже, з цього продукту
вартістю в 6000 одна третина (2000) заміщує сталий капітал підрозділу II,
а решта \sfrac{2}{3} — сталий капітал підрозділу І.

Сталий капітал І складається з маси різних груп капіталу, вкладених
у різні галузі продукції засобів продукції: стільки на домни, стільки на
кам’яновугільні шахти й~\abbr{т. ін.} Кожна з цих груп капіталу, або кожен
з цих суспільних групових капіталів і собі складається з більшого або
меншого числа самостійно діющих поодиноких капіталів. Поперше, капітал
суспільства, напр., 7500 (що може значити мільйони й~\abbr{т. ін.}) розпадається
на різні групи капіталу; суспільний капітал в 7500 розпадається
на особливі частини, що з них кожну вкладено в особливу галузь продукції;
вкладена в кожну особливу галузь продукції частина суспільної
\parbreak{}  %% абзац продовжується на наступній сторінці
