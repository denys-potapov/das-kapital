
\index{ii}{0332}  %% посилання на сторінку оригінального видання
Стала частина вартости, що лише знову з’являється, дорівнює вартості
тієї частини цього продукту, яка складається з засобів продукції;
вона втілюється в цій частині продукту.

Новоспродукована протягом року вартість \deq{} $v \dplus{} m$ дорівнює вартості
тієї частини цього продукту, яка складається з засобів споживання;
вона втілюється в цій частині продукту.

Але, лишаючи осторонь неважливі тут винятки, засоби продукції та
засоби споживання є цілком різні ґатунки товарів, продукти цілком різної
натуральної або споживної форми, отже, продукти цілком різних конкретних
відмін праці. Праця, яка застосовує машини, щоб продукувати
засоби існування, тут цілком відрізняється від праці, яка робить машини.
Здається, ніби ввесь річний робочий день, що його вираз вартости \deq{} 3000,
витрачено на продукцію засобів споживання \deq{} 3000, де не з’являється
знову жодної сталої частини вартости, бо ці $3000 \deq{} 1500 v \dplus{} 1500 m$
розкладаються лише на змінну капітальну вартість \dplus{} додаткова вартість. З
другого боку, стала капітальна вартість \deq{} 6000 знову з’являється в такому
ґатунку продуктів, що цілком відрізняється від засобів споживання,
в засобах продукції, і здається, ніби на продукцію цих нових продуктів
не витрачено жодної частини суспільного робочого дня; навпаки, здається,
ніби ввесь робочий день складається лише з таких ґатунків праці, що
результат їхній не засоби продукції, а тільки засоби споживання. Таємницю
вже розв’язано. Новоспродукована річною працею вартість дорівнює
вартості продукту підрозділу II, цілій вартості новоспродукованих засобів
споживання. Але ця вартість продукту на \sfrac{2}{3} більша, ніж частина річної
праці, витрачена в продукції засобів споживання (підрозділу II). На продукцію
їх витрачено лише \sfrac{1}{3} річної праці. \sfrac{2}{3} цієї річної праці витрачено
на продукцію засобів продукції, отже, в підрозділі І. Утворена протягом
цього часу в І нова вартість, рівна спродукованій в І змінній капітальній
вартості плюс додаткова вартість дорівнює сталій капітальній вартості II,
що знову з’являється в підрозділі II в засобах споживання. Отже, ці величини
можуть обмінятись одна на одну, можуть in natura замістити одна одну.
Тому вся вартість засобів споживання II дорівнює сумі нової новоспродукованої
вартости І \dplus{} II, або II (с \dplus{} $v \dplus{} m$) \deq{} І ($v \dplus{} m$) \dplus{} II ($v \dplus{} m$), отже, дорівнює
сумі нової вартости, спродукованої річною працею у формі $v \dplus{} m$.

З другого боку, вся вартість засобів продукції (І) дорівнює сумі сталої
капітальної вартости, яка знову з’являється в формі засобів продукції
(І) і в формі засобів споживання (II), отже, дорівнює сумі сталої
капітальної вартости, яка знову з’являється в цілому продукті суспільства.
Вся ця вартість дорівнює виразові вартости \sfrac{4}{3} минулого робочого дня,
витрачених у І підрозділі до початку продукційного процесу, і \sfrac{2}{3}
минулого робочого дня, витрачених в II до початку продукційного процесу,
отже, разом дорівнює виразові вартости двох цілих робочих днів.

Отже, в аналізі суспільного річного продукту трудноті постають тому,
що стала частина вартости виражається в продуктах цілком іншого ґатунку
— в засобах продукції — ніж долучена до цієї сталої частини
вартости нова вартість v \dplus{} т, яка виражається в формі засобів споживання.
\parbreak{}  %% абзац продовжується на наступній сторінці
