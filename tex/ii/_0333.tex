\parcont{}  %% абзац починається на попередній сторінці
\index{ii}{0333}  %% посилання на сторінку оригінального видання
Таким чином, здається, ніби \sfrac{2}{3} зужиткованої маси продуктів — коли
розглядати з погляду вартости — знову з’являються в новій формі, як
новий продукт, що на його продукцію суспільство не витратило
жодної праці. Щодо поодинокого капіталу, то цього не буває. Кожен індивідуальний
капіталіст застосовує певний конкретний ґатунок праці, що перетворює
на продукт відповідні йому засоби продукції. Хай, напр., капіталіст
буде машинобудівник, витрачений протягом року сталий капітал $\deq{} 6000 с$,
змінний $\deq{} 1500 v$, додаткова вартість \deq{} $1500 m$; продукт \deq{} 9000; скажімо,
напр., що цей продукт — 18 машин, що з них кожна \deq{} 500. Ввесь продукт
існує тут в тій самій формі, у формі машин. (Коли машінобудівник продукує
кілька ґатунків машин, то для кожного складається окремий рахунок).
Ввесь товаровий продукт є продукт праці, витраченої протягом
року в машинобудівництві, комбінація того самого конкретного ґатунку
праці з тими самими засобами продукції. Тому різні частини вартости
продукту можна виразити в тій самій натуральній формі: в 12 машинах
міститься $6000 с$, в 3 машинах — $1500 v$, в 3 машинах — $1500 m$. Тут
очевидно, що вартість 12 машин дорівнює $6000 с$ не тому, що в цих
12 машинах втілено тільки працю, яка минула до машинобудівництва й
витрачена не на машинобудізництво. Не сама собою вартість засобів
продукції, потрібних на ці 18 машин, перетворилась на 12 машин, а вартість
цих 12 машин (яка сама складається з $4000 с \dplus{} 1000 v \dplus{} 1000 m$)
просто дорівнює цілій сталій капітальній вартості, що міститься в 18 машинах.
Тому машинобудівник мусить продати 12 з 18 машин, щоб замістити
витрачений ним сталий капітал, потрібний йому для репродукції 18
нових машин. Навпаки, справу не можна було б з’ясувати, коли б, хоч
застосовувана праця складається тільки з машин, результат її був: з
одного боку, 6 машин \deq{} $1500 v \dplus{} 1500 m$, з другого боку, залізо,
мідь, гвинти, паси й~\abbr{т. ін.}, на суму вартости в 6000 с, тобто засоби продукції
машин в їхній натуральній формі, засоби продукції, що їх, як відомо,
поодинокий капіталіст, машинобудівник, сам не продукує, а мусить
заміщувати їх за допомогою процесу циркуляції. І однак на перший погляд
здається, ніби репродукція суспільного річного продукту відбувається
таким безглуздим способом.

Продукт індивідуального капіталу, тобто кожної самостійно діющої,
обдарованої власним життям частини суспільного капіталу, має якусь певну
натуральну форму. Єдина умова в тому, щоб він справді мав споживну
форму, споживну вартість, яка накладає на нього печать здатного до
циркуляції члена товарового світу. При цьому цілком байдужа й випадкова
та обставина, чи може він як засіб продукції знову ввійти в той
самий продукційний процес, що з нього він вийшов як продукт, отже, чи
має та частина вартости його продукту, що в ній виражається стала частина
капіталу, таку натуральну форму, в якій вона в дійсності може знову
функціонувати як сталий капітал. Коли цього немає, то ця частина вартости
продукту через продаж і купівлю знову перетворюєтьса на форму речових
елементів продукції цього продукту, і таким чином репродукується сталий
капітал у тій його натуральній формі, що в ній він може функціонувати.
