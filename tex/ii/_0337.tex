\index{ii}{0337}  %% посилання на сторінку оригінального видання
продуктивних споживачів. Але та обставина, що частина продукту мусить
бути спожита продуктивно, значить лише те, що вона мусить
функціонувати як капітал і не може бути спожита як дохід.

Навпаки, коли вартість сукупного продукту \deq{} 9000 ми розподілимо
на $6000 с \dplus{} 1500 v \dplus{} 1500 m$ і розглядатимемо 3000 ($v \dplus{} m$) лише
в їхній властивості бути доходом, то здається, ніби змінний капітал зник,
і що капітал, розглядуваний з суспільного погляду, складається виключно
з сталого капіталу. Бо те, що спочатку виступало як $1500 v$, розклалось
на частину суспільного доходу, на заробітну плату, дохід робітничої
кляси, — і разом з тим зник характер капіталу цієї частини. Рамсай
і справді зробив такий висновок. На його думку, капітал, розглядуваний
з суспільного погляду, складається лише з основного капіталу, але під
основним капіталом він розуміє сталий капітал, масу вартости, що є в
засобах продукції, хоч будуть ці засоби продукції лише засобами праці
або матеріялами праці, як от сировинний матеріял, напівфабрикат, допоміжний
матеріял тощо. Змінний капітал він зве обіговим капіталом. „Обіговий
капітал складається виключно з засобів існування та інших доконечних
речей, авансовуваних робітникам, поки вивершиться продукт
їхньої праці\dots{} Тільки основний капітал, а не обіговий є, власне кажучи,
джерело національного багатства\dots{} Обіговий капітал не є безпосередній
чинник продукції, і взагалі він не має для неї посутнього значення; це —
лише умова, що стала доконечною в наслідок гірких злиднів маси народу\dots{}
З національного погляду лише основний капітал є елемент витрат
продукції“.\footnote*{
„Circulating capital consists exclusively of subsistence and other necessaries
advanced to the workmen, previous to the completion of the produce of their labour\dots{}
Fixed capital\dots{} alone, not circulating, is properly speaking a source of national
wealth\dots{} Circulating capital is not an inmediate agent in production, nor even essential
to it at all, but merely a convenience rendered necessary by the deplorable poverty
of the mass of the people\dots{} Fixed capital alone constitutes an element of cost of
production in a national point of view“.
} (Ramsay, 1. с., стор. 23--26 passim). Ближче Рамсай так
пояснює основний капітал, що під ним він розуміє сталий: „Час, що
протягом його частина продукту цієї праці (а саме праці, застосованої на
продукцію якогобудь товару) існує як основний капітал, тобто в такій
формі, що в ній вона, хоч і сприяє продукції майбутнього товару, але
не утримує робітників“.\footnote*{
„The length of time during which any portion of the product of that labour
(а саме — labour bestowed on any commodity) has existed as fixed capital; that is
in a form in which, though assisting to raise the future commodity, it does not
maintain labourers“.
} (р. 59).

Тут ми знову бачимо оте лихо, що його наробив А. Сміс, потопивши
ріжницю між сталим та змінним капіталом у ріжниці між основним та
обіговим капіталом. Сталий капітал Рамсая складається з засобів праці,
його обіговий капітал — з засобів існування; і ті й ці є товари даної
вартости, і ті й ці однаково не можуть продукувати додаткову
вартість.

