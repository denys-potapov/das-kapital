
\index{ii}{0338}  %% посилання на сторінку оригінального видання
\subsection[Капітал і дохід: змінний капітал і заробітна плата]{Капітал і дохід: змінний капітал і заробітна плата\footnotemark{}}

\label{original-338}
\noindent{}Ціла
\footnotetext{Відси за рукописом VIII.}
річна репродукція, цілий продукт цього року є продукт корисної
праці за цей рік. Але вартість цього цілого продукту більша, ніж
та частина його вартости, що в ній втілюється річна праця, робоча
сила, витрачена протягом цього року. Новоспродукована вартість
цього року, вартість, новоутворена протягом цього року в товаровій
формі, менша, ніж вартість продукту, ніж вся вартість маси товарів,
виготовлених протягом цілого року. Різність, яка буде, коли з усієї
вартости річного продукту відлічити вартість, долучену до нього працею
поточного року, не є справді репродукована вартість, а вартість, що лише
знову з’явилася в новій формі існування: вартість, перенесена на річний
продукт вартістю, яка існувала раніше від цього продукту, яка — залежно
від тривалости складових частин сталого капіталу, що брали участь у
процесі суспільної праці цього року, — може бути раннішого або пізнішого
походження, яка можливо походить з вартости засобів продукції,
що з’явились на світ минулого року або протягом ряду попередніх років.
В усякому разі це — вартість, перенесена з торішніх засобів продукції
на продукт поточного року.

Коли ми звернемось до нашої схеми, то після обміну розглянутих досі
елементів між І і II і в межах II ми матимемо:

I) $4000 c \dplus{} 1000 v \dplus{} 1000 m$ (останні 2000 реалізуються в засобах
споживання II с) \deq{} 6000.

II) $2000 с$ [репродукуються через обмін з І ($v \dplus{} m$)] \dplus{} $500 v \dplus{} 500 m$ \deq{} 3000.

Сума вартости \deq{} 9000.

Вартість, новоспродукована протягом року, міститься тільки в $v$ і $m$.
Отже, сума новоспродукованої протягом цього року вартости дорівнює
сумі $v \dplus{} m$, \deq{} 2000 І ($v \dplus{} m$) \dplus{} 1000 II ($v \dplus{} m$) \deq{} 3000. Всі інші частини
вартости продукту цього року є лише вартість, перенесена з вартости
попередніх засобів продукції, зужиткованих на річну продукцію.
Крім вартости в 3000, праця поточного року не випродукувала жодної
іншої вартости; це — вся нова вартість, спродукована нею протягом року.

Але, як ми бачили, 2000 І ($v \dplus{} m$) заміщують для II підрозділу 2000
ІІ $с$ в натуральній формі засобів продукції. Отже, дві третини річної
праці, витрачені в категорії І, знову випродукували сталий капітал II, —
як усю його вартість, так і його натуральну форму. Отже, з суспільного
погляду, дві третини праці, витраченої протягом року, створили нову сталу
капітальну вартість, реалізовану в натуральній формі, відповідній підрозділові
II.~Отже, більшу частину річної суспільної праці витрачено на
продукцію нового сталого капіталу (капітальної вартости, що існує в
засобах продукції) для заміщення сталої капітальної вартости, витраченої
на продукцію засобів споживання. Капіталістичне суспільство в цьому
\parbreak{}  %% абзац продовжується на наступній сторінці
