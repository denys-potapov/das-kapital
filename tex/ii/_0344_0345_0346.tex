
\index{ii}{0344}  %% посилання на сторінку оригінального видання
Але ті 500 в грошах, що повернулися до капіталіста II, є разом з
тим відновлений потенціяльний змінний капітал у грошовій формі. Чому
це? Гроші, отже, і грошовий капітал є потенціяльний змінний капітал
лише тому й остільки, що й оскільки їх можна перетворити на робочу
силу. Поворот цих 500\pound{ф. стерл} грішми до капіталіста II супроводиться
поворотом робочої сили II на ринок. Поворот грошей і робочої сили на
протилежні полюси — а значить, і з’явлення знову цих 500 в грошовій
формі, не лише як грошей, а також і як змінного капіталу в грошовій
формі — зумовлено тією самою процедурою. Гроші \deq{} 500 повертаються до
капіталіста II тому, що він продав робітникові II засобів споживання на
суму 500, отже, тому, що робітник витратив свою заробітну плату й таким
чином дістав змогу утримувати себе й родину, а тим самим і свою
робочу силу. Щоб йому можна було й далі існувати, і далі виступати
покупцем товарів, він мусить знову продати свою робочу силу. Отже,
поворот до капіталіста II цих 500 грішми є разом з тим поворот, зглядно
збереження, робочої сили як товару, що його можна купити на ці 500
грішми, а тому це є поворот цих 500 грішми як потенціяльного змінного
капіталу.

Щодо категорії II~$b$, яка продукує речі розкошів, то з її $v$ —
(II~$b$) — справа така сама, як і з I~$v$. Гроші, що відновлюють капіталістам
II~$b$ їхній змінний капітал в грошовій формі, припливають до них
обкружним шляхом, через руки капіталістів II~$а$. А проте, є ріжниця в
тому, чи купують робітники засоби свого існування безпосередньо у тих
капіталістичних продуцентів, що їм вони продають свою робочу силу, чи
купують їх у другої категорії капіталістів, за посередництвом яких гроші
повертаються до перших лише обкружним шляхом. А що робітнича кляса
живе з дня на день, то вона купує, поки може купувати. Інша справа
з капіталістом, прим., при обміні 1000 II~$с$ на 1000 І~$v$. Капіталіст живе
не з дня на день. Рушійний мотив для нього — якомога значніше збільшення
вартости його капіталу. Тому, коли постають якісь обставини, що,
зважаючи на них, капіталістові II здається вигідніше, замість відновити
безпосередньо свій сталий капітал, хоча б почасти затримати його в
грошовій формі на більш-менш довгий час, то зворотний приплив цих
1000 ІІ~$с$ (в грошах) до І уповільнюється; уповільнюється, отже, і відновлення
$1000 v$ в грошовій формі, і капіталіст І може провадити далі
роботу в попередньому маштабі лише тоді, коли в його розпорядженні
є запасні гроші, як і взагалі потрібен запасний капітал в грошовій формі
для того, щоб можна було безперервно провадити роботу незалежно від
швидшого або повільнішого зворотного припливу змінної капітальної вартости
в грошах.

Коли треба дослідити обмін різних елементів поточної річної репродукції,
то при цьому треба дослідити й результат минулої річної праці,
праці вже закінченого року. Продукційний процес, що його результат є
цей річний продукт, лежить позад нас, минув, злився з своїм продуктом;
отже, то більше це має силу для процесу циркуляції, що передує
процесові продукції або відбувається рівнобіжно з ним, — для перетворення
\index{ii}{0345}  %% посилання на сторінку оригінального видання
потенціяльного на справжній змінний капітал, тобто для купівлі
й продажу робочої сили. Робочий ринок уже не становить частини того
товарового ринку, що тут є перед нами. Тут робітник не тільки вже продав
свою робочу силу, а й дав у товарі, крім додаткової вартости,
еквівалент ціни своєї робочої сили; з другого боку, заробітна плата є
вже в його кишені і в обміні він фігурує лише як покупець товару
(засобів споживання). Але далі річний продукт мусить мати в собі всі
елементи репродукції, мусить відновити всі елементи продуктивного капіталу,
— отже, насамперед, найважливіший елемент його — змінний капітал.
І ми справді бачили, що відносно до змінного капіталу результат обміну
такий: робітник як покупець товару, витрачаючи свою заробітну плату
й споживаючи куплений товар, зберігає й репродукує свою робочу силу
як єдиний товар, що його він може продавати; як гроші, авансовані
капіталістом на закуп цієї робочої сили, повертаються до капіталіста, так
і робоча сила, як товар, обмінюваний на ці гроші, повертається на робочий
ринок; в наслідок цього ми тут, а саме для 1000 I~$v$, маємо таке:
на боці капіталістів І — $1000v$ грішми; на протилежному боці, на
боці робітників І — робоча сила вартістю в 1000, отже, ввесь процес
репродукції І може початися знову. Це — один результат процесу
обміну.

З другого боку, витрачення заробітної плати робітників І забрало
<<<<<<< HEAD
в II засобів споживання на суму $1000с$ і таким чином перетворило їх з
товарової форми на грошову форму; II з цієї грошової форми перетворив
їх знову на натуральну форму свого сталого капіталу за допомогою
закупу товарів на суму $1000v$ у І; в наслідок цього до І повертається
=======
в II засобів споживання на суму $1000 с$ і таким чином перетворило їх з
товарової форми на грошову форму; II з цієї грошової форми перетворив
їх знову на натуральну форму свого сталого капіталу за допомогою
закупу товарів на суму $1000 v$ у І; в наслідок цього до І повертається
>>>>>>> c43b47a74b93878bdf3a775317e7242c2086753c
його змінна капітальна вартість знову в грошовій формі.

Змінний капітал І пророблює три перетворення, що зовсім не виявляються
при обміні річного продукту, або виявляються лише як натяк.

1) Перша форма, 1000 I~$v$ в грошах, які перетворюються на робочу
силу того ж розміру вартости. Саме це перетворення не виявляється в
товаровому обміні між І і II, але його результат виявляється в тому, що
кляса робітників І з 1000 в грошах протистоїть продавцеві товарів II,
цілком так само, як кляса робітників II з 500 в грошах протистоїть
продавцеві товарів — 500 II~$v$ в товаровій формі.

2) Друга форма, — єдина, що в ній змінний капітал справді змінюється,
функціонує як змінний, що в ній вартостетворча сила виступає замість
обміненої на неї даної вартости, — належить виключно до продукційного
процесу, що лежить позаду нас.

3) Третя форма, що в ній в наслідок продукційного процесу змінний
капітал виявляв себе як такий, є новоспродукована вартість, отже,
в І \deq{} $1000 v \dplus{} 1000 m$ \deq{} 2000 І ($v \dplus{} m$). Замість його первісної вартости \deq{}
1000 грішми виступає вдвоє більша вартість \deq{} 2000 в товарах.
А тому змінна капітальна вартість \deq{} 1000 в товарах становить лише
половину тієї нової вартости, що її утворив змінний капітал як елемент
продуктивного капіталу. Ці 1000 I~$v$ в товарах є точний еквівалент тієї
змінної за її призначенням частини всього капіталу, яку первісно І авансував
\index{ii}{0346}  %% посилання на сторінку оригінального видання
в $1000v$ грішми; але в товаровій формі вони є гроші лише потенціяльно
(дійсними грішми вони стають тільки в наслідок продажу), отже,
ще менше вони є безпосередньо змінний грошовий капітал. Кінець-кінцем,
вони стають ним через продаж товару 1000 I~$v$ покупцеві II~$c$ і через
те, що робоча сила одразу знову з’являється як продажний товар, як
матеріял, що на нього можуть перетворитись $1000v$ грішми.

Підчас усіх цих перетворень капіталіст І постійно має в своїх руках
змінний капітал: 1) спочатку як грошовий капітал; 2) потім як елемент
його продуктивного капіталу; 3) ще пізніше як частину вартости його
товарового капіталу, тобто в товаровій вартості; 4) нарешті, знову в грошах,
що їм знову протистоїть робоча сила, на яку їх можна перетворити.
Протягом процесу праці капіталіст має в своїх руках змінний капітал як
діющу вартостетворчу робочу силу, а не як вартість даної величини;
що капіталіст завжди оплачує робітника лише після того, як сила його
діяла вже певний коротший або довший час, то перш ніж оплатити її,
він уже одержує в свої руки утворену нею вартість як еквівалент її
самої плюс додаткова вартість.

А що змінний капітал в тій або іншій формі постійно
лишається в руках капіталіста, то ні в якому
разі не можна сказати, що він перетворюється для будького
на дохід. Навпаки, 1000 I~$v$ в товарі перетворюється на гроші
через продаж покупцеві II, що для нього таким чином заміщується
in natura половина його сталого капіталу.

Не змінний капітал І, $1000v$ в грошах, сходить на дохід. Ці гроші,
скоро вони перетворені на робочу силу, перестають функціонувати як
грошова форма змінного капіталу І, — так само, як і гроші всякого іншого
продавця товарів перестають репрезентувати щось належне йому,
скоро він їх перетворить на товар якогось продавця. Перетворення, що
їх пророблюють в руках робітничої кляси гроші, одержані як заробітна
плата, є перетворення не змінного капіталу, а перетвореної на гроші
вартости робочої сили робітничої кляси; цілком так само, як перетворення
новоутвореної робітником вартости [2000 І ($v \dplus{} m$)] є лише перетворення
належного капіталістові товару, перетворення, яке зовсім
не стосується до робітника. Але капіталіст, — а ще більше його теоретичний
тлумач, політикоеконом силу в силу може визволитись від уявлення,
ніби гроші, виплачені робітникові, все ще є його, капіталіста, гроші.
Коли капіталіст є продуцент золота, то змінна частина вартости, тобто
той еквівалент у товарі, що заміщує йому купівельну ціну праці, сама
безпосередньо з’являється в грошовій формі, а тому знову, без обкружних
шляхів зворотного припливу, може функціонувати як змінний грошовий
капітал. Але щодо робітника в II, — оскільки ми лишаємо осторонь
робітників, що продукують речі розкошів — то саме $500v$ існує в товарах,
призначених на споживання робітникові, і їх він, розглядуваний як
збірний робітник, безпосередньо знову купує в того самого збірного
капіталіста, що йому він продав свою робочу силу. Змінна частина
вартости капіталу II своєю натуральною формою складається з засобів
\parbreak{}  %% абзац продовжується на наступній сторінці
