\parcont{}  %% абзац починається на попередній сторінці
\index{ii}{0348}  %% посилання на сторінку оригінального видання
перетвориться на гроші, як і всякий інший, але після перетворення його
на гроші виявляється його відмінність від інших елементів вартости.
Щоб почалась репродукція товарів (взагалі, щоб процес продукції
товарів був безперервний), сировинні і допоміжні матеріяли, зужитковані на
продукцію товарів, мають бути заміщені in natura; витрачена на неї робоча
сила так само має бути заміщена новою робочою силою. Отже, вторговані
за товар гроші постійно доводиться знову й знову перетворювати на ці елементи
продуктивного капіталу, з грошової форми на товарову. Справа зовсім
не змінюється від того, що, наприклад, сировинні та допоміжні матеріяли
в певні строки закуповують великими масами, і вони утворюють
продукційні запаси, отже, що протягом певного часу не доводиться купувати
знову цих засобів продукції, а тому, від того, що поки ще
лишаються ці засоби продукції — гроші, одержувані від продажу товарів,
оскільки вони служать для цієї мети — можна нагромаджувати, і тому
ця частина сталого капіталу тимчасово являє собою грошовий капітал,
що його активне функціонування відкладено. Це — не капітал \deq{} дохід; це
продуктивний капітал, затримуваний у грошовій формі. Відновлення засобів
продукції має відбуватись постійно, хоч форма цього відновлення
— щодо циркуляції — може бути різна. Новий закуп, операція циркуляції,
що нею їх відновлюється, заміщується, може відбуватись через довгі переміжки
часу: в такому разі великі одночасні грошові витрати компенсовані
відповідним пропорційним запасом, або ця операція відбувається через невеликі
послідовні переміжки часу: в такому разі швидко одна по одній невеликі
порції грошових витрат, незначні продукційні запаси. Це зовсім не
змінює самої справи. Так само й з робочою силою. Коли ж продукцію
провадиться безперервно протягом року в тому самому маштабі: постійне
заміщення спожитої робочої сили на нову; коли робота має сезоновий характер
або в різні періоди прикладається різні маси праці, як от в хліборобстві,
— то відповідно до цього закуп раз меншої, раз більшої маси робочої
сили. Навпаки, гроші, вторговані від продажу товарів, оскільки вони
є та перетворена на гроші частина вартости товару, яка дорівнює зношуванню
основного капіталу, не перетворюються знову на складову частину
того продуктивного капіталу, що втрату вартости його вони заміщують.
Вони осаджуються поряд продуктивного капіталу й залишаються в своїй
грошовій формі. Таке осаджування грошей повторюється, поки мине
складена з більшого або меншого числа років доба репродукції, що
протягом її основний елемент сталого капіталу в своїй старій натуральній
формі й далі функціонує в процесі продукції. Скоро лише основний
елемент — будівлі, машини і т. ін. — відживе свій вік, втратить здібність
функціонувати в процесі репродукції, вартість його вже існує біля нього,
цілком заміщена грішми — сумою грошових осадів, вартостей, поступінно
перенесених з основного капіталу на товари, що в продукції їх він брав
участь, і перетворених на грошову форму через продаж товарів. Ці гроші
служать потім для того, щоб замістити in natura основний капітал (або
елементи його, бо різні елементи його мають різний протяг життя) і таким
чином справді відновити цю складову частину продуктивного капіталу.
\parbreak{}  %% абзац продовжується на наступній сторінці
