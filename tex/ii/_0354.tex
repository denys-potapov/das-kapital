\parcont{}  %% абзац починається на попередній сторінці
\index{ii}{0354}  %% посилання на сторінку оригінального видання
ролях „великої публіки“ робить політикоекономам ту „послугу“, що
пояснює нез’ясоване ними.

Так само мало допомагає, коли, замість безпосереднього обміну між
І і II — між двома великими підрозділами самих капіталістичних продуцентів,
— притягують торговця як посередника і за допомогою його „грошей“
перемагають усі труднощі. Напр., в даному разі 200 І m, кінець-кінцем,
і остаточно мусять бути продані промисловим капіталістам II. Хай вони
перейдуть через руки ряду торговців — і все ж останній з них буде, згідно
з гіпотезою, в такому ж відношенні проти II, в якому спочатку були
капіталістичні продуценти І, тобто вони не можуть продати ІІ-му 200 І m,
і заїжджена купівельна сума не може відновити той самий процес для I.

З цього видно, як конче потрібно, незалежно від нашої справжньої
мети, дослідити процес репродукції в його фундаментальній формі, —
де усунуто всі бічні обставини, що затемнюють справу, — як конче
потрібно це для того, щоб відкинути фалшиві викрути, які дають подобу
„наукового“ пояснення, коли за предмет аналізи з самого початку
береться суспільний процес репродукції, в його заплутаній конкретній
формі.

Отже, закон, що згідно з ним гроші, авансовані капіталістичним продуцентом
для циркуляції, при нормальному перебігу репродукції (хоч у
попередньому, хоч у поширеному маштабі) мусять повертатися до свого
вихідного пункту (при цьому байдуже, чи належать ці гроші капіталістичним
продуцентам, чи їх позичено), — цей закон раз назавжди виключає
ту гіпотезу, що 200 II с (d) перетворюються на гроші за допомогою
грошей, авансованих підрозділом I.

\subsubsection{Заміщення основного капіталу in naturа}

Після того, як ми відкинули щойно розглянуту гіпотезу, лишаються
тільки такі можливості, що, крім заміщення грішми зношеної частини, відбувається
також і заміщення цілком відмерлого основного капіталу in natura.

До цього часу ми припускали:

а) Що 1000\pound{ ф. стерл.}, видані І підрозділом на заробітну плату, витрачають
робітники на II с тієї самої величини вартости, тобто, що вони
купують на ці 1000\pound{ ф. стерл.} засоби споживання.

Що тут І авансує ці 1000\pound{ ф. стерл.} грішми, це є лише констатування
факту. Відповідні капіталістичні продуценти повинні виплатити заробітну
плату грішми; потім робітники витрачають ці гроші на засоби
існування, а для продавців засобів існування ці гроші знову таки правлять
за засіб циркуляції при перетворенні їхнього сталого капіталу
з товарового капіталу на продуктивний капітал. Вони переходять при
цьому через багато каналів (дрібні крамарі, домовласники, збирачі
податків, непродуктивні робітники, як от лікарі і т. ін., потрібні самому
робітникові), і тому лише частина їх безпосередньо з рук робітників І
припливає до рук кляси капіталістів II. Цей приплив може більш або
менш затриматись, а тому можуть бути потрібні нові грошові резерви
\parbreak{}  %% абзац продовжується на наступній сторінці
