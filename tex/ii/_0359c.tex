
\index{ii}{0359}  %% посилання на сторінку оригінального видання
Підрозділ I на 200\pound{ ф. стерл.} грішми купує в частини 1 товару на
200, і в наслідок цього до неї повертаються її 200\pound{ ф. стерл.} грішми,
авансовані на цей товаровий обмін; на другі 200\pound{ ф. стерл.}, — що їх він
теж одержав від частини 1, — I купує товарів на 200 у частини 2, і в
наслідок цього в останньої зношування її основного капіталу осідає в
формі грошей.

Справа зовсім не змінилась би, коли б ми припустили, що в випадку
$c$ не кляса II (частина 1), а кляса I авансує 200 грішми на обмін наявних
товарів. Потім, коли I купить спочатку у II, частини 2, товарів на
200, — припускається, що частині 2 лишається ще продати тільки цю
решту товару, — то ці 200\pound{ ф. стерл.} не повертаються до I, бо II, частина
2, не виступає знову як покупець; але II, частина 1, має тоді 200\pound{ ф. стерл.}
грішми для закупу, а також ще на 200 товарів для обміну, отже, всього
400 для обміну з I. Тоді 200\pound{ ф. стерл.} грішми від II, частини 1, повертаються
назад до I. Коли I знову витрачає їх, щоб купити товару на
200 у II, частини 1, то вони знову повернуться до нього, коли II,
частина 1, візьме в I другу половину 400 в товарах. Частина 1 (II)
витратила 200\pound{ ф. стерл.} грішми просто як покупець елементів основного
капіталу; тому вони не повертаються до неї, а служать для того,
щоб перетворити на гроші $200 c$, решту товарів II, частини 2, тимчасом
як до I гроші, витрачені ним на товаровий обмін, 200\pound{ ф. стерл.}, повертаються
не через II, частину 2, а через II, частину 1. За його товари
на 400 до нього повернувся товаровий еквівалент розміром в 400;
200\pound{ ф. стерл.} грішми, авансовані ним для обміну товарів на 800, теж
повернулись до нього й таким чином усе гаразд.

\pfbreak

Труднощі, які постали при обміні:
\[
I.    $1000 v \dplus{} 1000 m$

II.    2000 с, зійшли на труднощі при обміні решток:

I\dotfill $400 m$

II (1) 200 грішми \dplus{} $200 c$ товарами \dplus{} (2) $200 c$ товарами, або, щоб
подати справу ще ясніше:

I.    $200 m \dplus{} 200 m$

II. (1) 200 грішми \dplus{} $200 c$ товарами \dplus{} (2) $200 c$ товарами.
\]
А що для II, частини 1, $200 c$ товарами обмінюємся на 200 I$m$
(товарами) і що всі гроші, які циркулюють між І і II при цьому обміні
товарів на 400, повертаються назад до того, хто їх авансував, — або до
I або до II, — то ці гроші, як елемент обміну між I і II, в суті не є
елемент тієї проблеми, яка цікавить нас тепер. Або в іншому освітленні:
коли ми припустимо, що в обміні між 200 I$m$ (товарами) і 200 II$c$
(товарами II, частина 1) гроші функціонують як засіб виплати, а не як
купівельний засіб, а тому й не як „засіб циркуляції“ у вузькому значенні,
то зрозуміло, — бо товари 200 I$m$ і 200 II$c$ (частина 1)
рівні величиною вартості, — що засоби продукції вартістю в 200 обмінюються
на засоби споживання вартістю в 200, що гроші тут функціонують
лише ідеально, і що в дійсності зовсім не доводиться ні з того,
\parbreak{}  %% абзац продовжується на наступній сторінці
