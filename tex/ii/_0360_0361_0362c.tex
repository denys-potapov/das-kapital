\parcont{}  %% абзац починається на попередній сторінці
\index{ii}{0360}  %% посилання на сторінку оригінального видання
ні з другого боку подавати гроші в циркуляцію для оплати ріжниці.
Отже, проблема виступає в своєму чистому вигляді лише тоді, коли ми
викреслимо на обох боках, І і II, товар 200 I~$m$ і його еквівалент товар
200 ІІ~$с$ (частини 1).

Отже, усунувши ці дві товарові величини рівної вартости (І і II),
що навзаєм одна одну урівноважують, матимемо решту обміну, в якому
проблема виступає в чистому вигляді, а саме:

I. 200 $m$ товаром.

II. (1) 200$c$ грішми \dplus{} (2) 200$c$ товаром.

Тут очевидно: II, частина 1, на 200 грішми купує складові частини
свого основного капіталу, 200 І~$m$, в наслідок цього основний капітал II,
частини 1, відновлено in natura, а додаткова вартість І, вартістю в 200,
з товарової форми (засоби продукції, — а саме, елементи основного
капіталу) перетворена на грошову форму. На ці гроші І купує засоби
споживання у II, частини 2, а результат для II такий, що для частини 1
відновлено in natura основну складову частину її сталого капіталу; і що
для частини 2 друга складова частина (яка замішує зношування основного
капіталу) осіла в формі грошей, і це щороку повторюється доти,
доки й цю складову частину треба буде відновити in natura.

Попередня умова тут, очевидно, в тому, щоб ця основна складова частина
сталого капіталу II, яка в розмірі всієї своєї вартости зворотно перетворюється
на гроші, і яку, отже, кожного року треба відновлювати in natura
(частина 1), — щоб вона була рівна річному зношуванню другої основної
складової частини сталого капіталу II, яка все ще й далі функціонує в
своїй старій натуральній формі, і зношування якої — втрату вартости,
переношувану на товари, що в їхній продукції функціонує ця частина —
спочатку треба замістити грішми. Тому така рівновага з’являється як
закон репродукції в незмінному маштабі; це значить, інакше кажучи, що
пропорційний поділ праці в клясі 1, яка продукує засоби продукції,
мусить лишатись незмінний, оскільки вона дає, з одного боку, обігові,
а з другого — основні складові частини сталого капіталу підрозділові II.

Перш ніж ближче дослідити це, ми повинні розглянути, який вигляд
матиме справа, коли решта суми II~$с$ (1) не дорівнюватиме решті II~$с$
(2); вона може бути більша або менша за цю останню. Візьмімо один
по одному обидва ці випадки.

\so{Перший випадок:}

I.    200 $m$.

II. (1) 220 $с$ (грішми) \dplus{} (2) 200 $с$ (товаром).

Тут II~$с$ (1) на 220\pound{ ф. стерл.} грішми купує товари 200 I~$m$, а І на
ті самі гроші купує товари 200 ІІ~$с$ (2), тобто ту складову частину
основного капіталу, яка має осісти в грошовій формі; її перетворено
таким чином на гроші. Але 20 II~$с$ (1) грішми не сила перетворити на
основний капітал in natura.

Цьому лихові можна, здається, запобігти, коли ми припустимо, що решта
I~$m$ дорівнює не 200, а 220, так що з суми 2000 І попереднім обміном закінчено
\index{ii}{0361}  %% посилання на сторінку оригінального видання
справу не з 1800, а лише з 1780. Отже, в такому разі матимемо:

I.    220 $m$

II. (1) 220 $с$ (грішми) \dplus{} (2) 200 $с$ (товаром).

ІІ~$с$, частина 1, на 220\pound{ ф. стерл.} грішми купує 220 І~$m$, а І на
200\pound{ ф. стерл.} купує потім 200 II~$с$ (2) товаром. Але тоді на боці І
лишається 20\pound{ ф. стерл.} грішми — така частина додаткової вартости, яку
I може лише затримати в грошах, а не витрачати на засоби споживання.
Таким чином труднощі лише переміщено з II~$с$ (частина 1) на І~$m$.

Припустімо тепер, з другого боку, що ІІ~$с$, частина 1, менше, ніж ІІ~$с$
(частина 2), отже:

\so{Другий випадок:}

I.    200 $m$ (товаром).

II. (1) 180 $с$ (грішми) \dplus{} (2) 200 $с$ (товаром).

II. (частина 1), на 180\pound{ ф. стерл.} грішми купує товари 180 І~$m$; на ці
гроші І купує в II (частини 2) товари такої самої вартости, тобто 180

II.    $с$ (2); на одному боці лишається 20 І~$m$, що їх не сила продати, і так
само — 20 II~$с$ (2) на другому боці; товари вартістю в 40 не сила перетворити
на гроші.

Коли б ми припустили, що остача I \deq{} 180, це нам ані трохи не допомогло
б; правда, тоді в І не залишилося б жодного надлишку, але в II~$с$
(частині 2), як і раніш, був би надлишок в 20, що його не сила продати,
перетворити на гроші.

В першому випадку, де II (1) більше, ніж II (2), на боці II~$с$ (1)
лишається надлишок в грошах, що його не сила перетворити знову на
основний капітал, або, коли ми припустимо, що остача І~$m$ \deq{} ІІ~$с$ (1) на
боці І~$m$ буде такий самий надлишок у грошах, не перетворюваний на
засоби споживання.

В другому випадку, де ІІ~$с$ (1) менше, ніж ІІ~$с$ (2), виявляється грошовий
дефіцит на боці 200 І~$m$ і II~$с$ (2) і на обох боках товаровий надлишок
однакової величини, або коли припустити, що остача І~$m$ \deq{} ІІ~$с$ (2), дефіцит
в грошах і надлишок у товарі на боці II~$с$ (2).

Коли б ми припустили, що остачі І завжди дорівнюють ІІ~$с$ (1), — бо
продукція визначається замовленнями і в репродукції нічого не змінюється,
коли поточного року випродукувано більше основних складових частин
капіталу, а другого наступного року більше обігових складових частин
сталого капіталу II і І, — то в першому випадку І~$m$ можна було б знову
перетворити на засоби споживання лише тоді, коли б І купив на І~$m$
частину додаткової вартости у II, отже, коли б І її не споживав, а нагромаджував
як гроші; в другому випадку лихові можна було б запобігти
лише тоді, коли б І сам витратив гроші, — а цю гіпотезу ми
відкинули.

Коли ІІ~$с$ (1) більше, ніж ІІ~$с$ (2), то для реалізації грошового надлишку
в І~$m$ потрібен довіз закордонних товарів. Коли ІІ~$с$ (1) менше, ніж ІІ~$с$
(2), то для того, щоб реалізувати зношену частину II~$с$ в засобах продукції,
\index{ii}{0362}  %% посилання на сторінку оригінального видання
потрібен, навпаки, вивіз товару II (засобів споживання). Отже, в
обох випадках потрібна зовнішня торговля.

Даймо навіть, що при вивчанні репродукції в незмінному маштабі
треба припустити, що продуктивність усіх галузей продукції, а значить,
і пропорційні відношення вартостей товарових продуктів цих галузей,
лишаються незмінні, — все ж обидва останні випадки, де ІІ~$с$ (1) більше
або менше, ніж II~$с$ (2), являли б інтерес при вивчанні продукції в поширеному
маштабі, де, безперечно, можуть настати ці випадки.

\emph{3) Результати}  %% TODO Checck numbering


Щодо заміщення основного капіталу, то взагалі треба зазначити ось що.

Коли — припускаючи, що всі інші умови, а значить, не лише маштаб
продукції, а зокрема й продуктивність праці лишаються незмінні, —
поточного року відмирає більша частина основного елемента ІІ~$с$, ніж у
попередньому році, а тому й більшу частину треба відновлювати in
natura, то та частина основного капіталу, яка є лише на шляху до своєї
смерти й яку до моменту її смерти треба покищо заміщувати в грошах,
теж мусить зменшитись у такій самій пропорції, бо згідно з припущенням
сума (також і сума вартости) основної частини капіталу, діющої в II
лишається та сама. Але це тягне за собою такі обставини. \emph{Поперше}. Коли
більша частина товарового капіталу І складається з елементів основного
капіталу II~$с$, то відповідно менша частина складається з обігових складових
частин ІІ~$с$, бо вся продукція І для ІІ~$с$ лишається незмінна. Коли одна
частина збільшується, то друга зменшується й навпаки. Але, з другого
боку, величина всієї продукції кляси II такожа лишається незмінна. Як
же можливо це, коли меншає в неї сировинних матеріялів, напівфабрикатів,
допоміжних матеріялів (тобто обігових елементів сталого капіталу II)?
\emph{Подруге}. Більша частина основного капіталу II~$с$, знову відновленого в
грошовій формі, припливає до І, щоб знову перетворитись з грошової
форми на натуральну форму. Отже, до І, крім грошей, що циркулюють
між І і II для простого товарного обміну, припливає більше грошей;
більше таких грошей, які правлять не за посередника у взаємному товаровому
обміні, а однобічно виступають лише в функції купівельного
засобу. Але разом з тим пропорційно зменшилась би товарова маса
II~$с$, що є носій вартости на заміщення зношування, тобто та товарова
маса II, яка мусить бути перетворена не на товари І, а лише на гроші І.~Від II до І приплило б більше грошей як просто купівельних засобів і
було б менше товарів у II, що супроти них І мав би функціонувати як
простий покупець. Більшу частину І~$m$, — бо І~$v$ уже перетворено на товари
II, — не сила було б перетворити на товари II, її довелось би затримати в
грошовій формі.

Зворотний випадок, — коли протягом року репродукція відмерлого
основного капіталу II менша і навпаки, частина на заміщення зношування
більша, — після попереднього не потребує дальшого розгляду.

І, таким чином, настала б криза — криза продукції — не зважаючи на
репродукцію в незмінному маштабі.
