\parcont{}  %% абзац починається на попередній сторінці
\index{ii}{0368}  %% посилання на сторінку оригінального видання
від грошей, що в них вона витрачає дохід. Відки беруться ці останні
гроші? Та просто з тієї маси грошей, що є в руках кляси капіталістів,
отже, взагалі та в цілому з усієї маси грошей, яка є в суспільстві,
деяка частина служить для циркуляції доходу капіталістів. Ми
вже бачили вище, як кожен капіталіст, що відкриває нове підприємство,
витрачає гроші на засоби споживання для власного утримання, потім,
коли підприємство вже працює, знову виловлює їх як гроші, що служать
для перетворення на гроші його додаткової вартости. Але загалом кажучи,
всі труднощі походять з двох джерел.

Поперше, коли ми розглядатимемо тільки циркуляцію та оборот капіталу,
отже, і капіталіста лише як персоніфікацю капіталу, а не як капіталістичного
споживача та розкішника, то, хоч ми побачимо, що він
постійно подає в циркуляцію додаткову вартість як складову частину
свого товарового капіталу, але ми ніколи не побачимо в його руках грошей
як форми доходу; ми ніколи не побачимо, щоб він подавав у циркуляцію
гроші для споживання додаткової вартости.

Подруге, коли кляса капіталістів подає в циркуляцію певну грошову
суму в формі доходу, то здається, ніби вона виплачує еквівалент за цю
частину цілого річного продукту, і тому ця остання перестає вже репрезентувати
додаткову вартість. Але додатковий продукт, що в ньому
втілено додаткову вартість, нічого не коштує клясі капіталістів. Як кляса,
вона має й користається з неї безплатно, і грошова циркуляція нічого
не може змінити в цьому. Зміна, зумовлена циркуляцією, є просто в тому,
що кожен капіталіст, замість споживати свій додатковий продукт in natura,
а це здебільша зовсім неможливо, витягує з усієї маси річного суспільного
додаткового продукту й привлащує різного роду товари на суму
привлащеної ним додаткової вартости. Але механізм циркуляції показав,
що коли кляса капіталістів подає в циркуляцію гроші на витрачання
доходу, то вона знову й вилучає з циркуляції ці гроші, а тому завжди
може знову розпочати той самий процес; що, отже, розглядувана як
кляса капіталістів, вона, як і раніш, має цю грошову суму, потрібну для
перетворення додаткової вартости на гроші. Отже, коли капіталіст не
лише вилучає з товарового ринку для свого споживного фонду додаткову
вартість у формі товарів, але разом з тим до нього повертаються назад
і гроші, що на них він купив ці товари, то, очевидно, що він вилучив
з циркуляції товари, не давши за них жодного еквіваленту. Вони нічого
не коштують йому, хоч він заплатив за них гроші. Коли я купую товарів
на фунт стерлінґів, а продавець товару повертає мені цей фунт за додатковий
продукт, що нічого не коштував мені, то я, очевидно, безплатно
одержав товари. Постійне повторення цієї операції нічого не змінює в
тому, що я постійно вилучаю товари й постійно лишаюсь власником
фунта стерлінґів, хоч, щоб одержати товари, я на деякий час віддаю
його. Капіталіст постійно одержує ці гроші назад як перетворену на
гроші додаткову вартість, яка нічого не коштувала йому.

Ми бачили, що в А.~Сміса сукупна суспільна вартість продукту розкладається
на дохід, на $v \dplus{} m$, отже, що стала капітальна вартість у
\parbreak{}  %% абзац продовжується на наступній сторінці
