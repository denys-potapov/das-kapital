\parcont{}  %% абзац починається на попередній сторінці
\index{ii}{0376}  %% посилання на сторінку оригінального видання
мусить вистачити принаймні на те, щоб підтримати її існування й
працездатність, „придбати найпотрібніші засоби існування“ (стор. 180)
Коли робітники не одержують такої достатньої плати, то, за тим самим
Детю, це — „смерть для промисловости“ (стор. 208), отже, здавалось би,
зовсім не засіб збагачення для капіталістів. Але хоч яка буде висота заробітної
плати, виплачуваної клясою капіталістів робітничій клясі, ця плата
має певну вартість, напр., 80\pound{ ф. стерл}. Отже, коли кляса капіталістів
виплачує робітникам 80\pound{ ф. стерл.}, то вона повинна давати їм за ці 80\pound{ ф.
стерл.} товарову вартість в 80\pound{ ф. стерл.}, а тому зворотний приплив цих
80\pound{ ф. стерл.} не збагачує її. А коли вона виплачує їм грішми 100\pound{ ф. стерл.}
і продає їм за 100\pound{ ф. стерл.} товарову вартість в 80\pound{ ф. стерл.}, то вона
виплачує їм грішми на 25\% більше за їхню нормальну заробітну плату
і за те дає їм товарами на 25\% менше.

Інакше кажучи, фонд, що з нього взагалі кляса капіталістів бере свій
зиск, утворювався б з одрахувань з нормальної заробітної плати, в наслідок
оплати робочої сили нижче за її вартість, тобто нижче за вартість засобів
існування, доконечних для нормальної репродукції робочої сили як
найманих робітників. Отже, коли б виплачувалось нормальну заробітну
плату, а так повинно бути за Детю, то не існувало б жодного фонду
зиску ні для промисловців, ні для капіталістів-нероб.

Отже, усю таємницю, як збагачується кляса капіталістів, панові Детю
довелось би звести ось на що: в наслідок одрахувань з заробітної плати.
В такому разі не існує інших фондів додаткової вартости, що про них
він каже в пунктах 1 і 3.

Отже, в усіх країнах, де грошову заробітну плату робітників зведено
на вартість засобів споживання, потрібних на їхнє існування як кляси,
не існувало б ні фонду споживання, ні фонду нагромадження для капіталістів,
а значить, не існувало б і фонду існування кляси капіталістів, а
тому й кляси капіталістів. І за Детю, це саме так було б у всіх бабагатих,
розвинених країнах старої цивілізації, бо тут в „наших суспільствах
старого походження фонд, звідки покривається заробітну плату,
є\dots{} майже стала величина“ (стор. 202).

І коли урізується заробітну плату, збагачення капіталістів випливає
не з того, що вони спочатку виплачують робітникові 100\pound{ ф. стерл.}
грішми, а потім дають йому на ці 100\pound{ ф. стерл.} грішми 80\pound{ ф. стерл.}
товарами, — отже, в дійсності 80\pound{ ф. стерл.} товару пускають у циркуляцію
грошовою сумою а 100\pound{ ф. стерл.}, на 25\% більшою, — але з того, що капіталіст
привлащує собі з продукту робітника, крім додаткової вартости — тієї
частини продукту, що в ній втілюється додаткова вартість, — ще й
25\% тієї частини продукту, яка в формі заробітної плати повинна була
б дістатися робітникові. Таким безглуздим способом, як це уявляє собі Детю,
кляса капіталістів абсолютно нічого не виграла б. Вона платить робітникам
100\pound{ ф. стерл.} як заробітну плату і за ці 100\pound{ ф. стерл.} повертає робітникові
з власного його продукту 80\pound{ ф. стерл.} товарової вартости. Але
при наступній операції вона мусить знову авансувати на ту саму процедуру
100\pound{ ф. стерл}. Отже, вона вдається лише до марної гри, авансуючи
\parbreak{}  %% абзац продовжується на наступній сторінці
