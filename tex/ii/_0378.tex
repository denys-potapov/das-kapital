\index{ii}{0378}  %% посилання на сторінку оригінального видання
В наслідок цього відбувається зворотний приплив 100\pound{ ф. стерл.} грішми, що їх заплатили промислові
капіталісти капіталістам-неробам. Чи є цей зворотний приплив грошей, як фантазує Детю, засіб
збагачення для промислових капіталістів? До оборудки в них була сума вартості в 200\pound{ ф. стерл.}; 100\pound{ ф. стерл.} в грошах і 100\pound{ ф. стерл.} в засобах споживання. Після оборудки у них є лише половина
первісної суми вартости. У них знову є ці 100\pound{ ф. стерл.} грішми, але вони втратили ці 100\pound{ ф. стерл.} в
засобах споживання, що перейшли до рук капіталістів-нероб. Отже, вони зробились не багатші на 100\pound{ ф.
стерл.}, а бідніші на 100\pound{ ф. стерл}. Коли б замість такого обкружного шляху, — спочатку заплатити 100\pound{ ф. стерл.} грішми, а потім одержати ці 100\pound{ ф. стерл.} грішми назад як оплату 100\pound{ ф. стерл.} засобів
споживання, — коли б вони замість цього безпосередньо заплатили ренту, процент і т. ін. в
натуральній формі свого продукту, то до них не повернулось би з циркуляції жодних 100\pound{ ф. стерл.}
грішми, бо вони не подали в циркуляцію жодних 100\pound{ ф. стерл.} грішми. При виплаті натурою справа
стояла б просто так, що вони з додаткового продукту вартістю в 200\pound{ ф. стерл.} одну половину затримали
в себе, а другу половину віддали без еквіваленту капіталістам-неробам. Навіть Детю не міг би
наважитись оголосити це за засіб збагачення.

Земля й капітал, що їх промислові капіталісти позичили у капіталістів-нероб і що за них вони повинні
виплатити їм частину додаткової вартости в формі земельної ренти, проценту й т. інш., звичайно, були
для них зисковні, бо вони були за одну з умов продукції так продукту взагалі, як і тієї частини
продукту, яка становить додатковий продукт або в якій втілюється додаткова вартість. Цей зиск
випливає з користування орендованою землею та капіталом, а не з ціни, що її платиться за це.
Навпаки, ціна ця становить одрахування з зиску. Інакше довелось би стверджувати, що промислові
капіталісти зробились би не багатші, а бідніші, коли б вони могли затримати для самих себе й другу
половину додаткової вартости замість віддавати її. Але до такої плутанини доходять, коли явища
циркуляції, такі, як зворотний приплив грошей, звалюють докупи з розподілом продукту, тимчасом як
цей розподіл лише упосереднюється такими явищами циркуляції.

І все ж той самий Детю такий дотепний, що зауважує: „Відки походять доходи цих людей-нероб? Чи не
походять вони з ренти, яку з свого зиску платять їм люди, що пускають у роботу капітали перших,
тобто люди, які оплачують з фонду перших працю, яка продукує більш, ніж вона коштує, коротко кажучи
промисловці? Отже, щоб відкрити джерело всякого багатства, завжди доводиться повертатись до цих
промисловців. Вони — ось хто дійсно годує робітників, що їх наймають перші“ (стор. 246).

Тепер виходить, що виплата ренти і т. ін. є відрахування з зиску промисловців. Раніш це було за
засіб їхнього збагачення.

Та для нашого Детю лишилась ще одна втіха. Ці браві промисловці поводяться з капіталістами-неробами
так само, як один з одним і з робітниками.
\parbreak{}  %% абзац продовжується на наступній сторінці
