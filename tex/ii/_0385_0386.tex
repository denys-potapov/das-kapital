\parcont{}  %% абзац починається на попередній сторінці
\index{ii}{0385}  %% посилання на сторінку оригінального видання
школа від часів фізіократів і Адама Сміса. Ми знаємо, що основний
капітал, після того як витрату на нього вже раз зроблено, не відновлюється
протягом усього часу свого функціонування, а функціонує й
далі в старій формі, тммчасом як вартість його поступінно осаджується
в формі грошей. Ми бачили, що періодичне відновлення основного капіталу
II с [а вся капітальна вартість II с обмінюється на елементи вартістю
в І ($v \dplus{} m$)] має за передумову, з одного боку, просту купівлю
основної частини II с, яка зворотно перетворюється з грошової форми
на натуральну, при чому цій купівлі відповідає простий продаж І m;
з другого боку, воно має за передумову простий продаж з боку
II с, продаж тієї основної (зношеної) частини його вартости, яка осаджується
в формі грошей, при чому цьому продажеві відповідає проста
купівля І m. Для того, щоб обмін відбувався тут нормально, треба припустити,
що проста купівля з боку II с величиною вартости дорівнює
простому продажеві з боку II с, і так само, що простий продаж І m
1-ій частині II с дорівнює простій купівлі II с, частини 2 (стор. 360).
Інакше просту репродукцію порушиться; проста купівля тут мусить
покриватись простим продажем там. Так само тут треба припустити,
що простий продаж частини І m, яка утворює скарб для А,
А', А'', урівноважується простою купівлею частини І m з боку В, В', В''
і~\abbr{т. д.}, які перетворюють свій скарб на елементи додаткового продуктивного
капіталу.

Оскільки рівновага відновлюється через те, що покупець потім виступає
як продавець на таку саму суму вартости, і навпаки, остільки відбувається
зворотний приплив грошей до тієї сторони, яка авансувала їх
підчас купівлі, яка продала раніше, ніж знову купила. Але дійсна рівновага,
щодо самого товарового обміну, обміну різних частин річного продукту,
зумовлюється рівністю величини вартости обмінюваних один проти
одного товарів.

Але оскільки відбуваються просто однобічні перетворення, прості
купівлі, з одного боку, і прості продажі, з другого, — а ми бачили, що
нормальне перетворення річного продукту на капіталістичній основі зумовлює
такі однобічні метаморфози, — остільки рівновага буде лише при
тому припущенні, що сума вартости однобічних купівель і сума вартости
однобічних продажів покривають одна одну. Та обставина, що товарова
продукція є загальна форма капіталістичної продукції, включає вже й ту
ролю, що її відіграють у ній гроші не лише як засіб циркуляції, а й
як грошовий капітал, і утворює певні, властиві цьому способові продукції
умови нормального обміну, отже, нормального перебігу репродукції,
усе одно, чи в простому, чи в поширеному маштабі, — умови,
що перетворюються на так само численні умови ненормального перебігу
репродукції, на так само численні можливості криз, бо рівновага за стихійного
ладу (naturwüchsigen Gestaltung) цієї продукції — сама є випадок.

Так само ми бачили, що при обміні І v на відповідну суму вартости
II с, саме для II с, кінець-кінцем, відбувається заміщення товару II
рівною сумою вартости товару І, отже, що з боку збірного капіталіста
\index{ii}{0386}  %% посилання на сторінку оригінального видання
II тут продаж власного товару доповнюється купівлею товару І на
таку саму суму вартости. Таке заміщення відбувається; але не відбувається
обміну між самими капіталістами І і II при цьому перетворенні
їхніх товарів. II с продає свої товари робітничій клясі І; ця остання
протистоїть йому однобічно як покупець товарів, а II с протистоїть робітничій
клясі І однобічно як продавець товарів; з грішми, вторгованими
таким чином, II с протистоїть збірному капіталістові І однобічно як покупець
товарів, а збірний капіталіст І однобічно протистоїть йому як продавець
товарів на суму І v. Тільки через цей продаж товарів підрозділ І, кінецькінцем,
репродукує свій змінний капітал знову в формі грошового капіталу.
Коли капітал І протистоїть капіталові II однобічно як продавець
товару на суму І v, то своїй робітничій клясі він протистоїть як покупець
товарів, що купує її робочу силу; і коли робітнича кляса І протистоїть
капіталістові II однобічно як покупець товару (а саме як покупець
засобів існування), то капіталістові І вона протистоїть однобічно як продавець
товару, а саме як продавець своєї робочої сили.

Постійне подання робочої сили з боку робітничої кляси в І, зворотне
перетворення частини товарового капіталу І на грошову форму змінного
капіталу, заміщення частини товарового капіталу II натуральними елементами
сталого капіталу II с — всі ці доконечні передумови навзаєм зумовлюють
одна одну, але їх упосереднює дуже складний процес, який
має в собі три процеси циркуляції, що перебігають незалежно один від
одного, але в той самий час переплітаються один з одним. Складність самого
цього процесу дає так само численні нагоди до ненормального перебігу.

\subsubsection{Додатковий сталий капітал}

Додатковий продукт, носій додаткової вартости, нічого не коштує
капіталістам І, його привлащувачам. Їм не доводиться в жодній формі
авансувати гроші або товари, щоб його одержати. Аванс (avance)
уже у фізіократів є загальна форма вартости, реалізованої в елементах
продуктивного капіталу. Вони, отже, нічого не авансують, крім свого сталого
й змінного капіталу. Своєю працею робітник не лише зберігає їм
їхній сталий капітал; він не тільки заміщує їм змінну капітальну вартість,
утворюючи відповідну нову частину вартости в формі товару; своєю додатковою
працею він, крім того, дає їм додаткову вартість, що існує
в формі додаткового продукту. Послідовно продаючи цей додатковий
продукт, воии утворюють скарб, додатковий потенціальний грошовий капітал.
В розглядуваному тут випадку цей додатковий продукт складається
з самого початку із засобів продукції засобів продукції. Тільки в руках
$В$, $В'$, $В''$ і~\abbr{т. д.} (І) цей додатковий продукт функціонує як додатковий
сталий капітал; але віртуально він є ним раніше, ніж його продасться,
уже в руках утворювачів скарбу А, А', А'' (І). Коли ми розглядаємо
тільки розмір вартости репродукції на боці І, ми перебуваємо
ще в межах простої репродукції, бо жодного додаткового капіталу
не пущено в рух, щоб утворити цей віртуальний додатковий сталий
\parbreak{}  %% абзац продовжується на наступній сторінці
