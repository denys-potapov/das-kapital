
\index{ii}{0388}  %% посилання на сторінку оригінального видання
З цього далі випливає: чим більший продуктивний капітал, уже
діющий в даній країні (зараховуючи долучувану до нього робочу силу,
витворця додаткового продукту), чим більше розвинена продуктивна сила
праці, а разом з тим і технічні засоби для швидкого поширення продукції
засобів продукції, чим більша, отже, маса додаткового продукту
так своєю вартістю, як і кількістю споживних вартостей, що в них
втілюється додаткова вартість, тим більший:

1) віртуально додатковий продуктивний капітал в формі додаткового
продукту в руках $А$, $А'$, $А''$ і~\abbr{т. д.} і

2) тим більша маса цього додаткового продукту, перетвореного на
гроші, отже, віртуально додаткового грошового капіталу в руках $А$, $А'$, $А''$.
Отже, коли, напр., Фулартон нічого не хоче знати про перепродукцію
в звичайному розумінні цього слова, але визнає перепродукцію капіталу,
а саме грошового капіталу, то це ще раз доводить, що навіть найкращі
буржуазні економісти абсолютно нічого не розуміють у механізмі своєї
системи.

Коли додатковий продукт, безпосередньо продукований і привлашуваний
капіталістами $А$, $А'$, $А''$ (І), є реальна база акумуляції капіталу,
тобто поширеної репродукції, хоч він в цій властивості активно функціонує
лише в руках $В$, $В'$, $В''$ і~\abbr{т. д.} (І), — то, навпаки, в своїй грошовій
лялечці, — як скарб і як лише поступінно утворюваний віртуальний
грошовий капітал, — він абсолютно непродуктивний, рухається
в цій формі паралельно з процесом продукції, але перебуває поза ним.
Він є мертвий тягар (dead wight) капіталістичної продукції. Жадоба використати
для одержання зиску й доходу цю додаткову вартість, нагромаджувану
як скарб у формі віртуального грошового капіталу, знаходить
мету своїх прагнень в кредитовій системі і в „папірцях“. Тому грошовий
капітал в іншій формі набирає величезного впливу на перебіг і потужний
розвиток капіталістичної системи продукції.

Додатковий продукт, перетворений на віртуальний грошовий капітал,
буде своєю масою тим більший, чим більша була вся сума вже діющого
капіталу, що в наслідок його функціонування постав цей додатковий
продукт. Але при абсолютному збільшенні розміру щорічно репродуковуваного
віртуального грошового капіталу полегшується і його сегментація,
так що його швидше можна вкласти в особливе підприємство, хоч буде
воно у руках того капіталіста, хоч в інших руках (напр., членів його родини,
при поділі спадщини і~\abbr{т. ін.}). Сегментація грошового капіталу значить
тут, що він цілком відокремлюється від первісного капіталу, щоб як
новий грошовий капітал приміститись у новому самостійному підприємстві.

Коли продавці додаткового продукту $А$, $А'$, $А''$ і~\abbr{т. д.} (І) одержали
його як такий безпосередній наслідок процесу продукції, що, крім авансування
на сталий І змінний капітал, потрібного й за простої репродукції,
не має собі за передумову дальших актів циркуляції, коли вони
далі цим самим утворюють реальну базу для репродукції в поширеному
маштабі, дійсно Фабрикують віртуально додатковий капітал, то
\parbreak{}  %% абзац продовжується на наступній сторінці
