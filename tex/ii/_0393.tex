
\index{ii}{0393}  %% посилання на сторінку оригінального видання
Коли, напр., половину додаткового продукту І, отже, $\sfrac{1000}{2} m$ або
500 І m знову вводиться як сталий капітал у підрозділ І, то ця частина
додаткового продукту, затримана в І, не може замістити жодної частини
II с. Замість перетворитись на засоби споживання (а тут у цьому підрозділі
циркуляції — на відміну від заміщення 1000 II с через 1000 І v,
що його упосереднюють робітники, — між І і II відбувається справжній
взаємний обмін, тобто двобічне переміщення товарів) ця частина мусить
служити за додаткові засоби продукції в самому І.~Вона не може виконувати
таку функцію одночасно в І і II.~Капіталіст не може витрачати
вартість свого додаткового продукту на засоби споживання й разом
з тим продуктивно споживати самий додатковий продукт, тобто долучати
його до свого продуктивного капіталу. Отже, замість 2000 І ($v \dplus{} m$)
лише 1500, а саме ($1000 v \dplus{} 500 m$) І можна обміняти на 2000 II с;
отже, 500 II с не можуть знову перетворитися з своєї товарової форми
на продуктивний (сталий) капітал II.~Отже, в II сталася б перепродукція,
яка своїми розмірами точно відповідала б розмірам поширення продукції
в І, Можливо, що перепродукція в II так дуже впливала б на І,
що навіть зворотний приплив 1000, витрачених робітниками І на засоби
споживання II, відбувся б лише почасти, отже, ці 1000 не повернулися б
в формі змінного грошового капіталу до рук капіталістів І.~Таким чином,
для цих останніх утруднилось би навіть репродукцію в попередніх
розмірах і саме в наслідок простої лише спроби поширити її. І при цьому
треба взяти на увагу, що фактично в І відбулась лише проста репродукція,
і що елементи її, як їх подано в нашій схемі, лише згруповано інакше
задля майбутнього поширення її, напр. у наступному році.

Можна було б зробити спробу обійти цю трудність так: 500 II с,
що лежать на складі у капіталістів і що їх не можна безпосередньо
перетворити на продуктивний капітал, зовсім не являють перепродукції,
а, навпаки, являють доконечний елемент репродукції, що його ми досі
не брали на увагу. Ми бачили, що грошовий запас мусить нагромаджуватись
у багатьох пунктах, отже, гроші доводиться вилучати з циркуляції
почасти для того, щоб уможливити утворення нового грошового
капіталу в самому І, почасти для того, щоб вартість поступінно
зужиткованого основного капіталу деякий час затримувати у грошовій
формі. А що за нашою схемою всі гроші та всі товари з самого
початку перебувають виключно в руках капіталістів І і II і що
при цьому не існує ні купців, ні торговців грішми, ні банкірів, ні таких
кляс, що лише споживають і не беруть безпосередньої участи в продукції
товарів, — то тут для того, щоб підтримувати в русі механізм репродукції,
також конче потрібне постійне утворення товарових запасів
у руках самих відповідних продуцентів їх. Отже, 500 II с, що лежать
на складах у капіталістів II, репрезентують товаровий запас засобів споживання,
що забезпечує безперервність процесу споживання, включеного
в репродукцію, і значить, в даному разі забезпечує перехід від одного
року до наступного. Фонд споживання, який при цьому є ще в руках
\parbreak{}  %% абзац продовжується на наступній сторінці
