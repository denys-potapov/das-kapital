\parcont{}  %% абзац починається на попередній сторінці
\index{ii}{0399}  %% посилання на сторінку оригінального видання
його змінний капітал зростає з 750 до 800. Це збільшення сталого, а
також і змінного капіталу II, загалом на 150, покривається з його додаткової
вартости; отже, з 750 II $m$ тільки $600 m$ лишаються як споживний
фонд капіталістів II, що їхній річний продукт розподіляється тепер
так:

II. $1600 с \dplus{} 800 v \dplus{} 600m$ (споживний фонд) \deq{} 3000.    $150 m$,спродуковані
як засоби споживання й обмінені тут на ($100 с \dplus{} 50 v$) II,
в своїй натуральній формі цілком ідуть на споживання робітників: 100
споживають робітники І (100 I $v$), а 50 — робітники II (50 II $v$), як це
показано вище. В дійсності в II, де ввесь продукт його виготовляється
в формі, потрібній для акумуляції, більша на 150 частина додаткової
вартости мусить бути репродукована в формі \so{доконечних} засобів
споживання. Коли дійсно починається репродукція в поширеному
маштабі, то 100 змінного грошового капіталу І через руки робітничої
кляси І повертаються до II; навпаки, II передає $100 m$, як товаровий
запас, підрозділові І і разом з тим передає 50, як товаровий запас своїм
власним робітникам.

Розміщення, змінене з метою акумуляції, тепер є таке:
\begin{table}[h]
  \begin{center}
  \begin{tabular}{r@{ } l@{ } r@{ } r@{ } l}
    I. & $4400 с \dplus{} 1100 v \dplus{} 500 $& фонд споживання \deq{} & 6000 & \\

    II. & $1600 с \dplus{} \phantom{0}800 v \dplus{} 600 $& фонд споживання \deq{} & 3000 & \\
    \cmidrule(r){3-5}
        &                                            & Сума      & 9000 &, як вище.
  \end{tabular}
  \end{center}
\end{table}

З цього маємо капіталу:
\begin{center}
$
 \left.\begin{aligned}
        \text{ I. }4400 с \dplus{} 1100 v \text{ (грішми)} \deq{} 5500\\
        \text{II. }1600 с \dplus{} \phantom{0}800 v \text{ (грішми)} \deq{} 2400
       \end{aligned}
 \right\}
 \text{= 7900,}
$
\end{center}

тимчасом, як на початку продукції було:

\begin{center}
$
 \left.\begin{aligned}
        \text{ I. }4000 с \dplus{} 1000 v \deq{} 5000\\
        \text{II. }1500 с \dplus{} \phantom{0}750 v \deq{} 2250
       \end{aligned}
 \right\}
 \text{= 7250.}
$
\end{center}

Коли справжня акумуляція відбувається тепер на цій основі, тобто,
коли продукцію дійсно провадять з таким збільшеним капіталом, то
наприкінці другого року матимемо:

\begin{center}
$
 \left.\begin{aligned}
        \text{ I. }4400 с \dplus{} 1100 v \dplus{} 1100 m= 6600\\
        \text{II. }1600 с \dplus{} \phantom{0}800 v \dplus{} \phantom{0}800 v \deq{} 3200
       \end{aligned}
 \right\}
 \text{= 9800.}
$
\end{center}

Акумуляція в І триватиме далі в тій самій пропорції; отже, $550 m$
витрачатиметься як дохід, а $550 m$ акумулюватиметься. В такому разі насамперед
1100 I $v$ заміститься черга 1100 ІІ $с$; далі ще треба реалізувати
550 I $m$ в товарах II рівної вартости; отже, загалом 1650 І
($v \dplus{} m$). Але сталий капітал II, що його треба замістити, становить
тільки 1600, отже, решту 50 доводиться поповнити з 800 II $m$. Коли ми
\parbreak{}  %% абзац продовжується на наступній сторінці
