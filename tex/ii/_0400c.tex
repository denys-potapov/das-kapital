\parcont{}  %% абзац починається на попередній сторінці
\index{ii}{0400}  %% посилання на сторінку оригінального видання
тут покищо лишимо осторонь гроші, то як результат цієї оборудки
матимемо:

I.    $4400 с \dplus{} 50 m$ (що їх треба капіталізувати); крім того, в споживному
фонді капіталістів і робітників 1650 ($v \dplus{} m$) реалізовані в
товарах II~$с$.

II.    $1650 с$ (а саме 50, як сказано вище, додано з II~$m$) $\dplus{} 800 v \dplus{}
750 m$ (споживний фонд капіталістів).

Але коли в II між $v$ і $с$ зберігається попередня пропорція, то на $50 с$
доведеться витратити ще $25 v$; їх можна взяти з $750 m$; отже, маємо:
\[
\text{II. }1650 с \dplus{} 825 v \dplus{} 725 m
\]
В І треба капіталізувати $550 m$; коли зберігається попередня пропорція,
то 440 з них становлять сталий капітал, а 110 — змінний капітал.
Ці 110 можна взяти з 725 II~$m$, тобто засоби споживання вартістю в
110 споживуть робітники І, а не капіталісти II; отже, останні примушені
капіталізувати ці $110 m$, що їх вони не можуть спожити. Таким чином,
з 725 II~$m$ лишається 615 II~$m$. Але коли II таким чином перетворює
ці 110 на додатковий сталий капітал, то йому потрібно ще 55 додаткового
змінного капіталу; їх він мусить взяти знову таки з своєї додаткової
вартости; коли їх відлічити з 615 II~$m$, то для споживання капіталістів II
лишиться 560; зробивши всі ці справжні й потенціяльні переміщення,
матимемо таку капітальну вартість:
\[
 \left.\begin{array}{@{}r@{~}c@{~\dplus{}~}c@{~\deq{}~}r@{~}r@{~}l@{}}
        \text{I.}&
            (4400 с \dplus{} 440 с)
            &(1100 v \dplus{} 110 v)
            &4840 c \dplus{} & 1210 v & \deq{} 6050\\
        \text{II.}&
            (1600 с \dplus{} 50 с \dplus{} 110 с)
            &(800 v \dplus{} 25 v \dplus{} 55 v)
            &1760 c \dplus{} & 880 v & \deq{} 2640\\
       \end{array}
 \right\}
 \text{\deq{} 8690.}
\]
Для того, щоб справа йшла нормально, акумуляція II мусить відбуватись
швидше, ніж у І, бо інакше частина І ($v \dplus{} m$), яку треба обміняти
на товари II~$с$, зростала б швидше, ніж II~$с$, що на нього тільки й можна
її обміняти.

Коли на цій основі та в інших незмінних умовах репродукцію провадитиметься
й далі, то наприкінці наступного року матимемо:
\[
 \left.\begin{array}{r@{~}r@{~}r@{~}r@{~}l}
        \text{I. }&4840 с \dplus{}& 1210 v \dplus{}& 1210 m & \deq{} 7260\\
        \text{II. }&1760 с \dplus{}& 880 v \dplus{}& 880 m & \deq{} 3520
       \end{array}
 \right\}
 \text{\deq{} \num{10780}.}
\]
Насамперед при незмінній нормі розподілу додаткової вартости в І
треба витратити як дохід: $1210 v$ й половину $m$ \deq{} 605, разом 1815.
Цей фонд споживання знову на 55 більший, ніж II~$с$. Ці 55 слід відлічити
з $880 m$, лишається 825. Якщо 55 II~$m$ перетворюються на ІІ~$с$, то це
має собі за передумову дальше відлічення з II~$m$ для відповідного змінного
капіталу \deq{} 27\sfrac{1}{2}; лишається для споживання 797\sfrac{1}{2} II~$m$.

Тепер треба в І капіталізувати $605 m$; з них 484 сталого капіталу й
121 змінного; ці останні треба відлічити з II~$m$, що тепер дорівнює
\parbreak{}  %% абзац продовжується на наступній сторінці
