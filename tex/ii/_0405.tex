
\index{ii}{0405}  %% посилання на сторінку оригінального видання
Повертаючись до останнього, вищерозглянутого випадку, ми бачимо
в ньому ту особливість, що ІІ с менше, ніж І ($v \dplus{} \sfrac{1}{2} m$) ніж частина
продукту І, витрачувана як дохід на засоби споживання, так що через
обмін 1500 І ($v \dplus{} m$) реалізується й частина додаткового продукту II \deq{} 70.
Щодо ІІ с \deq{} 1430, то він, щоб можлива була проста репродукція в II,
за інших незмінних умов мусить бути заміщений на таку саму вартість
з І ($v \dplus{} m$), і тому його тут нічого далі розглядати. Інакше стоїть
справа з додатковими 70 II m. Те, що для І є просте заміщення доходу
засобами споживання, просто товаровий обмін для споживання, це для II
є тут не просто зворотне перетворення його сталого капіталу з форми
товарового капіталу на його натуральну форму, як при простій репродукції,
а безпосередній процес акумуляції, перетворення частини його
додаткового продукту з форми засобів споживання на форму сталого
капіталу. Коли І на 70\pound{ ф. стерл.} грішми (грошовий резерв для обміну
додаткової вартости) купує ці 70 II m, а II не купує на них 70 І m, але
нагромаджує ці 70\pound{ ф. стерл.} як грошовий капітал, то останній, правда,
завжди є вираз додаткового продукту (а саме додаткового продукту II,
що частину його він являє), однак, не такого, який знову входить у
продукцію; але в такому разі ця акумуляція грошей на боці II була б
разом з тим виразом того, що 70 І m в засобах продукції не сила продати.
Отже, в І сталась би відносна перепродукція, відповідно до одночасної
неможливости поширити репродукцію на боці II.

Але незалежно від цього: протягом того часу, поки ці 70 грішми,
які вийшли з рук І, ще не повернулись до нього або повернулись лише
почасти через купівлю 70 І m з боку II, 70 грішми всі цілком або почасти
фігурують в руках II як додатковий віртуальний грошовий капітал.
Це має силу щодо кожного обміну між І і II, поки взаємне заміщення
товарів з обох боків зумовить зворотний приплив грошей до їхнього
вихідного пункту. Але при нормальному перебігу справ гроші
фігурують тут в цій ролі лише тимчасово. При системі кредиту, коли
всі гроші, додатково звільнені хоч на самий короткий час, одразу ж повинні
функціонувати активно як додатковий грошовий капітал, такий лише
тимчасово вільний грошовий капітал можна закріпити, напр., він може
служити для нових підприємств в І, тимчасом як він мусів би пустити в
рух залежалий додатковий продукт в інших підприємствах. Далі треба
відзначити, що долучення 70 І m до сталого капіталу II потребує разом
з тим збільшення змінного капіталу II на суму в 14. Це має за передумову
— подібно до того, як в І при безпосередньому долученні додаткового
продукту І m до капіталу І с, — те, що репродукція в II уже відбувається
з тенденцією до дальшої капіталізації; отже, що вона включає збільшення
тієї частини додаткового продукту, яка складається з доконечних засобів
існування.

Як ми бачили, в другому прикладі продукт в 9000, коли 500 І m
треба капіталізувати, мусить розподілитись для завдань репродукції так.
При цьому ми беремо на увагу тільки товари й лишаємо осторонь грошову
циркуляцію.
