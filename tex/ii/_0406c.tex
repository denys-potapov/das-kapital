
\index{ii}{0406}  %% посилання на сторінку оригінального видання
I. $5000 с \dplus{} 500 m \text{ (що їх треба капіталізувати)} \dplus{} 1500 (v \dplus{} m)$ споживного
фонду \deq{} 7000 в товарах.

II. $1500 с \dplus{} 299 v \dplus{} 201 m \deq{} 2000$ в товарах. Загальна сума 9000 в
товаровому продукті.

Капіталізація тепер відбувається так:

В І підрозділі $500 m$, що їх капіталізується, поділяються на $\sfrac{5}{6} \deq{}
417 с \dplus{} \sfrac{1}{6} \deq{} 83 v$\footnote*{
\label{note-406}
Обчислено з закругленням дробів. \emph{Ред.}
}. Ці $83 v$ вилучають таку саму суму з $\text{II} m$, і на
неї купується елементи сталого капіталу, отже, їх долучається до $\text{II} с$.
Збільшення $\text{II} с$ на 83 зумовлює збільшення 
$\text{II} v$ на \sfrac{1}{5} від $83 \deq{} 17$\footref{note-406}.
Отже, після обміну ми маємо:
\[
 \left.\begin{array}{r@{~}r@{~}r@{~}r@{~}r@{~}r@{~}r@{~}l}
        \text{I. } & (5000 с \dplus{} 
          & 417 m) с \dplus{} & (1000 v & \dplus{} 83 m) v \deq{} & 
             5417 c \dplus{} & 1083 v \deq{} & 6500 \\

        \text{II. }& (1500 с \dplus{} 
          & 83 m) c \dplus{} & (299 v & \dplus{} 17 m) v \deq{} &
            1583 с \dplus{} & 316 v \deq{} & 1899
       \end{array}
 \right\}
 \text{\deq{} \num{8399}.}
\]
Капітал в І зріс з 6000 до 6500, отже, на \sfrac{1}{12}. В II з 1715 до 1899,
отже, майже на \sfrac{1}{9}.

На другий рік репродукція на такій основі дає наприкінці року
капітал:
\[
 \begin{array}{@{}r@{~}c@{~\dplus{}~}c@{~\deq{}~}r@{~}r@{~}l@{}}
  \text{I.}&
      (5417 \dplus{} 452 m) c
      &(1083 v \dplus{} 90 m) v
      &5869 с \dplus{} & 1173 v & \deq{} 7042\\
  \text{II.}&
     (1583 с \dplus{} 42 m \dplus{} 90 m) c
      &(316 v \dplus{} 8m \dplus{} 18 m) v
      &1715 c \dplus{} & 342 v& \deq{} 2057\text{,}\\
  \end{array}
\]
а наприкінці третього року дає продукт:
\[
 \begin{array}{r@{~}r@{~}r@{~}r@{~}r}
  \text{I.}& 5869 c \dplus{}& 1173 v \dplus{}& 1173 m \\
  \text{II.}& 1715 с \dplus{}& 342 v \dplus{}& 342 m& \text{.}
  \end{array}
\]
Коли І акумулює при цьому, як і раніше, половину додаткової вартости,
то І ($v \dplus{} \sfrac{1}{2} m$) дає $1173 v \dplus{} 587 (\sfrac{1}{2} m) \deq{} 1760$, отже, більше,
ніж усі $1715 \text{ ІІ} с$, а саме більше на 45. Отже, цю ріжницю знову треба
покрити переміщенням до II $с$ засобів продукції на таку саму суму. Отже,
ІІ $с$ зростає на 45, що зумовлює приріст в II $v$ на \sfrac{1}{5} \deq{} 9. Далі капіталізовані
$587 \text{ I} m$ поділяються на \sfrac{5}{6} і \sfrac{1}{6} на $489 с$ і $98 v$; ці 98 зумовлюють
в II нову додачу 98 до сталого капіталу, а це теж зумовлює
збільшення змінного капіталу II на \sfrac{1}{5} \deq{} 20. Ми маємо тоді:
\[
 \left.\begin{array}{@{}r@{~}c@{~\dplus{}~}c@{~\deq{}~}r@{~}r@{~}l@{}}
        \text{I.}&
            (5869 с \dplus{} 489 m) c
            &(1173 v \dplus{} 98 m) v
            &6358 с \dplus{} & 1271 v& \deq{} 7629\\
        \text{II.}&
            (1715 с \dplus{} 45 m \dplus{} 98 m) c
            &(342 v \dplus{} 9 m \dplus{} 20 m) v
            &1858 c \dplus{} & 371 v & \deq{} 2229\\
       \end{array}
 \right\}
 \text{\deq{} 9858.}
\]
Отже, при репродукції, що протягом трьох років зростала,
весь капітал підрозділу І зріс з 6000 до 7629, а ввесь капітал підрозділу
II з 1715 до 2229, сукупний суспільний капітал з 7715 до 9858.
