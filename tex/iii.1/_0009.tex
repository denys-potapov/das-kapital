
\index{iii1}{0009}  %% посилання на сторінку оригінального видання
\section*{Передмова}

Нарешті мені удалось опублікувати цю третю книгу головного
твору Маркса, закінчення теоретичної частини. При виданні
другої книги в 1885 році я гадав, що третя книга становитиме
хіба тільки технічні труднощі, за винятком, звичайно, деяких
дуже важливих відділів. Так воно й було в дійсності; але про ті
труднощі, які мене чекали саме в цих найважливіших відділах
цілого, я не мав тоді ніякого уявлення, так само як і про інші
перешкоди, які так дуже загаяли виготовлення книги.

Перш за все і більш за все мені заважала постійна слабість
зору, яка протягом багатьох років обмежувала до мінімуму мій
робочий час для письмових занять, та ще й тепер тільки винятками
дозволяє мені брати в руки перо при штучному освітленні.
До цього долучилися інші невідкладні роботи: нові видання й
переклади попередніх праць Маркса та моїх, отже, перегляди,
передмови, доповнення, часто неможливі без нових досліджень,
і т. д. Передусім англійське видання першої книги, за текст
якого кінець-кінцем відповідаю я і яке через це забрало в мене
багато часу. Хто скількинебудь стежив за колосальним ростом
інтернаціональної соціалістичної літератури протягом останніх
десяти років і особливо за числом перекладів раніших праць
Маркса та моїх, той визнає, що я мав підставу вітати себе
з тим, що число тих мов, де я міг бути корисним перекладачеві
і, отже, був зобов’язаний не відмовлятись від перегляду його
праці, дуже обмежене. Але ріст літератури був тільки симптомом
відповідного зростання самого інтернаціонального робітничого
руху. А це останнє накладало на мене нові обов’язки.
З перших днів нашої громадської діяльності чимала частина праці
щодо посередництва між національними рухами соціалістів і робітників
у різних країнах падала на мене і Маркса; ця праця зростала
відповідно до зміцнення всього руху. Але тимчасом як
до самої своєї смерті Маркс і в цьому головний тягар праці
брав на себе, після його смерті постійно наростаючу працю довелося
виконувати мені одному. Між тим, безпосередні зносини
окремих національних робітничих партій між собою стали загальним
правилом, і на щастя, з дня на день стають ним дедалі більше;
не зважаючи на це, моя допомога потрібна ще далеко частіше,
ніж мені хотілося б в інтересах моїх теоретичних праць. Але
\parbreak{}  %% абзац продовжується на наступній сторінці
