
\index{iii1}{0047}  %% посилання на сторінку оригінального видання

\chapter{Перетворення додаткової вартості в зиск і норми додаткової вартості в норму зиску}

\section{Витрати виробництва (kostpreis) і зиск}

В першій книзі були досліджені ті явища, які представляє капіталістичний
\emph{процес виробництва}, взятий сам по собі, як безпосередній
процес виробництва, при чому ще залишались осторонь
усі вторинні впливи чужих йому обставин. Але цей безпосередній
процес виробництва ще не вичерпує життьового шляху капіталу.
В дійсному світі він доповнюється \emph{процесом циркуляції},
який становив предмет досліджень другої книги. Там, саме в
третьому відділі, при розгляді процесу циркуляції як опосереднення
суспільного процесу репродукції, виявилось, що капіталістичний
процес виробництва, розглядуваний у цілому, є єдність
процесу виробництва і циркуляції. Завдання цієї третьої книги
не може полягати в тому, щоб дати загальні міркування про цю
єдність. Навпаки, тут треба знайти і описати ті конкретні форми,
які виростають з \emph{процесу руху капіталу}, \emph{розглядуваного як
ціле}. В своєму дійсному русі капітали протистоять один одному
в таких конкретних формах, для яких форма капіталу в безпосередньому
процесі виробництва, як і його форма в процесі циркуляції,
виступають тільки як особливі моменти. Отже, ті форми
капіталу, які ми описуємо в цій книзі, крок за кроком наближаються
до тієї форми, в якій вони виступають на поверхні
суспільства, в діянні різних капіталів один на одного, в конкуренції
і в звичайній свідомості самих діячів виробництва.

\pfbreak

Вартість кожного капіталістично виробленого товару $Т$ зображується
у формулі: $Т = c + v + m$. Якщо ми від цієї вартості
\parbreak{}  %% абзац продовжується на наступній сторінці
