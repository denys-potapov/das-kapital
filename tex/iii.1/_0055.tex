
\index{iii1}{0055}  %% посилання на сторінку оригінального видання
Тепер капіталістові ясно, що цей приріст вартості виникає
з продуктивних процесів, пророблених з капіталом, що він,
отже, виникає з самого капіталу; бо після процесу виробництва
він є, а перед процесом виробництва його не було. Насамперед,
щодо капіталу, витраченого на виробництво, то здається,
що додаткова вартість виникає рівномірно з різних елементів
його вартості, які існують у вигляді засобів виробництва і праці.
Адже ці елементи рівномірно входять в утворення витрат виробництва.
Вони рівномірно додають до вартості продукту свої
вартості, що наявні як авансування капіталу, і не відрізняються
один від одного як стала і змінна величини вартості. Це стає
цілком очевидним, коли ми на один момент припустимо, що
весь витрачений капітал складається або виключно з заробітної
плати, або виключно з вартості засобів виробництва. Ми мали б
тоді в першому випадку замість товарної вартості 400 с + 100 v +
100 m товарну вартість 500 v + 100 m. Витрачений на заробітну
плату капітал у 500 фунтів стерлінгів є вартість усієї праці,
вжитої на виробництво товарної вартості в 600 фунтів стерлінгів,
J. саме тому становить витрати виробництва всього продукту. Але
утворення цих витрат виробництва, в наслідок чого вартість витраченого
капіталу знову з’являється як складова частина вартості
продукту, є єдиний відомий нам процес в утворенні цієї товарної
вартості. Як виникає та її складова частина у 100 фунтів стерлінгів,
яка становить собою додаткову вартість, ми не знаємо.
Цілком те саме було б у другому випадку, де товарна вартість
була б = 500 с + 100 m. В обох випадках ми знаємо, що додаткова
вартість виникає з даної вартості тому, що ця вартість авансована
в формі продуктивного капіталу, однаково, чи то в формі праці,
чи в формі засобів виробництва. Але, з другого боку, авансована
капітальна вартість з тієї тільки причини, що вона витрачена
і тому становить витрати виробництва товару, не може
утворити додаткової вартості. Бо саме остільки, оскільки вона
становить витрати виробництва товару, вона утворює не додаткову
вартість, а тільки еквівалент, вартість, яка заміщає витрачений
капітал. Отже, оскільки вона утворює додаткову вартість,
вона утворює її не в наслідок своєї специфічної властивості як
витрачений капітал, а як авансований і тому застосований капітал
взагалі. Тому додаткова вартість в однаковій мірі виникає як
з тієї частини авансованого капіталу, що входить у витрати
виробництва товару, так і з тієї частини його, що не входить
у витрати виробництва; одним словом — в однаковій мірі з основних
і обігових складових частин застосованого капіталу. Весь
капітал речево служить як продуктотворець — засоби праці так
само, як і матеріали виробництва та праця. Весь капітал речево
входить у дійсний процес праці, хоч тільки частина його входить
у процес зростання вартості. Може, саме це і є причиною
того, що він тільки частиною бере участь в утворенні витрат виробництва,
але цілком — в утворенні додаткової вартості. Як би там
\parbreak{}  %% абзац продовжується на наступній сторінці
