
\index{iii1}{0075}  %% посилання на сторінку оригінального видання
Якщо $v$ з 30 впаде до 20 через те, що при зростанні сталого
капіталу вживатиметься на \sfrac{1}{3} менше робітників, то ми матимем
тут перед собою нормальний випадок сучасної промисловості:
ростущу продуктивність праці, опанування меншим числом робітників
більших мас засобів виробництва. Що такий рух необхідно
зв'язаний із зменшенням норми зиску, яке настає одночасно
з цим, — це виявиться в третьому відділі цієї книги.
       Але якщо $v$ з 30 знизиться до 20 через те, що вживається
те саме число робітників, але за нижчу заробітну плату, то при
незмінному робочому дні вся нововироблена вартість лишилася
б, як і раніш, $= 30 v + 15 m = 45$; через те що $v$ впало до 20,
додаткова вартість підвищилася б до 25, норма додаткової
вартості підвищилася б від 50\%  до 125\%, що суперечило б
припущенню. Щоб було додержано умов нашого випадку, додаткова
вартість, при нормі в 50\%, мусить, навпаки, впасти до
10, отже, вся нововироблена вартість — з 45 до 30, а це можливе
тільки при скороченні робочого дня на \sfrac{1}{3}. Тоді ми маємо,
як і вище:\[
    100 c + 20 v + 10 m; m' = 50\%, p' = 8\sfrac{1}{3}\%\]
      Звичайно, немає потреби згадувати, що такого скорочення
робочого часу при зменшенні заробітної плати на практиці не
було б. Однак, це не має значення. Норма зиску є функція
кількох змінних, і якщо ми хочемо знати, як ці змінні впливають
на норму зиску, то мусимо по черзі дослідити відокремлений
вплив кожної з них, однаково, чи такий ізольований вплив
економічно можливий для одного й того самого капіталу, чи ні.
\begin{center}
    \textbf{2. $m'$ не  змінюється, $v$ змінюється, $К$ змінюється в наслідок зміни $v$}
\end{center}
Цей випадок відрізняється від попереднього тільки щодо
ступеня. Замість того, щоб $c$ зменшувалось або збільшувалось
настільки, наскільки $v$ збільшується або зменшується, $c$ лишається
тут незмінним. Але при сучасних умовах великої промисловості
і землеробства змінний капітал є тільки відносно
незначна частина всього капіталу, а тому зменшення або зростання
цього останнього, оскільки воно визначається зміною
першого, так само відносно незначне. Якщо ми знову виходили
б з капіталу:

$\text{I. } 100 c + 20 v + 10 m; К = 120, m' = 50\%, p' = 8\sfrac{1}{3}\%$,

то він перетворився б, наприклад, у:

$\text{II. } 100 c + 30 v + 15 m; К = 130, m' = 50\%, p' = 11\sfrac{7}{13}\%$.

Протилежний випадок зменшення змінного капіталу знов
таки можна зробити наочним за допомогою зворотного переходу
від II до І.

Економічні умови в істотному були б такі самі, як і в попередньому
випадку, і тому вони не потребують повторного по-
\parbreak{}  %% абзац продовжується на наступній сторінці
