
\index{iii1}{0081}  %% посилання на сторінку оригінального видання
Наприклад:

80 с + 20 v + 20 m; m' = 100\%, p' = 20\%
80 с + 20 v + 10 m; m' = 50\%, p' = 10\%
20\% : 10\% = 100 × 20 : 50 × 20 = 20 m : 10 m.

Тепер ясно, що при капіталах однакового абсолютного чи
процентного складу норми додаткової вартості можуть бути
різні тільки в тому випадку, коли різні або заробітна плата,
або довжина робочого дня, або інтенсивність праці. В трьох
випадках:

I.  80 с + 20 v + 10 m; m' = 50\%, p' = 10\%,
II. 80 с + 20 v + 20 m; m' = 100\%, p' = 20\%,
III. 80 с + 20 v + 40 m; m' = 200\%, p' = 40\%,

вся нововироблена вартість буде в І 30 (20 v + 10 m), в II — 40,
в III — 60. Це може статись трояким способом.

Поперше, якщо заробітні плати різні, отже, якщо 20 v в кожному
окремому випадку виражає різне число робітників. Припустім,
що в І занято 15 робітників 10 годин при заробітній
платі в 1  1/3  фунтів стерлінгів і що вони виробляють вартість
у 30 фунтів стерлінгів, з яких 20 фунтів стерлінгів заміщають
заробітну плату, а 10 фунтів стерлінгів лишаються для додаткової
вартості. Якщо заробітна плата падає до 1 фунта стерлінгів,
то можуть бути заняті 20 робітників 10 годин; тоді вони
виробляють вартість у 40 фунтів стерлінгів, з яких 20 фунтів
стерлінгів для заробітної плати і 20 фунтів стерлінгів додаткової
вартості. Якщо заробітна плата падає ще далі, до 2/3 фунтів
стерлінгів, то можуть бути заняті 30 робітників по 10 годин,
які виробляють вартість у 60 фунтів стерлінгів, що з них після
відрахування 20 фунтів стерлінгів для заробітної плати залишиться
ще 40 фунтів стерлінгів для додаткової вартості.

Цей випадок: незмінний процентний склад капіталу, незмінний
робочий день, незмінна інтенсивність праці, зміна норми
додаткової вартості, спричинена зміною заробітної плати — є
єдиний випадок, на якому справджується положення Рікардо:
„profits would be high or low, exactly in proportion as wages
would be, low or high“ [„зиск буде високий чи низький точно
в такій пропорції, в якій заробітна плата буде низька чи висока“]
(„Principles of Political Economy“, розд. І, відділ III, стор. 18.
„Works of D. Ricardo“, вид. Mac Culloch, 1852).

Або, подруге, якщо інтенсивність праці різна. Тоді, наприклад,
20 робітників при однакових засобах праці за 10 робочих
годин на день виробляють у І — 30, у II — 40, у III — 60 штук
певного товару, кожна штука якого, крім вартості спожитих
на неї засобів виробництва, представляє нову вартість в 1 фунт
стерлінгів. Через те що в кожному випадку 20 штук, = 20
фунтам стерлінгів, заміщають заробітну плату, то для додаткової
\index{iii1}{0082}  %% посилання на сторінку оригінального видання
вартості лишаються в І — 10 штук = 10 фунтам стерлінгів,
в II — 20 штук = 20 фунтам стерлінгів, в III — 40 штук = 40 фунтам
стерлінгів.

Або, потретє, робочий день — різної довжини. Якщо 20 робітників
при однаковій інтенсивності працюють у І — дев’ять,
у II — дванадцять, у III — вісімнадцять годин на день, то весь їх
продукт 30 : 40 : 60 відноситься як 9 : 12 : 18, і тому що заробітна
плата в кожному випадку = 20, то знову лишається 10, відповідно
20 і 40 для додаткової вартості.

Отже, підвищення або зниження заробітної плати діє в зворотному
напрямі, підвищення або зниження інтенсивності праці
і здовження або скорочення робочого дня діє в тому самому
напрямі на висоту норми додаткової вартості, а тому, при незмінному
v/K, і на норму зиску.

2. m' і v змінюються, К не змінюється

В цьому випадку має силу пропорція:

p': p'1 = m' v/K : m'1 v1/K = m'v : m'1v1 = m : m1.

Норми зиску відносяться одна до одної, як відповідні маси
додаткової вартості.

Зміна норми додаткової вартості при незмінній величині змінного
капіталу означала зміну у величині й розподілі нововиробленої
вартості. Одночасна зміна v і m' так само завжди включає
інший розподіл, але не завжди зміну величини нововиробленої
вартості. Можливі три випадки:

a) Зміни v і m' відбуваються в протилежному напрямі, але
на однакову величину; наприклад:

80 с + 20 v + 10 m; m' = 50\%, p' = 10\%
90 с + 10 v + 20 m; m' = 200\%, p' = 20\%.

Нововироблена вартість в обох випадках однакова, отже, однакова
й кількість витраченої праці; 20 v + 10 m = 10 v + 20 m = 30.
Ріжниця тільки в тому, що в першому випадку 20 сплачується
як заробітна плата, а 10 лишається для додаткової вартості,
тимчасом як у другому випадку заробітна плата становить
тільки 10, а тому додаткова вартість — 20. Це єдиний випадок,
коли при одночасній зміні v і m' число робітників, інтенсивність
праці і довжина робочого дня лишаються незміненими.

b) Зміни m' і v відбуваються так само в протилежному напрямі,
але не на ту саму величину. Тоді перевага буде або на
стороні зміни v, або на стороні зміни m'.

I. 80 с + 20 v + 20 m; m' = 100\%, p' = 20\%
II.72 с + 28 v + 20 m; m' = 71 3/7\%, p' = 20\%
III. 84 с + 16 v + 20 m; m' = 125\%, p' = 20\%.
