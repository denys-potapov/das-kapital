
\index{iii1}{0133}  %% посилання на сторінку оригінального видання
„Стан справ покращав; але цикл сприятливих і несприятливих
періодів скорочується із збільшенням кількості машин, і
коли при цьому збільшується попит на сировинний матеріал,
то частіше повторюються також і коливання в стані справ...
В даний момент не тільки відновилось довір’я після паніки
1857 року, але й сама паніка, здається, майже цілком забута.
Чи це покращання буде тривалим, чи ні, це в дуже значній
мірі залежить від ціни сировинних матеріалів. Я бачу вже ознаки
того, що в деяких випадках уже досягнуто максимуму, після
якого фабрикація ставатиме все менш зисковною, поки, нарешті,
вона й зовсім перестане давати зиск. Якщо ми візьмемо, наприклад,
зисковні роки у підприємствах чесаної вовни, 1849 і 1850,
то ми побачимо, що ціна англійської чесаної вовни стояла на
рівні 13 пенсів, австралійської — від 14 до 17 пенсів за фунт,
і що на протязі десяти років, з 1841 до 1850, пересічна ціна
англійської вовни ніколи не перевищувала 14 пенсів, а австралійської
— 17 пенсів за фунт. Але на початку нещасливого
1857 року австралійська вовна стояла на рівні 23 пенсів; у грудні,
в найлихіший час паніки, вона впала до 18 пенсів, але на протязі
1858 року знову підвищилась до теперішньої ціни в 21 пенс.
Англійська вовна в 1857 році так само почала з 20 пенсів, у квітні
й вересні вона підвищилась до 21 пенса, в січні 1858 року впала
до 14 пенсів, а потім підвищилась до 17 пенсів, так що тепер
ціна її за фунт на 3 пенси вища, ніж пересічна ціна протягом
наведених 10 років... Це свідчить, на мою думку, про те, що або
банкрутства 1857 року, викликані подібними цінами, забуті, або
що вовни виробляється ледве-ледве стільки, скільки можуть
перепрясти наявні веретена, абож має місце тривале підвищення
ціни тканин... Але в моїй дотеперішній практиці я бачив,
як на протязі неймовірно короткого часу не тільки збільшилась
кількість веретен і ткацьких верстатів, але й швидкість
їх роботи; далі, що майже в тій самій мірі підвищився наш вивіз
вовни до Франції, тимчасом як пересічний вік овець, вирощуваних
як усередині країни, так і за кордоном, стає все нижчим,
бо населення швидко збільшується, і вівчарі бажають якомога
швидше перетворити своїх овець у гроші. Тому я часто переживав
тяжке почуття, коли бачив людей, які, не знаючи цього, вкладали
свою долю і свій капітал у підприємства, успіх яких залежить від
подання такого продукту, який може збільшуватись тільки
згідно з певними органічними законами... Стан попиту й подання
всіх сировинних матеріалів... пояснює, як видно, багато коливань
у бавовняній промисловості, а також стан англійського вовняного
ринку восени 1857 року і викликану ним промислову кризу“\footnote{Само собою зрозуміло, що ми не \emph{пояснюємо}, разом з паном Бекером,
вовняну кризу 1857 року невідповідністю між цінами сировинного матеріалу
і фабрикату. Ця невідповідність сама була тільки симптомом, а криза була
загальною. — \emph{Ф. Е.}}
(\emph{R. Baker} в „Rep. of. Insp. of Fact., Oct. 1858“, стор. 56—61).
