
\index{iii1}{0134}  %% посилання на сторінку оригінального видання
Періодом розквіту промисловості чесаної вовни Вест-Рідінга
у Йоркшірі були роки 1849—1850. В 1838 році там було занято
29 246 осіб, в 1843 році — 37 060, в 1845 році — 48 097, в 1850 році —
74 891. В тій самій окрузі було: в 1838 році — 2768 механічних
ткацьких верстатів, в 1841 році — 11 458, в 1843 році — 16 870,
в 1845 році — 19 121 і в 1850 році — 29 539 („Rep. of Insp. of Fact.,
[31 Oct.], 1850“, стор. 60). Цей розквіт промисловості чесаної
вовни уже в жовтні 1850 року почав ставати підозрілим. У квітневому
звіті за 1851 рік субінспектор Бекер каже про Лідсі Бредфорд:
„Стан справ з деякого часу дуже незадовільний. Прядільники
чесаної вовни швидко втрачають зиски 1850 року, і у більшості
ткачів справи йдуть також не дуже добре. Я гадаю, що
в даний момент у шерстяній промисловості стоїть без діла більше
машин, ніж будьколи раніш, і прядільники льону так само звільняють
робітників і спиняють машини. Цикли в текстильній промисловості
нині дійсно дуже непевні, і я гадаю, що незабаром
ми прийдемо до погляду... що не додержується ніякої відповідності
між виробничою спроможністю веретен, кількістю сировинного
матеріалу і збільшенням населення“ (стор. 52).

Те саме стосується до бавовняної промисловості. В щойно
цитованому жовтневому звіті за 1858 рік сказано: „З того часу,
як на фабриках установлені певні години праці, кількість споживаного
сировинного матеріалу, розміри виробництва і заробітних
плат в усіх галузях текстильної промисловості зведені
до простого трійного правила... Я цитую з недавньої доповіді...
пана Бейнса, теперішнього мера Блекберна, про бавовняну промисловість,
в якій він з найбільшою можливою точністю дає
зведення даних промислової статистики своєї округи:

„Кожна дійсна кінська сила приводить в рух 450 автоматичних
веретен з підготовчими прядільними машинами, або 200 веретен-throstle
[тонкопрядільних], або 15 верстатів для 40-дюймового
сукна разом з машинами для намотування, снування й
шліхтування. Кожна кінська сила потребує при прядінні 2 \sfrac{1}{2} робітників,
а при тканні — 10; їх пересічна заробітна плата становить
понад 10 \sfrac{1}{2}  шилінгів на кожного за тиждень... Пересічні
перероблювані нумери для основи 30—32 і для утка 34—36;
якщо ми припустимо, що вироблювана за тиждень пряжа становить
13 унцій на одно веретено, то це дасть 824 700 фунтів
пряжі на тиждень, для чого треба спожити 970 000 фунтів, або
2300 пак бавовни, ціною в 28 300 фунтів стерлінгів... В нашій
окрузі (навколо Блекберна радіусом у 5 англійських миль)
тижневе споживання бавовни становить 1 530 000 фунтів, або
3650 пак, ціною в 44 625 фунтів стерлінгів. Це становить \sfrac{1}{18}
всієї бавовнопрядільної промисловості Сполученого Королівства
і \sfrac{1}{16}  всього механічного ткацтва“.

„Отже, за обчисленнями пана Бейнса, загальне число бавовнопрядільних
веретен у Королівстві повинно було б бути 28 800 000,
і для того, щоб усі вони працювали на повний хід, потрібно
\parbreak{}  %% абзац продовжується на наступній сторінці
