\parcont{}  %% абзац починається на попередній сторінці
\index{iii1}{0148}  %% посилання на сторінку оригінального видання
сприятливого вибору ринку, може бути дуже різна залежно від
більшої чи меншої дешевини сировинного матеріалу, закупівлі
його з більшим чи меншим знанням справи; залежно від того,
наскільки застосовувані машини є продуктивні, доцільні й дешеві;
залежно від більшої чи меншої досконалості загальної
організації різних ступенів процесу виробництва, від того, наскільки
усунено марнування матеріалу, наскільки просто й доцільно
організовано управління й нагляд і т. п. Коротко кажучи,
якщо додаткова вартість для певного змінного капіталу є дана,
то та сама додаткова вартість може виражатися в більшій чи
меншій нормі зиску, отже, може давати більшу чи меншу масу
зиску залежно від особистої ділової спритності самого капіталіста
або його наглядачів і прикажчиків. Припустім, що та сама додаткова
вартість в 1000 фунтів стерлінгів, продукт 1000 фунтів
стерлінгів заробітної плати, в підприємстві $A$ припадає на
9000 фунтів стерлінгів, а в іншому підприємстві $В$ — на 11000
фунтів стерлінгів сталого капіталу. У випадку $А$ ми маємо
$р' = \frac{1000}{10000} = 10\%$. У випадку $В$ ми маємо $р' = \frac{1000}{12000} = 8\sfrac{1}{3}\%$.
Весь капітал виробляє в $А$ порівняно більше зиску, ніж у $В$, бо
там норма зиску вища, ніж тут, хоч в обох випадках авансований
змінний капітал = 1000 і здобута з нього додаткова
вартість також = 1000, отже в обох випадках має місце однакова
експлуатація однакового числа робітників. Ця ріжниця
виразу однієї і тієї ж маси додаткової вартості, або ріжниця
норм зиску, а тому й самих зисків, при однаковій експлуатації
праці, може походити і з інших джерел; але вона може також
походити цілком і виключно з ріжниці в діловій вправності, з
якою ведуться обидва підприємства. І ця обставина приводить
капіталіста до ілюзії — переконує його, — що його зиск завдячує
своє існування не експлуатації праці, а, принаймні почасти і
іншим, від цієї експлуатації праці незалежним, обставинам, особливо
його індивідуальній діяльності.

\pfbreak

З викладеного в цьому першому відділі видно помилковість
того погляду (Родбертуса), згідно з яким (відмінно від
земельної ренти, де, наприклад, площа землі лишається незмінною,
в той час як рента зростає) зміна величини капіталу не впливає
на відношення між зиском і капіталом, а тому й на норму
зиску, бо тоді, коли зростає маса зиску, зростає і маса капіталу,
на яку цей зиск обчислюється, і навпаки.

Це правильно тільки в двох випадках. Поперше, тоді, коли
при незмінності всіх інших умов, отже, особливо норми додаткової
вартості, настає зміна вартості товару, який є грошовим
товаром. (Те саме має місце при самій тільки номінальній
зміні вартості, підвищенні чи падінні знаків вартості при інших
однакових умовах). Припустім, що весь капітал = 100 фунтам
стерлінгів, зиск = 20 фунтам стерлінгів, отже, норма зиску = 20\%.
\parbreak{}  %% абзац продовжується на наступній сторінці
