\parcont{}  %% абзац починається на попередній сторінці
\index{iii1}{0163}  %% посилання на сторінку оригінального видання
тільки окремі частини (як, наприклад, на якійсь бавовняній фабриці, в різних відділах якої —
кардувальному, підготовчому, прядільному й ткацькому — існує різне відношення між змінним і
сталим капіталом і де пересічне відношення для всієї фабрики
ще тільки має бути обчислене), то, по-перше, пересічний склад
капіталу в 500 був би $= 390 c + 110 v$, або в процентах $78 c + 22 v$.
Кожний з цих капіталів в 100, розглядуваний тільки як \sfrac{1}{5} сукупного капіталу, мав би своїм складом цей
пересічний склад
в $78 c + 22 v$; так само на кожні 100 припадало б 22 як пересічна
додаткова вартість; тому пересічна норма зиску була б $=$ 22\%,
і, нарешті, ціна кожної п’ятої частини сукупного продукту, виробленого цими 500, дорівнювала б 122.
Отже, продукт кожної
п’ятої частини сукупного авансованого капіталу мусив би продаватись за 122.

Однак, щоб не прийти до цілком хибних висновків, не слід
усі витрати виробництва рахувати рівними 100.

При $80 c + 20 v$ і нормі додаткової вартості = 100\% вся вартість товару, виробленого капіталом І =
100, була б $= 80 c + 20 v + 20 m = 120$, коли б весь сталий капітал входив у річний продукт. При
певних обставинах це, звичайно, може мати місце
в певних сферах виробництва. Однак, ледве чи це можливе там,
де відношення $c : v = 4 : 1$. Отже, при дослідженні вартостей товарів, вироблюваних кожними 100
одиницями різних капіталів,
треба взяти до уваги те, що ці вартості можуть бути різні, залежно від різного складу $c$ з основних і
обігових складових
частин, і що основні складові частини різних капіталів, в свою
чергу, зношуються швидше або повільніше, отже, за однакові
періоди часу додають до продукту неоднакові кількості вартості. Але для норми зиску це не має
значення. Чи $80 c$ віддають
річному продуктові вартість в 80, чи в 50, чи в 5, отже, чи річний
продукт $= 80 c + 20 v + 20 m = 120$, чи $= 50 c + 20 v + 20 m = 90$, чи $= 5 c + 20 v + 20 m = 45$, — в
усіх цих випадках надлишок
вартості продукту понад його витрати виробництва = 20, і в усіх
цих випадках при встановленні норми зиску ці 20 обчислюються
на капітал в 100; отже, норма зиску для капіталу І в усіх випадках $=$ 20\%. Щоб зробити це ще яснішим,
ми в нижченаведеній
таблиці припускаємо для тих самих п’яти капіталів, про які мова
йшла вище, що у вартість продукту з цих п’яти капіталів входять різні частини сталого капіталу.
\begin{footnotesize}
\footnotesize
\begin{tabular}{c@{ } c@{ } c@{ } c@{ } c@{ } c@{ } c@{ } c@{ } }
\toprule
\multicolumn{2}{c}{Капітали} &
\makecell{Норма\\додаткової\\вартості} &
\makecell{Додаткова\\вартість} &
\makecell{Норма\\зиску} &
\makecell{Зношування\\$c$} &
\makecell{Вартість\\товарів} &
\makecell{Витрати\\виробництва} \\
\midrule
І.        & $\phantom{0}80 c + \phantom{0}20 v$ & 100\%  &  \phantom{0}20   & 20\%           & 50 & \phantom{0}90  & 70  \\
II.       & $\phantom{0}70 c + \phantom{0}30 v$ & 100\%  &  \phantom{0}30   & 30\%           & 51 & 111 & 81  \\
III.      & $\phantom{0}60 c + \phantom{0}40 v$ & 100\%  &  \phantom{0}40   & 40\%           & 51 & 131 & 91  \\
IV.       & $\phantom{0}85 c + \phantom{0}15 v$ & 100\%  &  \phantom{0}15   & 15\%           & 40 & \phantom{0}70  & 55  \\
V.        & $\phantom{0}95 c + \phantom{00}5 v$ & 100\%  &  \phantom{00}5   & \phantom{0}5\% & 10 & \phantom{0}20  & 15  \\
Сума      & $390 c + 110 v $                    & \textemdash  &  110             &  \textemdash   & \textemdash & \textemdash & \textemdash \\
Пересічно & $\phantom{0}78 c + \phantom{0}22 v$ & \textemdash &  \phantom{0}22   &  22\%          & \textemdash & \textemdash & \textemdash \\
\end{tabular}
\end{footnotesize}
