
\index{iii1}{0203}  %% посилання на сторінку оригінального видання
Візьмімо тепер капітал, склад якого є нижчий, ніж первісний
склад пересічного суспільного капіталу $80 c + 20 v$ (який
перетворився тепер в $76\sfrac{4}{21}c + 23\sfrac{17}{21}v$), наприклад, $50 c + 50 v$.
Тут ціна виробництва річного продукту, — якщо ми для спрощення
припустимо, що весь основний капітал увійшов як зношування
в річний продукт і що час обороту такий самий, як
і в випадку I, — становила перед підвищенням заробітної плати
$50 c + 50 v + 20 p = 120$. Підвищення заробітної плати на 25\%
дає для тієї самої кількості приведеної в рух праці підвищення
змінного капіталу з 50 до 62\sfrac{1}{2}. Коли б річний продукт був
проданий по попередній ціні виробництва в 120, то це дало б
$50 c + 62\sfrac{1}{2}v + 7\sfrac{1}{2}p$, тобто норму зиску в 6\sfrac{2}{3}\%.
Але нова пересічна норма зиску є 14\sfrac{2}{7}\%, і через те що ми всі інші умови
припускаємо незмінними, цей капітал в $50 c + 62\sfrac{1}{2}v$ так само
мусить дати вказаний зиск. Але капітал в 112\sfrac{1}{2}, при нормі зиску
в 14\sfrac{2}{7}, дає 16\sfrac{1}{14} зиску.\footnote*{
В першому німецькому виданні тут сказано: „в круглих числах 16\sfrac{1}{12}
зиску“; відповідно до цього Енгельс обчислює потім ціну виробництва в 128\sfrac{7}{12}
В рукопису Маркса дано точне число в 16\sfrac{3}{42}, яке нами взяте з відповідним
скороченням дробу і застосоване при обчисленні ціни виробництва.
\emph{Примітка ред. нім. вид. ІМЕЛ.}
} Отже, ціна виробництва вироблених
ним товарів є тепер $50 c + 62\sfrac{1}{2}v + 16\sfrac{1}{14}p = 128\sfrac{8}{14}$. Отже, в наслідок
підвищення заробітної плати на 25\% ціна виробництва
тієї самої кількості того самого товару підвищилась тут з 120
до 128\sfrac{8}{14}, або більше ніж на 7\%.

Візьмім, навпаки, сферу виробництва вищого складу, ніж пересічний
капітал, наприклад, $92 c + 8 v$. Отже, первісний пересічний
зиск і тут = 20, і якщо ми знову припустимо, що весь
основний капітал входить у річний продукт і що час обороту
такий самий, як і в випадках І і II, то ціна виробництва товару
й тут = 120.

В наслідок підвищення заробітної плати на 25\% змінний капітал
для тієї самої кількості праці зростає з 8 до 10, отже
витрати виробництва товарів зростають з 100 до 102; з другого
боку, пересічна норма зиску впала з 20\% до 14\sfrac{2}{7}\%. Але
$100 : 14\sfrac{2}{7} = 102 : 14\sfrac{4}{7}$\footnote*{
В першому німецькому виданні тут стоїть: „(приблизно)“. В рукопису
Маркса цього слова немає. В дійсності тут рівняння точне, а не тільки приблизне.
\emph{Примітка ред. нім. вид. ІМЕЛ.}
}. Отже, зиск, що припадає тепер на 102,
становить 14\sfrac{4}{7}. І тому весь продукт продається за
$k + kp' = 102 + 14\sfrac{4}{7} = 116\sfrac{4}{7}$. Отже, ціна виробництва впала
з 120 до 116\sfrac{4}{7}, або майже на 3\%\footnote*{
В першому німецькому виданні тут сказано: „більше ніж на 3\%. В рукопису
Маркса стоїть: „на 3\sfrac{3}{7}“, тобто дано абсолютне число. В процентах воно
дорівнює 2\sfrac{6}{7}\%. \emph{Примітка ред. нім. вид. ІМЕЛ.}
}.

Отже, в наслідок підвищення заробітної плати на 25\%:

1) для капіталу пересічного суспільного складу ціна виробництва
товару лишилась незмінною;

2) для капіталу нижчого складу ціна виробництва товару
\parbreak{}  %% абзац продовжується на наступній сторінці
