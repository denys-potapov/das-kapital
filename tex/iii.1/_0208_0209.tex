
\index{iii1}{0208}  %% посилання на сторінку оригінального видання
\subsection{Ціна виробництва товарів середнього складу}

Ми бачили, яким чином відхилення цін виробництва від вартостей постає в наслідок того:

1) що до витрат виробництва товару додається не додаткова вартість, вміщена в ньому, а
пересічний зиск;

2) що ціна виробництва товару, яка таким чином відхиляється від вартості, входить як елемент
у витрати виробництва інших товарів, в наслідок чого, отже, вже у витратах виробництва товару може
міститись відхилення від вартості спожитих на нього засобів виробництва, незалежно від того
відхилення, що може постати для самого цього товару в наслідок ріжниці між пересічним зиском і
додатковою вартістю.

Таким чином, можливо, що і в товарів, вироблених капіталами середнього складу, витрати виробництва
відхилятимуться від суми вартості елементів, з яких складається ця складова частина їх ціни
виробництва. Припустім, що середній склад є $80 c + 20 v$. Можливо, що в дійсних капіталах, які мають
такий склад, $80  c$ більше або менше вартості $с$, сталого капіталу, бо це $с$ складається з товарів,
ціна виробництва яких відхиляється від їх вартості. Так само $20 v$  могли б відхилятися від своєї
вартості, якщо в споживання заробітної плати входять товари, ціна виробництва яких відрізняється від
їх вартості; отже, робітник, щоб купити ці товари (замістити їх), мусить витратити більше або менше
робочого часу, отже, мусить виконати більше або менше необхідної праці, ніж потрібно було б, коли б
ціни виробництва
необхідних засобів існування збігалися з їх вартостями.

Однак, ця можливість зовсім не міняє правильності положень, встановлених для товарів
середнього складу. Кількість зиску, що припадає на ці товари, дорівнює кількості вміщеної в них
самих додаткової вартості. Наприклад, при наведеному вище капіталі з складом у $80 с + 20 v$ для
визначення додаткової вартості важливе не те, чи ці числа є вирази дійсних вартостей, а те, як вони
відносяться одне до одного; а саме, що $v = \frac{1}{5}$, а $с = \frac{4}{5}$  всього капіталу. Якщо це так, то
додаткова вартість, вироблена $v$, дорівнює, як ми це припустили вище, пересічному зискові. З другого
боку: через те що додаткова вартість дорівнює пересічному зискові, ціна виробництва = витратам
виробництва + зиск = $k + p = k + m$, на практиці дорівнює вартості товару. Тобто підвищення або
зниження заробітної плати лишає $k + p$  в цьому випадку так само незмінним, як воно лишило б
незмінною вартість товару, і викликає тільки відповідний зворотний рух,
зниження або підвищення, на стороні норми зиску. А саме, якщо в наслідок підвищення або зниження
заробітної плати тут змінилася б ціна товарів, то норма зиску в цих сферах середнього складу стала б
вищою або нижчою порівняно з її рівнем в інших
\index{iii1}{0209}  %% посилання на сторінку оригінального видання
сферах. Лиш оскільки ціна лишається незмінною, сфера
середнього складу зберігає свій рівень зиску однаковим з іншими сферами. Отже, на практиці в цій
сфері справа відбувається цілком так само, як коли б продукти цієї сфери продавались по їх дійсній
вартості. А саме, якщо товари продаються по їх дійсних вартостях, то очевидно, що при інших
однакових умовах підвищення або зниження заробітної плати викликає відповідне зниження або підвищення зиску,
але не викликає ніякої зміни вартості товарів, і що при всіх обставинах підвищення або зниження
заробітної плати ніколи не може вплинути на вартість товарів, а завжди тільки на величину додаткової
вартості.

\subsection{Підстави капіталіста для компенсації}

Уже було сказано, що, конкуренція вирівнює норми зиску різних сфер виробництва в пересічну норму
зиску і саме тим перетворює вартості продуктів цих різних сфер виробництва в ціни виробництва. І це
стається саме в наслідок постійного перенесення капіталу з однієї сфери виробництва до іншої, де в
даний момент зиск стоїть вище пересічного рівня; при цьому, однак, слід взяти до уваги коливання
зиску, зв’язані з чергуванням худих і ситих років в даній галузі промисловості на протязі даного
періоду часу. Ця безперервна еміграція та імміграція капіталу, яка відбувається між різними сферами
виробництва, породжує висхідні і низхідні рухи норми зиску, які більше чи менше взаємно
урівноважуються і через це мають тенденцію повсюди зводити норму зиску до того самого спільного й загального рівня.

Цей рух капіталів завжди викликається в першу чергу станом ринкових цін, які в одному місці
підвищують зиск понад загальний пересічний рівень, в другому — знижують його нижче цього рівня. Ми
покищо залишаємо осторонь купецький капітал, з яким ми тут ще не маємо справи і який, як це
показують пароксизми спекуляції з певними улюбленими товарами, що раптово вибухають, може з
надзвичайною швидкістю витягати маси капіталу з одної галузі застосування і так само швидко кидати
їх до іншої. Але в кожній сфері виробництва у власному розумінні слова — в промисловості,
землеробстві, рудниках і т. д. — перенесення капіталу з однієї сфери в іншу становить значні
труднощі, особливо в наслідок наявності основного капіталу. До того ж досвід показує, що коли
яканебудь галузь промисловості,
наприклад, бавовняна промисловість, в певний час дає надзвичайно високий зиск, то вона потім, в
інший час, дає дуже незначний зиск, а то навіть і збиток, так що за певний цикл років пересічний
зиск в ній приблизно такий самий, як і в інших галузях. І капітал швидко привчається зважати на цей
досвід.

Але чого конкуренція \emph{не} показує, так це визначення вартості, яке керує рухом виробництва; так це
вартостей, які стоять за
\parbreak{}  %% абзац продовжується на наступній сторінці
