
\index{iii1}{0227}  %% посилання на сторінку оригінального видання
[Норма зиску обчислюється на весь застосований капітал, але за певний час, фактично за один рік.
Відношення виробленої за рік і реалізованої додаткової вартості або зиску до всього капіталу,
обчислене в процентах, є норма зиску. Отже, вона не неодмінно дорівнює тій нормі зиску, при якій в
основу обчислення кладеться не рік, а період обороту капіталу, про який іде мова; тільки в тому
випадку, коли цей капітал обертається саме один раз за рік, обидві ці норми збігаються.

З другого боку, зиск, одержаний на протязі року, є тільки сума зисків на товари, вироблені і продані
на протязі того самого року. Якщо ж ми обчислюватимем зиск на витрати виробництва товарів, то
одержимо норму зиску = p/k, де р становить реалізований на протязі року зиск, а k — суму витрат
виробництва товарів, вироблених і проданих протягом того самого часу. Очевидно, що ця норма зиску
p/k тільки в тому випадку може збігатися з дійсною нормою зиску p/K, — маса зиску, поділена на весь
капітал, — коли k = К, тобто коли капітал обертається, саме один раз за рік.

Візьмімо три різні стани якогонебудь промислового капіталу.

І. Капітал в 8000 фунтів стерлінгів виробляє і продає щороку 5000 штук товару по 30 шилінгів за
штуку, отже, має річний оборот в 7500 фунтів стерлінгів. На кожну штуку товару
він дає зиск в 10 шилінгів = 2500 фунтам стерлінгів на рік. Отже, в кожній штуці містяться 20
шилінгів авансованого капіталу і 10 шилінгів зиску, отже норма зиску на кожну штуку становить 10/20
= 50\%. На суму в 7500 фунтів стерлінгів, що обернулась, припадає 5 000 фунтів стерлінгів
авансованого капіталу і 2 500 фунтів стерлінгів зиску; норма зиску на кожний оборот, р/k, так само =
50\%. Навпаки, норма зиску, обчислена на весь капітал, p/K = 2500/8000 = 31 1/4\%

II. Припустім, що капітал збільшується до 10000 фунтів стерлінгів. Припустім, що в наслідок
збільшеної продуктивної сили праці він може виробляти щороку 10000 штук товару при витратах
виробництва в 20 шилінгів на штуку. Він продає їх із зиском в 4 шилінги на штуку, отже, 'по 24
шилінги за штуку. Тоді ціна річного продукту = 12000 фунтам стерлінгів, з яких 10000 фунтів
стерлінгів авансованого капіталу і 2000 фунтів стерлінгів зиску. на кожну штуку = 20, для річного
обороту = 2000/10000, отже, в обох випадках = 20\%, а через те що весь
\parbreak{}  %% абзац продовжується на наступній сторінці
