
\index{iii1}{0243}  %% посилання на сторінку оригінального видання
Якщо дано певну норму зиску, то маса зиску завжди залежить
від величини авансованого капіталу. Але нагромадження
в такому випадку визначається тією частиною цієї маси, яка
зворотно перетворюється в капітал. Ця частина, однак, через
те що вона дорівнює зискові мінус спожитий капіталістами дохід,
залежатиме не тільки від вартості цієї маси, але й від дешевини
тих товарів, які капіталіст може купити на цю вартість;
товарів, які почасти входять у його споживання, — в його дохід,
а почасти в його сталий капітал. (Заробітну плату ми припускаємо
тут за дану.)

Маса капіталу, яку робітник приводить в рух і вартість якої
в наслідок його праці зберігається й знову з’являється в продукті,
цілком відмінна від тієї вартості, яку він додає. Якщо
маса капіталу = 1000, а додана праця = 100, то репродукований
капітал = 1100. Якщо маса = 100, а додана праця = 20, то
репродукований капітал = 120. Норма зиску в першому випадку
= 10\%, у другому = 20\%. І все ж із 100 може бути
нагромаджено більше, ніж з 20. І таким чином наростає потік
капіталу (його знецінення в наслідок підвищення продуктивної
сили ми залишаємо осторонь) або його нагромадження
пропорціонально до тієї маси, яку він уже становить, а не
пропорціонально до висоти норми зиску. Висока норма зиску,
оскільки вона грунтується на високій нормі додаткової вартості,
можлива, якщо робочий день дуже довгий, хоч би
праця була непродуктивна; вона можлива — не зважаючи на
те, що праця непродуктивна — тому що потреби робітників
дуже незначні і що через це пересічна заробітна плата дуже
низька. Низькій заробітній платі відповідатиме відсутність енергії
в робітників. Капітал при цьому нагромаджується повільно,
не зважаючи на високу норму зиску. Населення не збільшується,
а робочий час, що його коштує продукт, великий, не зважаючи
на те, що заробітна плата, яка виплачується робітникові,
мала.

Норма зиску падає не тому, що робітника менше експлуатують,
а тому, що взагалі вживається менше праці порівняно
з застосовуваним капіталом.

Якщо, як ми показали, падіння норми зиску збігається з підвищенням
маси зиску, то більша частина річного продукту праці
привласнюється капіталістом під категорією капіталу (як заміщення
спожитого капіталу) і відносно менша частина — під категорією
зиску. Звідси фантазія попа Чомерса, ніби чим меншу масу річного
продукту капіталісти витрачають як капітал, тим більші
вони поглинають зиски, при чому державна церква приходить
їм на поміч, щоб подбати про споживання значної частини додаткового
продукту, замість її капіталізації. Піп змішує причину
і наслідок. А втім, адже маса зиску зростає навіть при меншій
нормі разом з величиною витраченого капіталу. Однак, це зумовлює
разом з тим концентрацію капіталу, бо тепер умови
\parbreak{}  %% абзац продовжується на наступній сторінці
