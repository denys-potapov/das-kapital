
\index{iii1}{0312}  %% посилання на сторінку оригінального видання
Так само ясно і те, що їх зиск є тільки відрахування з додаткової
вартості, бо вони мають справу тільки з реалізованими
вже вартостями (навіть якщо вони реалізовані тільки в формі
боргових вимог).

Як і в торгівлі товарами, тут відбувається роздвоєння функції,
бо частина технічних операцій, зв’язаних з грошовою циркуляцією,
мусить виконуватись самими торговцями товарами
і виробниками товарів.

\section{З історії купецького капіталу}

Особлива форма грошового нагромадження товарно-торговельного
і грошево-торговельного капіталу буде розглянута лиш у
дальшому відділі.

З досі викладеного само собою випливає, що не може бути
нічого більш безглуздого, як розглядати купецький капітал —
чи то в формі товарно-торговельного капіталу, чи в формі грошево-торговельного
капіталу — як особливий вид промислового
капіталу, подібно до того, наприклад, як гірництво, землеробство,
скотарство, мануфактура, транспортна промисловість і т. д.
становлять дані суспільним поділом праці розгалуження, а тому й
особливі сфери застосування промислового капіталу. Вже те
просте спостереження, що кожний промисловий капітал, в той час
коли він перебуває у фазі циркуляції свого процесу репродукції,
як товарний капітал і грошовий капітал виконує цілком такі
самі функції, які виступають як виключні функції купецького
капіталу в обох його формах, — уже це спостереження мусило б
зробити неможливим таке грубе розуміння. Навпаки, в товарно-торговельному
капіталі і грошево-торговельному капіталі ріжниці
між промисловим капіталом як продуктивним і тим самим
капіталом у сфері циркуляції усамостійнюються в наслідок того,
що певні форми й функції, які тут тимчасово бере на себе капітал,
виступають як самостійні форми й функції частини капіталу,
що відокремилась, і властиві виключно їй. Перетворена форма
промислового капіталу і речові ріжниці між вкладеними в різні
виробництва продуктивними капіталами, які випливають з природи
різних галузей промисловості, є речі зовсім різні.

Крім тієї грубості, з якою економіст взагалі розглядає відмінності
форм, що цікавлять його в дійсності тільки з речової сторони,
у вульгарного економіста в основі такого переплутання
лежать ще двоякого роду обставини. Поперше, його неспроможність
пояснити торговельний зиск у його своєрідності; подруге,
його апологетичне намагання вивести форми товарного капіталу
й грошового капіталу і далі товарно-торговельного капіталу й
грошево-торговельного капіталу, — форми, що виникають з специфічної
форми капіталістичного способу виробництва, який
\parbreak{}  %% абзац продовжується на наступній сторінці
