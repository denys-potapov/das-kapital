\parcont{}  %% абзац починається на попередній сторінці
\index{iii1}{0313}  %% посилання на сторінку оригінального видання
передбачає як свою базу передусім товарну циркуляцію, а тому
й грошову циркуляцію, — вивести їх як форми, що необхідно
виникають з процесу виробництва як такого.

Якби між товарно-торговельним капіталом і грошево-торговельним
капіталом, з одного боку, і виробництвом зерна, з
другого, не було ніякої іншої ріжниці, крім такої, яка є між цим
останнім і скотарством та мануфактурою, то було б цілком
ясно, що виробництво і капіталістичне виробництво взагалі
тотожні, і що зокрема й розподіл суспільних продуктів між членами
суспільства як для продуктивного, так і для особистого
споживання так само мусить вічно відбуватись за допомогою
купців і банкірів, як споживання м’яса за допомогою скотарства
і споживання одягу за допомогою його фабрикації.\footnote{
Мудрий Рошер [„Die Grundlagen der Nationalökonomie“, 2 вид., Штутгарт
і Аугсбург 1857, стор. 102] додумався до того, що коли дехто характеризує
торгівлю як „посередництво“ між виробниками й споживачами, то з таким самим
успіхом можна характеризувати й саме виробництво як „посередництво“
споживання (між ким?), з чого, звичайно, випливає, що торговельний капітал
є частина продуктивного капіталу подібно до землеробського і промислового
капіталів. Отже, якщо можна сказати, що людина може опосереднювати своє
споживання тільки виробництвом (а це вона мусить зробити, навіть не здобувши
освіти в Лейпцігу), або що праця потрібна для присвоєння природи (що можна
назвати „посередництвом“), то звідси, звичайно, виходить, що суспільне „посередництво“, яке випливає
з специфічної суспільної форми виробництва, — \emph{тому
що} воно є посередництво, — має такий самий абсолютний характер необхідності,
такий самий ранг. Слово посередництво вирішує все. Зрештою, адже купці зовсім
не є посередники між виробниками й споживачами (споживачів у відміну від виробників, споживачів, які
не виробляють, ми спочатку не беремо до уваги), а посередники при обміні продуктів цих виробників
між собою; вони тільки проміжні
особи при обміні, який все ж у тисячі випадків відбувається без них.
}

Великі економісти, як Сміт, Рікардо і т. д., в наслідок того,
що вони розглядали основну форму капіталу, капітал як промисловий
капітал, а капітал циркуляції (грошовий капітал і товарний
капітал) фактично розглядали лиш остільки, оскільки він
сам є фазою в процесі репродукції всякого капіталу, попали в
скрутне становище з торговельним капіталом, як особливим видом
капіталу. Положення про утворення вартості, про зиск та
ін., безпосередньо виведені з розгляду промислового капіталу,
не можуть бути застосовані безпосередньо до купецького капіталу.
Тому ці економісти в дійсності лишають купецький капітал
цілком осторонь і згадують про нього тільки як про вид
промислового капіталу. Там, де вони окремо говорять про
нього, як от Рікардо в зв’язку з зовнішньою торгівлею, вони
намагаються довести, що він не утворює ніякої вартості (отже
й додаткової вартості). Але те, що має силу для зовнішньої
торгівлі, стосується й до внутрішньої.

\pfbreak{}

Досі ми розглядали купецький капітал з точки зору і в межах
капіталістичного способу виробництва. Однак не тільки торгівля,
але і торговельний капітал старіший за капіталістичний
\index{iii1}{0314}  %% посилання на сторінку оригінального видання
спосіб виробництва, є в дійсності історично найстаріша
вільна форма існування капіталу.

Оскільки ми вже бачили, що торгівля грішми і авансований
на неї капітал не потребують для свого розвитку нічого іншого,
крім існування гуртової торгівлі, і далі товарно-торговельного
капіталу, то ми тут розглядатимем тільки цей останній.

Через те що торговельний капітал замкнений в сфері циркуляції
і його функція полягає виключно в тому, щоб опосереднювати
обмін товарів, то — залишаючи осторонь нерозвинені форми,
які виникають з безпосередньої мінової торгівлі, — для його існування
не потрібно ніяких інших умов, крім тих, що потрібні для
простої товарної і грошової циркуляції. Або, краще сказати, ці
останні є умовою \emph{його} існування. Який би не був спосіб виробництва,
на основі якого виробляються продукти, що входять
у циркуляцію як товари, — чи виробляються вони на основі первісної
громади, чи рабського виробництва, чи дрібноселянського
і дрібнобуржуазного, або капіталістичного, — це нічого не змінює
в їх характері як товарів, і як товари вони повинні проробити
процес обміну і зміни форми, які супроводять його. Крайні
члени, між якими купецький капітал є посередником, є дані для
нього, цілком так само як вони дані для грошей і для руху грошей.
Єдине необхідне полягає в тому, щоб ці крайні члени були
в наявності як товари, однаково чи виробництво в усьому своєму
обсягу є товарне виробництво, чи на ринок подається тільки
надлишок, який лишається у самостійно господарюючих виробників
після задоволення їх безпосередніх потреб їх виробництвом.
Купецький капітал тільки опосереднює рух цих крайніх членів,
товарів, як даних для нього передумов.

Розмір, в якому виробництво входить у торгівлю, проходить
через руки купців, залежить від способу виробництва і досягає
свого максимуму при повному розвитку капіталістичного
виробництва, коли продукт виробляється вже тільки як товар,
а не як безпосередній засіб існування. З другого боку, на основі
всякого способу виробництва торгівля сприяє утворенню надлишкового
продукту, призначеного входити в обмін, щоб збільшити
споживання або скарби виробників (під якими тут слід
розуміти власників продуктів); отже, вона надає виробництву
характеру виробництва, що все більше й більше має своєю метою
мінову вартість.

Метаморфоза товарів, їх рух, полягає: 1) речово в обміні
різних товарів один на один, 2) формально в перетворенні товару
в гроші, в продажу, і в перетворенні грошей у товар,
в купівлі. І до цих функцій, до обміну товарів за допомогою
купівлі й продажу, зводиться функція купецького капіталу. Отже,
він опосереднює тільки обмін товарів, який, однак, з самого початку
не можна розуміти просто як обмін товарів між безпосередніми
виробниками. При відносинах рабства, при відносинах
кріпацтва, при відносинах данництва (оскільки мається на увазі
\parbreak{}  %% абзац продовжується на наступній сторінці
