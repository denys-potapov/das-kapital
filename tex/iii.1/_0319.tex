
\index{iii1}{0319}  %% посилання на сторінку оригінального видання
Поки торговельний капітал опосереднює обмін продуктів
нерозвинених країн, доти торговельний зиск не тільки виступає
як ошукання і шахрайство, але й дійсно здебільшого виникає з них.
Незалежно від того, що торговельний капітал використовує ріжницю
між цінами виробництва різних країн (і в цьому відношенні
він впливає на вирівнення і встановлення товарних вартостей),
указані способи виробництва приводять до того, що
купецький капітал привласнює собі переважну частину додаткового
продукту, почасти як посередник між суспільствами, виробництво
яких в істотному має ще своєю метою споживну вартість
і для економічної організації яких продаж тієї частини продукту,
яка взагалі йде в циркуляцію, отже, взагалі продаж продуктів по
їх вартості, має другорядне значення; почасти тому, що за цих
колишніх способів виробництва головні власники додаткового
продукту, з якими купець має справу, — рабовласник, феодальний
сеньйор, держава (наприклад, східний деспот), — є представниками
споживаючого багатства, що на нього купець розставляє
пастки, як це правильно відчув щодо феодальних часів
уже А. Сміт у наведеній цитаті. Таким чином, торговельний капітал,
коли йому належить переважне панування, являє собою повсюди
систему грабежу,\footnote{
„Тепер купці дуже скаржаться на дворян або на розбійників, на те, що
їм доводиться торгувати з великою небезпекою і, крім того, їх забирають у полон,
б’ють, оббирають і грабують. Коли б, однак, купці терпіли таке ради
справедливості, то вони, звичайно, були б святими людьми... Але оскільки така
велика несправедливість і нехристиянське злодійство та розбій чиниться купцями
по всьому світу і навіть поміж собою, то що ж дивного, коли волею
божою таке велике майно, неправедно придбане, знову втрачається або
грабується, а їх самих, крім того, б’ють або забирають у полон?... І князі
повинні за таку неправедну торгівлю карати належною владою і забороняти
купцям так безсоромно обдирати їх підданих. А через те що вони цього не
роблять, бог посилає рицарів та розбійників і їх руками карає купців за несправедливість,
і вони мусять бути його дияволами: подібно до того, як він мучить
дияволами або нищить ворогами землю Єгіпетську і весь світ. Так він б’є
одного шахрая руками другого, не даючи цим зрозуміти, що рицарі менші
розбійники, ніж купці: бо купці щодня грабують цілий світ, тоді як рицар за рік
пограбує раз або двічі, одного або двох“. — „Збувається пророкування Ісаї:
князі твої стали спільниками злодіїв. Бо вони вішають злодіїв, які вкрали гульден
або півгульдена, і діють заодно з тими, що грабують весь світ і крадуть
з більшою безпекою, ніж усі інші, ніби для того, щоб справдилося прислів’я:
великі злодії вішають дрібних злодіїв; і як казав римський сенатор Катон: поганенькі
злодії сидять у в’язницях і в кайданах, а громадські злодії ходять
у золоті та в шовках. Що ж кінець-кінцем скаже на це бог? Він зробить так,
як каже через Єзекііля, — князів і купців, одного злодія з другим, він стопить
в одно, як свинець і мідь, як тоді, коли вигорає місто, так що не лишиться більше
ні князів, ні купців“ (\emph{Martin Luther}: Bücher vom Kaufhandel und Wucher. 1524
[„Von Kauffshandlung und Wucher“, Wittenberg 1524. Luthers Werke, Wittenberg
1589, 6 Teil, стор. 296 і далі]).
} отже ж і розвиток його у торговельних
народів як стародавнього, так і нового часу безпосередньо зв’язаний
з насильницьким грабуванням, піратством, викраданням рабів,
поневоленням (у колоніях)\footnote*{
Дужки взяті з рукопису Маркса. \emph{Примітка ред. нім. вид. ІМЕЛ.}
}; так було в Карфагені, в Римі,
пізніше у венеціанців, португальців, голландців і т. д.
