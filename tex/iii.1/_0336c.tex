\parcont{}  %% абзац починається на попередній сторінці
\index{iii1}{0336}  %% посилання на сторінку оригінального видання
між власником капіталу і другою особою. Ми бачимо тільки
віддачу в позику і зворотну сплату. Все, що відбувається між
цими актами, стерлось.

Але через те що гроші, авансовані як капітал, мають властивість
повертатись до того, хто їх авансував, до того, хто
їх витратив як капітал; через те що $Г — Т — Г'$ є імманентна форма
руху капіталу, — саме через це власник грошей може віддавати
їх у позику як капітал, як щось таке, що має властивість повертатись
до своєї вихідної точки і зберігатися й збільшуватись як
вартість у тому русі, який вони пророблюють. Він віддає гроші як
капітал тому що після того, як вони застосовуються як капітал,
вони повертаються до своєї вихідної точки, отже, можуть бути
повернені позичальником через певний час саме тому, що вони
зворотно припливають до нього самого.

Віддача грошей у позику як капіталу — віддача їх під умовою
повернення через певний час — передбачає, отже, що гроші
дійсно застосовуються як капітал, дійсно зворотно припливають
до своєї вихідної точки. Отже, дійсний кругобіг грошей
як капіталу є передумовою юридичної угоди, згідно з якою
позичальник повинен повернути гроші позикодавцеві. Якщо позичальник
витрачає гроші не як капітал, це — його справа. Позикодавець
дає гроші в позику як капітал, і як такий вони повинні
проробити функції капіталу, які включають у собі кругобіг грошового
капіталу аж до зворотного припливу його в грошовій
формі до його вихідної точки.

Акти циркуляції $Г — Т$ і $Т — Г'$, в яких дана сума вартості функціонує
як гроші або як товар, є тільки опосереднюючі процеси,
окремі моменти всього її руху. Як капітал вона пророблює весь
рух $Г — Г'$. Вона авансується як гроші або як сума вартості
в якій-небудь іншій формі і повертається назад як сума вартості.
Той, хто дає в позику гроші, не витрачає їх на купівлю товару
або, якщо сума вартості існує в формі товарів, не продає їх за
гроші, але авансує їх як капітал, як $Г — Г'$, як вартість, яка
в певний строк знову повертається до своєї вихідної точки.
Замість того, щоб купувати або продавати, він дає в позику.
Отже, ця віддача в позику є відповідною формою для відчужування
їх \emph{як капіталу}, а не як грошей або товару. Але звідси ні
в якому разі не випливає, що віддача в позику не може бути
формою і для таких угод, які не мають ніякого відношення до
капіталістичного процесу репродукції.

\pfbreak{}

Досі ми розглядали тільки рух \emph{капіталу}, що віддається в позику,
між його власником і промисловим капіталістом. Тепер
нам треба дослідити \emph{процент}.

Позикодавець витрачає свої гроші як капітал; сума вартості,
яку він відчужує іншій особі, є капітал, і тому вона припливає
до нього назад. Але просте повернення до нього позиченої суми
\parbreak{}  %% абзац продовжується на наступній сторінці
