\parcont{}  %% абзац починається на попередній сторінці
\index{iii1}{0354}  %% посилання на сторінку оригінального видання
між ринковою ціною і витратами виробництва. І ці різні норми
зиску можуть вирівнюватись — спочатку в межах однієї і тієї
самої сфери, а потім між самими різними сферами — тільки за
допомогою постійних коливань.

\pfbreak

(Помітка для дальшого розроблення). Особлива форма кредиту.
Відомо, що коли гроші функціонують як платіжний засіб,
а не як купівельний засіб, то товар відчужується, але його
вартість реалізується тільки пізніше. Якщо платіж відбувається
вже після того, як товар знову продано, то цей продаж виступає
не як наслідок купівлі, а, навпаки, купівля реалізується за
допомогою продажу. Або продаж стає засобом купівлі. — Подруге:
боргові зобов’язання, векселі і т. д. стають платіжними
засобами для кредитора. — Потретє: компенсація боргових зобов’язань
заміняє гроші.

\section{Процент і підприємницький дохід}

Процент, як ми це бачили в двох попередніх розділах, виступає
первісно, є первісно і лишається в дійсності нічим
іншим, як тією частиною зиску, тобто додаткової вартості,
яку функціонуючий капіталіст, промисловець або купець,
оскільки він застосовує не власний, а взятий у позику капітал,
мусить виплатити власникові і позикодавцеві цього капіталу.
Якщо функціонуючий капіталіст застосовує тільки власний капітал,
то такого поділу зиску не відбувається; весь зиск належить
йому. Справді, оскільки власники капіталу самі застосовують
його в процесі репродукції, вони не беруть участі в
конкуренції, яка визначає норму процента, і уже в цьому виявляється,
наскільки категорія процента, яка неможлива без визначення
розміру процента, є чужа рухові промислового капіталу
самому по собі.

„The rate of interest may be defined to be that proportional sum
which the lender is content to receive, and the borrower to pay, for
a year or for any longer or shorter period for the use of a certain
amount of moneyed capital\dots{} when the owner of capital employs
it actively in reproduction, he does not come under the head of
those capitalists, the proportion of whom, to the number of borrowers,
determines the rate of interest“ [„Норма процента може
бути визначена як та пропорціональна сума, яку позикодавець
згоден одержати, а позичальник заплатити за користування певного
сумою грошового капіталу протягом року або якогось іншого
довшого чи коротшого періоду\dots{} якщо власник капіталу
активно застосовує його в репродукції, то він не належить до
\parbreak{}  %% абзац продовжується на наступній сторінці
