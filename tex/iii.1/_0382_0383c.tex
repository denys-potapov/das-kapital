\parcont{}  %% абзац починається на попередній сторінці
\index{iii1}{0382}  %% посилання на сторінку оригінального видання
покриваються, вони абсолютно функціонують як гроші, бо при
цьому не відбувається кінцевого перетворення їх у гроші. Подібно
до того як ці взаємні авансування, виробників і купців
один одному становлять власне основу кредиту, так і їх знаряддя
циркуляції, вексель, становить базу власне кредитних
грошей, банкнот і т. д. Ці останні грунтуються не на грошовій
циркуляції металічних грошей або державних паперових грошей,
а на вексельній циркуляції.

\begin{small}
\noindent{}\emph{W. Leatham} (банкір у Йоркшірі): „Letters on the Currency“, 2 вид., Лондон 1840:
„Я думаю, що загальна сума векселів за весь 1839 рік становила 528 493 842 фунтів
стерлінгів“ [він рахує суму закордонних векселів приблизно в \sfrac{1}{6} всієї суми],
„а сума векселів, які одночасно циркулювали в тому самому році, становила
132123460 фунтів стерлінгів“ (стор. 56). „Векселі є складова частина циркуляції,
розміром своїм більша, ніж усі інші частини, разом узяті“ (стор. З [4]). — „Ця
величезна надбудова з векселів грунтується (!) на основі, утвореній сумою
банкнот і золота; і якщо в ході подій ця основа занадто звужується, її міцності
і навіть її існуванню загрожує небезпека“ (стор. 8). — „Якщо оцінити всю
циркуляцію“ [він має на думці банкноти] „і суму зобов’язань усіх банків,
по яких може постати потреба негайного платежу готівкою, то я думаю, що
це становитиме суму в 153 мільйони, перетворення якої в золото можна вимагати
за законом, а для задоволення цієї вимоги є тільки 14 мільйонів золотом“
(стор. 11). — „Векселі не можуть бути поставлені під контроль, хіба тільки
коли буде ужито заходів проти надміру грошей і низького розміру процента
або дисконту, який породжує частину цих векселів і підохочує до цього великого
й небезпечного поширення їх. Неможливо встановити, скільки з них виникло
з дійсних операцій, наприклад, з дійсних купівель і продажів, і яка частина їх
штучно зроблена (fictitious) і складається тільки з бронзових векселів (Reitwechseln)\footnote*{
Фіктивний вексель, який не відбиває ніякої дійсної операції, чи то позики,
чи купівлі-продажу в кредит. Ред. укр. перекладу.
}, тобто коли
вексель видається, щоб замінити поточний вексель до прострочення платежу і таким чином створити
фіктивний капітал за допомогою утворення простих засобів циркуляції. Мені відомо, що за часів, коли
грошей є надмірна кількість і коли вони дешеві, це практикується в колосальних розмірах“ (стор. 43,
44). \emph{J. W. Bosanquet}: „Metallic, Paper, and Credit Currency“, London 1842: „Пересічна сума платежів,
зроблених у розрахунковій палаті [де лондонські банкіри взаємно обмінюються оплаченими чеками й
векселями, яким надійшов строк платежу], становить у кожний операційний день понад 3 мільйони фунтів
стерлінгів, а потрібний для цієї мети щоденний грошовий запас — дещо більший,
ніж 200000 фунтів стерлінгів“ (стор. 86). [В 1889 році загальний оборот розрахункової палати
становив 7618\sfrac{3}{4} мільйонів фунтів стерлінгів або, при круглому числі в 300 операційних днів,
пересічно 25\sfrac{1}{2} мільйонів щоденно. — \emph{Ф. Е.}]. „Векселі є, безперечно, засіб циркуляції (currency),
незалежно від грошей,
оскільки вони передають власність з рук у руки за допомогою передатного
напису“ (стор. 92 [93]). „Пересічно можна вважати, що кожний вексель,
який перебуває в циркуляції, має на собі два передатні написи і що кожний
вексель, таким чином, до скінчення його строку пересічно покриває два
платежі. Таким чином, можна вважати, що за допомогою самих тільки передатних
написів на протязі 1839 року векселі передали з рук у руки власність
на суму вдвоє більшу за 528 мільйонів, тобто на 1056 мільйонів фунтів стерлінгів,
— більше, ніж 3 мільйони щоденно. Тому можна з певністю сказати, що
векселі і вклади, разом узяті, шляхом передачі власності з рук у руки і без
допомоги грошей виконують грошові функції на суму щонайменше у 18 мільйонів
фунтів стерлінгів щоденно“ (стор. 93).

Про кредит взагалі \emph{Тук} каже таке: „У своєму найпростішому виразі кредит
є добре чи погано обгрунтоване довір’я, завдяки якому одна особа довіряє
другій певну суму капіталу, в грошах, або в товарах, оцінених у певній грошовій
\index{iii1}{0383}  %% посилання на сторінку оригінального видання
вартості, при чому цю суму, завжди, після скінчення певного строку, належить
повернути назад. Якщо капітал віддається в позику грішми, тобто банкнотами,
або банковим кредитом, або переказом на кореспондента, то на суму, яку
належить повернути при сплаті позики, додатково нараховується за користування
капіталом стільки то процентів. А при товарах, грошова вартість яких встановлюється
учасниками угоди і передача яких означає продаж, встановлена сума,
яка має бути сплачена, включає вже винагороду за користування капіталом
і за риск до скінчення строку. Писані платіжні зобов’язання з певним, строком
платежу здебільшого видаються для таких кредитів. І ці зобов’язання, які
можна передавати, або промеси, становлять засіб, за допомогою якого позикодавці,
коли їм трапляється нагода застосувати свій капітал у формі грошей чи
товарів до скінчення строку-цих векселів, здебільшого можуть дешевше взяти
позику або купити, бо їх власний кредит посилюється кредитом другого імени
на векселі“ („Inquiry into the Currency Principle“, crop. 87).

\emph{Ch. Coquelin}: „Du Crédit et des Banques dans l’Industrie“, в „\emph{Revue des deux Mondes}“,
1842, т. 31: „В кожній країні більшість кредитних операцій виконується в самій
сфері промислових відносин... Виробник сировинного матеріалу авансує його
фабрикантові, який обробляє цей матеріал, і одержує від нього платіжне зобов’язання
з певним строком платежу. Виконавши свою частину роботи, фабрикант
в свою чергу і на подібних же умовах авансує свій продукт іншому фабрикантові,
який мусить обробляти його далі, і таким чином кредит поширюється все далі і
далі, від одного до другого, аж до споживача. Гуртовий торговець дає в кредит
товари дрібному торговцеві, тимчасом як сам він одержує товари в кредит
від фабриканта або комісіонера. Кожен бере в позику однією рукою і дає в
позику другою, — іноді гроші, але далеко частіше продукти. Таким чином у промислових
відносинах відбувається безперестанний обмін позик, які комбінуються
і перехрещуються в усіх напрямах. Саме у множенні і зростанні цих взаємних
позик полягає розвиток кредиту, і тут справжнє місцеперебування його сили“.
\end{small}

\noindent{}Другий бік кредиту примикає до розвитку торгівлі грішми,
який у капіталістичному виробництві, звичайно, іде паралельно
з розвитком товарної торгівлі. В попередньому відділі (розділ XIX)
ми бачили, як у руках торговців грішми концентрується зберігання
резервних фондів ділових людей, технічні операції одержання
й сплати грошей, міжнародних платежів, і разом з цим
торгівля злитками. В зв’язку з цією торгівлею грішми розвивається
другий бік кредиту, — управління капіталом, що дає процент,
або грошовим капіталом, як особлива функція торговців
грішми. Одержання грошей у позику і віддача їх в позику стає
їх особливою справою. Вони виступають як посередники між дійсним
позикодавцем і позичальником грошового капіталу. Взагалі
кажучи, справа банкіра з цього боку полягає в тому,
щоб концентрувати в своїх руках великими масами капітал, що
дається в позику, так що замість окремого позикодавця промисловим
і торговельним капіталістам протистоять банкіри як представники
всіх грошових позикодавців. Вони стають загальними
управителями грошового капіталу. З другого боку, вони концентрують
позичальників супроти всіх позикодавців, бо вони
беруть позики для всього торговельного світу. З одного боку,
банк представляє централізацію грошового капіталу; позикодавців,
з другого боку — централізацію позичальників. Його зиск,
взагалі кажучи, полягає в тому, що він одержує позики за нижчі
проценти, ніж дає в позику.

Позиковий капітал, що ним порядкують банки, припливає до
\parbreak{}  %% абзац продовжується на наступній сторінці
