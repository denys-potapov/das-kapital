
\index{iii1}{0390}  %% посилання на сторінку оригінального видання
[Ця шахрайська процедура практикувалась доти, поки товари з Індії та до
Індії мусили на парусниках обходити мис Доброї Надії. З того часу, як товари
стали відправляти через Суецький канал і при тому на пароплавах, цей метод
здобування фіктивного капіталу втратив свою основу: довгочасність транспортування товарів. А з того
часу, як телеграф почав у той самий день давати відомості англійському купцеві про стан індійського
ринку і індійському торговцеві про стан англійського ринку, цей метод став зовсім неможливим. — \emph{Ф.
Е.}].

III. Нижченаведене взято з цитованого вже звіту „Commercial Distress“,
1847—48: „В останній тиждень квітня 1847 року Англійський банк повідомив
Royal Bank of Liverpool, що з цього моменту він наполовину зменшує свої дисконтні операції з ним. Це
повідомлення справило дуже поганий вплив, тому
що платежі в Ліверпулі за останній час далеко більше провадились векселями,
ніж готівкою, і тому що купці, які звичайно для оплати своїх акцептів вносили
в банк багато грошей готівкою, за останній час могли вносити тільки векселі,
які вони самі одержували за свою бавовну та інші продукти. Це явище дуже
поширилось, і разом з ним збільшились і труднощі в справах. Акцепти, які
банк повинен був оплачувати за купців, здебільшого видавались за кордоном
і досі покривалися в більшості випадків платежами, одержаними за продукти. Векселі, що їх тепер
подавали купці, замість колишніх грошей готівкою, були
різних строків і різного роду; значна частина їх складалася з банкових векселів на три місяці dato
[з дня видачі], велика кількість векселів була видана
під бавовну. Ці векселі акцептувались лондонськими банкірами, якщо вони були
банковими векселями, а в противному разі — всякого роду купцями, бразільськими,
американськими, канадськими, вест-індськими і т. д. фірмами\dots{} Купці не видавали векселів один на
одного, а клієнти всередині країни, які купували продукти в Ліверпулі, оплачували їх векселями на
лондонські банки, або векселями на інші фірми в Лондоні, або векселями на кого-небудь іншого.
Повідомлення Англійського банку привело до того, що для векселів під продані іноземні продукти був
скорочений строк, який до того часто перевищував три місяці“ (стор. [1], 2, 3).

Період процвітання 1844—1847 рр. в Англії, як відзначено вище, був зв’язаний
з першою великою залізничною гарячкою. Про вплив її на справи взагалі згаданий звіт каже таке: „У
квітні 1847 року майже всі торговельні фірми почали
більш чи менш виснажувати свої підприємства (to starve their business), вкладаючи частину свого
торговельного капіталу в залізниці“ (стор. 18). — „Під залізничні акції брались також і позики за
високі проценти, наприклад, по 8\%,
у приватних осіб, банкірів та страхових товариств“ (стор. 42 [43]). „Такі великі
авансування цих торговельних фірм на залізниці знов таки примушували їх брати
в банків за допомогою дисконту векселів занадто багато капіталу, щоб на нього
продовжувати ведення свого власного підприємства“ (стор. 43). — (Запитання:) „Чи
сказали б ви, що внески за залізничні акції значно сприяли пригніченню, яке
панувало“ [на грошовому ринку] „в квітні й жовтні [1847 р.]?“ (Відповідь:) „Я гадаю, що навряд чи
вони мали якийсь вплив на пригнічення в квітні. На мою думку,
вони до квітня і, мабуть, аж до літа скоріше підкріпляли, ніж ослабляли банкірів.
Бо дійсне застосування грошей зовсім не відбувалось так само швидко, як надходили внески; в наслідок
цього більшість банків мали в своїх руках на початку
року досить значну суму залізничних фондів“. [Це підтверджуються численними
свідченнями банкірів в комісії „Commercial Distress“ 1848/1857.] „Сума ця літом
помалу зменшувалась і на 31 грудня була значно менша. Одною з причин
пригнічення в жовтні було ступневе зменшення залізничних фондів у руках
банків; між 22 квітня і 31 грудня залізничні сальдо в наших руках зменшились
на третину. Такий вплив мали внески за залізничні акції в усій Великобританії;
вони помалу вичерпали вклади банків“ (стор. 19, 20). — Те саме каже і Samuel
Gurney (шеф відомої фірми Overend Gurney and С°): „В 1846 році був значно
більший попит на капітали для залізниць, але він не підвищив розміру процента. Відбулося злиття
дрібних сум у великі маси, і ці великі маси були витрачені на нашому ринку; так що загалом результат
був той, що на грошовий
ринок Сіті викидалося більше грошей, а звідти вони забиралися не так
швидко“ [стор. 135].

\index{iii1}{0391}  %% посилання на сторінку оригінального видання
A. Hodgson, директор ліверпульського Joint Stock Bank, показує, в якій великій мірі векселі можуть
сприяти утворенню резервів для банкірів: „Ми мали
звичай принаймні \sfrac{9}{10} усіх наших вкладів і всі гроші, які ми одержували від
інших осіб, тримати в нашому портфелі у векселях, строки яких кінчалися
з дня на день\dots{} настільки, що під час кризи сума надходжень по векселях,
строк яких щодня кінчався, майже дорівнювала сумі вимог платежу, які нам
щодня ставилися“ (стор. 29).

\emph{Спекулятивні векселі.} — „№ 5092. Ким головним чином акцептувалися векселі (за продану бавовну)?“ [R.
Gardner, бавовняний фабрикант, не раз згадуваний в цій праці:] „Товарними маклерами; торговець купує
бавовну,
передає її маклерові, виставляє вексель на цього маклера і дисконтує його“. — „№ 5094. І ці векселі
йдуть до ліверпульських банків і там дисконтуються? — Так, а також і в інших місцях\dots{} Коли б не
було цього надання кредитів, на яке
йшли головним чином ліверпульські банки, то бавовна була б у минулому році,
на мою думку, на 1\sfrac{1}{2} або 2 пенси на фунт дешевша“. — „№ 600. Ви сказали,
що в циркуляції було величезне число векселів, виставлених спекулянтами на
бавовняних маклерів у Ліверпулі; чи це саме стосується і до виданих вами
позик під векселі за інші колоніальні продукти, крім бавовни?“ — [A. Hodgson,
банкір у Ліверпулі:] „Це стосується до всіх сортів колоніальних продуктів,
особливо ж до бавовни“. — „№ 601. Чи стараєтесь ви, як банкір, уникати
такого роду векселів? — Аж ніяк; ми вважаємо їх цілком закономірними векселями, якщо тільки мати їх
у помірній кількості\dots{} По векселях цього роду
строки часто відсуваються“.

\emph{Шахрайства на ост-індсько-китайському ринку в 1847 році.} — Charles
Turner (шеф однієї з першорядних ост-індських фірм у Ліверпулі): „Всі ми
знаємо випадки, які мали місце в операціях на острові Маврікія і в інших
подібних справах. Маклери звикли брати позики під товари не тільки після
їх прибуття, на покриття векселів, виданих за ці товари, що є цілком нормальна річ, і позики під
накладні\dots{} але вони брали позики під продукт раніше,
ніж він був навантажений на судна, а в деяких випадках — раніше, ніж він був
вироблений. Я, наприклад, купив з особливої нагоди в Калькутті векселів на 6000—7000 фунтів
стерлінгів; виручка від цих векселів пішла до острову Маврікія
на сприяння культурі цукру; векселі прийшли до Англії, і половина з них була
опротестована; потім, коли нарешті прийшов вантаж цукру, яким мали б бути
оплачені векселі, то виявилось, що цей цукор був уже заставлений третім особам,
раніше ніж був навантажений на судна, в дійсності навіть майже раніше, ніж він
був вироблений“ (стор. 54). „Тепер товари для ост-індського ринку доводиться
оплачувати фабрикантові готівкою; але це не має великого значення, бо якщо покупець має в Лондоні
який-небудь кредит, він виставляє вексель на Лондон
і дисконтує його в Лондоні, де дисконт стоїть тепер низько; одержаними таким
чином грішми він платить фабрикантові\dots{} минає принаймні дванадцять місяців,
поки відправник товарів до Індії зможе одержати звідти виручені за них гроші;\dots{}
людина з 10000 або 15000 фунтів стерлінгів, яка береться вести операції з Індією, одержить у
лондонської фірми кредит на значну суму; цій фірмі вона платитиме 1\% і видаватиме на неї векселі під
умовою, що виручка від відправлених до
Індії товарів надсилатиметься цій лондонській фірмі; але при цьому обидві сторони мовчки
погоджуються, що лондонська фірма не повинна давати дійсної
позики готівкою, тобто що векселі будуть пролонговані, поки не надійдуть виручені за товари гроші.
Векселі дисконтувались у Ліверпулі, Манчестері, Лондоні, деякі з них перебувають у руках
шотландських банків“ (стор. 55). — „№ 786.
Ось фірма, яка недавно збанкрутувала в Лондоні; при розгляді книг виявили таке:
існувала одна фірма в Манчестері і друга в Калькутті; вони відкрили кредит
у лондонської фірми на 200000 фунтів стерлінгів, тобто ділові друзі цієї манчестерської фірми, які
посилали товари з Глазго й Манчестера на комісію фірмі в Калькутті, трасували на лондонську фірму в
сумі до 200000 фунтів стерлінгів; одночасно існувала угода, що калькуттська фірма видає векселів на
лондонську фірму теж на 200000 фунтів стерлінгів; ці векселі були продані у Калькутті; на виручені
гроші були куплені інші векселі, і ці останні були відіслані до Лондона,
щоб дати можливість лондонській фірмі оплатити перші векселі, видані в Глазго
або Манчестері. Таким чином за допомогою однієї тільки цієї операції було породжено
\index{iii1}{0392}  %% посилання на сторінку оригінального видання
на світ векселів на 600000 фунтів стерлінгів“ [стор. 61]. — „№ 971.
Тепер, якщо яканебудь фірма в Калькутті купує корабельний вантаж“ (для Англії)
„і оплачує його своїми власними траттами на свого лондонського кореспондента,
і накладні відсилаються сюди, то ці накладні відразу використовуються для одержання позик на Lombard
Street; отже, вона має вісім місяців, протягом яких
може користуватись цими грішми, раніше ніж її кореспонденти муситимуть
оплатити ці векселі“.

IV. В 1848 році засідала таємна комісія палати лордів для дослідження причин
кризи 1847 року. Свідчення перед цією комісією були, однак, опубліковані
лиш в 1857 році (Minutes of Evidence, taken before the Secret Committee of the
House of Lords appointed to inquire into the Causes of Distress etc.“ 1857; цитовано як: „Commercial
Distress“ 1848—1857). В цій комісії пан Lister, управитель
Union Bank of Liverpool, сказав між іншим таке:

„2444. Весною 1847 року кредит нечувано розширився\dots{} бо ділові люди
перенесли свій капітал з підприємств у залізниці і все ж хотіли й далі провадити свої підприємства в
попередніх розмірах. Кожен, мабуть, спочатку думав,
що зможе продати залізничні акції з зиском і таким чином вернути гроші в підприємство. Побачивши,
мабуть, що це неможливо, кожний почав брати для свого
підприємства в кредит там, де раніше платив готівкою. Звідси виникло розширення кредиту“.

„2500. Чи ці векселі, на яких банки, що їх прийняли, потерпіли збитки, — чи
були ці векселі видані головним чином під хліб, чи під бавовну?\dots{} Це були
векселі під продукти всякого роду, хліб, бавовну й цукор і іноземні продукти всякого роду. Тоді не
було майже жодного продукту, за винятком хіба
олії, який не впав би в ціні“. — „2506. Маклер, який акцептує вексель, не акцептує його без
достатнього покриття, включаючи й можливість падіння ціни того
товару, що служить покриттям“.

„2512. Під продукти видаються двоякого роду векселі. До першого роду належить первісний вексель,
який виставляється за кордоном на імпортера\dots{} Векселям,
які таким чином видаються під продукти, часто настає строк раніше, ніж прибувають продукти. Тому
купець, якщо прибуде товар і в нього немає достатнього
капіталу, мусить заставити його в маклера, поки не зможе продати його. Тоді
ліверпульським купцем негайно виставляється вексель другого роду на маклера,
під забезпечення цього товару\dots{} це вже тоді справа банкіра довідатись у маклера, чи є в нього товар
і скільки він дав під нього позики. Він мусить переконатися, що маклер має покриття, щоб підправити
свої справи в разі збитків“.

„2516. Ми одержуємо також векселі зза кордону\dots{} Хто-небудь купує за кордоном вексель на Англію і
надсилає його якійсь англійській фірмі; з цього векселя ми не можемо бачити, чи виданий він розумно
чи нерозумно, чи репрезентує він товар, чи вітер“.

„2533. Ви сказали, що закордонні продукти майже всякого роду були продані з великими збитками. Чи
думаєте ви, що це було наслідком неоправдуваної
спекуляції цими продуктами? — Збитки виникли від дуже великого довозу, тимчасом як не було
відповідного споживання для поглинення його. Як видно з
усього, споживання дуже впало“. — „2534. У жовтні\dots{} продуктів майже не можна
було продати“.

Як під час вищої точки краху лунає загальне sauve qui peut [рятуйся, хто
може], про це говорить в тому самому звіті першорядний знавець, вельмишановний бувалий квакер Samuel
Gurney з фірми Overend Gurney and С°: „1262. Коли
панує паніка, то ділова людина не запитує себе, по якій ціні вона може вмістити свої банкноти, або
чи втратить вона 1 чи 2\% при продажу своїх державних або трипроцентних цінних паперів. Раз вона
перебуває під впливом страху, її вже не цікавить ні зиск, ні збиток; вона забезпечує саму себе, всі
інші
можуть робити, що хочуть“.

V. Про взаємне переповнення двох ринків пан Alexander, купець, що веде
торгівлю з Ост-Індією, показав перед комісією нижньої палати в справі банкового акту 1857 року
(цитується як „Bank Committee“ 1857) таке: „4330. В даний
момент, якщо я витрачаю в Манчестері 6 шилінгів, то в Індії одержую назад
5 шилінгів. Якщо я витрачаю в Індії 6 шилінгів, то в Лондоні одержую назад
\index{iii1}{0393}  %% посилання на сторінку оригінального видання
5 шилінгів“. Таким чином, індійський ринок переповнений Англією, а англійський — Індією в однаковій
мірі. І цей випадок мав місце якраз влітку 1857 року,
через неповних десять років після гіркого досвіду 1847 року!

\section{Нагромадження грошового капіталу; Його вплив на розмір процента}

„В Англії відбувається постійне нагромадження додаткового
багатства, яке має тенденцію, кінець-кінцем, набрати грошової
форми. Але бажання набувати гроші супроводиться найнастійливішим бажанням знову звільнитись від них
шляхом якого-небудь
застосування, що дає процент або зиск; бо гроші як гроші не
дають нічого. Тому, якщо одночасно з цим постійним припливом надлишкового капіталу не відбувається
ступневого і достатнього розширення поля діяльності для нього, то у нас
періодично нагромаджуються гроші, які шукають застосування,
при чому ці нагромадження, залежно від обставин, мають
більше чи менше значення. Протягом довгого ряду років державні борги були головним засобом
поглинення надлишкового
багатства Англії. З того часу, як державний борг досяг у
1816 році свого максимуму і більше вже не поглинає багатства, щороку виявлялася сума принаймні в 27
мільйонів, яка
шукала іншого застосування. До того ж відбувались ще зворотні
виплати капіталу різного роду\dots{} Підприємства, які для свого
здійснення потребують великих капіталів і які час від часу
відтягають надлишок незайнятого капіталу\dots{} принаймні для нашої країни абсолютно необхідні для того,
щоб відводити періодичні нагромадження надлишкового багатства суспільства, які не
можуть знайти собі місця в звичайних галузях застосування“
(„The Currency Theory Reviewed“, London 1845, стор. 32 [33, 34]).
Про 1845 рік сказано там же: „Протягом дуже короткого періоду ціни від найнижчої точки депресії
підскочили вгору\dots{} трипроцентна державна позика стоїть майже al pari [на рівні номінальної
вартості]\dots{} золото в підвалах Англійського банку перевищує всяку суму, яка будь-коли там
нагромаджувалась. На
акції всякого роду стоять ціни, майже ніколи нечувані, а розмір
процента так упав, що він майже номінальний\dots{} Все це доводить,
що в Англії тепер знову наявне тяжке нагромадження незайнятого
багатства, що в недалекому майбутньому знову матимемо період
спекулятивної гарячки“ (там же, стор. 36).

„Хоча довіз золота не є певною ознакою зисків у зовнішній
торгівлі, все ж частина цього золотого довозу, за відсутністю
іншого способу пояснення, репрезентує prima facie такий зиск“
(\emph{J. G. Hubbard}: „The Currency and the Country“. London 1843,
стор. [40] 41). „Припустімо, що в такий період, коли справи
весь час добрі, ціни зисковні і грошовий обіг добре заповнений,
\index{iii1}{0394}  %% посилання на сторінку оригінального видання
внаслідок неврожаю довелося б вивезти 5 мільйонів
золота і довезти хліба на таку ж суму. Циркуляція“ [це означає, як зараз виявиться, не засоби
циркуляції, а незайнятий грошовий капітал — \emph{Ф. Е.}] „зменшиться на таку ж суму. Можливо, що приватні
люди матимуть ще досить засобів циркуляції,
але вклади купців у їх банках, сальдо банків у їх грошових маклерів і резерви в касах банків всі
зменшаться, і безпосереднім
наслідком цього зменшення суми незайнятого капіталу буде підвищення розміру процента, наприклад, з 4
до 6\%. Тому що стан
справ здоровий, то й довір’я не буде захитане, але кредит цінуватиметься вище“ (там же, стор. 42).
„Якщо відбувається загальне
падіння товарних цін, то надлишкові гроші припливають назад
до банків у формі збільшених вкладів, надлишок незайнятого капіталу знижує розмір процента до
мінімуму, і цей стан речей
триває доти, поки вищі ціни або пожвавлення справ не покличуть
до роботи бездіяльні гроші, або поки ці гроші не будуть поглинені в купівлі іноземних цінних паперів
чи іноземних товарів“ (стор. 68).

Дальші витяги знов таки взяті з парламентського звіту про
„Commercial Distress“ 1847—1848 рр. — В наслідок неврожаю і голоду 1846—1847 років став потрібним
великий довіз засобів харчування. „Звідси велике перевищення довозу над вивозом\dots{} Звідси
значний відплив грошей з банків і посилений наплив до дисконтних маклерів осіб, яким треба було
дисконтувати векселі;
маклери починають обережніше приймати векселі. Одержуваний
досі кредит зазнав серйозного обмеження, і серед слабих фірм
почались банкрутства. Ті, що цілком покладались на кредит,
зруйнувалися вкрай. Це збільшило тривогу, яка відчувалася ще
раніше; банкіри та інші побачили, що вони не можуть з такою
певністю, як раніше, розраховувати на перетворення своїх векселів та інших цінних паперів у
банкноти, щоб виконати свої зобов’язання; вони ще більше обмежили надання позик і часто
зовсім відмовляли в цьому; в багатьох випадках вони приховували свої банкноти для майбутнього
покриття своїх власних зобов’язань; вони вважали за краще зовсім не випускати їх. Тривога
й замішання зростали з кожним днем, і якби не лист лорда
Джона Росселя, то настало б загальне банкрутство“ (стор. 50, 51).
Лист Росселя припинював чинність банкового акту. — Вищезгаданий Charles Turner свідчить: „Деякі
фірми мали великі кошти, але
вони не були вільні. Весь їх капітал міцно засів у земельній
власності на острові Маврікія або в фабриках індиго та цукрових заводах. Прийнявши на себе раніше
зобов’язання на 500000 — 600000 фунтів стерлінгів, вони не мали вільних коштів для
оплати цих векселів і кінець-кінцем виявилось, що вони могли
оплатити свої векселі тільки за допомогою свого кредиту і лиш
остільки, оскільки його вистачало“ (стор. 57). — Згаданий S. Gurney
заявив: „Нині (1848 р.) панує обмеження оборотів і великий
надмір грошей“. — „№ 1763. Я не думаю, щоб причиною, яка так
\parbreak{}  %% абзац продовжується на наступній сторінці
