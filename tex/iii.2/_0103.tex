
\index{iii2}{0103}  %% посилання на сторінку оригінального видання
\chapter{Перетворення надзиску на земельну ренту}

\section{Вступ}

Аналіза земельної власности в її різних історичних формах лежить поза
межами цієї праці. Ми спиняємось на ній лише остільки, оскільки частина додаткової
вартости, випродукуваної капіталом, припадає земельному власникові.
Отже, ми припускаємо, що в хліборобстві, цілком так само, як в мануфактурі
панує капіталістичний спосіб продукції, тобто що сільське господарство провадять
капіталісти, які відрізняються від решти капіталістів передусім лише тим
елементом, до якого прикладається їхній капітал та наймана праця, яку цей
капітал пускає в рух. На наш погляд фармер продукує пшеницю і т. ін. так
само, як фабрикант — пряжу або машини. Та передумова, що капіталістичний спосіб
продукції опанував сільське господарство, має в собі й те, що цей спосіб продукції
опановує всі сфери продукції й буржуазного суспільства, що, отже,
є наявні і його цілком розвинуті умови, як от вільна конкуренція капіталів,
змога переносити їх з однієї сфери продукції до іншої, однакова висота пересічного
зиску і т. ін. Та форма земельної власности, що її ми розглядаємо,
становить специфічно історичну її форму, \emph{перетворену} — в наслідок впливу
капіталу та капіталістичного способу продукції — форму або февдальної земельної
власности, або дрібно-селянського хліборобства, що провадиться як ділянка
для прохарчування, хліборобства, що в ньому \emph{володіння} землею для безпосередного
продуцента є одна з умов продукції, а його, того продуцента \emph{власність}
на землю є найвигідніша умова розцвіту \emph{його} способу продукції. Якщо
капіталістичний спосіб продукції взагалі має собі за передумову експропріацію
умов праці в робітників, то в хліборобстві він має собі за передумову
експропріацію землі в сільських робітників та підпорядкування їх капіталістові,
що провадить хліборобство за-для зиску. Отже, для нашої аналізи цілком байдуже,
коли нам заперечуватимуть, нагадуючи, що були або ще й досі є й інші
форми земельної власности та хліборобства. Це заперечення може вразити тільки
тих економістів, що розглядають капіталістичний спосіб продукції в сільському
господарстві та відповідну йому форму земельної власности не як історичні, а як
вічні категорії.

Для нас розгляд новітньої форми земельної власности потрібний тому, що
взагалі справа йде про розгляд тих певних відносин продукції й обміну, що
\parbreak{}  %% абзац продовжується на наступній сторінці
