
\index{iii2}{0156}  %% посилання на сторінку оригінального видання
Загальна сума грошової ренти становила б якраз половину того, що було
в таблиці II, де додаткові капітали були вкладені за незмінних цін продукції.

Найважливіше є порівняти вищенаведені таблиці з таблицею І.

Ми бачимо, що з пониженням ціни продукції на половину, з 60 шил. до
30 шил. за квартер, загальна сума грошової ренти залишилась та сама = 18 ф.
ст. і відповідно до цього збіжжева рента подвоїлась, саме зросла з 6 кварт. до
12 кварт. Рента з В відпала; з С грошова рента в ІVс збільшилась на половину,
але на половину зменшилась в ІVс; з D вона лишилась та сама = 9 ф.
стерл. у таблиці ІVс, і піднеслась з 9 ф. стерл. до 15 ф. стерл. у таблиції ІVd.
Продукц я піднеслась з 10 квартерів до 34 в ІVс, і до 30 квартер в в IVd;
зиск підвищився з 2 ф. стерл. до 5\sfrac{1}{2} в ІVс і до 4 \sfrac{1}{2} в IVd. Загальна сума
вкладеного капіталу зросла в одному випадку з 10 ф. стерл. до 27\sfrac{1}{2} ф. стерл.,
в другому — з 10 до 22\sfrac{1}{2} ф. стерл.; отже, обидва рази більше, ніж удвоє. Норма
ренти, рента, обчислена у відношенні до авансованого капіталу, в усіх таблицях
від IV до IVd для кожного роду землі всюди та сама, що вже було дано тим припущенням,
що норма продуктивности обох послідовних витрат капіталу на землях
усіх родів не змінюється. Проти таблиці І вона, проте, понизилась пересічно
щодо всіх родів землі і для кожного окремого роду землі. В таблиці І вона =
180\% пересічно, в таблиці ІVс вона$ = \frac{18}{27\sfrac{1}{2}} × 100 = 65\sfrac{5}{11}\%$ і
IVd = $\frac{18}{22\sfrac{1}{2}} × 100 = 80\%$. Пересічна грошова рента з акра підвищилась. Її пересічна
величина давніш в таблиці І була 4\sfrac{1}{2} ф. стерл. з акра для всіх 4 акрів,
а тепер у таблицях IVс і d вона дорівнює 6 ф. стерл. з акра для 3 акрів.
Її пересічна величина для землі, що дає ренту, була раніш 6 ф. стерл., а тепер
зона дорівнює 9 ф. стерл. з акра. Отже, грошова вартість ренти з акра підвищилась
і репрезентує тепер удвоє більше продукту в збіжжі, ніж давніш, але
12 квартерів збіжжевої ренти тепер становлять менше, ніж половину всього продукту
в 34, зглядно 30\footnote*{В німецькому тексті стоїть: усього «продукту в 33, зглядно 27 квартерів» Явна помилка,
як це можна бачити з таблиць ІVс і IVd. \emph{Прим. Ред.}} квартерів, тимчасом як у таблиці І 6 квартерів становлять
\sfrac{3}{5}  усього продукту в 10 квартерів. Отже, хоч рента, коли розглядати
її як відповідну частину всього продукту, а також коли обчислити її у відношенні
до витраченого капіталу, і знизилась, одначе її грошова вартість,
обчислена на акр. збільшилась, а її вартість в продукті, збільшилась ще дужче.
Коли ми візьмемо землю D в таблиці IVd, то ціна продукції тут дорівнює
15 ф. стерл., що з них витрачений капітал = 12\sfrac{1}{2} ф. стерл. Грошова рента = 15
ф. стер. У таблиці І на тій самій землі D ціна продукції була 3 ф. стерл., витрачений
капітал = 2\sfrac{1}{2} ф. стерл., грошова рента = 9 ф. стерл., отже, остання
утроє більша за ціну продукції й майже у чотири рази більша за витрачений
капітал. У таблиці IVd для D грошова рента в 15 ф. стерл. якраз дорівнює ціні
продукції і лише на \sfrac{1}{5}  більша за витрачений капітал. А все ж грошова рента
з акра на \sfrac{2}{3}  більша, 15 ф. стерл. замість 9 ф. стерл. В таблиці І збіжжева
рента в 3 квартери = \sfrac{3}{4}  усього продукту, що становить 4 квартери, в таблиці
IVd вона = 10 квартерам, половині всього продукту (20 квартерів) з акра
землі D. Це показує, що грошова і збіжжева рента з акра може зрости, хоч
вона і становить відносно меншу частину всього здобутку і знизилась у відношенні
до авансованого капіталу.

Вартість всього продукту в таблиці І = 30 ф. стерл.; рента = 18 ф.
стерл. більше від половини цієї вартости. Вартість усього продукту в таблиці
IV = 45 ф. стерл., що з них 18 ф. стерл., менш від половини, становлять
ренту.
